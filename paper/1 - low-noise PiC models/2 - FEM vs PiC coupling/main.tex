\section{Numerical Models}\label{cha:FEM vs. PiC}
    $\delta\!f$ decompositions induce both a fluid (\ref{eqn:PiC-coupled mass conservation}--\ref{eqn:PiC-coupled Maxwell's equations steady-state}) and a kinetic component (\ref{eqn:linearized Boltzmann equation}) that require numerical simulation. When considered individually, each of these two layers offers a choice of possible discretization techniques.
    
    \line

    \documentclass[12pt, a4paper]{report}

\documentclass[12pt, a4paper]{report}

\documentclass[12pt, a4paper]{report}

\input{template/main.tex}

\title{\BA{Title in Progress...}}
\author{Boris Andrews}
\affil{Mathematical Institute, University of Oxford}
\date{\today}


\begin{document}
    \pagenumbering{gobble}
    \maketitle
    
    
    \begin{abstract}
        Magnetic confinement reactors---in particular tokamaks---offer one of the most promising options for achieving practical nuclear fusion, with the potential to provide virtually limitless, clean energy. The theoretical and numerical modeling of tokamak plasmas is simultaneously an essential component of effective reactor design, and a great research barrier. Tokamak operational conditions exhibit comparatively low Knudsen numbers. Kinetic effects, including kinetic waves and instabilities, Landau damping, bump-on-tail instabilities and more, are therefore highly influential in tokamak plasma dynamics. Purely fluid models are inherently incapable of capturing these effects, whereas the high dimensionality in purely kinetic models render them practically intractable for most relevant purposes.

        We consider a $\delta\!f$ decomposition model, with a macroscopic fluid background and microscopic kinetic correction, both fully coupled to each other. A similar manner of discretization is proposed to that used in the recent \texttt{STRUPHY} code \cite{Holderied_Possanner_Wang_2021, Holderied_2022, Li_et_al_2023} with a finite-element model for the background and a pseudo-particle/PiC model for the correction.

        The fluid background satisfies the full, non-linear, resistive, compressible, Hall MHD equations. \cite{Laakmann_Hu_Farrell_2022} introduces finite-element(-in-space) implicit timesteppers for the incompressible analogue to this system with structure-preserving (SP) properties in the ideal case, alongside parameter-robust preconditioners. We show that these timesteppers can derive from a finite-element-in-time (FET) (and finite-element-in-space) interpretation. The benefits of this reformulation are discussed, including the derivation of timesteppers that are higher order in time, and the quantifiable dissipative SP properties in the non-ideal, resistive case.
        
        We discuss possible options for extending this FET approach to timesteppers for the compressible case.

        The kinetic corrections satisfy linearized Boltzmann equations. Using a Lénard--Bernstein collision operator, these take Fokker--Planck-like forms \cite{Fokker_1914, Planck_1917} wherein pseudo-particles in the numerical model obey the neoclassical transport equations, with particle-independent Brownian drift terms. This offers a rigorous methodology for incorporating collisions into the particle transport model, without coupling the equations of motions for each particle.
        
        Works by Chen, Chacón et al. \cite{Chen_Chacón_Barnes_2011, Chacón_Chen_Barnes_2013, Chen_Chacón_2014, Chen_Chacón_2015} have developed structure-preserving particle pushers for neoclassical transport in the Vlasov equations, derived from Crank--Nicolson integrators. We show these too can can derive from a FET interpretation, similarly offering potential extensions to higher-order-in-time particle pushers. The FET formulation is used also to consider how the stochastic drift terms can be incorporated into the pushers. Stochastic gyrokinetic expansions are also discussed.

        Different options for the numerical implementation of these schemes are considered.

        Due to the efficacy of FET in the development of SP timesteppers for both the fluid and kinetic component, we hope this approach will prove effective in the future for developing SP timesteppers for the full hybrid model. We hope this will give us the opportunity to incorporate previously inaccessible kinetic effects into the highly effective, modern, finite-element MHD models.
    \end{abstract}
    
    
    \newpage
    \tableofcontents
    
    
    \newpage
    \pagenumbering{arabic}
    %\linenumbers\renewcommand\thelinenumber{\color{black!50}\arabic{linenumber}}
            \input{0 - introduction/main.tex}
        \part{Research}
            \input{1 - low-noise PiC models/main.tex}
            \input{2 - kinetic component/main.tex}
            \input{3 - fluid component/main.tex}
            \input{4 - numerical implementation/main.tex}
        \part{Project Overview}
            \input{5 - research plan/main.tex}
            \input{6 - summary/main.tex}
    
    
    %\section{}
    \newpage
    \pagenumbering{gobble}
        \printbibliography


    \newpage
    \pagenumbering{roman}
    \appendix
        \part{Appendices}
            \input{8 - Hilbert complexes/main.tex}
            \input{9 - weak conservation proofs/main.tex}
\end{document}


\title{\BA{Title in Progress...}}
\author{Boris Andrews}
\affil{Mathematical Institute, University of Oxford}
\date{\today}


\begin{document}
    \pagenumbering{gobble}
    \maketitle
    
    
    \begin{abstract}
        Magnetic confinement reactors---in particular tokamaks---offer one of the most promising options for achieving practical nuclear fusion, with the potential to provide virtually limitless, clean energy. The theoretical and numerical modeling of tokamak plasmas is simultaneously an essential component of effective reactor design, and a great research barrier. Tokamak operational conditions exhibit comparatively low Knudsen numbers. Kinetic effects, including kinetic waves and instabilities, Landau damping, bump-on-tail instabilities and more, are therefore highly influential in tokamak plasma dynamics. Purely fluid models are inherently incapable of capturing these effects, whereas the high dimensionality in purely kinetic models render them practically intractable for most relevant purposes.

        We consider a $\delta\!f$ decomposition model, with a macroscopic fluid background and microscopic kinetic correction, both fully coupled to each other. A similar manner of discretization is proposed to that used in the recent \texttt{STRUPHY} code \cite{Holderied_Possanner_Wang_2021, Holderied_2022, Li_et_al_2023} with a finite-element model for the background and a pseudo-particle/PiC model for the correction.

        The fluid background satisfies the full, non-linear, resistive, compressible, Hall MHD equations. \cite{Laakmann_Hu_Farrell_2022} introduces finite-element(-in-space) implicit timesteppers for the incompressible analogue to this system with structure-preserving (SP) properties in the ideal case, alongside parameter-robust preconditioners. We show that these timesteppers can derive from a finite-element-in-time (FET) (and finite-element-in-space) interpretation. The benefits of this reformulation are discussed, including the derivation of timesteppers that are higher order in time, and the quantifiable dissipative SP properties in the non-ideal, resistive case.
        
        We discuss possible options for extending this FET approach to timesteppers for the compressible case.

        The kinetic corrections satisfy linearized Boltzmann equations. Using a Lénard--Bernstein collision operator, these take Fokker--Planck-like forms \cite{Fokker_1914, Planck_1917} wherein pseudo-particles in the numerical model obey the neoclassical transport equations, with particle-independent Brownian drift terms. This offers a rigorous methodology for incorporating collisions into the particle transport model, without coupling the equations of motions for each particle.
        
        Works by Chen, Chacón et al. \cite{Chen_Chacón_Barnes_2011, Chacón_Chen_Barnes_2013, Chen_Chacón_2014, Chen_Chacón_2015} have developed structure-preserving particle pushers for neoclassical transport in the Vlasov equations, derived from Crank--Nicolson integrators. We show these too can can derive from a FET interpretation, similarly offering potential extensions to higher-order-in-time particle pushers. The FET formulation is used also to consider how the stochastic drift terms can be incorporated into the pushers. Stochastic gyrokinetic expansions are also discussed.

        Different options for the numerical implementation of these schemes are considered.

        Due to the efficacy of FET in the development of SP timesteppers for both the fluid and kinetic component, we hope this approach will prove effective in the future for developing SP timesteppers for the full hybrid model. We hope this will give us the opportunity to incorporate previously inaccessible kinetic effects into the highly effective, modern, finite-element MHD models.
    \end{abstract}
    
    
    \newpage
    \tableofcontents
    
    
    \newpage
    \pagenumbering{arabic}
    %\linenumbers\renewcommand\thelinenumber{\color{black!50}\arabic{linenumber}}
            \documentclass[12pt, a4paper]{report}

\input{template/main.tex}

\title{\BA{Title in Progress...}}
\author{Boris Andrews}
\affil{Mathematical Institute, University of Oxford}
\date{\today}


\begin{document}
    \pagenumbering{gobble}
    \maketitle
    
    
    \begin{abstract}
        Magnetic confinement reactors---in particular tokamaks---offer one of the most promising options for achieving practical nuclear fusion, with the potential to provide virtually limitless, clean energy. The theoretical and numerical modeling of tokamak plasmas is simultaneously an essential component of effective reactor design, and a great research barrier. Tokamak operational conditions exhibit comparatively low Knudsen numbers. Kinetic effects, including kinetic waves and instabilities, Landau damping, bump-on-tail instabilities and more, are therefore highly influential in tokamak plasma dynamics. Purely fluid models are inherently incapable of capturing these effects, whereas the high dimensionality in purely kinetic models render them practically intractable for most relevant purposes.

        We consider a $\delta\!f$ decomposition model, with a macroscopic fluid background and microscopic kinetic correction, both fully coupled to each other. A similar manner of discretization is proposed to that used in the recent \texttt{STRUPHY} code \cite{Holderied_Possanner_Wang_2021, Holderied_2022, Li_et_al_2023} with a finite-element model for the background and a pseudo-particle/PiC model for the correction.

        The fluid background satisfies the full, non-linear, resistive, compressible, Hall MHD equations. \cite{Laakmann_Hu_Farrell_2022} introduces finite-element(-in-space) implicit timesteppers for the incompressible analogue to this system with structure-preserving (SP) properties in the ideal case, alongside parameter-robust preconditioners. We show that these timesteppers can derive from a finite-element-in-time (FET) (and finite-element-in-space) interpretation. The benefits of this reformulation are discussed, including the derivation of timesteppers that are higher order in time, and the quantifiable dissipative SP properties in the non-ideal, resistive case.
        
        We discuss possible options for extending this FET approach to timesteppers for the compressible case.

        The kinetic corrections satisfy linearized Boltzmann equations. Using a Lénard--Bernstein collision operator, these take Fokker--Planck-like forms \cite{Fokker_1914, Planck_1917} wherein pseudo-particles in the numerical model obey the neoclassical transport equations, with particle-independent Brownian drift terms. This offers a rigorous methodology for incorporating collisions into the particle transport model, without coupling the equations of motions for each particle.
        
        Works by Chen, Chacón et al. \cite{Chen_Chacón_Barnes_2011, Chacón_Chen_Barnes_2013, Chen_Chacón_2014, Chen_Chacón_2015} have developed structure-preserving particle pushers for neoclassical transport in the Vlasov equations, derived from Crank--Nicolson integrators. We show these too can can derive from a FET interpretation, similarly offering potential extensions to higher-order-in-time particle pushers. The FET formulation is used also to consider how the stochastic drift terms can be incorporated into the pushers. Stochastic gyrokinetic expansions are also discussed.

        Different options for the numerical implementation of these schemes are considered.

        Due to the efficacy of FET in the development of SP timesteppers for both the fluid and kinetic component, we hope this approach will prove effective in the future for developing SP timesteppers for the full hybrid model. We hope this will give us the opportunity to incorporate previously inaccessible kinetic effects into the highly effective, modern, finite-element MHD models.
    \end{abstract}
    
    
    \newpage
    \tableofcontents
    
    
    \newpage
    \pagenumbering{arabic}
    %\linenumbers\renewcommand\thelinenumber{\color{black!50}\arabic{linenumber}}
            \input{0 - introduction/main.tex}
        \part{Research}
            \input{1 - low-noise PiC models/main.tex}
            \input{2 - kinetic component/main.tex}
            \input{3 - fluid component/main.tex}
            \input{4 - numerical implementation/main.tex}
        \part{Project Overview}
            \input{5 - research plan/main.tex}
            \input{6 - summary/main.tex}
    
    
    %\section{}
    \newpage
    \pagenumbering{gobble}
        \printbibliography


    \newpage
    \pagenumbering{roman}
    \appendix
        \part{Appendices}
            \input{8 - Hilbert complexes/main.tex}
            \input{9 - weak conservation proofs/main.tex}
\end{document}

        \part{Research}
            \documentclass[12pt, a4paper]{report}

\input{template/main.tex}

\title{\BA{Title in Progress...}}
\author{Boris Andrews}
\affil{Mathematical Institute, University of Oxford}
\date{\today}


\begin{document}
    \pagenumbering{gobble}
    \maketitle
    
    
    \begin{abstract}
        Magnetic confinement reactors---in particular tokamaks---offer one of the most promising options for achieving practical nuclear fusion, with the potential to provide virtually limitless, clean energy. The theoretical and numerical modeling of tokamak plasmas is simultaneously an essential component of effective reactor design, and a great research barrier. Tokamak operational conditions exhibit comparatively low Knudsen numbers. Kinetic effects, including kinetic waves and instabilities, Landau damping, bump-on-tail instabilities and more, are therefore highly influential in tokamak plasma dynamics. Purely fluid models are inherently incapable of capturing these effects, whereas the high dimensionality in purely kinetic models render them practically intractable for most relevant purposes.

        We consider a $\delta\!f$ decomposition model, with a macroscopic fluid background and microscopic kinetic correction, both fully coupled to each other. A similar manner of discretization is proposed to that used in the recent \texttt{STRUPHY} code \cite{Holderied_Possanner_Wang_2021, Holderied_2022, Li_et_al_2023} with a finite-element model for the background and a pseudo-particle/PiC model for the correction.

        The fluid background satisfies the full, non-linear, resistive, compressible, Hall MHD equations. \cite{Laakmann_Hu_Farrell_2022} introduces finite-element(-in-space) implicit timesteppers for the incompressible analogue to this system with structure-preserving (SP) properties in the ideal case, alongside parameter-robust preconditioners. We show that these timesteppers can derive from a finite-element-in-time (FET) (and finite-element-in-space) interpretation. The benefits of this reformulation are discussed, including the derivation of timesteppers that are higher order in time, and the quantifiable dissipative SP properties in the non-ideal, resistive case.
        
        We discuss possible options for extending this FET approach to timesteppers for the compressible case.

        The kinetic corrections satisfy linearized Boltzmann equations. Using a Lénard--Bernstein collision operator, these take Fokker--Planck-like forms \cite{Fokker_1914, Planck_1917} wherein pseudo-particles in the numerical model obey the neoclassical transport equations, with particle-independent Brownian drift terms. This offers a rigorous methodology for incorporating collisions into the particle transport model, without coupling the equations of motions for each particle.
        
        Works by Chen, Chacón et al. \cite{Chen_Chacón_Barnes_2011, Chacón_Chen_Barnes_2013, Chen_Chacón_2014, Chen_Chacón_2015} have developed structure-preserving particle pushers for neoclassical transport in the Vlasov equations, derived from Crank--Nicolson integrators. We show these too can can derive from a FET interpretation, similarly offering potential extensions to higher-order-in-time particle pushers. The FET formulation is used also to consider how the stochastic drift terms can be incorporated into the pushers. Stochastic gyrokinetic expansions are also discussed.

        Different options for the numerical implementation of these schemes are considered.

        Due to the efficacy of FET in the development of SP timesteppers for both the fluid and kinetic component, we hope this approach will prove effective in the future for developing SP timesteppers for the full hybrid model. We hope this will give us the opportunity to incorporate previously inaccessible kinetic effects into the highly effective, modern, finite-element MHD models.
    \end{abstract}
    
    
    \newpage
    \tableofcontents
    
    
    \newpage
    \pagenumbering{arabic}
    %\linenumbers\renewcommand\thelinenumber{\color{black!50}\arabic{linenumber}}
            \input{0 - introduction/main.tex}
        \part{Research}
            \input{1 - low-noise PiC models/main.tex}
            \input{2 - kinetic component/main.tex}
            \input{3 - fluid component/main.tex}
            \input{4 - numerical implementation/main.tex}
        \part{Project Overview}
            \input{5 - research plan/main.tex}
            \input{6 - summary/main.tex}
    
    
    %\section{}
    \newpage
    \pagenumbering{gobble}
        \printbibliography


    \newpage
    \pagenumbering{roman}
    \appendix
        \part{Appendices}
            \input{8 - Hilbert complexes/main.tex}
            \input{9 - weak conservation proofs/main.tex}
\end{document}

            \documentclass[12pt, a4paper]{report}

\input{template/main.tex}

\title{\BA{Title in Progress...}}
\author{Boris Andrews}
\affil{Mathematical Institute, University of Oxford}
\date{\today}


\begin{document}
    \pagenumbering{gobble}
    \maketitle
    
    
    \begin{abstract}
        Magnetic confinement reactors---in particular tokamaks---offer one of the most promising options for achieving practical nuclear fusion, with the potential to provide virtually limitless, clean energy. The theoretical and numerical modeling of tokamak plasmas is simultaneously an essential component of effective reactor design, and a great research barrier. Tokamak operational conditions exhibit comparatively low Knudsen numbers. Kinetic effects, including kinetic waves and instabilities, Landau damping, bump-on-tail instabilities and more, are therefore highly influential in tokamak plasma dynamics. Purely fluid models are inherently incapable of capturing these effects, whereas the high dimensionality in purely kinetic models render them practically intractable for most relevant purposes.

        We consider a $\delta\!f$ decomposition model, with a macroscopic fluid background and microscopic kinetic correction, both fully coupled to each other. A similar manner of discretization is proposed to that used in the recent \texttt{STRUPHY} code \cite{Holderied_Possanner_Wang_2021, Holderied_2022, Li_et_al_2023} with a finite-element model for the background and a pseudo-particle/PiC model for the correction.

        The fluid background satisfies the full, non-linear, resistive, compressible, Hall MHD equations. \cite{Laakmann_Hu_Farrell_2022} introduces finite-element(-in-space) implicit timesteppers for the incompressible analogue to this system with structure-preserving (SP) properties in the ideal case, alongside parameter-robust preconditioners. We show that these timesteppers can derive from a finite-element-in-time (FET) (and finite-element-in-space) interpretation. The benefits of this reformulation are discussed, including the derivation of timesteppers that are higher order in time, and the quantifiable dissipative SP properties in the non-ideal, resistive case.
        
        We discuss possible options for extending this FET approach to timesteppers for the compressible case.

        The kinetic corrections satisfy linearized Boltzmann equations. Using a Lénard--Bernstein collision operator, these take Fokker--Planck-like forms \cite{Fokker_1914, Planck_1917} wherein pseudo-particles in the numerical model obey the neoclassical transport equations, with particle-independent Brownian drift terms. This offers a rigorous methodology for incorporating collisions into the particle transport model, without coupling the equations of motions for each particle.
        
        Works by Chen, Chacón et al. \cite{Chen_Chacón_Barnes_2011, Chacón_Chen_Barnes_2013, Chen_Chacón_2014, Chen_Chacón_2015} have developed structure-preserving particle pushers for neoclassical transport in the Vlasov equations, derived from Crank--Nicolson integrators. We show these too can can derive from a FET interpretation, similarly offering potential extensions to higher-order-in-time particle pushers. The FET formulation is used also to consider how the stochastic drift terms can be incorporated into the pushers. Stochastic gyrokinetic expansions are also discussed.

        Different options for the numerical implementation of these schemes are considered.

        Due to the efficacy of FET in the development of SP timesteppers for both the fluid and kinetic component, we hope this approach will prove effective in the future for developing SP timesteppers for the full hybrid model. We hope this will give us the opportunity to incorporate previously inaccessible kinetic effects into the highly effective, modern, finite-element MHD models.
    \end{abstract}
    
    
    \newpage
    \tableofcontents
    
    
    \newpage
    \pagenumbering{arabic}
    %\linenumbers\renewcommand\thelinenumber{\color{black!50}\arabic{linenumber}}
            \input{0 - introduction/main.tex}
        \part{Research}
            \input{1 - low-noise PiC models/main.tex}
            \input{2 - kinetic component/main.tex}
            \input{3 - fluid component/main.tex}
            \input{4 - numerical implementation/main.tex}
        \part{Project Overview}
            \input{5 - research plan/main.tex}
            \input{6 - summary/main.tex}
    
    
    %\section{}
    \newpage
    \pagenumbering{gobble}
        \printbibliography


    \newpage
    \pagenumbering{roman}
    \appendix
        \part{Appendices}
            \input{8 - Hilbert complexes/main.tex}
            \input{9 - weak conservation proofs/main.tex}
\end{document}

            \documentclass[12pt, a4paper]{report}

\input{template/main.tex}

\title{\BA{Title in Progress...}}
\author{Boris Andrews}
\affil{Mathematical Institute, University of Oxford}
\date{\today}


\begin{document}
    \pagenumbering{gobble}
    \maketitle
    
    
    \begin{abstract}
        Magnetic confinement reactors---in particular tokamaks---offer one of the most promising options for achieving practical nuclear fusion, with the potential to provide virtually limitless, clean energy. The theoretical and numerical modeling of tokamak plasmas is simultaneously an essential component of effective reactor design, and a great research barrier. Tokamak operational conditions exhibit comparatively low Knudsen numbers. Kinetic effects, including kinetic waves and instabilities, Landau damping, bump-on-tail instabilities and more, are therefore highly influential in tokamak plasma dynamics. Purely fluid models are inherently incapable of capturing these effects, whereas the high dimensionality in purely kinetic models render them practically intractable for most relevant purposes.

        We consider a $\delta\!f$ decomposition model, with a macroscopic fluid background and microscopic kinetic correction, both fully coupled to each other. A similar manner of discretization is proposed to that used in the recent \texttt{STRUPHY} code \cite{Holderied_Possanner_Wang_2021, Holderied_2022, Li_et_al_2023} with a finite-element model for the background and a pseudo-particle/PiC model for the correction.

        The fluid background satisfies the full, non-linear, resistive, compressible, Hall MHD equations. \cite{Laakmann_Hu_Farrell_2022} introduces finite-element(-in-space) implicit timesteppers for the incompressible analogue to this system with structure-preserving (SP) properties in the ideal case, alongside parameter-robust preconditioners. We show that these timesteppers can derive from a finite-element-in-time (FET) (and finite-element-in-space) interpretation. The benefits of this reformulation are discussed, including the derivation of timesteppers that are higher order in time, and the quantifiable dissipative SP properties in the non-ideal, resistive case.
        
        We discuss possible options for extending this FET approach to timesteppers for the compressible case.

        The kinetic corrections satisfy linearized Boltzmann equations. Using a Lénard--Bernstein collision operator, these take Fokker--Planck-like forms \cite{Fokker_1914, Planck_1917} wherein pseudo-particles in the numerical model obey the neoclassical transport equations, with particle-independent Brownian drift terms. This offers a rigorous methodology for incorporating collisions into the particle transport model, without coupling the equations of motions for each particle.
        
        Works by Chen, Chacón et al. \cite{Chen_Chacón_Barnes_2011, Chacón_Chen_Barnes_2013, Chen_Chacón_2014, Chen_Chacón_2015} have developed structure-preserving particle pushers for neoclassical transport in the Vlasov equations, derived from Crank--Nicolson integrators. We show these too can can derive from a FET interpretation, similarly offering potential extensions to higher-order-in-time particle pushers. The FET formulation is used also to consider how the stochastic drift terms can be incorporated into the pushers. Stochastic gyrokinetic expansions are also discussed.

        Different options for the numerical implementation of these schemes are considered.

        Due to the efficacy of FET in the development of SP timesteppers for both the fluid and kinetic component, we hope this approach will prove effective in the future for developing SP timesteppers for the full hybrid model. We hope this will give us the opportunity to incorporate previously inaccessible kinetic effects into the highly effective, modern, finite-element MHD models.
    \end{abstract}
    
    
    \newpage
    \tableofcontents
    
    
    \newpage
    \pagenumbering{arabic}
    %\linenumbers\renewcommand\thelinenumber{\color{black!50}\arabic{linenumber}}
            \input{0 - introduction/main.tex}
        \part{Research}
            \input{1 - low-noise PiC models/main.tex}
            \input{2 - kinetic component/main.tex}
            \input{3 - fluid component/main.tex}
            \input{4 - numerical implementation/main.tex}
        \part{Project Overview}
            \input{5 - research plan/main.tex}
            \input{6 - summary/main.tex}
    
    
    %\section{}
    \newpage
    \pagenumbering{gobble}
        \printbibliography


    \newpage
    \pagenumbering{roman}
    \appendix
        \part{Appendices}
            \input{8 - Hilbert complexes/main.tex}
            \input{9 - weak conservation proofs/main.tex}
\end{document}

            \documentclass[12pt, a4paper]{report}

\input{template/main.tex}

\title{\BA{Title in Progress...}}
\author{Boris Andrews}
\affil{Mathematical Institute, University of Oxford}
\date{\today}


\begin{document}
    \pagenumbering{gobble}
    \maketitle
    
    
    \begin{abstract}
        Magnetic confinement reactors---in particular tokamaks---offer one of the most promising options for achieving practical nuclear fusion, with the potential to provide virtually limitless, clean energy. The theoretical and numerical modeling of tokamak plasmas is simultaneously an essential component of effective reactor design, and a great research barrier. Tokamak operational conditions exhibit comparatively low Knudsen numbers. Kinetic effects, including kinetic waves and instabilities, Landau damping, bump-on-tail instabilities and more, are therefore highly influential in tokamak plasma dynamics. Purely fluid models are inherently incapable of capturing these effects, whereas the high dimensionality in purely kinetic models render them practically intractable for most relevant purposes.

        We consider a $\delta\!f$ decomposition model, with a macroscopic fluid background and microscopic kinetic correction, both fully coupled to each other. A similar manner of discretization is proposed to that used in the recent \texttt{STRUPHY} code \cite{Holderied_Possanner_Wang_2021, Holderied_2022, Li_et_al_2023} with a finite-element model for the background and a pseudo-particle/PiC model for the correction.

        The fluid background satisfies the full, non-linear, resistive, compressible, Hall MHD equations. \cite{Laakmann_Hu_Farrell_2022} introduces finite-element(-in-space) implicit timesteppers for the incompressible analogue to this system with structure-preserving (SP) properties in the ideal case, alongside parameter-robust preconditioners. We show that these timesteppers can derive from a finite-element-in-time (FET) (and finite-element-in-space) interpretation. The benefits of this reformulation are discussed, including the derivation of timesteppers that are higher order in time, and the quantifiable dissipative SP properties in the non-ideal, resistive case.
        
        We discuss possible options for extending this FET approach to timesteppers for the compressible case.

        The kinetic corrections satisfy linearized Boltzmann equations. Using a Lénard--Bernstein collision operator, these take Fokker--Planck-like forms \cite{Fokker_1914, Planck_1917} wherein pseudo-particles in the numerical model obey the neoclassical transport equations, with particle-independent Brownian drift terms. This offers a rigorous methodology for incorporating collisions into the particle transport model, without coupling the equations of motions for each particle.
        
        Works by Chen, Chacón et al. \cite{Chen_Chacón_Barnes_2011, Chacón_Chen_Barnes_2013, Chen_Chacón_2014, Chen_Chacón_2015} have developed structure-preserving particle pushers for neoclassical transport in the Vlasov equations, derived from Crank--Nicolson integrators. We show these too can can derive from a FET interpretation, similarly offering potential extensions to higher-order-in-time particle pushers. The FET formulation is used also to consider how the stochastic drift terms can be incorporated into the pushers. Stochastic gyrokinetic expansions are also discussed.

        Different options for the numerical implementation of these schemes are considered.

        Due to the efficacy of FET in the development of SP timesteppers for both the fluid and kinetic component, we hope this approach will prove effective in the future for developing SP timesteppers for the full hybrid model. We hope this will give us the opportunity to incorporate previously inaccessible kinetic effects into the highly effective, modern, finite-element MHD models.
    \end{abstract}
    
    
    \newpage
    \tableofcontents
    
    
    \newpage
    \pagenumbering{arabic}
    %\linenumbers\renewcommand\thelinenumber{\color{black!50}\arabic{linenumber}}
            \input{0 - introduction/main.tex}
        \part{Research}
            \input{1 - low-noise PiC models/main.tex}
            \input{2 - kinetic component/main.tex}
            \input{3 - fluid component/main.tex}
            \input{4 - numerical implementation/main.tex}
        \part{Project Overview}
            \input{5 - research plan/main.tex}
            \input{6 - summary/main.tex}
    
    
    %\section{}
    \newpage
    \pagenumbering{gobble}
        \printbibliography


    \newpage
    \pagenumbering{roman}
    \appendix
        \part{Appendices}
            \input{8 - Hilbert complexes/main.tex}
            \input{9 - weak conservation proofs/main.tex}
\end{document}

        \part{Project Overview}
            \documentclass[12pt, a4paper]{report}

\input{template/main.tex}

\title{\BA{Title in Progress...}}
\author{Boris Andrews}
\affil{Mathematical Institute, University of Oxford}
\date{\today}


\begin{document}
    \pagenumbering{gobble}
    \maketitle
    
    
    \begin{abstract}
        Magnetic confinement reactors---in particular tokamaks---offer one of the most promising options for achieving practical nuclear fusion, with the potential to provide virtually limitless, clean energy. The theoretical and numerical modeling of tokamak plasmas is simultaneously an essential component of effective reactor design, and a great research barrier. Tokamak operational conditions exhibit comparatively low Knudsen numbers. Kinetic effects, including kinetic waves and instabilities, Landau damping, bump-on-tail instabilities and more, are therefore highly influential in tokamak plasma dynamics. Purely fluid models are inherently incapable of capturing these effects, whereas the high dimensionality in purely kinetic models render them practically intractable for most relevant purposes.

        We consider a $\delta\!f$ decomposition model, with a macroscopic fluid background and microscopic kinetic correction, both fully coupled to each other. A similar manner of discretization is proposed to that used in the recent \texttt{STRUPHY} code \cite{Holderied_Possanner_Wang_2021, Holderied_2022, Li_et_al_2023} with a finite-element model for the background and a pseudo-particle/PiC model for the correction.

        The fluid background satisfies the full, non-linear, resistive, compressible, Hall MHD equations. \cite{Laakmann_Hu_Farrell_2022} introduces finite-element(-in-space) implicit timesteppers for the incompressible analogue to this system with structure-preserving (SP) properties in the ideal case, alongside parameter-robust preconditioners. We show that these timesteppers can derive from a finite-element-in-time (FET) (and finite-element-in-space) interpretation. The benefits of this reformulation are discussed, including the derivation of timesteppers that are higher order in time, and the quantifiable dissipative SP properties in the non-ideal, resistive case.
        
        We discuss possible options for extending this FET approach to timesteppers for the compressible case.

        The kinetic corrections satisfy linearized Boltzmann equations. Using a Lénard--Bernstein collision operator, these take Fokker--Planck-like forms \cite{Fokker_1914, Planck_1917} wherein pseudo-particles in the numerical model obey the neoclassical transport equations, with particle-independent Brownian drift terms. This offers a rigorous methodology for incorporating collisions into the particle transport model, without coupling the equations of motions for each particle.
        
        Works by Chen, Chacón et al. \cite{Chen_Chacón_Barnes_2011, Chacón_Chen_Barnes_2013, Chen_Chacón_2014, Chen_Chacón_2015} have developed structure-preserving particle pushers for neoclassical transport in the Vlasov equations, derived from Crank--Nicolson integrators. We show these too can can derive from a FET interpretation, similarly offering potential extensions to higher-order-in-time particle pushers. The FET formulation is used also to consider how the stochastic drift terms can be incorporated into the pushers. Stochastic gyrokinetic expansions are also discussed.

        Different options for the numerical implementation of these schemes are considered.

        Due to the efficacy of FET in the development of SP timesteppers for both the fluid and kinetic component, we hope this approach will prove effective in the future for developing SP timesteppers for the full hybrid model. We hope this will give us the opportunity to incorporate previously inaccessible kinetic effects into the highly effective, modern, finite-element MHD models.
    \end{abstract}
    
    
    \newpage
    \tableofcontents
    
    
    \newpage
    \pagenumbering{arabic}
    %\linenumbers\renewcommand\thelinenumber{\color{black!50}\arabic{linenumber}}
            \input{0 - introduction/main.tex}
        \part{Research}
            \input{1 - low-noise PiC models/main.tex}
            \input{2 - kinetic component/main.tex}
            \input{3 - fluid component/main.tex}
            \input{4 - numerical implementation/main.tex}
        \part{Project Overview}
            \input{5 - research plan/main.tex}
            \input{6 - summary/main.tex}
    
    
    %\section{}
    \newpage
    \pagenumbering{gobble}
        \printbibliography


    \newpage
    \pagenumbering{roman}
    \appendix
        \part{Appendices}
            \input{8 - Hilbert complexes/main.tex}
            \input{9 - weak conservation proofs/main.tex}
\end{document}

            \documentclass[12pt, a4paper]{report}

\input{template/main.tex}

\title{\BA{Title in Progress...}}
\author{Boris Andrews}
\affil{Mathematical Institute, University of Oxford}
\date{\today}


\begin{document}
    \pagenumbering{gobble}
    \maketitle
    
    
    \begin{abstract}
        Magnetic confinement reactors---in particular tokamaks---offer one of the most promising options for achieving practical nuclear fusion, with the potential to provide virtually limitless, clean energy. The theoretical and numerical modeling of tokamak plasmas is simultaneously an essential component of effective reactor design, and a great research barrier. Tokamak operational conditions exhibit comparatively low Knudsen numbers. Kinetic effects, including kinetic waves and instabilities, Landau damping, bump-on-tail instabilities and more, are therefore highly influential in tokamak plasma dynamics. Purely fluid models are inherently incapable of capturing these effects, whereas the high dimensionality in purely kinetic models render them practically intractable for most relevant purposes.

        We consider a $\delta\!f$ decomposition model, with a macroscopic fluid background and microscopic kinetic correction, both fully coupled to each other. A similar manner of discretization is proposed to that used in the recent \texttt{STRUPHY} code \cite{Holderied_Possanner_Wang_2021, Holderied_2022, Li_et_al_2023} with a finite-element model for the background and a pseudo-particle/PiC model for the correction.

        The fluid background satisfies the full, non-linear, resistive, compressible, Hall MHD equations. \cite{Laakmann_Hu_Farrell_2022} introduces finite-element(-in-space) implicit timesteppers for the incompressible analogue to this system with structure-preserving (SP) properties in the ideal case, alongside parameter-robust preconditioners. We show that these timesteppers can derive from a finite-element-in-time (FET) (and finite-element-in-space) interpretation. The benefits of this reformulation are discussed, including the derivation of timesteppers that are higher order in time, and the quantifiable dissipative SP properties in the non-ideal, resistive case.
        
        We discuss possible options for extending this FET approach to timesteppers for the compressible case.

        The kinetic corrections satisfy linearized Boltzmann equations. Using a Lénard--Bernstein collision operator, these take Fokker--Planck-like forms \cite{Fokker_1914, Planck_1917} wherein pseudo-particles in the numerical model obey the neoclassical transport equations, with particle-independent Brownian drift terms. This offers a rigorous methodology for incorporating collisions into the particle transport model, without coupling the equations of motions for each particle.
        
        Works by Chen, Chacón et al. \cite{Chen_Chacón_Barnes_2011, Chacón_Chen_Barnes_2013, Chen_Chacón_2014, Chen_Chacón_2015} have developed structure-preserving particle pushers for neoclassical transport in the Vlasov equations, derived from Crank--Nicolson integrators. We show these too can can derive from a FET interpretation, similarly offering potential extensions to higher-order-in-time particle pushers. The FET formulation is used also to consider how the stochastic drift terms can be incorporated into the pushers. Stochastic gyrokinetic expansions are also discussed.

        Different options for the numerical implementation of these schemes are considered.

        Due to the efficacy of FET in the development of SP timesteppers for both the fluid and kinetic component, we hope this approach will prove effective in the future for developing SP timesteppers for the full hybrid model. We hope this will give us the opportunity to incorporate previously inaccessible kinetic effects into the highly effective, modern, finite-element MHD models.
    \end{abstract}
    
    
    \newpage
    \tableofcontents
    
    
    \newpage
    \pagenumbering{arabic}
    %\linenumbers\renewcommand\thelinenumber{\color{black!50}\arabic{linenumber}}
            \input{0 - introduction/main.tex}
        \part{Research}
            \input{1 - low-noise PiC models/main.tex}
            \input{2 - kinetic component/main.tex}
            \input{3 - fluid component/main.tex}
            \input{4 - numerical implementation/main.tex}
        \part{Project Overview}
            \input{5 - research plan/main.tex}
            \input{6 - summary/main.tex}
    
    
    %\section{}
    \newpage
    \pagenumbering{gobble}
        \printbibliography


    \newpage
    \pagenumbering{roman}
    \appendix
        \part{Appendices}
            \input{8 - Hilbert complexes/main.tex}
            \input{9 - weak conservation proofs/main.tex}
\end{document}

    
    
    %\section{}
    \newpage
    \pagenumbering{gobble}
        \printbibliography


    \newpage
    \pagenumbering{roman}
    \appendix
        \part{Appendices}
            \documentclass[12pt, a4paper]{report}

\input{template/main.tex}

\title{\BA{Title in Progress...}}
\author{Boris Andrews}
\affil{Mathematical Institute, University of Oxford}
\date{\today}


\begin{document}
    \pagenumbering{gobble}
    \maketitle
    
    
    \begin{abstract}
        Magnetic confinement reactors---in particular tokamaks---offer one of the most promising options for achieving practical nuclear fusion, with the potential to provide virtually limitless, clean energy. The theoretical and numerical modeling of tokamak plasmas is simultaneously an essential component of effective reactor design, and a great research barrier. Tokamak operational conditions exhibit comparatively low Knudsen numbers. Kinetic effects, including kinetic waves and instabilities, Landau damping, bump-on-tail instabilities and more, are therefore highly influential in tokamak plasma dynamics. Purely fluid models are inherently incapable of capturing these effects, whereas the high dimensionality in purely kinetic models render them practically intractable for most relevant purposes.

        We consider a $\delta\!f$ decomposition model, with a macroscopic fluid background and microscopic kinetic correction, both fully coupled to each other. A similar manner of discretization is proposed to that used in the recent \texttt{STRUPHY} code \cite{Holderied_Possanner_Wang_2021, Holderied_2022, Li_et_al_2023} with a finite-element model for the background and a pseudo-particle/PiC model for the correction.

        The fluid background satisfies the full, non-linear, resistive, compressible, Hall MHD equations. \cite{Laakmann_Hu_Farrell_2022} introduces finite-element(-in-space) implicit timesteppers for the incompressible analogue to this system with structure-preserving (SP) properties in the ideal case, alongside parameter-robust preconditioners. We show that these timesteppers can derive from a finite-element-in-time (FET) (and finite-element-in-space) interpretation. The benefits of this reformulation are discussed, including the derivation of timesteppers that are higher order in time, and the quantifiable dissipative SP properties in the non-ideal, resistive case.
        
        We discuss possible options for extending this FET approach to timesteppers for the compressible case.

        The kinetic corrections satisfy linearized Boltzmann equations. Using a Lénard--Bernstein collision operator, these take Fokker--Planck-like forms \cite{Fokker_1914, Planck_1917} wherein pseudo-particles in the numerical model obey the neoclassical transport equations, with particle-independent Brownian drift terms. This offers a rigorous methodology for incorporating collisions into the particle transport model, without coupling the equations of motions for each particle.
        
        Works by Chen, Chacón et al. \cite{Chen_Chacón_Barnes_2011, Chacón_Chen_Barnes_2013, Chen_Chacón_2014, Chen_Chacón_2015} have developed structure-preserving particle pushers for neoclassical transport in the Vlasov equations, derived from Crank--Nicolson integrators. We show these too can can derive from a FET interpretation, similarly offering potential extensions to higher-order-in-time particle pushers. The FET formulation is used also to consider how the stochastic drift terms can be incorporated into the pushers. Stochastic gyrokinetic expansions are also discussed.

        Different options for the numerical implementation of these schemes are considered.

        Due to the efficacy of FET in the development of SP timesteppers for both the fluid and kinetic component, we hope this approach will prove effective in the future for developing SP timesteppers for the full hybrid model. We hope this will give us the opportunity to incorporate previously inaccessible kinetic effects into the highly effective, modern, finite-element MHD models.
    \end{abstract}
    
    
    \newpage
    \tableofcontents
    
    
    \newpage
    \pagenumbering{arabic}
    %\linenumbers\renewcommand\thelinenumber{\color{black!50}\arabic{linenumber}}
            \input{0 - introduction/main.tex}
        \part{Research}
            \input{1 - low-noise PiC models/main.tex}
            \input{2 - kinetic component/main.tex}
            \input{3 - fluid component/main.tex}
            \input{4 - numerical implementation/main.tex}
        \part{Project Overview}
            \input{5 - research plan/main.tex}
            \input{6 - summary/main.tex}
    
    
    %\section{}
    \newpage
    \pagenumbering{gobble}
        \printbibliography


    \newpage
    \pagenumbering{roman}
    \appendix
        \part{Appendices}
            \input{8 - Hilbert complexes/main.tex}
            \input{9 - weak conservation proofs/main.tex}
\end{document}

            \documentclass[12pt, a4paper]{report}

\input{template/main.tex}

\title{\BA{Title in Progress...}}
\author{Boris Andrews}
\affil{Mathematical Institute, University of Oxford}
\date{\today}


\begin{document}
    \pagenumbering{gobble}
    \maketitle
    
    
    \begin{abstract}
        Magnetic confinement reactors---in particular tokamaks---offer one of the most promising options for achieving practical nuclear fusion, with the potential to provide virtually limitless, clean energy. The theoretical and numerical modeling of tokamak plasmas is simultaneously an essential component of effective reactor design, and a great research barrier. Tokamak operational conditions exhibit comparatively low Knudsen numbers. Kinetic effects, including kinetic waves and instabilities, Landau damping, bump-on-tail instabilities and more, are therefore highly influential in tokamak plasma dynamics. Purely fluid models are inherently incapable of capturing these effects, whereas the high dimensionality in purely kinetic models render them practically intractable for most relevant purposes.

        We consider a $\delta\!f$ decomposition model, with a macroscopic fluid background and microscopic kinetic correction, both fully coupled to each other. A similar manner of discretization is proposed to that used in the recent \texttt{STRUPHY} code \cite{Holderied_Possanner_Wang_2021, Holderied_2022, Li_et_al_2023} with a finite-element model for the background and a pseudo-particle/PiC model for the correction.

        The fluid background satisfies the full, non-linear, resistive, compressible, Hall MHD equations. \cite{Laakmann_Hu_Farrell_2022} introduces finite-element(-in-space) implicit timesteppers for the incompressible analogue to this system with structure-preserving (SP) properties in the ideal case, alongside parameter-robust preconditioners. We show that these timesteppers can derive from a finite-element-in-time (FET) (and finite-element-in-space) interpretation. The benefits of this reformulation are discussed, including the derivation of timesteppers that are higher order in time, and the quantifiable dissipative SP properties in the non-ideal, resistive case.
        
        We discuss possible options for extending this FET approach to timesteppers for the compressible case.

        The kinetic corrections satisfy linearized Boltzmann equations. Using a Lénard--Bernstein collision operator, these take Fokker--Planck-like forms \cite{Fokker_1914, Planck_1917} wherein pseudo-particles in the numerical model obey the neoclassical transport equations, with particle-independent Brownian drift terms. This offers a rigorous methodology for incorporating collisions into the particle transport model, without coupling the equations of motions for each particle.
        
        Works by Chen, Chacón et al. \cite{Chen_Chacón_Barnes_2011, Chacón_Chen_Barnes_2013, Chen_Chacón_2014, Chen_Chacón_2015} have developed structure-preserving particle pushers for neoclassical transport in the Vlasov equations, derived from Crank--Nicolson integrators. We show these too can can derive from a FET interpretation, similarly offering potential extensions to higher-order-in-time particle pushers. The FET formulation is used also to consider how the stochastic drift terms can be incorporated into the pushers. Stochastic gyrokinetic expansions are also discussed.

        Different options for the numerical implementation of these schemes are considered.

        Due to the efficacy of FET in the development of SP timesteppers for both the fluid and kinetic component, we hope this approach will prove effective in the future for developing SP timesteppers for the full hybrid model. We hope this will give us the opportunity to incorporate previously inaccessible kinetic effects into the highly effective, modern, finite-element MHD models.
    \end{abstract}
    
    
    \newpage
    \tableofcontents
    
    
    \newpage
    \pagenumbering{arabic}
    %\linenumbers\renewcommand\thelinenumber{\color{black!50}\arabic{linenumber}}
            \input{0 - introduction/main.tex}
        \part{Research}
            \input{1 - low-noise PiC models/main.tex}
            \input{2 - kinetic component/main.tex}
            \input{3 - fluid component/main.tex}
            \input{4 - numerical implementation/main.tex}
        \part{Project Overview}
            \input{5 - research plan/main.tex}
            \input{6 - summary/main.tex}
    
    
    %\section{}
    \newpage
    \pagenumbering{gobble}
        \printbibliography


    \newpage
    \pagenumbering{roman}
    \appendix
        \part{Appendices}
            \input{8 - Hilbert complexes/main.tex}
            \input{9 - weak conservation proofs/main.tex}
\end{document}

\end{document}


\title{\BA{Title in Progress...}}
\author{Boris Andrews}
\affil{Mathematical Institute, University of Oxford}
\date{\today}


\begin{document}
    \pagenumbering{gobble}
    \maketitle
    
    
    \begin{abstract}
        Magnetic confinement reactors---in particular tokamaks---offer one of the most promising options for achieving practical nuclear fusion, with the potential to provide virtually limitless, clean energy. The theoretical and numerical modeling of tokamak plasmas is simultaneously an essential component of effective reactor design, and a great research barrier. Tokamak operational conditions exhibit comparatively low Knudsen numbers. Kinetic effects, including kinetic waves and instabilities, Landau damping, bump-on-tail instabilities and more, are therefore highly influential in tokamak plasma dynamics. Purely fluid models are inherently incapable of capturing these effects, whereas the high dimensionality in purely kinetic models render them practically intractable for most relevant purposes.

        We consider a $\delta\!f$ decomposition model, with a macroscopic fluid background and microscopic kinetic correction, both fully coupled to each other. A similar manner of discretization is proposed to that used in the recent \texttt{STRUPHY} code \cite{Holderied_Possanner_Wang_2021, Holderied_2022, Li_et_al_2023} with a finite-element model for the background and a pseudo-particle/PiC model for the correction.

        The fluid background satisfies the full, non-linear, resistive, compressible, Hall MHD equations. \cite{Laakmann_Hu_Farrell_2022} introduces finite-element(-in-space) implicit timesteppers for the incompressible analogue to this system with structure-preserving (SP) properties in the ideal case, alongside parameter-robust preconditioners. We show that these timesteppers can derive from a finite-element-in-time (FET) (and finite-element-in-space) interpretation. The benefits of this reformulation are discussed, including the derivation of timesteppers that are higher order in time, and the quantifiable dissipative SP properties in the non-ideal, resistive case.
        
        We discuss possible options for extending this FET approach to timesteppers for the compressible case.

        The kinetic corrections satisfy linearized Boltzmann equations. Using a Lénard--Bernstein collision operator, these take Fokker--Planck-like forms \cite{Fokker_1914, Planck_1917} wherein pseudo-particles in the numerical model obey the neoclassical transport equations, with particle-independent Brownian drift terms. This offers a rigorous methodology for incorporating collisions into the particle transport model, without coupling the equations of motions for each particle.
        
        Works by Chen, Chacón et al. \cite{Chen_Chacón_Barnes_2011, Chacón_Chen_Barnes_2013, Chen_Chacón_2014, Chen_Chacón_2015} have developed structure-preserving particle pushers for neoclassical transport in the Vlasov equations, derived from Crank--Nicolson integrators. We show these too can can derive from a FET interpretation, similarly offering potential extensions to higher-order-in-time particle pushers. The FET formulation is used also to consider how the stochastic drift terms can be incorporated into the pushers. Stochastic gyrokinetic expansions are also discussed.

        Different options for the numerical implementation of these schemes are considered.

        Due to the efficacy of FET in the development of SP timesteppers for both the fluid and kinetic component, we hope this approach will prove effective in the future for developing SP timesteppers for the full hybrid model. We hope this will give us the opportunity to incorporate previously inaccessible kinetic effects into the highly effective, modern, finite-element MHD models.
    \end{abstract}
    
    
    \newpage
    \tableofcontents
    
    
    \newpage
    \pagenumbering{arabic}
    %\linenumbers\renewcommand\thelinenumber{\color{black!50}\arabic{linenumber}}
            \documentclass[12pt, a4paper]{report}

\documentclass[12pt, a4paper]{report}

\input{template/main.tex}

\title{\BA{Title in Progress...}}
\author{Boris Andrews}
\affil{Mathematical Institute, University of Oxford}
\date{\today}


\begin{document}
    \pagenumbering{gobble}
    \maketitle
    
    
    \begin{abstract}
        Magnetic confinement reactors---in particular tokamaks---offer one of the most promising options for achieving practical nuclear fusion, with the potential to provide virtually limitless, clean energy. The theoretical and numerical modeling of tokamak plasmas is simultaneously an essential component of effective reactor design, and a great research barrier. Tokamak operational conditions exhibit comparatively low Knudsen numbers. Kinetic effects, including kinetic waves and instabilities, Landau damping, bump-on-tail instabilities and more, are therefore highly influential in tokamak plasma dynamics. Purely fluid models are inherently incapable of capturing these effects, whereas the high dimensionality in purely kinetic models render them practically intractable for most relevant purposes.

        We consider a $\delta\!f$ decomposition model, with a macroscopic fluid background and microscopic kinetic correction, both fully coupled to each other. A similar manner of discretization is proposed to that used in the recent \texttt{STRUPHY} code \cite{Holderied_Possanner_Wang_2021, Holderied_2022, Li_et_al_2023} with a finite-element model for the background and a pseudo-particle/PiC model for the correction.

        The fluid background satisfies the full, non-linear, resistive, compressible, Hall MHD equations. \cite{Laakmann_Hu_Farrell_2022} introduces finite-element(-in-space) implicit timesteppers for the incompressible analogue to this system with structure-preserving (SP) properties in the ideal case, alongside parameter-robust preconditioners. We show that these timesteppers can derive from a finite-element-in-time (FET) (and finite-element-in-space) interpretation. The benefits of this reformulation are discussed, including the derivation of timesteppers that are higher order in time, and the quantifiable dissipative SP properties in the non-ideal, resistive case.
        
        We discuss possible options for extending this FET approach to timesteppers for the compressible case.

        The kinetic corrections satisfy linearized Boltzmann equations. Using a Lénard--Bernstein collision operator, these take Fokker--Planck-like forms \cite{Fokker_1914, Planck_1917} wherein pseudo-particles in the numerical model obey the neoclassical transport equations, with particle-independent Brownian drift terms. This offers a rigorous methodology for incorporating collisions into the particle transport model, without coupling the equations of motions for each particle.
        
        Works by Chen, Chacón et al. \cite{Chen_Chacón_Barnes_2011, Chacón_Chen_Barnes_2013, Chen_Chacón_2014, Chen_Chacón_2015} have developed structure-preserving particle pushers for neoclassical transport in the Vlasov equations, derived from Crank--Nicolson integrators. We show these too can can derive from a FET interpretation, similarly offering potential extensions to higher-order-in-time particle pushers. The FET formulation is used also to consider how the stochastic drift terms can be incorporated into the pushers. Stochastic gyrokinetic expansions are also discussed.

        Different options for the numerical implementation of these schemes are considered.

        Due to the efficacy of FET in the development of SP timesteppers for both the fluid and kinetic component, we hope this approach will prove effective in the future for developing SP timesteppers for the full hybrid model. We hope this will give us the opportunity to incorporate previously inaccessible kinetic effects into the highly effective, modern, finite-element MHD models.
    \end{abstract}
    
    
    \newpage
    \tableofcontents
    
    
    \newpage
    \pagenumbering{arabic}
    %\linenumbers\renewcommand\thelinenumber{\color{black!50}\arabic{linenumber}}
            \input{0 - introduction/main.tex}
        \part{Research}
            \input{1 - low-noise PiC models/main.tex}
            \input{2 - kinetic component/main.tex}
            \input{3 - fluid component/main.tex}
            \input{4 - numerical implementation/main.tex}
        \part{Project Overview}
            \input{5 - research plan/main.tex}
            \input{6 - summary/main.tex}
    
    
    %\section{}
    \newpage
    \pagenumbering{gobble}
        \printbibliography


    \newpage
    \pagenumbering{roman}
    \appendix
        \part{Appendices}
            \input{8 - Hilbert complexes/main.tex}
            \input{9 - weak conservation proofs/main.tex}
\end{document}


\title{\BA{Title in Progress...}}
\author{Boris Andrews}
\affil{Mathematical Institute, University of Oxford}
\date{\today}


\begin{document}
    \pagenumbering{gobble}
    \maketitle
    
    
    \begin{abstract}
        Magnetic confinement reactors---in particular tokamaks---offer one of the most promising options for achieving practical nuclear fusion, with the potential to provide virtually limitless, clean energy. The theoretical and numerical modeling of tokamak plasmas is simultaneously an essential component of effective reactor design, and a great research barrier. Tokamak operational conditions exhibit comparatively low Knudsen numbers. Kinetic effects, including kinetic waves and instabilities, Landau damping, bump-on-tail instabilities and more, are therefore highly influential in tokamak plasma dynamics. Purely fluid models are inherently incapable of capturing these effects, whereas the high dimensionality in purely kinetic models render them practically intractable for most relevant purposes.

        We consider a $\delta\!f$ decomposition model, with a macroscopic fluid background and microscopic kinetic correction, both fully coupled to each other. A similar manner of discretization is proposed to that used in the recent \texttt{STRUPHY} code \cite{Holderied_Possanner_Wang_2021, Holderied_2022, Li_et_al_2023} with a finite-element model for the background and a pseudo-particle/PiC model for the correction.

        The fluid background satisfies the full, non-linear, resistive, compressible, Hall MHD equations. \cite{Laakmann_Hu_Farrell_2022} introduces finite-element(-in-space) implicit timesteppers for the incompressible analogue to this system with structure-preserving (SP) properties in the ideal case, alongside parameter-robust preconditioners. We show that these timesteppers can derive from a finite-element-in-time (FET) (and finite-element-in-space) interpretation. The benefits of this reformulation are discussed, including the derivation of timesteppers that are higher order in time, and the quantifiable dissipative SP properties in the non-ideal, resistive case.
        
        We discuss possible options for extending this FET approach to timesteppers for the compressible case.

        The kinetic corrections satisfy linearized Boltzmann equations. Using a Lénard--Bernstein collision operator, these take Fokker--Planck-like forms \cite{Fokker_1914, Planck_1917} wherein pseudo-particles in the numerical model obey the neoclassical transport equations, with particle-independent Brownian drift terms. This offers a rigorous methodology for incorporating collisions into the particle transport model, without coupling the equations of motions for each particle.
        
        Works by Chen, Chacón et al. \cite{Chen_Chacón_Barnes_2011, Chacón_Chen_Barnes_2013, Chen_Chacón_2014, Chen_Chacón_2015} have developed structure-preserving particle pushers for neoclassical transport in the Vlasov equations, derived from Crank--Nicolson integrators. We show these too can can derive from a FET interpretation, similarly offering potential extensions to higher-order-in-time particle pushers. The FET formulation is used also to consider how the stochastic drift terms can be incorporated into the pushers. Stochastic gyrokinetic expansions are also discussed.

        Different options for the numerical implementation of these schemes are considered.

        Due to the efficacy of FET in the development of SP timesteppers for both the fluid and kinetic component, we hope this approach will prove effective in the future for developing SP timesteppers for the full hybrid model. We hope this will give us the opportunity to incorporate previously inaccessible kinetic effects into the highly effective, modern, finite-element MHD models.
    \end{abstract}
    
    
    \newpage
    \tableofcontents
    
    
    \newpage
    \pagenumbering{arabic}
    %\linenumbers\renewcommand\thelinenumber{\color{black!50}\arabic{linenumber}}
            \documentclass[12pt, a4paper]{report}

\input{template/main.tex}

\title{\BA{Title in Progress...}}
\author{Boris Andrews}
\affil{Mathematical Institute, University of Oxford}
\date{\today}


\begin{document}
    \pagenumbering{gobble}
    \maketitle
    
    
    \begin{abstract}
        Magnetic confinement reactors---in particular tokamaks---offer one of the most promising options for achieving practical nuclear fusion, with the potential to provide virtually limitless, clean energy. The theoretical and numerical modeling of tokamak plasmas is simultaneously an essential component of effective reactor design, and a great research barrier. Tokamak operational conditions exhibit comparatively low Knudsen numbers. Kinetic effects, including kinetic waves and instabilities, Landau damping, bump-on-tail instabilities and more, are therefore highly influential in tokamak plasma dynamics. Purely fluid models are inherently incapable of capturing these effects, whereas the high dimensionality in purely kinetic models render them practically intractable for most relevant purposes.

        We consider a $\delta\!f$ decomposition model, with a macroscopic fluid background and microscopic kinetic correction, both fully coupled to each other. A similar manner of discretization is proposed to that used in the recent \texttt{STRUPHY} code \cite{Holderied_Possanner_Wang_2021, Holderied_2022, Li_et_al_2023} with a finite-element model for the background and a pseudo-particle/PiC model for the correction.

        The fluid background satisfies the full, non-linear, resistive, compressible, Hall MHD equations. \cite{Laakmann_Hu_Farrell_2022} introduces finite-element(-in-space) implicit timesteppers for the incompressible analogue to this system with structure-preserving (SP) properties in the ideal case, alongside parameter-robust preconditioners. We show that these timesteppers can derive from a finite-element-in-time (FET) (and finite-element-in-space) interpretation. The benefits of this reformulation are discussed, including the derivation of timesteppers that are higher order in time, and the quantifiable dissipative SP properties in the non-ideal, resistive case.
        
        We discuss possible options for extending this FET approach to timesteppers for the compressible case.

        The kinetic corrections satisfy linearized Boltzmann equations. Using a Lénard--Bernstein collision operator, these take Fokker--Planck-like forms \cite{Fokker_1914, Planck_1917} wherein pseudo-particles in the numerical model obey the neoclassical transport equations, with particle-independent Brownian drift terms. This offers a rigorous methodology for incorporating collisions into the particle transport model, without coupling the equations of motions for each particle.
        
        Works by Chen, Chacón et al. \cite{Chen_Chacón_Barnes_2011, Chacón_Chen_Barnes_2013, Chen_Chacón_2014, Chen_Chacón_2015} have developed structure-preserving particle pushers for neoclassical transport in the Vlasov equations, derived from Crank--Nicolson integrators. We show these too can can derive from a FET interpretation, similarly offering potential extensions to higher-order-in-time particle pushers. The FET formulation is used also to consider how the stochastic drift terms can be incorporated into the pushers. Stochastic gyrokinetic expansions are also discussed.

        Different options for the numerical implementation of these schemes are considered.

        Due to the efficacy of FET in the development of SP timesteppers for both the fluid and kinetic component, we hope this approach will prove effective in the future for developing SP timesteppers for the full hybrid model. We hope this will give us the opportunity to incorporate previously inaccessible kinetic effects into the highly effective, modern, finite-element MHD models.
    \end{abstract}
    
    
    \newpage
    \tableofcontents
    
    
    \newpage
    \pagenumbering{arabic}
    %\linenumbers\renewcommand\thelinenumber{\color{black!50}\arabic{linenumber}}
            \input{0 - introduction/main.tex}
        \part{Research}
            \input{1 - low-noise PiC models/main.tex}
            \input{2 - kinetic component/main.tex}
            \input{3 - fluid component/main.tex}
            \input{4 - numerical implementation/main.tex}
        \part{Project Overview}
            \input{5 - research plan/main.tex}
            \input{6 - summary/main.tex}
    
    
    %\section{}
    \newpage
    \pagenumbering{gobble}
        \printbibliography


    \newpage
    \pagenumbering{roman}
    \appendix
        \part{Appendices}
            \input{8 - Hilbert complexes/main.tex}
            \input{9 - weak conservation proofs/main.tex}
\end{document}

        \part{Research}
            \documentclass[12pt, a4paper]{report}

\input{template/main.tex}

\title{\BA{Title in Progress...}}
\author{Boris Andrews}
\affil{Mathematical Institute, University of Oxford}
\date{\today}


\begin{document}
    \pagenumbering{gobble}
    \maketitle
    
    
    \begin{abstract}
        Magnetic confinement reactors---in particular tokamaks---offer one of the most promising options for achieving practical nuclear fusion, with the potential to provide virtually limitless, clean energy. The theoretical and numerical modeling of tokamak plasmas is simultaneously an essential component of effective reactor design, and a great research barrier. Tokamak operational conditions exhibit comparatively low Knudsen numbers. Kinetic effects, including kinetic waves and instabilities, Landau damping, bump-on-tail instabilities and more, are therefore highly influential in tokamak plasma dynamics. Purely fluid models are inherently incapable of capturing these effects, whereas the high dimensionality in purely kinetic models render them practically intractable for most relevant purposes.

        We consider a $\delta\!f$ decomposition model, with a macroscopic fluid background and microscopic kinetic correction, both fully coupled to each other. A similar manner of discretization is proposed to that used in the recent \texttt{STRUPHY} code \cite{Holderied_Possanner_Wang_2021, Holderied_2022, Li_et_al_2023} with a finite-element model for the background and a pseudo-particle/PiC model for the correction.

        The fluid background satisfies the full, non-linear, resistive, compressible, Hall MHD equations. \cite{Laakmann_Hu_Farrell_2022} introduces finite-element(-in-space) implicit timesteppers for the incompressible analogue to this system with structure-preserving (SP) properties in the ideal case, alongside parameter-robust preconditioners. We show that these timesteppers can derive from a finite-element-in-time (FET) (and finite-element-in-space) interpretation. The benefits of this reformulation are discussed, including the derivation of timesteppers that are higher order in time, and the quantifiable dissipative SP properties in the non-ideal, resistive case.
        
        We discuss possible options for extending this FET approach to timesteppers for the compressible case.

        The kinetic corrections satisfy linearized Boltzmann equations. Using a Lénard--Bernstein collision operator, these take Fokker--Planck-like forms \cite{Fokker_1914, Planck_1917} wherein pseudo-particles in the numerical model obey the neoclassical transport equations, with particle-independent Brownian drift terms. This offers a rigorous methodology for incorporating collisions into the particle transport model, without coupling the equations of motions for each particle.
        
        Works by Chen, Chacón et al. \cite{Chen_Chacón_Barnes_2011, Chacón_Chen_Barnes_2013, Chen_Chacón_2014, Chen_Chacón_2015} have developed structure-preserving particle pushers for neoclassical transport in the Vlasov equations, derived from Crank--Nicolson integrators. We show these too can can derive from a FET interpretation, similarly offering potential extensions to higher-order-in-time particle pushers. The FET formulation is used also to consider how the stochastic drift terms can be incorporated into the pushers. Stochastic gyrokinetic expansions are also discussed.

        Different options for the numerical implementation of these schemes are considered.

        Due to the efficacy of FET in the development of SP timesteppers for both the fluid and kinetic component, we hope this approach will prove effective in the future for developing SP timesteppers for the full hybrid model. We hope this will give us the opportunity to incorporate previously inaccessible kinetic effects into the highly effective, modern, finite-element MHD models.
    \end{abstract}
    
    
    \newpage
    \tableofcontents
    
    
    \newpage
    \pagenumbering{arabic}
    %\linenumbers\renewcommand\thelinenumber{\color{black!50}\arabic{linenumber}}
            \input{0 - introduction/main.tex}
        \part{Research}
            \input{1 - low-noise PiC models/main.tex}
            \input{2 - kinetic component/main.tex}
            \input{3 - fluid component/main.tex}
            \input{4 - numerical implementation/main.tex}
        \part{Project Overview}
            \input{5 - research plan/main.tex}
            \input{6 - summary/main.tex}
    
    
    %\section{}
    \newpage
    \pagenumbering{gobble}
        \printbibliography


    \newpage
    \pagenumbering{roman}
    \appendix
        \part{Appendices}
            \input{8 - Hilbert complexes/main.tex}
            \input{9 - weak conservation proofs/main.tex}
\end{document}

            \documentclass[12pt, a4paper]{report}

\input{template/main.tex}

\title{\BA{Title in Progress...}}
\author{Boris Andrews}
\affil{Mathematical Institute, University of Oxford}
\date{\today}


\begin{document}
    \pagenumbering{gobble}
    \maketitle
    
    
    \begin{abstract}
        Magnetic confinement reactors---in particular tokamaks---offer one of the most promising options for achieving practical nuclear fusion, with the potential to provide virtually limitless, clean energy. The theoretical and numerical modeling of tokamak plasmas is simultaneously an essential component of effective reactor design, and a great research barrier. Tokamak operational conditions exhibit comparatively low Knudsen numbers. Kinetic effects, including kinetic waves and instabilities, Landau damping, bump-on-tail instabilities and more, are therefore highly influential in tokamak plasma dynamics. Purely fluid models are inherently incapable of capturing these effects, whereas the high dimensionality in purely kinetic models render them practically intractable for most relevant purposes.

        We consider a $\delta\!f$ decomposition model, with a macroscopic fluid background and microscopic kinetic correction, both fully coupled to each other. A similar manner of discretization is proposed to that used in the recent \texttt{STRUPHY} code \cite{Holderied_Possanner_Wang_2021, Holderied_2022, Li_et_al_2023} with a finite-element model for the background and a pseudo-particle/PiC model for the correction.

        The fluid background satisfies the full, non-linear, resistive, compressible, Hall MHD equations. \cite{Laakmann_Hu_Farrell_2022} introduces finite-element(-in-space) implicit timesteppers for the incompressible analogue to this system with structure-preserving (SP) properties in the ideal case, alongside parameter-robust preconditioners. We show that these timesteppers can derive from a finite-element-in-time (FET) (and finite-element-in-space) interpretation. The benefits of this reformulation are discussed, including the derivation of timesteppers that are higher order in time, and the quantifiable dissipative SP properties in the non-ideal, resistive case.
        
        We discuss possible options for extending this FET approach to timesteppers for the compressible case.

        The kinetic corrections satisfy linearized Boltzmann equations. Using a Lénard--Bernstein collision operator, these take Fokker--Planck-like forms \cite{Fokker_1914, Planck_1917} wherein pseudo-particles in the numerical model obey the neoclassical transport equations, with particle-independent Brownian drift terms. This offers a rigorous methodology for incorporating collisions into the particle transport model, without coupling the equations of motions for each particle.
        
        Works by Chen, Chacón et al. \cite{Chen_Chacón_Barnes_2011, Chacón_Chen_Barnes_2013, Chen_Chacón_2014, Chen_Chacón_2015} have developed structure-preserving particle pushers for neoclassical transport in the Vlasov equations, derived from Crank--Nicolson integrators. We show these too can can derive from a FET interpretation, similarly offering potential extensions to higher-order-in-time particle pushers. The FET formulation is used also to consider how the stochastic drift terms can be incorporated into the pushers. Stochastic gyrokinetic expansions are also discussed.

        Different options for the numerical implementation of these schemes are considered.

        Due to the efficacy of FET in the development of SP timesteppers for both the fluid and kinetic component, we hope this approach will prove effective in the future for developing SP timesteppers for the full hybrid model. We hope this will give us the opportunity to incorporate previously inaccessible kinetic effects into the highly effective, modern, finite-element MHD models.
    \end{abstract}
    
    
    \newpage
    \tableofcontents
    
    
    \newpage
    \pagenumbering{arabic}
    %\linenumbers\renewcommand\thelinenumber{\color{black!50}\arabic{linenumber}}
            \input{0 - introduction/main.tex}
        \part{Research}
            \input{1 - low-noise PiC models/main.tex}
            \input{2 - kinetic component/main.tex}
            \input{3 - fluid component/main.tex}
            \input{4 - numerical implementation/main.tex}
        \part{Project Overview}
            \input{5 - research plan/main.tex}
            \input{6 - summary/main.tex}
    
    
    %\section{}
    \newpage
    \pagenumbering{gobble}
        \printbibliography


    \newpage
    \pagenumbering{roman}
    \appendix
        \part{Appendices}
            \input{8 - Hilbert complexes/main.tex}
            \input{9 - weak conservation proofs/main.tex}
\end{document}

            \documentclass[12pt, a4paper]{report}

\input{template/main.tex}

\title{\BA{Title in Progress...}}
\author{Boris Andrews}
\affil{Mathematical Institute, University of Oxford}
\date{\today}


\begin{document}
    \pagenumbering{gobble}
    \maketitle
    
    
    \begin{abstract}
        Magnetic confinement reactors---in particular tokamaks---offer one of the most promising options for achieving practical nuclear fusion, with the potential to provide virtually limitless, clean energy. The theoretical and numerical modeling of tokamak plasmas is simultaneously an essential component of effective reactor design, and a great research barrier. Tokamak operational conditions exhibit comparatively low Knudsen numbers. Kinetic effects, including kinetic waves and instabilities, Landau damping, bump-on-tail instabilities and more, are therefore highly influential in tokamak plasma dynamics. Purely fluid models are inherently incapable of capturing these effects, whereas the high dimensionality in purely kinetic models render them practically intractable for most relevant purposes.

        We consider a $\delta\!f$ decomposition model, with a macroscopic fluid background and microscopic kinetic correction, both fully coupled to each other. A similar manner of discretization is proposed to that used in the recent \texttt{STRUPHY} code \cite{Holderied_Possanner_Wang_2021, Holderied_2022, Li_et_al_2023} with a finite-element model for the background and a pseudo-particle/PiC model for the correction.

        The fluid background satisfies the full, non-linear, resistive, compressible, Hall MHD equations. \cite{Laakmann_Hu_Farrell_2022} introduces finite-element(-in-space) implicit timesteppers for the incompressible analogue to this system with structure-preserving (SP) properties in the ideal case, alongside parameter-robust preconditioners. We show that these timesteppers can derive from a finite-element-in-time (FET) (and finite-element-in-space) interpretation. The benefits of this reformulation are discussed, including the derivation of timesteppers that are higher order in time, and the quantifiable dissipative SP properties in the non-ideal, resistive case.
        
        We discuss possible options for extending this FET approach to timesteppers for the compressible case.

        The kinetic corrections satisfy linearized Boltzmann equations. Using a Lénard--Bernstein collision operator, these take Fokker--Planck-like forms \cite{Fokker_1914, Planck_1917} wherein pseudo-particles in the numerical model obey the neoclassical transport equations, with particle-independent Brownian drift terms. This offers a rigorous methodology for incorporating collisions into the particle transport model, without coupling the equations of motions for each particle.
        
        Works by Chen, Chacón et al. \cite{Chen_Chacón_Barnes_2011, Chacón_Chen_Barnes_2013, Chen_Chacón_2014, Chen_Chacón_2015} have developed structure-preserving particle pushers for neoclassical transport in the Vlasov equations, derived from Crank--Nicolson integrators. We show these too can can derive from a FET interpretation, similarly offering potential extensions to higher-order-in-time particle pushers. The FET formulation is used also to consider how the stochastic drift terms can be incorporated into the pushers. Stochastic gyrokinetic expansions are also discussed.

        Different options for the numerical implementation of these schemes are considered.

        Due to the efficacy of FET in the development of SP timesteppers for both the fluid and kinetic component, we hope this approach will prove effective in the future for developing SP timesteppers for the full hybrid model. We hope this will give us the opportunity to incorporate previously inaccessible kinetic effects into the highly effective, modern, finite-element MHD models.
    \end{abstract}
    
    
    \newpage
    \tableofcontents
    
    
    \newpage
    \pagenumbering{arabic}
    %\linenumbers\renewcommand\thelinenumber{\color{black!50}\arabic{linenumber}}
            \input{0 - introduction/main.tex}
        \part{Research}
            \input{1 - low-noise PiC models/main.tex}
            \input{2 - kinetic component/main.tex}
            \input{3 - fluid component/main.tex}
            \input{4 - numerical implementation/main.tex}
        \part{Project Overview}
            \input{5 - research plan/main.tex}
            \input{6 - summary/main.tex}
    
    
    %\section{}
    \newpage
    \pagenumbering{gobble}
        \printbibliography


    \newpage
    \pagenumbering{roman}
    \appendix
        \part{Appendices}
            \input{8 - Hilbert complexes/main.tex}
            \input{9 - weak conservation proofs/main.tex}
\end{document}

            \documentclass[12pt, a4paper]{report}

\input{template/main.tex}

\title{\BA{Title in Progress...}}
\author{Boris Andrews}
\affil{Mathematical Institute, University of Oxford}
\date{\today}


\begin{document}
    \pagenumbering{gobble}
    \maketitle
    
    
    \begin{abstract}
        Magnetic confinement reactors---in particular tokamaks---offer one of the most promising options for achieving practical nuclear fusion, with the potential to provide virtually limitless, clean energy. The theoretical and numerical modeling of tokamak plasmas is simultaneously an essential component of effective reactor design, and a great research barrier. Tokamak operational conditions exhibit comparatively low Knudsen numbers. Kinetic effects, including kinetic waves and instabilities, Landau damping, bump-on-tail instabilities and more, are therefore highly influential in tokamak plasma dynamics. Purely fluid models are inherently incapable of capturing these effects, whereas the high dimensionality in purely kinetic models render them practically intractable for most relevant purposes.

        We consider a $\delta\!f$ decomposition model, with a macroscopic fluid background and microscopic kinetic correction, both fully coupled to each other. A similar manner of discretization is proposed to that used in the recent \texttt{STRUPHY} code \cite{Holderied_Possanner_Wang_2021, Holderied_2022, Li_et_al_2023} with a finite-element model for the background and a pseudo-particle/PiC model for the correction.

        The fluid background satisfies the full, non-linear, resistive, compressible, Hall MHD equations. \cite{Laakmann_Hu_Farrell_2022} introduces finite-element(-in-space) implicit timesteppers for the incompressible analogue to this system with structure-preserving (SP) properties in the ideal case, alongside parameter-robust preconditioners. We show that these timesteppers can derive from a finite-element-in-time (FET) (and finite-element-in-space) interpretation. The benefits of this reformulation are discussed, including the derivation of timesteppers that are higher order in time, and the quantifiable dissipative SP properties in the non-ideal, resistive case.
        
        We discuss possible options for extending this FET approach to timesteppers for the compressible case.

        The kinetic corrections satisfy linearized Boltzmann equations. Using a Lénard--Bernstein collision operator, these take Fokker--Planck-like forms \cite{Fokker_1914, Planck_1917} wherein pseudo-particles in the numerical model obey the neoclassical transport equations, with particle-independent Brownian drift terms. This offers a rigorous methodology for incorporating collisions into the particle transport model, without coupling the equations of motions for each particle.
        
        Works by Chen, Chacón et al. \cite{Chen_Chacón_Barnes_2011, Chacón_Chen_Barnes_2013, Chen_Chacón_2014, Chen_Chacón_2015} have developed structure-preserving particle pushers for neoclassical transport in the Vlasov equations, derived from Crank--Nicolson integrators. We show these too can can derive from a FET interpretation, similarly offering potential extensions to higher-order-in-time particle pushers. The FET formulation is used also to consider how the stochastic drift terms can be incorporated into the pushers. Stochastic gyrokinetic expansions are also discussed.

        Different options for the numerical implementation of these schemes are considered.

        Due to the efficacy of FET in the development of SP timesteppers for both the fluid and kinetic component, we hope this approach will prove effective in the future for developing SP timesteppers for the full hybrid model. We hope this will give us the opportunity to incorporate previously inaccessible kinetic effects into the highly effective, modern, finite-element MHD models.
    \end{abstract}
    
    
    \newpage
    \tableofcontents
    
    
    \newpage
    \pagenumbering{arabic}
    %\linenumbers\renewcommand\thelinenumber{\color{black!50}\arabic{linenumber}}
            \input{0 - introduction/main.tex}
        \part{Research}
            \input{1 - low-noise PiC models/main.tex}
            \input{2 - kinetic component/main.tex}
            \input{3 - fluid component/main.tex}
            \input{4 - numerical implementation/main.tex}
        \part{Project Overview}
            \input{5 - research plan/main.tex}
            \input{6 - summary/main.tex}
    
    
    %\section{}
    \newpage
    \pagenumbering{gobble}
        \printbibliography


    \newpage
    \pagenumbering{roman}
    \appendix
        \part{Appendices}
            \input{8 - Hilbert complexes/main.tex}
            \input{9 - weak conservation proofs/main.tex}
\end{document}

        \part{Project Overview}
            \documentclass[12pt, a4paper]{report}

\input{template/main.tex}

\title{\BA{Title in Progress...}}
\author{Boris Andrews}
\affil{Mathematical Institute, University of Oxford}
\date{\today}


\begin{document}
    \pagenumbering{gobble}
    \maketitle
    
    
    \begin{abstract}
        Magnetic confinement reactors---in particular tokamaks---offer one of the most promising options for achieving practical nuclear fusion, with the potential to provide virtually limitless, clean energy. The theoretical and numerical modeling of tokamak plasmas is simultaneously an essential component of effective reactor design, and a great research barrier. Tokamak operational conditions exhibit comparatively low Knudsen numbers. Kinetic effects, including kinetic waves and instabilities, Landau damping, bump-on-tail instabilities and more, are therefore highly influential in tokamak plasma dynamics. Purely fluid models are inherently incapable of capturing these effects, whereas the high dimensionality in purely kinetic models render them practically intractable for most relevant purposes.

        We consider a $\delta\!f$ decomposition model, with a macroscopic fluid background and microscopic kinetic correction, both fully coupled to each other. A similar manner of discretization is proposed to that used in the recent \texttt{STRUPHY} code \cite{Holderied_Possanner_Wang_2021, Holderied_2022, Li_et_al_2023} with a finite-element model for the background and a pseudo-particle/PiC model for the correction.

        The fluid background satisfies the full, non-linear, resistive, compressible, Hall MHD equations. \cite{Laakmann_Hu_Farrell_2022} introduces finite-element(-in-space) implicit timesteppers for the incompressible analogue to this system with structure-preserving (SP) properties in the ideal case, alongside parameter-robust preconditioners. We show that these timesteppers can derive from a finite-element-in-time (FET) (and finite-element-in-space) interpretation. The benefits of this reformulation are discussed, including the derivation of timesteppers that are higher order in time, and the quantifiable dissipative SP properties in the non-ideal, resistive case.
        
        We discuss possible options for extending this FET approach to timesteppers for the compressible case.

        The kinetic corrections satisfy linearized Boltzmann equations. Using a Lénard--Bernstein collision operator, these take Fokker--Planck-like forms \cite{Fokker_1914, Planck_1917} wherein pseudo-particles in the numerical model obey the neoclassical transport equations, with particle-independent Brownian drift terms. This offers a rigorous methodology for incorporating collisions into the particle transport model, without coupling the equations of motions for each particle.
        
        Works by Chen, Chacón et al. \cite{Chen_Chacón_Barnes_2011, Chacón_Chen_Barnes_2013, Chen_Chacón_2014, Chen_Chacón_2015} have developed structure-preserving particle pushers for neoclassical transport in the Vlasov equations, derived from Crank--Nicolson integrators. We show these too can can derive from a FET interpretation, similarly offering potential extensions to higher-order-in-time particle pushers. The FET formulation is used also to consider how the stochastic drift terms can be incorporated into the pushers. Stochastic gyrokinetic expansions are also discussed.

        Different options for the numerical implementation of these schemes are considered.

        Due to the efficacy of FET in the development of SP timesteppers for both the fluid and kinetic component, we hope this approach will prove effective in the future for developing SP timesteppers for the full hybrid model. We hope this will give us the opportunity to incorporate previously inaccessible kinetic effects into the highly effective, modern, finite-element MHD models.
    \end{abstract}
    
    
    \newpage
    \tableofcontents
    
    
    \newpage
    \pagenumbering{arabic}
    %\linenumbers\renewcommand\thelinenumber{\color{black!50}\arabic{linenumber}}
            \input{0 - introduction/main.tex}
        \part{Research}
            \input{1 - low-noise PiC models/main.tex}
            \input{2 - kinetic component/main.tex}
            \input{3 - fluid component/main.tex}
            \input{4 - numerical implementation/main.tex}
        \part{Project Overview}
            \input{5 - research plan/main.tex}
            \input{6 - summary/main.tex}
    
    
    %\section{}
    \newpage
    \pagenumbering{gobble}
        \printbibliography


    \newpage
    \pagenumbering{roman}
    \appendix
        \part{Appendices}
            \input{8 - Hilbert complexes/main.tex}
            \input{9 - weak conservation proofs/main.tex}
\end{document}

            \documentclass[12pt, a4paper]{report}

\input{template/main.tex}

\title{\BA{Title in Progress...}}
\author{Boris Andrews}
\affil{Mathematical Institute, University of Oxford}
\date{\today}


\begin{document}
    \pagenumbering{gobble}
    \maketitle
    
    
    \begin{abstract}
        Magnetic confinement reactors---in particular tokamaks---offer one of the most promising options for achieving practical nuclear fusion, with the potential to provide virtually limitless, clean energy. The theoretical and numerical modeling of tokamak plasmas is simultaneously an essential component of effective reactor design, and a great research barrier. Tokamak operational conditions exhibit comparatively low Knudsen numbers. Kinetic effects, including kinetic waves and instabilities, Landau damping, bump-on-tail instabilities and more, are therefore highly influential in tokamak plasma dynamics. Purely fluid models are inherently incapable of capturing these effects, whereas the high dimensionality in purely kinetic models render them practically intractable for most relevant purposes.

        We consider a $\delta\!f$ decomposition model, with a macroscopic fluid background and microscopic kinetic correction, both fully coupled to each other. A similar manner of discretization is proposed to that used in the recent \texttt{STRUPHY} code \cite{Holderied_Possanner_Wang_2021, Holderied_2022, Li_et_al_2023} with a finite-element model for the background and a pseudo-particle/PiC model for the correction.

        The fluid background satisfies the full, non-linear, resistive, compressible, Hall MHD equations. \cite{Laakmann_Hu_Farrell_2022} introduces finite-element(-in-space) implicit timesteppers for the incompressible analogue to this system with structure-preserving (SP) properties in the ideal case, alongside parameter-robust preconditioners. We show that these timesteppers can derive from a finite-element-in-time (FET) (and finite-element-in-space) interpretation. The benefits of this reformulation are discussed, including the derivation of timesteppers that are higher order in time, and the quantifiable dissipative SP properties in the non-ideal, resistive case.
        
        We discuss possible options for extending this FET approach to timesteppers for the compressible case.

        The kinetic corrections satisfy linearized Boltzmann equations. Using a Lénard--Bernstein collision operator, these take Fokker--Planck-like forms \cite{Fokker_1914, Planck_1917} wherein pseudo-particles in the numerical model obey the neoclassical transport equations, with particle-independent Brownian drift terms. This offers a rigorous methodology for incorporating collisions into the particle transport model, without coupling the equations of motions for each particle.
        
        Works by Chen, Chacón et al. \cite{Chen_Chacón_Barnes_2011, Chacón_Chen_Barnes_2013, Chen_Chacón_2014, Chen_Chacón_2015} have developed structure-preserving particle pushers for neoclassical transport in the Vlasov equations, derived from Crank--Nicolson integrators. We show these too can can derive from a FET interpretation, similarly offering potential extensions to higher-order-in-time particle pushers. The FET formulation is used also to consider how the stochastic drift terms can be incorporated into the pushers. Stochastic gyrokinetic expansions are also discussed.

        Different options for the numerical implementation of these schemes are considered.

        Due to the efficacy of FET in the development of SP timesteppers for both the fluid and kinetic component, we hope this approach will prove effective in the future for developing SP timesteppers for the full hybrid model. We hope this will give us the opportunity to incorporate previously inaccessible kinetic effects into the highly effective, modern, finite-element MHD models.
    \end{abstract}
    
    
    \newpage
    \tableofcontents
    
    
    \newpage
    \pagenumbering{arabic}
    %\linenumbers\renewcommand\thelinenumber{\color{black!50}\arabic{linenumber}}
            \input{0 - introduction/main.tex}
        \part{Research}
            \input{1 - low-noise PiC models/main.tex}
            \input{2 - kinetic component/main.tex}
            \input{3 - fluid component/main.tex}
            \input{4 - numerical implementation/main.tex}
        \part{Project Overview}
            \input{5 - research plan/main.tex}
            \input{6 - summary/main.tex}
    
    
    %\section{}
    \newpage
    \pagenumbering{gobble}
        \printbibliography


    \newpage
    \pagenumbering{roman}
    \appendix
        \part{Appendices}
            \input{8 - Hilbert complexes/main.tex}
            \input{9 - weak conservation proofs/main.tex}
\end{document}

    
    
    %\section{}
    \newpage
    \pagenumbering{gobble}
        \printbibliography


    \newpage
    \pagenumbering{roman}
    \appendix
        \part{Appendices}
            \documentclass[12pt, a4paper]{report}

\input{template/main.tex}

\title{\BA{Title in Progress...}}
\author{Boris Andrews}
\affil{Mathematical Institute, University of Oxford}
\date{\today}


\begin{document}
    \pagenumbering{gobble}
    \maketitle
    
    
    \begin{abstract}
        Magnetic confinement reactors---in particular tokamaks---offer one of the most promising options for achieving practical nuclear fusion, with the potential to provide virtually limitless, clean energy. The theoretical and numerical modeling of tokamak plasmas is simultaneously an essential component of effective reactor design, and a great research barrier. Tokamak operational conditions exhibit comparatively low Knudsen numbers. Kinetic effects, including kinetic waves and instabilities, Landau damping, bump-on-tail instabilities and more, are therefore highly influential in tokamak plasma dynamics. Purely fluid models are inherently incapable of capturing these effects, whereas the high dimensionality in purely kinetic models render them practically intractable for most relevant purposes.

        We consider a $\delta\!f$ decomposition model, with a macroscopic fluid background and microscopic kinetic correction, both fully coupled to each other. A similar manner of discretization is proposed to that used in the recent \texttt{STRUPHY} code \cite{Holderied_Possanner_Wang_2021, Holderied_2022, Li_et_al_2023} with a finite-element model for the background and a pseudo-particle/PiC model for the correction.

        The fluid background satisfies the full, non-linear, resistive, compressible, Hall MHD equations. \cite{Laakmann_Hu_Farrell_2022} introduces finite-element(-in-space) implicit timesteppers for the incompressible analogue to this system with structure-preserving (SP) properties in the ideal case, alongside parameter-robust preconditioners. We show that these timesteppers can derive from a finite-element-in-time (FET) (and finite-element-in-space) interpretation. The benefits of this reformulation are discussed, including the derivation of timesteppers that are higher order in time, and the quantifiable dissipative SP properties in the non-ideal, resistive case.
        
        We discuss possible options for extending this FET approach to timesteppers for the compressible case.

        The kinetic corrections satisfy linearized Boltzmann equations. Using a Lénard--Bernstein collision operator, these take Fokker--Planck-like forms \cite{Fokker_1914, Planck_1917} wherein pseudo-particles in the numerical model obey the neoclassical transport equations, with particle-independent Brownian drift terms. This offers a rigorous methodology for incorporating collisions into the particle transport model, without coupling the equations of motions for each particle.
        
        Works by Chen, Chacón et al. \cite{Chen_Chacón_Barnes_2011, Chacón_Chen_Barnes_2013, Chen_Chacón_2014, Chen_Chacón_2015} have developed structure-preserving particle pushers for neoclassical transport in the Vlasov equations, derived from Crank--Nicolson integrators. We show these too can can derive from a FET interpretation, similarly offering potential extensions to higher-order-in-time particle pushers. The FET formulation is used also to consider how the stochastic drift terms can be incorporated into the pushers. Stochastic gyrokinetic expansions are also discussed.

        Different options for the numerical implementation of these schemes are considered.

        Due to the efficacy of FET in the development of SP timesteppers for both the fluid and kinetic component, we hope this approach will prove effective in the future for developing SP timesteppers for the full hybrid model. We hope this will give us the opportunity to incorporate previously inaccessible kinetic effects into the highly effective, modern, finite-element MHD models.
    \end{abstract}
    
    
    \newpage
    \tableofcontents
    
    
    \newpage
    \pagenumbering{arabic}
    %\linenumbers\renewcommand\thelinenumber{\color{black!50}\arabic{linenumber}}
            \input{0 - introduction/main.tex}
        \part{Research}
            \input{1 - low-noise PiC models/main.tex}
            \input{2 - kinetic component/main.tex}
            \input{3 - fluid component/main.tex}
            \input{4 - numerical implementation/main.tex}
        \part{Project Overview}
            \input{5 - research plan/main.tex}
            \input{6 - summary/main.tex}
    
    
    %\section{}
    \newpage
    \pagenumbering{gobble}
        \printbibliography


    \newpage
    \pagenumbering{roman}
    \appendix
        \part{Appendices}
            \input{8 - Hilbert complexes/main.tex}
            \input{9 - weak conservation proofs/main.tex}
\end{document}

            \documentclass[12pt, a4paper]{report}

\input{template/main.tex}

\title{\BA{Title in Progress...}}
\author{Boris Andrews}
\affil{Mathematical Institute, University of Oxford}
\date{\today}


\begin{document}
    \pagenumbering{gobble}
    \maketitle
    
    
    \begin{abstract}
        Magnetic confinement reactors---in particular tokamaks---offer one of the most promising options for achieving practical nuclear fusion, with the potential to provide virtually limitless, clean energy. The theoretical and numerical modeling of tokamak plasmas is simultaneously an essential component of effective reactor design, and a great research barrier. Tokamak operational conditions exhibit comparatively low Knudsen numbers. Kinetic effects, including kinetic waves and instabilities, Landau damping, bump-on-tail instabilities and more, are therefore highly influential in tokamak plasma dynamics. Purely fluid models are inherently incapable of capturing these effects, whereas the high dimensionality in purely kinetic models render them practically intractable for most relevant purposes.

        We consider a $\delta\!f$ decomposition model, with a macroscopic fluid background and microscopic kinetic correction, both fully coupled to each other. A similar manner of discretization is proposed to that used in the recent \texttt{STRUPHY} code \cite{Holderied_Possanner_Wang_2021, Holderied_2022, Li_et_al_2023} with a finite-element model for the background and a pseudo-particle/PiC model for the correction.

        The fluid background satisfies the full, non-linear, resistive, compressible, Hall MHD equations. \cite{Laakmann_Hu_Farrell_2022} introduces finite-element(-in-space) implicit timesteppers for the incompressible analogue to this system with structure-preserving (SP) properties in the ideal case, alongside parameter-robust preconditioners. We show that these timesteppers can derive from a finite-element-in-time (FET) (and finite-element-in-space) interpretation. The benefits of this reformulation are discussed, including the derivation of timesteppers that are higher order in time, and the quantifiable dissipative SP properties in the non-ideal, resistive case.
        
        We discuss possible options for extending this FET approach to timesteppers for the compressible case.

        The kinetic corrections satisfy linearized Boltzmann equations. Using a Lénard--Bernstein collision operator, these take Fokker--Planck-like forms \cite{Fokker_1914, Planck_1917} wherein pseudo-particles in the numerical model obey the neoclassical transport equations, with particle-independent Brownian drift terms. This offers a rigorous methodology for incorporating collisions into the particle transport model, without coupling the equations of motions for each particle.
        
        Works by Chen, Chacón et al. \cite{Chen_Chacón_Barnes_2011, Chacón_Chen_Barnes_2013, Chen_Chacón_2014, Chen_Chacón_2015} have developed structure-preserving particle pushers for neoclassical transport in the Vlasov equations, derived from Crank--Nicolson integrators. We show these too can can derive from a FET interpretation, similarly offering potential extensions to higher-order-in-time particle pushers. The FET formulation is used also to consider how the stochastic drift terms can be incorporated into the pushers. Stochastic gyrokinetic expansions are also discussed.

        Different options for the numerical implementation of these schemes are considered.

        Due to the efficacy of FET in the development of SP timesteppers for both the fluid and kinetic component, we hope this approach will prove effective in the future for developing SP timesteppers for the full hybrid model. We hope this will give us the opportunity to incorporate previously inaccessible kinetic effects into the highly effective, modern, finite-element MHD models.
    \end{abstract}
    
    
    \newpage
    \tableofcontents
    
    
    \newpage
    \pagenumbering{arabic}
    %\linenumbers\renewcommand\thelinenumber{\color{black!50}\arabic{linenumber}}
            \input{0 - introduction/main.tex}
        \part{Research}
            \input{1 - low-noise PiC models/main.tex}
            \input{2 - kinetic component/main.tex}
            \input{3 - fluid component/main.tex}
            \input{4 - numerical implementation/main.tex}
        \part{Project Overview}
            \input{5 - research plan/main.tex}
            \input{6 - summary/main.tex}
    
    
    %\section{}
    \newpage
    \pagenumbering{gobble}
        \printbibliography


    \newpage
    \pagenumbering{roman}
    \appendix
        \part{Appendices}
            \input{8 - Hilbert complexes/main.tex}
            \input{9 - weak conservation proofs/main.tex}
\end{document}

\end{document}

        \part{Research}
            \documentclass[12pt, a4paper]{report}

\documentclass[12pt, a4paper]{report}

\input{template/main.tex}

\title{\BA{Title in Progress...}}
\author{Boris Andrews}
\affil{Mathematical Institute, University of Oxford}
\date{\today}


\begin{document}
    \pagenumbering{gobble}
    \maketitle
    
    
    \begin{abstract}
        Magnetic confinement reactors---in particular tokamaks---offer one of the most promising options for achieving practical nuclear fusion, with the potential to provide virtually limitless, clean energy. The theoretical and numerical modeling of tokamak plasmas is simultaneously an essential component of effective reactor design, and a great research barrier. Tokamak operational conditions exhibit comparatively low Knudsen numbers. Kinetic effects, including kinetic waves and instabilities, Landau damping, bump-on-tail instabilities and more, are therefore highly influential in tokamak plasma dynamics. Purely fluid models are inherently incapable of capturing these effects, whereas the high dimensionality in purely kinetic models render them practically intractable for most relevant purposes.

        We consider a $\delta\!f$ decomposition model, with a macroscopic fluid background and microscopic kinetic correction, both fully coupled to each other. A similar manner of discretization is proposed to that used in the recent \texttt{STRUPHY} code \cite{Holderied_Possanner_Wang_2021, Holderied_2022, Li_et_al_2023} with a finite-element model for the background and a pseudo-particle/PiC model for the correction.

        The fluid background satisfies the full, non-linear, resistive, compressible, Hall MHD equations. \cite{Laakmann_Hu_Farrell_2022} introduces finite-element(-in-space) implicit timesteppers for the incompressible analogue to this system with structure-preserving (SP) properties in the ideal case, alongside parameter-robust preconditioners. We show that these timesteppers can derive from a finite-element-in-time (FET) (and finite-element-in-space) interpretation. The benefits of this reformulation are discussed, including the derivation of timesteppers that are higher order in time, and the quantifiable dissipative SP properties in the non-ideal, resistive case.
        
        We discuss possible options for extending this FET approach to timesteppers for the compressible case.

        The kinetic corrections satisfy linearized Boltzmann equations. Using a Lénard--Bernstein collision operator, these take Fokker--Planck-like forms \cite{Fokker_1914, Planck_1917} wherein pseudo-particles in the numerical model obey the neoclassical transport equations, with particle-independent Brownian drift terms. This offers a rigorous methodology for incorporating collisions into the particle transport model, without coupling the equations of motions for each particle.
        
        Works by Chen, Chacón et al. \cite{Chen_Chacón_Barnes_2011, Chacón_Chen_Barnes_2013, Chen_Chacón_2014, Chen_Chacón_2015} have developed structure-preserving particle pushers for neoclassical transport in the Vlasov equations, derived from Crank--Nicolson integrators. We show these too can can derive from a FET interpretation, similarly offering potential extensions to higher-order-in-time particle pushers. The FET formulation is used also to consider how the stochastic drift terms can be incorporated into the pushers. Stochastic gyrokinetic expansions are also discussed.

        Different options for the numerical implementation of these schemes are considered.

        Due to the efficacy of FET in the development of SP timesteppers for both the fluid and kinetic component, we hope this approach will prove effective in the future for developing SP timesteppers for the full hybrid model. We hope this will give us the opportunity to incorporate previously inaccessible kinetic effects into the highly effective, modern, finite-element MHD models.
    \end{abstract}
    
    
    \newpage
    \tableofcontents
    
    
    \newpage
    \pagenumbering{arabic}
    %\linenumbers\renewcommand\thelinenumber{\color{black!50}\arabic{linenumber}}
            \input{0 - introduction/main.tex}
        \part{Research}
            \input{1 - low-noise PiC models/main.tex}
            \input{2 - kinetic component/main.tex}
            \input{3 - fluid component/main.tex}
            \input{4 - numerical implementation/main.tex}
        \part{Project Overview}
            \input{5 - research plan/main.tex}
            \input{6 - summary/main.tex}
    
    
    %\section{}
    \newpage
    \pagenumbering{gobble}
        \printbibliography


    \newpage
    \pagenumbering{roman}
    \appendix
        \part{Appendices}
            \input{8 - Hilbert complexes/main.tex}
            \input{9 - weak conservation proofs/main.tex}
\end{document}


\title{\BA{Title in Progress...}}
\author{Boris Andrews}
\affil{Mathematical Institute, University of Oxford}
\date{\today}


\begin{document}
    \pagenumbering{gobble}
    \maketitle
    
    
    \begin{abstract}
        Magnetic confinement reactors---in particular tokamaks---offer one of the most promising options for achieving practical nuclear fusion, with the potential to provide virtually limitless, clean energy. The theoretical and numerical modeling of tokamak plasmas is simultaneously an essential component of effective reactor design, and a great research barrier. Tokamak operational conditions exhibit comparatively low Knudsen numbers. Kinetic effects, including kinetic waves and instabilities, Landau damping, bump-on-tail instabilities and more, are therefore highly influential in tokamak plasma dynamics. Purely fluid models are inherently incapable of capturing these effects, whereas the high dimensionality in purely kinetic models render them practically intractable for most relevant purposes.

        We consider a $\delta\!f$ decomposition model, with a macroscopic fluid background and microscopic kinetic correction, both fully coupled to each other. A similar manner of discretization is proposed to that used in the recent \texttt{STRUPHY} code \cite{Holderied_Possanner_Wang_2021, Holderied_2022, Li_et_al_2023} with a finite-element model for the background and a pseudo-particle/PiC model for the correction.

        The fluid background satisfies the full, non-linear, resistive, compressible, Hall MHD equations. \cite{Laakmann_Hu_Farrell_2022} introduces finite-element(-in-space) implicit timesteppers for the incompressible analogue to this system with structure-preserving (SP) properties in the ideal case, alongside parameter-robust preconditioners. We show that these timesteppers can derive from a finite-element-in-time (FET) (and finite-element-in-space) interpretation. The benefits of this reformulation are discussed, including the derivation of timesteppers that are higher order in time, and the quantifiable dissipative SP properties in the non-ideal, resistive case.
        
        We discuss possible options for extending this FET approach to timesteppers for the compressible case.

        The kinetic corrections satisfy linearized Boltzmann equations. Using a Lénard--Bernstein collision operator, these take Fokker--Planck-like forms \cite{Fokker_1914, Planck_1917} wherein pseudo-particles in the numerical model obey the neoclassical transport equations, with particle-independent Brownian drift terms. This offers a rigorous methodology for incorporating collisions into the particle transport model, without coupling the equations of motions for each particle.
        
        Works by Chen, Chacón et al. \cite{Chen_Chacón_Barnes_2011, Chacón_Chen_Barnes_2013, Chen_Chacón_2014, Chen_Chacón_2015} have developed structure-preserving particle pushers for neoclassical transport in the Vlasov equations, derived from Crank--Nicolson integrators. We show these too can can derive from a FET interpretation, similarly offering potential extensions to higher-order-in-time particle pushers. The FET formulation is used also to consider how the stochastic drift terms can be incorporated into the pushers. Stochastic gyrokinetic expansions are also discussed.

        Different options for the numerical implementation of these schemes are considered.

        Due to the efficacy of FET in the development of SP timesteppers for both the fluid and kinetic component, we hope this approach will prove effective in the future for developing SP timesteppers for the full hybrid model. We hope this will give us the opportunity to incorporate previously inaccessible kinetic effects into the highly effective, modern, finite-element MHD models.
    \end{abstract}
    
    
    \newpage
    \tableofcontents
    
    
    \newpage
    \pagenumbering{arabic}
    %\linenumbers\renewcommand\thelinenumber{\color{black!50}\arabic{linenumber}}
            \documentclass[12pt, a4paper]{report}

\input{template/main.tex}

\title{\BA{Title in Progress...}}
\author{Boris Andrews}
\affil{Mathematical Institute, University of Oxford}
\date{\today}


\begin{document}
    \pagenumbering{gobble}
    \maketitle
    
    
    \begin{abstract}
        Magnetic confinement reactors---in particular tokamaks---offer one of the most promising options for achieving practical nuclear fusion, with the potential to provide virtually limitless, clean energy. The theoretical and numerical modeling of tokamak plasmas is simultaneously an essential component of effective reactor design, and a great research barrier. Tokamak operational conditions exhibit comparatively low Knudsen numbers. Kinetic effects, including kinetic waves and instabilities, Landau damping, bump-on-tail instabilities and more, are therefore highly influential in tokamak plasma dynamics. Purely fluid models are inherently incapable of capturing these effects, whereas the high dimensionality in purely kinetic models render them practically intractable for most relevant purposes.

        We consider a $\delta\!f$ decomposition model, with a macroscopic fluid background and microscopic kinetic correction, both fully coupled to each other. A similar manner of discretization is proposed to that used in the recent \texttt{STRUPHY} code \cite{Holderied_Possanner_Wang_2021, Holderied_2022, Li_et_al_2023} with a finite-element model for the background and a pseudo-particle/PiC model for the correction.

        The fluid background satisfies the full, non-linear, resistive, compressible, Hall MHD equations. \cite{Laakmann_Hu_Farrell_2022} introduces finite-element(-in-space) implicit timesteppers for the incompressible analogue to this system with structure-preserving (SP) properties in the ideal case, alongside parameter-robust preconditioners. We show that these timesteppers can derive from a finite-element-in-time (FET) (and finite-element-in-space) interpretation. The benefits of this reformulation are discussed, including the derivation of timesteppers that are higher order in time, and the quantifiable dissipative SP properties in the non-ideal, resistive case.
        
        We discuss possible options for extending this FET approach to timesteppers for the compressible case.

        The kinetic corrections satisfy linearized Boltzmann equations. Using a Lénard--Bernstein collision operator, these take Fokker--Planck-like forms \cite{Fokker_1914, Planck_1917} wherein pseudo-particles in the numerical model obey the neoclassical transport equations, with particle-independent Brownian drift terms. This offers a rigorous methodology for incorporating collisions into the particle transport model, without coupling the equations of motions for each particle.
        
        Works by Chen, Chacón et al. \cite{Chen_Chacón_Barnes_2011, Chacón_Chen_Barnes_2013, Chen_Chacón_2014, Chen_Chacón_2015} have developed structure-preserving particle pushers for neoclassical transport in the Vlasov equations, derived from Crank--Nicolson integrators. We show these too can can derive from a FET interpretation, similarly offering potential extensions to higher-order-in-time particle pushers. The FET formulation is used also to consider how the stochastic drift terms can be incorporated into the pushers. Stochastic gyrokinetic expansions are also discussed.

        Different options for the numerical implementation of these schemes are considered.

        Due to the efficacy of FET in the development of SP timesteppers for both the fluid and kinetic component, we hope this approach will prove effective in the future for developing SP timesteppers for the full hybrid model. We hope this will give us the opportunity to incorporate previously inaccessible kinetic effects into the highly effective, modern, finite-element MHD models.
    \end{abstract}
    
    
    \newpage
    \tableofcontents
    
    
    \newpage
    \pagenumbering{arabic}
    %\linenumbers\renewcommand\thelinenumber{\color{black!50}\arabic{linenumber}}
            \input{0 - introduction/main.tex}
        \part{Research}
            \input{1 - low-noise PiC models/main.tex}
            \input{2 - kinetic component/main.tex}
            \input{3 - fluid component/main.tex}
            \input{4 - numerical implementation/main.tex}
        \part{Project Overview}
            \input{5 - research plan/main.tex}
            \input{6 - summary/main.tex}
    
    
    %\section{}
    \newpage
    \pagenumbering{gobble}
        \printbibliography


    \newpage
    \pagenumbering{roman}
    \appendix
        \part{Appendices}
            \input{8 - Hilbert complexes/main.tex}
            \input{9 - weak conservation proofs/main.tex}
\end{document}

        \part{Research}
            \documentclass[12pt, a4paper]{report}

\input{template/main.tex}

\title{\BA{Title in Progress...}}
\author{Boris Andrews}
\affil{Mathematical Institute, University of Oxford}
\date{\today}


\begin{document}
    \pagenumbering{gobble}
    \maketitle
    
    
    \begin{abstract}
        Magnetic confinement reactors---in particular tokamaks---offer one of the most promising options for achieving practical nuclear fusion, with the potential to provide virtually limitless, clean energy. The theoretical and numerical modeling of tokamak plasmas is simultaneously an essential component of effective reactor design, and a great research barrier. Tokamak operational conditions exhibit comparatively low Knudsen numbers. Kinetic effects, including kinetic waves and instabilities, Landau damping, bump-on-tail instabilities and more, are therefore highly influential in tokamak plasma dynamics. Purely fluid models are inherently incapable of capturing these effects, whereas the high dimensionality in purely kinetic models render them practically intractable for most relevant purposes.

        We consider a $\delta\!f$ decomposition model, with a macroscopic fluid background and microscopic kinetic correction, both fully coupled to each other. A similar manner of discretization is proposed to that used in the recent \texttt{STRUPHY} code \cite{Holderied_Possanner_Wang_2021, Holderied_2022, Li_et_al_2023} with a finite-element model for the background and a pseudo-particle/PiC model for the correction.

        The fluid background satisfies the full, non-linear, resistive, compressible, Hall MHD equations. \cite{Laakmann_Hu_Farrell_2022} introduces finite-element(-in-space) implicit timesteppers for the incompressible analogue to this system with structure-preserving (SP) properties in the ideal case, alongside parameter-robust preconditioners. We show that these timesteppers can derive from a finite-element-in-time (FET) (and finite-element-in-space) interpretation. The benefits of this reformulation are discussed, including the derivation of timesteppers that are higher order in time, and the quantifiable dissipative SP properties in the non-ideal, resistive case.
        
        We discuss possible options for extending this FET approach to timesteppers for the compressible case.

        The kinetic corrections satisfy linearized Boltzmann equations. Using a Lénard--Bernstein collision operator, these take Fokker--Planck-like forms \cite{Fokker_1914, Planck_1917} wherein pseudo-particles in the numerical model obey the neoclassical transport equations, with particle-independent Brownian drift terms. This offers a rigorous methodology for incorporating collisions into the particle transport model, without coupling the equations of motions for each particle.
        
        Works by Chen, Chacón et al. \cite{Chen_Chacón_Barnes_2011, Chacón_Chen_Barnes_2013, Chen_Chacón_2014, Chen_Chacón_2015} have developed structure-preserving particle pushers for neoclassical transport in the Vlasov equations, derived from Crank--Nicolson integrators. We show these too can can derive from a FET interpretation, similarly offering potential extensions to higher-order-in-time particle pushers. The FET formulation is used also to consider how the stochastic drift terms can be incorporated into the pushers. Stochastic gyrokinetic expansions are also discussed.

        Different options for the numerical implementation of these schemes are considered.

        Due to the efficacy of FET in the development of SP timesteppers for both the fluid and kinetic component, we hope this approach will prove effective in the future for developing SP timesteppers for the full hybrid model. We hope this will give us the opportunity to incorporate previously inaccessible kinetic effects into the highly effective, modern, finite-element MHD models.
    \end{abstract}
    
    
    \newpage
    \tableofcontents
    
    
    \newpage
    \pagenumbering{arabic}
    %\linenumbers\renewcommand\thelinenumber{\color{black!50}\arabic{linenumber}}
            \input{0 - introduction/main.tex}
        \part{Research}
            \input{1 - low-noise PiC models/main.tex}
            \input{2 - kinetic component/main.tex}
            \input{3 - fluid component/main.tex}
            \input{4 - numerical implementation/main.tex}
        \part{Project Overview}
            \input{5 - research plan/main.tex}
            \input{6 - summary/main.tex}
    
    
    %\section{}
    \newpage
    \pagenumbering{gobble}
        \printbibliography


    \newpage
    \pagenumbering{roman}
    \appendix
        \part{Appendices}
            \input{8 - Hilbert complexes/main.tex}
            \input{9 - weak conservation proofs/main.tex}
\end{document}

            \documentclass[12pt, a4paper]{report}

\input{template/main.tex}

\title{\BA{Title in Progress...}}
\author{Boris Andrews}
\affil{Mathematical Institute, University of Oxford}
\date{\today}


\begin{document}
    \pagenumbering{gobble}
    \maketitle
    
    
    \begin{abstract}
        Magnetic confinement reactors---in particular tokamaks---offer one of the most promising options for achieving practical nuclear fusion, with the potential to provide virtually limitless, clean energy. The theoretical and numerical modeling of tokamak plasmas is simultaneously an essential component of effective reactor design, and a great research barrier. Tokamak operational conditions exhibit comparatively low Knudsen numbers. Kinetic effects, including kinetic waves and instabilities, Landau damping, bump-on-tail instabilities and more, are therefore highly influential in tokamak plasma dynamics. Purely fluid models are inherently incapable of capturing these effects, whereas the high dimensionality in purely kinetic models render them practically intractable for most relevant purposes.

        We consider a $\delta\!f$ decomposition model, with a macroscopic fluid background and microscopic kinetic correction, both fully coupled to each other. A similar manner of discretization is proposed to that used in the recent \texttt{STRUPHY} code \cite{Holderied_Possanner_Wang_2021, Holderied_2022, Li_et_al_2023} with a finite-element model for the background and a pseudo-particle/PiC model for the correction.

        The fluid background satisfies the full, non-linear, resistive, compressible, Hall MHD equations. \cite{Laakmann_Hu_Farrell_2022} introduces finite-element(-in-space) implicit timesteppers for the incompressible analogue to this system with structure-preserving (SP) properties in the ideal case, alongside parameter-robust preconditioners. We show that these timesteppers can derive from a finite-element-in-time (FET) (and finite-element-in-space) interpretation. The benefits of this reformulation are discussed, including the derivation of timesteppers that are higher order in time, and the quantifiable dissipative SP properties in the non-ideal, resistive case.
        
        We discuss possible options for extending this FET approach to timesteppers for the compressible case.

        The kinetic corrections satisfy linearized Boltzmann equations. Using a Lénard--Bernstein collision operator, these take Fokker--Planck-like forms \cite{Fokker_1914, Planck_1917} wherein pseudo-particles in the numerical model obey the neoclassical transport equations, with particle-independent Brownian drift terms. This offers a rigorous methodology for incorporating collisions into the particle transport model, without coupling the equations of motions for each particle.
        
        Works by Chen, Chacón et al. \cite{Chen_Chacón_Barnes_2011, Chacón_Chen_Barnes_2013, Chen_Chacón_2014, Chen_Chacón_2015} have developed structure-preserving particle pushers for neoclassical transport in the Vlasov equations, derived from Crank--Nicolson integrators. We show these too can can derive from a FET interpretation, similarly offering potential extensions to higher-order-in-time particle pushers. The FET formulation is used also to consider how the stochastic drift terms can be incorporated into the pushers. Stochastic gyrokinetic expansions are also discussed.

        Different options for the numerical implementation of these schemes are considered.

        Due to the efficacy of FET in the development of SP timesteppers for both the fluid and kinetic component, we hope this approach will prove effective in the future for developing SP timesteppers for the full hybrid model. We hope this will give us the opportunity to incorporate previously inaccessible kinetic effects into the highly effective, modern, finite-element MHD models.
    \end{abstract}
    
    
    \newpage
    \tableofcontents
    
    
    \newpage
    \pagenumbering{arabic}
    %\linenumbers\renewcommand\thelinenumber{\color{black!50}\arabic{linenumber}}
            \input{0 - introduction/main.tex}
        \part{Research}
            \input{1 - low-noise PiC models/main.tex}
            \input{2 - kinetic component/main.tex}
            \input{3 - fluid component/main.tex}
            \input{4 - numerical implementation/main.tex}
        \part{Project Overview}
            \input{5 - research plan/main.tex}
            \input{6 - summary/main.tex}
    
    
    %\section{}
    \newpage
    \pagenumbering{gobble}
        \printbibliography


    \newpage
    \pagenumbering{roman}
    \appendix
        \part{Appendices}
            \input{8 - Hilbert complexes/main.tex}
            \input{9 - weak conservation proofs/main.tex}
\end{document}

            \documentclass[12pt, a4paper]{report}

\input{template/main.tex}

\title{\BA{Title in Progress...}}
\author{Boris Andrews}
\affil{Mathematical Institute, University of Oxford}
\date{\today}


\begin{document}
    \pagenumbering{gobble}
    \maketitle
    
    
    \begin{abstract}
        Magnetic confinement reactors---in particular tokamaks---offer one of the most promising options for achieving practical nuclear fusion, with the potential to provide virtually limitless, clean energy. The theoretical and numerical modeling of tokamak plasmas is simultaneously an essential component of effective reactor design, and a great research barrier. Tokamak operational conditions exhibit comparatively low Knudsen numbers. Kinetic effects, including kinetic waves and instabilities, Landau damping, bump-on-tail instabilities and more, are therefore highly influential in tokamak plasma dynamics. Purely fluid models are inherently incapable of capturing these effects, whereas the high dimensionality in purely kinetic models render them practically intractable for most relevant purposes.

        We consider a $\delta\!f$ decomposition model, with a macroscopic fluid background and microscopic kinetic correction, both fully coupled to each other. A similar manner of discretization is proposed to that used in the recent \texttt{STRUPHY} code \cite{Holderied_Possanner_Wang_2021, Holderied_2022, Li_et_al_2023} with a finite-element model for the background and a pseudo-particle/PiC model for the correction.

        The fluid background satisfies the full, non-linear, resistive, compressible, Hall MHD equations. \cite{Laakmann_Hu_Farrell_2022} introduces finite-element(-in-space) implicit timesteppers for the incompressible analogue to this system with structure-preserving (SP) properties in the ideal case, alongside parameter-robust preconditioners. We show that these timesteppers can derive from a finite-element-in-time (FET) (and finite-element-in-space) interpretation. The benefits of this reformulation are discussed, including the derivation of timesteppers that are higher order in time, and the quantifiable dissipative SP properties in the non-ideal, resistive case.
        
        We discuss possible options for extending this FET approach to timesteppers for the compressible case.

        The kinetic corrections satisfy linearized Boltzmann equations. Using a Lénard--Bernstein collision operator, these take Fokker--Planck-like forms \cite{Fokker_1914, Planck_1917} wherein pseudo-particles in the numerical model obey the neoclassical transport equations, with particle-independent Brownian drift terms. This offers a rigorous methodology for incorporating collisions into the particle transport model, without coupling the equations of motions for each particle.
        
        Works by Chen, Chacón et al. \cite{Chen_Chacón_Barnes_2011, Chacón_Chen_Barnes_2013, Chen_Chacón_2014, Chen_Chacón_2015} have developed structure-preserving particle pushers for neoclassical transport in the Vlasov equations, derived from Crank--Nicolson integrators. We show these too can can derive from a FET interpretation, similarly offering potential extensions to higher-order-in-time particle pushers. The FET formulation is used also to consider how the stochastic drift terms can be incorporated into the pushers. Stochastic gyrokinetic expansions are also discussed.

        Different options for the numerical implementation of these schemes are considered.

        Due to the efficacy of FET in the development of SP timesteppers for both the fluid and kinetic component, we hope this approach will prove effective in the future for developing SP timesteppers for the full hybrid model. We hope this will give us the opportunity to incorporate previously inaccessible kinetic effects into the highly effective, modern, finite-element MHD models.
    \end{abstract}
    
    
    \newpage
    \tableofcontents
    
    
    \newpage
    \pagenumbering{arabic}
    %\linenumbers\renewcommand\thelinenumber{\color{black!50}\arabic{linenumber}}
            \input{0 - introduction/main.tex}
        \part{Research}
            \input{1 - low-noise PiC models/main.tex}
            \input{2 - kinetic component/main.tex}
            \input{3 - fluid component/main.tex}
            \input{4 - numerical implementation/main.tex}
        \part{Project Overview}
            \input{5 - research plan/main.tex}
            \input{6 - summary/main.tex}
    
    
    %\section{}
    \newpage
    \pagenumbering{gobble}
        \printbibliography


    \newpage
    \pagenumbering{roman}
    \appendix
        \part{Appendices}
            \input{8 - Hilbert complexes/main.tex}
            \input{9 - weak conservation proofs/main.tex}
\end{document}

            \documentclass[12pt, a4paper]{report}

\input{template/main.tex}

\title{\BA{Title in Progress...}}
\author{Boris Andrews}
\affil{Mathematical Institute, University of Oxford}
\date{\today}


\begin{document}
    \pagenumbering{gobble}
    \maketitle
    
    
    \begin{abstract}
        Magnetic confinement reactors---in particular tokamaks---offer one of the most promising options for achieving practical nuclear fusion, with the potential to provide virtually limitless, clean energy. The theoretical and numerical modeling of tokamak plasmas is simultaneously an essential component of effective reactor design, and a great research barrier. Tokamak operational conditions exhibit comparatively low Knudsen numbers. Kinetic effects, including kinetic waves and instabilities, Landau damping, bump-on-tail instabilities and more, are therefore highly influential in tokamak plasma dynamics. Purely fluid models are inherently incapable of capturing these effects, whereas the high dimensionality in purely kinetic models render them practically intractable for most relevant purposes.

        We consider a $\delta\!f$ decomposition model, with a macroscopic fluid background and microscopic kinetic correction, both fully coupled to each other. A similar manner of discretization is proposed to that used in the recent \texttt{STRUPHY} code \cite{Holderied_Possanner_Wang_2021, Holderied_2022, Li_et_al_2023} with a finite-element model for the background and a pseudo-particle/PiC model for the correction.

        The fluid background satisfies the full, non-linear, resistive, compressible, Hall MHD equations. \cite{Laakmann_Hu_Farrell_2022} introduces finite-element(-in-space) implicit timesteppers for the incompressible analogue to this system with structure-preserving (SP) properties in the ideal case, alongside parameter-robust preconditioners. We show that these timesteppers can derive from a finite-element-in-time (FET) (and finite-element-in-space) interpretation. The benefits of this reformulation are discussed, including the derivation of timesteppers that are higher order in time, and the quantifiable dissipative SP properties in the non-ideal, resistive case.
        
        We discuss possible options for extending this FET approach to timesteppers for the compressible case.

        The kinetic corrections satisfy linearized Boltzmann equations. Using a Lénard--Bernstein collision operator, these take Fokker--Planck-like forms \cite{Fokker_1914, Planck_1917} wherein pseudo-particles in the numerical model obey the neoclassical transport equations, with particle-independent Brownian drift terms. This offers a rigorous methodology for incorporating collisions into the particle transport model, without coupling the equations of motions for each particle.
        
        Works by Chen, Chacón et al. \cite{Chen_Chacón_Barnes_2011, Chacón_Chen_Barnes_2013, Chen_Chacón_2014, Chen_Chacón_2015} have developed structure-preserving particle pushers for neoclassical transport in the Vlasov equations, derived from Crank--Nicolson integrators. We show these too can can derive from a FET interpretation, similarly offering potential extensions to higher-order-in-time particle pushers. The FET formulation is used also to consider how the stochastic drift terms can be incorporated into the pushers. Stochastic gyrokinetic expansions are also discussed.

        Different options for the numerical implementation of these schemes are considered.

        Due to the efficacy of FET in the development of SP timesteppers for both the fluid and kinetic component, we hope this approach will prove effective in the future for developing SP timesteppers for the full hybrid model. We hope this will give us the opportunity to incorporate previously inaccessible kinetic effects into the highly effective, modern, finite-element MHD models.
    \end{abstract}
    
    
    \newpage
    \tableofcontents
    
    
    \newpage
    \pagenumbering{arabic}
    %\linenumbers\renewcommand\thelinenumber{\color{black!50}\arabic{linenumber}}
            \input{0 - introduction/main.tex}
        \part{Research}
            \input{1 - low-noise PiC models/main.tex}
            \input{2 - kinetic component/main.tex}
            \input{3 - fluid component/main.tex}
            \input{4 - numerical implementation/main.tex}
        \part{Project Overview}
            \input{5 - research plan/main.tex}
            \input{6 - summary/main.tex}
    
    
    %\section{}
    \newpage
    \pagenumbering{gobble}
        \printbibliography


    \newpage
    \pagenumbering{roman}
    \appendix
        \part{Appendices}
            \input{8 - Hilbert complexes/main.tex}
            \input{9 - weak conservation proofs/main.tex}
\end{document}

        \part{Project Overview}
            \documentclass[12pt, a4paper]{report}

\input{template/main.tex}

\title{\BA{Title in Progress...}}
\author{Boris Andrews}
\affil{Mathematical Institute, University of Oxford}
\date{\today}


\begin{document}
    \pagenumbering{gobble}
    \maketitle
    
    
    \begin{abstract}
        Magnetic confinement reactors---in particular tokamaks---offer one of the most promising options for achieving practical nuclear fusion, with the potential to provide virtually limitless, clean energy. The theoretical and numerical modeling of tokamak plasmas is simultaneously an essential component of effective reactor design, and a great research barrier. Tokamak operational conditions exhibit comparatively low Knudsen numbers. Kinetic effects, including kinetic waves and instabilities, Landau damping, bump-on-tail instabilities and more, are therefore highly influential in tokamak plasma dynamics. Purely fluid models are inherently incapable of capturing these effects, whereas the high dimensionality in purely kinetic models render them practically intractable for most relevant purposes.

        We consider a $\delta\!f$ decomposition model, with a macroscopic fluid background and microscopic kinetic correction, both fully coupled to each other. A similar manner of discretization is proposed to that used in the recent \texttt{STRUPHY} code \cite{Holderied_Possanner_Wang_2021, Holderied_2022, Li_et_al_2023} with a finite-element model for the background and a pseudo-particle/PiC model for the correction.

        The fluid background satisfies the full, non-linear, resistive, compressible, Hall MHD equations. \cite{Laakmann_Hu_Farrell_2022} introduces finite-element(-in-space) implicit timesteppers for the incompressible analogue to this system with structure-preserving (SP) properties in the ideal case, alongside parameter-robust preconditioners. We show that these timesteppers can derive from a finite-element-in-time (FET) (and finite-element-in-space) interpretation. The benefits of this reformulation are discussed, including the derivation of timesteppers that are higher order in time, and the quantifiable dissipative SP properties in the non-ideal, resistive case.
        
        We discuss possible options for extending this FET approach to timesteppers for the compressible case.

        The kinetic corrections satisfy linearized Boltzmann equations. Using a Lénard--Bernstein collision operator, these take Fokker--Planck-like forms \cite{Fokker_1914, Planck_1917} wherein pseudo-particles in the numerical model obey the neoclassical transport equations, with particle-independent Brownian drift terms. This offers a rigorous methodology for incorporating collisions into the particle transport model, without coupling the equations of motions for each particle.
        
        Works by Chen, Chacón et al. \cite{Chen_Chacón_Barnes_2011, Chacón_Chen_Barnes_2013, Chen_Chacón_2014, Chen_Chacón_2015} have developed structure-preserving particle pushers for neoclassical transport in the Vlasov equations, derived from Crank--Nicolson integrators. We show these too can can derive from a FET interpretation, similarly offering potential extensions to higher-order-in-time particle pushers. The FET formulation is used also to consider how the stochastic drift terms can be incorporated into the pushers. Stochastic gyrokinetic expansions are also discussed.

        Different options for the numerical implementation of these schemes are considered.

        Due to the efficacy of FET in the development of SP timesteppers for both the fluid and kinetic component, we hope this approach will prove effective in the future for developing SP timesteppers for the full hybrid model. We hope this will give us the opportunity to incorporate previously inaccessible kinetic effects into the highly effective, modern, finite-element MHD models.
    \end{abstract}
    
    
    \newpage
    \tableofcontents
    
    
    \newpage
    \pagenumbering{arabic}
    %\linenumbers\renewcommand\thelinenumber{\color{black!50}\arabic{linenumber}}
            \input{0 - introduction/main.tex}
        \part{Research}
            \input{1 - low-noise PiC models/main.tex}
            \input{2 - kinetic component/main.tex}
            \input{3 - fluid component/main.tex}
            \input{4 - numerical implementation/main.tex}
        \part{Project Overview}
            \input{5 - research plan/main.tex}
            \input{6 - summary/main.tex}
    
    
    %\section{}
    \newpage
    \pagenumbering{gobble}
        \printbibliography


    \newpage
    \pagenumbering{roman}
    \appendix
        \part{Appendices}
            \input{8 - Hilbert complexes/main.tex}
            \input{9 - weak conservation proofs/main.tex}
\end{document}

            \documentclass[12pt, a4paper]{report}

\input{template/main.tex}

\title{\BA{Title in Progress...}}
\author{Boris Andrews}
\affil{Mathematical Institute, University of Oxford}
\date{\today}


\begin{document}
    \pagenumbering{gobble}
    \maketitle
    
    
    \begin{abstract}
        Magnetic confinement reactors---in particular tokamaks---offer one of the most promising options for achieving practical nuclear fusion, with the potential to provide virtually limitless, clean energy. The theoretical and numerical modeling of tokamak plasmas is simultaneously an essential component of effective reactor design, and a great research barrier. Tokamak operational conditions exhibit comparatively low Knudsen numbers. Kinetic effects, including kinetic waves and instabilities, Landau damping, bump-on-tail instabilities and more, are therefore highly influential in tokamak plasma dynamics. Purely fluid models are inherently incapable of capturing these effects, whereas the high dimensionality in purely kinetic models render them practically intractable for most relevant purposes.

        We consider a $\delta\!f$ decomposition model, with a macroscopic fluid background and microscopic kinetic correction, both fully coupled to each other. A similar manner of discretization is proposed to that used in the recent \texttt{STRUPHY} code \cite{Holderied_Possanner_Wang_2021, Holderied_2022, Li_et_al_2023} with a finite-element model for the background and a pseudo-particle/PiC model for the correction.

        The fluid background satisfies the full, non-linear, resistive, compressible, Hall MHD equations. \cite{Laakmann_Hu_Farrell_2022} introduces finite-element(-in-space) implicit timesteppers for the incompressible analogue to this system with structure-preserving (SP) properties in the ideal case, alongside parameter-robust preconditioners. We show that these timesteppers can derive from a finite-element-in-time (FET) (and finite-element-in-space) interpretation. The benefits of this reformulation are discussed, including the derivation of timesteppers that are higher order in time, and the quantifiable dissipative SP properties in the non-ideal, resistive case.
        
        We discuss possible options for extending this FET approach to timesteppers for the compressible case.

        The kinetic corrections satisfy linearized Boltzmann equations. Using a Lénard--Bernstein collision operator, these take Fokker--Planck-like forms \cite{Fokker_1914, Planck_1917} wherein pseudo-particles in the numerical model obey the neoclassical transport equations, with particle-independent Brownian drift terms. This offers a rigorous methodology for incorporating collisions into the particle transport model, without coupling the equations of motions for each particle.
        
        Works by Chen, Chacón et al. \cite{Chen_Chacón_Barnes_2011, Chacón_Chen_Barnes_2013, Chen_Chacón_2014, Chen_Chacón_2015} have developed structure-preserving particle pushers for neoclassical transport in the Vlasov equations, derived from Crank--Nicolson integrators. We show these too can can derive from a FET interpretation, similarly offering potential extensions to higher-order-in-time particle pushers. The FET formulation is used also to consider how the stochastic drift terms can be incorporated into the pushers. Stochastic gyrokinetic expansions are also discussed.

        Different options for the numerical implementation of these schemes are considered.

        Due to the efficacy of FET in the development of SP timesteppers for both the fluid and kinetic component, we hope this approach will prove effective in the future for developing SP timesteppers for the full hybrid model. We hope this will give us the opportunity to incorporate previously inaccessible kinetic effects into the highly effective, modern, finite-element MHD models.
    \end{abstract}
    
    
    \newpage
    \tableofcontents
    
    
    \newpage
    \pagenumbering{arabic}
    %\linenumbers\renewcommand\thelinenumber{\color{black!50}\arabic{linenumber}}
            \input{0 - introduction/main.tex}
        \part{Research}
            \input{1 - low-noise PiC models/main.tex}
            \input{2 - kinetic component/main.tex}
            \input{3 - fluid component/main.tex}
            \input{4 - numerical implementation/main.tex}
        \part{Project Overview}
            \input{5 - research plan/main.tex}
            \input{6 - summary/main.tex}
    
    
    %\section{}
    \newpage
    \pagenumbering{gobble}
        \printbibliography


    \newpage
    \pagenumbering{roman}
    \appendix
        \part{Appendices}
            \input{8 - Hilbert complexes/main.tex}
            \input{9 - weak conservation proofs/main.tex}
\end{document}

    
    
    %\section{}
    \newpage
    \pagenumbering{gobble}
        \printbibliography


    \newpage
    \pagenumbering{roman}
    \appendix
        \part{Appendices}
            \documentclass[12pt, a4paper]{report}

\input{template/main.tex}

\title{\BA{Title in Progress...}}
\author{Boris Andrews}
\affil{Mathematical Institute, University of Oxford}
\date{\today}


\begin{document}
    \pagenumbering{gobble}
    \maketitle
    
    
    \begin{abstract}
        Magnetic confinement reactors---in particular tokamaks---offer one of the most promising options for achieving practical nuclear fusion, with the potential to provide virtually limitless, clean energy. The theoretical and numerical modeling of tokamak plasmas is simultaneously an essential component of effective reactor design, and a great research barrier. Tokamak operational conditions exhibit comparatively low Knudsen numbers. Kinetic effects, including kinetic waves and instabilities, Landau damping, bump-on-tail instabilities and more, are therefore highly influential in tokamak plasma dynamics. Purely fluid models are inherently incapable of capturing these effects, whereas the high dimensionality in purely kinetic models render them practically intractable for most relevant purposes.

        We consider a $\delta\!f$ decomposition model, with a macroscopic fluid background and microscopic kinetic correction, both fully coupled to each other. A similar manner of discretization is proposed to that used in the recent \texttt{STRUPHY} code \cite{Holderied_Possanner_Wang_2021, Holderied_2022, Li_et_al_2023} with a finite-element model for the background and a pseudo-particle/PiC model for the correction.

        The fluid background satisfies the full, non-linear, resistive, compressible, Hall MHD equations. \cite{Laakmann_Hu_Farrell_2022} introduces finite-element(-in-space) implicit timesteppers for the incompressible analogue to this system with structure-preserving (SP) properties in the ideal case, alongside parameter-robust preconditioners. We show that these timesteppers can derive from a finite-element-in-time (FET) (and finite-element-in-space) interpretation. The benefits of this reformulation are discussed, including the derivation of timesteppers that are higher order in time, and the quantifiable dissipative SP properties in the non-ideal, resistive case.
        
        We discuss possible options for extending this FET approach to timesteppers for the compressible case.

        The kinetic corrections satisfy linearized Boltzmann equations. Using a Lénard--Bernstein collision operator, these take Fokker--Planck-like forms \cite{Fokker_1914, Planck_1917} wherein pseudo-particles in the numerical model obey the neoclassical transport equations, with particle-independent Brownian drift terms. This offers a rigorous methodology for incorporating collisions into the particle transport model, without coupling the equations of motions for each particle.
        
        Works by Chen, Chacón et al. \cite{Chen_Chacón_Barnes_2011, Chacón_Chen_Barnes_2013, Chen_Chacón_2014, Chen_Chacón_2015} have developed structure-preserving particle pushers for neoclassical transport in the Vlasov equations, derived from Crank--Nicolson integrators. We show these too can can derive from a FET interpretation, similarly offering potential extensions to higher-order-in-time particle pushers. The FET formulation is used also to consider how the stochastic drift terms can be incorporated into the pushers. Stochastic gyrokinetic expansions are also discussed.

        Different options for the numerical implementation of these schemes are considered.

        Due to the efficacy of FET in the development of SP timesteppers for both the fluid and kinetic component, we hope this approach will prove effective in the future for developing SP timesteppers for the full hybrid model. We hope this will give us the opportunity to incorporate previously inaccessible kinetic effects into the highly effective, modern, finite-element MHD models.
    \end{abstract}
    
    
    \newpage
    \tableofcontents
    
    
    \newpage
    \pagenumbering{arabic}
    %\linenumbers\renewcommand\thelinenumber{\color{black!50}\arabic{linenumber}}
            \input{0 - introduction/main.tex}
        \part{Research}
            \input{1 - low-noise PiC models/main.tex}
            \input{2 - kinetic component/main.tex}
            \input{3 - fluid component/main.tex}
            \input{4 - numerical implementation/main.tex}
        \part{Project Overview}
            \input{5 - research plan/main.tex}
            \input{6 - summary/main.tex}
    
    
    %\section{}
    \newpage
    \pagenumbering{gobble}
        \printbibliography


    \newpage
    \pagenumbering{roman}
    \appendix
        \part{Appendices}
            \input{8 - Hilbert complexes/main.tex}
            \input{9 - weak conservation proofs/main.tex}
\end{document}

            \documentclass[12pt, a4paper]{report}

\input{template/main.tex}

\title{\BA{Title in Progress...}}
\author{Boris Andrews}
\affil{Mathematical Institute, University of Oxford}
\date{\today}


\begin{document}
    \pagenumbering{gobble}
    \maketitle
    
    
    \begin{abstract}
        Magnetic confinement reactors---in particular tokamaks---offer one of the most promising options for achieving practical nuclear fusion, with the potential to provide virtually limitless, clean energy. The theoretical and numerical modeling of tokamak plasmas is simultaneously an essential component of effective reactor design, and a great research barrier. Tokamak operational conditions exhibit comparatively low Knudsen numbers. Kinetic effects, including kinetic waves and instabilities, Landau damping, bump-on-tail instabilities and more, are therefore highly influential in tokamak plasma dynamics. Purely fluid models are inherently incapable of capturing these effects, whereas the high dimensionality in purely kinetic models render them practically intractable for most relevant purposes.

        We consider a $\delta\!f$ decomposition model, with a macroscopic fluid background and microscopic kinetic correction, both fully coupled to each other. A similar manner of discretization is proposed to that used in the recent \texttt{STRUPHY} code \cite{Holderied_Possanner_Wang_2021, Holderied_2022, Li_et_al_2023} with a finite-element model for the background and a pseudo-particle/PiC model for the correction.

        The fluid background satisfies the full, non-linear, resistive, compressible, Hall MHD equations. \cite{Laakmann_Hu_Farrell_2022} introduces finite-element(-in-space) implicit timesteppers for the incompressible analogue to this system with structure-preserving (SP) properties in the ideal case, alongside parameter-robust preconditioners. We show that these timesteppers can derive from a finite-element-in-time (FET) (and finite-element-in-space) interpretation. The benefits of this reformulation are discussed, including the derivation of timesteppers that are higher order in time, and the quantifiable dissipative SP properties in the non-ideal, resistive case.
        
        We discuss possible options for extending this FET approach to timesteppers for the compressible case.

        The kinetic corrections satisfy linearized Boltzmann equations. Using a Lénard--Bernstein collision operator, these take Fokker--Planck-like forms \cite{Fokker_1914, Planck_1917} wherein pseudo-particles in the numerical model obey the neoclassical transport equations, with particle-independent Brownian drift terms. This offers a rigorous methodology for incorporating collisions into the particle transport model, without coupling the equations of motions for each particle.
        
        Works by Chen, Chacón et al. \cite{Chen_Chacón_Barnes_2011, Chacón_Chen_Barnes_2013, Chen_Chacón_2014, Chen_Chacón_2015} have developed structure-preserving particle pushers for neoclassical transport in the Vlasov equations, derived from Crank--Nicolson integrators. We show these too can can derive from a FET interpretation, similarly offering potential extensions to higher-order-in-time particle pushers. The FET formulation is used also to consider how the stochastic drift terms can be incorporated into the pushers. Stochastic gyrokinetic expansions are also discussed.

        Different options for the numerical implementation of these schemes are considered.

        Due to the efficacy of FET in the development of SP timesteppers for both the fluid and kinetic component, we hope this approach will prove effective in the future for developing SP timesteppers for the full hybrid model. We hope this will give us the opportunity to incorporate previously inaccessible kinetic effects into the highly effective, modern, finite-element MHD models.
    \end{abstract}
    
    
    \newpage
    \tableofcontents
    
    
    \newpage
    \pagenumbering{arabic}
    %\linenumbers\renewcommand\thelinenumber{\color{black!50}\arabic{linenumber}}
            \input{0 - introduction/main.tex}
        \part{Research}
            \input{1 - low-noise PiC models/main.tex}
            \input{2 - kinetic component/main.tex}
            \input{3 - fluid component/main.tex}
            \input{4 - numerical implementation/main.tex}
        \part{Project Overview}
            \input{5 - research plan/main.tex}
            \input{6 - summary/main.tex}
    
    
    %\section{}
    \newpage
    \pagenumbering{gobble}
        \printbibliography


    \newpage
    \pagenumbering{roman}
    \appendix
        \part{Appendices}
            \input{8 - Hilbert complexes/main.tex}
            \input{9 - weak conservation proofs/main.tex}
\end{document}

\end{document}

            \documentclass[12pt, a4paper]{report}

\documentclass[12pt, a4paper]{report}

\input{template/main.tex}

\title{\BA{Title in Progress...}}
\author{Boris Andrews}
\affil{Mathematical Institute, University of Oxford}
\date{\today}


\begin{document}
    \pagenumbering{gobble}
    \maketitle
    
    
    \begin{abstract}
        Magnetic confinement reactors---in particular tokamaks---offer one of the most promising options for achieving practical nuclear fusion, with the potential to provide virtually limitless, clean energy. The theoretical and numerical modeling of tokamak plasmas is simultaneously an essential component of effective reactor design, and a great research barrier. Tokamak operational conditions exhibit comparatively low Knudsen numbers. Kinetic effects, including kinetic waves and instabilities, Landau damping, bump-on-tail instabilities and more, are therefore highly influential in tokamak plasma dynamics. Purely fluid models are inherently incapable of capturing these effects, whereas the high dimensionality in purely kinetic models render them practically intractable for most relevant purposes.

        We consider a $\delta\!f$ decomposition model, with a macroscopic fluid background and microscopic kinetic correction, both fully coupled to each other. A similar manner of discretization is proposed to that used in the recent \texttt{STRUPHY} code \cite{Holderied_Possanner_Wang_2021, Holderied_2022, Li_et_al_2023} with a finite-element model for the background and a pseudo-particle/PiC model for the correction.

        The fluid background satisfies the full, non-linear, resistive, compressible, Hall MHD equations. \cite{Laakmann_Hu_Farrell_2022} introduces finite-element(-in-space) implicit timesteppers for the incompressible analogue to this system with structure-preserving (SP) properties in the ideal case, alongside parameter-robust preconditioners. We show that these timesteppers can derive from a finite-element-in-time (FET) (and finite-element-in-space) interpretation. The benefits of this reformulation are discussed, including the derivation of timesteppers that are higher order in time, and the quantifiable dissipative SP properties in the non-ideal, resistive case.
        
        We discuss possible options for extending this FET approach to timesteppers for the compressible case.

        The kinetic corrections satisfy linearized Boltzmann equations. Using a Lénard--Bernstein collision operator, these take Fokker--Planck-like forms \cite{Fokker_1914, Planck_1917} wherein pseudo-particles in the numerical model obey the neoclassical transport equations, with particle-independent Brownian drift terms. This offers a rigorous methodology for incorporating collisions into the particle transport model, without coupling the equations of motions for each particle.
        
        Works by Chen, Chacón et al. \cite{Chen_Chacón_Barnes_2011, Chacón_Chen_Barnes_2013, Chen_Chacón_2014, Chen_Chacón_2015} have developed structure-preserving particle pushers for neoclassical transport in the Vlasov equations, derived from Crank--Nicolson integrators. We show these too can can derive from a FET interpretation, similarly offering potential extensions to higher-order-in-time particle pushers. The FET formulation is used also to consider how the stochastic drift terms can be incorporated into the pushers. Stochastic gyrokinetic expansions are also discussed.

        Different options for the numerical implementation of these schemes are considered.

        Due to the efficacy of FET in the development of SP timesteppers for both the fluid and kinetic component, we hope this approach will prove effective in the future for developing SP timesteppers for the full hybrid model. We hope this will give us the opportunity to incorporate previously inaccessible kinetic effects into the highly effective, modern, finite-element MHD models.
    \end{abstract}
    
    
    \newpage
    \tableofcontents
    
    
    \newpage
    \pagenumbering{arabic}
    %\linenumbers\renewcommand\thelinenumber{\color{black!50}\arabic{linenumber}}
            \input{0 - introduction/main.tex}
        \part{Research}
            \input{1 - low-noise PiC models/main.tex}
            \input{2 - kinetic component/main.tex}
            \input{3 - fluid component/main.tex}
            \input{4 - numerical implementation/main.tex}
        \part{Project Overview}
            \input{5 - research plan/main.tex}
            \input{6 - summary/main.tex}
    
    
    %\section{}
    \newpage
    \pagenumbering{gobble}
        \printbibliography


    \newpage
    \pagenumbering{roman}
    \appendix
        \part{Appendices}
            \input{8 - Hilbert complexes/main.tex}
            \input{9 - weak conservation proofs/main.tex}
\end{document}


\title{\BA{Title in Progress...}}
\author{Boris Andrews}
\affil{Mathematical Institute, University of Oxford}
\date{\today}


\begin{document}
    \pagenumbering{gobble}
    \maketitle
    
    
    \begin{abstract}
        Magnetic confinement reactors---in particular tokamaks---offer one of the most promising options for achieving practical nuclear fusion, with the potential to provide virtually limitless, clean energy. The theoretical and numerical modeling of tokamak plasmas is simultaneously an essential component of effective reactor design, and a great research barrier. Tokamak operational conditions exhibit comparatively low Knudsen numbers. Kinetic effects, including kinetic waves and instabilities, Landau damping, bump-on-tail instabilities and more, are therefore highly influential in tokamak plasma dynamics. Purely fluid models are inherently incapable of capturing these effects, whereas the high dimensionality in purely kinetic models render them practically intractable for most relevant purposes.

        We consider a $\delta\!f$ decomposition model, with a macroscopic fluid background and microscopic kinetic correction, both fully coupled to each other. A similar manner of discretization is proposed to that used in the recent \texttt{STRUPHY} code \cite{Holderied_Possanner_Wang_2021, Holderied_2022, Li_et_al_2023} with a finite-element model for the background and a pseudo-particle/PiC model for the correction.

        The fluid background satisfies the full, non-linear, resistive, compressible, Hall MHD equations. \cite{Laakmann_Hu_Farrell_2022} introduces finite-element(-in-space) implicit timesteppers for the incompressible analogue to this system with structure-preserving (SP) properties in the ideal case, alongside parameter-robust preconditioners. We show that these timesteppers can derive from a finite-element-in-time (FET) (and finite-element-in-space) interpretation. The benefits of this reformulation are discussed, including the derivation of timesteppers that are higher order in time, and the quantifiable dissipative SP properties in the non-ideal, resistive case.
        
        We discuss possible options for extending this FET approach to timesteppers for the compressible case.

        The kinetic corrections satisfy linearized Boltzmann equations. Using a Lénard--Bernstein collision operator, these take Fokker--Planck-like forms \cite{Fokker_1914, Planck_1917} wherein pseudo-particles in the numerical model obey the neoclassical transport equations, with particle-independent Brownian drift terms. This offers a rigorous methodology for incorporating collisions into the particle transport model, without coupling the equations of motions for each particle.
        
        Works by Chen, Chacón et al. \cite{Chen_Chacón_Barnes_2011, Chacón_Chen_Barnes_2013, Chen_Chacón_2014, Chen_Chacón_2015} have developed structure-preserving particle pushers for neoclassical transport in the Vlasov equations, derived from Crank--Nicolson integrators. We show these too can can derive from a FET interpretation, similarly offering potential extensions to higher-order-in-time particle pushers. The FET formulation is used also to consider how the stochastic drift terms can be incorporated into the pushers. Stochastic gyrokinetic expansions are also discussed.

        Different options for the numerical implementation of these schemes are considered.

        Due to the efficacy of FET in the development of SP timesteppers for both the fluid and kinetic component, we hope this approach will prove effective in the future for developing SP timesteppers for the full hybrid model. We hope this will give us the opportunity to incorporate previously inaccessible kinetic effects into the highly effective, modern, finite-element MHD models.
    \end{abstract}
    
    
    \newpage
    \tableofcontents
    
    
    \newpage
    \pagenumbering{arabic}
    %\linenumbers\renewcommand\thelinenumber{\color{black!50}\arabic{linenumber}}
            \documentclass[12pt, a4paper]{report}

\input{template/main.tex}

\title{\BA{Title in Progress...}}
\author{Boris Andrews}
\affil{Mathematical Institute, University of Oxford}
\date{\today}


\begin{document}
    \pagenumbering{gobble}
    \maketitle
    
    
    \begin{abstract}
        Magnetic confinement reactors---in particular tokamaks---offer one of the most promising options for achieving practical nuclear fusion, with the potential to provide virtually limitless, clean energy. The theoretical and numerical modeling of tokamak plasmas is simultaneously an essential component of effective reactor design, and a great research barrier. Tokamak operational conditions exhibit comparatively low Knudsen numbers. Kinetic effects, including kinetic waves and instabilities, Landau damping, bump-on-tail instabilities and more, are therefore highly influential in tokamak plasma dynamics. Purely fluid models are inherently incapable of capturing these effects, whereas the high dimensionality in purely kinetic models render them practically intractable for most relevant purposes.

        We consider a $\delta\!f$ decomposition model, with a macroscopic fluid background and microscopic kinetic correction, both fully coupled to each other. A similar manner of discretization is proposed to that used in the recent \texttt{STRUPHY} code \cite{Holderied_Possanner_Wang_2021, Holderied_2022, Li_et_al_2023} with a finite-element model for the background and a pseudo-particle/PiC model for the correction.

        The fluid background satisfies the full, non-linear, resistive, compressible, Hall MHD equations. \cite{Laakmann_Hu_Farrell_2022} introduces finite-element(-in-space) implicit timesteppers for the incompressible analogue to this system with structure-preserving (SP) properties in the ideal case, alongside parameter-robust preconditioners. We show that these timesteppers can derive from a finite-element-in-time (FET) (and finite-element-in-space) interpretation. The benefits of this reformulation are discussed, including the derivation of timesteppers that are higher order in time, and the quantifiable dissipative SP properties in the non-ideal, resistive case.
        
        We discuss possible options for extending this FET approach to timesteppers for the compressible case.

        The kinetic corrections satisfy linearized Boltzmann equations. Using a Lénard--Bernstein collision operator, these take Fokker--Planck-like forms \cite{Fokker_1914, Planck_1917} wherein pseudo-particles in the numerical model obey the neoclassical transport equations, with particle-independent Brownian drift terms. This offers a rigorous methodology for incorporating collisions into the particle transport model, without coupling the equations of motions for each particle.
        
        Works by Chen, Chacón et al. \cite{Chen_Chacón_Barnes_2011, Chacón_Chen_Barnes_2013, Chen_Chacón_2014, Chen_Chacón_2015} have developed structure-preserving particle pushers for neoclassical transport in the Vlasov equations, derived from Crank--Nicolson integrators. We show these too can can derive from a FET interpretation, similarly offering potential extensions to higher-order-in-time particle pushers. The FET formulation is used also to consider how the stochastic drift terms can be incorporated into the pushers. Stochastic gyrokinetic expansions are also discussed.

        Different options for the numerical implementation of these schemes are considered.

        Due to the efficacy of FET in the development of SP timesteppers for both the fluid and kinetic component, we hope this approach will prove effective in the future for developing SP timesteppers for the full hybrid model. We hope this will give us the opportunity to incorporate previously inaccessible kinetic effects into the highly effective, modern, finite-element MHD models.
    \end{abstract}
    
    
    \newpage
    \tableofcontents
    
    
    \newpage
    \pagenumbering{arabic}
    %\linenumbers\renewcommand\thelinenumber{\color{black!50}\arabic{linenumber}}
            \input{0 - introduction/main.tex}
        \part{Research}
            \input{1 - low-noise PiC models/main.tex}
            \input{2 - kinetic component/main.tex}
            \input{3 - fluid component/main.tex}
            \input{4 - numerical implementation/main.tex}
        \part{Project Overview}
            \input{5 - research plan/main.tex}
            \input{6 - summary/main.tex}
    
    
    %\section{}
    \newpage
    \pagenumbering{gobble}
        \printbibliography


    \newpage
    \pagenumbering{roman}
    \appendix
        \part{Appendices}
            \input{8 - Hilbert complexes/main.tex}
            \input{9 - weak conservation proofs/main.tex}
\end{document}

        \part{Research}
            \documentclass[12pt, a4paper]{report}

\input{template/main.tex}

\title{\BA{Title in Progress...}}
\author{Boris Andrews}
\affil{Mathematical Institute, University of Oxford}
\date{\today}


\begin{document}
    \pagenumbering{gobble}
    \maketitle
    
    
    \begin{abstract}
        Magnetic confinement reactors---in particular tokamaks---offer one of the most promising options for achieving practical nuclear fusion, with the potential to provide virtually limitless, clean energy. The theoretical and numerical modeling of tokamak plasmas is simultaneously an essential component of effective reactor design, and a great research barrier. Tokamak operational conditions exhibit comparatively low Knudsen numbers. Kinetic effects, including kinetic waves and instabilities, Landau damping, bump-on-tail instabilities and more, are therefore highly influential in tokamak plasma dynamics. Purely fluid models are inherently incapable of capturing these effects, whereas the high dimensionality in purely kinetic models render them practically intractable for most relevant purposes.

        We consider a $\delta\!f$ decomposition model, with a macroscopic fluid background and microscopic kinetic correction, both fully coupled to each other. A similar manner of discretization is proposed to that used in the recent \texttt{STRUPHY} code \cite{Holderied_Possanner_Wang_2021, Holderied_2022, Li_et_al_2023} with a finite-element model for the background and a pseudo-particle/PiC model for the correction.

        The fluid background satisfies the full, non-linear, resistive, compressible, Hall MHD equations. \cite{Laakmann_Hu_Farrell_2022} introduces finite-element(-in-space) implicit timesteppers for the incompressible analogue to this system with structure-preserving (SP) properties in the ideal case, alongside parameter-robust preconditioners. We show that these timesteppers can derive from a finite-element-in-time (FET) (and finite-element-in-space) interpretation. The benefits of this reformulation are discussed, including the derivation of timesteppers that are higher order in time, and the quantifiable dissipative SP properties in the non-ideal, resistive case.
        
        We discuss possible options for extending this FET approach to timesteppers for the compressible case.

        The kinetic corrections satisfy linearized Boltzmann equations. Using a Lénard--Bernstein collision operator, these take Fokker--Planck-like forms \cite{Fokker_1914, Planck_1917} wherein pseudo-particles in the numerical model obey the neoclassical transport equations, with particle-independent Brownian drift terms. This offers a rigorous methodology for incorporating collisions into the particle transport model, without coupling the equations of motions for each particle.
        
        Works by Chen, Chacón et al. \cite{Chen_Chacón_Barnes_2011, Chacón_Chen_Barnes_2013, Chen_Chacón_2014, Chen_Chacón_2015} have developed structure-preserving particle pushers for neoclassical transport in the Vlasov equations, derived from Crank--Nicolson integrators. We show these too can can derive from a FET interpretation, similarly offering potential extensions to higher-order-in-time particle pushers. The FET formulation is used also to consider how the stochastic drift terms can be incorporated into the pushers. Stochastic gyrokinetic expansions are also discussed.

        Different options for the numerical implementation of these schemes are considered.

        Due to the efficacy of FET in the development of SP timesteppers for both the fluid and kinetic component, we hope this approach will prove effective in the future for developing SP timesteppers for the full hybrid model. We hope this will give us the opportunity to incorporate previously inaccessible kinetic effects into the highly effective, modern, finite-element MHD models.
    \end{abstract}
    
    
    \newpage
    \tableofcontents
    
    
    \newpage
    \pagenumbering{arabic}
    %\linenumbers\renewcommand\thelinenumber{\color{black!50}\arabic{linenumber}}
            \input{0 - introduction/main.tex}
        \part{Research}
            \input{1 - low-noise PiC models/main.tex}
            \input{2 - kinetic component/main.tex}
            \input{3 - fluid component/main.tex}
            \input{4 - numerical implementation/main.tex}
        \part{Project Overview}
            \input{5 - research plan/main.tex}
            \input{6 - summary/main.tex}
    
    
    %\section{}
    \newpage
    \pagenumbering{gobble}
        \printbibliography


    \newpage
    \pagenumbering{roman}
    \appendix
        \part{Appendices}
            \input{8 - Hilbert complexes/main.tex}
            \input{9 - weak conservation proofs/main.tex}
\end{document}

            \documentclass[12pt, a4paper]{report}

\input{template/main.tex}

\title{\BA{Title in Progress...}}
\author{Boris Andrews}
\affil{Mathematical Institute, University of Oxford}
\date{\today}


\begin{document}
    \pagenumbering{gobble}
    \maketitle
    
    
    \begin{abstract}
        Magnetic confinement reactors---in particular tokamaks---offer one of the most promising options for achieving practical nuclear fusion, with the potential to provide virtually limitless, clean energy. The theoretical and numerical modeling of tokamak plasmas is simultaneously an essential component of effective reactor design, and a great research barrier. Tokamak operational conditions exhibit comparatively low Knudsen numbers. Kinetic effects, including kinetic waves and instabilities, Landau damping, bump-on-tail instabilities and more, are therefore highly influential in tokamak plasma dynamics. Purely fluid models are inherently incapable of capturing these effects, whereas the high dimensionality in purely kinetic models render them practically intractable for most relevant purposes.

        We consider a $\delta\!f$ decomposition model, with a macroscopic fluid background and microscopic kinetic correction, both fully coupled to each other. A similar manner of discretization is proposed to that used in the recent \texttt{STRUPHY} code \cite{Holderied_Possanner_Wang_2021, Holderied_2022, Li_et_al_2023} with a finite-element model for the background and a pseudo-particle/PiC model for the correction.

        The fluid background satisfies the full, non-linear, resistive, compressible, Hall MHD equations. \cite{Laakmann_Hu_Farrell_2022} introduces finite-element(-in-space) implicit timesteppers for the incompressible analogue to this system with structure-preserving (SP) properties in the ideal case, alongside parameter-robust preconditioners. We show that these timesteppers can derive from a finite-element-in-time (FET) (and finite-element-in-space) interpretation. The benefits of this reformulation are discussed, including the derivation of timesteppers that are higher order in time, and the quantifiable dissipative SP properties in the non-ideal, resistive case.
        
        We discuss possible options for extending this FET approach to timesteppers for the compressible case.

        The kinetic corrections satisfy linearized Boltzmann equations. Using a Lénard--Bernstein collision operator, these take Fokker--Planck-like forms \cite{Fokker_1914, Planck_1917} wherein pseudo-particles in the numerical model obey the neoclassical transport equations, with particle-independent Brownian drift terms. This offers a rigorous methodology for incorporating collisions into the particle transport model, without coupling the equations of motions for each particle.
        
        Works by Chen, Chacón et al. \cite{Chen_Chacón_Barnes_2011, Chacón_Chen_Barnes_2013, Chen_Chacón_2014, Chen_Chacón_2015} have developed structure-preserving particle pushers for neoclassical transport in the Vlasov equations, derived from Crank--Nicolson integrators. We show these too can can derive from a FET interpretation, similarly offering potential extensions to higher-order-in-time particle pushers. The FET formulation is used also to consider how the stochastic drift terms can be incorporated into the pushers. Stochastic gyrokinetic expansions are also discussed.

        Different options for the numerical implementation of these schemes are considered.

        Due to the efficacy of FET in the development of SP timesteppers for both the fluid and kinetic component, we hope this approach will prove effective in the future for developing SP timesteppers for the full hybrid model. We hope this will give us the opportunity to incorporate previously inaccessible kinetic effects into the highly effective, modern, finite-element MHD models.
    \end{abstract}
    
    
    \newpage
    \tableofcontents
    
    
    \newpage
    \pagenumbering{arabic}
    %\linenumbers\renewcommand\thelinenumber{\color{black!50}\arabic{linenumber}}
            \input{0 - introduction/main.tex}
        \part{Research}
            \input{1 - low-noise PiC models/main.tex}
            \input{2 - kinetic component/main.tex}
            \input{3 - fluid component/main.tex}
            \input{4 - numerical implementation/main.tex}
        \part{Project Overview}
            \input{5 - research plan/main.tex}
            \input{6 - summary/main.tex}
    
    
    %\section{}
    \newpage
    \pagenumbering{gobble}
        \printbibliography


    \newpage
    \pagenumbering{roman}
    \appendix
        \part{Appendices}
            \input{8 - Hilbert complexes/main.tex}
            \input{9 - weak conservation proofs/main.tex}
\end{document}

            \documentclass[12pt, a4paper]{report}

\input{template/main.tex}

\title{\BA{Title in Progress...}}
\author{Boris Andrews}
\affil{Mathematical Institute, University of Oxford}
\date{\today}


\begin{document}
    \pagenumbering{gobble}
    \maketitle
    
    
    \begin{abstract}
        Magnetic confinement reactors---in particular tokamaks---offer one of the most promising options for achieving practical nuclear fusion, with the potential to provide virtually limitless, clean energy. The theoretical and numerical modeling of tokamak plasmas is simultaneously an essential component of effective reactor design, and a great research barrier. Tokamak operational conditions exhibit comparatively low Knudsen numbers. Kinetic effects, including kinetic waves and instabilities, Landau damping, bump-on-tail instabilities and more, are therefore highly influential in tokamak plasma dynamics. Purely fluid models are inherently incapable of capturing these effects, whereas the high dimensionality in purely kinetic models render them practically intractable for most relevant purposes.

        We consider a $\delta\!f$ decomposition model, with a macroscopic fluid background and microscopic kinetic correction, both fully coupled to each other. A similar manner of discretization is proposed to that used in the recent \texttt{STRUPHY} code \cite{Holderied_Possanner_Wang_2021, Holderied_2022, Li_et_al_2023} with a finite-element model for the background and a pseudo-particle/PiC model for the correction.

        The fluid background satisfies the full, non-linear, resistive, compressible, Hall MHD equations. \cite{Laakmann_Hu_Farrell_2022} introduces finite-element(-in-space) implicit timesteppers for the incompressible analogue to this system with structure-preserving (SP) properties in the ideal case, alongside parameter-robust preconditioners. We show that these timesteppers can derive from a finite-element-in-time (FET) (and finite-element-in-space) interpretation. The benefits of this reformulation are discussed, including the derivation of timesteppers that are higher order in time, and the quantifiable dissipative SP properties in the non-ideal, resistive case.
        
        We discuss possible options for extending this FET approach to timesteppers for the compressible case.

        The kinetic corrections satisfy linearized Boltzmann equations. Using a Lénard--Bernstein collision operator, these take Fokker--Planck-like forms \cite{Fokker_1914, Planck_1917} wherein pseudo-particles in the numerical model obey the neoclassical transport equations, with particle-independent Brownian drift terms. This offers a rigorous methodology for incorporating collisions into the particle transport model, without coupling the equations of motions for each particle.
        
        Works by Chen, Chacón et al. \cite{Chen_Chacón_Barnes_2011, Chacón_Chen_Barnes_2013, Chen_Chacón_2014, Chen_Chacón_2015} have developed structure-preserving particle pushers for neoclassical transport in the Vlasov equations, derived from Crank--Nicolson integrators. We show these too can can derive from a FET interpretation, similarly offering potential extensions to higher-order-in-time particle pushers. The FET formulation is used also to consider how the stochastic drift terms can be incorporated into the pushers. Stochastic gyrokinetic expansions are also discussed.

        Different options for the numerical implementation of these schemes are considered.

        Due to the efficacy of FET in the development of SP timesteppers for both the fluid and kinetic component, we hope this approach will prove effective in the future for developing SP timesteppers for the full hybrid model. We hope this will give us the opportunity to incorporate previously inaccessible kinetic effects into the highly effective, modern, finite-element MHD models.
    \end{abstract}
    
    
    \newpage
    \tableofcontents
    
    
    \newpage
    \pagenumbering{arabic}
    %\linenumbers\renewcommand\thelinenumber{\color{black!50}\arabic{linenumber}}
            \input{0 - introduction/main.tex}
        \part{Research}
            \input{1 - low-noise PiC models/main.tex}
            \input{2 - kinetic component/main.tex}
            \input{3 - fluid component/main.tex}
            \input{4 - numerical implementation/main.tex}
        \part{Project Overview}
            \input{5 - research plan/main.tex}
            \input{6 - summary/main.tex}
    
    
    %\section{}
    \newpage
    \pagenumbering{gobble}
        \printbibliography


    \newpage
    \pagenumbering{roman}
    \appendix
        \part{Appendices}
            \input{8 - Hilbert complexes/main.tex}
            \input{9 - weak conservation proofs/main.tex}
\end{document}

            \documentclass[12pt, a4paper]{report}

\input{template/main.tex}

\title{\BA{Title in Progress...}}
\author{Boris Andrews}
\affil{Mathematical Institute, University of Oxford}
\date{\today}


\begin{document}
    \pagenumbering{gobble}
    \maketitle
    
    
    \begin{abstract}
        Magnetic confinement reactors---in particular tokamaks---offer one of the most promising options for achieving practical nuclear fusion, with the potential to provide virtually limitless, clean energy. The theoretical and numerical modeling of tokamak plasmas is simultaneously an essential component of effective reactor design, and a great research barrier. Tokamak operational conditions exhibit comparatively low Knudsen numbers. Kinetic effects, including kinetic waves and instabilities, Landau damping, bump-on-tail instabilities and more, are therefore highly influential in tokamak plasma dynamics. Purely fluid models are inherently incapable of capturing these effects, whereas the high dimensionality in purely kinetic models render them practically intractable for most relevant purposes.

        We consider a $\delta\!f$ decomposition model, with a macroscopic fluid background and microscopic kinetic correction, both fully coupled to each other. A similar manner of discretization is proposed to that used in the recent \texttt{STRUPHY} code \cite{Holderied_Possanner_Wang_2021, Holderied_2022, Li_et_al_2023} with a finite-element model for the background and a pseudo-particle/PiC model for the correction.

        The fluid background satisfies the full, non-linear, resistive, compressible, Hall MHD equations. \cite{Laakmann_Hu_Farrell_2022} introduces finite-element(-in-space) implicit timesteppers for the incompressible analogue to this system with structure-preserving (SP) properties in the ideal case, alongside parameter-robust preconditioners. We show that these timesteppers can derive from a finite-element-in-time (FET) (and finite-element-in-space) interpretation. The benefits of this reformulation are discussed, including the derivation of timesteppers that are higher order in time, and the quantifiable dissipative SP properties in the non-ideal, resistive case.
        
        We discuss possible options for extending this FET approach to timesteppers for the compressible case.

        The kinetic corrections satisfy linearized Boltzmann equations. Using a Lénard--Bernstein collision operator, these take Fokker--Planck-like forms \cite{Fokker_1914, Planck_1917} wherein pseudo-particles in the numerical model obey the neoclassical transport equations, with particle-independent Brownian drift terms. This offers a rigorous methodology for incorporating collisions into the particle transport model, without coupling the equations of motions for each particle.
        
        Works by Chen, Chacón et al. \cite{Chen_Chacón_Barnes_2011, Chacón_Chen_Barnes_2013, Chen_Chacón_2014, Chen_Chacón_2015} have developed structure-preserving particle pushers for neoclassical transport in the Vlasov equations, derived from Crank--Nicolson integrators. We show these too can can derive from a FET interpretation, similarly offering potential extensions to higher-order-in-time particle pushers. The FET formulation is used also to consider how the stochastic drift terms can be incorporated into the pushers. Stochastic gyrokinetic expansions are also discussed.

        Different options for the numerical implementation of these schemes are considered.

        Due to the efficacy of FET in the development of SP timesteppers for both the fluid and kinetic component, we hope this approach will prove effective in the future for developing SP timesteppers for the full hybrid model. We hope this will give us the opportunity to incorporate previously inaccessible kinetic effects into the highly effective, modern, finite-element MHD models.
    \end{abstract}
    
    
    \newpage
    \tableofcontents
    
    
    \newpage
    \pagenumbering{arabic}
    %\linenumbers\renewcommand\thelinenumber{\color{black!50}\arabic{linenumber}}
            \input{0 - introduction/main.tex}
        \part{Research}
            \input{1 - low-noise PiC models/main.tex}
            \input{2 - kinetic component/main.tex}
            \input{3 - fluid component/main.tex}
            \input{4 - numerical implementation/main.tex}
        \part{Project Overview}
            \input{5 - research plan/main.tex}
            \input{6 - summary/main.tex}
    
    
    %\section{}
    \newpage
    \pagenumbering{gobble}
        \printbibliography


    \newpage
    \pagenumbering{roman}
    \appendix
        \part{Appendices}
            \input{8 - Hilbert complexes/main.tex}
            \input{9 - weak conservation proofs/main.tex}
\end{document}

        \part{Project Overview}
            \documentclass[12pt, a4paper]{report}

\input{template/main.tex}

\title{\BA{Title in Progress...}}
\author{Boris Andrews}
\affil{Mathematical Institute, University of Oxford}
\date{\today}


\begin{document}
    \pagenumbering{gobble}
    \maketitle
    
    
    \begin{abstract}
        Magnetic confinement reactors---in particular tokamaks---offer one of the most promising options for achieving practical nuclear fusion, with the potential to provide virtually limitless, clean energy. The theoretical and numerical modeling of tokamak plasmas is simultaneously an essential component of effective reactor design, and a great research barrier. Tokamak operational conditions exhibit comparatively low Knudsen numbers. Kinetic effects, including kinetic waves and instabilities, Landau damping, bump-on-tail instabilities and more, are therefore highly influential in tokamak plasma dynamics. Purely fluid models are inherently incapable of capturing these effects, whereas the high dimensionality in purely kinetic models render them practically intractable for most relevant purposes.

        We consider a $\delta\!f$ decomposition model, with a macroscopic fluid background and microscopic kinetic correction, both fully coupled to each other. A similar manner of discretization is proposed to that used in the recent \texttt{STRUPHY} code \cite{Holderied_Possanner_Wang_2021, Holderied_2022, Li_et_al_2023} with a finite-element model for the background and a pseudo-particle/PiC model for the correction.

        The fluid background satisfies the full, non-linear, resistive, compressible, Hall MHD equations. \cite{Laakmann_Hu_Farrell_2022} introduces finite-element(-in-space) implicit timesteppers for the incompressible analogue to this system with structure-preserving (SP) properties in the ideal case, alongside parameter-robust preconditioners. We show that these timesteppers can derive from a finite-element-in-time (FET) (and finite-element-in-space) interpretation. The benefits of this reformulation are discussed, including the derivation of timesteppers that are higher order in time, and the quantifiable dissipative SP properties in the non-ideal, resistive case.
        
        We discuss possible options for extending this FET approach to timesteppers for the compressible case.

        The kinetic corrections satisfy linearized Boltzmann equations. Using a Lénard--Bernstein collision operator, these take Fokker--Planck-like forms \cite{Fokker_1914, Planck_1917} wherein pseudo-particles in the numerical model obey the neoclassical transport equations, with particle-independent Brownian drift terms. This offers a rigorous methodology for incorporating collisions into the particle transport model, without coupling the equations of motions for each particle.
        
        Works by Chen, Chacón et al. \cite{Chen_Chacón_Barnes_2011, Chacón_Chen_Barnes_2013, Chen_Chacón_2014, Chen_Chacón_2015} have developed structure-preserving particle pushers for neoclassical transport in the Vlasov equations, derived from Crank--Nicolson integrators. We show these too can can derive from a FET interpretation, similarly offering potential extensions to higher-order-in-time particle pushers. The FET formulation is used also to consider how the stochastic drift terms can be incorporated into the pushers. Stochastic gyrokinetic expansions are also discussed.

        Different options for the numerical implementation of these schemes are considered.

        Due to the efficacy of FET in the development of SP timesteppers for both the fluid and kinetic component, we hope this approach will prove effective in the future for developing SP timesteppers for the full hybrid model. We hope this will give us the opportunity to incorporate previously inaccessible kinetic effects into the highly effective, modern, finite-element MHD models.
    \end{abstract}
    
    
    \newpage
    \tableofcontents
    
    
    \newpage
    \pagenumbering{arabic}
    %\linenumbers\renewcommand\thelinenumber{\color{black!50}\arabic{linenumber}}
            \input{0 - introduction/main.tex}
        \part{Research}
            \input{1 - low-noise PiC models/main.tex}
            \input{2 - kinetic component/main.tex}
            \input{3 - fluid component/main.tex}
            \input{4 - numerical implementation/main.tex}
        \part{Project Overview}
            \input{5 - research plan/main.tex}
            \input{6 - summary/main.tex}
    
    
    %\section{}
    \newpage
    \pagenumbering{gobble}
        \printbibliography


    \newpage
    \pagenumbering{roman}
    \appendix
        \part{Appendices}
            \input{8 - Hilbert complexes/main.tex}
            \input{9 - weak conservation proofs/main.tex}
\end{document}

            \documentclass[12pt, a4paper]{report}

\input{template/main.tex}

\title{\BA{Title in Progress...}}
\author{Boris Andrews}
\affil{Mathematical Institute, University of Oxford}
\date{\today}


\begin{document}
    \pagenumbering{gobble}
    \maketitle
    
    
    \begin{abstract}
        Magnetic confinement reactors---in particular tokamaks---offer one of the most promising options for achieving practical nuclear fusion, with the potential to provide virtually limitless, clean energy. The theoretical and numerical modeling of tokamak plasmas is simultaneously an essential component of effective reactor design, and a great research barrier. Tokamak operational conditions exhibit comparatively low Knudsen numbers. Kinetic effects, including kinetic waves and instabilities, Landau damping, bump-on-tail instabilities and more, are therefore highly influential in tokamak plasma dynamics. Purely fluid models are inherently incapable of capturing these effects, whereas the high dimensionality in purely kinetic models render them practically intractable for most relevant purposes.

        We consider a $\delta\!f$ decomposition model, with a macroscopic fluid background and microscopic kinetic correction, both fully coupled to each other. A similar manner of discretization is proposed to that used in the recent \texttt{STRUPHY} code \cite{Holderied_Possanner_Wang_2021, Holderied_2022, Li_et_al_2023} with a finite-element model for the background and a pseudo-particle/PiC model for the correction.

        The fluid background satisfies the full, non-linear, resistive, compressible, Hall MHD equations. \cite{Laakmann_Hu_Farrell_2022} introduces finite-element(-in-space) implicit timesteppers for the incompressible analogue to this system with structure-preserving (SP) properties in the ideal case, alongside parameter-robust preconditioners. We show that these timesteppers can derive from a finite-element-in-time (FET) (and finite-element-in-space) interpretation. The benefits of this reformulation are discussed, including the derivation of timesteppers that are higher order in time, and the quantifiable dissipative SP properties in the non-ideal, resistive case.
        
        We discuss possible options for extending this FET approach to timesteppers for the compressible case.

        The kinetic corrections satisfy linearized Boltzmann equations. Using a Lénard--Bernstein collision operator, these take Fokker--Planck-like forms \cite{Fokker_1914, Planck_1917} wherein pseudo-particles in the numerical model obey the neoclassical transport equations, with particle-independent Brownian drift terms. This offers a rigorous methodology for incorporating collisions into the particle transport model, without coupling the equations of motions for each particle.
        
        Works by Chen, Chacón et al. \cite{Chen_Chacón_Barnes_2011, Chacón_Chen_Barnes_2013, Chen_Chacón_2014, Chen_Chacón_2015} have developed structure-preserving particle pushers for neoclassical transport in the Vlasov equations, derived from Crank--Nicolson integrators. We show these too can can derive from a FET interpretation, similarly offering potential extensions to higher-order-in-time particle pushers. The FET formulation is used also to consider how the stochastic drift terms can be incorporated into the pushers. Stochastic gyrokinetic expansions are also discussed.

        Different options for the numerical implementation of these schemes are considered.

        Due to the efficacy of FET in the development of SP timesteppers for both the fluid and kinetic component, we hope this approach will prove effective in the future for developing SP timesteppers for the full hybrid model. We hope this will give us the opportunity to incorporate previously inaccessible kinetic effects into the highly effective, modern, finite-element MHD models.
    \end{abstract}
    
    
    \newpage
    \tableofcontents
    
    
    \newpage
    \pagenumbering{arabic}
    %\linenumbers\renewcommand\thelinenumber{\color{black!50}\arabic{linenumber}}
            \input{0 - introduction/main.tex}
        \part{Research}
            \input{1 - low-noise PiC models/main.tex}
            \input{2 - kinetic component/main.tex}
            \input{3 - fluid component/main.tex}
            \input{4 - numerical implementation/main.tex}
        \part{Project Overview}
            \input{5 - research plan/main.tex}
            \input{6 - summary/main.tex}
    
    
    %\section{}
    \newpage
    \pagenumbering{gobble}
        \printbibliography


    \newpage
    \pagenumbering{roman}
    \appendix
        \part{Appendices}
            \input{8 - Hilbert complexes/main.tex}
            \input{9 - weak conservation proofs/main.tex}
\end{document}

    
    
    %\section{}
    \newpage
    \pagenumbering{gobble}
        \printbibliography


    \newpage
    \pagenumbering{roman}
    \appendix
        \part{Appendices}
            \documentclass[12pt, a4paper]{report}

\input{template/main.tex}

\title{\BA{Title in Progress...}}
\author{Boris Andrews}
\affil{Mathematical Institute, University of Oxford}
\date{\today}


\begin{document}
    \pagenumbering{gobble}
    \maketitle
    
    
    \begin{abstract}
        Magnetic confinement reactors---in particular tokamaks---offer one of the most promising options for achieving practical nuclear fusion, with the potential to provide virtually limitless, clean energy. The theoretical and numerical modeling of tokamak plasmas is simultaneously an essential component of effective reactor design, and a great research barrier. Tokamak operational conditions exhibit comparatively low Knudsen numbers. Kinetic effects, including kinetic waves and instabilities, Landau damping, bump-on-tail instabilities and more, are therefore highly influential in tokamak plasma dynamics. Purely fluid models are inherently incapable of capturing these effects, whereas the high dimensionality in purely kinetic models render them practically intractable for most relevant purposes.

        We consider a $\delta\!f$ decomposition model, with a macroscopic fluid background and microscopic kinetic correction, both fully coupled to each other. A similar manner of discretization is proposed to that used in the recent \texttt{STRUPHY} code \cite{Holderied_Possanner_Wang_2021, Holderied_2022, Li_et_al_2023} with a finite-element model for the background and a pseudo-particle/PiC model for the correction.

        The fluid background satisfies the full, non-linear, resistive, compressible, Hall MHD equations. \cite{Laakmann_Hu_Farrell_2022} introduces finite-element(-in-space) implicit timesteppers for the incompressible analogue to this system with structure-preserving (SP) properties in the ideal case, alongside parameter-robust preconditioners. We show that these timesteppers can derive from a finite-element-in-time (FET) (and finite-element-in-space) interpretation. The benefits of this reformulation are discussed, including the derivation of timesteppers that are higher order in time, and the quantifiable dissipative SP properties in the non-ideal, resistive case.
        
        We discuss possible options for extending this FET approach to timesteppers for the compressible case.

        The kinetic corrections satisfy linearized Boltzmann equations. Using a Lénard--Bernstein collision operator, these take Fokker--Planck-like forms \cite{Fokker_1914, Planck_1917} wherein pseudo-particles in the numerical model obey the neoclassical transport equations, with particle-independent Brownian drift terms. This offers a rigorous methodology for incorporating collisions into the particle transport model, without coupling the equations of motions for each particle.
        
        Works by Chen, Chacón et al. \cite{Chen_Chacón_Barnes_2011, Chacón_Chen_Barnes_2013, Chen_Chacón_2014, Chen_Chacón_2015} have developed structure-preserving particle pushers for neoclassical transport in the Vlasov equations, derived from Crank--Nicolson integrators. We show these too can can derive from a FET interpretation, similarly offering potential extensions to higher-order-in-time particle pushers. The FET formulation is used also to consider how the stochastic drift terms can be incorporated into the pushers. Stochastic gyrokinetic expansions are also discussed.

        Different options for the numerical implementation of these schemes are considered.

        Due to the efficacy of FET in the development of SP timesteppers for both the fluid and kinetic component, we hope this approach will prove effective in the future for developing SP timesteppers for the full hybrid model. We hope this will give us the opportunity to incorporate previously inaccessible kinetic effects into the highly effective, modern, finite-element MHD models.
    \end{abstract}
    
    
    \newpage
    \tableofcontents
    
    
    \newpage
    \pagenumbering{arabic}
    %\linenumbers\renewcommand\thelinenumber{\color{black!50}\arabic{linenumber}}
            \input{0 - introduction/main.tex}
        \part{Research}
            \input{1 - low-noise PiC models/main.tex}
            \input{2 - kinetic component/main.tex}
            \input{3 - fluid component/main.tex}
            \input{4 - numerical implementation/main.tex}
        \part{Project Overview}
            \input{5 - research plan/main.tex}
            \input{6 - summary/main.tex}
    
    
    %\section{}
    \newpage
    \pagenumbering{gobble}
        \printbibliography


    \newpage
    \pagenumbering{roman}
    \appendix
        \part{Appendices}
            \input{8 - Hilbert complexes/main.tex}
            \input{9 - weak conservation proofs/main.tex}
\end{document}

            \documentclass[12pt, a4paper]{report}

\input{template/main.tex}

\title{\BA{Title in Progress...}}
\author{Boris Andrews}
\affil{Mathematical Institute, University of Oxford}
\date{\today}


\begin{document}
    \pagenumbering{gobble}
    \maketitle
    
    
    \begin{abstract}
        Magnetic confinement reactors---in particular tokamaks---offer one of the most promising options for achieving practical nuclear fusion, with the potential to provide virtually limitless, clean energy. The theoretical and numerical modeling of tokamak plasmas is simultaneously an essential component of effective reactor design, and a great research barrier. Tokamak operational conditions exhibit comparatively low Knudsen numbers. Kinetic effects, including kinetic waves and instabilities, Landau damping, bump-on-tail instabilities and more, are therefore highly influential in tokamak plasma dynamics. Purely fluid models are inherently incapable of capturing these effects, whereas the high dimensionality in purely kinetic models render them practically intractable for most relevant purposes.

        We consider a $\delta\!f$ decomposition model, with a macroscopic fluid background and microscopic kinetic correction, both fully coupled to each other. A similar manner of discretization is proposed to that used in the recent \texttt{STRUPHY} code \cite{Holderied_Possanner_Wang_2021, Holderied_2022, Li_et_al_2023} with a finite-element model for the background and a pseudo-particle/PiC model for the correction.

        The fluid background satisfies the full, non-linear, resistive, compressible, Hall MHD equations. \cite{Laakmann_Hu_Farrell_2022} introduces finite-element(-in-space) implicit timesteppers for the incompressible analogue to this system with structure-preserving (SP) properties in the ideal case, alongside parameter-robust preconditioners. We show that these timesteppers can derive from a finite-element-in-time (FET) (and finite-element-in-space) interpretation. The benefits of this reformulation are discussed, including the derivation of timesteppers that are higher order in time, and the quantifiable dissipative SP properties in the non-ideal, resistive case.
        
        We discuss possible options for extending this FET approach to timesteppers for the compressible case.

        The kinetic corrections satisfy linearized Boltzmann equations. Using a Lénard--Bernstein collision operator, these take Fokker--Planck-like forms \cite{Fokker_1914, Planck_1917} wherein pseudo-particles in the numerical model obey the neoclassical transport equations, with particle-independent Brownian drift terms. This offers a rigorous methodology for incorporating collisions into the particle transport model, without coupling the equations of motions for each particle.
        
        Works by Chen, Chacón et al. \cite{Chen_Chacón_Barnes_2011, Chacón_Chen_Barnes_2013, Chen_Chacón_2014, Chen_Chacón_2015} have developed structure-preserving particle pushers for neoclassical transport in the Vlasov equations, derived from Crank--Nicolson integrators. We show these too can can derive from a FET interpretation, similarly offering potential extensions to higher-order-in-time particle pushers. The FET formulation is used also to consider how the stochastic drift terms can be incorporated into the pushers. Stochastic gyrokinetic expansions are also discussed.

        Different options for the numerical implementation of these schemes are considered.

        Due to the efficacy of FET in the development of SP timesteppers for both the fluid and kinetic component, we hope this approach will prove effective in the future for developing SP timesteppers for the full hybrid model. We hope this will give us the opportunity to incorporate previously inaccessible kinetic effects into the highly effective, modern, finite-element MHD models.
    \end{abstract}
    
    
    \newpage
    \tableofcontents
    
    
    \newpage
    \pagenumbering{arabic}
    %\linenumbers\renewcommand\thelinenumber{\color{black!50}\arabic{linenumber}}
            \input{0 - introduction/main.tex}
        \part{Research}
            \input{1 - low-noise PiC models/main.tex}
            \input{2 - kinetic component/main.tex}
            \input{3 - fluid component/main.tex}
            \input{4 - numerical implementation/main.tex}
        \part{Project Overview}
            \input{5 - research plan/main.tex}
            \input{6 - summary/main.tex}
    
    
    %\section{}
    \newpage
    \pagenumbering{gobble}
        \printbibliography


    \newpage
    \pagenumbering{roman}
    \appendix
        \part{Appendices}
            \input{8 - Hilbert complexes/main.tex}
            \input{9 - weak conservation proofs/main.tex}
\end{document}

\end{document}

            \documentclass[12pt, a4paper]{report}

\documentclass[12pt, a4paper]{report}

\input{template/main.tex}

\title{\BA{Title in Progress...}}
\author{Boris Andrews}
\affil{Mathematical Institute, University of Oxford}
\date{\today}


\begin{document}
    \pagenumbering{gobble}
    \maketitle
    
    
    \begin{abstract}
        Magnetic confinement reactors---in particular tokamaks---offer one of the most promising options for achieving practical nuclear fusion, with the potential to provide virtually limitless, clean energy. The theoretical and numerical modeling of tokamak plasmas is simultaneously an essential component of effective reactor design, and a great research barrier. Tokamak operational conditions exhibit comparatively low Knudsen numbers. Kinetic effects, including kinetic waves and instabilities, Landau damping, bump-on-tail instabilities and more, are therefore highly influential in tokamak plasma dynamics. Purely fluid models are inherently incapable of capturing these effects, whereas the high dimensionality in purely kinetic models render them practically intractable for most relevant purposes.

        We consider a $\delta\!f$ decomposition model, with a macroscopic fluid background and microscopic kinetic correction, both fully coupled to each other. A similar manner of discretization is proposed to that used in the recent \texttt{STRUPHY} code \cite{Holderied_Possanner_Wang_2021, Holderied_2022, Li_et_al_2023} with a finite-element model for the background and a pseudo-particle/PiC model for the correction.

        The fluid background satisfies the full, non-linear, resistive, compressible, Hall MHD equations. \cite{Laakmann_Hu_Farrell_2022} introduces finite-element(-in-space) implicit timesteppers for the incompressible analogue to this system with structure-preserving (SP) properties in the ideal case, alongside parameter-robust preconditioners. We show that these timesteppers can derive from a finite-element-in-time (FET) (and finite-element-in-space) interpretation. The benefits of this reformulation are discussed, including the derivation of timesteppers that are higher order in time, and the quantifiable dissipative SP properties in the non-ideal, resistive case.
        
        We discuss possible options for extending this FET approach to timesteppers for the compressible case.

        The kinetic corrections satisfy linearized Boltzmann equations. Using a Lénard--Bernstein collision operator, these take Fokker--Planck-like forms \cite{Fokker_1914, Planck_1917} wherein pseudo-particles in the numerical model obey the neoclassical transport equations, with particle-independent Brownian drift terms. This offers a rigorous methodology for incorporating collisions into the particle transport model, without coupling the equations of motions for each particle.
        
        Works by Chen, Chacón et al. \cite{Chen_Chacón_Barnes_2011, Chacón_Chen_Barnes_2013, Chen_Chacón_2014, Chen_Chacón_2015} have developed structure-preserving particle pushers for neoclassical transport in the Vlasov equations, derived from Crank--Nicolson integrators. We show these too can can derive from a FET interpretation, similarly offering potential extensions to higher-order-in-time particle pushers. The FET formulation is used also to consider how the stochastic drift terms can be incorporated into the pushers. Stochastic gyrokinetic expansions are also discussed.

        Different options for the numerical implementation of these schemes are considered.

        Due to the efficacy of FET in the development of SP timesteppers for both the fluid and kinetic component, we hope this approach will prove effective in the future for developing SP timesteppers for the full hybrid model. We hope this will give us the opportunity to incorporate previously inaccessible kinetic effects into the highly effective, modern, finite-element MHD models.
    \end{abstract}
    
    
    \newpage
    \tableofcontents
    
    
    \newpage
    \pagenumbering{arabic}
    %\linenumbers\renewcommand\thelinenumber{\color{black!50}\arabic{linenumber}}
            \input{0 - introduction/main.tex}
        \part{Research}
            \input{1 - low-noise PiC models/main.tex}
            \input{2 - kinetic component/main.tex}
            \input{3 - fluid component/main.tex}
            \input{4 - numerical implementation/main.tex}
        \part{Project Overview}
            \input{5 - research plan/main.tex}
            \input{6 - summary/main.tex}
    
    
    %\section{}
    \newpage
    \pagenumbering{gobble}
        \printbibliography


    \newpage
    \pagenumbering{roman}
    \appendix
        \part{Appendices}
            \input{8 - Hilbert complexes/main.tex}
            \input{9 - weak conservation proofs/main.tex}
\end{document}


\title{\BA{Title in Progress...}}
\author{Boris Andrews}
\affil{Mathematical Institute, University of Oxford}
\date{\today}


\begin{document}
    \pagenumbering{gobble}
    \maketitle
    
    
    \begin{abstract}
        Magnetic confinement reactors---in particular tokamaks---offer one of the most promising options for achieving practical nuclear fusion, with the potential to provide virtually limitless, clean energy. The theoretical and numerical modeling of tokamak plasmas is simultaneously an essential component of effective reactor design, and a great research barrier. Tokamak operational conditions exhibit comparatively low Knudsen numbers. Kinetic effects, including kinetic waves and instabilities, Landau damping, bump-on-tail instabilities and more, are therefore highly influential in tokamak plasma dynamics. Purely fluid models are inherently incapable of capturing these effects, whereas the high dimensionality in purely kinetic models render them practically intractable for most relevant purposes.

        We consider a $\delta\!f$ decomposition model, with a macroscopic fluid background and microscopic kinetic correction, both fully coupled to each other. A similar manner of discretization is proposed to that used in the recent \texttt{STRUPHY} code \cite{Holderied_Possanner_Wang_2021, Holderied_2022, Li_et_al_2023} with a finite-element model for the background and a pseudo-particle/PiC model for the correction.

        The fluid background satisfies the full, non-linear, resistive, compressible, Hall MHD equations. \cite{Laakmann_Hu_Farrell_2022} introduces finite-element(-in-space) implicit timesteppers for the incompressible analogue to this system with structure-preserving (SP) properties in the ideal case, alongside parameter-robust preconditioners. We show that these timesteppers can derive from a finite-element-in-time (FET) (and finite-element-in-space) interpretation. The benefits of this reformulation are discussed, including the derivation of timesteppers that are higher order in time, and the quantifiable dissipative SP properties in the non-ideal, resistive case.
        
        We discuss possible options for extending this FET approach to timesteppers for the compressible case.

        The kinetic corrections satisfy linearized Boltzmann equations. Using a Lénard--Bernstein collision operator, these take Fokker--Planck-like forms \cite{Fokker_1914, Planck_1917} wherein pseudo-particles in the numerical model obey the neoclassical transport equations, with particle-independent Brownian drift terms. This offers a rigorous methodology for incorporating collisions into the particle transport model, without coupling the equations of motions for each particle.
        
        Works by Chen, Chacón et al. \cite{Chen_Chacón_Barnes_2011, Chacón_Chen_Barnes_2013, Chen_Chacón_2014, Chen_Chacón_2015} have developed structure-preserving particle pushers for neoclassical transport in the Vlasov equations, derived from Crank--Nicolson integrators. We show these too can can derive from a FET interpretation, similarly offering potential extensions to higher-order-in-time particle pushers. The FET formulation is used also to consider how the stochastic drift terms can be incorporated into the pushers. Stochastic gyrokinetic expansions are also discussed.

        Different options for the numerical implementation of these schemes are considered.

        Due to the efficacy of FET in the development of SP timesteppers for both the fluid and kinetic component, we hope this approach will prove effective in the future for developing SP timesteppers for the full hybrid model. We hope this will give us the opportunity to incorporate previously inaccessible kinetic effects into the highly effective, modern, finite-element MHD models.
    \end{abstract}
    
    
    \newpage
    \tableofcontents
    
    
    \newpage
    \pagenumbering{arabic}
    %\linenumbers\renewcommand\thelinenumber{\color{black!50}\arabic{linenumber}}
            \documentclass[12pt, a4paper]{report}

\input{template/main.tex}

\title{\BA{Title in Progress...}}
\author{Boris Andrews}
\affil{Mathematical Institute, University of Oxford}
\date{\today}


\begin{document}
    \pagenumbering{gobble}
    \maketitle
    
    
    \begin{abstract}
        Magnetic confinement reactors---in particular tokamaks---offer one of the most promising options for achieving practical nuclear fusion, with the potential to provide virtually limitless, clean energy. The theoretical and numerical modeling of tokamak plasmas is simultaneously an essential component of effective reactor design, and a great research barrier. Tokamak operational conditions exhibit comparatively low Knudsen numbers. Kinetic effects, including kinetic waves and instabilities, Landau damping, bump-on-tail instabilities and more, are therefore highly influential in tokamak plasma dynamics. Purely fluid models are inherently incapable of capturing these effects, whereas the high dimensionality in purely kinetic models render them practically intractable for most relevant purposes.

        We consider a $\delta\!f$ decomposition model, with a macroscopic fluid background and microscopic kinetic correction, both fully coupled to each other. A similar manner of discretization is proposed to that used in the recent \texttt{STRUPHY} code \cite{Holderied_Possanner_Wang_2021, Holderied_2022, Li_et_al_2023} with a finite-element model for the background and a pseudo-particle/PiC model for the correction.

        The fluid background satisfies the full, non-linear, resistive, compressible, Hall MHD equations. \cite{Laakmann_Hu_Farrell_2022} introduces finite-element(-in-space) implicit timesteppers for the incompressible analogue to this system with structure-preserving (SP) properties in the ideal case, alongside parameter-robust preconditioners. We show that these timesteppers can derive from a finite-element-in-time (FET) (and finite-element-in-space) interpretation. The benefits of this reformulation are discussed, including the derivation of timesteppers that are higher order in time, and the quantifiable dissipative SP properties in the non-ideal, resistive case.
        
        We discuss possible options for extending this FET approach to timesteppers for the compressible case.

        The kinetic corrections satisfy linearized Boltzmann equations. Using a Lénard--Bernstein collision operator, these take Fokker--Planck-like forms \cite{Fokker_1914, Planck_1917} wherein pseudo-particles in the numerical model obey the neoclassical transport equations, with particle-independent Brownian drift terms. This offers a rigorous methodology for incorporating collisions into the particle transport model, without coupling the equations of motions for each particle.
        
        Works by Chen, Chacón et al. \cite{Chen_Chacón_Barnes_2011, Chacón_Chen_Barnes_2013, Chen_Chacón_2014, Chen_Chacón_2015} have developed structure-preserving particle pushers for neoclassical transport in the Vlasov equations, derived from Crank--Nicolson integrators. We show these too can can derive from a FET interpretation, similarly offering potential extensions to higher-order-in-time particle pushers. The FET formulation is used also to consider how the stochastic drift terms can be incorporated into the pushers. Stochastic gyrokinetic expansions are also discussed.

        Different options for the numerical implementation of these schemes are considered.

        Due to the efficacy of FET in the development of SP timesteppers for both the fluid and kinetic component, we hope this approach will prove effective in the future for developing SP timesteppers for the full hybrid model. We hope this will give us the opportunity to incorporate previously inaccessible kinetic effects into the highly effective, modern, finite-element MHD models.
    \end{abstract}
    
    
    \newpage
    \tableofcontents
    
    
    \newpage
    \pagenumbering{arabic}
    %\linenumbers\renewcommand\thelinenumber{\color{black!50}\arabic{linenumber}}
            \input{0 - introduction/main.tex}
        \part{Research}
            \input{1 - low-noise PiC models/main.tex}
            \input{2 - kinetic component/main.tex}
            \input{3 - fluid component/main.tex}
            \input{4 - numerical implementation/main.tex}
        \part{Project Overview}
            \input{5 - research plan/main.tex}
            \input{6 - summary/main.tex}
    
    
    %\section{}
    \newpage
    \pagenumbering{gobble}
        \printbibliography


    \newpage
    \pagenumbering{roman}
    \appendix
        \part{Appendices}
            \input{8 - Hilbert complexes/main.tex}
            \input{9 - weak conservation proofs/main.tex}
\end{document}

        \part{Research}
            \documentclass[12pt, a4paper]{report}

\input{template/main.tex}

\title{\BA{Title in Progress...}}
\author{Boris Andrews}
\affil{Mathematical Institute, University of Oxford}
\date{\today}


\begin{document}
    \pagenumbering{gobble}
    \maketitle
    
    
    \begin{abstract}
        Magnetic confinement reactors---in particular tokamaks---offer one of the most promising options for achieving practical nuclear fusion, with the potential to provide virtually limitless, clean energy. The theoretical and numerical modeling of tokamak plasmas is simultaneously an essential component of effective reactor design, and a great research barrier. Tokamak operational conditions exhibit comparatively low Knudsen numbers. Kinetic effects, including kinetic waves and instabilities, Landau damping, bump-on-tail instabilities and more, are therefore highly influential in tokamak plasma dynamics. Purely fluid models are inherently incapable of capturing these effects, whereas the high dimensionality in purely kinetic models render them practically intractable for most relevant purposes.

        We consider a $\delta\!f$ decomposition model, with a macroscopic fluid background and microscopic kinetic correction, both fully coupled to each other. A similar manner of discretization is proposed to that used in the recent \texttt{STRUPHY} code \cite{Holderied_Possanner_Wang_2021, Holderied_2022, Li_et_al_2023} with a finite-element model for the background and a pseudo-particle/PiC model for the correction.

        The fluid background satisfies the full, non-linear, resistive, compressible, Hall MHD equations. \cite{Laakmann_Hu_Farrell_2022} introduces finite-element(-in-space) implicit timesteppers for the incompressible analogue to this system with structure-preserving (SP) properties in the ideal case, alongside parameter-robust preconditioners. We show that these timesteppers can derive from a finite-element-in-time (FET) (and finite-element-in-space) interpretation. The benefits of this reformulation are discussed, including the derivation of timesteppers that are higher order in time, and the quantifiable dissipative SP properties in the non-ideal, resistive case.
        
        We discuss possible options for extending this FET approach to timesteppers for the compressible case.

        The kinetic corrections satisfy linearized Boltzmann equations. Using a Lénard--Bernstein collision operator, these take Fokker--Planck-like forms \cite{Fokker_1914, Planck_1917} wherein pseudo-particles in the numerical model obey the neoclassical transport equations, with particle-independent Brownian drift terms. This offers a rigorous methodology for incorporating collisions into the particle transport model, without coupling the equations of motions for each particle.
        
        Works by Chen, Chacón et al. \cite{Chen_Chacón_Barnes_2011, Chacón_Chen_Barnes_2013, Chen_Chacón_2014, Chen_Chacón_2015} have developed structure-preserving particle pushers for neoclassical transport in the Vlasov equations, derived from Crank--Nicolson integrators. We show these too can can derive from a FET interpretation, similarly offering potential extensions to higher-order-in-time particle pushers. The FET formulation is used also to consider how the stochastic drift terms can be incorporated into the pushers. Stochastic gyrokinetic expansions are also discussed.

        Different options for the numerical implementation of these schemes are considered.

        Due to the efficacy of FET in the development of SP timesteppers for both the fluid and kinetic component, we hope this approach will prove effective in the future for developing SP timesteppers for the full hybrid model. We hope this will give us the opportunity to incorporate previously inaccessible kinetic effects into the highly effective, modern, finite-element MHD models.
    \end{abstract}
    
    
    \newpage
    \tableofcontents
    
    
    \newpage
    \pagenumbering{arabic}
    %\linenumbers\renewcommand\thelinenumber{\color{black!50}\arabic{linenumber}}
            \input{0 - introduction/main.tex}
        \part{Research}
            \input{1 - low-noise PiC models/main.tex}
            \input{2 - kinetic component/main.tex}
            \input{3 - fluid component/main.tex}
            \input{4 - numerical implementation/main.tex}
        \part{Project Overview}
            \input{5 - research plan/main.tex}
            \input{6 - summary/main.tex}
    
    
    %\section{}
    \newpage
    \pagenumbering{gobble}
        \printbibliography


    \newpage
    \pagenumbering{roman}
    \appendix
        \part{Appendices}
            \input{8 - Hilbert complexes/main.tex}
            \input{9 - weak conservation proofs/main.tex}
\end{document}

            \documentclass[12pt, a4paper]{report}

\input{template/main.tex}

\title{\BA{Title in Progress...}}
\author{Boris Andrews}
\affil{Mathematical Institute, University of Oxford}
\date{\today}


\begin{document}
    \pagenumbering{gobble}
    \maketitle
    
    
    \begin{abstract}
        Magnetic confinement reactors---in particular tokamaks---offer one of the most promising options for achieving practical nuclear fusion, with the potential to provide virtually limitless, clean energy. The theoretical and numerical modeling of tokamak plasmas is simultaneously an essential component of effective reactor design, and a great research barrier. Tokamak operational conditions exhibit comparatively low Knudsen numbers. Kinetic effects, including kinetic waves and instabilities, Landau damping, bump-on-tail instabilities and more, are therefore highly influential in tokamak plasma dynamics. Purely fluid models are inherently incapable of capturing these effects, whereas the high dimensionality in purely kinetic models render them practically intractable for most relevant purposes.

        We consider a $\delta\!f$ decomposition model, with a macroscopic fluid background and microscopic kinetic correction, both fully coupled to each other. A similar manner of discretization is proposed to that used in the recent \texttt{STRUPHY} code \cite{Holderied_Possanner_Wang_2021, Holderied_2022, Li_et_al_2023} with a finite-element model for the background and a pseudo-particle/PiC model for the correction.

        The fluid background satisfies the full, non-linear, resistive, compressible, Hall MHD equations. \cite{Laakmann_Hu_Farrell_2022} introduces finite-element(-in-space) implicit timesteppers for the incompressible analogue to this system with structure-preserving (SP) properties in the ideal case, alongside parameter-robust preconditioners. We show that these timesteppers can derive from a finite-element-in-time (FET) (and finite-element-in-space) interpretation. The benefits of this reformulation are discussed, including the derivation of timesteppers that are higher order in time, and the quantifiable dissipative SP properties in the non-ideal, resistive case.
        
        We discuss possible options for extending this FET approach to timesteppers for the compressible case.

        The kinetic corrections satisfy linearized Boltzmann equations. Using a Lénard--Bernstein collision operator, these take Fokker--Planck-like forms \cite{Fokker_1914, Planck_1917} wherein pseudo-particles in the numerical model obey the neoclassical transport equations, with particle-independent Brownian drift terms. This offers a rigorous methodology for incorporating collisions into the particle transport model, without coupling the equations of motions for each particle.
        
        Works by Chen, Chacón et al. \cite{Chen_Chacón_Barnes_2011, Chacón_Chen_Barnes_2013, Chen_Chacón_2014, Chen_Chacón_2015} have developed structure-preserving particle pushers for neoclassical transport in the Vlasov equations, derived from Crank--Nicolson integrators. We show these too can can derive from a FET interpretation, similarly offering potential extensions to higher-order-in-time particle pushers. The FET formulation is used also to consider how the stochastic drift terms can be incorporated into the pushers. Stochastic gyrokinetic expansions are also discussed.

        Different options for the numerical implementation of these schemes are considered.

        Due to the efficacy of FET in the development of SP timesteppers for both the fluid and kinetic component, we hope this approach will prove effective in the future for developing SP timesteppers for the full hybrid model. We hope this will give us the opportunity to incorporate previously inaccessible kinetic effects into the highly effective, modern, finite-element MHD models.
    \end{abstract}
    
    
    \newpage
    \tableofcontents
    
    
    \newpage
    \pagenumbering{arabic}
    %\linenumbers\renewcommand\thelinenumber{\color{black!50}\arabic{linenumber}}
            \input{0 - introduction/main.tex}
        \part{Research}
            \input{1 - low-noise PiC models/main.tex}
            \input{2 - kinetic component/main.tex}
            \input{3 - fluid component/main.tex}
            \input{4 - numerical implementation/main.tex}
        \part{Project Overview}
            \input{5 - research plan/main.tex}
            \input{6 - summary/main.tex}
    
    
    %\section{}
    \newpage
    \pagenumbering{gobble}
        \printbibliography


    \newpage
    \pagenumbering{roman}
    \appendix
        \part{Appendices}
            \input{8 - Hilbert complexes/main.tex}
            \input{9 - weak conservation proofs/main.tex}
\end{document}

            \documentclass[12pt, a4paper]{report}

\input{template/main.tex}

\title{\BA{Title in Progress...}}
\author{Boris Andrews}
\affil{Mathematical Institute, University of Oxford}
\date{\today}


\begin{document}
    \pagenumbering{gobble}
    \maketitle
    
    
    \begin{abstract}
        Magnetic confinement reactors---in particular tokamaks---offer one of the most promising options for achieving practical nuclear fusion, with the potential to provide virtually limitless, clean energy. The theoretical and numerical modeling of tokamak plasmas is simultaneously an essential component of effective reactor design, and a great research barrier. Tokamak operational conditions exhibit comparatively low Knudsen numbers. Kinetic effects, including kinetic waves and instabilities, Landau damping, bump-on-tail instabilities and more, are therefore highly influential in tokamak plasma dynamics. Purely fluid models are inherently incapable of capturing these effects, whereas the high dimensionality in purely kinetic models render them practically intractable for most relevant purposes.

        We consider a $\delta\!f$ decomposition model, with a macroscopic fluid background and microscopic kinetic correction, both fully coupled to each other. A similar manner of discretization is proposed to that used in the recent \texttt{STRUPHY} code \cite{Holderied_Possanner_Wang_2021, Holderied_2022, Li_et_al_2023} with a finite-element model for the background and a pseudo-particle/PiC model for the correction.

        The fluid background satisfies the full, non-linear, resistive, compressible, Hall MHD equations. \cite{Laakmann_Hu_Farrell_2022} introduces finite-element(-in-space) implicit timesteppers for the incompressible analogue to this system with structure-preserving (SP) properties in the ideal case, alongside parameter-robust preconditioners. We show that these timesteppers can derive from a finite-element-in-time (FET) (and finite-element-in-space) interpretation. The benefits of this reformulation are discussed, including the derivation of timesteppers that are higher order in time, and the quantifiable dissipative SP properties in the non-ideal, resistive case.
        
        We discuss possible options for extending this FET approach to timesteppers for the compressible case.

        The kinetic corrections satisfy linearized Boltzmann equations. Using a Lénard--Bernstein collision operator, these take Fokker--Planck-like forms \cite{Fokker_1914, Planck_1917} wherein pseudo-particles in the numerical model obey the neoclassical transport equations, with particle-independent Brownian drift terms. This offers a rigorous methodology for incorporating collisions into the particle transport model, without coupling the equations of motions for each particle.
        
        Works by Chen, Chacón et al. \cite{Chen_Chacón_Barnes_2011, Chacón_Chen_Barnes_2013, Chen_Chacón_2014, Chen_Chacón_2015} have developed structure-preserving particle pushers for neoclassical transport in the Vlasov equations, derived from Crank--Nicolson integrators. We show these too can can derive from a FET interpretation, similarly offering potential extensions to higher-order-in-time particle pushers. The FET formulation is used also to consider how the stochastic drift terms can be incorporated into the pushers. Stochastic gyrokinetic expansions are also discussed.

        Different options for the numerical implementation of these schemes are considered.

        Due to the efficacy of FET in the development of SP timesteppers for both the fluid and kinetic component, we hope this approach will prove effective in the future for developing SP timesteppers for the full hybrid model. We hope this will give us the opportunity to incorporate previously inaccessible kinetic effects into the highly effective, modern, finite-element MHD models.
    \end{abstract}
    
    
    \newpage
    \tableofcontents
    
    
    \newpage
    \pagenumbering{arabic}
    %\linenumbers\renewcommand\thelinenumber{\color{black!50}\arabic{linenumber}}
            \input{0 - introduction/main.tex}
        \part{Research}
            \input{1 - low-noise PiC models/main.tex}
            \input{2 - kinetic component/main.tex}
            \input{3 - fluid component/main.tex}
            \input{4 - numerical implementation/main.tex}
        \part{Project Overview}
            \input{5 - research plan/main.tex}
            \input{6 - summary/main.tex}
    
    
    %\section{}
    \newpage
    \pagenumbering{gobble}
        \printbibliography


    \newpage
    \pagenumbering{roman}
    \appendix
        \part{Appendices}
            \input{8 - Hilbert complexes/main.tex}
            \input{9 - weak conservation proofs/main.tex}
\end{document}

            \documentclass[12pt, a4paper]{report}

\input{template/main.tex}

\title{\BA{Title in Progress...}}
\author{Boris Andrews}
\affil{Mathematical Institute, University of Oxford}
\date{\today}


\begin{document}
    \pagenumbering{gobble}
    \maketitle
    
    
    \begin{abstract}
        Magnetic confinement reactors---in particular tokamaks---offer one of the most promising options for achieving practical nuclear fusion, with the potential to provide virtually limitless, clean energy. The theoretical and numerical modeling of tokamak plasmas is simultaneously an essential component of effective reactor design, and a great research barrier. Tokamak operational conditions exhibit comparatively low Knudsen numbers. Kinetic effects, including kinetic waves and instabilities, Landau damping, bump-on-tail instabilities and more, are therefore highly influential in tokamak plasma dynamics. Purely fluid models are inherently incapable of capturing these effects, whereas the high dimensionality in purely kinetic models render them practically intractable for most relevant purposes.

        We consider a $\delta\!f$ decomposition model, with a macroscopic fluid background and microscopic kinetic correction, both fully coupled to each other. A similar manner of discretization is proposed to that used in the recent \texttt{STRUPHY} code \cite{Holderied_Possanner_Wang_2021, Holderied_2022, Li_et_al_2023} with a finite-element model for the background and a pseudo-particle/PiC model for the correction.

        The fluid background satisfies the full, non-linear, resistive, compressible, Hall MHD equations. \cite{Laakmann_Hu_Farrell_2022} introduces finite-element(-in-space) implicit timesteppers for the incompressible analogue to this system with structure-preserving (SP) properties in the ideal case, alongside parameter-robust preconditioners. We show that these timesteppers can derive from a finite-element-in-time (FET) (and finite-element-in-space) interpretation. The benefits of this reformulation are discussed, including the derivation of timesteppers that are higher order in time, and the quantifiable dissipative SP properties in the non-ideal, resistive case.
        
        We discuss possible options for extending this FET approach to timesteppers for the compressible case.

        The kinetic corrections satisfy linearized Boltzmann equations. Using a Lénard--Bernstein collision operator, these take Fokker--Planck-like forms \cite{Fokker_1914, Planck_1917} wherein pseudo-particles in the numerical model obey the neoclassical transport equations, with particle-independent Brownian drift terms. This offers a rigorous methodology for incorporating collisions into the particle transport model, without coupling the equations of motions for each particle.
        
        Works by Chen, Chacón et al. \cite{Chen_Chacón_Barnes_2011, Chacón_Chen_Barnes_2013, Chen_Chacón_2014, Chen_Chacón_2015} have developed structure-preserving particle pushers for neoclassical transport in the Vlasov equations, derived from Crank--Nicolson integrators. We show these too can can derive from a FET interpretation, similarly offering potential extensions to higher-order-in-time particle pushers. The FET formulation is used also to consider how the stochastic drift terms can be incorporated into the pushers. Stochastic gyrokinetic expansions are also discussed.

        Different options for the numerical implementation of these schemes are considered.

        Due to the efficacy of FET in the development of SP timesteppers for both the fluid and kinetic component, we hope this approach will prove effective in the future for developing SP timesteppers for the full hybrid model. We hope this will give us the opportunity to incorporate previously inaccessible kinetic effects into the highly effective, modern, finite-element MHD models.
    \end{abstract}
    
    
    \newpage
    \tableofcontents
    
    
    \newpage
    \pagenumbering{arabic}
    %\linenumbers\renewcommand\thelinenumber{\color{black!50}\arabic{linenumber}}
            \input{0 - introduction/main.tex}
        \part{Research}
            \input{1 - low-noise PiC models/main.tex}
            \input{2 - kinetic component/main.tex}
            \input{3 - fluid component/main.tex}
            \input{4 - numerical implementation/main.tex}
        \part{Project Overview}
            \input{5 - research plan/main.tex}
            \input{6 - summary/main.tex}
    
    
    %\section{}
    \newpage
    \pagenumbering{gobble}
        \printbibliography


    \newpage
    \pagenumbering{roman}
    \appendix
        \part{Appendices}
            \input{8 - Hilbert complexes/main.tex}
            \input{9 - weak conservation proofs/main.tex}
\end{document}

        \part{Project Overview}
            \documentclass[12pt, a4paper]{report}

\input{template/main.tex}

\title{\BA{Title in Progress...}}
\author{Boris Andrews}
\affil{Mathematical Institute, University of Oxford}
\date{\today}


\begin{document}
    \pagenumbering{gobble}
    \maketitle
    
    
    \begin{abstract}
        Magnetic confinement reactors---in particular tokamaks---offer one of the most promising options for achieving practical nuclear fusion, with the potential to provide virtually limitless, clean energy. The theoretical and numerical modeling of tokamak plasmas is simultaneously an essential component of effective reactor design, and a great research barrier. Tokamak operational conditions exhibit comparatively low Knudsen numbers. Kinetic effects, including kinetic waves and instabilities, Landau damping, bump-on-tail instabilities and more, are therefore highly influential in tokamak plasma dynamics. Purely fluid models are inherently incapable of capturing these effects, whereas the high dimensionality in purely kinetic models render them practically intractable for most relevant purposes.

        We consider a $\delta\!f$ decomposition model, with a macroscopic fluid background and microscopic kinetic correction, both fully coupled to each other. A similar manner of discretization is proposed to that used in the recent \texttt{STRUPHY} code \cite{Holderied_Possanner_Wang_2021, Holderied_2022, Li_et_al_2023} with a finite-element model for the background and a pseudo-particle/PiC model for the correction.

        The fluid background satisfies the full, non-linear, resistive, compressible, Hall MHD equations. \cite{Laakmann_Hu_Farrell_2022} introduces finite-element(-in-space) implicit timesteppers for the incompressible analogue to this system with structure-preserving (SP) properties in the ideal case, alongside parameter-robust preconditioners. We show that these timesteppers can derive from a finite-element-in-time (FET) (and finite-element-in-space) interpretation. The benefits of this reformulation are discussed, including the derivation of timesteppers that are higher order in time, and the quantifiable dissipative SP properties in the non-ideal, resistive case.
        
        We discuss possible options for extending this FET approach to timesteppers for the compressible case.

        The kinetic corrections satisfy linearized Boltzmann equations. Using a Lénard--Bernstein collision operator, these take Fokker--Planck-like forms \cite{Fokker_1914, Planck_1917} wherein pseudo-particles in the numerical model obey the neoclassical transport equations, with particle-independent Brownian drift terms. This offers a rigorous methodology for incorporating collisions into the particle transport model, without coupling the equations of motions for each particle.
        
        Works by Chen, Chacón et al. \cite{Chen_Chacón_Barnes_2011, Chacón_Chen_Barnes_2013, Chen_Chacón_2014, Chen_Chacón_2015} have developed structure-preserving particle pushers for neoclassical transport in the Vlasov equations, derived from Crank--Nicolson integrators. We show these too can can derive from a FET interpretation, similarly offering potential extensions to higher-order-in-time particle pushers. The FET formulation is used also to consider how the stochastic drift terms can be incorporated into the pushers. Stochastic gyrokinetic expansions are also discussed.

        Different options for the numerical implementation of these schemes are considered.

        Due to the efficacy of FET in the development of SP timesteppers for both the fluid and kinetic component, we hope this approach will prove effective in the future for developing SP timesteppers for the full hybrid model. We hope this will give us the opportunity to incorporate previously inaccessible kinetic effects into the highly effective, modern, finite-element MHD models.
    \end{abstract}
    
    
    \newpage
    \tableofcontents
    
    
    \newpage
    \pagenumbering{arabic}
    %\linenumbers\renewcommand\thelinenumber{\color{black!50}\arabic{linenumber}}
            \input{0 - introduction/main.tex}
        \part{Research}
            \input{1 - low-noise PiC models/main.tex}
            \input{2 - kinetic component/main.tex}
            \input{3 - fluid component/main.tex}
            \input{4 - numerical implementation/main.tex}
        \part{Project Overview}
            \input{5 - research plan/main.tex}
            \input{6 - summary/main.tex}
    
    
    %\section{}
    \newpage
    \pagenumbering{gobble}
        \printbibliography


    \newpage
    \pagenumbering{roman}
    \appendix
        \part{Appendices}
            \input{8 - Hilbert complexes/main.tex}
            \input{9 - weak conservation proofs/main.tex}
\end{document}

            \documentclass[12pt, a4paper]{report}

\input{template/main.tex}

\title{\BA{Title in Progress...}}
\author{Boris Andrews}
\affil{Mathematical Institute, University of Oxford}
\date{\today}


\begin{document}
    \pagenumbering{gobble}
    \maketitle
    
    
    \begin{abstract}
        Magnetic confinement reactors---in particular tokamaks---offer one of the most promising options for achieving practical nuclear fusion, with the potential to provide virtually limitless, clean energy. The theoretical and numerical modeling of tokamak plasmas is simultaneously an essential component of effective reactor design, and a great research barrier. Tokamak operational conditions exhibit comparatively low Knudsen numbers. Kinetic effects, including kinetic waves and instabilities, Landau damping, bump-on-tail instabilities and more, are therefore highly influential in tokamak plasma dynamics. Purely fluid models are inherently incapable of capturing these effects, whereas the high dimensionality in purely kinetic models render them practically intractable for most relevant purposes.

        We consider a $\delta\!f$ decomposition model, with a macroscopic fluid background and microscopic kinetic correction, both fully coupled to each other. A similar manner of discretization is proposed to that used in the recent \texttt{STRUPHY} code \cite{Holderied_Possanner_Wang_2021, Holderied_2022, Li_et_al_2023} with a finite-element model for the background and a pseudo-particle/PiC model for the correction.

        The fluid background satisfies the full, non-linear, resistive, compressible, Hall MHD equations. \cite{Laakmann_Hu_Farrell_2022} introduces finite-element(-in-space) implicit timesteppers for the incompressible analogue to this system with structure-preserving (SP) properties in the ideal case, alongside parameter-robust preconditioners. We show that these timesteppers can derive from a finite-element-in-time (FET) (and finite-element-in-space) interpretation. The benefits of this reformulation are discussed, including the derivation of timesteppers that are higher order in time, and the quantifiable dissipative SP properties in the non-ideal, resistive case.
        
        We discuss possible options for extending this FET approach to timesteppers for the compressible case.

        The kinetic corrections satisfy linearized Boltzmann equations. Using a Lénard--Bernstein collision operator, these take Fokker--Planck-like forms \cite{Fokker_1914, Planck_1917} wherein pseudo-particles in the numerical model obey the neoclassical transport equations, with particle-independent Brownian drift terms. This offers a rigorous methodology for incorporating collisions into the particle transport model, without coupling the equations of motions for each particle.
        
        Works by Chen, Chacón et al. \cite{Chen_Chacón_Barnes_2011, Chacón_Chen_Barnes_2013, Chen_Chacón_2014, Chen_Chacón_2015} have developed structure-preserving particle pushers for neoclassical transport in the Vlasov equations, derived from Crank--Nicolson integrators. We show these too can can derive from a FET interpretation, similarly offering potential extensions to higher-order-in-time particle pushers. The FET formulation is used also to consider how the stochastic drift terms can be incorporated into the pushers. Stochastic gyrokinetic expansions are also discussed.

        Different options for the numerical implementation of these schemes are considered.

        Due to the efficacy of FET in the development of SP timesteppers for both the fluid and kinetic component, we hope this approach will prove effective in the future for developing SP timesteppers for the full hybrid model. We hope this will give us the opportunity to incorporate previously inaccessible kinetic effects into the highly effective, modern, finite-element MHD models.
    \end{abstract}
    
    
    \newpage
    \tableofcontents
    
    
    \newpage
    \pagenumbering{arabic}
    %\linenumbers\renewcommand\thelinenumber{\color{black!50}\arabic{linenumber}}
            \input{0 - introduction/main.tex}
        \part{Research}
            \input{1 - low-noise PiC models/main.tex}
            \input{2 - kinetic component/main.tex}
            \input{3 - fluid component/main.tex}
            \input{4 - numerical implementation/main.tex}
        \part{Project Overview}
            \input{5 - research plan/main.tex}
            \input{6 - summary/main.tex}
    
    
    %\section{}
    \newpage
    \pagenumbering{gobble}
        \printbibliography


    \newpage
    \pagenumbering{roman}
    \appendix
        \part{Appendices}
            \input{8 - Hilbert complexes/main.tex}
            \input{9 - weak conservation proofs/main.tex}
\end{document}

    
    
    %\section{}
    \newpage
    \pagenumbering{gobble}
        \printbibliography


    \newpage
    \pagenumbering{roman}
    \appendix
        \part{Appendices}
            \documentclass[12pt, a4paper]{report}

\input{template/main.tex}

\title{\BA{Title in Progress...}}
\author{Boris Andrews}
\affil{Mathematical Institute, University of Oxford}
\date{\today}


\begin{document}
    \pagenumbering{gobble}
    \maketitle
    
    
    \begin{abstract}
        Magnetic confinement reactors---in particular tokamaks---offer one of the most promising options for achieving practical nuclear fusion, with the potential to provide virtually limitless, clean energy. The theoretical and numerical modeling of tokamak plasmas is simultaneously an essential component of effective reactor design, and a great research barrier. Tokamak operational conditions exhibit comparatively low Knudsen numbers. Kinetic effects, including kinetic waves and instabilities, Landau damping, bump-on-tail instabilities and more, are therefore highly influential in tokamak plasma dynamics. Purely fluid models are inherently incapable of capturing these effects, whereas the high dimensionality in purely kinetic models render them practically intractable for most relevant purposes.

        We consider a $\delta\!f$ decomposition model, with a macroscopic fluid background and microscopic kinetic correction, both fully coupled to each other. A similar manner of discretization is proposed to that used in the recent \texttt{STRUPHY} code \cite{Holderied_Possanner_Wang_2021, Holderied_2022, Li_et_al_2023} with a finite-element model for the background and a pseudo-particle/PiC model for the correction.

        The fluid background satisfies the full, non-linear, resistive, compressible, Hall MHD equations. \cite{Laakmann_Hu_Farrell_2022} introduces finite-element(-in-space) implicit timesteppers for the incompressible analogue to this system with structure-preserving (SP) properties in the ideal case, alongside parameter-robust preconditioners. We show that these timesteppers can derive from a finite-element-in-time (FET) (and finite-element-in-space) interpretation. The benefits of this reformulation are discussed, including the derivation of timesteppers that are higher order in time, and the quantifiable dissipative SP properties in the non-ideal, resistive case.
        
        We discuss possible options for extending this FET approach to timesteppers for the compressible case.

        The kinetic corrections satisfy linearized Boltzmann equations. Using a Lénard--Bernstein collision operator, these take Fokker--Planck-like forms \cite{Fokker_1914, Planck_1917} wherein pseudo-particles in the numerical model obey the neoclassical transport equations, with particle-independent Brownian drift terms. This offers a rigorous methodology for incorporating collisions into the particle transport model, without coupling the equations of motions for each particle.
        
        Works by Chen, Chacón et al. \cite{Chen_Chacón_Barnes_2011, Chacón_Chen_Barnes_2013, Chen_Chacón_2014, Chen_Chacón_2015} have developed structure-preserving particle pushers for neoclassical transport in the Vlasov equations, derived from Crank--Nicolson integrators. We show these too can can derive from a FET interpretation, similarly offering potential extensions to higher-order-in-time particle pushers. The FET formulation is used also to consider how the stochastic drift terms can be incorporated into the pushers. Stochastic gyrokinetic expansions are also discussed.

        Different options for the numerical implementation of these schemes are considered.

        Due to the efficacy of FET in the development of SP timesteppers for both the fluid and kinetic component, we hope this approach will prove effective in the future for developing SP timesteppers for the full hybrid model. We hope this will give us the opportunity to incorporate previously inaccessible kinetic effects into the highly effective, modern, finite-element MHD models.
    \end{abstract}
    
    
    \newpage
    \tableofcontents
    
    
    \newpage
    \pagenumbering{arabic}
    %\linenumbers\renewcommand\thelinenumber{\color{black!50}\arabic{linenumber}}
            \input{0 - introduction/main.tex}
        \part{Research}
            \input{1 - low-noise PiC models/main.tex}
            \input{2 - kinetic component/main.tex}
            \input{3 - fluid component/main.tex}
            \input{4 - numerical implementation/main.tex}
        \part{Project Overview}
            \input{5 - research plan/main.tex}
            \input{6 - summary/main.tex}
    
    
    %\section{}
    \newpage
    \pagenumbering{gobble}
        \printbibliography


    \newpage
    \pagenumbering{roman}
    \appendix
        \part{Appendices}
            \input{8 - Hilbert complexes/main.tex}
            \input{9 - weak conservation proofs/main.tex}
\end{document}

            \documentclass[12pt, a4paper]{report}

\input{template/main.tex}

\title{\BA{Title in Progress...}}
\author{Boris Andrews}
\affil{Mathematical Institute, University of Oxford}
\date{\today}


\begin{document}
    \pagenumbering{gobble}
    \maketitle
    
    
    \begin{abstract}
        Magnetic confinement reactors---in particular tokamaks---offer one of the most promising options for achieving practical nuclear fusion, with the potential to provide virtually limitless, clean energy. The theoretical and numerical modeling of tokamak plasmas is simultaneously an essential component of effective reactor design, and a great research barrier. Tokamak operational conditions exhibit comparatively low Knudsen numbers. Kinetic effects, including kinetic waves and instabilities, Landau damping, bump-on-tail instabilities and more, are therefore highly influential in tokamak plasma dynamics. Purely fluid models are inherently incapable of capturing these effects, whereas the high dimensionality in purely kinetic models render them practically intractable for most relevant purposes.

        We consider a $\delta\!f$ decomposition model, with a macroscopic fluid background and microscopic kinetic correction, both fully coupled to each other. A similar manner of discretization is proposed to that used in the recent \texttt{STRUPHY} code \cite{Holderied_Possanner_Wang_2021, Holderied_2022, Li_et_al_2023} with a finite-element model for the background and a pseudo-particle/PiC model for the correction.

        The fluid background satisfies the full, non-linear, resistive, compressible, Hall MHD equations. \cite{Laakmann_Hu_Farrell_2022} introduces finite-element(-in-space) implicit timesteppers for the incompressible analogue to this system with structure-preserving (SP) properties in the ideal case, alongside parameter-robust preconditioners. We show that these timesteppers can derive from a finite-element-in-time (FET) (and finite-element-in-space) interpretation. The benefits of this reformulation are discussed, including the derivation of timesteppers that are higher order in time, and the quantifiable dissipative SP properties in the non-ideal, resistive case.
        
        We discuss possible options for extending this FET approach to timesteppers for the compressible case.

        The kinetic corrections satisfy linearized Boltzmann equations. Using a Lénard--Bernstein collision operator, these take Fokker--Planck-like forms \cite{Fokker_1914, Planck_1917} wherein pseudo-particles in the numerical model obey the neoclassical transport equations, with particle-independent Brownian drift terms. This offers a rigorous methodology for incorporating collisions into the particle transport model, without coupling the equations of motions for each particle.
        
        Works by Chen, Chacón et al. \cite{Chen_Chacón_Barnes_2011, Chacón_Chen_Barnes_2013, Chen_Chacón_2014, Chen_Chacón_2015} have developed structure-preserving particle pushers for neoclassical transport in the Vlasov equations, derived from Crank--Nicolson integrators. We show these too can can derive from a FET interpretation, similarly offering potential extensions to higher-order-in-time particle pushers. The FET formulation is used also to consider how the stochastic drift terms can be incorporated into the pushers. Stochastic gyrokinetic expansions are also discussed.

        Different options for the numerical implementation of these schemes are considered.

        Due to the efficacy of FET in the development of SP timesteppers for both the fluid and kinetic component, we hope this approach will prove effective in the future for developing SP timesteppers for the full hybrid model. We hope this will give us the opportunity to incorporate previously inaccessible kinetic effects into the highly effective, modern, finite-element MHD models.
    \end{abstract}
    
    
    \newpage
    \tableofcontents
    
    
    \newpage
    \pagenumbering{arabic}
    %\linenumbers\renewcommand\thelinenumber{\color{black!50}\arabic{linenumber}}
            \input{0 - introduction/main.tex}
        \part{Research}
            \input{1 - low-noise PiC models/main.tex}
            \input{2 - kinetic component/main.tex}
            \input{3 - fluid component/main.tex}
            \input{4 - numerical implementation/main.tex}
        \part{Project Overview}
            \input{5 - research plan/main.tex}
            \input{6 - summary/main.tex}
    
    
    %\section{}
    \newpage
    \pagenumbering{gobble}
        \printbibliography


    \newpage
    \pagenumbering{roman}
    \appendix
        \part{Appendices}
            \input{8 - Hilbert complexes/main.tex}
            \input{9 - weak conservation proofs/main.tex}
\end{document}

\end{document}

            \documentclass[12pt, a4paper]{report}

\documentclass[12pt, a4paper]{report}

\input{template/main.tex}

\title{\BA{Title in Progress...}}
\author{Boris Andrews}
\affil{Mathematical Institute, University of Oxford}
\date{\today}


\begin{document}
    \pagenumbering{gobble}
    \maketitle
    
    
    \begin{abstract}
        Magnetic confinement reactors---in particular tokamaks---offer one of the most promising options for achieving practical nuclear fusion, with the potential to provide virtually limitless, clean energy. The theoretical and numerical modeling of tokamak plasmas is simultaneously an essential component of effective reactor design, and a great research barrier. Tokamak operational conditions exhibit comparatively low Knudsen numbers. Kinetic effects, including kinetic waves and instabilities, Landau damping, bump-on-tail instabilities and more, are therefore highly influential in tokamak plasma dynamics. Purely fluid models are inherently incapable of capturing these effects, whereas the high dimensionality in purely kinetic models render them practically intractable for most relevant purposes.

        We consider a $\delta\!f$ decomposition model, with a macroscopic fluid background and microscopic kinetic correction, both fully coupled to each other. A similar manner of discretization is proposed to that used in the recent \texttt{STRUPHY} code \cite{Holderied_Possanner_Wang_2021, Holderied_2022, Li_et_al_2023} with a finite-element model for the background and a pseudo-particle/PiC model for the correction.

        The fluid background satisfies the full, non-linear, resistive, compressible, Hall MHD equations. \cite{Laakmann_Hu_Farrell_2022} introduces finite-element(-in-space) implicit timesteppers for the incompressible analogue to this system with structure-preserving (SP) properties in the ideal case, alongside parameter-robust preconditioners. We show that these timesteppers can derive from a finite-element-in-time (FET) (and finite-element-in-space) interpretation. The benefits of this reformulation are discussed, including the derivation of timesteppers that are higher order in time, and the quantifiable dissipative SP properties in the non-ideal, resistive case.
        
        We discuss possible options for extending this FET approach to timesteppers for the compressible case.

        The kinetic corrections satisfy linearized Boltzmann equations. Using a Lénard--Bernstein collision operator, these take Fokker--Planck-like forms \cite{Fokker_1914, Planck_1917} wherein pseudo-particles in the numerical model obey the neoclassical transport equations, with particle-independent Brownian drift terms. This offers a rigorous methodology for incorporating collisions into the particle transport model, without coupling the equations of motions for each particle.
        
        Works by Chen, Chacón et al. \cite{Chen_Chacón_Barnes_2011, Chacón_Chen_Barnes_2013, Chen_Chacón_2014, Chen_Chacón_2015} have developed structure-preserving particle pushers for neoclassical transport in the Vlasov equations, derived from Crank--Nicolson integrators. We show these too can can derive from a FET interpretation, similarly offering potential extensions to higher-order-in-time particle pushers. The FET formulation is used also to consider how the stochastic drift terms can be incorporated into the pushers. Stochastic gyrokinetic expansions are also discussed.

        Different options for the numerical implementation of these schemes are considered.

        Due to the efficacy of FET in the development of SP timesteppers for both the fluid and kinetic component, we hope this approach will prove effective in the future for developing SP timesteppers for the full hybrid model. We hope this will give us the opportunity to incorporate previously inaccessible kinetic effects into the highly effective, modern, finite-element MHD models.
    \end{abstract}
    
    
    \newpage
    \tableofcontents
    
    
    \newpage
    \pagenumbering{arabic}
    %\linenumbers\renewcommand\thelinenumber{\color{black!50}\arabic{linenumber}}
            \input{0 - introduction/main.tex}
        \part{Research}
            \input{1 - low-noise PiC models/main.tex}
            \input{2 - kinetic component/main.tex}
            \input{3 - fluid component/main.tex}
            \input{4 - numerical implementation/main.tex}
        \part{Project Overview}
            \input{5 - research plan/main.tex}
            \input{6 - summary/main.tex}
    
    
    %\section{}
    \newpage
    \pagenumbering{gobble}
        \printbibliography


    \newpage
    \pagenumbering{roman}
    \appendix
        \part{Appendices}
            \input{8 - Hilbert complexes/main.tex}
            \input{9 - weak conservation proofs/main.tex}
\end{document}


\title{\BA{Title in Progress...}}
\author{Boris Andrews}
\affil{Mathematical Institute, University of Oxford}
\date{\today}


\begin{document}
    \pagenumbering{gobble}
    \maketitle
    
    
    \begin{abstract}
        Magnetic confinement reactors---in particular tokamaks---offer one of the most promising options for achieving practical nuclear fusion, with the potential to provide virtually limitless, clean energy. The theoretical and numerical modeling of tokamak plasmas is simultaneously an essential component of effective reactor design, and a great research barrier. Tokamak operational conditions exhibit comparatively low Knudsen numbers. Kinetic effects, including kinetic waves and instabilities, Landau damping, bump-on-tail instabilities and more, are therefore highly influential in tokamak plasma dynamics. Purely fluid models are inherently incapable of capturing these effects, whereas the high dimensionality in purely kinetic models render them practically intractable for most relevant purposes.

        We consider a $\delta\!f$ decomposition model, with a macroscopic fluid background and microscopic kinetic correction, both fully coupled to each other. A similar manner of discretization is proposed to that used in the recent \texttt{STRUPHY} code \cite{Holderied_Possanner_Wang_2021, Holderied_2022, Li_et_al_2023} with a finite-element model for the background and a pseudo-particle/PiC model for the correction.

        The fluid background satisfies the full, non-linear, resistive, compressible, Hall MHD equations. \cite{Laakmann_Hu_Farrell_2022} introduces finite-element(-in-space) implicit timesteppers for the incompressible analogue to this system with structure-preserving (SP) properties in the ideal case, alongside parameter-robust preconditioners. We show that these timesteppers can derive from a finite-element-in-time (FET) (and finite-element-in-space) interpretation. The benefits of this reformulation are discussed, including the derivation of timesteppers that are higher order in time, and the quantifiable dissipative SP properties in the non-ideal, resistive case.
        
        We discuss possible options for extending this FET approach to timesteppers for the compressible case.

        The kinetic corrections satisfy linearized Boltzmann equations. Using a Lénard--Bernstein collision operator, these take Fokker--Planck-like forms \cite{Fokker_1914, Planck_1917} wherein pseudo-particles in the numerical model obey the neoclassical transport equations, with particle-independent Brownian drift terms. This offers a rigorous methodology for incorporating collisions into the particle transport model, without coupling the equations of motions for each particle.
        
        Works by Chen, Chacón et al. \cite{Chen_Chacón_Barnes_2011, Chacón_Chen_Barnes_2013, Chen_Chacón_2014, Chen_Chacón_2015} have developed structure-preserving particle pushers for neoclassical transport in the Vlasov equations, derived from Crank--Nicolson integrators. We show these too can can derive from a FET interpretation, similarly offering potential extensions to higher-order-in-time particle pushers. The FET formulation is used also to consider how the stochastic drift terms can be incorporated into the pushers. Stochastic gyrokinetic expansions are also discussed.

        Different options for the numerical implementation of these schemes are considered.

        Due to the efficacy of FET in the development of SP timesteppers for both the fluid and kinetic component, we hope this approach will prove effective in the future for developing SP timesteppers for the full hybrid model. We hope this will give us the opportunity to incorporate previously inaccessible kinetic effects into the highly effective, modern, finite-element MHD models.
    \end{abstract}
    
    
    \newpage
    \tableofcontents
    
    
    \newpage
    \pagenumbering{arabic}
    %\linenumbers\renewcommand\thelinenumber{\color{black!50}\arabic{linenumber}}
            \documentclass[12pt, a4paper]{report}

\input{template/main.tex}

\title{\BA{Title in Progress...}}
\author{Boris Andrews}
\affil{Mathematical Institute, University of Oxford}
\date{\today}


\begin{document}
    \pagenumbering{gobble}
    \maketitle
    
    
    \begin{abstract}
        Magnetic confinement reactors---in particular tokamaks---offer one of the most promising options for achieving practical nuclear fusion, with the potential to provide virtually limitless, clean energy. The theoretical and numerical modeling of tokamak plasmas is simultaneously an essential component of effective reactor design, and a great research barrier. Tokamak operational conditions exhibit comparatively low Knudsen numbers. Kinetic effects, including kinetic waves and instabilities, Landau damping, bump-on-tail instabilities and more, are therefore highly influential in tokamak plasma dynamics. Purely fluid models are inherently incapable of capturing these effects, whereas the high dimensionality in purely kinetic models render them practically intractable for most relevant purposes.

        We consider a $\delta\!f$ decomposition model, with a macroscopic fluid background and microscopic kinetic correction, both fully coupled to each other. A similar manner of discretization is proposed to that used in the recent \texttt{STRUPHY} code \cite{Holderied_Possanner_Wang_2021, Holderied_2022, Li_et_al_2023} with a finite-element model for the background and a pseudo-particle/PiC model for the correction.

        The fluid background satisfies the full, non-linear, resistive, compressible, Hall MHD equations. \cite{Laakmann_Hu_Farrell_2022} introduces finite-element(-in-space) implicit timesteppers for the incompressible analogue to this system with structure-preserving (SP) properties in the ideal case, alongside parameter-robust preconditioners. We show that these timesteppers can derive from a finite-element-in-time (FET) (and finite-element-in-space) interpretation. The benefits of this reformulation are discussed, including the derivation of timesteppers that are higher order in time, and the quantifiable dissipative SP properties in the non-ideal, resistive case.
        
        We discuss possible options for extending this FET approach to timesteppers for the compressible case.

        The kinetic corrections satisfy linearized Boltzmann equations. Using a Lénard--Bernstein collision operator, these take Fokker--Planck-like forms \cite{Fokker_1914, Planck_1917} wherein pseudo-particles in the numerical model obey the neoclassical transport equations, with particle-independent Brownian drift terms. This offers a rigorous methodology for incorporating collisions into the particle transport model, without coupling the equations of motions for each particle.
        
        Works by Chen, Chacón et al. \cite{Chen_Chacón_Barnes_2011, Chacón_Chen_Barnes_2013, Chen_Chacón_2014, Chen_Chacón_2015} have developed structure-preserving particle pushers for neoclassical transport in the Vlasov equations, derived from Crank--Nicolson integrators. We show these too can can derive from a FET interpretation, similarly offering potential extensions to higher-order-in-time particle pushers. The FET formulation is used also to consider how the stochastic drift terms can be incorporated into the pushers. Stochastic gyrokinetic expansions are also discussed.

        Different options for the numerical implementation of these schemes are considered.

        Due to the efficacy of FET in the development of SP timesteppers for both the fluid and kinetic component, we hope this approach will prove effective in the future for developing SP timesteppers for the full hybrid model. We hope this will give us the opportunity to incorporate previously inaccessible kinetic effects into the highly effective, modern, finite-element MHD models.
    \end{abstract}
    
    
    \newpage
    \tableofcontents
    
    
    \newpage
    \pagenumbering{arabic}
    %\linenumbers\renewcommand\thelinenumber{\color{black!50}\arabic{linenumber}}
            \input{0 - introduction/main.tex}
        \part{Research}
            \input{1 - low-noise PiC models/main.tex}
            \input{2 - kinetic component/main.tex}
            \input{3 - fluid component/main.tex}
            \input{4 - numerical implementation/main.tex}
        \part{Project Overview}
            \input{5 - research plan/main.tex}
            \input{6 - summary/main.tex}
    
    
    %\section{}
    \newpage
    \pagenumbering{gobble}
        \printbibliography


    \newpage
    \pagenumbering{roman}
    \appendix
        \part{Appendices}
            \input{8 - Hilbert complexes/main.tex}
            \input{9 - weak conservation proofs/main.tex}
\end{document}

        \part{Research}
            \documentclass[12pt, a4paper]{report}

\input{template/main.tex}

\title{\BA{Title in Progress...}}
\author{Boris Andrews}
\affil{Mathematical Institute, University of Oxford}
\date{\today}


\begin{document}
    \pagenumbering{gobble}
    \maketitle
    
    
    \begin{abstract}
        Magnetic confinement reactors---in particular tokamaks---offer one of the most promising options for achieving practical nuclear fusion, with the potential to provide virtually limitless, clean energy. The theoretical and numerical modeling of tokamak plasmas is simultaneously an essential component of effective reactor design, and a great research barrier. Tokamak operational conditions exhibit comparatively low Knudsen numbers. Kinetic effects, including kinetic waves and instabilities, Landau damping, bump-on-tail instabilities and more, are therefore highly influential in tokamak plasma dynamics. Purely fluid models are inherently incapable of capturing these effects, whereas the high dimensionality in purely kinetic models render them practically intractable for most relevant purposes.

        We consider a $\delta\!f$ decomposition model, with a macroscopic fluid background and microscopic kinetic correction, both fully coupled to each other. A similar manner of discretization is proposed to that used in the recent \texttt{STRUPHY} code \cite{Holderied_Possanner_Wang_2021, Holderied_2022, Li_et_al_2023} with a finite-element model for the background and a pseudo-particle/PiC model for the correction.

        The fluid background satisfies the full, non-linear, resistive, compressible, Hall MHD equations. \cite{Laakmann_Hu_Farrell_2022} introduces finite-element(-in-space) implicit timesteppers for the incompressible analogue to this system with structure-preserving (SP) properties in the ideal case, alongside parameter-robust preconditioners. We show that these timesteppers can derive from a finite-element-in-time (FET) (and finite-element-in-space) interpretation. The benefits of this reformulation are discussed, including the derivation of timesteppers that are higher order in time, and the quantifiable dissipative SP properties in the non-ideal, resistive case.
        
        We discuss possible options for extending this FET approach to timesteppers for the compressible case.

        The kinetic corrections satisfy linearized Boltzmann equations. Using a Lénard--Bernstein collision operator, these take Fokker--Planck-like forms \cite{Fokker_1914, Planck_1917} wherein pseudo-particles in the numerical model obey the neoclassical transport equations, with particle-independent Brownian drift terms. This offers a rigorous methodology for incorporating collisions into the particle transport model, without coupling the equations of motions for each particle.
        
        Works by Chen, Chacón et al. \cite{Chen_Chacón_Barnes_2011, Chacón_Chen_Barnes_2013, Chen_Chacón_2014, Chen_Chacón_2015} have developed structure-preserving particle pushers for neoclassical transport in the Vlasov equations, derived from Crank--Nicolson integrators. We show these too can can derive from a FET interpretation, similarly offering potential extensions to higher-order-in-time particle pushers. The FET formulation is used also to consider how the stochastic drift terms can be incorporated into the pushers. Stochastic gyrokinetic expansions are also discussed.

        Different options for the numerical implementation of these schemes are considered.

        Due to the efficacy of FET in the development of SP timesteppers for both the fluid and kinetic component, we hope this approach will prove effective in the future for developing SP timesteppers for the full hybrid model. We hope this will give us the opportunity to incorporate previously inaccessible kinetic effects into the highly effective, modern, finite-element MHD models.
    \end{abstract}
    
    
    \newpage
    \tableofcontents
    
    
    \newpage
    \pagenumbering{arabic}
    %\linenumbers\renewcommand\thelinenumber{\color{black!50}\arabic{linenumber}}
            \input{0 - introduction/main.tex}
        \part{Research}
            \input{1 - low-noise PiC models/main.tex}
            \input{2 - kinetic component/main.tex}
            \input{3 - fluid component/main.tex}
            \input{4 - numerical implementation/main.tex}
        \part{Project Overview}
            \input{5 - research plan/main.tex}
            \input{6 - summary/main.tex}
    
    
    %\section{}
    \newpage
    \pagenumbering{gobble}
        \printbibliography


    \newpage
    \pagenumbering{roman}
    \appendix
        \part{Appendices}
            \input{8 - Hilbert complexes/main.tex}
            \input{9 - weak conservation proofs/main.tex}
\end{document}

            \documentclass[12pt, a4paper]{report}

\input{template/main.tex}

\title{\BA{Title in Progress...}}
\author{Boris Andrews}
\affil{Mathematical Institute, University of Oxford}
\date{\today}


\begin{document}
    \pagenumbering{gobble}
    \maketitle
    
    
    \begin{abstract}
        Magnetic confinement reactors---in particular tokamaks---offer one of the most promising options for achieving practical nuclear fusion, with the potential to provide virtually limitless, clean energy. The theoretical and numerical modeling of tokamak plasmas is simultaneously an essential component of effective reactor design, and a great research barrier. Tokamak operational conditions exhibit comparatively low Knudsen numbers. Kinetic effects, including kinetic waves and instabilities, Landau damping, bump-on-tail instabilities and more, are therefore highly influential in tokamak plasma dynamics. Purely fluid models are inherently incapable of capturing these effects, whereas the high dimensionality in purely kinetic models render them practically intractable for most relevant purposes.

        We consider a $\delta\!f$ decomposition model, with a macroscopic fluid background and microscopic kinetic correction, both fully coupled to each other. A similar manner of discretization is proposed to that used in the recent \texttt{STRUPHY} code \cite{Holderied_Possanner_Wang_2021, Holderied_2022, Li_et_al_2023} with a finite-element model for the background and a pseudo-particle/PiC model for the correction.

        The fluid background satisfies the full, non-linear, resistive, compressible, Hall MHD equations. \cite{Laakmann_Hu_Farrell_2022} introduces finite-element(-in-space) implicit timesteppers for the incompressible analogue to this system with structure-preserving (SP) properties in the ideal case, alongside parameter-robust preconditioners. We show that these timesteppers can derive from a finite-element-in-time (FET) (and finite-element-in-space) interpretation. The benefits of this reformulation are discussed, including the derivation of timesteppers that are higher order in time, and the quantifiable dissipative SP properties in the non-ideal, resistive case.
        
        We discuss possible options for extending this FET approach to timesteppers for the compressible case.

        The kinetic corrections satisfy linearized Boltzmann equations. Using a Lénard--Bernstein collision operator, these take Fokker--Planck-like forms \cite{Fokker_1914, Planck_1917} wherein pseudo-particles in the numerical model obey the neoclassical transport equations, with particle-independent Brownian drift terms. This offers a rigorous methodology for incorporating collisions into the particle transport model, without coupling the equations of motions for each particle.
        
        Works by Chen, Chacón et al. \cite{Chen_Chacón_Barnes_2011, Chacón_Chen_Barnes_2013, Chen_Chacón_2014, Chen_Chacón_2015} have developed structure-preserving particle pushers for neoclassical transport in the Vlasov equations, derived from Crank--Nicolson integrators. We show these too can can derive from a FET interpretation, similarly offering potential extensions to higher-order-in-time particle pushers. The FET formulation is used also to consider how the stochastic drift terms can be incorporated into the pushers. Stochastic gyrokinetic expansions are also discussed.

        Different options for the numerical implementation of these schemes are considered.

        Due to the efficacy of FET in the development of SP timesteppers for both the fluid and kinetic component, we hope this approach will prove effective in the future for developing SP timesteppers for the full hybrid model. We hope this will give us the opportunity to incorporate previously inaccessible kinetic effects into the highly effective, modern, finite-element MHD models.
    \end{abstract}
    
    
    \newpage
    \tableofcontents
    
    
    \newpage
    \pagenumbering{arabic}
    %\linenumbers\renewcommand\thelinenumber{\color{black!50}\arabic{linenumber}}
            \input{0 - introduction/main.tex}
        \part{Research}
            \input{1 - low-noise PiC models/main.tex}
            \input{2 - kinetic component/main.tex}
            \input{3 - fluid component/main.tex}
            \input{4 - numerical implementation/main.tex}
        \part{Project Overview}
            \input{5 - research plan/main.tex}
            \input{6 - summary/main.tex}
    
    
    %\section{}
    \newpage
    \pagenumbering{gobble}
        \printbibliography


    \newpage
    \pagenumbering{roman}
    \appendix
        \part{Appendices}
            \input{8 - Hilbert complexes/main.tex}
            \input{9 - weak conservation proofs/main.tex}
\end{document}

            \documentclass[12pt, a4paper]{report}

\input{template/main.tex}

\title{\BA{Title in Progress...}}
\author{Boris Andrews}
\affil{Mathematical Institute, University of Oxford}
\date{\today}


\begin{document}
    \pagenumbering{gobble}
    \maketitle
    
    
    \begin{abstract}
        Magnetic confinement reactors---in particular tokamaks---offer one of the most promising options for achieving practical nuclear fusion, with the potential to provide virtually limitless, clean energy. The theoretical and numerical modeling of tokamak plasmas is simultaneously an essential component of effective reactor design, and a great research barrier. Tokamak operational conditions exhibit comparatively low Knudsen numbers. Kinetic effects, including kinetic waves and instabilities, Landau damping, bump-on-tail instabilities and more, are therefore highly influential in tokamak plasma dynamics. Purely fluid models are inherently incapable of capturing these effects, whereas the high dimensionality in purely kinetic models render them practically intractable for most relevant purposes.

        We consider a $\delta\!f$ decomposition model, with a macroscopic fluid background and microscopic kinetic correction, both fully coupled to each other. A similar manner of discretization is proposed to that used in the recent \texttt{STRUPHY} code \cite{Holderied_Possanner_Wang_2021, Holderied_2022, Li_et_al_2023} with a finite-element model for the background and a pseudo-particle/PiC model for the correction.

        The fluid background satisfies the full, non-linear, resistive, compressible, Hall MHD equations. \cite{Laakmann_Hu_Farrell_2022} introduces finite-element(-in-space) implicit timesteppers for the incompressible analogue to this system with structure-preserving (SP) properties in the ideal case, alongside parameter-robust preconditioners. We show that these timesteppers can derive from a finite-element-in-time (FET) (and finite-element-in-space) interpretation. The benefits of this reformulation are discussed, including the derivation of timesteppers that are higher order in time, and the quantifiable dissipative SP properties in the non-ideal, resistive case.
        
        We discuss possible options for extending this FET approach to timesteppers for the compressible case.

        The kinetic corrections satisfy linearized Boltzmann equations. Using a Lénard--Bernstein collision operator, these take Fokker--Planck-like forms \cite{Fokker_1914, Planck_1917} wherein pseudo-particles in the numerical model obey the neoclassical transport equations, with particle-independent Brownian drift terms. This offers a rigorous methodology for incorporating collisions into the particle transport model, without coupling the equations of motions for each particle.
        
        Works by Chen, Chacón et al. \cite{Chen_Chacón_Barnes_2011, Chacón_Chen_Barnes_2013, Chen_Chacón_2014, Chen_Chacón_2015} have developed structure-preserving particle pushers for neoclassical transport in the Vlasov equations, derived from Crank--Nicolson integrators. We show these too can can derive from a FET interpretation, similarly offering potential extensions to higher-order-in-time particle pushers. The FET formulation is used also to consider how the stochastic drift terms can be incorporated into the pushers. Stochastic gyrokinetic expansions are also discussed.

        Different options for the numerical implementation of these schemes are considered.

        Due to the efficacy of FET in the development of SP timesteppers for both the fluid and kinetic component, we hope this approach will prove effective in the future for developing SP timesteppers for the full hybrid model. We hope this will give us the opportunity to incorporate previously inaccessible kinetic effects into the highly effective, modern, finite-element MHD models.
    \end{abstract}
    
    
    \newpage
    \tableofcontents
    
    
    \newpage
    \pagenumbering{arabic}
    %\linenumbers\renewcommand\thelinenumber{\color{black!50}\arabic{linenumber}}
            \input{0 - introduction/main.tex}
        \part{Research}
            \input{1 - low-noise PiC models/main.tex}
            \input{2 - kinetic component/main.tex}
            \input{3 - fluid component/main.tex}
            \input{4 - numerical implementation/main.tex}
        \part{Project Overview}
            \input{5 - research plan/main.tex}
            \input{6 - summary/main.tex}
    
    
    %\section{}
    \newpage
    \pagenumbering{gobble}
        \printbibliography


    \newpage
    \pagenumbering{roman}
    \appendix
        \part{Appendices}
            \input{8 - Hilbert complexes/main.tex}
            \input{9 - weak conservation proofs/main.tex}
\end{document}

            \documentclass[12pt, a4paper]{report}

\input{template/main.tex}

\title{\BA{Title in Progress...}}
\author{Boris Andrews}
\affil{Mathematical Institute, University of Oxford}
\date{\today}


\begin{document}
    \pagenumbering{gobble}
    \maketitle
    
    
    \begin{abstract}
        Magnetic confinement reactors---in particular tokamaks---offer one of the most promising options for achieving practical nuclear fusion, with the potential to provide virtually limitless, clean energy. The theoretical and numerical modeling of tokamak plasmas is simultaneously an essential component of effective reactor design, and a great research barrier. Tokamak operational conditions exhibit comparatively low Knudsen numbers. Kinetic effects, including kinetic waves and instabilities, Landau damping, bump-on-tail instabilities and more, are therefore highly influential in tokamak plasma dynamics. Purely fluid models are inherently incapable of capturing these effects, whereas the high dimensionality in purely kinetic models render them practically intractable for most relevant purposes.

        We consider a $\delta\!f$ decomposition model, with a macroscopic fluid background and microscopic kinetic correction, both fully coupled to each other. A similar manner of discretization is proposed to that used in the recent \texttt{STRUPHY} code \cite{Holderied_Possanner_Wang_2021, Holderied_2022, Li_et_al_2023} with a finite-element model for the background and a pseudo-particle/PiC model for the correction.

        The fluid background satisfies the full, non-linear, resistive, compressible, Hall MHD equations. \cite{Laakmann_Hu_Farrell_2022} introduces finite-element(-in-space) implicit timesteppers for the incompressible analogue to this system with structure-preserving (SP) properties in the ideal case, alongside parameter-robust preconditioners. We show that these timesteppers can derive from a finite-element-in-time (FET) (and finite-element-in-space) interpretation. The benefits of this reformulation are discussed, including the derivation of timesteppers that are higher order in time, and the quantifiable dissipative SP properties in the non-ideal, resistive case.
        
        We discuss possible options for extending this FET approach to timesteppers for the compressible case.

        The kinetic corrections satisfy linearized Boltzmann equations. Using a Lénard--Bernstein collision operator, these take Fokker--Planck-like forms \cite{Fokker_1914, Planck_1917} wherein pseudo-particles in the numerical model obey the neoclassical transport equations, with particle-independent Brownian drift terms. This offers a rigorous methodology for incorporating collisions into the particle transport model, without coupling the equations of motions for each particle.
        
        Works by Chen, Chacón et al. \cite{Chen_Chacón_Barnes_2011, Chacón_Chen_Barnes_2013, Chen_Chacón_2014, Chen_Chacón_2015} have developed structure-preserving particle pushers for neoclassical transport in the Vlasov equations, derived from Crank--Nicolson integrators. We show these too can can derive from a FET interpretation, similarly offering potential extensions to higher-order-in-time particle pushers. The FET formulation is used also to consider how the stochastic drift terms can be incorporated into the pushers. Stochastic gyrokinetic expansions are also discussed.

        Different options for the numerical implementation of these schemes are considered.

        Due to the efficacy of FET in the development of SP timesteppers for both the fluid and kinetic component, we hope this approach will prove effective in the future for developing SP timesteppers for the full hybrid model. We hope this will give us the opportunity to incorporate previously inaccessible kinetic effects into the highly effective, modern, finite-element MHD models.
    \end{abstract}
    
    
    \newpage
    \tableofcontents
    
    
    \newpage
    \pagenumbering{arabic}
    %\linenumbers\renewcommand\thelinenumber{\color{black!50}\arabic{linenumber}}
            \input{0 - introduction/main.tex}
        \part{Research}
            \input{1 - low-noise PiC models/main.tex}
            \input{2 - kinetic component/main.tex}
            \input{3 - fluid component/main.tex}
            \input{4 - numerical implementation/main.tex}
        \part{Project Overview}
            \input{5 - research plan/main.tex}
            \input{6 - summary/main.tex}
    
    
    %\section{}
    \newpage
    \pagenumbering{gobble}
        \printbibliography


    \newpage
    \pagenumbering{roman}
    \appendix
        \part{Appendices}
            \input{8 - Hilbert complexes/main.tex}
            \input{9 - weak conservation proofs/main.tex}
\end{document}

        \part{Project Overview}
            \documentclass[12pt, a4paper]{report}

\input{template/main.tex}

\title{\BA{Title in Progress...}}
\author{Boris Andrews}
\affil{Mathematical Institute, University of Oxford}
\date{\today}


\begin{document}
    \pagenumbering{gobble}
    \maketitle
    
    
    \begin{abstract}
        Magnetic confinement reactors---in particular tokamaks---offer one of the most promising options for achieving practical nuclear fusion, with the potential to provide virtually limitless, clean energy. The theoretical and numerical modeling of tokamak plasmas is simultaneously an essential component of effective reactor design, and a great research barrier. Tokamak operational conditions exhibit comparatively low Knudsen numbers. Kinetic effects, including kinetic waves and instabilities, Landau damping, bump-on-tail instabilities and more, are therefore highly influential in tokamak plasma dynamics. Purely fluid models are inherently incapable of capturing these effects, whereas the high dimensionality in purely kinetic models render them practically intractable for most relevant purposes.

        We consider a $\delta\!f$ decomposition model, with a macroscopic fluid background and microscopic kinetic correction, both fully coupled to each other. A similar manner of discretization is proposed to that used in the recent \texttt{STRUPHY} code \cite{Holderied_Possanner_Wang_2021, Holderied_2022, Li_et_al_2023} with a finite-element model for the background and a pseudo-particle/PiC model for the correction.

        The fluid background satisfies the full, non-linear, resistive, compressible, Hall MHD equations. \cite{Laakmann_Hu_Farrell_2022} introduces finite-element(-in-space) implicit timesteppers for the incompressible analogue to this system with structure-preserving (SP) properties in the ideal case, alongside parameter-robust preconditioners. We show that these timesteppers can derive from a finite-element-in-time (FET) (and finite-element-in-space) interpretation. The benefits of this reformulation are discussed, including the derivation of timesteppers that are higher order in time, and the quantifiable dissipative SP properties in the non-ideal, resistive case.
        
        We discuss possible options for extending this FET approach to timesteppers for the compressible case.

        The kinetic corrections satisfy linearized Boltzmann equations. Using a Lénard--Bernstein collision operator, these take Fokker--Planck-like forms \cite{Fokker_1914, Planck_1917} wherein pseudo-particles in the numerical model obey the neoclassical transport equations, with particle-independent Brownian drift terms. This offers a rigorous methodology for incorporating collisions into the particle transport model, without coupling the equations of motions for each particle.
        
        Works by Chen, Chacón et al. \cite{Chen_Chacón_Barnes_2011, Chacón_Chen_Barnes_2013, Chen_Chacón_2014, Chen_Chacón_2015} have developed structure-preserving particle pushers for neoclassical transport in the Vlasov equations, derived from Crank--Nicolson integrators. We show these too can can derive from a FET interpretation, similarly offering potential extensions to higher-order-in-time particle pushers. The FET formulation is used also to consider how the stochastic drift terms can be incorporated into the pushers. Stochastic gyrokinetic expansions are also discussed.

        Different options for the numerical implementation of these schemes are considered.

        Due to the efficacy of FET in the development of SP timesteppers for both the fluid and kinetic component, we hope this approach will prove effective in the future for developing SP timesteppers for the full hybrid model. We hope this will give us the opportunity to incorporate previously inaccessible kinetic effects into the highly effective, modern, finite-element MHD models.
    \end{abstract}
    
    
    \newpage
    \tableofcontents
    
    
    \newpage
    \pagenumbering{arabic}
    %\linenumbers\renewcommand\thelinenumber{\color{black!50}\arabic{linenumber}}
            \input{0 - introduction/main.tex}
        \part{Research}
            \input{1 - low-noise PiC models/main.tex}
            \input{2 - kinetic component/main.tex}
            \input{3 - fluid component/main.tex}
            \input{4 - numerical implementation/main.tex}
        \part{Project Overview}
            \input{5 - research plan/main.tex}
            \input{6 - summary/main.tex}
    
    
    %\section{}
    \newpage
    \pagenumbering{gobble}
        \printbibliography


    \newpage
    \pagenumbering{roman}
    \appendix
        \part{Appendices}
            \input{8 - Hilbert complexes/main.tex}
            \input{9 - weak conservation proofs/main.tex}
\end{document}

            \documentclass[12pt, a4paper]{report}

\input{template/main.tex}

\title{\BA{Title in Progress...}}
\author{Boris Andrews}
\affil{Mathematical Institute, University of Oxford}
\date{\today}


\begin{document}
    \pagenumbering{gobble}
    \maketitle
    
    
    \begin{abstract}
        Magnetic confinement reactors---in particular tokamaks---offer one of the most promising options for achieving practical nuclear fusion, with the potential to provide virtually limitless, clean energy. The theoretical and numerical modeling of tokamak plasmas is simultaneously an essential component of effective reactor design, and a great research barrier. Tokamak operational conditions exhibit comparatively low Knudsen numbers. Kinetic effects, including kinetic waves and instabilities, Landau damping, bump-on-tail instabilities and more, are therefore highly influential in tokamak plasma dynamics. Purely fluid models are inherently incapable of capturing these effects, whereas the high dimensionality in purely kinetic models render them practically intractable for most relevant purposes.

        We consider a $\delta\!f$ decomposition model, with a macroscopic fluid background and microscopic kinetic correction, both fully coupled to each other. A similar manner of discretization is proposed to that used in the recent \texttt{STRUPHY} code \cite{Holderied_Possanner_Wang_2021, Holderied_2022, Li_et_al_2023} with a finite-element model for the background and a pseudo-particle/PiC model for the correction.

        The fluid background satisfies the full, non-linear, resistive, compressible, Hall MHD equations. \cite{Laakmann_Hu_Farrell_2022} introduces finite-element(-in-space) implicit timesteppers for the incompressible analogue to this system with structure-preserving (SP) properties in the ideal case, alongside parameter-robust preconditioners. We show that these timesteppers can derive from a finite-element-in-time (FET) (and finite-element-in-space) interpretation. The benefits of this reformulation are discussed, including the derivation of timesteppers that are higher order in time, and the quantifiable dissipative SP properties in the non-ideal, resistive case.
        
        We discuss possible options for extending this FET approach to timesteppers for the compressible case.

        The kinetic corrections satisfy linearized Boltzmann equations. Using a Lénard--Bernstein collision operator, these take Fokker--Planck-like forms \cite{Fokker_1914, Planck_1917} wherein pseudo-particles in the numerical model obey the neoclassical transport equations, with particle-independent Brownian drift terms. This offers a rigorous methodology for incorporating collisions into the particle transport model, without coupling the equations of motions for each particle.
        
        Works by Chen, Chacón et al. \cite{Chen_Chacón_Barnes_2011, Chacón_Chen_Barnes_2013, Chen_Chacón_2014, Chen_Chacón_2015} have developed structure-preserving particle pushers for neoclassical transport in the Vlasov equations, derived from Crank--Nicolson integrators. We show these too can can derive from a FET interpretation, similarly offering potential extensions to higher-order-in-time particle pushers. The FET formulation is used also to consider how the stochastic drift terms can be incorporated into the pushers. Stochastic gyrokinetic expansions are also discussed.

        Different options for the numerical implementation of these schemes are considered.

        Due to the efficacy of FET in the development of SP timesteppers for both the fluid and kinetic component, we hope this approach will prove effective in the future for developing SP timesteppers for the full hybrid model. We hope this will give us the opportunity to incorporate previously inaccessible kinetic effects into the highly effective, modern, finite-element MHD models.
    \end{abstract}
    
    
    \newpage
    \tableofcontents
    
    
    \newpage
    \pagenumbering{arabic}
    %\linenumbers\renewcommand\thelinenumber{\color{black!50}\arabic{linenumber}}
            \input{0 - introduction/main.tex}
        \part{Research}
            \input{1 - low-noise PiC models/main.tex}
            \input{2 - kinetic component/main.tex}
            \input{3 - fluid component/main.tex}
            \input{4 - numerical implementation/main.tex}
        \part{Project Overview}
            \input{5 - research plan/main.tex}
            \input{6 - summary/main.tex}
    
    
    %\section{}
    \newpage
    \pagenumbering{gobble}
        \printbibliography


    \newpage
    \pagenumbering{roman}
    \appendix
        \part{Appendices}
            \input{8 - Hilbert complexes/main.tex}
            \input{9 - weak conservation proofs/main.tex}
\end{document}

    
    
    %\section{}
    \newpage
    \pagenumbering{gobble}
        \printbibliography


    \newpage
    \pagenumbering{roman}
    \appendix
        \part{Appendices}
            \documentclass[12pt, a4paper]{report}

\input{template/main.tex}

\title{\BA{Title in Progress...}}
\author{Boris Andrews}
\affil{Mathematical Institute, University of Oxford}
\date{\today}


\begin{document}
    \pagenumbering{gobble}
    \maketitle
    
    
    \begin{abstract}
        Magnetic confinement reactors---in particular tokamaks---offer one of the most promising options for achieving practical nuclear fusion, with the potential to provide virtually limitless, clean energy. The theoretical and numerical modeling of tokamak plasmas is simultaneously an essential component of effective reactor design, and a great research barrier. Tokamak operational conditions exhibit comparatively low Knudsen numbers. Kinetic effects, including kinetic waves and instabilities, Landau damping, bump-on-tail instabilities and more, are therefore highly influential in tokamak plasma dynamics. Purely fluid models are inherently incapable of capturing these effects, whereas the high dimensionality in purely kinetic models render them practically intractable for most relevant purposes.

        We consider a $\delta\!f$ decomposition model, with a macroscopic fluid background and microscopic kinetic correction, both fully coupled to each other. A similar manner of discretization is proposed to that used in the recent \texttt{STRUPHY} code \cite{Holderied_Possanner_Wang_2021, Holderied_2022, Li_et_al_2023} with a finite-element model for the background and a pseudo-particle/PiC model for the correction.

        The fluid background satisfies the full, non-linear, resistive, compressible, Hall MHD equations. \cite{Laakmann_Hu_Farrell_2022} introduces finite-element(-in-space) implicit timesteppers for the incompressible analogue to this system with structure-preserving (SP) properties in the ideal case, alongside parameter-robust preconditioners. We show that these timesteppers can derive from a finite-element-in-time (FET) (and finite-element-in-space) interpretation. The benefits of this reformulation are discussed, including the derivation of timesteppers that are higher order in time, and the quantifiable dissipative SP properties in the non-ideal, resistive case.
        
        We discuss possible options for extending this FET approach to timesteppers for the compressible case.

        The kinetic corrections satisfy linearized Boltzmann equations. Using a Lénard--Bernstein collision operator, these take Fokker--Planck-like forms \cite{Fokker_1914, Planck_1917} wherein pseudo-particles in the numerical model obey the neoclassical transport equations, with particle-independent Brownian drift terms. This offers a rigorous methodology for incorporating collisions into the particle transport model, without coupling the equations of motions for each particle.
        
        Works by Chen, Chacón et al. \cite{Chen_Chacón_Barnes_2011, Chacón_Chen_Barnes_2013, Chen_Chacón_2014, Chen_Chacón_2015} have developed structure-preserving particle pushers for neoclassical transport in the Vlasov equations, derived from Crank--Nicolson integrators. We show these too can can derive from a FET interpretation, similarly offering potential extensions to higher-order-in-time particle pushers. The FET formulation is used also to consider how the stochastic drift terms can be incorporated into the pushers. Stochastic gyrokinetic expansions are also discussed.

        Different options for the numerical implementation of these schemes are considered.

        Due to the efficacy of FET in the development of SP timesteppers for both the fluid and kinetic component, we hope this approach will prove effective in the future for developing SP timesteppers for the full hybrid model. We hope this will give us the opportunity to incorporate previously inaccessible kinetic effects into the highly effective, modern, finite-element MHD models.
    \end{abstract}
    
    
    \newpage
    \tableofcontents
    
    
    \newpage
    \pagenumbering{arabic}
    %\linenumbers\renewcommand\thelinenumber{\color{black!50}\arabic{linenumber}}
            \input{0 - introduction/main.tex}
        \part{Research}
            \input{1 - low-noise PiC models/main.tex}
            \input{2 - kinetic component/main.tex}
            \input{3 - fluid component/main.tex}
            \input{4 - numerical implementation/main.tex}
        \part{Project Overview}
            \input{5 - research plan/main.tex}
            \input{6 - summary/main.tex}
    
    
    %\section{}
    \newpage
    \pagenumbering{gobble}
        \printbibliography


    \newpage
    \pagenumbering{roman}
    \appendix
        \part{Appendices}
            \input{8 - Hilbert complexes/main.tex}
            \input{9 - weak conservation proofs/main.tex}
\end{document}

            \documentclass[12pt, a4paper]{report}

\input{template/main.tex}

\title{\BA{Title in Progress...}}
\author{Boris Andrews}
\affil{Mathematical Institute, University of Oxford}
\date{\today}


\begin{document}
    \pagenumbering{gobble}
    \maketitle
    
    
    \begin{abstract}
        Magnetic confinement reactors---in particular tokamaks---offer one of the most promising options for achieving practical nuclear fusion, with the potential to provide virtually limitless, clean energy. The theoretical and numerical modeling of tokamak plasmas is simultaneously an essential component of effective reactor design, and a great research barrier. Tokamak operational conditions exhibit comparatively low Knudsen numbers. Kinetic effects, including kinetic waves and instabilities, Landau damping, bump-on-tail instabilities and more, are therefore highly influential in tokamak plasma dynamics. Purely fluid models are inherently incapable of capturing these effects, whereas the high dimensionality in purely kinetic models render them practically intractable for most relevant purposes.

        We consider a $\delta\!f$ decomposition model, with a macroscopic fluid background and microscopic kinetic correction, both fully coupled to each other. A similar manner of discretization is proposed to that used in the recent \texttt{STRUPHY} code \cite{Holderied_Possanner_Wang_2021, Holderied_2022, Li_et_al_2023} with a finite-element model for the background and a pseudo-particle/PiC model for the correction.

        The fluid background satisfies the full, non-linear, resistive, compressible, Hall MHD equations. \cite{Laakmann_Hu_Farrell_2022} introduces finite-element(-in-space) implicit timesteppers for the incompressible analogue to this system with structure-preserving (SP) properties in the ideal case, alongside parameter-robust preconditioners. We show that these timesteppers can derive from a finite-element-in-time (FET) (and finite-element-in-space) interpretation. The benefits of this reformulation are discussed, including the derivation of timesteppers that are higher order in time, and the quantifiable dissipative SP properties in the non-ideal, resistive case.
        
        We discuss possible options for extending this FET approach to timesteppers for the compressible case.

        The kinetic corrections satisfy linearized Boltzmann equations. Using a Lénard--Bernstein collision operator, these take Fokker--Planck-like forms \cite{Fokker_1914, Planck_1917} wherein pseudo-particles in the numerical model obey the neoclassical transport equations, with particle-independent Brownian drift terms. This offers a rigorous methodology for incorporating collisions into the particle transport model, without coupling the equations of motions for each particle.
        
        Works by Chen, Chacón et al. \cite{Chen_Chacón_Barnes_2011, Chacón_Chen_Barnes_2013, Chen_Chacón_2014, Chen_Chacón_2015} have developed structure-preserving particle pushers for neoclassical transport in the Vlasov equations, derived from Crank--Nicolson integrators. We show these too can can derive from a FET interpretation, similarly offering potential extensions to higher-order-in-time particle pushers. The FET formulation is used also to consider how the stochastic drift terms can be incorporated into the pushers. Stochastic gyrokinetic expansions are also discussed.

        Different options for the numerical implementation of these schemes are considered.

        Due to the efficacy of FET in the development of SP timesteppers for both the fluid and kinetic component, we hope this approach will prove effective in the future for developing SP timesteppers for the full hybrid model. We hope this will give us the opportunity to incorporate previously inaccessible kinetic effects into the highly effective, modern, finite-element MHD models.
    \end{abstract}
    
    
    \newpage
    \tableofcontents
    
    
    \newpage
    \pagenumbering{arabic}
    %\linenumbers\renewcommand\thelinenumber{\color{black!50}\arabic{linenumber}}
            \input{0 - introduction/main.tex}
        \part{Research}
            \input{1 - low-noise PiC models/main.tex}
            \input{2 - kinetic component/main.tex}
            \input{3 - fluid component/main.tex}
            \input{4 - numerical implementation/main.tex}
        \part{Project Overview}
            \input{5 - research plan/main.tex}
            \input{6 - summary/main.tex}
    
    
    %\section{}
    \newpage
    \pagenumbering{gobble}
        \printbibliography


    \newpage
    \pagenumbering{roman}
    \appendix
        \part{Appendices}
            \input{8 - Hilbert complexes/main.tex}
            \input{9 - weak conservation proofs/main.tex}
\end{document}

\end{document}

        \part{Project Overview}
            \documentclass[12pt, a4paper]{report}

\documentclass[12pt, a4paper]{report}

\input{template/main.tex}

\title{\BA{Title in Progress...}}
\author{Boris Andrews}
\affil{Mathematical Institute, University of Oxford}
\date{\today}


\begin{document}
    \pagenumbering{gobble}
    \maketitle
    
    
    \begin{abstract}
        Magnetic confinement reactors---in particular tokamaks---offer one of the most promising options for achieving practical nuclear fusion, with the potential to provide virtually limitless, clean energy. The theoretical and numerical modeling of tokamak plasmas is simultaneously an essential component of effective reactor design, and a great research barrier. Tokamak operational conditions exhibit comparatively low Knudsen numbers. Kinetic effects, including kinetic waves and instabilities, Landau damping, bump-on-tail instabilities and more, are therefore highly influential in tokamak plasma dynamics. Purely fluid models are inherently incapable of capturing these effects, whereas the high dimensionality in purely kinetic models render them practically intractable for most relevant purposes.

        We consider a $\delta\!f$ decomposition model, with a macroscopic fluid background and microscopic kinetic correction, both fully coupled to each other. A similar manner of discretization is proposed to that used in the recent \texttt{STRUPHY} code \cite{Holderied_Possanner_Wang_2021, Holderied_2022, Li_et_al_2023} with a finite-element model for the background and a pseudo-particle/PiC model for the correction.

        The fluid background satisfies the full, non-linear, resistive, compressible, Hall MHD equations. \cite{Laakmann_Hu_Farrell_2022} introduces finite-element(-in-space) implicit timesteppers for the incompressible analogue to this system with structure-preserving (SP) properties in the ideal case, alongside parameter-robust preconditioners. We show that these timesteppers can derive from a finite-element-in-time (FET) (and finite-element-in-space) interpretation. The benefits of this reformulation are discussed, including the derivation of timesteppers that are higher order in time, and the quantifiable dissipative SP properties in the non-ideal, resistive case.
        
        We discuss possible options for extending this FET approach to timesteppers for the compressible case.

        The kinetic corrections satisfy linearized Boltzmann equations. Using a Lénard--Bernstein collision operator, these take Fokker--Planck-like forms \cite{Fokker_1914, Planck_1917} wherein pseudo-particles in the numerical model obey the neoclassical transport equations, with particle-independent Brownian drift terms. This offers a rigorous methodology for incorporating collisions into the particle transport model, without coupling the equations of motions for each particle.
        
        Works by Chen, Chacón et al. \cite{Chen_Chacón_Barnes_2011, Chacón_Chen_Barnes_2013, Chen_Chacón_2014, Chen_Chacón_2015} have developed structure-preserving particle pushers for neoclassical transport in the Vlasov equations, derived from Crank--Nicolson integrators. We show these too can can derive from a FET interpretation, similarly offering potential extensions to higher-order-in-time particle pushers. The FET formulation is used also to consider how the stochastic drift terms can be incorporated into the pushers. Stochastic gyrokinetic expansions are also discussed.

        Different options for the numerical implementation of these schemes are considered.

        Due to the efficacy of FET in the development of SP timesteppers for both the fluid and kinetic component, we hope this approach will prove effective in the future for developing SP timesteppers for the full hybrid model. We hope this will give us the opportunity to incorporate previously inaccessible kinetic effects into the highly effective, modern, finite-element MHD models.
    \end{abstract}
    
    
    \newpage
    \tableofcontents
    
    
    \newpage
    \pagenumbering{arabic}
    %\linenumbers\renewcommand\thelinenumber{\color{black!50}\arabic{linenumber}}
            \input{0 - introduction/main.tex}
        \part{Research}
            \input{1 - low-noise PiC models/main.tex}
            \input{2 - kinetic component/main.tex}
            \input{3 - fluid component/main.tex}
            \input{4 - numerical implementation/main.tex}
        \part{Project Overview}
            \input{5 - research plan/main.tex}
            \input{6 - summary/main.tex}
    
    
    %\section{}
    \newpage
    \pagenumbering{gobble}
        \printbibliography


    \newpage
    \pagenumbering{roman}
    \appendix
        \part{Appendices}
            \input{8 - Hilbert complexes/main.tex}
            \input{9 - weak conservation proofs/main.tex}
\end{document}


\title{\BA{Title in Progress...}}
\author{Boris Andrews}
\affil{Mathematical Institute, University of Oxford}
\date{\today}


\begin{document}
    \pagenumbering{gobble}
    \maketitle
    
    
    \begin{abstract}
        Magnetic confinement reactors---in particular tokamaks---offer one of the most promising options for achieving practical nuclear fusion, with the potential to provide virtually limitless, clean energy. The theoretical and numerical modeling of tokamak plasmas is simultaneously an essential component of effective reactor design, and a great research barrier. Tokamak operational conditions exhibit comparatively low Knudsen numbers. Kinetic effects, including kinetic waves and instabilities, Landau damping, bump-on-tail instabilities and more, are therefore highly influential in tokamak plasma dynamics. Purely fluid models are inherently incapable of capturing these effects, whereas the high dimensionality in purely kinetic models render them practically intractable for most relevant purposes.

        We consider a $\delta\!f$ decomposition model, with a macroscopic fluid background and microscopic kinetic correction, both fully coupled to each other. A similar manner of discretization is proposed to that used in the recent \texttt{STRUPHY} code \cite{Holderied_Possanner_Wang_2021, Holderied_2022, Li_et_al_2023} with a finite-element model for the background and a pseudo-particle/PiC model for the correction.

        The fluid background satisfies the full, non-linear, resistive, compressible, Hall MHD equations. \cite{Laakmann_Hu_Farrell_2022} introduces finite-element(-in-space) implicit timesteppers for the incompressible analogue to this system with structure-preserving (SP) properties in the ideal case, alongside parameter-robust preconditioners. We show that these timesteppers can derive from a finite-element-in-time (FET) (and finite-element-in-space) interpretation. The benefits of this reformulation are discussed, including the derivation of timesteppers that are higher order in time, and the quantifiable dissipative SP properties in the non-ideal, resistive case.
        
        We discuss possible options for extending this FET approach to timesteppers for the compressible case.

        The kinetic corrections satisfy linearized Boltzmann equations. Using a Lénard--Bernstein collision operator, these take Fokker--Planck-like forms \cite{Fokker_1914, Planck_1917} wherein pseudo-particles in the numerical model obey the neoclassical transport equations, with particle-independent Brownian drift terms. This offers a rigorous methodology for incorporating collisions into the particle transport model, without coupling the equations of motions for each particle.
        
        Works by Chen, Chacón et al. \cite{Chen_Chacón_Barnes_2011, Chacón_Chen_Barnes_2013, Chen_Chacón_2014, Chen_Chacón_2015} have developed structure-preserving particle pushers for neoclassical transport in the Vlasov equations, derived from Crank--Nicolson integrators. We show these too can can derive from a FET interpretation, similarly offering potential extensions to higher-order-in-time particle pushers. The FET formulation is used also to consider how the stochastic drift terms can be incorporated into the pushers. Stochastic gyrokinetic expansions are also discussed.

        Different options for the numerical implementation of these schemes are considered.

        Due to the efficacy of FET in the development of SP timesteppers for both the fluid and kinetic component, we hope this approach will prove effective in the future for developing SP timesteppers for the full hybrid model. We hope this will give us the opportunity to incorporate previously inaccessible kinetic effects into the highly effective, modern, finite-element MHD models.
    \end{abstract}
    
    
    \newpage
    \tableofcontents
    
    
    \newpage
    \pagenumbering{arabic}
    %\linenumbers\renewcommand\thelinenumber{\color{black!50}\arabic{linenumber}}
            \documentclass[12pt, a4paper]{report}

\input{template/main.tex}

\title{\BA{Title in Progress...}}
\author{Boris Andrews}
\affil{Mathematical Institute, University of Oxford}
\date{\today}


\begin{document}
    \pagenumbering{gobble}
    \maketitle
    
    
    \begin{abstract}
        Magnetic confinement reactors---in particular tokamaks---offer one of the most promising options for achieving practical nuclear fusion, with the potential to provide virtually limitless, clean energy. The theoretical and numerical modeling of tokamak plasmas is simultaneously an essential component of effective reactor design, and a great research barrier. Tokamak operational conditions exhibit comparatively low Knudsen numbers. Kinetic effects, including kinetic waves and instabilities, Landau damping, bump-on-tail instabilities and more, are therefore highly influential in tokamak plasma dynamics. Purely fluid models are inherently incapable of capturing these effects, whereas the high dimensionality in purely kinetic models render them practically intractable for most relevant purposes.

        We consider a $\delta\!f$ decomposition model, with a macroscopic fluid background and microscopic kinetic correction, both fully coupled to each other. A similar manner of discretization is proposed to that used in the recent \texttt{STRUPHY} code \cite{Holderied_Possanner_Wang_2021, Holderied_2022, Li_et_al_2023} with a finite-element model for the background and a pseudo-particle/PiC model for the correction.

        The fluid background satisfies the full, non-linear, resistive, compressible, Hall MHD equations. \cite{Laakmann_Hu_Farrell_2022} introduces finite-element(-in-space) implicit timesteppers for the incompressible analogue to this system with structure-preserving (SP) properties in the ideal case, alongside parameter-robust preconditioners. We show that these timesteppers can derive from a finite-element-in-time (FET) (and finite-element-in-space) interpretation. The benefits of this reformulation are discussed, including the derivation of timesteppers that are higher order in time, and the quantifiable dissipative SP properties in the non-ideal, resistive case.
        
        We discuss possible options for extending this FET approach to timesteppers for the compressible case.

        The kinetic corrections satisfy linearized Boltzmann equations. Using a Lénard--Bernstein collision operator, these take Fokker--Planck-like forms \cite{Fokker_1914, Planck_1917} wherein pseudo-particles in the numerical model obey the neoclassical transport equations, with particle-independent Brownian drift terms. This offers a rigorous methodology for incorporating collisions into the particle transport model, without coupling the equations of motions for each particle.
        
        Works by Chen, Chacón et al. \cite{Chen_Chacón_Barnes_2011, Chacón_Chen_Barnes_2013, Chen_Chacón_2014, Chen_Chacón_2015} have developed structure-preserving particle pushers for neoclassical transport in the Vlasov equations, derived from Crank--Nicolson integrators. We show these too can can derive from a FET interpretation, similarly offering potential extensions to higher-order-in-time particle pushers. The FET formulation is used also to consider how the stochastic drift terms can be incorporated into the pushers. Stochastic gyrokinetic expansions are also discussed.

        Different options for the numerical implementation of these schemes are considered.

        Due to the efficacy of FET in the development of SP timesteppers for both the fluid and kinetic component, we hope this approach will prove effective in the future for developing SP timesteppers for the full hybrid model. We hope this will give us the opportunity to incorporate previously inaccessible kinetic effects into the highly effective, modern, finite-element MHD models.
    \end{abstract}
    
    
    \newpage
    \tableofcontents
    
    
    \newpage
    \pagenumbering{arabic}
    %\linenumbers\renewcommand\thelinenumber{\color{black!50}\arabic{linenumber}}
            \input{0 - introduction/main.tex}
        \part{Research}
            \input{1 - low-noise PiC models/main.tex}
            \input{2 - kinetic component/main.tex}
            \input{3 - fluid component/main.tex}
            \input{4 - numerical implementation/main.tex}
        \part{Project Overview}
            \input{5 - research plan/main.tex}
            \input{6 - summary/main.tex}
    
    
    %\section{}
    \newpage
    \pagenumbering{gobble}
        \printbibliography


    \newpage
    \pagenumbering{roman}
    \appendix
        \part{Appendices}
            \input{8 - Hilbert complexes/main.tex}
            \input{9 - weak conservation proofs/main.tex}
\end{document}

        \part{Research}
            \documentclass[12pt, a4paper]{report}

\input{template/main.tex}

\title{\BA{Title in Progress...}}
\author{Boris Andrews}
\affil{Mathematical Institute, University of Oxford}
\date{\today}


\begin{document}
    \pagenumbering{gobble}
    \maketitle
    
    
    \begin{abstract}
        Magnetic confinement reactors---in particular tokamaks---offer one of the most promising options for achieving practical nuclear fusion, with the potential to provide virtually limitless, clean energy. The theoretical and numerical modeling of tokamak plasmas is simultaneously an essential component of effective reactor design, and a great research barrier. Tokamak operational conditions exhibit comparatively low Knudsen numbers. Kinetic effects, including kinetic waves and instabilities, Landau damping, bump-on-tail instabilities and more, are therefore highly influential in tokamak plasma dynamics. Purely fluid models are inherently incapable of capturing these effects, whereas the high dimensionality in purely kinetic models render them practically intractable for most relevant purposes.

        We consider a $\delta\!f$ decomposition model, with a macroscopic fluid background and microscopic kinetic correction, both fully coupled to each other. A similar manner of discretization is proposed to that used in the recent \texttt{STRUPHY} code \cite{Holderied_Possanner_Wang_2021, Holderied_2022, Li_et_al_2023} with a finite-element model for the background and a pseudo-particle/PiC model for the correction.

        The fluid background satisfies the full, non-linear, resistive, compressible, Hall MHD equations. \cite{Laakmann_Hu_Farrell_2022} introduces finite-element(-in-space) implicit timesteppers for the incompressible analogue to this system with structure-preserving (SP) properties in the ideal case, alongside parameter-robust preconditioners. We show that these timesteppers can derive from a finite-element-in-time (FET) (and finite-element-in-space) interpretation. The benefits of this reformulation are discussed, including the derivation of timesteppers that are higher order in time, and the quantifiable dissipative SP properties in the non-ideal, resistive case.
        
        We discuss possible options for extending this FET approach to timesteppers for the compressible case.

        The kinetic corrections satisfy linearized Boltzmann equations. Using a Lénard--Bernstein collision operator, these take Fokker--Planck-like forms \cite{Fokker_1914, Planck_1917} wherein pseudo-particles in the numerical model obey the neoclassical transport equations, with particle-independent Brownian drift terms. This offers a rigorous methodology for incorporating collisions into the particle transport model, without coupling the equations of motions for each particle.
        
        Works by Chen, Chacón et al. \cite{Chen_Chacón_Barnes_2011, Chacón_Chen_Barnes_2013, Chen_Chacón_2014, Chen_Chacón_2015} have developed structure-preserving particle pushers for neoclassical transport in the Vlasov equations, derived from Crank--Nicolson integrators. We show these too can can derive from a FET interpretation, similarly offering potential extensions to higher-order-in-time particle pushers. The FET formulation is used also to consider how the stochastic drift terms can be incorporated into the pushers. Stochastic gyrokinetic expansions are also discussed.

        Different options for the numerical implementation of these schemes are considered.

        Due to the efficacy of FET in the development of SP timesteppers for both the fluid and kinetic component, we hope this approach will prove effective in the future for developing SP timesteppers for the full hybrid model. We hope this will give us the opportunity to incorporate previously inaccessible kinetic effects into the highly effective, modern, finite-element MHD models.
    \end{abstract}
    
    
    \newpage
    \tableofcontents
    
    
    \newpage
    \pagenumbering{arabic}
    %\linenumbers\renewcommand\thelinenumber{\color{black!50}\arabic{linenumber}}
            \input{0 - introduction/main.tex}
        \part{Research}
            \input{1 - low-noise PiC models/main.tex}
            \input{2 - kinetic component/main.tex}
            \input{3 - fluid component/main.tex}
            \input{4 - numerical implementation/main.tex}
        \part{Project Overview}
            \input{5 - research plan/main.tex}
            \input{6 - summary/main.tex}
    
    
    %\section{}
    \newpage
    \pagenumbering{gobble}
        \printbibliography


    \newpage
    \pagenumbering{roman}
    \appendix
        \part{Appendices}
            \input{8 - Hilbert complexes/main.tex}
            \input{9 - weak conservation proofs/main.tex}
\end{document}

            \documentclass[12pt, a4paper]{report}

\input{template/main.tex}

\title{\BA{Title in Progress...}}
\author{Boris Andrews}
\affil{Mathematical Institute, University of Oxford}
\date{\today}


\begin{document}
    \pagenumbering{gobble}
    \maketitle
    
    
    \begin{abstract}
        Magnetic confinement reactors---in particular tokamaks---offer one of the most promising options for achieving practical nuclear fusion, with the potential to provide virtually limitless, clean energy. The theoretical and numerical modeling of tokamak plasmas is simultaneously an essential component of effective reactor design, and a great research barrier. Tokamak operational conditions exhibit comparatively low Knudsen numbers. Kinetic effects, including kinetic waves and instabilities, Landau damping, bump-on-tail instabilities and more, are therefore highly influential in tokamak plasma dynamics. Purely fluid models are inherently incapable of capturing these effects, whereas the high dimensionality in purely kinetic models render them practically intractable for most relevant purposes.

        We consider a $\delta\!f$ decomposition model, with a macroscopic fluid background and microscopic kinetic correction, both fully coupled to each other. A similar manner of discretization is proposed to that used in the recent \texttt{STRUPHY} code \cite{Holderied_Possanner_Wang_2021, Holderied_2022, Li_et_al_2023} with a finite-element model for the background and a pseudo-particle/PiC model for the correction.

        The fluid background satisfies the full, non-linear, resistive, compressible, Hall MHD equations. \cite{Laakmann_Hu_Farrell_2022} introduces finite-element(-in-space) implicit timesteppers for the incompressible analogue to this system with structure-preserving (SP) properties in the ideal case, alongside parameter-robust preconditioners. We show that these timesteppers can derive from a finite-element-in-time (FET) (and finite-element-in-space) interpretation. The benefits of this reformulation are discussed, including the derivation of timesteppers that are higher order in time, and the quantifiable dissipative SP properties in the non-ideal, resistive case.
        
        We discuss possible options for extending this FET approach to timesteppers for the compressible case.

        The kinetic corrections satisfy linearized Boltzmann equations. Using a Lénard--Bernstein collision operator, these take Fokker--Planck-like forms \cite{Fokker_1914, Planck_1917} wherein pseudo-particles in the numerical model obey the neoclassical transport equations, with particle-independent Brownian drift terms. This offers a rigorous methodology for incorporating collisions into the particle transport model, without coupling the equations of motions for each particle.
        
        Works by Chen, Chacón et al. \cite{Chen_Chacón_Barnes_2011, Chacón_Chen_Barnes_2013, Chen_Chacón_2014, Chen_Chacón_2015} have developed structure-preserving particle pushers for neoclassical transport in the Vlasov equations, derived from Crank--Nicolson integrators. We show these too can can derive from a FET interpretation, similarly offering potential extensions to higher-order-in-time particle pushers. The FET formulation is used also to consider how the stochastic drift terms can be incorporated into the pushers. Stochastic gyrokinetic expansions are also discussed.

        Different options for the numerical implementation of these schemes are considered.

        Due to the efficacy of FET in the development of SP timesteppers for both the fluid and kinetic component, we hope this approach will prove effective in the future for developing SP timesteppers for the full hybrid model. We hope this will give us the opportunity to incorporate previously inaccessible kinetic effects into the highly effective, modern, finite-element MHD models.
    \end{abstract}
    
    
    \newpage
    \tableofcontents
    
    
    \newpage
    \pagenumbering{arabic}
    %\linenumbers\renewcommand\thelinenumber{\color{black!50}\arabic{linenumber}}
            \input{0 - introduction/main.tex}
        \part{Research}
            \input{1 - low-noise PiC models/main.tex}
            \input{2 - kinetic component/main.tex}
            \input{3 - fluid component/main.tex}
            \input{4 - numerical implementation/main.tex}
        \part{Project Overview}
            \input{5 - research plan/main.tex}
            \input{6 - summary/main.tex}
    
    
    %\section{}
    \newpage
    \pagenumbering{gobble}
        \printbibliography


    \newpage
    \pagenumbering{roman}
    \appendix
        \part{Appendices}
            \input{8 - Hilbert complexes/main.tex}
            \input{9 - weak conservation proofs/main.tex}
\end{document}

            \documentclass[12pt, a4paper]{report}

\input{template/main.tex}

\title{\BA{Title in Progress...}}
\author{Boris Andrews}
\affil{Mathematical Institute, University of Oxford}
\date{\today}


\begin{document}
    \pagenumbering{gobble}
    \maketitle
    
    
    \begin{abstract}
        Magnetic confinement reactors---in particular tokamaks---offer one of the most promising options for achieving practical nuclear fusion, with the potential to provide virtually limitless, clean energy. The theoretical and numerical modeling of tokamak plasmas is simultaneously an essential component of effective reactor design, and a great research barrier. Tokamak operational conditions exhibit comparatively low Knudsen numbers. Kinetic effects, including kinetic waves and instabilities, Landau damping, bump-on-tail instabilities and more, are therefore highly influential in tokamak plasma dynamics. Purely fluid models are inherently incapable of capturing these effects, whereas the high dimensionality in purely kinetic models render them practically intractable for most relevant purposes.

        We consider a $\delta\!f$ decomposition model, with a macroscopic fluid background and microscopic kinetic correction, both fully coupled to each other. A similar manner of discretization is proposed to that used in the recent \texttt{STRUPHY} code \cite{Holderied_Possanner_Wang_2021, Holderied_2022, Li_et_al_2023} with a finite-element model for the background and a pseudo-particle/PiC model for the correction.

        The fluid background satisfies the full, non-linear, resistive, compressible, Hall MHD equations. \cite{Laakmann_Hu_Farrell_2022} introduces finite-element(-in-space) implicit timesteppers for the incompressible analogue to this system with structure-preserving (SP) properties in the ideal case, alongside parameter-robust preconditioners. We show that these timesteppers can derive from a finite-element-in-time (FET) (and finite-element-in-space) interpretation. The benefits of this reformulation are discussed, including the derivation of timesteppers that are higher order in time, and the quantifiable dissipative SP properties in the non-ideal, resistive case.
        
        We discuss possible options for extending this FET approach to timesteppers for the compressible case.

        The kinetic corrections satisfy linearized Boltzmann equations. Using a Lénard--Bernstein collision operator, these take Fokker--Planck-like forms \cite{Fokker_1914, Planck_1917} wherein pseudo-particles in the numerical model obey the neoclassical transport equations, with particle-independent Brownian drift terms. This offers a rigorous methodology for incorporating collisions into the particle transport model, without coupling the equations of motions for each particle.
        
        Works by Chen, Chacón et al. \cite{Chen_Chacón_Barnes_2011, Chacón_Chen_Barnes_2013, Chen_Chacón_2014, Chen_Chacón_2015} have developed structure-preserving particle pushers for neoclassical transport in the Vlasov equations, derived from Crank--Nicolson integrators. We show these too can can derive from a FET interpretation, similarly offering potential extensions to higher-order-in-time particle pushers. The FET formulation is used also to consider how the stochastic drift terms can be incorporated into the pushers. Stochastic gyrokinetic expansions are also discussed.

        Different options for the numerical implementation of these schemes are considered.

        Due to the efficacy of FET in the development of SP timesteppers for both the fluid and kinetic component, we hope this approach will prove effective in the future for developing SP timesteppers for the full hybrid model. We hope this will give us the opportunity to incorporate previously inaccessible kinetic effects into the highly effective, modern, finite-element MHD models.
    \end{abstract}
    
    
    \newpage
    \tableofcontents
    
    
    \newpage
    \pagenumbering{arabic}
    %\linenumbers\renewcommand\thelinenumber{\color{black!50}\arabic{linenumber}}
            \input{0 - introduction/main.tex}
        \part{Research}
            \input{1 - low-noise PiC models/main.tex}
            \input{2 - kinetic component/main.tex}
            \input{3 - fluid component/main.tex}
            \input{4 - numerical implementation/main.tex}
        \part{Project Overview}
            \input{5 - research plan/main.tex}
            \input{6 - summary/main.tex}
    
    
    %\section{}
    \newpage
    \pagenumbering{gobble}
        \printbibliography


    \newpage
    \pagenumbering{roman}
    \appendix
        \part{Appendices}
            \input{8 - Hilbert complexes/main.tex}
            \input{9 - weak conservation proofs/main.tex}
\end{document}

            \documentclass[12pt, a4paper]{report}

\input{template/main.tex}

\title{\BA{Title in Progress...}}
\author{Boris Andrews}
\affil{Mathematical Institute, University of Oxford}
\date{\today}


\begin{document}
    \pagenumbering{gobble}
    \maketitle
    
    
    \begin{abstract}
        Magnetic confinement reactors---in particular tokamaks---offer one of the most promising options for achieving practical nuclear fusion, with the potential to provide virtually limitless, clean energy. The theoretical and numerical modeling of tokamak plasmas is simultaneously an essential component of effective reactor design, and a great research barrier. Tokamak operational conditions exhibit comparatively low Knudsen numbers. Kinetic effects, including kinetic waves and instabilities, Landau damping, bump-on-tail instabilities and more, are therefore highly influential in tokamak plasma dynamics. Purely fluid models are inherently incapable of capturing these effects, whereas the high dimensionality in purely kinetic models render them practically intractable for most relevant purposes.

        We consider a $\delta\!f$ decomposition model, with a macroscopic fluid background and microscopic kinetic correction, both fully coupled to each other. A similar manner of discretization is proposed to that used in the recent \texttt{STRUPHY} code \cite{Holderied_Possanner_Wang_2021, Holderied_2022, Li_et_al_2023} with a finite-element model for the background and a pseudo-particle/PiC model for the correction.

        The fluid background satisfies the full, non-linear, resistive, compressible, Hall MHD equations. \cite{Laakmann_Hu_Farrell_2022} introduces finite-element(-in-space) implicit timesteppers for the incompressible analogue to this system with structure-preserving (SP) properties in the ideal case, alongside parameter-robust preconditioners. We show that these timesteppers can derive from a finite-element-in-time (FET) (and finite-element-in-space) interpretation. The benefits of this reformulation are discussed, including the derivation of timesteppers that are higher order in time, and the quantifiable dissipative SP properties in the non-ideal, resistive case.
        
        We discuss possible options for extending this FET approach to timesteppers for the compressible case.

        The kinetic corrections satisfy linearized Boltzmann equations. Using a Lénard--Bernstein collision operator, these take Fokker--Planck-like forms \cite{Fokker_1914, Planck_1917} wherein pseudo-particles in the numerical model obey the neoclassical transport equations, with particle-independent Brownian drift terms. This offers a rigorous methodology for incorporating collisions into the particle transport model, without coupling the equations of motions for each particle.
        
        Works by Chen, Chacón et al. \cite{Chen_Chacón_Barnes_2011, Chacón_Chen_Barnes_2013, Chen_Chacón_2014, Chen_Chacón_2015} have developed structure-preserving particle pushers for neoclassical transport in the Vlasov equations, derived from Crank--Nicolson integrators. We show these too can can derive from a FET interpretation, similarly offering potential extensions to higher-order-in-time particle pushers. The FET formulation is used also to consider how the stochastic drift terms can be incorporated into the pushers. Stochastic gyrokinetic expansions are also discussed.

        Different options for the numerical implementation of these schemes are considered.

        Due to the efficacy of FET in the development of SP timesteppers for both the fluid and kinetic component, we hope this approach will prove effective in the future for developing SP timesteppers for the full hybrid model. We hope this will give us the opportunity to incorporate previously inaccessible kinetic effects into the highly effective, modern, finite-element MHD models.
    \end{abstract}
    
    
    \newpage
    \tableofcontents
    
    
    \newpage
    \pagenumbering{arabic}
    %\linenumbers\renewcommand\thelinenumber{\color{black!50}\arabic{linenumber}}
            \input{0 - introduction/main.tex}
        \part{Research}
            \input{1 - low-noise PiC models/main.tex}
            \input{2 - kinetic component/main.tex}
            \input{3 - fluid component/main.tex}
            \input{4 - numerical implementation/main.tex}
        \part{Project Overview}
            \input{5 - research plan/main.tex}
            \input{6 - summary/main.tex}
    
    
    %\section{}
    \newpage
    \pagenumbering{gobble}
        \printbibliography


    \newpage
    \pagenumbering{roman}
    \appendix
        \part{Appendices}
            \input{8 - Hilbert complexes/main.tex}
            \input{9 - weak conservation proofs/main.tex}
\end{document}

        \part{Project Overview}
            \documentclass[12pt, a4paper]{report}

\input{template/main.tex}

\title{\BA{Title in Progress...}}
\author{Boris Andrews}
\affil{Mathematical Institute, University of Oxford}
\date{\today}


\begin{document}
    \pagenumbering{gobble}
    \maketitle
    
    
    \begin{abstract}
        Magnetic confinement reactors---in particular tokamaks---offer one of the most promising options for achieving practical nuclear fusion, with the potential to provide virtually limitless, clean energy. The theoretical and numerical modeling of tokamak plasmas is simultaneously an essential component of effective reactor design, and a great research barrier. Tokamak operational conditions exhibit comparatively low Knudsen numbers. Kinetic effects, including kinetic waves and instabilities, Landau damping, bump-on-tail instabilities and more, are therefore highly influential in tokamak plasma dynamics. Purely fluid models are inherently incapable of capturing these effects, whereas the high dimensionality in purely kinetic models render them practically intractable for most relevant purposes.

        We consider a $\delta\!f$ decomposition model, with a macroscopic fluid background and microscopic kinetic correction, both fully coupled to each other. A similar manner of discretization is proposed to that used in the recent \texttt{STRUPHY} code \cite{Holderied_Possanner_Wang_2021, Holderied_2022, Li_et_al_2023} with a finite-element model for the background and a pseudo-particle/PiC model for the correction.

        The fluid background satisfies the full, non-linear, resistive, compressible, Hall MHD equations. \cite{Laakmann_Hu_Farrell_2022} introduces finite-element(-in-space) implicit timesteppers for the incompressible analogue to this system with structure-preserving (SP) properties in the ideal case, alongside parameter-robust preconditioners. We show that these timesteppers can derive from a finite-element-in-time (FET) (and finite-element-in-space) interpretation. The benefits of this reformulation are discussed, including the derivation of timesteppers that are higher order in time, and the quantifiable dissipative SP properties in the non-ideal, resistive case.
        
        We discuss possible options for extending this FET approach to timesteppers for the compressible case.

        The kinetic corrections satisfy linearized Boltzmann equations. Using a Lénard--Bernstein collision operator, these take Fokker--Planck-like forms \cite{Fokker_1914, Planck_1917} wherein pseudo-particles in the numerical model obey the neoclassical transport equations, with particle-independent Brownian drift terms. This offers a rigorous methodology for incorporating collisions into the particle transport model, without coupling the equations of motions for each particle.
        
        Works by Chen, Chacón et al. \cite{Chen_Chacón_Barnes_2011, Chacón_Chen_Barnes_2013, Chen_Chacón_2014, Chen_Chacón_2015} have developed structure-preserving particle pushers for neoclassical transport in the Vlasov equations, derived from Crank--Nicolson integrators. We show these too can can derive from a FET interpretation, similarly offering potential extensions to higher-order-in-time particle pushers. The FET formulation is used also to consider how the stochastic drift terms can be incorporated into the pushers. Stochastic gyrokinetic expansions are also discussed.

        Different options for the numerical implementation of these schemes are considered.

        Due to the efficacy of FET in the development of SP timesteppers for both the fluid and kinetic component, we hope this approach will prove effective in the future for developing SP timesteppers for the full hybrid model. We hope this will give us the opportunity to incorporate previously inaccessible kinetic effects into the highly effective, modern, finite-element MHD models.
    \end{abstract}
    
    
    \newpage
    \tableofcontents
    
    
    \newpage
    \pagenumbering{arabic}
    %\linenumbers\renewcommand\thelinenumber{\color{black!50}\arabic{linenumber}}
            \input{0 - introduction/main.tex}
        \part{Research}
            \input{1 - low-noise PiC models/main.tex}
            \input{2 - kinetic component/main.tex}
            \input{3 - fluid component/main.tex}
            \input{4 - numerical implementation/main.tex}
        \part{Project Overview}
            \input{5 - research plan/main.tex}
            \input{6 - summary/main.tex}
    
    
    %\section{}
    \newpage
    \pagenumbering{gobble}
        \printbibliography


    \newpage
    \pagenumbering{roman}
    \appendix
        \part{Appendices}
            \input{8 - Hilbert complexes/main.tex}
            \input{9 - weak conservation proofs/main.tex}
\end{document}

            \documentclass[12pt, a4paper]{report}

\input{template/main.tex}

\title{\BA{Title in Progress...}}
\author{Boris Andrews}
\affil{Mathematical Institute, University of Oxford}
\date{\today}


\begin{document}
    \pagenumbering{gobble}
    \maketitle
    
    
    \begin{abstract}
        Magnetic confinement reactors---in particular tokamaks---offer one of the most promising options for achieving practical nuclear fusion, with the potential to provide virtually limitless, clean energy. The theoretical and numerical modeling of tokamak plasmas is simultaneously an essential component of effective reactor design, and a great research barrier. Tokamak operational conditions exhibit comparatively low Knudsen numbers. Kinetic effects, including kinetic waves and instabilities, Landau damping, bump-on-tail instabilities and more, are therefore highly influential in tokamak plasma dynamics. Purely fluid models are inherently incapable of capturing these effects, whereas the high dimensionality in purely kinetic models render them practically intractable for most relevant purposes.

        We consider a $\delta\!f$ decomposition model, with a macroscopic fluid background and microscopic kinetic correction, both fully coupled to each other. A similar manner of discretization is proposed to that used in the recent \texttt{STRUPHY} code \cite{Holderied_Possanner_Wang_2021, Holderied_2022, Li_et_al_2023} with a finite-element model for the background and a pseudo-particle/PiC model for the correction.

        The fluid background satisfies the full, non-linear, resistive, compressible, Hall MHD equations. \cite{Laakmann_Hu_Farrell_2022} introduces finite-element(-in-space) implicit timesteppers for the incompressible analogue to this system with structure-preserving (SP) properties in the ideal case, alongside parameter-robust preconditioners. We show that these timesteppers can derive from a finite-element-in-time (FET) (and finite-element-in-space) interpretation. The benefits of this reformulation are discussed, including the derivation of timesteppers that are higher order in time, and the quantifiable dissipative SP properties in the non-ideal, resistive case.
        
        We discuss possible options for extending this FET approach to timesteppers for the compressible case.

        The kinetic corrections satisfy linearized Boltzmann equations. Using a Lénard--Bernstein collision operator, these take Fokker--Planck-like forms \cite{Fokker_1914, Planck_1917} wherein pseudo-particles in the numerical model obey the neoclassical transport equations, with particle-independent Brownian drift terms. This offers a rigorous methodology for incorporating collisions into the particle transport model, without coupling the equations of motions for each particle.
        
        Works by Chen, Chacón et al. \cite{Chen_Chacón_Barnes_2011, Chacón_Chen_Barnes_2013, Chen_Chacón_2014, Chen_Chacón_2015} have developed structure-preserving particle pushers for neoclassical transport in the Vlasov equations, derived from Crank--Nicolson integrators. We show these too can can derive from a FET interpretation, similarly offering potential extensions to higher-order-in-time particle pushers. The FET formulation is used also to consider how the stochastic drift terms can be incorporated into the pushers. Stochastic gyrokinetic expansions are also discussed.

        Different options for the numerical implementation of these schemes are considered.

        Due to the efficacy of FET in the development of SP timesteppers for both the fluid and kinetic component, we hope this approach will prove effective in the future for developing SP timesteppers for the full hybrid model. We hope this will give us the opportunity to incorporate previously inaccessible kinetic effects into the highly effective, modern, finite-element MHD models.
    \end{abstract}
    
    
    \newpage
    \tableofcontents
    
    
    \newpage
    \pagenumbering{arabic}
    %\linenumbers\renewcommand\thelinenumber{\color{black!50}\arabic{linenumber}}
            \input{0 - introduction/main.tex}
        \part{Research}
            \input{1 - low-noise PiC models/main.tex}
            \input{2 - kinetic component/main.tex}
            \input{3 - fluid component/main.tex}
            \input{4 - numerical implementation/main.tex}
        \part{Project Overview}
            \input{5 - research plan/main.tex}
            \input{6 - summary/main.tex}
    
    
    %\section{}
    \newpage
    \pagenumbering{gobble}
        \printbibliography


    \newpage
    \pagenumbering{roman}
    \appendix
        \part{Appendices}
            \input{8 - Hilbert complexes/main.tex}
            \input{9 - weak conservation proofs/main.tex}
\end{document}

    
    
    %\section{}
    \newpage
    \pagenumbering{gobble}
        \printbibliography


    \newpage
    \pagenumbering{roman}
    \appendix
        \part{Appendices}
            \documentclass[12pt, a4paper]{report}

\input{template/main.tex}

\title{\BA{Title in Progress...}}
\author{Boris Andrews}
\affil{Mathematical Institute, University of Oxford}
\date{\today}


\begin{document}
    \pagenumbering{gobble}
    \maketitle
    
    
    \begin{abstract}
        Magnetic confinement reactors---in particular tokamaks---offer one of the most promising options for achieving practical nuclear fusion, with the potential to provide virtually limitless, clean energy. The theoretical and numerical modeling of tokamak plasmas is simultaneously an essential component of effective reactor design, and a great research barrier. Tokamak operational conditions exhibit comparatively low Knudsen numbers. Kinetic effects, including kinetic waves and instabilities, Landau damping, bump-on-tail instabilities and more, are therefore highly influential in tokamak plasma dynamics. Purely fluid models are inherently incapable of capturing these effects, whereas the high dimensionality in purely kinetic models render them practically intractable for most relevant purposes.

        We consider a $\delta\!f$ decomposition model, with a macroscopic fluid background and microscopic kinetic correction, both fully coupled to each other. A similar manner of discretization is proposed to that used in the recent \texttt{STRUPHY} code \cite{Holderied_Possanner_Wang_2021, Holderied_2022, Li_et_al_2023} with a finite-element model for the background and a pseudo-particle/PiC model for the correction.

        The fluid background satisfies the full, non-linear, resistive, compressible, Hall MHD equations. \cite{Laakmann_Hu_Farrell_2022} introduces finite-element(-in-space) implicit timesteppers for the incompressible analogue to this system with structure-preserving (SP) properties in the ideal case, alongside parameter-robust preconditioners. We show that these timesteppers can derive from a finite-element-in-time (FET) (and finite-element-in-space) interpretation. The benefits of this reformulation are discussed, including the derivation of timesteppers that are higher order in time, and the quantifiable dissipative SP properties in the non-ideal, resistive case.
        
        We discuss possible options for extending this FET approach to timesteppers for the compressible case.

        The kinetic corrections satisfy linearized Boltzmann equations. Using a Lénard--Bernstein collision operator, these take Fokker--Planck-like forms \cite{Fokker_1914, Planck_1917} wherein pseudo-particles in the numerical model obey the neoclassical transport equations, with particle-independent Brownian drift terms. This offers a rigorous methodology for incorporating collisions into the particle transport model, without coupling the equations of motions for each particle.
        
        Works by Chen, Chacón et al. \cite{Chen_Chacón_Barnes_2011, Chacón_Chen_Barnes_2013, Chen_Chacón_2014, Chen_Chacón_2015} have developed structure-preserving particle pushers for neoclassical transport in the Vlasov equations, derived from Crank--Nicolson integrators. We show these too can can derive from a FET interpretation, similarly offering potential extensions to higher-order-in-time particle pushers. The FET formulation is used also to consider how the stochastic drift terms can be incorporated into the pushers. Stochastic gyrokinetic expansions are also discussed.

        Different options for the numerical implementation of these schemes are considered.

        Due to the efficacy of FET in the development of SP timesteppers for both the fluid and kinetic component, we hope this approach will prove effective in the future for developing SP timesteppers for the full hybrid model. We hope this will give us the opportunity to incorporate previously inaccessible kinetic effects into the highly effective, modern, finite-element MHD models.
    \end{abstract}
    
    
    \newpage
    \tableofcontents
    
    
    \newpage
    \pagenumbering{arabic}
    %\linenumbers\renewcommand\thelinenumber{\color{black!50}\arabic{linenumber}}
            \input{0 - introduction/main.tex}
        \part{Research}
            \input{1 - low-noise PiC models/main.tex}
            \input{2 - kinetic component/main.tex}
            \input{3 - fluid component/main.tex}
            \input{4 - numerical implementation/main.tex}
        \part{Project Overview}
            \input{5 - research plan/main.tex}
            \input{6 - summary/main.tex}
    
    
    %\section{}
    \newpage
    \pagenumbering{gobble}
        \printbibliography


    \newpage
    \pagenumbering{roman}
    \appendix
        \part{Appendices}
            \input{8 - Hilbert complexes/main.tex}
            \input{9 - weak conservation proofs/main.tex}
\end{document}

            \documentclass[12pt, a4paper]{report}

\input{template/main.tex}

\title{\BA{Title in Progress...}}
\author{Boris Andrews}
\affil{Mathematical Institute, University of Oxford}
\date{\today}


\begin{document}
    \pagenumbering{gobble}
    \maketitle
    
    
    \begin{abstract}
        Magnetic confinement reactors---in particular tokamaks---offer one of the most promising options for achieving practical nuclear fusion, with the potential to provide virtually limitless, clean energy. The theoretical and numerical modeling of tokamak plasmas is simultaneously an essential component of effective reactor design, and a great research barrier. Tokamak operational conditions exhibit comparatively low Knudsen numbers. Kinetic effects, including kinetic waves and instabilities, Landau damping, bump-on-tail instabilities and more, are therefore highly influential in tokamak plasma dynamics. Purely fluid models are inherently incapable of capturing these effects, whereas the high dimensionality in purely kinetic models render them practically intractable for most relevant purposes.

        We consider a $\delta\!f$ decomposition model, with a macroscopic fluid background and microscopic kinetic correction, both fully coupled to each other. A similar manner of discretization is proposed to that used in the recent \texttt{STRUPHY} code \cite{Holderied_Possanner_Wang_2021, Holderied_2022, Li_et_al_2023} with a finite-element model for the background and a pseudo-particle/PiC model for the correction.

        The fluid background satisfies the full, non-linear, resistive, compressible, Hall MHD equations. \cite{Laakmann_Hu_Farrell_2022} introduces finite-element(-in-space) implicit timesteppers for the incompressible analogue to this system with structure-preserving (SP) properties in the ideal case, alongside parameter-robust preconditioners. We show that these timesteppers can derive from a finite-element-in-time (FET) (and finite-element-in-space) interpretation. The benefits of this reformulation are discussed, including the derivation of timesteppers that are higher order in time, and the quantifiable dissipative SP properties in the non-ideal, resistive case.
        
        We discuss possible options for extending this FET approach to timesteppers for the compressible case.

        The kinetic corrections satisfy linearized Boltzmann equations. Using a Lénard--Bernstein collision operator, these take Fokker--Planck-like forms \cite{Fokker_1914, Planck_1917} wherein pseudo-particles in the numerical model obey the neoclassical transport equations, with particle-independent Brownian drift terms. This offers a rigorous methodology for incorporating collisions into the particle transport model, without coupling the equations of motions for each particle.
        
        Works by Chen, Chacón et al. \cite{Chen_Chacón_Barnes_2011, Chacón_Chen_Barnes_2013, Chen_Chacón_2014, Chen_Chacón_2015} have developed structure-preserving particle pushers for neoclassical transport in the Vlasov equations, derived from Crank--Nicolson integrators. We show these too can can derive from a FET interpretation, similarly offering potential extensions to higher-order-in-time particle pushers. The FET formulation is used also to consider how the stochastic drift terms can be incorporated into the pushers. Stochastic gyrokinetic expansions are also discussed.

        Different options for the numerical implementation of these schemes are considered.

        Due to the efficacy of FET in the development of SP timesteppers for both the fluid and kinetic component, we hope this approach will prove effective in the future for developing SP timesteppers for the full hybrid model. We hope this will give us the opportunity to incorporate previously inaccessible kinetic effects into the highly effective, modern, finite-element MHD models.
    \end{abstract}
    
    
    \newpage
    \tableofcontents
    
    
    \newpage
    \pagenumbering{arabic}
    %\linenumbers\renewcommand\thelinenumber{\color{black!50}\arabic{linenumber}}
            \input{0 - introduction/main.tex}
        \part{Research}
            \input{1 - low-noise PiC models/main.tex}
            \input{2 - kinetic component/main.tex}
            \input{3 - fluid component/main.tex}
            \input{4 - numerical implementation/main.tex}
        \part{Project Overview}
            \input{5 - research plan/main.tex}
            \input{6 - summary/main.tex}
    
    
    %\section{}
    \newpage
    \pagenumbering{gobble}
        \printbibliography


    \newpage
    \pagenumbering{roman}
    \appendix
        \part{Appendices}
            \input{8 - Hilbert complexes/main.tex}
            \input{9 - weak conservation proofs/main.tex}
\end{document}

\end{document}

            \documentclass[12pt, a4paper]{report}

\documentclass[12pt, a4paper]{report}

\input{template/main.tex}

\title{\BA{Title in Progress...}}
\author{Boris Andrews}
\affil{Mathematical Institute, University of Oxford}
\date{\today}


\begin{document}
    \pagenumbering{gobble}
    \maketitle
    
    
    \begin{abstract}
        Magnetic confinement reactors---in particular tokamaks---offer one of the most promising options for achieving practical nuclear fusion, with the potential to provide virtually limitless, clean energy. The theoretical and numerical modeling of tokamak plasmas is simultaneously an essential component of effective reactor design, and a great research barrier. Tokamak operational conditions exhibit comparatively low Knudsen numbers. Kinetic effects, including kinetic waves and instabilities, Landau damping, bump-on-tail instabilities and more, are therefore highly influential in tokamak plasma dynamics. Purely fluid models are inherently incapable of capturing these effects, whereas the high dimensionality in purely kinetic models render them practically intractable for most relevant purposes.

        We consider a $\delta\!f$ decomposition model, with a macroscopic fluid background and microscopic kinetic correction, both fully coupled to each other. A similar manner of discretization is proposed to that used in the recent \texttt{STRUPHY} code \cite{Holderied_Possanner_Wang_2021, Holderied_2022, Li_et_al_2023} with a finite-element model for the background and a pseudo-particle/PiC model for the correction.

        The fluid background satisfies the full, non-linear, resistive, compressible, Hall MHD equations. \cite{Laakmann_Hu_Farrell_2022} introduces finite-element(-in-space) implicit timesteppers for the incompressible analogue to this system with structure-preserving (SP) properties in the ideal case, alongside parameter-robust preconditioners. We show that these timesteppers can derive from a finite-element-in-time (FET) (and finite-element-in-space) interpretation. The benefits of this reformulation are discussed, including the derivation of timesteppers that are higher order in time, and the quantifiable dissipative SP properties in the non-ideal, resistive case.
        
        We discuss possible options for extending this FET approach to timesteppers for the compressible case.

        The kinetic corrections satisfy linearized Boltzmann equations. Using a Lénard--Bernstein collision operator, these take Fokker--Planck-like forms \cite{Fokker_1914, Planck_1917} wherein pseudo-particles in the numerical model obey the neoclassical transport equations, with particle-independent Brownian drift terms. This offers a rigorous methodology for incorporating collisions into the particle transport model, without coupling the equations of motions for each particle.
        
        Works by Chen, Chacón et al. \cite{Chen_Chacón_Barnes_2011, Chacón_Chen_Barnes_2013, Chen_Chacón_2014, Chen_Chacón_2015} have developed structure-preserving particle pushers for neoclassical transport in the Vlasov equations, derived from Crank--Nicolson integrators. We show these too can can derive from a FET interpretation, similarly offering potential extensions to higher-order-in-time particle pushers. The FET formulation is used also to consider how the stochastic drift terms can be incorporated into the pushers. Stochastic gyrokinetic expansions are also discussed.

        Different options for the numerical implementation of these schemes are considered.

        Due to the efficacy of FET in the development of SP timesteppers for both the fluid and kinetic component, we hope this approach will prove effective in the future for developing SP timesteppers for the full hybrid model. We hope this will give us the opportunity to incorporate previously inaccessible kinetic effects into the highly effective, modern, finite-element MHD models.
    \end{abstract}
    
    
    \newpage
    \tableofcontents
    
    
    \newpage
    \pagenumbering{arabic}
    %\linenumbers\renewcommand\thelinenumber{\color{black!50}\arabic{linenumber}}
            \input{0 - introduction/main.tex}
        \part{Research}
            \input{1 - low-noise PiC models/main.tex}
            \input{2 - kinetic component/main.tex}
            \input{3 - fluid component/main.tex}
            \input{4 - numerical implementation/main.tex}
        \part{Project Overview}
            \input{5 - research plan/main.tex}
            \input{6 - summary/main.tex}
    
    
    %\section{}
    \newpage
    \pagenumbering{gobble}
        \printbibliography


    \newpage
    \pagenumbering{roman}
    \appendix
        \part{Appendices}
            \input{8 - Hilbert complexes/main.tex}
            \input{9 - weak conservation proofs/main.tex}
\end{document}


\title{\BA{Title in Progress...}}
\author{Boris Andrews}
\affil{Mathematical Institute, University of Oxford}
\date{\today}


\begin{document}
    \pagenumbering{gobble}
    \maketitle
    
    
    \begin{abstract}
        Magnetic confinement reactors---in particular tokamaks---offer one of the most promising options for achieving practical nuclear fusion, with the potential to provide virtually limitless, clean energy. The theoretical and numerical modeling of tokamak plasmas is simultaneously an essential component of effective reactor design, and a great research barrier. Tokamak operational conditions exhibit comparatively low Knudsen numbers. Kinetic effects, including kinetic waves and instabilities, Landau damping, bump-on-tail instabilities and more, are therefore highly influential in tokamak plasma dynamics. Purely fluid models are inherently incapable of capturing these effects, whereas the high dimensionality in purely kinetic models render them practically intractable for most relevant purposes.

        We consider a $\delta\!f$ decomposition model, with a macroscopic fluid background and microscopic kinetic correction, both fully coupled to each other. A similar manner of discretization is proposed to that used in the recent \texttt{STRUPHY} code \cite{Holderied_Possanner_Wang_2021, Holderied_2022, Li_et_al_2023} with a finite-element model for the background and a pseudo-particle/PiC model for the correction.

        The fluid background satisfies the full, non-linear, resistive, compressible, Hall MHD equations. \cite{Laakmann_Hu_Farrell_2022} introduces finite-element(-in-space) implicit timesteppers for the incompressible analogue to this system with structure-preserving (SP) properties in the ideal case, alongside parameter-robust preconditioners. We show that these timesteppers can derive from a finite-element-in-time (FET) (and finite-element-in-space) interpretation. The benefits of this reformulation are discussed, including the derivation of timesteppers that are higher order in time, and the quantifiable dissipative SP properties in the non-ideal, resistive case.
        
        We discuss possible options for extending this FET approach to timesteppers for the compressible case.

        The kinetic corrections satisfy linearized Boltzmann equations. Using a Lénard--Bernstein collision operator, these take Fokker--Planck-like forms \cite{Fokker_1914, Planck_1917} wherein pseudo-particles in the numerical model obey the neoclassical transport equations, with particle-independent Brownian drift terms. This offers a rigorous methodology for incorporating collisions into the particle transport model, without coupling the equations of motions for each particle.
        
        Works by Chen, Chacón et al. \cite{Chen_Chacón_Barnes_2011, Chacón_Chen_Barnes_2013, Chen_Chacón_2014, Chen_Chacón_2015} have developed structure-preserving particle pushers for neoclassical transport in the Vlasov equations, derived from Crank--Nicolson integrators. We show these too can can derive from a FET interpretation, similarly offering potential extensions to higher-order-in-time particle pushers. The FET formulation is used also to consider how the stochastic drift terms can be incorporated into the pushers. Stochastic gyrokinetic expansions are also discussed.

        Different options for the numerical implementation of these schemes are considered.

        Due to the efficacy of FET in the development of SP timesteppers for both the fluid and kinetic component, we hope this approach will prove effective in the future for developing SP timesteppers for the full hybrid model. We hope this will give us the opportunity to incorporate previously inaccessible kinetic effects into the highly effective, modern, finite-element MHD models.
    \end{abstract}
    
    
    \newpage
    \tableofcontents
    
    
    \newpage
    \pagenumbering{arabic}
    %\linenumbers\renewcommand\thelinenumber{\color{black!50}\arabic{linenumber}}
            \documentclass[12pt, a4paper]{report}

\input{template/main.tex}

\title{\BA{Title in Progress...}}
\author{Boris Andrews}
\affil{Mathematical Institute, University of Oxford}
\date{\today}


\begin{document}
    \pagenumbering{gobble}
    \maketitle
    
    
    \begin{abstract}
        Magnetic confinement reactors---in particular tokamaks---offer one of the most promising options for achieving practical nuclear fusion, with the potential to provide virtually limitless, clean energy. The theoretical and numerical modeling of tokamak plasmas is simultaneously an essential component of effective reactor design, and a great research barrier. Tokamak operational conditions exhibit comparatively low Knudsen numbers. Kinetic effects, including kinetic waves and instabilities, Landau damping, bump-on-tail instabilities and more, are therefore highly influential in tokamak plasma dynamics. Purely fluid models are inherently incapable of capturing these effects, whereas the high dimensionality in purely kinetic models render them practically intractable for most relevant purposes.

        We consider a $\delta\!f$ decomposition model, with a macroscopic fluid background and microscopic kinetic correction, both fully coupled to each other. A similar manner of discretization is proposed to that used in the recent \texttt{STRUPHY} code \cite{Holderied_Possanner_Wang_2021, Holderied_2022, Li_et_al_2023} with a finite-element model for the background and a pseudo-particle/PiC model for the correction.

        The fluid background satisfies the full, non-linear, resistive, compressible, Hall MHD equations. \cite{Laakmann_Hu_Farrell_2022} introduces finite-element(-in-space) implicit timesteppers for the incompressible analogue to this system with structure-preserving (SP) properties in the ideal case, alongside parameter-robust preconditioners. We show that these timesteppers can derive from a finite-element-in-time (FET) (and finite-element-in-space) interpretation. The benefits of this reformulation are discussed, including the derivation of timesteppers that are higher order in time, and the quantifiable dissipative SP properties in the non-ideal, resistive case.
        
        We discuss possible options for extending this FET approach to timesteppers for the compressible case.

        The kinetic corrections satisfy linearized Boltzmann equations. Using a Lénard--Bernstein collision operator, these take Fokker--Planck-like forms \cite{Fokker_1914, Planck_1917} wherein pseudo-particles in the numerical model obey the neoclassical transport equations, with particle-independent Brownian drift terms. This offers a rigorous methodology for incorporating collisions into the particle transport model, without coupling the equations of motions for each particle.
        
        Works by Chen, Chacón et al. \cite{Chen_Chacón_Barnes_2011, Chacón_Chen_Barnes_2013, Chen_Chacón_2014, Chen_Chacón_2015} have developed structure-preserving particle pushers for neoclassical transport in the Vlasov equations, derived from Crank--Nicolson integrators. We show these too can can derive from a FET interpretation, similarly offering potential extensions to higher-order-in-time particle pushers. The FET formulation is used also to consider how the stochastic drift terms can be incorporated into the pushers. Stochastic gyrokinetic expansions are also discussed.

        Different options for the numerical implementation of these schemes are considered.

        Due to the efficacy of FET in the development of SP timesteppers for both the fluid and kinetic component, we hope this approach will prove effective in the future for developing SP timesteppers for the full hybrid model. We hope this will give us the opportunity to incorporate previously inaccessible kinetic effects into the highly effective, modern, finite-element MHD models.
    \end{abstract}
    
    
    \newpage
    \tableofcontents
    
    
    \newpage
    \pagenumbering{arabic}
    %\linenumbers\renewcommand\thelinenumber{\color{black!50}\arabic{linenumber}}
            \input{0 - introduction/main.tex}
        \part{Research}
            \input{1 - low-noise PiC models/main.tex}
            \input{2 - kinetic component/main.tex}
            \input{3 - fluid component/main.tex}
            \input{4 - numerical implementation/main.tex}
        \part{Project Overview}
            \input{5 - research plan/main.tex}
            \input{6 - summary/main.tex}
    
    
    %\section{}
    \newpage
    \pagenumbering{gobble}
        \printbibliography


    \newpage
    \pagenumbering{roman}
    \appendix
        \part{Appendices}
            \input{8 - Hilbert complexes/main.tex}
            \input{9 - weak conservation proofs/main.tex}
\end{document}

        \part{Research}
            \documentclass[12pt, a4paper]{report}

\input{template/main.tex}

\title{\BA{Title in Progress...}}
\author{Boris Andrews}
\affil{Mathematical Institute, University of Oxford}
\date{\today}


\begin{document}
    \pagenumbering{gobble}
    \maketitle
    
    
    \begin{abstract}
        Magnetic confinement reactors---in particular tokamaks---offer one of the most promising options for achieving practical nuclear fusion, with the potential to provide virtually limitless, clean energy. The theoretical and numerical modeling of tokamak plasmas is simultaneously an essential component of effective reactor design, and a great research barrier. Tokamak operational conditions exhibit comparatively low Knudsen numbers. Kinetic effects, including kinetic waves and instabilities, Landau damping, bump-on-tail instabilities and more, are therefore highly influential in tokamak plasma dynamics. Purely fluid models are inherently incapable of capturing these effects, whereas the high dimensionality in purely kinetic models render them practically intractable for most relevant purposes.

        We consider a $\delta\!f$ decomposition model, with a macroscopic fluid background and microscopic kinetic correction, both fully coupled to each other. A similar manner of discretization is proposed to that used in the recent \texttt{STRUPHY} code \cite{Holderied_Possanner_Wang_2021, Holderied_2022, Li_et_al_2023} with a finite-element model for the background and a pseudo-particle/PiC model for the correction.

        The fluid background satisfies the full, non-linear, resistive, compressible, Hall MHD equations. \cite{Laakmann_Hu_Farrell_2022} introduces finite-element(-in-space) implicit timesteppers for the incompressible analogue to this system with structure-preserving (SP) properties in the ideal case, alongside parameter-robust preconditioners. We show that these timesteppers can derive from a finite-element-in-time (FET) (and finite-element-in-space) interpretation. The benefits of this reformulation are discussed, including the derivation of timesteppers that are higher order in time, and the quantifiable dissipative SP properties in the non-ideal, resistive case.
        
        We discuss possible options for extending this FET approach to timesteppers for the compressible case.

        The kinetic corrections satisfy linearized Boltzmann equations. Using a Lénard--Bernstein collision operator, these take Fokker--Planck-like forms \cite{Fokker_1914, Planck_1917} wherein pseudo-particles in the numerical model obey the neoclassical transport equations, with particle-independent Brownian drift terms. This offers a rigorous methodology for incorporating collisions into the particle transport model, without coupling the equations of motions for each particle.
        
        Works by Chen, Chacón et al. \cite{Chen_Chacón_Barnes_2011, Chacón_Chen_Barnes_2013, Chen_Chacón_2014, Chen_Chacón_2015} have developed structure-preserving particle pushers for neoclassical transport in the Vlasov equations, derived from Crank--Nicolson integrators. We show these too can can derive from a FET interpretation, similarly offering potential extensions to higher-order-in-time particle pushers. The FET formulation is used also to consider how the stochastic drift terms can be incorporated into the pushers. Stochastic gyrokinetic expansions are also discussed.

        Different options for the numerical implementation of these schemes are considered.

        Due to the efficacy of FET in the development of SP timesteppers for both the fluid and kinetic component, we hope this approach will prove effective in the future for developing SP timesteppers for the full hybrid model. We hope this will give us the opportunity to incorporate previously inaccessible kinetic effects into the highly effective, modern, finite-element MHD models.
    \end{abstract}
    
    
    \newpage
    \tableofcontents
    
    
    \newpage
    \pagenumbering{arabic}
    %\linenumbers\renewcommand\thelinenumber{\color{black!50}\arabic{linenumber}}
            \input{0 - introduction/main.tex}
        \part{Research}
            \input{1 - low-noise PiC models/main.tex}
            \input{2 - kinetic component/main.tex}
            \input{3 - fluid component/main.tex}
            \input{4 - numerical implementation/main.tex}
        \part{Project Overview}
            \input{5 - research plan/main.tex}
            \input{6 - summary/main.tex}
    
    
    %\section{}
    \newpage
    \pagenumbering{gobble}
        \printbibliography


    \newpage
    \pagenumbering{roman}
    \appendix
        \part{Appendices}
            \input{8 - Hilbert complexes/main.tex}
            \input{9 - weak conservation proofs/main.tex}
\end{document}

            \documentclass[12pt, a4paper]{report}

\input{template/main.tex}

\title{\BA{Title in Progress...}}
\author{Boris Andrews}
\affil{Mathematical Institute, University of Oxford}
\date{\today}


\begin{document}
    \pagenumbering{gobble}
    \maketitle
    
    
    \begin{abstract}
        Magnetic confinement reactors---in particular tokamaks---offer one of the most promising options for achieving practical nuclear fusion, with the potential to provide virtually limitless, clean energy. The theoretical and numerical modeling of tokamak plasmas is simultaneously an essential component of effective reactor design, and a great research barrier. Tokamak operational conditions exhibit comparatively low Knudsen numbers. Kinetic effects, including kinetic waves and instabilities, Landau damping, bump-on-tail instabilities and more, are therefore highly influential in tokamak plasma dynamics. Purely fluid models are inherently incapable of capturing these effects, whereas the high dimensionality in purely kinetic models render them practically intractable for most relevant purposes.

        We consider a $\delta\!f$ decomposition model, with a macroscopic fluid background and microscopic kinetic correction, both fully coupled to each other. A similar manner of discretization is proposed to that used in the recent \texttt{STRUPHY} code \cite{Holderied_Possanner_Wang_2021, Holderied_2022, Li_et_al_2023} with a finite-element model for the background and a pseudo-particle/PiC model for the correction.

        The fluid background satisfies the full, non-linear, resistive, compressible, Hall MHD equations. \cite{Laakmann_Hu_Farrell_2022} introduces finite-element(-in-space) implicit timesteppers for the incompressible analogue to this system with structure-preserving (SP) properties in the ideal case, alongside parameter-robust preconditioners. We show that these timesteppers can derive from a finite-element-in-time (FET) (and finite-element-in-space) interpretation. The benefits of this reformulation are discussed, including the derivation of timesteppers that are higher order in time, and the quantifiable dissipative SP properties in the non-ideal, resistive case.
        
        We discuss possible options for extending this FET approach to timesteppers for the compressible case.

        The kinetic corrections satisfy linearized Boltzmann equations. Using a Lénard--Bernstein collision operator, these take Fokker--Planck-like forms \cite{Fokker_1914, Planck_1917} wherein pseudo-particles in the numerical model obey the neoclassical transport equations, with particle-independent Brownian drift terms. This offers a rigorous methodology for incorporating collisions into the particle transport model, without coupling the equations of motions for each particle.
        
        Works by Chen, Chacón et al. \cite{Chen_Chacón_Barnes_2011, Chacón_Chen_Barnes_2013, Chen_Chacón_2014, Chen_Chacón_2015} have developed structure-preserving particle pushers for neoclassical transport in the Vlasov equations, derived from Crank--Nicolson integrators. We show these too can can derive from a FET interpretation, similarly offering potential extensions to higher-order-in-time particle pushers. The FET formulation is used also to consider how the stochastic drift terms can be incorporated into the pushers. Stochastic gyrokinetic expansions are also discussed.

        Different options for the numerical implementation of these schemes are considered.

        Due to the efficacy of FET in the development of SP timesteppers for both the fluid and kinetic component, we hope this approach will prove effective in the future for developing SP timesteppers for the full hybrid model. We hope this will give us the opportunity to incorporate previously inaccessible kinetic effects into the highly effective, modern, finite-element MHD models.
    \end{abstract}
    
    
    \newpage
    \tableofcontents
    
    
    \newpage
    \pagenumbering{arabic}
    %\linenumbers\renewcommand\thelinenumber{\color{black!50}\arabic{linenumber}}
            \input{0 - introduction/main.tex}
        \part{Research}
            \input{1 - low-noise PiC models/main.tex}
            \input{2 - kinetic component/main.tex}
            \input{3 - fluid component/main.tex}
            \input{4 - numerical implementation/main.tex}
        \part{Project Overview}
            \input{5 - research plan/main.tex}
            \input{6 - summary/main.tex}
    
    
    %\section{}
    \newpage
    \pagenumbering{gobble}
        \printbibliography


    \newpage
    \pagenumbering{roman}
    \appendix
        \part{Appendices}
            \input{8 - Hilbert complexes/main.tex}
            \input{9 - weak conservation proofs/main.tex}
\end{document}

            \documentclass[12pt, a4paper]{report}

\input{template/main.tex}

\title{\BA{Title in Progress...}}
\author{Boris Andrews}
\affil{Mathematical Institute, University of Oxford}
\date{\today}


\begin{document}
    \pagenumbering{gobble}
    \maketitle
    
    
    \begin{abstract}
        Magnetic confinement reactors---in particular tokamaks---offer one of the most promising options for achieving practical nuclear fusion, with the potential to provide virtually limitless, clean energy. The theoretical and numerical modeling of tokamak plasmas is simultaneously an essential component of effective reactor design, and a great research barrier. Tokamak operational conditions exhibit comparatively low Knudsen numbers. Kinetic effects, including kinetic waves and instabilities, Landau damping, bump-on-tail instabilities and more, are therefore highly influential in tokamak plasma dynamics. Purely fluid models are inherently incapable of capturing these effects, whereas the high dimensionality in purely kinetic models render them practically intractable for most relevant purposes.

        We consider a $\delta\!f$ decomposition model, with a macroscopic fluid background and microscopic kinetic correction, both fully coupled to each other. A similar manner of discretization is proposed to that used in the recent \texttt{STRUPHY} code \cite{Holderied_Possanner_Wang_2021, Holderied_2022, Li_et_al_2023} with a finite-element model for the background and a pseudo-particle/PiC model for the correction.

        The fluid background satisfies the full, non-linear, resistive, compressible, Hall MHD equations. \cite{Laakmann_Hu_Farrell_2022} introduces finite-element(-in-space) implicit timesteppers for the incompressible analogue to this system with structure-preserving (SP) properties in the ideal case, alongside parameter-robust preconditioners. We show that these timesteppers can derive from a finite-element-in-time (FET) (and finite-element-in-space) interpretation. The benefits of this reformulation are discussed, including the derivation of timesteppers that are higher order in time, and the quantifiable dissipative SP properties in the non-ideal, resistive case.
        
        We discuss possible options for extending this FET approach to timesteppers for the compressible case.

        The kinetic corrections satisfy linearized Boltzmann equations. Using a Lénard--Bernstein collision operator, these take Fokker--Planck-like forms \cite{Fokker_1914, Planck_1917} wherein pseudo-particles in the numerical model obey the neoclassical transport equations, with particle-independent Brownian drift terms. This offers a rigorous methodology for incorporating collisions into the particle transport model, without coupling the equations of motions for each particle.
        
        Works by Chen, Chacón et al. \cite{Chen_Chacón_Barnes_2011, Chacón_Chen_Barnes_2013, Chen_Chacón_2014, Chen_Chacón_2015} have developed structure-preserving particle pushers for neoclassical transport in the Vlasov equations, derived from Crank--Nicolson integrators. We show these too can can derive from a FET interpretation, similarly offering potential extensions to higher-order-in-time particle pushers. The FET formulation is used also to consider how the stochastic drift terms can be incorporated into the pushers. Stochastic gyrokinetic expansions are also discussed.

        Different options for the numerical implementation of these schemes are considered.

        Due to the efficacy of FET in the development of SP timesteppers for both the fluid and kinetic component, we hope this approach will prove effective in the future for developing SP timesteppers for the full hybrid model. We hope this will give us the opportunity to incorporate previously inaccessible kinetic effects into the highly effective, modern, finite-element MHD models.
    \end{abstract}
    
    
    \newpage
    \tableofcontents
    
    
    \newpage
    \pagenumbering{arabic}
    %\linenumbers\renewcommand\thelinenumber{\color{black!50}\arabic{linenumber}}
            \input{0 - introduction/main.tex}
        \part{Research}
            \input{1 - low-noise PiC models/main.tex}
            \input{2 - kinetic component/main.tex}
            \input{3 - fluid component/main.tex}
            \input{4 - numerical implementation/main.tex}
        \part{Project Overview}
            \input{5 - research plan/main.tex}
            \input{6 - summary/main.tex}
    
    
    %\section{}
    \newpage
    \pagenumbering{gobble}
        \printbibliography


    \newpage
    \pagenumbering{roman}
    \appendix
        \part{Appendices}
            \input{8 - Hilbert complexes/main.tex}
            \input{9 - weak conservation proofs/main.tex}
\end{document}

            \documentclass[12pt, a4paper]{report}

\input{template/main.tex}

\title{\BA{Title in Progress...}}
\author{Boris Andrews}
\affil{Mathematical Institute, University of Oxford}
\date{\today}


\begin{document}
    \pagenumbering{gobble}
    \maketitle
    
    
    \begin{abstract}
        Magnetic confinement reactors---in particular tokamaks---offer one of the most promising options for achieving practical nuclear fusion, with the potential to provide virtually limitless, clean energy. The theoretical and numerical modeling of tokamak plasmas is simultaneously an essential component of effective reactor design, and a great research barrier. Tokamak operational conditions exhibit comparatively low Knudsen numbers. Kinetic effects, including kinetic waves and instabilities, Landau damping, bump-on-tail instabilities and more, are therefore highly influential in tokamak plasma dynamics. Purely fluid models are inherently incapable of capturing these effects, whereas the high dimensionality in purely kinetic models render them practically intractable for most relevant purposes.

        We consider a $\delta\!f$ decomposition model, with a macroscopic fluid background and microscopic kinetic correction, both fully coupled to each other. A similar manner of discretization is proposed to that used in the recent \texttt{STRUPHY} code \cite{Holderied_Possanner_Wang_2021, Holderied_2022, Li_et_al_2023} with a finite-element model for the background and a pseudo-particle/PiC model for the correction.

        The fluid background satisfies the full, non-linear, resistive, compressible, Hall MHD equations. \cite{Laakmann_Hu_Farrell_2022} introduces finite-element(-in-space) implicit timesteppers for the incompressible analogue to this system with structure-preserving (SP) properties in the ideal case, alongside parameter-robust preconditioners. We show that these timesteppers can derive from a finite-element-in-time (FET) (and finite-element-in-space) interpretation. The benefits of this reformulation are discussed, including the derivation of timesteppers that are higher order in time, and the quantifiable dissipative SP properties in the non-ideal, resistive case.
        
        We discuss possible options for extending this FET approach to timesteppers for the compressible case.

        The kinetic corrections satisfy linearized Boltzmann equations. Using a Lénard--Bernstein collision operator, these take Fokker--Planck-like forms \cite{Fokker_1914, Planck_1917} wherein pseudo-particles in the numerical model obey the neoclassical transport equations, with particle-independent Brownian drift terms. This offers a rigorous methodology for incorporating collisions into the particle transport model, without coupling the equations of motions for each particle.
        
        Works by Chen, Chacón et al. \cite{Chen_Chacón_Barnes_2011, Chacón_Chen_Barnes_2013, Chen_Chacón_2014, Chen_Chacón_2015} have developed structure-preserving particle pushers for neoclassical transport in the Vlasov equations, derived from Crank--Nicolson integrators. We show these too can can derive from a FET interpretation, similarly offering potential extensions to higher-order-in-time particle pushers. The FET formulation is used also to consider how the stochastic drift terms can be incorporated into the pushers. Stochastic gyrokinetic expansions are also discussed.

        Different options for the numerical implementation of these schemes are considered.

        Due to the efficacy of FET in the development of SP timesteppers for both the fluid and kinetic component, we hope this approach will prove effective in the future for developing SP timesteppers for the full hybrid model. We hope this will give us the opportunity to incorporate previously inaccessible kinetic effects into the highly effective, modern, finite-element MHD models.
    \end{abstract}
    
    
    \newpage
    \tableofcontents
    
    
    \newpage
    \pagenumbering{arabic}
    %\linenumbers\renewcommand\thelinenumber{\color{black!50}\arabic{linenumber}}
            \input{0 - introduction/main.tex}
        \part{Research}
            \input{1 - low-noise PiC models/main.tex}
            \input{2 - kinetic component/main.tex}
            \input{3 - fluid component/main.tex}
            \input{4 - numerical implementation/main.tex}
        \part{Project Overview}
            \input{5 - research plan/main.tex}
            \input{6 - summary/main.tex}
    
    
    %\section{}
    \newpage
    \pagenumbering{gobble}
        \printbibliography


    \newpage
    \pagenumbering{roman}
    \appendix
        \part{Appendices}
            \input{8 - Hilbert complexes/main.tex}
            \input{9 - weak conservation proofs/main.tex}
\end{document}

        \part{Project Overview}
            \documentclass[12pt, a4paper]{report}

\input{template/main.tex}

\title{\BA{Title in Progress...}}
\author{Boris Andrews}
\affil{Mathematical Institute, University of Oxford}
\date{\today}


\begin{document}
    \pagenumbering{gobble}
    \maketitle
    
    
    \begin{abstract}
        Magnetic confinement reactors---in particular tokamaks---offer one of the most promising options for achieving practical nuclear fusion, with the potential to provide virtually limitless, clean energy. The theoretical and numerical modeling of tokamak plasmas is simultaneously an essential component of effective reactor design, and a great research barrier. Tokamak operational conditions exhibit comparatively low Knudsen numbers. Kinetic effects, including kinetic waves and instabilities, Landau damping, bump-on-tail instabilities and more, are therefore highly influential in tokamak plasma dynamics. Purely fluid models are inherently incapable of capturing these effects, whereas the high dimensionality in purely kinetic models render them practically intractable for most relevant purposes.

        We consider a $\delta\!f$ decomposition model, with a macroscopic fluid background and microscopic kinetic correction, both fully coupled to each other. A similar manner of discretization is proposed to that used in the recent \texttt{STRUPHY} code \cite{Holderied_Possanner_Wang_2021, Holderied_2022, Li_et_al_2023} with a finite-element model for the background and a pseudo-particle/PiC model for the correction.

        The fluid background satisfies the full, non-linear, resistive, compressible, Hall MHD equations. \cite{Laakmann_Hu_Farrell_2022} introduces finite-element(-in-space) implicit timesteppers for the incompressible analogue to this system with structure-preserving (SP) properties in the ideal case, alongside parameter-robust preconditioners. We show that these timesteppers can derive from a finite-element-in-time (FET) (and finite-element-in-space) interpretation. The benefits of this reformulation are discussed, including the derivation of timesteppers that are higher order in time, and the quantifiable dissipative SP properties in the non-ideal, resistive case.
        
        We discuss possible options for extending this FET approach to timesteppers for the compressible case.

        The kinetic corrections satisfy linearized Boltzmann equations. Using a Lénard--Bernstein collision operator, these take Fokker--Planck-like forms \cite{Fokker_1914, Planck_1917} wherein pseudo-particles in the numerical model obey the neoclassical transport equations, with particle-independent Brownian drift terms. This offers a rigorous methodology for incorporating collisions into the particle transport model, without coupling the equations of motions for each particle.
        
        Works by Chen, Chacón et al. \cite{Chen_Chacón_Barnes_2011, Chacón_Chen_Barnes_2013, Chen_Chacón_2014, Chen_Chacón_2015} have developed structure-preserving particle pushers for neoclassical transport in the Vlasov equations, derived from Crank--Nicolson integrators. We show these too can can derive from a FET interpretation, similarly offering potential extensions to higher-order-in-time particle pushers. The FET formulation is used also to consider how the stochastic drift terms can be incorporated into the pushers. Stochastic gyrokinetic expansions are also discussed.

        Different options for the numerical implementation of these schemes are considered.

        Due to the efficacy of FET in the development of SP timesteppers for both the fluid and kinetic component, we hope this approach will prove effective in the future for developing SP timesteppers for the full hybrid model. We hope this will give us the opportunity to incorporate previously inaccessible kinetic effects into the highly effective, modern, finite-element MHD models.
    \end{abstract}
    
    
    \newpage
    \tableofcontents
    
    
    \newpage
    \pagenumbering{arabic}
    %\linenumbers\renewcommand\thelinenumber{\color{black!50}\arabic{linenumber}}
            \input{0 - introduction/main.tex}
        \part{Research}
            \input{1 - low-noise PiC models/main.tex}
            \input{2 - kinetic component/main.tex}
            \input{3 - fluid component/main.tex}
            \input{4 - numerical implementation/main.tex}
        \part{Project Overview}
            \input{5 - research plan/main.tex}
            \input{6 - summary/main.tex}
    
    
    %\section{}
    \newpage
    \pagenumbering{gobble}
        \printbibliography


    \newpage
    \pagenumbering{roman}
    \appendix
        \part{Appendices}
            \input{8 - Hilbert complexes/main.tex}
            \input{9 - weak conservation proofs/main.tex}
\end{document}

            \documentclass[12pt, a4paper]{report}

\input{template/main.tex}

\title{\BA{Title in Progress...}}
\author{Boris Andrews}
\affil{Mathematical Institute, University of Oxford}
\date{\today}


\begin{document}
    \pagenumbering{gobble}
    \maketitle
    
    
    \begin{abstract}
        Magnetic confinement reactors---in particular tokamaks---offer one of the most promising options for achieving practical nuclear fusion, with the potential to provide virtually limitless, clean energy. The theoretical and numerical modeling of tokamak plasmas is simultaneously an essential component of effective reactor design, and a great research barrier. Tokamak operational conditions exhibit comparatively low Knudsen numbers. Kinetic effects, including kinetic waves and instabilities, Landau damping, bump-on-tail instabilities and more, are therefore highly influential in tokamak plasma dynamics. Purely fluid models are inherently incapable of capturing these effects, whereas the high dimensionality in purely kinetic models render them practically intractable for most relevant purposes.

        We consider a $\delta\!f$ decomposition model, with a macroscopic fluid background and microscopic kinetic correction, both fully coupled to each other. A similar manner of discretization is proposed to that used in the recent \texttt{STRUPHY} code \cite{Holderied_Possanner_Wang_2021, Holderied_2022, Li_et_al_2023} with a finite-element model for the background and a pseudo-particle/PiC model for the correction.

        The fluid background satisfies the full, non-linear, resistive, compressible, Hall MHD equations. \cite{Laakmann_Hu_Farrell_2022} introduces finite-element(-in-space) implicit timesteppers for the incompressible analogue to this system with structure-preserving (SP) properties in the ideal case, alongside parameter-robust preconditioners. We show that these timesteppers can derive from a finite-element-in-time (FET) (and finite-element-in-space) interpretation. The benefits of this reformulation are discussed, including the derivation of timesteppers that are higher order in time, and the quantifiable dissipative SP properties in the non-ideal, resistive case.
        
        We discuss possible options for extending this FET approach to timesteppers for the compressible case.

        The kinetic corrections satisfy linearized Boltzmann equations. Using a Lénard--Bernstein collision operator, these take Fokker--Planck-like forms \cite{Fokker_1914, Planck_1917} wherein pseudo-particles in the numerical model obey the neoclassical transport equations, with particle-independent Brownian drift terms. This offers a rigorous methodology for incorporating collisions into the particle transport model, without coupling the equations of motions for each particle.
        
        Works by Chen, Chacón et al. \cite{Chen_Chacón_Barnes_2011, Chacón_Chen_Barnes_2013, Chen_Chacón_2014, Chen_Chacón_2015} have developed structure-preserving particle pushers for neoclassical transport in the Vlasov equations, derived from Crank--Nicolson integrators. We show these too can can derive from a FET interpretation, similarly offering potential extensions to higher-order-in-time particle pushers. The FET formulation is used also to consider how the stochastic drift terms can be incorporated into the pushers. Stochastic gyrokinetic expansions are also discussed.

        Different options for the numerical implementation of these schemes are considered.

        Due to the efficacy of FET in the development of SP timesteppers for both the fluid and kinetic component, we hope this approach will prove effective in the future for developing SP timesteppers for the full hybrid model. We hope this will give us the opportunity to incorporate previously inaccessible kinetic effects into the highly effective, modern, finite-element MHD models.
    \end{abstract}
    
    
    \newpage
    \tableofcontents
    
    
    \newpage
    \pagenumbering{arabic}
    %\linenumbers\renewcommand\thelinenumber{\color{black!50}\arabic{linenumber}}
            \input{0 - introduction/main.tex}
        \part{Research}
            \input{1 - low-noise PiC models/main.tex}
            \input{2 - kinetic component/main.tex}
            \input{3 - fluid component/main.tex}
            \input{4 - numerical implementation/main.tex}
        \part{Project Overview}
            \input{5 - research plan/main.tex}
            \input{6 - summary/main.tex}
    
    
    %\section{}
    \newpage
    \pagenumbering{gobble}
        \printbibliography


    \newpage
    \pagenumbering{roman}
    \appendix
        \part{Appendices}
            \input{8 - Hilbert complexes/main.tex}
            \input{9 - weak conservation proofs/main.tex}
\end{document}

    
    
    %\section{}
    \newpage
    \pagenumbering{gobble}
        \printbibliography


    \newpage
    \pagenumbering{roman}
    \appendix
        \part{Appendices}
            \documentclass[12pt, a4paper]{report}

\input{template/main.tex}

\title{\BA{Title in Progress...}}
\author{Boris Andrews}
\affil{Mathematical Institute, University of Oxford}
\date{\today}


\begin{document}
    \pagenumbering{gobble}
    \maketitle
    
    
    \begin{abstract}
        Magnetic confinement reactors---in particular tokamaks---offer one of the most promising options for achieving practical nuclear fusion, with the potential to provide virtually limitless, clean energy. The theoretical and numerical modeling of tokamak plasmas is simultaneously an essential component of effective reactor design, and a great research barrier. Tokamak operational conditions exhibit comparatively low Knudsen numbers. Kinetic effects, including kinetic waves and instabilities, Landau damping, bump-on-tail instabilities and more, are therefore highly influential in tokamak plasma dynamics. Purely fluid models are inherently incapable of capturing these effects, whereas the high dimensionality in purely kinetic models render them practically intractable for most relevant purposes.

        We consider a $\delta\!f$ decomposition model, with a macroscopic fluid background and microscopic kinetic correction, both fully coupled to each other. A similar manner of discretization is proposed to that used in the recent \texttt{STRUPHY} code \cite{Holderied_Possanner_Wang_2021, Holderied_2022, Li_et_al_2023} with a finite-element model for the background and a pseudo-particle/PiC model for the correction.

        The fluid background satisfies the full, non-linear, resistive, compressible, Hall MHD equations. \cite{Laakmann_Hu_Farrell_2022} introduces finite-element(-in-space) implicit timesteppers for the incompressible analogue to this system with structure-preserving (SP) properties in the ideal case, alongside parameter-robust preconditioners. We show that these timesteppers can derive from a finite-element-in-time (FET) (and finite-element-in-space) interpretation. The benefits of this reformulation are discussed, including the derivation of timesteppers that are higher order in time, and the quantifiable dissipative SP properties in the non-ideal, resistive case.
        
        We discuss possible options for extending this FET approach to timesteppers for the compressible case.

        The kinetic corrections satisfy linearized Boltzmann equations. Using a Lénard--Bernstein collision operator, these take Fokker--Planck-like forms \cite{Fokker_1914, Planck_1917} wherein pseudo-particles in the numerical model obey the neoclassical transport equations, with particle-independent Brownian drift terms. This offers a rigorous methodology for incorporating collisions into the particle transport model, without coupling the equations of motions for each particle.
        
        Works by Chen, Chacón et al. \cite{Chen_Chacón_Barnes_2011, Chacón_Chen_Barnes_2013, Chen_Chacón_2014, Chen_Chacón_2015} have developed structure-preserving particle pushers for neoclassical transport in the Vlasov equations, derived from Crank--Nicolson integrators. We show these too can can derive from a FET interpretation, similarly offering potential extensions to higher-order-in-time particle pushers. The FET formulation is used also to consider how the stochastic drift terms can be incorporated into the pushers. Stochastic gyrokinetic expansions are also discussed.

        Different options for the numerical implementation of these schemes are considered.

        Due to the efficacy of FET in the development of SP timesteppers for both the fluid and kinetic component, we hope this approach will prove effective in the future for developing SP timesteppers for the full hybrid model. We hope this will give us the opportunity to incorporate previously inaccessible kinetic effects into the highly effective, modern, finite-element MHD models.
    \end{abstract}
    
    
    \newpage
    \tableofcontents
    
    
    \newpage
    \pagenumbering{arabic}
    %\linenumbers\renewcommand\thelinenumber{\color{black!50}\arabic{linenumber}}
            \input{0 - introduction/main.tex}
        \part{Research}
            \input{1 - low-noise PiC models/main.tex}
            \input{2 - kinetic component/main.tex}
            \input{3 - fluid component/main.tex}
            \input{4 - numerical implementation/main.tex}
        \part{Project Overview}
            \input{5 - research plan/main.tex}
            \input{6 - summary/main.tex}
    
    
    %\section{}
    \newpage
    \pagenumbering{gobble}
        \printbibliography


    \newpage
    \pagenumbering{roman}
    \appendix
        \part{Appendices}
            \input{8 - Hilbert complexes/main.tex}
            \input{9 - weak conservation proofs/main.tex}
\end{document}

            \documentclass[12pt, a4paper]{report}

\input{template/main.tex}

\title{\BA{Title in Progress...}}
\author{Boris Andrews}
\affil{Mathematical Institute, University of Oxford}
\date{\today}


\begin{document}
    \pagenumbering{gobble}
    \maketitle
    
    
    \begin{abstract}
        Magnetic confinement reactors---in particular tokamaks---offer one of the most promising options for achieving practical nuclear fusion, with the potential to provide virtually limitless, clean energy. The theoretical and numerical modeling of tokamak plasmas is simultaneously an essential component of effective reactor design, and a great research barrier. Tokamak operational conditions exhibit comparatively low Knudsen numbers. Kinetic effects, including kinetic waves and instabilities, Landau damping, bump-on-tail instabilities and more, are therefore highly influential in tokamak plasma dynamics. Purely fluid models are inherently incapable of capturing these effects, whereas the high dimensionality in purely kinetic models render them practically intractable for most relevant purposes.

        We consider a $\delta\!f$ decomposition model, with a macroscopic fluid background and microscopic kinetic correction, both fully coupled to each other. A similar manner of discretization is proposed to that used in the recent \texttt{STRUPHY} code \cite{Holderied_Possanner_Wang_2021, Holderied_2022, Li_et_al_2023} with a finite-element model for the background and a pseudo-particle/PiC model for the correction.

        The fluid background satisfies the full, non-linear, resistive, compressible, Hall MHD equations. \cite{Laakmann_Hu_Farrell_2022} introduces finite-element(-in-space) implicit timesteppers for the incompressible analogue to this system with structure-preserving (SP) properties in the ideal case, alongside parameter-robust preconditioners. We show that these timesteppers can derive from a finite-element-in-time (FET) (and finite-element-in-space) interpretation. The benefits of this reformulation are discussed, including the derivation of timesteppers that are higher order in time, and the quantifiable dissipative SP properties in the non-ideal, resistive case.
        
        We discuss possible options for extending this FET approach to timesteppers for the compressible case.

        The kinetic corrections satisfy linearized Boltzmann equations. Using a Lénard--Bernstein collision operator, these take Fokker--Planck-like forms \cite{Fokker_1914, Planck_1917} wherein pseudo-particles in the numerical model obey the neoclassical transport equations, with particle-independent Brownian drift terms. This offers a rigorous methodology for incorporating collisions into the particle transport model, without coupling the equations of motions for each particle.
        
        Works by Chen, Chacón et al. \cite{Chen_Chacón_Barnes_2011, Chacón_Chen_Barnes_2013, Chen_Chacón_2014, Chen_Chacón_2015} have developed structure-preserving particle pushers for neoclassical transport in the Vlasov equations, derived from Crank--Nicolson integrators. We show these too can can derive from a FET interpretation, similarly offering potential extensions to higher-order-in-time particle pushers. The FET formulation is used also to consider how the stochastic drift terms can be incorporated into the pushers. Stochastic gyrokinetic expansions are also discussed.

        Different options for the numerical implementation of these schemes are considered.

        Due to the efficacy of FET in the development of SP timesteppers for both the fluid and kinetic component, we hope this approach will prove effective in the future for developing SP timesteppers for the full hybrid model. We hope this will give us the opportunity to incorporate previously inaccessible kinetic effects into the highly effective, modern, finite-element MHD models.
    \end{abstract}
    
    
    \newpage
    \tableofcontents
    
    
    \newpage
    \pagenumbering{arabic}
    %\linenumbers\renewcommand\thelinenumber{\color{black!50}\arabic{linenumber}}
            \input{0 - introduction/main.tex}
        \part{Research}
            \input{1 - low-noise PiC models/main.tex}
            \input{2 - kinetic component/main.tex}
            \input{3 - fluid component/main.tex}
            \input{4 - numerical implementation/main.tex}
        \part{Project Overview}
            \input{5 - research plan/main.tex}
            \input{6 - summary/main.tex}
    
    
    %\section{}
    \newpage
    \pagenumbering{gobble}
        \printbibliography


    \newpage
    \pagenumbering{roman}
    \appendix
        \part{Appendices}
            \input{8 - Hilbert complexes/main.tex}
            \input{9 - weak conservation proofs/main.tex}
\end{document}

\end{document}

    
    
    %\section{}
    \newpage
    \pagenumbering{gobble}
        \printbibliography


    \newpage
    \pagenumbering{roman}
    \appendix
        \part{Appendices}
            \documentclass[12pt, a4paper]{report}

\documentclass[12pt, a4paper]{report}

\input{template/main.tex}

\title{\BA{Title in Progress...}}
\author{Boris Andrews}
\affil{Mathematical Institute, University of Oxford}
\date{\today}


\begin{document}
    \pagenumbering{gobble}
    \maketitle
    
    
    \begin{abstract}
        Magnetic confinement reactors---in particular tokamaks---offer one of the most promising options for achieving practical nuclear fusion, with the potential to provide virtually limitless, clean energy. The theoretical and numerical modeling of tokamak plasmas is simultaneously an essential component of effective reactor design, and a great research barrier. Tokamak operational conditions exhibit comparatively low Knudsen numbers. Kinetic effects, including kinetic waves and instabilities, Landau damping, bump-on-tail instabilities and more, are therefore highly influential in tokamak plasma dynamics. Purely fluid models are inherently incapable of capturing these effects, whereas the high dimensionality in purely kinetic models render them practically intractable for most relevant purposes.

        We consider a $\delta\!f$ decomposition model, with a macroscopic fluid background and microscopic kinetic correction, both fully coupled to each other. A similar manner of discretization is proposed to that used in the recent \texttt{STRUPHY} code \cite{Holderied_Possanner_Wang_2021, Holderied_2022, Li_et_al_2023} with a finite-element model for the background and a pseudo-particle/PiC model for the correction.

        The fluid background satisfies the full, non-linear, resistive, compressible, Hall MHD equations. \cite{Laakmann_Hu_Farrell_2022} introduces finite-element(-in-space) implicit timesteppers for the incompressible analogue to this system with structure-preserving (SP) properties in the ideal case, alongside parameter-robust preconditioners. We show that these timesteppers can derive from a finite-element-in-time (FET) (and finite-element-in-space) interpretation. The benefits of this reformulation are discussed, including the derivation of timesteppers that are higher order in time, and the quantifiable dissipative SP properties in the non-ideal, resistive case.
        
        We discuss possible options for extending this FET approach to timesteppers for the compressible case.

        The kinetic corrections satisfy linearized Boltzmann equations. Using a Lénard--Bernstein collision operator, these take Fokker--Planck-like forms \cite{Fokker_1914, Planck_1917} wherein pseudo-particles in the numerical model obey the neoclassical transport equations, with particle-independent Brownian drift terms. This offers a rigorous methodology for incorporating collisions into the particle transport model, without coupling the equations of motions for each particle.
        
        Works by Chen, Chacón et al. \cite{Chen_Chacón_Barnes_2011, Chacón_Chen_Barnes_2013, Chen_Chacón_2014, Chen_Chacón_2015} have developed structure-preserving particle pushers for neoclassical transport in the Vlasov equations, derived from Crank--Nicolson integrators. We show these too can can derive from a FET interpretation, similarly offering potential extensions to higher-order-in-time particle pushers. The FET formulation is used also to consider how the stochastic drift terms can be incorporated into the pushers. Stochastic gyrokinetic expansions are also discussed.

        Different options for the numerical implementation of these schemes are considered.

        Due to the efficacy of FET in the development of SP timesteppers for both the fluid and kinetic component, we hope this approach will prove effective in the future for developing SP timesteppers for the full hybrid model. We hope this will give us the opportunity to incorporate previously inaccessible kinetic effects into the highly effective, modern, finite-element MHD models.
    \end{abstract}
    
    
    \newpage
    \tableofcontents
    
    
    \newpage
    \pagenumbering{arabic}
    %\linenumbers\renewcommand\thelinenumber{\color{black!50}\arabic{linenumber}}
            \input{0 - introduction/main.tex}
        \part{Research}
            \input{1 - low-noise PiC models/main.tex}
            \input{2 - kinetic component/main.tex}
            \input{3 - fluid component/main.tex}
            \input{4 - numerical implementation/main.tex}
        \part{Project Overview}
            \input{5 - research plan/main.tex}
            \input{6 - summary/main.tex}
    
    
    %\section{}
    \newpage
    \pagenumbering{gobble}
        \printbibliography


    \newpage
    \pagenumbering{roman}
    \appendix
        \part{Appendices}
            \input{8 - Hilbert complexes/main.tex}
            \input{9 - weak conservation proofs/main.tex}
\end{document}


\title{\BA{Title in Progress...}}
\author{Boris Andrews}
\affil{Mathematical Institute, University of Oxford}
\date{\today}


\begin{document}
    \pagenumbering{gobble}
    \maketitle
    
    
    \begin{abstract}
        Magnetic confinement reactors---in particular tokamaks---offer one of the most promising options for achieving practical nuclear fusion, with the potential to provide virtually limitless, clean energy. The theoretical and numerical modeling of tokamak plasmas is simultaneously an essential component of effective reactor design, and a great research barrier. Tokamak operational conditions exhibit comparatively low Knudsen numbers. Kinetic effects, including kinetic waves and instabilities, Landau damping, bump-on-tail instabilities and more, are therefore highly influential in tokamak plasma dynamics. Purely fluid models are inherently incapable of capturing these effects, whereas the high dimensionality in purely kinetic models render them practically intractable for most relevant purposes.

        We consider a $\delta\!f$ decomposition model, with a macroscopic fluid background and microscopic kinetic correction, both fully coupled to each other. A similar manner of discretization is proposed to that used in the recent \texttt{STRUPHY} code \cite{Holderied_Possanner_Wang_2021, Holderied_2022, Li_et_al_2023} with a finite-element model for the background and a pseudo-particle/PiC model for the correction.

        The fluid background satisfies the full, non-linear, resistive, compressible, Hall MHD equations. \cite{Laakmann_Hu_Farrell_2022} introduces finite-element(-in-space) implicit timesteppers for the incompressible analogue to this system with structure-preserving (SP) properties in the ideal case, alongside parameter-robust preconditioners. We show that these timesteppers can derive from a finite-element-in-time (FET) (and finite-element-in-space) interpretation. The benefits of this reformulation are discussed, including the derivation of timesteppers that are higher order in time, and the quantifiable dissipative SP properties in the non-ideal, resistive case.
        
        We discuss possible options for extending this FET approach to timesteppers for the compressible case.

        The kinetic corrections satisfy linearized Boltzmann equations. Using a Lénard--Bernstein collision operator, these take Fokker--Planck-like forms \cite{Fokker_1914, Planck_1917} wherein pseudo-particles in the numerical model obey the neoclassical transport equations, with particle-independent Brownian drift terms. This offers a rigorous methodology for incorporating collisions into the particle transport model, without coupling the equations of motions for each particle.
        
        Works by Chen, Chacón et al. \cite{Chen_Chacón_Barnes_2011, Chacón_Chen_Barnes_2013, Chen_Chacón_2014, Chen_Chacón_2015} have developed structure-preserving particle pushers for neoclassical transport in the Vlasov equations, derived from Crank--Nicolson integrators. We show these too can can derive from a FET interpretation, similarly offering potential extensions to higher-order-in-time particle pushers. The FET formulation is used also to consider how the stochastic drift terms can be incorporated into the pushers. Stochastic gyrokinetic expansions are also discussed.

        Different options for the numerical implementation of these schemes are considered.

        Due to the efficacy of FET in the development of SP timesteppers for both the fluid and kinetic component, we hope this approach will prove effective in the future for developing SP timesteppers for the full hybrid model. We hope this will give us the opportunity to incorporate previously inaccessible kinetic effects into the highly effective, modern, finite-element MHD models.
    \end{abstract}
    
    
    \newpage
    \tableofcontents
    
    
    \newpage
    \pagenumbering{arabic}
    %\linenumbers\renewcommand\thelinenumber{\color{black!50}\arabic{linenumber}}
            \documentclass[12pt, a4paper]{report}

\input{template/main.tex}

\title{\BA{Title in Progress...}}
\author{Boris Andrews}
\affil{Mathematical Institute, University of Oxford}
\date{\today}


\begin{document}
    \pagenumbering{gobble}
    \maketitle
    
    
    \begin{abstract}
        Magnetic confinement reactors---in particular tokamaks---offer one of the most promising options for achieving practical nuclear fusion, with the potential to provide virtually limitless, clean energy. The theoretical and numerical modeling of tokamak plasmas is simultaneously an essential component of effective reactor design, and a great research barrier. Tokamak operational conditions exhibit comparatively low Knudsen numbers. Kinetic effects, including kinetic waves and instabilities, Landau damping, bump-on-tail instabilities and more, are therefore highly influential in tokamak plasma dynamics. Purely fluid models are inherently incapable of capturing these effects, whereas the high dimensionality in purely kinetic models render them practically intractable for most relevant purposes.

        We consider a $\delta\!f$ decomposition model, with a macroscopic fluid background and microscopic kinetic correction, both fully coupled to each other. A similar manner of discretization is proposed to that used in the recent \texttt{STRUPHY} code \cite{Holderied_Possanner_Wang_2021, Holderied_2022, Li_et_al_2023} with a finite-element model for the background and a pseudo-particle/PiC model for the correction.

        The fluid background satisfies the full, non-linear, resistive, compressible, Hall MHD equations. \cite{Laakmann_Hu_Farrell_2022} introduces finite-element(-in-space) implicit timesteppers for the incompressible analogue to this system with structure-preserving (SP) properties in the ideal case, alongside parameter-robust preconditioners. We show that these timesteppers can derive from a finite-element-in-time (FET) (and finite-element-in-space) interpretation. The benefits of this reformulation are discussed, including the derivation of timesteppers that are higher order in time, and the quantifiable dissipative SP properties in the non-ideal, resistive case.
        
        We discuss possible options for extending this FET approach to timesteppers for the compressible case.

        The kinetic corrections satisfy linearized Boltzmann equations. Using a Lénard--Bernstein collision operator, these take Fokker--Planck-like forms \cite{Fokker_1914, Planck_1917} wherein pseudo-particles in the numerical model obey the neoclassical transport equations, with particle-independent Brownian drift terms. This offers a rigorous methodology for incorporating collisions into the particle transport model, without coupling the equations of motions for each particle.
        
        Works by Chen, Chacón et al. \cite{Chen_Chacón_Barnes_2011, Chacón_Chen_Barnes_2013, Chen_Chacón_2014, Chen_Chacón_2015} have developed structure-preserving particle pushers for neoclassical transport in the Vlasov equations, derived from Crank--Nicolson integrators. We show these too can can derive from a FET interpretation, similarly offering potential extensions to higher-order-in-time particle pushers. The FET formulation is used also to consider how the stochastic drift terms can be incorporated into the pushers. Stochastic gyrokinetic expansions are also discussed.

        Different options for the numerical implementation of these schemes are considered.

        Due to the efficacy of FET in the development of SP timesteppers for both the fluid and kinetic component, we hope this approach will prove effective in the future for developing SP timesteppers for the full hybrid model. We hope this will give us the opportunity to incorporate previously inaccessible kinetic effects into the highly effective, modern, finite-element MHD models.
    \end{abstract}
    
    
    \newpage
    \tableofcontents
    
    
    \newpage
    \pagenumbering{arabic}
    %\linenumbers\renewcommand\thelinenumber{\color{black!50}\arabic{linenumber}}
            \input{0 - introduction/main.tex}
        \part{Research}
            \input{1 - low-noise PiC models/main.tex}
            \input{2 - kinetic component/main.tex}
            \input{3 - fluid component/main.tex}
            \input{4 - numerical implementation/main.tex}
        \part{Project Overview}
            \input{5 - research plan/main.tex}
            \input{6 - summary/main.tex}
    
    
    %\section{}
    \newpage
    \pagenumbering{gobble}
        \printbibliography


    \newpage
    \pagenumbering{roman}
    \appendix
        \part{Appendices}
            \input{8 - Hilbert complexes/main.tex}
            \input{9 - weak conservation proofs/main.tex}
\end{document}

        \part{Research}
            \documentclass[12pt, a4paper]{report}

\input{template/main.tex}

\title{\BA{Title in Progress...}}
\author{Boris Andrews}
\affil{Mathematical Institute, University of Oxford}
\date{\today}


\begin{document}
    \pagenumbering{gobble}
    \maketitle
    
    
    \begin{abstract}
        Magnetic confinement reactors---in particular tokamaks---offer one of the most promising options for achieving practical nuclear fusion, with the potential to provide virtually limitless, clean energy. The theoretical and numerical modeling of tokamak plasmas is simultaneously an essential component of effective reactor design, and a great research barrier. Tokamak operational conditions exhibit comparatively low Knudsen numbers. Kinetic effects, including kinetic waves and instabilities, Landau damping, bump-on-tail instabilities and more, are therefore highly influential in tokamak plasma dynamics. Purely fluid models are inherently incapable of capturing these effects, whereas the high dimensionality in purely kinetic models render them practically intractable for most relevant purposes.

        We consider a $\delta\!f$ decomposition model, with a macroscopic fluid background and microscopic kinetic correction, both fully coupled to each other. A similar manner of discretization is proposed to that used in the recent \texttt{STRUPHY} code \cite{Holderied_Possanner_Wang_2021, Holderied_2022, Li_et_al_2023} with a finite-element model for the background and a pseudo-particle/PiC model for the correction.

        The fluid background satisfies the full, non-linear, resistive, compressible, Hall MHD equations. \cite{Laakmann_Hu_Farrell_2022} introduces finite-element(-in-space) implicit timesteppers for the incompressible analogue to this system with structure-preserving (SP) properties in the ideal case, alongside parameter-robust preconditioners. We show that these timesteppers can derive from a finite-element-in-time (FET) (and finite-element-in-space) interpretation. The benefits of this reformulation are discussed, including the derivation of timesteppers that are higher order in time, and the quantifiable dissipative SP properties in the non-ideal, resistive case.
        
        We discuss possible options for extending this FET approach to timesteppers for the compressible case.

        The kinetic corrections satisfy linearized Boltzmann equations. Using a Lénard--Bernstein collision operator, these take Fokker--Planck-like forms \cite{Fokker_1914, Planck_1917} wherein pseudo-particles in the numerical model obey the neoclassical transport equations, with particle-independent Brownian drift terms. This offers a rigorous methodology for incorporating collisions into the particle transport model, without coupling the equations of motions for each particle.
        
        Works by Chen, Chacón et al. \cite{Chen_Chacón_Barnes_2011, Chacón_Chen_Barnes_2013, Chen_Chacón_2014, Chen_Chacón_2015} have developed structure-preserving particle pushers for neoclassical transport in the Vlasov equations, derived from Crank--Nicolson integrators. We show these too can can derive from a FET interpretation, similarly offering potential extensions to higher-order-in-time particle pushers. The FET formulation is used also to consider how the stochastic drift terms can be incorporated into the pushers. Stochastic gyrokinetic expansions are also discussed.

        Different options for the numerical implementation of these schemes are considered.

        Due to the efficacy of FET in the development of SP timesteppers for both the fluid and kinetic component, we hope this approach will prove effective in the future for developing SP timesteppers for the full hybrid model. We hope this will give us the opportunity to incorporate previously inaccessible kinetic effects into the highly effective, modern, finite-element MHD models.
    \end{abstract}
    
    
    \newpage
    \tableofcontents
    
    
    \newpage
    \pagenumbering{arabic}
    %\linenumbers\renewcommand\thelinenumber{\color{black!50}\arabic{linenumber}}
            \input{0 - introduction/main.tex}
        \part{Research}
            \input{1 - low-noise PiC models/main.tex}
            \input{2 - kinetic component/main.tex}
            \input{3 - fluid component/main.tex}
            \input{4 - numerical implementation/main.tex}
        \part{Project Overview}
            \input{5 - research plan/main.tex}
            \input{6 - summary/main.tex}
    
    
    %\section{}
    \newpage
    \pagenumbering{gobble}
        \printbibliography


    \newpage
    \pagenumbering{roman}
    \appendix
        \part{Appendices}
            \input{8 - Hilbert complexes/main.tex}
            \input{9 - weak conservation proofs/main.tex}
\end{document}

            \documentclass[12pt, a4paper]{report}

\input{template/main.tex}

\title{\BA{Title in Progress...}}
\author{Boris Andrews}
\affil{Mathematical Institute, University of Oxford}
\date{\today}


\begin{document}
    \pagenumbering{gobble}
    \maketitle
    
    
    \begin{abstract}
        Magnetic confinement reactors---in particular tokamaks---offer one of the most promising options for achieving practical nuclear fusion, with the potential to provide virtually limitless, clean energy. The theoretical and numerical modeling of tokamak plasmas is simultaneously an essential component of effective reactor design, and a great research barrier. Tokamak operational conditions exhibit comparatively low Knudsen numbers. Kinetic effects, including kinetic waves and instabilities, Landau damping, bump-on-tail instabilities and more, are therefore highly influential in tokamak plasma dynamics. Purely fluid models are inherently incapable of capturing these effects, whereas the high dimensionality in purely kinetic models render them practically intractable for most relevant purposes.

        We consider a $\delta\!f$ decomposition model, with a macroscopic fluid background and microscopic kinetic correction, both fully coupled to each other. A similar manner of discretization is proposed to that used in the recent \texttt{STRUPHY} code \cite{Holderied_Possanner_Wang_2021, Holderied_2022, Li_et_al_2023} with a finite-element model for the background and a pseudo-particle/PiC model for the correction.

        The fluid background satisfies the full, non-linear, resistive, compressible, Hall MHD equations. \cite{Laakmann_Hu_Farrell_2022} introduces finite-element(-in-space) implicit timesteppers for the incompressible analogue to this system with structure-preserving (SP) properties in the ideal case, alongside parameter-robust preconditioners. We show that these timesteppers can derive from a finite-element-in-time (FET) (and finite-element-in-space) interpretation. The benefits of this reformulation are discussed, including the derivation of timesteppers that are higher order in time, and the quantifiable dissipative SP properties in the non-ideal, resistive case.
        
        We discuss possible options for extending this FET approach to timesteppers for the compressible case.

        The kinetic corrections satisfy linearized Boltzmann equations. Using a Lénard--Bernstein collision operator, these take Fokker--Planck-like forms \cite{Fokker_1914, Planck_1917} wherein pseudo-particles in the numerical model obey the neoclassical transport equations, with particle-independent Brownian drift terms. This offers a rigorous methodology for incorporating collisions into the particle transport model, without coupling the equations of motions for each particle.
        
        Works by Chen, Chacón et al. \cite{Chen_Chacón_Barnes_2011, Chacón_Chen_Barnes_2013, Chen_Chacón_2014, Chen_Chacón_2015} have developed structure-preserving particle pushers for neoclassical transport in the Vlasov equations, derived from Crank--Nicolson integrators. We show these too can can derive from a FET interpretation, similarly offering potential extensions to higher-order-in-time particle pushers. The FET formulation is used also to consider how the stochastic drift terms can be incorporated into the pushers. Stochastic gyrokinetic expansions are also discussed.

        Different options for the numerical implementation of these schemes are considered.

        Due to the efficacy of FET in the development of SP timesteppers for both the fluid and kinetic component, we hope this approach will prove effective in the future for developing SP timesteppers for the full hybrid model. We hope this will give us the opportunity to incorporate previously inaccessible kinetic effects into the highly effective, modern, finite-element MHD models.
    \end{abstract}
    
    
    \newpage
    \tableofcontents
    
    
    \newpage
    \pagenumbering{arabic}
    %\linenumbers\renewcommand\thelinenumber{\color{black!50}\arabic{linenumber}}
            \input{0 - introduction/main.tex}
        \part{Research}
            \input{1 - low-noise PiC models/main.tex}
            \input{2 - kinetic component/main.tex}
            \input{3 - fluid component/main.tex}
            \input{4 - numerical implementation/main.tex}
        \part{Project Overview}
            \input{5 - research plan/main.tex}
            \input{6 - summary/main.tex}
    
    
    %\section{}
    \newpage
    \pagenumbering{gobble}
        \printbibliography


    \newpage
    \pagenumbering{roman}
    \appendix
        \part{Appendices}
            \input{8 - Hilbert complexes/main.tex}
            \input{9 - weak conservation proofs/main.tex}
\end{document}

            \documentclass[12pt, a4paper]{report}

\input{template/main.tex}

\title{\BA{Title in Progress...}}
\author{Boris Andrews}
\affil{Mathematical Institute, University of Oxford}
\date{\today}


\begin{document}
    \pagenumbering{gobble}
    \maketitle
    
    
    \begin{abstract}
        Magnetic confinement reactors---in particular tokamaks---offer one of the most promising options for achieving practical nuclear fusion, with the potential to provide virtually limitless, clean energy. The theoretical and numerical modeling of tokamak plasmas is simultaneously an essential component of effective reactor design, and a great research barrier. Tokamak operational conditions exhibit comparatively low Knudsen numbers. Kinetic effects, including kinetic waves and instabilities, Landau damping, bump-on-tail instabilities and more, are therefore highly influential in tokamak plasma dynamics. Purely fluid models are inherently incapable of capturing these effects, whereas the high dimensionality in purely kinetic models render them practically intractable for most relevant purposes.

        We consider a $\delta\!f$ decomposition model, with a macroscopic fluid background and microscopic kinetic correction, both fully coupled to each other. A similar manner of discretization is proposed to that used in the recent \texttt{STRUPHY} code \cite{Holderied_Possanner_Wang_2021, Holderied_2022, Li_et_al_2023} with a finite-element model for the background and a pseudo-particle/PiC model for the correction.

        The fluid background satisfies the full, non-linear, resistive, compressible, Hall MHD equations. \cite{Laakmann_Hu_Farrell_2022} introduces finite-element(-in-space) implicit timesteppers for the incompressible analogue to this system with structure-preserving (SP) properties in the ideal case, alongside parameter-robust preconditioners. We show that these timesteppers can derive from a finite-element-in-time (FET) (and finite-element-in-space) interpretation. The benefits of this reformulation are discussed, including the derivation of timesteppers that are higher order in time, and the quantifiable dissipative SP properties in the non-ideal, resistive case.
        
        We discuss possible options for extending this FET approach to timesteppers for the compressible case.

        The kinetic corrections satisfy linearized Boltzmann equations. Using a Lénard--Bernstein collision operator, these take Fokker--Planck-like forms \cite{Fokker_1914, Planck_1917} wherein pseudo-particles in the numerical model obey the neoclassical transport equations, with particle-independent Brownian drift terms. This offers a rigorous methodology for incorporating collisions into the particle transport model, without coupling the equations of motions for each particle.
        
        Works by Chen, Chacón et al. \cite{Chen_Chacón_Barnes_2011, Chacón_Chen_Barnes_2013, Chen_Chacón_2014, Chen_Chacón_2015} have developed structure-preserving particle pushers for neoclassical transport in the Vlasov equations, derived from Crank--Nicolson integrators. We show these too can can derive from a FET interpretation, similarly offering potential extensions to higher-order-in-time particle pushers. The FET formulation is used also to consider how the stochastic drift terms can be incorporated into the pushers. Stochastic gyrokinetic expansions are also discussed.

        Different options for the numerical implementation of these schemes are considered.

        Due to the efficacy of FET in the development of SP timesteppers for both the fluid and kinetic component, we hope this approach will prove effective in the future for developing SP timesteppers for the full hybrid model. We hope this will give us the opportunity to incorporate previously inaccessible kinetic effects into the highly effective, modern, finite-element MHD models.
    \end{abstract}
    
    
    \newpage
    \tableofcontents
    
    
    \newpage
    \pagenumbering{arabic}
    %\linenumbers\renewcommand\thelinenumber{\color{black!50}\arabic{linenumber}}
            \input{0 - introduction/main.tex}
        \part{Research}
            \input{1 - low-noise PiC models/main.tex}
            \input{2 - kinetic component/main.tex}
            \input{3 - fluid component/main.tex}
            \input{4 - numerical implementation/main.tex}
        \part{Project Overview}
            \input{5 - research plan/main.tex}
            \input{6 - summary/main.tex}
    
    
    %\section{}
    \newpage
    \pagenumbering{gobble}
        \printbibliography


    \newpage
    \pagenumbering{roman}
    \appendix
        \part{Appendices}
            \input{8 - Hilbert complexes/main.tex}
            \input{9 - weak conservation proofs/main.tex}
\end{document}

            \documentclass[12pt, a4paper]{report}

\input{template/main.tex}

\title{\BA{Title in Progress...}}
\author{Boris Andrews}
\affil{Mathematical Institute, University of Oxford}
\date{\today}


\begin{document}
    \pagenumbering{gobble}
    \maketitle
    
    
    \begin{abstract}
        Magnetic confinement reactors---in particular tokamaks---offer one of the most promising options for achieving practical nuclear fusion, with the potential to provide virtually limitless, clean energy. The theoretical and numerical modeling of tokamak plasmas is simultaneously an essential component of effective reactor design, and a great research barrier. Tokamak operational conditions exhibit comparatively low Knudsen numbers. Kinetic effects, including kinetic waves and instabilities, Landau damping, bump-on-tail instabilities and more, are therefore highly influential in tokamak plasma dynamics. Purely fluid models are inherently incapable of capturing these effects, whereas the high dimensionality in purely kinetic models render them practically intractable for most relevant purposes.

        We consider a $\delta\!f$ decomposition model, with a macroscopic fluid background and microscopic kinetic correction, both fully coupled to each other. A similar manner of discretization is proposed to that used in the recent \texttt{STRUPHY} code \cite{Holderied_Possanner_Wang_2021, Holderied_2022, Li_et_al_2023} with a finite-element model for the background and a pseudo-particle/PiC model for the correction.

        The fluid background satisfies the full, non-linear, resistive, compressible, Hall MHD equations. \cite{Laakmann_Hu_Farrell_2022} introduces finite-element(-in-space) implicit timesteppers for the incompressible analogue to this system with structure-preserving (SP) properties in the ideal case, alongside parameter-robust preconditioners. We show that these timesteppers can derive from a finite-element-in-time (FET) (and finite-element-in-space) interpretation. The benefits of this reformulation are discussed, including the derivation of timesteppers that are higher order in time, and the quantifiable dissipative SP properties in the non-ideal, resistive case.
        
        We discuss possible options for extending this FET approach to timesteppers for the compressible case.

        The kinetic corrections satisfy linearized Boltzmann equations. Using a Lénard--Bernstein collision operator, these take Fokker--Planck-like forms \cite{Fokker_1914, Planck_1917} wherein pseudo-particles in the numerical model obey the neoclassical transport equations, with particle-independent Brownian drift terms. This offers a rigorous methodology for incorporating collisions into the particle transport model, without coupling the equations of motions for each particle.
        
        Works by Chen, Chacón et al. \cite{Chen_Chacón_Barnes_2011, Chacón_Chen_Barnes_2013, Chen_Chacón_2014, Chen_Chacón_2015} have developed structure-preserving particle pushers for neoclassical transport in the Vlasov equations, derived from Crank--Nicolson integrators. We show these too can can derive from a FET interpretation, similarly offering potential extensions to higher-order-in-time particle pushers. The FET formulation is used also to consider how the stochastic drift terms can be incorporated into the pushers. Stochastic gyrokinetic expansions are also discussed.

        Different options for the numerical implementation of these schemes are considered.

        Due to the efficacy of FET in the development of SP timesteppers for both the fluid and kinetic component, we hope this approach will prove effective in the future for developing SP timesteppers for the full hybrid model. We hope this will give us the opportunity to incorporate previously inaccessible kinetic effects into the highly effective, modern, finite-element MHD models.
    \end{abstract}
    
    
    \newpage
    \tableofcontents
    
    
    \newpage
    \pagenumbering{arabic}
    %\linenumbers\renewcommand\thelinenumber{\color{black!50}\arabic{linenumber}}
            \input{0 - introduction/main.tex}
        \part{Research}
            \input{1 - low-noise PiC models/main.tex}
            \input{2 - kinetic component/main.tex}
            \input{3 - fluid component/main.tex}
            \input{4 - numerical implementation/main.tex}
        \part{Project Overview}
            \input{5 - research plan/main.tex}
            \input{6 - summary/main.tex}
    
    
    %\section{}
    \newpage
    \pagenumbering{gobble}
        \printbibliography


    \newpage
    \pagenumbering{roman}
    \appendix
        \part{Appendices}
            \input{8 - Hilbert complexes/main.tex}
            \input{9 - weak conservation proofs/main.tex}
\end{document}

        \part{Project Overview}
            \documentclass[12pt, a4paper]{report}

\input{template/main.tex}

\title{\BA{Title in Progress...}}
\author{Boris Andrews}
\affil{Mathematical Institute, University of Oxford}
\date{\today}


\begin{document}
    \pagenumbering{gobble}
    \maketitle
    
    
    \begin{abstract}
        Magnetic confinement reactors---in particular tokamaks---offer one of the most promising options for achieving practical nuclear fusion, with the potential to provide virtually limitless, clean energy. The theoretical and numerical modeling of tokamak plasmas is simultaneously an essential component of effective reactor design, and a great research barrier. Tokamak operational conditions exhibit comparatively low Knudsen numbers. Kinetic effects, including kinetic waves and instabilities, Landau damping, bump-on-tail instabilities and more, are therefore highly influential in tokamak plasma dynamics. Purely fluid models are inherently incapable of capturing these effects, whereas the high dimensionality in purely kinetic models render them practically intractable for most relevant purposes.

        We consider a $\delta\!f$ decomposition model, with a macroscopic fluid background and microscopic kinetic correction, both fully coupled to each other. A similar manner of discretization is proposed to that used in the recent \texttt{STRUPHY} code \cite{Holderied_Possanner_Wang_2021, Holderied_2022, Li_et_al_2023} with a finite-element model for the background and a pseudo-particle/PiC model for the correction.

        The fluid background satisfies the full, non-linear, resistive, compressible, Hall MHD equations. \cite{Laakmann_Hu_Farrell_2022} introduces finite-element(-in-space) implicit timesteppers for the incompressible analogue to this system with structure-preserving (SP) properties in the ideal case, alongside parameter-robust preconditioners. We show that these timesteppers can derive from a finite-element-in-time (FET) (and finite-element-in-space) interpretation. The benefits of this reformulation are discussed, including the derivation of timesteppers that are higher order in time, and the quantifiable dissipative SP properties in the non-ideal, resistive case.
        
        We discuss possible options for extending this FET approach to timesteppers for the compressible case.

        The kinetic corrections satisfy linearized Boltzmann equations. Using a Lénard--Bernstein collision operator, these take Fokker--Planck-like forms \cite{Fokker_1914, Planck_1917} wherein pseudo-particles in the numerical model obey the neoclassical transport equations, with particle-independent Brownian drift terms. This offers a rigorous methodology for incorporating collisions into the particle transport model, without coupling the equations of motions for each particle.
        
        Works by Chen, Chacón et al. \cite{Chen_Chacón_Barnes_2011, Chacón_Chen_Barnes_2013, Chen_Chacón_2014, Chen_Chacón_2015} have developed structure-preserving particle pushers for neoclassical transport in the Vlasov equations, derived from Crank--Nicolson integrators. We show these too can can derive from a FET interpretation, similarly offering potential extensions to higher-order-in-time particle pushers. The FET formulation is used also to consider how the stochastic drift terms can be incorporated into the pushers. Stochastic gyrokinetic expansions are also discussed.

        Different options for the numerical implementation of these schemes are considered.

        Due to the efficacy of FET in the development of SP timesteppers for both the fluid and kinetic component, we hope this approach will prove effective in the future for developing SP timesteppers for the full hybrid model. We hope this will give us the opportunity to incorporate previously inaccessible kinetic effects into the highly effective, modern, finite-element MHD models.
    \end{abstract}
    
    
    \newpage
    \tableofcontents
    
    
    \newpage
    \pagenumbering{arabic}
    %\linenumbers\renewcommand\thelinenumber{\color{black!50}\arabic{linenumber}}
            \input{0 - introduction/main.tex}
        \part{Research}
            \input{1 - low-noise PiC models/main.tex}
            \input{2 - kinetic component/main.tex}
            \input{3 - fluid component/main.tex}
            \input{4 - numerical implementation/main.tex}
        \part{Project Overview}
            \input{5 - research plan/main.tex}
            \input{6 - summary/main.tex}
    
    
    %\section{}
    \newpage
    \pagenumbering{gobble}
        \printbibliography


    \newpage
    \pagenumbering{roman}
    \appendix
        \part{Appendices}
            \input{8 - Hilbert complexes/main.tex}
            \input{9 - weak conservation proofs/main.tex}
\end{document}

            \documentclass[12pt, a4paper]{report}

\input{template/main.tex}

\title{\BA{Title in Progress...}}
\author{Boris Andrews}
\affil{Mathematical Institute, University of Oxford}
\date{\today}


\begin{document}
    \pagenumbering{gobble}
    \maketitle
    
    
    \begin{abstract}
        Magnetic confinement reactors---in particular tokamaks---offer one of the most promising options for achieving practical nuclear fusion, with the potential to provide virtually limitless, clean energy. The theoretical and numerical modeling of tokamak plasmas is simultaneously an essential component of effective reactor design, and a great research barrier. Tokamak operational conditions exhibit comparatively low Knudsen numbers. Kinetic effects, including kinetic waves and instabilities, Landau damping, bump-on-tail instabilities and more, are therefore highly influential in tokamak plasma dynamics. Purely fluid models are inherently incapable of capturing these effects, whereas the high dimensionality in purely kinetic models render them practically intractable for most relevant purposes.

        We consider a $\delta\!f$ decomposition model, with a macroscopic fluid background and microscopic kinetic correction, both fully coupled to each other. A similar manner of discretization is proposed to that used in the recent \texttt{STRUPHY} code \cite{Holderied_Possanner_Wang_2021, Holderied_2022, Li_et_al_2023} with a finite-element model for the background and a pseudo-particle/PiC model for the correction.

        The fluid background satisfies the full, non-linear, resistive, compressible, Hall MHD equations. \cite{Laakmann_Hu_Farrell_2022} introduces finite-element(-in-space) implicit timesteppers for the incompressible analogue to this system with structure-preserving (SP) properties in the ideal case, alongside parameter-robust preconditioners. We show that these timesteppers can derive from a finite-element-in-time (FET) (and finite-element-in-space) interpretation. The benefits of this reformulation are discussed, including the derivation of timesteppers that are higher order in time, and the quantifiable dissipative SP properties in the non-ideal, resistive case.
        
        We discuss possible options for extending this FET approach to timesteppers for the compressible case.

        The kinetic corrections satisfy linearized Boltzmann equations. Using a Lénard--Bernstein collision operator, these take Fokker--Planck-like forms \cite{Fokker_1914, Planck_1917} wherein pseudo-particles in the numerical model obey the neoclassical transport equations, with particle-independent Brownian drift terms. This offers a rigorous methodology for incorporating collisions into the particle transport model, without coupling the equations of motions for each particle.
        
        Works by Chen, Chacón et al. \cite{Chen_Chacón_Barnes_2011, Chacón_Chen_Barnes_2013, Chen_Chacón_2014, Chen_Chacón_2015} have developed structure-preserving particle pushers for neoclassical transport in the Vlasov equations, derived from Crank--Nicolson integrators. We show these too can can derive from a FET interpretation, similarly offering potential extensions to higher-order-in-time particle pushers. The FET formulation is used also to consider how the stochastic drift terms can be incorporated into the pushers. Stochastic gyrokinetic expansions are also discussed.

        Different options for the numerical implementation of these schemes are considered.

        Due to the efficacy of FET in the development of SP timesteppers for both the fluid and kinetic component, we hope this approach will prove effective in the future for developing SP timesteppers for the full hybrid model. We hope this will give us the opportunity to incorporate previously inaccessible kinetic effects into the highly effective, modern, finite-element MHD models.
    \end{abstract}
    
    
    \newpage
    \tableofcontents
    
    
    \newpage
    \pagenumbering{arabic}
    %\linenumbers\renewcommand\thelinenumber{\color{black!50}\arabic{linenumber}}
            \input{0 - introduction/main.tex}
        \part{Research}
            \input{1 - low-noise PiC models/main.tex}
            \input{2 - kinetic component/main.tex}
            \input{3 - fluid component/main.tex}
            \input{4 - numerical implementation/main.tex}
        \part{Project Overview}
            \input{5 - research plan/main.tex}
            \input{6 - summary/main.tex}
    
    
    %\section{}
    \newpage
    \pagenumbering{gobble}
        \printbibliography


    \newpage
    \pagenumbering{roman}
    \appendix
        \part{Appendices}
            \input{8 - Hilbert complexes/main.tex}
            \input{9 - weak conservation proofs/main.tex}
\end{document}

    
    
    %\section{}
    \newpage
    \pagenumbering{gobble}
        \printbibliography


    \newpage
    \pagenumbering{roman}
    \appendix
        \part{Appendices}
            \documentclass[12pt, a4paper]{report}

\input{template/main.tex}

\title{\BA{Title in Progress...}}
\author{Boris Andrews}
\affil{Mathematical Institute, University of Oxford}
\date{\today}


\begin{document}
    \pagenumbering{gobble}
    \maketitle
    
    
    \begin{abstract}
        Magnetic confinement reactors---in particular tokamaks---offer one of the most promising options for achieving practical nuclear fusion, with the potential to provide virtually limitless, clean energy. The theoretical and numerical modeling of tokamak plasmas is simultaneously an essential component of effective reactor design, and a great research barrier. Tokamak operational conditions exhibit comparatively low Knudsen numbers. Kinetic effects, including kinetic waves and instabilities, Landau damping, bump-on-tail instabilities and more, are therefore highly influential in tokamak plasma dynamics. Purely fluid models are inherently incapable of capturing these effects, whereas the high dimensionality in purely kinetic models render them practically intractable for most relevant purposes.

        We consider a $\delta\!f$ decomposition model, with a macroscopic fluid background and microscopic kinetic correction, both fully coupled to each other. A similar manner of discretization is proposed to that used in the recent \texttt{STRUPHY} code \cite{Holderied_Possanner_Wang_2021, Holderied_2022, Li_et_al_2023} with a finite-element model for the background and a pseudo-particle/PiC model for the correction.

        The fluid background satisfies the full, non-linear, resistive, compressible, Hall MHD equations. \cite{Laakmann_Hu_Farrell_2022} introduces finite-element(-in-space) implicit timesteppers for the incompressible analogue to this system with structure-preserving (SP) properties in the ideal case, alongside parameter-robust preconditioners. We show that these timesteppers can derive from a finite-element-in-time (FET) (and finite-element-in-space) interpretation. The benefits of this reformulation are discussed, including the derivation of timesteppers that are higher order in time, and the quantifiable dissipative SP properties in the non-ideal, resistive case.
        
        We discuss possible options for extending this FET approach to timesteppers for the compressible case.

        The kinetic corrections satisfy linearized Boltzmann equations. Using a Lénard--Bernstein collision operator, these take Fokker--Planck-like forms \cite{Fokker_1914, Planck_1917} wherein pseudo-particles in the numerical model obey the neoclassical transport equations, with particle-independent Brownian drift terms. This offers a rigorous methodology for incorporating collisions into the particle transport model, without coupling the equations of motions for each particle.
        
        Works by Chen, Chacón et al. \cite{Chen_Chacón_Barnes_2011, Chacón_Chen_Barnes_2013, Chen_Chacón_2014, Chen_Chacón_2015} have developed structure-preserving particle pushers for neoclassical transport in the Vlasov equations, derived from Crank--Nicolson integrators. We show these too can can derive from a FET interpretation, similarly offering potential extensions to higher-order-in-time particle pushers. The FET formulation is used also to consider how the stochastic drift terms can be incorporated into the pushers. Stochastic gyrokinetic expansions are also discussed.

        Different options for the numerical implementation of these schemes are considered.

        Due to the efficacy of FET in the development of SP timesteppers for both the fluid and kinetic component, we hope this approach will prove effective in the future for developing SP timesteppers for the full hybrid model. We hope this will give us the opportunity to incorporate previously inaccessible kinetic effects into the highly effective, modern, finite-element MHD models.
    \end{abstract}
    
    
    \newpage
    \tableofcontents
    
    
    \newpage
    \pagenumbering{arabic}
    %\linenumbers\renewcommand\thelinenumber{\color{black!50}\arabic{linenumber}}
            \input{0 - introduction/main.tex}
        \part{Research}
            \input{1 - low-noise PiC models/main.tex}
            \input{2 - kinetic component/main.tex}
            \input{3 - fluid component/main.tex}
            \input{4 - numerical implementation/main.tex}
        \part{Project Overview}
            \input{5 - research plan/main.tex}
            \input{6 - summary/main.tex}
    
    
    %\section{}
    \newpage
    \pagenumbering{gobble}
        \printbibliography


    \newpage
    \pagenumbering{roman}
    \appendix
        \part{Appendices}
            \input{8 - Hilbert complexes/main.tex}
            \input{9 - weak conservation proofs/main.tex}
\end{document}

            \documentclass[12pt, a4paper]{report}

\input{template/main.tex}

\title{\BA{Title in Progress...}}
\author{Boris Andrews}
\affil{Mathematical Institute, University of Oxford}
\date{\today}


\begin{document}
    \pagenumbering{gobble}
    \maketitle
    
    
    \begin{abstract}
        Magnetic confinement reactors---in particular tokamaks---offer one of the most promising options for achieving practical nuclear fusion, with the potential to provide virtually limitless, clean energy. The theoretical and numerical modeling of tokamak plasmas is simultaneously an essential component of effective reactor design, and a great research barrier. Tokamak operational conditions exhibit comparatively low Knudsen numbers. Kinetic effects, including kinetic waves and instabilities, Landau damping, bump-on-tail instabilities and more, are therefore highly influential in tokamak plasma dynamics. Purely fluid models are inherently incapable of capturing these effects, whereas the high dimensionality in purely kinetic models render them practically intractable for most relevant purposes.

        We consider a $\delta\!f$ decomposition model, with a macroscopic fluid background and microscopic kinetic correction, both fully coupled to each other. A similar manner of discretization is proposed to that used in the recent \texttt{STRUPHY} code \cite{Holderied_Possanner_Wang_2021, Holderied_2022, Li_et_al_2023} with a finite-element model for the background and a pseudo-particle/PiC model for the correction.

        The fluid background satisfies the full, non-linear, resistive, compressible, Hall MHD equations. \cite{Laakmann_Hu_Farrell_2022} introduces finite-element(-in-space) implicit timesteppers for the incompressible analogue to this system with structure-preserving (SP) properties in the ideal case, alongside parameter-robust preconditioners. We show that these timesteppers can derive from a finite-element-in-time (FET) (and finite-element-in-space) interpretation. The benefits of this reformulation are discussed, including the derivation of timesteppers that are higher order in time, and the quantifiable dissipative SP properties in the non-ideal, resistive case.
        
        We discuss possible options for extending this FET approach to timesteppers for the compressible case.

        The kinetic corrections satisfy linearized Boltzmann equations. Using a Lénard--Bernstein collision operator, these take Fokker--Planck-like forms \cite{Fokker_1914, Planck_1917} wherein pseudo-particles in the numerical model obey the neoclassical transport equations, with particle-independent Brownian drift terms. This offers a rigorous methodology for incorporating collisions into the particle transport model, without coupling the equations of motions for each particle.
        
        Works by Chen, Chacón et al. \cite{Chen_Chacón_Barnes_2011, Chacón_Chen_Barnes_2013, Chen_Chacón_2014, Chen_Chacón_2015} have developed structure-preserving particle pushers for neoclassical transport in the Vlasov equations, derived from Crank--Nicolson integrators. We show these too can can derive from a FET interpretation, similarly offering potential extensions to higher-order-in-time particle pushers. The FET formulation is used also to consider how the stochastic drift terms can be incorporated into the pushers. Stochastic gyrokinetic expansions are also discussed.

        Different options for the numerical implementation of these schemes are considered.

        Due to the efficacy of FET in the development of SP timesteppers for both the fluid and kinetic component, we hope this approach will prove effective in the future for developing SP timesteppers for the full hybrid model. We hope this will give us the opportunity to incorporate previously inaccessible kinetic effects into the highly effective, modern, finite-element MHD models.
    \end{abstract}
    
    
    \newpage
    \tableofcontents
    
    
    \newpage
    \pagenumbering{arabic}
    %\linenumbers\renewcommand\thelinenumber{\color{black!50}\arabic{linenumber}}
            \input{0 - introduction/main.tex}
        \part{Research}
            \input{1 - low-noise PiC models/main.tex}
            \input{2 - kinetic component/main.tex}
            \input{3 - fluid component/main.tex}
            \input{4 - numerical implementation/main.tex}
        \part{Project Overview}
            \input{5 - research plan/main.tex}
            \input{6 - summary/main.tex}
    
    
    %\section{}
    \newpage
    \pagenumbering{gobble}
        \printbibliography


    \newpage
    \pagenumbering{roman}
    \appendix
        \part{Appendices}
            \input{8 - Hilbert complexes/main.tex}
            \input{9 - weak conservation proofs/main.tex}
\end{document}

\end{document}

            \documentclass[12pt, a4paper]{report}

\documentclass[12pt, a4paper]{report}

\input{template/main.tex}

\title{\BA{Title in Progress...}}
\author{Boris Andrews}
\affil{Mathematical Institute, University of Oxford}
\date{\today}


\begin{document}
    \pagenumbering{gobble}
    \maketitle
    
    
    \begin{abstract}
        Magnetic confinement reactors---in particular tokamaks---offer one of the most promising options for achieving practical nuclear fusion, with the potential to provide virtually limitless, clean energy. The theoretical and numerical modeling of tokamak plasmas is simultaneously an essential component of effective reactor design, and a great research barrier. Tokamak operational conditions exhibit comparatively low Knudsen numbers. Kinetic effects, including kinetic waves and instabilities, Landau damping, bump-on-tail instabilities and more, are therefore highly influential in tokamak plasma dynamics. Purely fluid models are inherently incapable of capturing these effects, whereas the high dimensionality in purely kinetic models render them practically intractable for most relevant purposes.

        We consider a $\delta\!f$ decomposition model, with a macroscopic fluid background and microscopic kinetic correction, both fully coupled to each other. A similar manner of discretization is proposed to that used in the recent \texttt{STRUPHY} code \cite{Holderied_Possanner_Wang_2021, Holderied_2022, Li_et_al_2023} with a finite-element model for the background and a pseudo-particle/PiC model for the correction.

        The fluid background satisfies the full, non-linear, resistive, compressible, Hall MHD equations. \cite{Laakmann_Hu_Farrell_2022} introduces finite-element(-in-space) implicit timesteppers for the incompressible analogue to this system with structure-preserving (SP) properties in the ideal case, alongside parameter-robust preconditioners. We show that these timesteppers can derive from a finite-element-in-time (FET) (and finite-element-in-space) interpretation. The benefits of this reformulation are discussed, including the derivation of timesteppers that are higher order in time, and the quantifiable dissipative SP properties in the non-ideal, resistive case.
        
        We discuss possible options for extending this FET approach to timesteppers for the compressible case.

        The kinetic corrections satisfy linearized Boltzmann equations. Using a Lénard--Bernstein collision operator, these take Fokker--Planck-like forms \cite{Fokker_1914, Planck_1917} wherein pseudo-particles in the numerical model obey the neoclassical transport equations, with particle-independent Brownian drift terms. This offers a rigorous methodology for incorporating collisions into the particle transport model, without coupling the equations of motions for each particle.
        
        Works by Chen, Chacón et al. \cite{Chen_Chacón_Barnes_2011, Chacón_Chen_Barnes_2013, Chen_Chacón_2014, Chen_Chacón_2015} have developed structure-preserving particle pushers for neoclassical transport in the Vlasov equations, derived from Crank--Nicolson integrators. We show these too can can derive from a FET interpretation, similarly offering potential extensions to higher-order-in-time particle pushers. The FET formulation is used also to consider how the stochastic drift terms can be incorporated into the pushers. Stochastic gyrokinetic expansions are also discussed.

        Different options for the numerical implementation of these schemes are considered.

        Due to the efficacy of FET in the development of SP timesteppers for both the fluid and kinetic component, we hope this approach will prove effective in the future for developing SP timesteppers for the full hybrid model. We hope this will give us the opportunity to incorporate previously inaccessible kinetic effects into the highly effective, modern, finite-element MHD models.
    \end{abstract}
    
    
    \newpage
    \tableofcontents
    
    
    \newpage
    \pagenumbering{arabic}
    %\linenumbers\renewcommand\thelinenumber{\color{black!50}\arabic{linenumber}}
            \input{0 - introduction/main.tex}
        \part{Research}
            \input{1 - low-noise PiC models/main.tex}
            \input{2 - kinetic component/main.tex}
            \input{3 - fluid component/main.tex}
            \input{4 - numerical implementation/main.tex}
        \part{Project Overview}
            \input{5 - research plan/main.tex}
            \input{6 - summary/main.tex}
    
    
    %\section{}
    \newpage
    \pagenumbering{gobble}
        \printbibliography


    \newpage
    \pagenumbering{roman}
    \appendix
        \part{Appendices}
            \input{8 - Hilbert complexes/main.tex}
            \input{9 - weak conservation proofs/main.tex}
\end{document}


\title{\BA{Title in Progress...}}
\author{Boris Andrews}
\affil{Mathematical Institute, University of Oxford}
\date{\today}


\begin{document}
    \pagenumbering{gobble}
    \maketitle
    
    
    \begin{abstract}
        Magnetic confinement reactors---in particular tokamaks---offer one of the most promising options for achieving practical nuclear fusion, with the potential to provide virtually limitless, clean energy. The theoretical and numerical modeling of tokamak plasmas is simultaneously an essential component of effective reactor design, and a great research barrier. Tokamak operational conditions exhibit comparatively low Knudsen numbers. Kinetic effects, including kinetic waves and instabilities, Landau damping, bump-on-tail instabilities and more, are therefore highly influential in tokamak plasma dynamics. Purely fluid models are inherently incapable of capturing these effects, whereas the high dimensionality in purely kinetic models render them practically intractable for most relevant purposes.

        We consider a $\delta\!f$ decomposition model, with a macroscopic fluid background and microscopic kinetic correction, both fully coupled to each other. A similar manner of discretization is proposed to that used in the recent \texttt{STRUPHY} code \cite{Holderied_Possanner_Wang_2021, Holderied_2022, Li_et_al_2023} with a finite-element model for the background and a pseudo-particle/PiC model for the correction.

        The fluid background satisfies the full, non-linear, resistive, compressible, Hall MHD equations. \cite{Laakmann_Hu_Farrell_2022} introduces finite-element(-in-space) implicit timesteppers for the incompressible analogue to this system with structure-preserving (SP) properties in the ideal case, alongside parameter-robust preconditioners. We show that these timesteppers can derive from a finite-element-in-time (FET) (and finite-element-in-space) interpretation. The benefits of this reformulation are discussed, including the derivation of timesteppers that are higher order in time, and the quantifiable dissipative SP properties in the non-ideal, resistive case.
        
        We discuss possible options for extending this FET approach to timesteppers for the compressible case.

        The kinetic corrections satisfy linearized Boltzmann equations. Using a Lénard--Bernstein collision operator, these take Fokker--Planck-like forms \cite{Fokker_1914, Planck_1917} wherein pseudo-particles in the numerical model obey the neoclassical transport equations, with particle-independent Brownian drift terms. This offers a rigorous methodology for incorporating collisions into the particle transport model, without coupling the equations of motions for each particle.
        
        Works by Chen, Chacón et al. \cite{Chen_Chacón_Barnes_2011, Chacón_Chen_Barnes_2013, Chen_Chacón_2014, Chen_Chacón_2015} have developed structure-preserving particle pushers for neoclassical transport in the Vlasov equations, derived from Crank--Nicolson integrators. We show these too can can derive from a FET interpretation, similarly offering potential extensions to higher-order-in-time particle pushers. The FET formulation is used also to consider how the stochastic drift terms can be incorporated into the pushers. Stochastic gyrokinetic expansions are also discussed.

        Different options for the numerical implementation of these schemes are considered.

        Due to the efficacy of FET in the development of SP timesteppers for both the fluid and kinetic component, we hope this approach will prove effective in the future for developing SP timesteppers for the full hybrid model. We hope this will give us the opportunity to incorporate previously inaccessible kinetic effects into the highly effective, modern, finite-element MHD models.
    \end{abstract}
    
    
    \newpage
    \tableofcontents
    
    
    \newpage
    \pagenumbering{arabic}
    %\linenumbers\renewcommand\thelinenumber{\color{black!50}\arabic{linenumber}}
            \documentclass[12pt, a4paper]{report}

\input{template/main.tex}

\title{\BA{Title in Progress...}}
\author{Boris Andrews}
\affil{Mathematical Institute, University of Oxford}
\date{\today}


\begin{document}
    \pagenumbering{gobble}
    \maketitle
    
    
    \begin{abstract}
        Magnetic confinement reactors---in particular tokamaks---offer one of the most promising options for achieving practical nuclear fusion, with the potential to provide virtually limitless, clean energy. The theoretical and numerical modeling of tokamak plasmas is simultaneously an essential component of effective reactor design, and a great research barrier. Tokamak operational conditions exhibit comparatively low Knudsen numbers. Kinetic effects, including kinetic waves and instabilities, Landau damping, bump-on-tail instabilities and more, are therefore highly influential in tokamak plasma dynamics. Purely fluid models are inherently incapable of capturing these effects, whereas the high dimensionality in purely kinetic models render them practically intractable for most relevant purposes.

        We consider a $\delta\!f$ decomposition model, with a macroscopic fluid background and microscopic kinetic correction, both fully coupled to each other. A similar manner of discretization is proposed to that used in the recent \texttt{STRUPHY} code \cite{Holderied_Possanner_Wang_2021, Holderied_2022, Li_et_al_2023} with a finite-element model for the background and a pseudo-particle/PiC model for the correction.

        The fluid background satisfies the full, non-linear, resistive, compressible, Hall MHD equations. \cite{Laakmann_Hu_Farrell_2022} introduces finite-element(-in-space) implicit timesteppers for the incompressible analogue to this system with structure-preserving (SP) properties in the ideal case, alongside parameter-robust preconditioners. We show that these timesteppers can derive from a finite-element-in-time (FET) (and finite-element-in-space) interpretation. The benefits of this reformulation are discussed, including the derivation of timesteppers that are higher order in time, and the quantifiable dissipative SP properties in the non-ideal, resistive case.
        
        We discuss possible options for extending this FET approach to timesteppers for the compressible case.

        The kinetic corrections satisfy linearized Boltzmann equations. Using a Lénard--Bernstein collision operator, these take Fokker--Planck-like forms \cite{Fokker_1914, Planck_1917} wherein pseudo-particles in the numerical model obey the neoclassical transport equations, with particle-independent Brownian drift terms. This offers a rigorous methodology for incorporating collisions into the particle transport model, without coupling the equations of motions for each particle.
        
        Works by Chen, Chacón et al. \cite{Chen_Chacón_Barnes_2011, Chacón_Chen_Barnes_2013, Chen_Chacón_2014, Chen_Chacón_2015} have developed structure-preserving particle pushers for neoclassical transport in the Vlasov equations, derived from Crank--Nicolson integrators. We show these too can can derive from a FET interpretation, similarly offering potential extensions to higher-order-in-time particle pushers. The FET formulation is used also to consider how the stochastic drift terms can be incorporated into the pushers. Stochastic gyrokinetic expansions are also discussed.

        Different options for the numerical implementation of these schemes are considered.

        Due to the efficacy of FET in the development of SP timesteppers for both the fluid and kinetic component, we hope this approach will prove effective in the future for developing SP timesteppers for the full hybrid model. We hope this will give us the opportunity to incorporate previously inaccessible kinetic effects into the highly effective, modern, finite-element MHD models.
    \end{abstract}
    
    
    \newpage
    \tableofcontents
    
    
    \newpage
    \pagenumbering{arabic}
    %\linenumbers\renewcommand\thelinenumber{\color{black!50}\arabic{linenumber}}
            \input{0 - introduction/main.tex}
        \part{Research}
            \input{1 - low-noise PiC models/main.tex}
            \input{2 - kinetic component/main.tex}
            \input{3 - fluid component/main.tex}
            \input{4 - numerical implementation/main.tex}
        \part{Project Overview}
            \input{5 - research plan/main.tex}
            \input{6 - summary/main.tex}
    
    
    %\section{}
    \newpage
    \pagenumbering{gobble}
        \printbibliography


    \newpage
    \pagenumbering{roman}
    \appendix
        \part{Appendices}
            \input{8 - Hilbert complexes/main.tex}
            \input{9 - weak conservation proofs/main.tex}
\end{document}

        \part{Research}
            \documentclass[12pt, a4paper]{report}

\input{template/main.tex}

\title{\BA{Title in Progress...}}
\author{Boris Andrews}
\affil{Mathematical Institute, University of Oxford}
\date{\today}


\begin{document}
    \pagenumbering{gobble}
    \maketitle
    
    
    \begin{abstract}
        Magnetic confinement reactors---in particular tokamaks---offer one of the most promising options for achieving practical nuclear fusion, with the potential to provide virtually limitless, clean energy. The theoretical and numerical modeling of tokamak plasmas is simultaneously an essential component of effective reactor design, and a great research barrier. Tokamak operational conditions exhibit comparatively low Knudsen numbers. Kinetic effects, including kinetic waves and instabilities, Landau damping, bump-on-tail instabilities and more, are therefore highly influential in tokamak plasma dynamics. Purely fluid models are inherently incapable of capturing these effects, whereas the high dimensionality in purely kinetic models render them practically intractable for most relevant purposes.

        We consider a $\delta\!f$ decomposition model, with a macroscopic fluid background and microscopic kinetic correction, both fully coupled to each other. A similar manner of discretization is proposed to that used in the recent \texttt{STRUPHY} code \cite{Holderied_Possanner_Wang_2021, Holderied_2022, Li_et_al_2023} with a finite-element model for the background and a pseudo-particle/PiC model for the correction.

        The fluid background satisfies the full, non-linear, resistive, compressible, Hall MHD equations. \cite{Laakmann_Hu_Farrell_2022} introduces finite-element(-in-space) implicit timesteppers for the incompressible analogue to this system with structure-preserving (SP) properties in the ideal case, alongside parameter-robust preconditioners. We show that these timesteppers can derive from a finite-element-in-time (FET) (and finite-element-in-space) interpretation. The benefits of this reformulation are discussed, including the derivation of timesteppers that are higher order in time, and the quantifiable dissipative SP properties in the non-ideal, resistive case.
        
        We discuss possible options for extending this FET approach to timesteppers for the compressible case.

        The kinetic corrections satisfy linearized Boltzmann equations. Using a Lénard--Bernstein collision operator, these take Fokker--Planck-like forms \cite{Fokker_1914, Planck_1917} wherein pseudo-particles in the numerical model obey the neoclassical transport equations, with particle-independent Brownian drift terms. This offers a rigorous methodology for incorporating collisions into the particle transport model, without coupling the equations of motions for each particle.
        
        Works by Chen, Chacón et al. \cite{Chen_Chacón_Barnes_2011, Chacón_Chen_Barnes_2013, Chen_Chacón_2014, Chen_Chacón_2015} have developed structure-preserving particle pushers for neoclassical transport in the Vlasov equations, derived from Crank--Nicolson integrators. We show these too can can derive from a FET interpretation, similarly offering potential extensions to higher-order-in-time particle pushers. The FET formulation is used also to consider how the stochastic drift terms can be incorporated into the pushers. Stochastic gyrokinetic expansions are also discussed.

        Different options for the numerical implementation of these schemes are considered.

        Due to the efficacy of FET in the development of SP timesteppers for both the fluid and kinetic component, we hope this approach will prove effective in the future for developing SP timesteppers for the full hybrid model. We hope this will give us the opportunity to incorporate previously inaccessible kinetic effects into the highly effective, modern, finite-element MHD models.
    \end{abstract}
    
    
    \newpage
    \tableofcontents
    
    
    \newpage
    \pagenumbering{arabic}
    %\linenumbers\renewcommand\thelinenumber{\color{black!50}\arabic{linenumber}}
            \input{0 - introduction/main.tex}
        \part{Research}
            \input{1 - low-noise PiC models/main.tex}
            \input{2 - kinetic component/main.tex}
            \input{3 - fluid component/main.tex}
            \input{4 - numerical implementation/main.tex}
        \part{Project Overview}
            \input{5 - research plan/main.tex}
            \input{6 - summary/main.tex}
    
    
    %\section{}
    \newpage
    \pagenumbering{gobble}
        \printbibliography


    \newpage
    \pagenumbering{roman}
    \appendix
        \part{Appendices}
            \input{8 - Hilbert complexes/main.tex}
            \input{9 - weak conservation proofs/main.tex}
\end{document}

            \documentclass[12pt, a4paper]{report}

\input{template/main.tex}

\title{\BA{Title in Progress...}}
\author{Boris Andrews}
\affil{Mathematical Institute, University of Oxford}
\date{\today}


\begin{document}
    \pagenumbering{gobble}
    \maketitle
    
    
    \begin{abstract}
        Magnetic confinement reactors---in particular tokamaks---offer one of the most promising options for achieving practical nuclear fusion, with the potential to provide virtually limitless, clean energy. The theoretical and numerical modeling of tokamak plasmas is simultaneously an essential component of effective reactor design, and a great research barrier. Tokamak operational conditions exhibit comparatively low Knudsen numbers. Kinetic effects, including kinetic waves and instabilities, Landau damping, bump-on-tail instabilities and more, are therefore highly influential in tokamak plasma dynamics. Purely fluid models are inherently incapable of capturing these effects, whereas the high dimensionality in purely kinetic models render them practically intractable for most relevant purposes.

        We consider a $\delta\!f$ decomposition model, with a macroscopic fluid background and microscopic kinetic correction, both fully coupled to each other. A similar manner of discretization is proposed to that used in the recent \texttt{STRUPHY} code \cite{Holderied_Possanner_Wang_2021, Holderied_2022, Li_et_al_2023} with a finite-element model for the background and a pseudo-particle/PiC model for the correction.

        The fluid background satisfies the full, non-linear, resistive, compressible, Hall MHD equations. \cite{Laakmann_Hu_Farrell_2022} introduces finite-element(-in-space) implicit timesteppers for the incompressible analogue to this system with structure-preserving (SP) properties in the ideal case, alongside parameter-robust preconditioners. We show that these timesteppers can derive from a finite-element-in-time (FET) (and finite-element-in-space) interpretation. The benefits of this reformulation are discussed, including the derivation of timesteppers that are higher order in time, and the quantifiable dissipative SP properties in the non-ideal, resistive case.
        
        We discuss possible options for extending this FET approach to timesteppers for the compressible case.

        The kinetic corrections satisfy linearized Boltzmann equations. Using a Lénard--Bernstein collision operator, these take Fokker--Planck-like forms \cite{Fokker_1914, Planck_1917} wherein pseudo-particles in the numerical model obey the neoclassical transport equations, with particle-independent Brownian drift terms. This offers a rigorous methodology for incorporating collisions into the particle transport model, without coupling the equations of motions for each particle.
        
        Works by Chen, Chacón et al. \cite{Chen_Chacón_Barnes_2011, Chacón_Chen_Barnes_2013, Chen_Chacón_2014, Chen_Chacón_2015} have developed structure-preserving particle pushers for neoclassical transport in the Vlasov equations, derived from Crank--Nicolson integrators. We show these too can can derive from a FET interpretation, similarly offering potential extensions to higher-order-in-time particle pushers. The FET formulation is used also to consider how the stochastic drift terms can be incorporated into the pushers. Stochastic gyrokinetic expansions are also discussed.

        Different options for the numerical implementation of these schemes are considered.

        Due to the efficacy of FET in the development of SP timesteppers for both the fluid and kinetic component, we hope this approach will prove effective in the future for developing SP timesteppers for the full hybrid model. We hope this will give us the opportunity to incorporate previously inaccessible kinetic effects into the highly effective, modern, finite-element MHD models.
    \end{abstract}
    
    
    \newpage
    \tableofcontents
    
    
    \newpage
    \pagenumbering{arabic}
    %\linenumbers\renewcommand\thelinenumber{\color{black!50}\arabic{linenumber}}
            \input{0 - introduction/main.tex}
        \part{Research}
            \input{1 - low-noise PiC models/main.tex}
            \input{2 - kinetic component/main.tex}
            \input{3 - fluid component/main.tex}
            \input{4 - numerical implementation/main.tex}
        \part{Project Overview}
            \input{5 - research plan/main.tex}
            \input{6 - summary/main.tex}
    
    
    %\section{}
    \newpage
    \pagenumbering{gobble}
        \printbibliography


    \newpage
    \pagenumbering{roman}
    \appendix
        \part{Appendices}
            \input{8 - Hilbert complexes/main.tex}
            \input{9 - weak conservation proofs/main.tex}
\end{document}

            \documentclass[12pt, a4paper]{report}

\input{template/main.tex}

\title{\BA{Title in Progress...}}
\author{Boris Andrews}
\affil{Mathematical Institute, University of Oxford}
\date{\today}


\begin{document}
    \pagenumbering{gobble}
    \maketitle
    
    
    \begin{abstract}
        Magnetic confinement reactors---in particular tokamaks---offer one of the most promising options for achieving practical nuclear fusion, with the potential to provide virtually limitless, clean energy. The theoretical and numerical modeling of tokamak plasmas is simultaneously an essential component of effective reactor design, and a great research barrier. Tokamak operational conditions exhibit comparatively low Knudsen numbers. Kinetic effects, including kinetic waves and instabilities, Landau damping, bump-on-tail instabilities and more, are therefore highly influential in tokamak plasma dynamics. Purely fluid models are inherently incapable of capturing these effects, whereas the high dimensionality in purely kinetic models render them practically intractable for most relevant purposes.

        We consider a $\delta\!f$ decomposition model, with a macroscopic fluid background and microscopic kinetic correction, both fully coupled to each other. A similar manner of discretization is proposed to that used in the recent \texttt{STRUPHY} code \cite{Holderied_Possanner_Wang_2021, Holderied_2022, Li_et_al_2023} with a finite-element model for the background and a pseudo-particle/PiC model for the correction.

        The fluid background satisfies the full, non-linear, resistive, compressible, Hall MHD equations. \cite{Laakmann_Hu_Farrell_2022} introduces finite-element(-in-space) implicit timesteppers for the incompressible analogue to this system with structure-preserving (SP) properties in the ideal case, alongside parameter-robust preconditioners. We show that these timesteppers can derive from a finite-element-in-time (FET) (and finite-element-in-space) interpretation. The benefits of this reformulation are discussed, including the derivation of timesteppers that are higher order in time, and the quantifiable dissipative SP properties in the non-ideal, resistive case.
        
        We discuss possible options for extending this FET approach to timesteppers for the compressible case.

        The kinetic corrections satisfy linearized Boltzmann equations. Using a Lénard--Bernstein collision operator, these take Fokker--Planck-like forms \cite{Fokker_1914, Planck_1917} wherein pseudo-particles in the numerical model obey the neoclassical transport equations, with particle-independent Brownian drift terms. This offers a rigorous methodology for incorporating collisions into the particle transport model, without coupling the equations of motions for each particle.
        
        Works by Chen, Chacón et al. \cite{Chen_Chacón_Barnes_2011, Chacón_Chen_Barnes_2013, Chen_Chacón_2014, Chen_Chacón_2015} have developed structure-preserving particle pushers for neoclassical transport in the Vlasov equations, derived from Crank--Nicolson integrators. We show these too can can derive from a FET interpretation, similarly offering potential extensions to higher-order-in-time particle pushers. The FET formulation is used also to consider how the stochastic drift terms can be incorporated into the pushers. Stochastic gyrokinetic expansions are also discussed.

        Different options for the numerical implementation of these schemes are considered.

        Due to the efficacy of FET in the development of SP timesteppers for both the fluid and kinetic component, we hope this approach will prove effective in the future for developing SP timesteppers for the full hybrid model. We hope this will give us the opportunity to incorporate previously inaccessible kinetic effects into the highly effective, modern, finite-element MHD models.
    \end{abstract}
    
    
    \newpage
    \tableofcontents
    
    
    \newpage
    \pagenumbering{arabic}
    %\linenumbers\renewcommand\thelinenumber{\color{black!50}\arabic{linenumber}}
            \input{0 - introduction/main.tex}
        \part{Research}
            \input{1 - low-noise PiC models/main.tex}
            \input{2 - kinetic component/main.tex}
            \input{3 - fluid component/main.tex}
            \input{4 - numerical implementation/main.tex}
        \part{Project Overview}
            \input{5 - research plan/main.tex}
            \input{6 - summary/main.tex}
    
    
    %\section{}
    \newpage
    \pagenumbering{gobble}
        \printbibliography


    \newpage
    \pagenumbering{roman}
    \appendix
        \part{Appendices}
            \input{8 - Hilbert complexes/main.tex}
            \input{9 - weak conservation proofs/main.tex}
\end{document}

            \documentclass[12pt, a4paper]{report}

\input{template/main.tex}

\title{\BA{Title in Progress...}}
\author{Boris Andrews}
\affil{Mathematical Institute, University of Oxford}
\date{\today}


\begin{document}
    \pagenumbering{gobble}
    \maketitle
    
    
    \begin{abstract}
        Magnetic confinement reactors---in particular tokamaks---offer one of the most promising options for achieving practical nuclear fusion, with the potential to provide virtually limitless, clean energy. The theoretical and numerical modeling of tokamak plasmas is simultaneously an essential component of effective reactor design, and a great research barrier. Tokamak operational conditions exhibit comparatively low Knudsen numbers. Kinetic effects, including kinetic waves and instabilities, Landau damping, bump-on-tail instabilities and more, are therefore highly influential in tokamak plasma dynamics. Purely fluid models are inherently incapable of capturing these effects, whereas the high dimensionality in purely kinetic models render them practically intractable for most relevant purposes.

        We consider a $\delta\!f$ decomposition model, with a macroscopic fluid background and microscopic kinetic correction, both fully coupled to each other. A similar manner of discretization is proposed to that used in the recent \texttt{STRUPHY} code \cite{Holderied_Possanner_Wang_2021, Holderied_2022, Li_et_al_2023} with a finite-element model for the background and a pseudo-particle/PiC model for the correction.

        The fluid background satisfies the full, non-linear, resistive, compressible, Hall MHD equations. \cite{Laakmann_Hu_Farrell_2022} introduces finite-element(-in-space) implicit timesteppers for the incompressible analogue to this system with structure-preserving (SP) properties in the ideal case, alongside parameter-robust preconditioners. We show that these timesteppers can derive from a finite-element-in-time (FET) (and finite-element-in-space) interpretation. The benefits of this reformulation are discussed, including the derivation of timesteppers that are higher order in time, and the quantifiable dissipative SP properties in the non-ideal, resistive case.
        
        We discuss possible options for extending this FET approach to timesteppers for the compressible case.

        The kinetic corrections satisfy linearized Boltzmann equations. Using a Lénard--Bernstein collision operator, these take Fokker--Planck-like forms \cite{Fokker_1914, Planck_1917} wherein pseudo-particles in the numerical model obey the neoclassical transport equations, with particle-independent Brownian drift terms. This offers a rigorous methodology for incorporating collisions into the particle transport model, without coupling the equations of motions for each particle.
        
        Works by Chen, Chacón et al. \cite{Chen_Chacón_Barnes_2011, Chacón_Chen_Barnes_2013, Chen_Chacón_2014, Chen_Chacón_2015} have developed structure-preserving particle pushers for neoclassical transport in the Vlasov equations, derived from Crank--Nicolson integrators. We show these too can can derive from a FET interpretation, similarly offering potential extensions to higher-order-in-time particle pushers. The FET formulation is used also to consider how the stochastic drift terms can be incorporated into the pushers. Stochastic gyrokinetic expansions are also discussed.

        Different options for the numerical implementation of these schemes are considered.

        Due to the efficacy of FET in the development of SP timesteppers for both the fluid and kinetic component, we hope this approach will prove effective in the future for developing SP timesteppers for the full hybrid model. We hope this will give us the opportunity to incorporate previously inaccessible kinetic effects into the highly effective, modern, finite-element MHD models.
    \end{abstract}
    
    
    \newpage
    \tableofcontents
    
    
    \newpage
    \pagenumbering{arabic}
    %\linenumbers\renewcommand\thelinenumber{\color{black!50}\arabic{linenumber}}
            \input{0 - introduction/main.tex}
        \part{Research}
            \input{1 - low-noise PiC models/main.tex}
            \input{2 - kinetic component/main.tex}
            \input{3 - fluid component/main.tex}
            \input{4 - numerical implementation/main.tex}
        \part{Project Overview}
            \input{5 - research plan/main.tex}
            \input{6 - summary/main.tex}
    
    
    %\section{}
    \newpage
    \pagenumbering{gobble}
        \printbibliography


    \newpage
    \pagenumbering{roman}
    \appendix
        \part{Appendices}
            \input{8 - Hilbert complexes/main.tex}
            \input{9 - weak conservation proofs/main.tex}
\end{document}

        \part{Project Overview}
            \documentclass[12pt, a4paper]{report}

\input{template/main.tex}

\title{\BA{Title in Progress...}}
\author{Boris Andrews}
\affil{Mathematical Institute, University of Oxford}
\date{\today}


\begin{document}
    \pagenumbering{gobble}
    \maketitle
    
    
    \begin{abstract}
        Magnetic confinement reactors---in particular tokamaks---offer one of the most promising options for achieving practical nuclear fusion, with the potential to provide virtually limitless, clean energy. The theoretical and numerical modeling of tokamak plasmas is simultaneously an essential component of effective reactor design, and a great research barrier. Tokamak operational conditions exhibit comparatively low Knudsen numbers. Kinetic effects, including kinetic waves and instabilities, Landau damping, bump-on-tail instabilities and more, are therefore highly influential in tokamak plasma dynamics. Purely fluid models are inherently incapable of capturing these effects, whereas the high dimensionality in purely kinetic models render them practically intractable for most relevant purposes.

        We consider a $\delta\!f$ decomposition model, with a macroscopic fluid background and microscopic kinetic correction, both fully coupled to each other. A similar manner of discretization is proposed to that used in the recent \texttt{STRUPHY} code \cite{Holderied_Possanner_Wang_2021, Holderied_2022, Li_et_al_2023} with a finite-element model for the background and a pseudo-particle/PiC model for the correction.

        The fluid background satisfies the full, non-linear, resistive, compressible, Hall MHD equations. \cite{Laakmann_Hu_Farrell_2022} introduces finite-element(-in-space) implicit timesteppers for the incompressible analogue to this system with structure-preserving (SP) properties in the ideal case, alongside parameter-robust preconditioners. We show that these timesteppers can derive from a finite-element-in-time (FET) (and finite-element-in-space) interpretation. The benefits of this reformulation are discussed, including the derivation of timesteppers that are higher order in time, and the quantifiable dissipative SP properties in the non-ideal, resistive case.
        
        We discuss possible options for extending this FET approach to timesteppers for the compressible case.

        The kinetic corrections satisfy linearized Boltzmann equations. Using a Lénard--Bernstein collision operator, these take Fokker--Planck-like forms \cite{Fokker_1914, Planck_1917} wherein pseudo-particles in the numerical model obey the neoclassical transport equations, with particle-independent Brownian drift terms. This offers a rigorous methodology for incorporating collisions into the particle transport model, without coupling the equations of motions for each particle.
        
        Works by Chen, Chacón et al. \cite{Chen_Chacón_Barnes_2011, Chacón_Chen_Barnes_2013, Chen_Chacón_2014, Chen_Chacón_2015} have developed structure-preserving particle pushers for neoclassical transport in the Vlasov equations, derived from Crank--Nicolson integrators. We show these too can can derive from a FET interpretation, similarly offering potential extensions to higher-order-in-time particle pushers. The FET formulation is used also to consider how the stochastic drift terms can be incorporated into the pushers. Stochastic gyrokinetic expansions are also discussed.

        Different options for the numerical implementation of these schemes are considered.

        Due to the efficacy of FET in the development of SP timesteppers for both the fluid and kinetic component, we hope this approach will prove effective in the future for developing SP timesteppers for the full hybrid model. We hope this will give us the opportunity to incorporate previously inaccessible kinetic effects into the highly effective, modern, finite-element MHD models.
    \end{abstract}
    
    
    \newpage
    \tableofcontents
    
    
    \newpage
    \pagenumbering{arabic}
    %\linenumbers\renewcommand\thelinenumber{\color{black!50}\arabic{linenumber}}
            \input{0 - introduction/main.tex}
        \part{Research}
            \input{1 - low-noise PiC models/main.tex}
            \input{2 - kinetic component/main.tex}
            \input{3 - fluid component/main.tex}
            \input{4 - numerical implementation/main.tex}
        \part{Project Overview}
            \input{5 - research plan/main.tex}
            \input{6 - summary/main.tex}
    
    
    %\section{}
    \newpage
    \pagenumbering{gobble}
        \printbibliography


    \newpage
    \pagenumbering{roman}
    \appendix
        \part{Appendices}
            \input{8 - Hilbert complexes/main.tex}
            \input{9 - weak conservation proofs/main.tex}
\end{document}

            \documentclass[12pt, a4paper]{report}

\input{template/main.tex}

\title{\BA{Title in Progress...}}
\author{Boris Andrews}
\affil{Mathematical Institute, University of Oxford}
\date{\today}


\begin{document}
    \pagenumbering{gobble}
    \maketitle
    
    
    \begin{abstract}
        Magnetic confinement reactors---in particular tokamaks---offer one of the most promising options for achieving practical nuclear fusion, with the potential to provide virtually limitless, clean energy. The theoretical and numerical modeling of tokamak plasmas is simultaneously an essential component of effective reactor design, and a great research barrier. Tokamak operational conditions exhibit comparatively low Knudsen numbers. Kinetic effects, including kinetic waves and instabilities, Landau damping, bump-on-tail instabilities and more, are therefore highly influential in tokamak plasma dynamics. Purely fluid models are inherently incapable of capturing these effects, whereas the high dimensionality in purely kinetic models render them practically intractable for most relevant purposes.

        We consider a $\delta\!f$ decomposition model, with a macroscopic fluid background and microscopic kinetic correction, both fully coupled to each other. A similar manner of discretization is proposed to that used in the recent \texttt{STRUPHY} code \cite{Holderied_Possanner_Wang_2021, Holderied_2022, Li_et_al_2023} with a finite-element model for the background and a pseudo-particle/PiC model for the correction.

        The fluid background satisfies the full, non-linear, resistive, compressible, Hall MHD equations. \cite{Laakmann_Hu_Farrell_2022} introduces finite-element(-in-space) implicit timesteppers for the incompressible analogue to this system with structure-preserving (SP) properties in the ideal case, alongside parameter-robust preconditioners. We show that these timesteppers can derive from a finite-element-in-time (FET) (and finite-element-in-space) interpretation. The benefits of this reformulation are discussed, including the derivation of timesteppers that are higher order in time, and the quantifiable dissipative SP properties in the non-ideal, resistive case.
        
        We discuss possible options for extending this FET approach to timesteppers for the compressible case.

        The kinetic corrections satisfy linearized Boltzmann equations. Using a Lénard--Bernstein collision operator, these take Fokker--Planck-like forms \cite{Fokker_1914, Planck_1917} wherein pseudo-particles in the numerical model obey the neoclassical transport equations, with particle-independent Brownian drift terms. This offers a rigorous methodology for incorporating collisions into the particle transport model, without coupling the equations of motions for each particle.
        
        Works by Chen, Chacón et al. \cite{Chen_Chacón_Barnes_2011, Chacón_Chen_Barnes_2013, Chen_Chacón_2014, Chen_Chacón_2015} have developed structure-preserving particle pushers for neoclassical transport in the Vlasov equations, derived from Crank--Nicolson integrators. We show these too can can derive from a FET interpretation, similarly offering potential extensions to higher-order-in-time particle pushers. The FET formulation is used also to consider how the stochastic drift terms can be incorporated into the pushers. Stochastic gyrokinetic expansions are also discussed.

        Different options for the numerical implementation of these schemes are considered.

        Due to the efficacy of FET in the development of SP timesteppers for both the fluid and kinetic component, we hope this approach will prove effective in the future for developing SP timesteppers for the full hybrid model. We hope this will give us the opportunity to incorporate previously inaccessible kinetic effects into the highly effective, modern, finite-element MHD models.
    \end{abstract}
    
    
    \newpage
    \tableofcontents
    
    
    \newpage
    \pagenumbering{arabic}
    %\linenumbers\renewcommand\thelinenumber{\color{black!50}\arabic{linenumber}}
            \input{0 - introduction/main.tex}
        \part{Research}
            \input{1 - low-noise PiC models/main.tex}
            \input{2 - kinetic component/main.tex}
            \input{3 - fluid component/main.tex}
            \input{4 - numerical implementation/main.tex}
        \part{Project Overview}
            \input{5 - research plan/main.tex}
            \input{6 - summary/main.tex}
    
    
    %\section{}
    \newpage
    \pagenumbering{gobble}
        \printbibliography


    \newpage
    \pagenumbering{roman}
    \appendix
        \part{Appendices}
            \input{8 - Hilbert complexes/main.tex}
            \input{9 - weak conservation proofs/main.tex}
\end{document}

    
    
    %\section{}
    \newpage
    \pagenumbering{gobble}
        \printbibliography


    \newpage
    \pagenumbering{roman}
    \appendix
        \part{Appendices}
            \documentclass[12pt, a4paper]{report}

\input{template/main.tex}

\title{\BA{Title in Progress...}}
\author{Boris Andrews}
\affil{Mathematical Institute, University of Oxford}
\date{\today}


\begin{document}
    \pagenumbering{gobble}
    \maketitle
    
    
    \begin{abstract}
        Magnetic confinement reactors---in particular tokamaks---offer one of the most promising options for achieving practical nuclear fusion, with the potential to provide virtually limitless, clean energy. The theoretical and numerical modeling of tokamak plasmas is simultaneously an essential component of effective reactor design, and a great research barrier. Tokamak operational conditions exhibit comparatively low Knudsen numbers. Kinetic effects, including kinetic waves and instabilities, Landau damping, bump-on-tail instabilities and more, are therefore highly influential in tokamak plasma dynamics. Purely fluid models are inherently incapable of capturing these effects, whereas the high dimensionality in purely kinetic models render them practically intractable for most relevant purposes.

        We consider a $\delta\!f$ decomposition model, with a macroscopic fluid background and microscopic kinetic correction, both fully coupled to each other. A similar manner of discretization is proposed to that used in the recent \texttt{STRUPHY} code \cite{Holderied_Possanner_Wang_2021, Holderied_2022, Li_et_al_2023} with a finite-element model for the background and a pseudo-particle/PiC model for the correction.

        The fluid background satisfies the full, non-linear, resistive, compressible, Hall MHD equations. \cite{Laakmann_Hu_Farrell_2022} introduces finite-element(-in-space) implicit timesteppers for the incompressible analogue to this system with structure-preserving (SP) properties in the ideal case, alongside parameter-robust preconditioners. We show that these timesteppers can derive from a finite-element-in-time (FET) (and finite-element-in-space) interpretation. The benefits of this reformulation are discussed, including the derivation of timesteppers that are higher order in time, and the quantifiable dissipative SP properties in the non-ideal, resistive case.
        
        We discuss possible options for extending this FET approach to timesteppers for the compressible case.

        The kinetic corrections satisfy linearized Boltzmann equations. Using a Lénard--Bernstein collision operator, these take Fokker--Planck-like forms \cite{Fokker_1914, Planck_1917} wherein pseudo-particles in the numerical model obey the neoclassical transport equations, with particle-independent Brownian drift terms. This offers a rigorous methodology for incorporating collisions into the particle transport model, without coupling the equations of motions for each particle.
        
        Works by Chen, Chacón et al. \cite{Chen_Chacón_Barnes_2011, Chacón_Chen_Barnes_2013, Chen_Chacón_2014, Chen_Chacón_2015} have developed structure-preserving particle pushers for neoclassical transport in the Vlasov equations, derived from Crank--Nicolson integrators. We show these too can can derive from a FET interpretation, similarly offering potential extensions to higher-order-in-time particle pushers. The FET formulation is used also to consider how the stochastic drift terms can be incorporated into the pushers. Stochastic gyrokinetic expansions are also discussed.

        Different options for the numerical implementation of these schemes are considered.

        Due to the efficacy of FET in the development of SP timesteppers for both the fluid and kinetic component, we hope this approach will prove effective in the future for developing SP timesteppers for the full hybrid model. We hope this will give us the opportunity to incorporate previously inaccessible kinetic effects into the highly effective, modern, finite-element MHD models.
    \end{abstract}
    
    
    \newpage
    \tableofcontents
    
    
    \newpage
    \pagenumbering{arabic}
    %\linenumbers\renewcommand\thelinenumber{\color{black!50}\arabic{linenumber}}
            \input{0 - introduction/main.tex}
        \part{Research}
            \input{1 - low-noise PiC models/main.tex}
            \input{2 - kinetic component/main.tex}
            \input{3 - fluid component/main.tex}
            \input{4 - numerical implementation/main.tex}
        \part{Project Overview}
            \input{5 - research plan/main.tex}
            \input{6 - summary/main.tex}
    
    
    %\section{}
    \newpage
    \pagenumbering{gobble}
        \printbibliography


    \newpage
    \pagenumbering{roman}
    \appendix
        \part{Appendices}
            \input{8 - Hilbert complexes/main.tex}
            \input{9 - weak conservation proofs/main.tex}
\end{document}

            \documentclass[12pt, a4paper]{report}

\input{template/main.tex}

\title{\BA{Title in Progress...}}
\author{Boris Andrews}
\affil{Mathematical Institute, University of Oxford}
\date{\today}


\begin{document}
    \pagenumbering{gobble}
    \maketitle
    
    
    \begin{abstract}
        Magnetic confinement reactors---in particular tokamaks---offer one of the most promising options for achieving practical nuclear fusion, with the potential to provide virtually limitless, clean energy. The theoretical and numerical modeling of tokamak plasmas is simultaneously an essential component of effective reactor design, and a great research barrier. Tokamak operational conditions exhibit comparatively low Knudsen numbers. Kinetic effects, including kinetic waves and instabilities, Landau damping, bump-on-tail instabilities and more, are therefore highly influential in tokamak plasma dynamics. Purely fluid models are inherently incapable of capturing these effects, whereas the high dimensionality in purely kinetic models render them practically intractable for most relevant purposes.

        We consider a $\delta\!f$ decomposition model, with a macroscopic fluid background and microscopic kinetic correction, both fully coupled to each other. A similar manner of discretization is proposed to that used in the recent \texttt{STRUPHY} code \cite{Holderied_Possanner_Wang_2021, Holderied_2022, Li_et_al_2023} with a finite-element model for the background and a pseudo-particle/PiC model for the correction.

        The fluid background satisfies the full, non-linear, resistive, compressible, Hall MHD equations. \cite{Laakmann_Hu_Farrell_2022} introduces finite-element(-in-space) implicit timesteppers for the incompressible analogue to this system with structure-preserving (SP) properties in the ideal case, alongside parameter-robust preconditioners. We show that these timesteppers can derive from a finite-element-in-time (FET) (and finite-element-in-space) interpretation. The benefits of this reformulation are discussed, including the derivation of timesteppers that are higher order in time, and the quantifiable dissipative SP properties in the non-ideal, resistive case.
        
        We discuss possible options for extending this FET approach to timesteppers for the compressible case.

        The kinetic corrections satisfy linearized Boltzmann equations. Using a Lénard--Bernstein collision operator, these take Fokker--Planck-like forms \cite{Fokker_1914, Planck_1917} wherein pseudo-particles in the numerical model obey the neoclassical transport equations, with particle-independent Brownian drift terms. This offers a rigorous methodology for incorporating collisions into the particle transport model, without coupling the equations of motions for each particle.
        
        Works by Chen, Chacón et al. \cite{Chen_Chacón_Barnes_2011, Chacón_Chen_Barnes_2013, Chen_Chacón_2014, Chen_Chacón_2015} have developed structure-preserving particle pushers for neoclassical transport in the Vlasov equations, derived from Crank--Nicolson integrators. We show these too can can derive from a FET interpretation, similarly offering potential extensions to higher-order-in-time particle pushers. The FET formulation is used also to consider how the stochastic drift terms can be incorporated into the pushers. Stochastic gyrokinetic expansions are also discussed.

        Different options for the numerical implementation of these schemes are considered.

        Due to the efficacy of FET in the development of SP timesteppers for both the fluid and kinetic component, we hope this approach will prove effective in the future for developing SP timesteppers for the full hybrid model. We hope this will give us the opportunity to incorporate previously inaccessible kinetic effects into the highly effective, modern, finite-element MHD models.
    \end{abstract}
    
    
    \newpage
    \tableofcontents
    
    
    \newpage
    \pagenumbering{arabic}
    %\linenumbers\renewcommand\thelinenumber{\color{black!50}\arabic{linenumber}}
            \input{0 - introduction/main.tex}
        \part{Research}
            \input{1 - low-noise PiC models/main.tex}
            \input{2 - kinetic component/main.tex}
            \input{3 - fluid component/main.tex}
            \input{4 - numerical implementation/main.tex}
        \part{Project Overview}
            \input{5 - research plan/main.tex}
            \input{6 - summary/main.tex}
    
    
    %\section{}
    \newpage
    \pagenumbering{gobble}
        \printbibliography


    \newpage
    \pagenumbering{roman}
    \appendix
        \part{Appendices}
            \input{8 - Hilbert complexes/main.tex}
            \input{9 - weak conservation proofs/main.tex}
\end{document}

\end{document}

\end{document}

    \documentclass[12pt, a4paper]{report}

\documentclass[12pt, a4paper]{report}

\documentclass[12pt, a4paper]{report}

\input{template/main.tex}

\title{\BA{Title in Progress...}}
\author{Boris Andrews}
\affil{Mathematical Institute, University of Oxford}
\date{\today}


\begin{document}
    \pagenumbering{gobble}
    \maketitle
    
    
    \begin{abstract}
        Magnetic confinement reactors---in particular tokamaks---offer one of the most promising options for achieving practical nuclear fusion, with the potential to provide virtually limitless, clean energy. The theoretical and numerical modeling of tokamak plasmas is simultaneously an essential component of effective reactor design, and a great research barrier. Tokamak operational conditions exhibit comparatively low Knudsen numbers. Kinetic effects, including kinetic waves and instabilities, Landau damping, bump-on-tail instabilities and more, are therefore highly influential in tokamak plasma dynamics. Purely fluid models are inherently incapable of capturing these effects, whereas the high dimensionality in purely kinetic models render them practically intractable for most relevant purposes.

        We consider a $\delta\!f$ decomposition model, with a macroscopic fluid background and microscopic kinetic correction, both fully coupled to each other. A similar manner of discretization is proposed to that used in the recent \texttt{STRUPHY} code \cite{Holderied_Possanner_Wang_2021, Holderied_2022, Li_et_al_2023} with a finite-element model for the background and a pseudo-particle/PiC model for the correction.

        The fluid background satisfies the full, non-linear, resistive, compressible, Hall MHD equations. \cite{Laakmann_Hu_Farrell_2022} introduces finite-element(-in-space) implicit timesteppers for the incompressible analogue to this system with structure-preserving (SP) properties in the ideal case, alongside parameter-robust preconditioners. We show that these timesteppers can derive from a finite-element-in-time (FET) (and finite-element-in-space) interpretation. The benefits of this reformulation are discussed, including the derivation of timesteppers that are higher order in time, and the quantifiable dissipative SP properties in the non-ideal, resistive case.
        
        We discuss possible options for extending this FET approach to timesteppers for the compressible case.

        The kinetic corrections satisfy linearized Boltzmann equations. Using a Lénard--Bernstein collision operator, these take Fokker--Planck-like forms \cite{Fokker_1914, Planck_1917} wherein pseudo-particles in the numerical model obey the neoclassical transport equations, with particle-independent Brownian drift terms. This offers a rigorous methodology for incorporating collisions into the particle transport model, without coupling the equations of motions for each particle.
        
        Works by Chen, Chacón et al. \cite{Chen_Chacón_Barnes_2011, Chacón_Chen_Barnes_2013, Chen_Chacón_2014, Chen_Chacón_2015} have developed structure-preserving particle pushers for neoclassical transport in the Vlasov equations, derived from Crank--Nicolson integrators. We show these too can can derive from a FET interpretation, similarly offering potential extensions to higher-order-in-time particle pushers. The FET formulation is used also to consider how the stochastic drift terms can be incorporated into the pushers. Stochastic gyrokinetic expansions are also discussed.

        Different options for the numerical implementation of these schemes are considered.

        Due to the efficacy of FET in the development of SP timesteppers for both the fluid and kinetic component, we hope this approach will prove effective in the future for developing SP timesteppers for the full hybrid model. We hope this will give us the opportunity to incorporate previously inaccessible kinetic effects into the highly effective, modern, finite-element MHD models.
    \end{abstract}
    
    
    \newpage
    \tableofcontents
    
    
    \newpage
    \pagenumbering{arabic}
    %\linenumbers\renewcommand\thelinenumber{\color{black!50}\arabic{linenumber}}
            \input{0 - introduction/main.tex}
        \part{Research}
            \input{1 - low-noise PiC models/main.tex}
            \input{2 - kinetic component/main.tex}
            \input{3 - fluid component/main.tex}
            \input{4 - numerical implementation/main.tex}
        \part{Project Overview}
            \input{5 - research plan/main.tex}
            \input{6 - summary/main.tex}
    
    
    %\section{}
    \newpage
    \pagenumbering{gobble}
        \printbibliography


    \newpage
    \pagenumbering{roman}
    \appendix
        \part{Appendices}
            \input{8 - Hilbert complexes/main.tex}
            \input{9 - weak conservation proofs/main.tex}
\end{document}


\title{\BA{Title in Progress...}}
\author{Boris Andrews}
\affil{Mathematical Institute, University of Oxford}
\date{\today}


\begin{document}
    \pagenumbering{gobble}
    \maketitle
    
    
    \begin{abstract}
        Magnetic confinement reactors---in particular tokamaks---offer one of the most promising options for achieving practical nuclear fusion, with the potential to provide virtually limitless, clean energy. The theoretical and numerical modeling of tokamak plasmas is simultaneously an essential component of effective reactor design, and a great research barrier. Tokamak operational conditions exhibit comparatively low Knudsen numbers. Kinetic effects, including kinetic waves and instabilities, Landau damping, bump-on-tail instabilities and more, are therefore highly influential in tokamak plasma dynamics. Purely fluid models are inherently incapable of capturing these effects, whereas the high dimensionality in purely kinetic models render them practically intractable for most relevant purposes.

        We consider a $\delta\!f$ decomposition model, with a macroscopic fluid background and microscopic kinetic correction, both fully coupled to each other. A similar manner of discretization is proposed to that used in the recent \texttt{STRUPHY} code \cite{Holderied_Possanner_Wang_2021, Holderied_2022, Li_et_al_2023} with a finite-element model for the background and a pseudo-particle/PiC model for the correction.

        The fluid background satisfies the full, non-linear, resistive, compressible, Hall MHD equations. \cite{Laakmann_Hu_Farrell_2022} introduces finite-element(-in-space) implicit timesteppers for the incompressible analogue to this system with structure-preserving (SP) properties in the ideal case, alongside parameter-robust preconditioners. We show that these timesteppers can derive from a finite-element-in-time (FET) (and finite-element-in-space) interpretation. The benefits of this reformulation are discussed, including the derivation of timesteppers that are higher order in time, and the quantifiable dissipative SP properties in the non-ideal, resistive case.
        
        We discuss possible options for extending this FET approach to timesteppers for the compressible case.

        The kinetic corrections satisfy linearized Boltzmann equations. Using a Lénard--Bernstein collision operator, these take Fokker--Planck-like forms \cite{Fokker_1914, Planck_1917} wherein pseudo-particles in the numerical model obey the neoclassical transport equations, with particle-independent Brownian drift terms. This offers a rigorous methodology for incorporating collisions into the particle transport model, without coupling the equations of motions for each particle.
        
        Works by Chen, Chacón et al. \cite{Chen_Chacón_Barnes_2011, Chacón_Chen_Barnes_2013, Chen_Chacón_2014, Chen_Chacón_2015} have developed structure-preserving particle pushers for neoclassical transport in the Vlasov equations, derived from Crank--Nicolson integrators. We show these too can can derive from a FET interpretation, similarly offering potential extensions to higher-order-in-time particle pushers. The FET formulation is used also to consider how the stochastic drift terms can be incorporated into the pushers. Stochastic gyrokinetic expansions are also discussed.

        Different options for the numerical implementation of these schemes are considered.

        Due to the efficacy of FET in the development of SP timesteppers for both the fluid and kinetic component, we hope this approach will prove effective in the future for developing SP timesteppers for the full hybrid model. We hope this will give us the opportunity to incorporate previously inaccessible kinetic effects into the highly effective, modern, finite-element MHD models.
    \end{abstract}
    
    
    \newpage
    \tableofcontents
    
    
    \newpage
    \pagenumbering{arabic}
    %\linenumbers\renewcommand\thelinenumber{\color{black!50}\arabic{linenumber}}
            \documentclass[12pt, a4paper]{report}

\input{template/main.tex}

\title{\BA{Title in Progress...}}
\author{Boris Andrews}
\affil{Mathematical Institute, University of Oxford}
\date{\today}


\begin{document}
    \pagenumbering{gobble}
    \maketitle
    
    
    \begin{abstract}
        Magnetic confinement reactors---in particular tokamaks---offer one of the most promising options for achieving practical nuclear fusion, with the potential to provide virtually limitless, clean energy. The theoretical and numerical modeling of tokamak plasmas is simultaneously an essential component of effective reactor design, and a great research barrier. Tokamak operational conditions exhibit comparatively low Knudsen numbers. Kinetic effects, including kinetic waves and instabilities, Landau damping, bump-on-tail instabilities and more, are therefore highly influential in tokamak plasma dynamics. Purely fluid models are inherently incapable of capturing these effects, whereas the high dimensionality in purely kinetic models render them practically intractable for most relevant purposes.

        We consider a $\delta\!f$ decomposition model, with a macroscopic fluid background and microscopic kinetic correction, both fully coupled to each other. A similar manner of discretization is proposed to that used in the recent \texttt{STRUPHY} code \cite{Holderied_Possanner_Wang_2021, Holderied_2022, Li_et_al_2023} with a finite-element model for the background and a pseudo-particle/PiC model for the correction.

        The fluid background satisfies the full, non-linear, resistive, compressible, Hall MHD equations. \cite{Laakmann_Hu_Farrell_2022} introduces finite-element(-in-space) implicit timesteppers for the incompressible analogue to this system with structure-preserving (SP) properties in the ideal case, alongside parameter-robust preconditioners. We show that these timesteppers can derive from a finite-element-in-time (FET) (and finite-element-in-space) interpretation. The benefits of this reformulation are discussed, including the derivation of timesteppers that are higher order in time, and the quantifiable dissipative SP properties in the non-ideal, resistive case.
        
        We discuss possible options for extending this FET approach to timesteppers for the compressible case.

        The kinetic corrections satisfy linearized Boltzmann equations. Using a Lénard--Bernstein collision operator, these take Fokker--Planck-like forms \cite{Fokker_1914, Planck_1917} wherein pseudo-particles in the numerical model obey the neoclassical transport equations, with particle-independent Brownian drift terms. This offers a rigorous methodology for incorporating collisions into the particle transport model, without coupling the equations of motions for each particle.
        
        Works by Chen, Chacón et al. \cite{Chen_Chacón_Barnes_2011, Chacón_Chen_Barnes_2013, Chen_Chacón_2014, Chen_Chacón_2015} have developed structure-preserving particle pushers for neoclassical transport in the Vlasov equations, derived from Crank--Nicolson integrators. We show these too can can derive from a FET interpretation, similarly offering potential extensions to higher-order-in-time particle pushers. The FET formulation is used also to consider how the stochastic drift terms can be incorporated into the pushers. Stochastic gyrokinetic expansions are also discussed.

        Different options for the numerical implementation of these schemes are considered.

        Due to the efficacy of FET in the development of SP timesteppers for both the fluid and kinetic component, we hope this approach will prove effective in the future for developing SP timesteppers for the full hybrid model. We hope this will give us the opportunity to incorporate previously inaccessible kinetic effects into the highly effective, modern, finite-element MHD models.
    \end{abstract}
    
    
    \newpage
    \tableofcontents
    
    
    \newpage
    \pagenumbering{arabic}
    %\linenumbers\renewcommand\thelinenumber{\color{black!50}\arabic{linenumber}}
            \input{0 - introduction/main.tex}
        \part{Research}
            \input{1 - low-noise PiC models/main.tex}
            \input{2 - kinetic component/main.tex}
            \input{3 - fluid component/main.tex}
            \input{4 - numerical implementation/main.tex}
        \part{Project Overview}
            \input{5 - research plan/main.tex}
            \input{6 - summary/main.tex}
    
    
    %\section{}
    \newpage
    \pagenumbering{gobble}
        \printbibliography


    \newpage
    \pagenumbering{roman}
    \appendix
        \part{Appendices}
            \input{8 - Hilbert complexes/main.tex}
            \input{9 - weak conservation proofs/main.tex}
\end{document}

        \part{Research}
            \documentclass[12pt, a4paper]{report}

\input{template/main.tex}

\title{\BA{Title in Progress...}}
\author{Boris Andrews}
\affil{Mathematical Institute, University of Oxford}
\date{\today}


\begin{document}
    \pagenumbering{gobble}
    \maketitle
    
    
    \begin{abstract}
        Magnetic confinement reactors---in particular tokamaks---offer one of the most promising options for achieving practical nuclear fusion, with the potential to provide virtually limitless, clean energy. The theoretical and numerical modeling of tokamak plasmas is simultaneously an essential component of effective reactor design, and a great research barrier. Tokamak operational conditions exhibit comparatively low Knudsen numbers. Kinetic effects, including kinetic waves and instabilities, Landau damping, bump-on-tail instabilities and more, are therefore highly influential in tokamak plasma dynamics. Purely fluid models are inherently incapable of capturing these effects, whereas the high dimensionality in purely kinetic models render them practically intractable for most relevant purposes.

        We consider a $\delta\!f$ decomposition model, with a macroscopic fluid background and microscopic kinetic correction, both fully coupled to each other. A similar manner of discretization is proposed to that used in the recent \texttt{STRUPHY} code \cite{Holderied_Possanner_Wang_2021, Holderied_2022, Li_et_al_2023} with a finite-element model for the background and a pseudo-particle/PiC model for the correction.

        The fluid background satisfies the full, non-linear, resistive, compressible, Hall MHD equations. \cite{Laakmann_Hu_Farrell_2022} introduces finite-element(-in-space) implicit timesteppers for the incompressible analogue to this system with structure-preserving (SP) properties in the ideal case, alongside parameter-robust preconditioners. We show that these timesteppers can derive from a finite-element-in-time (FET) (and finite-element-in-space) interpretation. The benefits of this reformulation are discussed, including the derivation of timesteppers that are higher order in time, and the quantifiable dissipative SP properties in the non-ideal, resistive case.
        
        We discuss possible options for extending this FET approach to timesteppers for the compressible case.

        The kinetic corrections satisfy linearized Boltzmann equations. Using a Lénard--Bernstein collision operator, these take Fokker--Planck-like forms \cite{Fokker_1914, Planck_1917} wherein pseudo-particles in the numerical model obey the neoclassical transport equations, with particle-independent Brownian drift terms. This offers a rigorous methodology for incorporating collisions into the particle transport model, without coupling the equations of motions for each particle.
        
        Works by Chen, Chacón et al. \cite{Chen_Chacón_Barnes_2011, Chacón_Chen_Barnes_2013, Chen_Chacón_2014, Chen_Chacón_2015} have developed structure-preserving particle pushers for neoclassical transport in the Vlasov equations, derived from Crank--Nicolson integrators. We show these too can can derive from a FET interpretation, similarly offering potential extensions to higher-order-in-time particle pushers. The FET formulation is used also to consider how the stochastic drift terms can be incorporated into the pushers. Stochastic gyrokinetic expansions are also discussed.

        Different options for the numerical implementation of these schemes are considered.

        Due to the efficacy of FET in the development of SP timesteppers for both the fluid and kinetic component, we hope this approach will prove effective in the future for developing SP timesteppers for the full hybrid model. We hope this will give us the opportunity to incorporate previously inaccessible kinetic effects into the highly effective, modern, finite-element MHD models.
    \end{abstract}
    
    
    \newpage
    \tableofcontents
    
    
    \newpage
    \pagenumbering{arabic}
    %\linenumbers\renewcommand\thelinenumber{\color{black!50}\arabic{linenumber}}
            \input{0 - introduction/main.tex}
        \part{Research}
            \input{1 - low-noise PiC models/main.tex}
            \input{2 - kinetic component/main.tex}
            \input{3 - fluid component/main.tex}
            \input{4 - numerical implementation/main.tex}
        \part{Project Overview}
            \input{5 - research plan/main.tex}
            \input{6 - summary/main.tex}
    
    
    %\section{}
    \newpage
    \pagenumbering{gobble}
        \printbibliography


    \newpage
    \pagenumbering{roman}
    \appendix
        \part{Appendices}
            \input{8 - Hilbert complexes/main.tex}
            \input{9 - weak conservation proofs/main.tex}
\end{document}

            \documentclass[12pt, a4paper]{report}

\input{template/main.tex}

\title{\BA{Title in Progress...}}
\author{Boris Andrews}
\affil{Mathematical Institute, University of Oxford}
\date{\today}


\begin{document}
    \pagenumbering{gobble}
    \maketitle
    
    
    \begin{abstract}
        Magnetic confinement reactors---in particular tokamaks---offer one of the most promising options for achieving practical nuclear fusion, with the potential to provide virtually limitless, clean energy. The theoretical and numerical modeling of tokamak plasmas is simultaneously an essential component of effective reactor design, and a great research barrier. Tokamak operational conditions exhibit comparatively low Knudsen numbers. Kinetic effects, including kinetic waves and instabilities, Landau damping, bump-on-tail instabilities and more, are therefore highly influential in tokamak plasma dynamics. Purely fluid models are inherently incapable of capturing these effects, whereas the high dimensionality in purely kinetic models render them practically intractable for most relevant purposes.

        We consider a $\delta\!f$ decomposition model, with a macroscopic fluid background and microscopic kinetic correction, both fully coupled to each other. A similar manner of discretization is proposed to that used in the recent \texttt{STRUPHY} code \cite{Holderied_Possanner_Wang_2021, Holderied_2022, Li_et_al_2023} with a finite-element model for the background and a pseudo-particle/PiC model for the correction.

        The fluid background satisfies the full, non-linear, resistive, compressible, Hall MHD equations. \cite{Laakmann_Hu_Farrell_2022} introduces finite-element(-in-space) implicit timesteppers for the incompressible analogue to this system with structure-preserving (SP) properties in the ideal case, alongside parameter-robust preconditioners. We show that these timesteppers can derive from a finite-element-in-time (FET) (and finite-element-in-space) interpretation. The benefits of this reformulation are discussed, including the derivation of timesteppers that are higher order in time, and the quantifiable dissipative SP properties in the non-ideal, resistive case.
        
        We discuss possible options for extending this FET approach to timesteppers for the compressible case.

        The kinetic corrections satisfy linearized Boltzmann equations. Using a Lénard--Bernstein collision operator, these take Fokker--Planck-like forms \cite{Fokker_1914, Planck_1917} wherein pseudo-particles in the numerical model obey the neoclassical transport equations, with particle-independent Brownian drift terms. This offers a rigorous methodology for incorporating collisions into the particle transport model, without coupling the equations of motions for each particle.
        
        Works by Chen, Chacón et al. \cite{Chen_Chacón_Barnes_2011, Chacón_Chen_Barnes_2013, Chen_Chacón_2014, Chen_Chacón_2015} have developed structure-preserving particle pushers for neoclassical transport in the Vlasov equations, derived from Crank--Nicolson integrators. We show these too can can derive from a FET interpretation, similarly offering potential extensions to higher-order-in-time particle pushers. The FET formulation is used also to consider how the stochastic drift terms can be incorporated into the pushers. Stochastic gyrokinetic expansions are also discussed.

        Different options for the numerical implementation of these schemes are considered.

        Due to the efficacy of FET in the development of SP timesteppers for both the fluid and kinetic component, we hope this approach will prove effective in the future for developing SP timesteppers for the full hybrid model. We hope this will give us the opportunity to incorporate previously inaccessible kinetic effects into the highly effective, modern, finite-element MHD models.
    \end{abstract}
    
    
    \newpage
    \tableofcontents
    
    
    \newpage
    \pagenumbering{arabic}
    %\linenumbers\renewcommand\thelinenumber{\color{black!50}\arabic{linenumber}}
            \input{0 - introduction/main.tex}
        \part{Research}
            \input{1 - low-noise PiC models/main.tex}
            \input{2 - kinetic component/main.tex}
            \input{3 - fluid component/main.tex}
            \input{4 - numerical implementation/main.tex}
        \part{Project Overview}
            \input{5 - research plan/main.tex}
            \input{6 - summary/main.tex}
    
    
    %\section{}
    \newpage
    \pagenumbering{gobble}
        \printbibliography


    \newpage
    \pagenumbering{roman}
    \appendix
        \part{Appendices}
            \input{8 - Hilbert complexes/main.tex}
            \input{9 - weak conservation proofs/main.tex}
\end{document}

            \documentclass[12pt, a4paper]{report}

\input{template/main.tex}

\title{\BA{Title in Progress...}}
\author{Boris Andrews}
\affil{Mathematical Institute, University of Oxford}
\date{\today}


\begin{document}
    \pagenumbering{gobble}
    \maketitle
    
    
    \begin{abstract}
        Magnetic confinement reactors---in particular tokamaks---offer one of the most promising options for achieving practical nuclear fusion, with the potential to provide virtually limitless, clean energy. The theoretical and numerical modeling of tokamak plasmas is simultaneously an essential component of effective reactor design, and a great research barrier. Tokamak operational conditions exhibit comparatively low Knudsen numbers. Kinetic effects, including kinetic waves and instabilities, Landau damping, bump-on-tail instabilities and more, are therefore highly influential in tokamak plasma dynamics. Purely fluid models are inherently incapable of capturing these effects, whereas the high dimensionality in purely kinetic models render them practically intractable for most relevant purposes.

        We consider a $\delta\!f$ decomposition model, with a macroscopic fluid background and microscopic kinetic correction, both fully coupled to each other. A similar manner of discretization is proposed to that used in the recent \texttt{STRUPHY} code \cite{Holderied_Possanner_Wang_2021, Holderied_2022, Li_et_al_2023} with a finite-element model for the background and a pseudo-particle/PiC model for the correction.

        The fluid background satisfies the full, non-linear, resistive, compressible, Hall MHD equations. \cite{Laakmann_Hu_Farrell_2022} introduces finite-element(-in-space) implicit timesteppers for the incompressible analogue to this system with structure-preserving (SP) properties in the ideal case, alongside parameter-robust preconditioners. We show that these timesteppers can derive from a finite-element-in-time (FET) (and finite-element-in-space) interpretation. The benefits of this reformulation are discussed, including the derivation of timesteppers that are higher order in time, and the quantifiable dissipative SP properties in the non-ideal, resistive case.
        
        We discuss possible options for extending this FET approach to timesteppers for the compressible case.

        The kinetic corrections satisfy linearized Boltzmann equations. Using a Lénard--Bernstein collision operator, these take Fokker--Planck-like forms \cite{Fokker_1914, Planck_1917} wherein pseudo-particles in the numerical model obey the neoclassical transport equations, with particle-independent Brownian drift terms. This offers a rigorous methodology for incorporating collisions into the particle transport model, without coupling the equations of motions for each particle.
        
        Works by Chen, Chacón et al. \cite{Chen_Chacón_Barnes_2011, Chacón_Chen_Barnes_2013, Chen_Chacón_2014, Chen_Chacón_2015} have developed structure-preserving particle pushers for neoclassical transport in the Vlasov equations, derived from Crank--Nicolson integrators. We show these too can can derive from a FET interpretation, similarly offering potential extensions to higher-order-in-time particle pushers. The FET formulation is used also to consider how the stochastic drift terms can be incorporated into the pushers. Stochastic gyrokinetic expansions are also discussed.

        Different options for the numerical implementation of these schemes are considered.

        Due to the efficacy of FET in the development of SP timesteppers for both the fluid and kinetic component, we hope this approach will prove effective in the future for developing SP timesteppers for the full hybrid model. We hope this will give us the opportunity to incorporate previously inaccessible kinetic effects into the highly effective, modern, finite-element MHD models.
    \end{abstract}
    
    
    \newpage
    \tableofcontents
    
    
    \newpage
    \pagenumbering{arabic}
    %\linenumbers\renewcommand\thelinenumber{\color{black!50}\arabic{linenumber}}
            \input{0 - introduction/main.tex}
        \part{Research}
            \input{1 - low-noise PiC models/main.tex}
            \input{2 - kinetic component/main.tex}
            \input{3 - fluid component/main.tex}
            \input{4 - numerical implementation/main.tex}
        \part{Project Overview}
            \input{5 - research plan/main.tex}
            \input{6 - summary/main.tex}
    
    
    %\section{}
    \newpage
    \pagenumbering{gobble}
        \printbibliography


    \newpage
    \pagenumbering{roman}
    \appendix
        \part{Appendices}
            \input{8 - Hilbert complexes/main.tex}
            \input{9 - weak conservation proofs/main.tex}
\end{document}

            \documentclass[12pt, a4paper]{report}

\input{template/main.tex}

\title{\BA{Title in Progress...}}
\author{Boris Andrews}
\affil{Mathematical Institute, University of Oxford}
\date{\today}


\begin{document}
    \pagenumbering{gobble}
    \maketitle
    
    
    \begin{abstract}
        Magnetic confinement reactors---in particular tokamaks---offer one of the most promising options for achieving practical nuclear fusion, with the potential to provide virtually limitless, clean energy. The theoretical and numerical modeling of tokamak plasmas is simultaneously an essential component of effective reactor design, and a great research barrier. Tokamak operational conditions exhibit comparatively low Knudsen numbers. Kinetic effects, including kinetic waves and instabilities, Landau damping, bump-on-tail instabilities and more, are therefore highly influential in tokamak plasma dynamics. Purely fluid models are inherently incapable of capturing these effects, whereas the high dimensionality in purely kinetic models render them practically intractable for most relevant purposes.

        We consider a $\delta\!f$ decomposition model, with a macroscopic fluid background and microscopic kinetic correction, both fully coupled to each other. A similar manner of discretization is proposed to that used in the recent \texttt{STRUPHY} code \cite{Holderied_Possanner_Wang_2021, Holderied_2022, Li_et_al_2023} with a finite-element model for the background and a pseudo-particle/PiC model for the correction.

        The fluid background satisfies the full, non-linear, resistive, compressible, Hall MHD equations. \cite{Laakmann_Hu_Farrell_2022} introduces finite-element(-in-space) implicit timesteppers for the incompressible analogue to this system with structure-preserving (SP) properties in the ideal case, alongside parameter-robust preconditioners. We show that these timesteppers can derive from a finite-element-in-time (FET) (and finite-element-in-space) interpretation. The benefits of this reformulation are discussed, including the derivation of timesteppers that are higher order in time, and the quantifiable dissipative SP properties in the non-ideal, resistive case.
        
        We discuss possible options for extending this FET approach to timesteppers for the compressible case.

        The kinetic corrections satisfy linearized Boltzmann equations. Using a Lénard--Bernstein collision operator, these take Fokker--Planck-like forms \cite{Fokker_1914, Planck_1917} wherein pseudo-particles in the numerical model obey the neoclassical transport equations, with particle-independent Brownian drift terms. This offers a rigorous methodology for incorporating collisions into the particle transport model, without coupling the equations of motions for each particle.
        
        Works by Chen, Chacón et al. \cite{Chen_Chacón_Barnes_2011, Chacón_Chen_Barnes_2013, Chen_Chacón_2014, Chen_Chacón_2015} have developed structure-preserving particle pushers for neoclassical transport in the Vlasov equations, derived from Crank--Nicolson integrators. We show these too can can derive from a FET interpretation, similarly offering potential extensions to higher-order-in-time particle pushers. The FET formulation is used also to consider how the stochastic drift terms can be incorporated into the pushers. Stochastic gyrokinetic expansions are also discussed.

        Different options for the numerical implementation of these schemes are considered.

        Due to the efficacy of FET in the development of SP timesteppers for both the fluid and kinetic component, we hope this approach will prove effective in the future for developing SP timesteppers for the full hybrid model. We hope this will give us the opportunity to incorporate previously inaccessible kinetic effects into the highly effective, modern, finite-element MHD models.
    \end{abstract}
    
    
    \newpage
    \tableofcontents
    
    
    \newpage
    \pagenumbering{arabic}
    %\linenumbers\renewcommand\thelinenumber{\color{black!50}\arabic{linenumber}}
            \input{0 - introduction/main.tex}
        \part{Research}
            \input{1 - low-noise PiC models/main.tex}
            \input{2 - kinetic component/main.tex}
            \input{3 - fluid component/main.tex}
            \input{4 - numerical implementation/main.tex}
        \part{Project Overview}
            \input{5 - research plan/main.tex}
            \input{6 - summary/main.tex}
    
    
    %\section{}
    \newpage
    \pagenumbering{gobble}
        \printbibliography


    \newpage
    \pagenumbering{roman}
    \appendix
        \part{Appendices}
            \input{8 - Hilbert complexes/main.tex}
            \input{9 - weak conservation proofs/main.tex}
\end{document}

        \part{Project Overview}
            \documentclass[12pt, a4paper]{report}

\input{template/main.tex}

\title{\BA{Title in Progress...}}
\author{Boris Andrews}
\affil{Mathematical Institute, University of Oxford}
\date{\today}


\begin{document}
    \pagenumbering{gobble}
    \maketitle
    
    
    \begin{abstract}
        Magnetic confinement reactors---in particular tokamaks---offer one of the most promising options for achieving practical nuclear fusion, with the potential to provide virtually limitless, clean energy. The theoretical and numerical modeling of tokamak plasmas is simultaneously an essential component of effective reactor design, and a great research barrier. Tokamak operational conditions exhibit comparatively low Knudsen numbers. Kinetic effects, including kinetic waves and instabilities, Landau damping, bump-on-tail instabilities and more, are therefore highly influential in tokamak plasma dynamics. Purely fluid models are inherently incapable of capturing these effects, whereas the high dimensionality in purely kinetic models render them practically intractable for most relevant purposes.

        We consider a $\delta\!f$ decomposition model, with a macroscopic fluid background and microscopic kinetic correction, both fully coupled to each other. A similar manner of discretization is proposed to that used in the recent \texttt{STRUPHY} code \cite{Holderied_Possanner_Wang_2021, Holderied_2022, Li_et_al_2023} with a finite-element model for the background and a pseudo-particle/PiC model for the correction.

        The fluid background satisfies the full, non-linear, resistive, compressible, Hall MHD equations. \cite{Laakmann_Hu_Farrell_2022} introduces finite-element(-in-space) implicit timesteppers for the incompressible analogue to this system with structure-preserving (SP) properties in the ideal case, alongside parameter-robust preconditioners. We show that these timesteppers can derive from a finite-element-in-time (FET) (and finite-element-in-space) interpretation. The benefits of this reformulation are discussed, including the derivation of timesteppers that are higher order in time, and the quantifiable dissipative SP properties in the non-ideal, resistive case.
        
        We discuss possible options for extending this FET approach to timesteppers for the compressible case.

        The kinetic corrections satisfy linearized Boltzmann equations. Using a Lénard--Bernstein collision operator, these take Fokker--Planck-like forms \cite{Fokker_1914, Planck_1917} wherein pseudo-particles in the numerical model obey the neoclassical transport equations, with particle-independent Brownian drift terms. This offers a rigorous methodology for incorporating collisions into the particle transport model, without coupling the equations of motions for each particle.
        
        Works by Chen, Chacón et al. \cite{Chen_Chacón_Barnes_2011, Chacón_Chen_Barnes_2013, Chen_Chacón_2014, Chen_Chacón_2015} have developed structure-preserving particle pushers for neoclassical transport in the Vlasov equations, derived from Crank--Nicolson integrators. We show these too can can derive from a FET interpretation, similarly offering potential extensions to higher-order-in-time particle pushers. The FET formulation is used also to consider how the stochastic drift terms can be incorporated into the pushers. Stochastic gyrokinetic expansions are also discussed.

        Different options for the numerical implementation of these schemes are considered.

        Due to the efficacy of FET in the development of SP timesteppers for both the fluid and kinetic component, we hope this approach will prove effective in the future for developing SP timesteppers for the full hybrid model. We hope this will give us the opportunity to incorporate previously inaccessible kinetic effects into the highly effective, modern, finite-element MHD models.
    \end{abstract}
    
    
    \newpage
    \tableofcontents
    
    
    \newpage
    \pagenumbering{arabic}
    %\linenumbers\renewcommand\thelinenumber{\color{black!50}\arabic{linenumber}}
            \input{0 - introduction/main.tex}
        \part{Research}
            \input{1 - low-noise PiC models/main.tex}
            \input{2 - kinetic component/main.tex}
            \input{3 - fluid component/main.tex}
            \input{4 - numerical implementation/main.tex}
        \part{Project Overview}
            \input{5 - research plan/main.tex}
            \input{6 - summary/main.tex}
    
    
    %\section{}
    \newpage
    \pagenumbering{gobble}
        \printbibliography


    \newpage
    \pagenumbering{roman}
    \appendix
        \part{Appendices}
            \input{8 - Hilbert complexes/main.tex}
            \input{9 - weak conservation proofs/main.tex}
\end{document}

            \documentclass[12pt, a4paper]{report}

\input{template/main.tex}

\title{\BA{Title in Progress...}}
\author{Boris Andrews}
\affil{Mathematical Institute, University of Oxford}
\date{\today}


\begin{document}
    \pagenumbering{gobble}
    \maketitle
    
    
    \begin{abstract}
        Magnetic confinement reactors---in particular tokamaks---offer one of the most promising options for achieving practical nuclear fusion, with the potential to provide virtually limitless, clean energy. The theoretical and numerical modeling of tokamak plasmas is simultaneously an essential component of effective reactor design, and a great research barrier. Tokamak operational conditions exhibit comparatively low Knudsen numbers. Kinetic effects, including kinetic waves and instabilities, Landau damping, bump-on-tail instabilities and more, are therefore highly influential in tokamak plasma dynamics. Purely fluid models are inherently incapable of capturing these effects, whereas the high dimensionality in purely kinetic models render them practically intractable for most relevant purposes.

        We consider a $\delta\!f$ decomposition model, with a macroscopic fluid background and microscopic kinetic correction, both fully coupled to each other. A similar manner of discretization is proposed to that used in the recent \texttt{STRUPHY} code \cite{Holderied_Possanner_Wang_2021, Holderied_2022, Li_et_al_2023} with a finite-element model for the background and a pseudo-particle/PiC model for the correction.

        The fluid background satisfies the full, non-linear, resistive, compressible, Hall MHD equations. \cite{Laakmann_Hu_Farrell_2022} introduces finite-element(-in-space) implicit timesteppers for the incompressible analogue to this system with structure-preserving (SP) properties in the ideal case, alongside parameter-robust preconditioners. We show that these timesteppers can derive from a finite-element-in-time (FET) (and finite-element-in-space) interpretation. The benefits of this reformulation are discussed, including the derivation of timesteppers that are higher order in time, and the quantifiable dissipative SP properties in the non-ideal, resistive case.
        
        We discuss possible options for extending this FET approach to timesteppers for the compressible case.

        The kinetic corrections satisfy linearized Boltzmann equations. Using a Lénard--Bernstein collision operator, these take Fokker--Planck-like forms \cite{Fokker_1914, Planck_1917} wherein pseudo-particles in the numerical model obey the neoclassical transport equations, with particle-independent Brownian drift terms. This offers a rigorous methodology for incorporating collisions into the particle transport model, without coupling the equations of motions for each particle.
        
        Works by Chen, Chacón et al. \cite{Chen_Chacón_Barnes_2011, Chacón_Chen_Barnes_2013, Chen_Chacón_2014, Chen_Chacón_2015} have developed structure-preserving particle pushers for neoclassical transport in the Vlasov equations, derived from Crank--Nicolson integrators. We show these too can can derive from a FET interpretation, similarly offering potential extensions to higher-order-in-time particle pushers. The FET formulation is used also to consider how the stochastic drift terms can be incorporated into the pushers. Stochastic gyrokinetic expansions are also discussed.

        Different options for the numerical implementation of these schemes are considered.

        Due to the efficacy of FET in the development of SP timesteppers for both the fluid and kinetic component, we hope this approach will prove effective in the future for developing SP timesteppers for the full hybrid model. We hope this will give us the opportunity to incorporate previously inaccessible kinetic effects into the highly effective, modern, finite-element MHD models.
    \end{abstract}
    
    
    \newpage
    \tableofcontents
    
    
    \newpage
    \pagenumbering{arabic}
    %\linenumbers\renewcommand\thelinenumber{\color{black!50}\arabic{linenumber}}
            \input{0 - introduction/main.tex}
        \part{Research}
            \input{1 - low-noise PiC models/main.tex}
            \input{2 - kinetic component/main.tex}
            \input{3 - fluid component/main.tex}
            \input{4 - numerical implementation/main.tex}
        \part{Project Overview}
            \input{5 - research plan/main.tex}
            \input{6 - summary/main.tex}
    
    
    %\section{}
    \newpage
    \pagenumbering{gobble}
        \printbibliography


    \newpage
    \pagenumbering{roman}
    \appendix
        \part{Appendices}
            \input{8 - Hilbert complexes/main.tex}
            \input{9 - weak conservation proofs/main.tex}
\end{document}

    
    
    %\section{}
    \newpage
    \pagenumbering{gobble}
        \printbibliography


    \newpage
    \pagenumbering{roman}
    \appendix
        \part{Appendices}
            \documentclass[12pt, a4paper]{report}

\input{template/main.tex}

\title{\BA{Title in Progress...}}
\author{Boris Andrews}
\affil{Mathematical Institute, University of Oxford}
\date{\today}


\begin{document}
    \pagenumbering{gobble}
    \maketitle
    
    
    \begin{abstract}
        Magnetic confinement reactors---in particular tokamaks---offer one of the most promising options for achieving practical nuclear fusion, with the potential to provide virtually limitless, clean energy. The theoretical and numerical modeling of tokamak plasmas is simultaneously an essential component of effective reactor design, and a great research barrier. Tokamak operational conditions exhibit comparatively low Knudsen numbers. Kinetic effects, including kinetic waves and instabilities, Landau damping, bump-on-tail instabilities and more, are therefore highly influential in tokamak plasma dynamics. Purely fluid models are inherently incapable of capturing these effects, whereas the high dimensionality in purely kinetic models render them practically intractable for most relevant purposes.

        We consider a $\delta\!f$ decomposition model, with a macroscopic fluid background and microscopic kinetic correction, both fully coupled to each other. A similar manner of discretization is proposed to that used in the recent \texttt{STRUPHY} code \cite{Holderied_Possanner_Wang_2021, Holderied_2022, Li_et_al_2023} with a finite-element model for the background and a pseudo-particle/PiC model for the correction.

        The fluid background satisfies the full, non-linear, resistive, compressible, Hall MHD equations. \cite{Laakmann_Hu_Farrell_2022} introduces finite-element(-in-space) implicit timesteppers for the incompressible analogue to this system with structure-preserving (SP) properties in the ideal case, alongside parameter-robust preconditioners. We show that these timesteppers can derive from a finite-element-in-time (FET) (and finite-element-in-space) interpretation. The benefits of this reformulation are discussed, including the derivation of timesteppers that are higher order in time, and the quantifiable dissipative SP properties in the non-ideal, resistive case.
        
        We discuss possible options for extending this FET approach to timesteppers for the compressible case.

        The kinetic corrections satisfy linearized Boltzmann equations. Using a Lénard--Bernstein collision operator, these take Fokker--Planck-like forms \cite{Fokker_1914, Planck_1917} wherein pseudo-particles in the numerical model obey the neoclassical transport equations, with particle-independent Brownian drift terms. This offers a rigorous methodology for incorporating collisions into the particle transport model, without coupling the equations of motions for each particle.
        
        Works by Chen, Chacón et al. \cite{Chen_Chacón_Barnes_2011, Chacón_Chen_Barnes_2013, Chen_Chacón_2014, Chen_Chacón_2015} have developed structure-preserving particle pushers for neoclassical transport in the Vlasov equations, derived from Crank--Nicolson integrators. We show these too can can derive from a FET interpretation, similarly offering potential extensions to higher-order-in-time particle pushers. The FET formulation is used also to consider how the stochastic drift terms can be incorporated into the pushers. Stochastic gyrokinetic expansions are also discussed.

        Different options for the numerical implementation of these schemes are considered.

        Due to the efficacy of FET in the development of SP timesteppers for both the fluid and kinetic component, we hope this approach will prove effective in the future for developing SP timesteppers for the full hybrid model. We hope this will give us the opportunity to incorporate previously inaccessible kinetic effects into the highly effective, modern, finite-element MHD models.
    \end{abstract}
    
    
    \newpage
    \tableofcontents
    
    
    \newpage
    \pagenumbering{arabic}
    %\linenumbers\renewcommand\thelinenumber{\color{black!50}\arabic{linenumber}}
            \input{0 - introduction/main.tex}
        \part{Research}
            \input{1 - low-noise PiC models/main.tex}
            \input{2 - kinetic component/main.tex}
            \input{3 - fluid component/main.tex}
            \input{4 - numerical implementation/main.tex}
        \part{Project Overview}
            \input{5 - research plan/main.tex}
            \input{6 - summary/main.tex}
    
    
    %\section{}
    \newpage
    \pagenumbering{gobble}
        \printbibliography


    \newpage
    \pagenumbering{roman}
    \appendix
        \part{Appendices}
            \input{8 - Hilbert complexes/main.tex}
            \input{9 - weak conservation proofs/main.tex}
\end{document}

            \documentclass[12pt, a4paper]{report}

\input{template/main.tex}

\title{\BA{Title in Progress...}}
\author{Boris Andrews}
\affil{Mathematical Institute, University of Oxford}
\date{\today}


\begin{document}
    \pagenumbering{gobble}
    \maketitle
    
    
    \begin{abstract}
        Magnetic confinement reactors---in particular tokamaks---offer one of the most promising options for achieving practical nuclear fusion, with the potential to provide virtually limitless, clean energy. The theoretical and numerical modeling of tokamak plasmas is simultaneously an essential component of effective reactor design, and a great research barrier. Tokamak operational conditions exhibit comparatively low Knudsen numbers. Kinetic effects, including kinetic waves and instabilities, Landau damping, bump-on-tail instabilities and more, are therefore highly influential in tokamak plasma dynamics. Purely fluid models are inherently incapable of capturing these effects, whereas the high dimensionality in purely kinetic models render them practically intractable for most relevant purposes.

        We consider a $\delta\!f$ decomposition model, with a macroscopic fluid background and microscopic kinetic correction, both fully coupled to each other. A similar manner of discretization is proposed to that used in the recent \texttt{STRUPHY} code \cite{Holderied_Possanner_Wang_2021, Holderied_2022, Li_et_al_2023} with a finite-element model for the background and a pseudo-particle/PiC model for the correction.

        The fluid background satisfies the full, non-linear, resistive, compressible, Hall MHD equations. \cite{Laakmann_Hu_Farrell_2022} introduces finite-element(-in-space) implicit timesteppers for the incompressible analogue to this system with structure-preserving (SP) properties in the ideal case, alongside parameter-robust preconditioners. We show that these timesteppers can derive from a finite-element-in-time (FET) (and finite-element-in-space) interpretation. The benefits of this reformulation are discussed, including the derivation of timesteppers that are higher order in time, and the quantifiable dissipative SP properties in the non-ideal, resistive case.
        
        We discuss possible options for extending this FET approach to timesteppers for the compressible case.

        The kinetic corrections satisfy linearized Boltzmann equations. Using a Lénard--Bernstein collision operator, these take Fokker--Planck-like forms \cite{Fokker_1914, Planck_1917} wherein pseudo-particles in the numerical model obey the neoclassical transport equations, with particle-independent Brownian drift terms. This offers a rigorous methodology for incorporating collisions into the particle transport model, without coupling the equations of motions for each particle.
        
        Works by Chen, Chacón et al. \cite{Chen_Chacón_Barnes_2011, Chacón_Chen_Barnes_2013, Chen_Chacón_2014, Chen_Chacón_2015} have developed structure-preserving particle pushers for neoclassical transport in the Vlasov equations, derived from Crank--Nicolson integrators. We show these too can can derive from a FET interpretation, similarly offering potential extensions to higher-order-in-time particle pushers. The FET formulation is used also to consider how the stochastic drift terms can be incorporated into the pushers. Stochastic gyrokinetic expansions are also discussed.

        Different options for the numerical implementation of these schemes are considered.

        Due to the efficacy of FET in the development of SP timesteppers for both the fluid and kinetic component, we hope this approach will prove effective in the future for developing SP timesteppers for the full hybrid model. We hope this will give us the opportunity to incorporate previously inaccessible kinetic effects into the highly effective, modern, finite-element MHD models.
    \end{abstract}
    
    
    \newpage
    \tableofcontents
    
    
    \newpage
    \pagenumbering{arabic}
    %\linenumbers\renewcommand\thelinenumber{\color{black!50}\arabic{linenumber}}
            \input{0 - introduction/main.tex}
        \part{Research}
            \input{1 - low-noise PiC models/main.tex}
            \input{2 - kinetic component/main.tex}
            \input{3 - fluid component/main.tex}
            \input{4 - numerical implementation/main.tex}
        \part{Project Overview}
            \input{5 - research plan/main.tex}
            \input{6 - summary/main.tex}
    
    
    %\section{}
    \newpage
    \pagenumbering{gobble}
        \printbibliography


    \newpage
    \pagenumbering{roman}
    \appendix
        \part{Appendices}
            \input{8 - Hilbert complexes/main.tex}
            \input{9 - weak conservation proofs/main.tex}
\end{document}

\end{document}


\title{\BA{Title in Progress...}}
\author{Boris Andrews}
\affil{Mathematical Institute, University of Oxford}
\date{\today}


\begin{document}
    \pagenumbering{gobble}
    \maketitle
    
    
    \begin{abstract}
        Magnetic confinement reactors---in particular tokamaks---offer one of the most promising options for achieving practical nuclear fusion, with the potential to provide virtually limitless, clean energy. The theoretical and numerical modeling of tokamak plasmas is simultaneously an essential component of effective reactor design, and a great research barrier. Tokamak operational conditions exhibit comparatively low Knudsen numbers. Kinetic effects, including kinetic waves and instabilities, Landau damping, bump-on-tail instabilities and more, are therefore highly influential in tokamak plasma dynamics. Purely fluid models are inherently incapable of capturing these effects, whereas the high dimensionality in purely kinetic models render them practically intractable for most relevant purposes.

        We consider a $\delta\!f$ decomposition model, with a macroscopic fluid background and microscopic kinetic correction, both fully coupled to each other. A similar manner of discretization is proposed to that used in the recent \texttt{STRUPHY} code \cite{Holderied_Possanner_Wang_2021, Holderied_2022, Li_et_al_2023} with a finite-element model for the background and a pseudo-particle/PiC model for the correction.

        The fluid background satisfies the full, non-linear, resistive, compressible, Hall MHD equations. \cite{Laakmann_Hu_Farrell_2022} introduces finite-element(-in-space) implicit timesteppers for the incompressible analogue to this system with structure-preserving (SP) properties in the ideal case, alongside parameter-robust preconditioners. We show that these timesteppers can derive from a finite-element-in-time (FET) (and finite-element-in-space) interpretation. The benefits of this reformulation are discussed, including the derivation of timesteppers that are higher order in time, and the quantifiable dissipative SP properties in the non-ideal, resistive case.
        
        We discuss possible options for extending this FET approach to timesteppers for the compressible case.

        The kinetic corrections satisfy linearized Boltzmann equations. Using a Lénard--Bernstein collision operator, these take Fokker--Planck-like forms \cite{Fokker_1914, Planck_1917} wherein pseudo-particles in the numerical model obey the neoclassical transport equations, with particle-independent Brownian drift terms. This offers a rigorous methodology for incorporating collisions into the particle transport model, without coupling the equations of motions for each particle.
        
        Works by Chen, Chacón et al. \cite{Chen_Chacón_Barnes_2011, Chacón_Chen_Barnes_2013, Chen_Chacón_2014, Chen_Chacón_2015} have developed structure-preserving particle pushers for neoclassical transport in the Vlasov equations, derived from Crank--Nicolson integrators. We show these too can can derive from a FET interpretation, similarly offering potential extensions to higher-order-in-time particle pushers. The FET formulation is used also to consider how the stochastic drift terms can be incorporated into the pushers. Stochastic gyrokinetic expansions are also discussed.

        Different options for the numerical implementation of these schemes are considered.

        Due to the efficacy of FET in the development of SP timesteppers for both the fluid and kinetic component, we hope this approach will prove effective in the future for developing SP timesteppers for the full hybrid model. We hope this will give us the opportunity to incorporate previously inaccessible kinetic effects into the highly effective, modern, finite-element MHD models.
    \end{abstract}
    
    
    \newpage
    \tableofcontents
    
    
    \newpage
    \pagenumbering{arabic}
    %\linenumbers\renewcommand\thelinenumber{\color{black!50}\arabic{linenumber}}
            \documentclass[12pt, a4paper]{report}

\documentclass[12pt, a4paper]{report}

\input{template/main.tex}

\title{\BA{Title in Progress...}}
\author{Boris Andrews}
\affil{Mathematical Institute, University of Oxford}
\date{\today}


\begin{document}
    \pagenumbering{gobble}
    \maketitle
    
    
    \begin{abstract}
        Magnetic confinement reactors---in particular tokamaks---offer one of the most promising options for achieving practical nuclear fusion, with the potential to provide virtually limitless, clean energy. The theoretical and numerical modeling of tokamak plasmas is simultaneously an essential component of effective reactor design, and a great research barrier. Tokamak operational conditions exhibit comparatively low Knudsen numbers. Kinetic effects, including kinetic waves and instabilities, Landau damping, bump-on-tail instabilities and more, are therefore highly influential in tokamak plasma dynamics. Purely fluid models are inherently incapable of capturing these effects, whereas the high dimensionality in purely kinetic models render them practically intractable for most relevant purposes.

        We consider a $\delta\!f$ decomposition model, with a macroscopic fluid background and microscopic kinetic correction, both fully coupled to each other. A similar manner of discretization is proposed to that used in the recent \texttt{STRUPHY} code \cite{Holderied_Possanner_Wang_2021, Holderied_2022, Li_et_al_2023} with a finite-element model for the background and a pseudo-particle/PiC model for the correction.

        The fluid background satisfies the full, non-linear, resistive, compressible, Hall MHD equations. \cite{Laakmann_Hu_Farrell_2022} introduces finite-element(-in-space) implicit timesteppers for the incompressible analogue to this system with structure-preserving (SP) properties in the ideal case, alongside parameter-robust preconditioners. We show that these timesteppers can derive from a finite-element-in-time (FET) (and finite-element-in-space) interpretation. The benefits of this reformulation are discussed, including the derivation of timesteppers that are higher order in time, and the quantifiable dissipative SP properties in the non-ideal, resistive case.
        
        We discuss possible options for extending this FET approach to timesteppers for the compressible case.

        The kinetic corrections satisfy linearized Boltzmann equations. Using a Lénard--Bernstein collision operator, these take Fokker--Planck-like forms \cite{Fokker_1914, Planck_1917} wherein pseudo-particles in the numerical model obey the neoclassical transport equations, with particle-independent Brownian drift terms. This offers a rigorous methodology for incorporating collisions into the particle transport model, without coupling the equations of motions for each particle.
        
        Works by Chen, Chacón et al. \cite{Chen_Chacón_Barnes_2011, Chacón_Chen_Barnes_2013, Chen_Chacón_2014, Chen_Chacón_2015} have developed structure-preserving particle pushers for neoclassical transport in the Vlasov equations, derived from Crank--Nicolson integrators. We show these too can can derive from a FET interpretation, similarly offering potential extensions to higher-order-in-time particle pushers. The FET formulation is used also to consider how the stochastic drift terms can be incorporated into the pushers. Stochastic gyrokinetic expansions are also discussed.

        Different options for the numerical implementation of these schemes are considered.

        Due to the efficacy of FET in the development of SP timesteppers for both the fluid and kinetic component, we hope this approach will prove effective in the future for developing SP timesteppers for the full hybrid model. We hope this will give us the opportunity to incorporate previously inaccessible kinetic effects into the highly effective, modern, finite-element MHD models.
    \end{abstract}
    
    
    \newpage
    \tableofcontents
    
    
    \newpage
    \pagenumbering{arabic}
    %\linenumbers\renewcommand\thelinenumber{\color{black!50}\arabic{linenumber}}
            \input{0 - introduction/main.tex}
        \part{Research}
            \input{1 - low-noise PiC models/main.tex}
            \input{2 - kinetic component/main.tex}
            \input{3 - fluid component/main.tex}
            \input{4 - numerical implementation/main.tex}
        \part{Project Overview}
            \input{5 - research plan/main.tex}
            \input{6 - summary/main.tex}
    
    
    %\section{}
    \newpage
    \pagenumbering{gobble}
        \printbibliography


    \newpage
    \pagenumbering{roman}
    \appendix
        \part{Appendices}
            \input{8 - Hilbert complexes/main.tex}
            \input{9 - weak conservation proofs/main.tex}
\end{document}


\title{\BA{Title in Progress...}}
\author{Boris Andrews}
\affil{Mathematical Institute, University of Oxford}
\date{\today}


\begin{document}
    \pagenumbering{gobble}
    \maketitle
    
    
    \begin{abstract}
        Magnetic confinement reactors---in particular tokamaks---offer one of the most promising options for achieving practical nuclear fusion, with the potential to provide virtually limitless, clean energy. The theoretical and numerical modeling of tokamak plasmas is simultaneously an essential component of effective reactor design, and a great research barrier. Tokamak operational conditions exhibit comparatively low Knudsen numbers. Kinetic effects, including kinetic waves and instabilities, Landau damping, bump-on-tail instabilities and more, are therefore highly influential in tokamak plasma dynamics. Purely fluid models are inherently incapable of capturing these effects, whereas the high dimensionality in purely kinetic models render them practically intractable for most relevant purposes.

        We consider a $\delta\!f$ decomposition model, with a macroscopic fluid background and microscopic kinetic correction, both fully coupled to each other. A similar manner of discretization is proposed to that used in the recent \texttt{STRUPHY} code \cite{Holderied_Possanner_Wang_2021, Holderied_2022, Li_et_al_2023} with a finite-element model for the background and a pseudo-particle/PiC model for the correction.

        The fluid background satisfies the full, non-linear, resistive, compressible, Hall MHD equations. \cite{Laakmann_Hu_Farrell_2022} introduces finite-element(-in-space) implicit timesteppers for the incompressible analogue to this system with structure-preserving (SP) properties in the ideal case, alongside parameter-robust preconditioners. We show that these timesteppers can derive from a finite-element-in-time (FET) (and finite-element-in-space) interpretation. The benefits of this reformulation are discussed, including the derivation of timesteppers that are higher order in time, and the quantifiable dissipative SP properties in the non-ideal, resistive case.
        
        We discuss possible options for extending this FET approach to timesteppers for the compressible case.

        The kinetic corrections satisfy linearized Boltzmann equations. Using a Lénard--Bernstein collision operator, these take Fokker--Planck-like forms \cite{Fokker_1914, Planck_1917} wherein pseudo-particles in the numerical model obey the neoclassical transport equations, with particle-independent Brownian drift terms. This offers a rigorous methodology for incorporating collisions into the particle transport model, without coupling the equations of motions for each particle.
        
        Works by Chen, Chacón et al. \cite{Chen_Chacón_Barnes_2011, Chacón_Chen_Barnes_2013, Chen_Chacón_2014, Chen_Chacón_2015} have developed structure-preserving particle pushers for neoclassical transport in the Vlasov equations, derived from Crank--Nicolson integrators. We show these too can can derive from a FET interpretation, similarly offering potential extensions to higher-order-in-time particle pushers. The FET formulation is used also to consider how the stochastic drift terms can be incorporated into the pushers. Stochastic gyrokinetic expansions are also discussed.

        Different options for the numerical implementation of these schemes are considered.

        Due to the efficacy of FET in the development of SP timesteppers for both the fluid and kinetic component, we hope this approach will prove effective in the future for developing SP timesteppers for the full hybrid model. We hope this will give us the opportunity to incorporate previously inaccessible kinetic effects into the highly effective, modern, finite-element MHD models.
    \end{abstract}
    
    
    \newpage
    \tableofcontents
    
    
    \newpage
    \pagenumbering{arabic}
    %\linenumbers\renewcommand\thelinenumber{\color{black!50}\arabic{linenumber}}
            \documentclass[12pt, a4paper]{report}

\input{template/main.tex}

\title{\BA{Title in Progress...}}
\author{Boris Andrews}
\affil{Mathematical Institute, University of Oxford}
\date{\today}


\begin{document}
    \pagenumbering{gobble}
    \maketitle
    
    
    \begin{abstract}
        Magnetic confinement reactors---in particular tokamaks---offer one of the most promising options for achieving practical nuclear fusion, with the potential to provide virtually limitless, clean energy. The theoretical and numerical modeling of tokamak plasmas is simultaneously an essential component of effective reactor design, and a great research barrier. Tokamak operational conditions exhibit comparatively low Knudsen numbers. Kinetic effects, including kinetic waves and instabilities, Landau damping, bump-on-tail instabilities and more, are therefore highly influential in tokamak plasma dynamics. Purely fluid models are inherently incapable of capturing these effects, whereas the high dimensionality in purely kinetic models render them practically intractable for most relevant purposes.

        We consider a $\delta\!f$ decomposition model, with a macroscopic fluid background and microscopic kinetic correction, both fully coupled to each other. A similar manner of discretization is proposed to that used in the recent \texttt{STRUPHY} code \cite{Holderied_Possanner_Wang_2021, Holderied_2022, Li_et_al_2023} with a finite-element model for the background and a pseudo-particle/PiC model for the correction.

        The fluid background satisfies the full, non-linear, resistive, compressible, Hall MHD equations. \cite{Laakmann_Hu_Farrell_2022} introduces finite-element(-in-space) implicit timesteppers for the incompressible analogue to this system with structure-preserving (SP) properties in the ideal case, alongside parameter-robust preconditioners. We show that these timesteppers can derive from a finite-element-in-time (FET) (and finite-element-in-space) interpretation. The benefits of this reformulation are discussed, including the derivation of timesteppers that are higher order in time, and the quantifiable dissipative SP properties in the non-ideal, resistive case.
        
        We discuss possible options for extending this FET approach to timesteppers for the compressible case.

        The kinetic corrections satisfy linearized Boltzmann equations. Using a Lénard--Bernstein collision operator, these take Fokker--Planck-like forms \cite{Fokker_1914, Planck_1917} wherein pseudo-particles in the numerical model obey the neoclassical transport equations, with particle-independent Brownian drift terms. This offers a rigorous methodology for incorporating collisions into the particle transport model, without coupling the equations of motions for each particle.
        
        Works by Chen, Chacón et al. \cite{Chen_Chacón_Barnes_2011, Chacón_Chen_Barnes_2013, Chen_Chacón_2014, Chen_Chacón_2015} have developed structure-preserving particle pushers for neoclassical transport in the Vlasov equations, derived from Crank--Nicolson integrators. We show these too can can derive from a FET interpretation, similarly offering potential extensions to higher-order-in-time particle pushers. The FET formulation is used also to consider how the stochastic drift terms can be incorporated into the pushers. Stochastic gyrokinetic expansions are also discussed.

        Different options for the numerical implementation of these schemes are considered.

        Due to the efficacy of FET in the development of SP timesteppers for both the fluid and kinetic component, we hope this approach will prove effective in the future for developing SP timesteppers for the full hybrid model. We hope this will give us the opportunity to incorporate previously inaccessible kinetic effects into the highly effective, modern, finite-element MHD models.
    \end{abstract}
    
    
    \newpage
    \tableofcontents
    
    
    \newpage
    \pagenumbering{arabic}
    %\linenumbers\renewcommand\thelinenumber{\color{black!50}\arabic{linenumber}}
            \input{0 - introduction/main.tex}
        \part{Research}
            \input{1 - low-noise PiC models/main.tex}
            \input{2 - kinetic component/main.tex}
            \input{3 - fluid component/main.tex}
            \input{4 - numerical implementation/main.tex}
        \part{Project Overview}
            \input{5 - research plan/main.tex}
            \input{6 - summary/main.tex}
    
    
    %\section{}
    \newpage
    \pagenumbering{gobble}
        \printbibliography


    \newpage
    \pagenumbering{roman}
    \appendix
        \part{Appendices}
            \input{8 - Hilbert complexes/main.tex}
            \input{9 - weak conservation proofs/main.tex}
\end{document}

        \part{Research}
            \documentclass[12pt, a4paper]{report}

\input{template/main.tex}

\title{\BA{Title in Progress...}}
\author{Boris Andrews}
\affil{Mathematical Institute, University of Oxford}
\date{\today}


\begin{document}
    \pagenumbering{gobble}
    \maketitle
    
    
    \begin{abstract}
        Magnetic confinement reactors---in particular tokamaks---offer one of the most promising options for achieving practical nuclear fusion, with the potential to provide virtually limitless, clean energy. The theoretical and numerical modeling of tokamak plasmas is simultaneously an essential component of effective reactor design, and a great research barrier. Tokamak operational conditions exhibit comparatively low Knudsen numbers. Kinetic effects, including kinetic waves and instabilities, Landau damping, bump-on-tail instabilities and more, are therefore highly influential in tokamak plasma dynamics. Purely fluid models are inherently incapable of capturing these effects, whereas the high dimensionality in purely kinetic models render them practically intractable for most relevant purposes.

        We consider a $\delta\!f$ decomposition model, with a macroscopic fluid background and microscopic kinetic correction, both fully coupled to each other. A similar manner of discretization is proposed to that used in the recent \texttt{STRUPHY} code \cite{Holderied_Possanner_Wang_2021, Holderied_2022, Li_et_al_2023} with a finite-element model for the background and a pseudo-particle/PiC model for the correction.

        The fluid background satisfies the full, non-linear, resistive, compressible, Hall MHD equations. \cite{Laakmann_Hu_Farrell_2022} introduces finite-element(-in-space) implicit timesteppers for the incompressible analogue to this system with structure-preserving (SP) properties in the ideal case, alongside parameter-robust preconditioners. We show that these timesteppers can derive from a finite-element-in-time (FET) (and finite-element-in-space) interpretation. The benefits of this reformulation are discussed, including the derivation of timesteppers that are higher order in time, and the quantifiable dissipative SP properties in the non-ideal, resistive case.
        
        We discuss possible options for extending this FET approach to timesteppers for the compressible case.

        The kinetic corrections satisfy linearized Boltzmann equations. Using a Lénard--Bernstein collision operator, these take Fokker--Planck-like forms \cite{Fokker_1914, Planck_1917} wherein pseudo-particles in the numerical model obey the neoclassical transport equations, with particle-independent Brownian drift terms. This offers a rigorous methodology for incorporating collisions into the particle transport model, without coupling the equations of motions for each particle.
        
        Works by Chen, Chacón et al. \cite{Chen_Chacón_Barnes_2011, Chacón_Chen_Barnes_2013, Chen_Chacón_2014, Chen_Chacón_2015} have developed structure-preserving particle pushers for neoclassical transport in the Vlasov equations, derived from Crank--Nicolson integrators. We show these too can can derive from a FET interpretation, similarly offering potential extensions to higher-order-in-time particle pushers. The FET formulation is used also to consider how the stochastic drift terms can be incorporated into the pushers. Stochastic gyrokinetic expansions are also discussed.

        Different options for the numerical implementation of these schemes are considered.

        Due to the efficacy of FET in the development of SP timesteppers for both the fluid and kinetic component, we hope this approach will prove effective in the future for developing SP timesteppers for the full hybrid model. We hope this will give us the opportunity to incorporate previously inaccessible kinetic effects into the highly effective, modern, finite-element MHD models.
    \end{abstract}
    
    
    \newpage
    \tableofcontents
    
    
    \newpage
    \pagenumbering{arabic}
    %\linenumbers\renewcommand\thelinenumber{\color{black!50}\arabic{linenumber}}
            \input{0 - introduction/main.tex}
        \part{Research}
            \input{1 - low-noise PiC models/main.tex}
            \input{2 - kinetic component/main.tex}
            \input{3 - fluid component/main.tex}
            \input{4 - numerical implementation/main.tex}
        \part{Project Overview}
            \input{5 - research plan/main.tex}
            \input{6 - summary/main.tex}
    
    
    %\section{}
    \newpage
    \pagenumbering{gobble}
        \printbibliography


    \newpage
    \pagenumbering{roman}
    \appendix
        \part{Appendices}
            \input{8 - Hilbert complexes/main.tex}
            \input{9 - weak conservation proofs/main.tex}
\end{document}

            \documentclass[12pt, a4paper]{report}

\input{template/main.tex}

\title{\BA{Title in Progress...}}
\author{Boris Andrews}
\affil{Mathematical Institute, University of Oxford}
\date{\today}


\begin{document}
    \pagenumbering{gobble}
    \maketitle
    
    
    \begin{abstract}
        Magnetic confinement reactors---in particular tokamaks---offer one of the most promising options for achieving practical nuclear fusion, with the potential to provide virtually limitless, clean energy. The theoretical and numerical modeling of tokamak plasmas is simultaneously an essential component of effective reactor design, and a great research barrier. Tokamak operational conditions exhibit comparatively low Knudsen numbers. Kinetic effects, including kinetic waves and instabilities, Landau damping, bump-on-tail instabilities and more, are therefore highly influential in tokamak plasma dynamics. Purely fluid models are inherently incapable of capturing these effects, whereas the high dimensionality in purely kinetic models render them practically intractable for most relevant purposes.

        We consider a $\delta\!f$ decomposition model, with a macroscopic fluid background and microscopic kinetic correction, both fully coupled to each other. A similar manner of discretization is proposed to that used in the recent \texttt{STRUPHY} code \cite{Holderied_Possanner_Wang_2021, Holderied_2022, Li_et_al_2023} with a finite-element model for the background and a pseudo-particle/PiC model for the correction.

        The fluid background satisfies the full, non-linear, resistive, compressible, Hall MHD equations. \cite{Laakmann_Hu_Farrell_2022} introduces finite-element(-in-space) implicit timesteppers for the incompressible analogue to this system with structure-preserving (SP) properties in the ideal case, alongside parameter-robust preconditioners. We show that these timesteppers can derive from a finite-element-in-time (FET) (and finite-element-in-space) interpretation. The benefits of this reformulation are discussed, including the derivation of timesteppers that are higher order in time, and the quantifiable dissipative SP properties in the non-ideal, resistive case.
        
        We discuss possible options for extending this FET approach to timesteppers for the compressible case.

        The kinetic corrections satisfy linearized Boltzmann equations. Using a Lénard--Bernstein collision operator, these take Fokker--Planck-like forms \cite{Fokker_1914, Planck_1917} wherein pseudo-particles in the numerical model obey the neoclassical transport equations, with particle-independent Brownian drift terms. This offers a rigorous methodology for incorporating collisions into the particle transport model, without coupling the equations of motions for each particle.
        
        Works by Chen, Chacón et al. \cite{Chen_Chacón_Barnes_2011, Chacón_Chen_Barnes_2013, Chen_Chacón_2014, Chen_Chacón_2015} have developed structure-preserving particle pushers for neoclassical transport in the Vlasov equations, derived from Crank--Nicolson integrators. We show these too can can derive from a FET interpretation, similarly offering potential extensions to higher-order-in-time particle pushers. The FET formulation is used also to consider how the stochastic drift terms can be incorporated into the pushers. Stochastic gyrokinetic expansions are also discussed.

        Different options for the numerical implementation of these schemes are considered.

        Due to the efficacy of FET in the development of SP timesteppers for both the fluid and kinetic component, we hope this approach will prove effective in the future for developing SP timesteppers for the full hybrid model. We hope this will give us the opportunity to incorporate previously inaccessible kinetic effects into the highly effective, modern, finite-element MHD models.
    \end{abstract}
    
    
    \newpage
    \tableofcontents
    
    
    \newpage
    \pagenumbering{arabic}
    %\linenumbers\renewcommand\thelinenumber{\color{black!50}\arabic{linenumber}}
            \input{0 - introduction/main.tex}
        \part{Research}
            \input{1 - low-noise PiC models/main.tex}
            \input{2 - kinetic component/main.tex}
            \input{3 - fluid component/main.tex}
            \input{4 - numerical implementation/main.tex}
        \part{Project Overview}
            \input{5 - research plan/main.tex}
            \input{6 - summary/main.tex}
    
    
    %\section{}
    \newpage
    \pagenumbering{gobble}
        \printbibliography


    \newpage
    \pagenumbering{roman}
    \appendix
        \part{Appendices}
            \input{8 - Hilbert complexes/main.tex}
            \input{9 - weak conservation proofs/main.tex}
\end{document}

            \documentclass[12pt, a4paper]{report}

\input{template/main.tex}

\title{\BA{Title in Progress...}}
\author{Boris Andrews}
\affil{Mathematical Institute, University of Oxford}
\date{\today}


\begin{document}
    \pagenumbering{gobble}
    \maketitle
    
    
    \begin{abstract}
        Magnetic confinement reactors---in particular tokamaks---offer one of the most promising options for achieving practical nuclear fusion, with the potential to provide virtually limitless, clean energy. The theoretical and numerical modeling of tokamak plasmas is simultaneously an essential component of effective reactor design, and a great research barrier. Tokamak operational conditions exhibit comparatively low Knudsen numbers. Kinetic effects, including kinetic waves and instabilities, Landau damping, bump-on-tail instabilities and more, are therefore highly influential in tokamak plasma dynamics. Purely fluid models are inherently incapable of capturing these effects, whereas the high dimensionality in purely kinetic models render them practically intractable for most relevant purposes.

        We consider a $\delta\!f$ decomposition model, with a macroscopic fluid background and microscopic kinetic correction, both fully coupled to each other. A similar manner of discretization is proposed to that used in the recent \texttt{STRUPHY} code \cite{Holderied_Possanner_Wang_2021, Holderied_2022, Li_et_al_2023} with a finite-element model for the background and a pseudo-particle/PiC model for the correction.

        The fluid background satisfies the full, non-linear, resistive, compressible, Hall MHD equations. \cite{Laakmann_Hu_Farrell_2022} introduces finite-element(-in-space) implicit timesteppers for the incompressible analogue to this system with structure-preserving (SP) properties in the ideal case, alongside parameter-robust preconditioners. We show that these timesteppers can derive from a finite-element-in-time (FET) (and finite-element-in-space) interpretation. The benefits of this reformulation are discussed, including the derivation of timesteppers that are higher order in time, and the quantifiable dissipative SP properties in the non-ideal, resistive case.
        
        We discuss possible options for extending this FET approach to timesteppers for the compressible case.

        The kinetic corrections satisfy linearized Boltzmann equations. Using a Lénard--Bernstein collision operator, these take Fokker--Planck-like forms \cite{Fokker_1914, Planck_1917} wherein pseudo-particles in the numerical model obey the neoclassical transport equations, with particle-independent Brownian drift terms. This offers a rigorous methodology for incorporating collisions into the particle transport model, without coupling the equations of motions for each particle.
        
        Works by Chen, Chacón et al. \cite{Chen_Chacón_Barnes_2011, Chacón_Chen_Barnes_2013, Chen_Chacón_2014, Chen_Chacón_2015} have developed structure-preserving particle pushers for neoclassical transport in the Vlasov equations, derived from Crank--Nicolson integrators. We show these too can can derive from a FET interpretation, similarly offering potential extensions to higher-order-in-time particle pushers. The FET formulation is used also to consider how the stochastic drift terms can be incorporated into the pushers. Stochastic gyrokinetic expansions are also discussed.

        Different options for the numerical implementation of these schemes are considered.

        Due to the efficacy of FET in the development of SP timesteppers for both the fluid and kinetic component, we hope this approach will prove effective in the future for developing SP timesteppers for the full hybrid model. We hope this will give us the opportunity to incorporate previously inaccessible kinetic effects into the highly effective, modern, finite-element MHD models.
    \end{abstract}
    
    
    \newpage
    \tableofcontents
    
    
    \newpage
    \pagenumbering{arabic}
    %\linenumbers\renewcommand\thelinenumber{\color{black!50}\arabic{linenumber}}
            \input{0 - introduction/main.tex}
        \part{Research}
            \input{1 - low-noise PiC models/main.tex}
            \input{2 - kinetic component/main.tex}
            \input{3 - fluid component/main.tex}
            \input{4 - numerical implementation/main.tex}
        \part{Project Overview}
            \input{5 - research plan/main.tex}
            \input{6 - summary/main.tex}
    
    
    %\section{}
    \newpage
    \pagenumbering{gobble}
        \printbibliography


    \newpage
    \pagenumbering{roman}
    \appendix
        \part{Appendices}
            \input{8 - Hilbert complexes/main.tex}
            \input{9 - weak conservation proofs/main.tex}
\end{document}

            \documentclass[12pt, a4paper]{report}

\input{template/main.tex}

\title{\BA{Title in Progress...}}
\author{Boris Andrews}
\affil{Mathematical Institute, University of Oxford}
\date{\today}


\begin{document}
    \pagenumbering{gobble}
    \maketitle
    
    
    \begin{abstract}
        Magnetic confinement reactors---in particular tokamaks---offer one of the most promising options for achieving practical nuclear fusion, with the potential to provide virtually limitless, clean energy. The theoretical and numerical modeling of tokamak plasmas is simultaneously an essential component of effective reactor design, and a great research barrier. Tokamak operational conditions exhibit comparatively low Knudsen numbers. Kinetic effects, including kinetic waves and instabilities, Landau damping, bump-on-tail instabilities and more, are therefore highly influential in tokamak plasma dynamics. Purely fluid models are inherently incapable of capturing these effects, whereas the high dimensionality in purely kinetic models render them practically intractable for most relevant purposes.

        We consider a $\delta\!f$ decomposition model, with a macroscopic fluid background and microscopic kinetic correction, both fully coupled to each other. A similar manner of discretization is proposed to that used in the recent \texttt{STRUPHY} code \cite{Holderied_Possanner_Wang_2021, Holderied_2022, Li_et_al_2023} with a finite-element model for the background and a pseudo-particle/PiC model for the correction.

        The fluid background satisfies the full, non-linear, resistive, compressible, Hall MHD equations. \cite{Laakmann_Hu_Farrell_2022} introduces finite-element(-in-space) implicit timesteppers for the incompressible analogue to this system with structure-preserving (SP) properties in the ideal case, alongside parameter-robust preconditioners. We show that these timesteppers can derive from a finite-element-in-time (FET) (and finite-element-in-space) interpretation. The benefits of this reformulation are discussed, including the derivation of timesteppers that are higher order in time, and the quantifiable dissipative SP properties in the non-ideal, resistive case.
        
        We discuss possible options for extending this FET approach to timesteppers for the compressible case.

        The kinetic corrections satisfy linearized Boltzmann equations. Using a Lénard--Bernstein collision operator, these take Fokker--Planck-like forms \cite{Fokker_1914, Planck_1917} wherein pseudo-particles in the numerical model obey the neoclassical transport equations, with particle-independent Brownian drift terms. This offers a rigorous methodology for incorporating collisions into the particle transport model, without coupling the equations of motions for each particle.
        
        Works by Chen, Chacón et al. \cite{Chen_Chacón_Barnes_2011, Chacón_Chen_Barnes_2013, Chen_Chacón_2014, Chen_Chacón_2015} have developed structure-preserving particle pushers for neoclassical transport in the Vlasov equations, derived from Crank--Nicolson integrators. We show these too can can derive from a FET interpretation, similarly offering potential extensions to higher-order-in-time particle pushers. The FET formulation is used also to consider how the stochastic drift terms can be incorporated into the pushers. Stochastic gyrokinetic expansions are also discussed.

        Different options for the numerical implementation of these schemes are considered.

        Due to the efficacy of FET in the development of SP timesteppers for both the fluid and kinetic component, we hope this approach will prove effective in the future for developing SP timesteppers for the full hybrid model. We hope this will give us the opportunity to incorporate previously inaccessible kinetic effects into the highly effective, modern, finite-element MHD models.
    \end{abstract}
    
    
    \newpage
    \tableofcontents
    
    
    \newpage
    \pagenumbering{arabic}
    %\linenumbers\renewcommand\thelinenumber{\color{black!50}\arabic{linenumber}}
            \input{0 - introduction/main.tex}
        \part{Research}
            \input{1 - low-noise PiC models/main.tex}
            \input{2 - kinetic component/main.tex}
            \input{3 - fluid component/main.tex}
            \input{4 - numerical implementation/main.tex}
        \part{Project Overview}
            \input{5 - research plan/main.tex}
            \input{6 - summary/main.tex}
    
    
    %\section{}
    \newpage
    \pagenumbering{gobble}
        \printbibliography


    \newpage
    \pagenumbering{roman}
    \appendix
        \part{Appendices}
            \input{8 - Hilbert complexes/main.tex}
            \input{9 - weak conservation proofs/main.tex}
\end{document}

        \part{Project Overview}
            \documentclass[12pt, a4paper]{report}

\input{template/main.tex}

\title{\BA{Title in Progress...}}
\author{Boris Andrews}
\affil{Mathematical Institute, University of Oxford}
\date{\today}


\begin{document}
    \pagenumbering{gobble}
    \maketitle
    
    
    \begin{abstract}
        Magnetic confinement reactors---in particular tokamaks---offer one of the most promising options for achieving practical nuclear fusion, with the potential to provide virtually limitless, clean energy. The theoretical and numerical modeling of tokamak plasmas is simultaneously an essential component of effective reactor design, and a great research barrier. Tokamak operational conditions exhibit comparatively low Knudsen numbers. Kinetic effects, including kinetic waves and instabilities, Landau damping, bump-on-tail instabilities and more, are therefore highly influential in tokamak plasma dynamics. Purely fluid models are inherently incapable of capturing these effects, whereas the high dimensionality in purely kinetic models render them practically intractable for most relevant purposes.

        We consider a $\delta\!f$ decomposition model, with a macroscopic fluid background and microscopic kinetic correction, both fully coupled to each other. A similar manner of discretization is proposed to that used in the recent \texttt{STRUPHY} code \cite{Holderied_Possanner_Wang_2021, Holderied_2022, Li_et_al_2023} with a finite-element model for the background and a pseudo-particle/PiC model for the correction.

        The fluid background satisfies the full, non-linear, resistive, compressible, Hall MHD equations. \cite{Laakmann_Hu_Farrell_2022} introduces finite-element(-in-space) implicit timesteppers for the incompressible analogue to this system with structure-preserving (SP) properties in the ideal case, alongside parameter-robust preconditioners. We show that these timesteppers can derive from a finite-element-in-time (FET) (and finite-element-in-space) interpretation. The benefits of this reformulation are discussed, including the derivation of timesteppers that are higher order in time, and the quantifiable dissipative SP properties in the non-ideal, resistive case.
        
        We discuss possible options for extending this FET approach to timesteppers for the compressible case.

        The kinetic corrections satisfy linearized Boltzmann equations. Using a Lénard--Bernstein collision operator, these take Fokker--Planck-like forms \cite{Fokker_1914, Planck_1917} wherein pseudo-particles in the numerical model obey the neoclassical transport equations, with particle-independent Brownian drift terms. This offers a rigorous methodology for incorporating collisions into the particle transport model, without coupling the equations of motions for each particle.
        
        Works by Chen, Chacón et al. \cite{Chen_Chacón_Barnes_2011, Chacón_Chen_Barnes_2013, Chen_Chacón_2014, Chen_Chacón_2015} have developed structure-preserving particle pushers for neoclassical transport in the Vlasov equations, derived from Crank--Nicolson integrators. We show these too can can derive from a FET interpretation, similarly offering potential extensions to higher-order-in-time particle pushers. The FET formulation is used also to consider how the stochastic drift terms can be incorporated into the pushers. Stochastic gyrokinetic expansions are also discussed.

        Different options for the numerical implementation of these schemes are considered.

        Due to the efficacy of FET in the development of SP timesteppers for both the fluid and kinetic component, we hope this approach will prove effective in the future for developing SP timesteppers for the full hybrid model. We hope this will give us the opportunity to incorporate previously inaccessible kinetic effects into the highly effective, modern, finite-element MHD models.
    \end{abstract}
    
    
    \newpage
    \tableofcontents
    
    
    \newpage
    \pagenumbering{arabic}
    %\linenumbers\renewcommand\thelinenumber{\color{black!50}\arabic{linenumber}}
            \input{0 - introduction/main.tex}
        \part{Research}
            \input{1 - low-noise PiC models/main.tex}
            \input{2 - kinetic component/main.tex}
            \input{3 - fluid component/main.tex}
            \input{4 - numerical implementation/main.tex}
        \part{Project Overview}
            \input{5 - research plan/main.tex}
            \input{6 - summary/main.tex}
    
    
    %\section{}
    \newpage
    \pagenumbering{gobble}
        \printbibliography


    \newpage
    \pagenumbering{roman}
    \appendix
        \part{Appendices}
            \input{8 - Hilbert complexes/main.tex}
            \input{9 - weak conservation proofs/main.tex}
\end{document}

            \documentclass[12pt, a4paper]{report}

\input{template/main.tex}

\title{\BA{Title in Progress...}}
\author{Boris Andrews}
\affil{Mathematical Institute, University of Oxford}
\date{\today}


\begin{document}
    \pagenumbering{gobble}
    \maketitle
    
    
    \begin{abstract}
        Magnetic confinement reactors---in particular tokamaks---offer one of the most promising options for achieving practical nuclear fusion, with the potential to provide virtually limitless, clean energy. The theoretical and numerical modeling of tokamak plasmas is simultaneously an essential component of effective reactor design, and a great research barrier. Tokamak operational conditions exhibit comparatively low Knudsen numbers. Kinetic effects, including kinetic waves and instabilities, Landau damping, bump-on-tail instabilities and more, are therefore highly influential in tokamak plasma dynamics. Purely fluid models are inherently incapable of capturing these effects, whereas the high dimensionality in purely kinetic models render them practically intractable for most relevant purposes.

        We consider a $\delta\!f$ decomposition model, with a macroscopic fluid background and microscopic kinetic correction, both fully coupled to each other. A similar manner of discretization is proposed to that used in the recent \texttt{STRUPHY} code \cite{Holderied_Possanner_Wang_2021, Holderied_2022, Li_et_al_2023} with a finite-element model for the background and a pseudo-particle/PiC model for the correction.

        The fluid background satisfies the full, non-linear, resistive, compressible, Hall MHD equations. \cite{Laakmann_Hu_Farrell_2022} introduces finite-element(-in-space) implicit timesteppers for the incompressible analogue to this system with structure-preserving (SP) properties in the ideal case, alongside parameter-robust preconditioners. We show that these timesteppers can derive from a finite-element-in-time (FET) (and finite-element-in-space) interpretation. The benefits of this reformulation are discussed, including the derivation of timesteppers that are higher order in time, and the quantifiable dissipative SP properties in the non-ideal, resistive case.
        
        We discuss possible options for extending this FET approach to timesteppers for the compressible case.

        The kinetic corrections satisfy linearized Boltzmann equations. Using a Lénard--Bernstein collision operator, these take Fokker--Planck-like forms \cite{Fokker_1914, Planck_1917} wherein pseudo-particles in the numerical model obey the neoclassical transport equations, with particle-independent Brownian drift terms. This offers a rigorous methodology for incorporating collisions into the particle transport model, without coupling the equations of motions for each particle.
        
        Works by Chen, Chacón et al. \cite{Chen_Chacón_Barnes_2011, Chacón_Chen_Barnes_2013, Chen_Chacón_2014, Chen_Chacón_2015} have developed structure-preserving particle pushers for neoclassical transport in the Vlasov equations, derived from Crank--Nicolson integrators. We show these too can can derive from a FET interpretation, similarly offering potential extensions to higher-order-in-time particle pushers. The FET formulation is used also to consider how the stochastic drift terms can be incorporated into the pushers. Stochastic gyrokinetic expansions are also discussed.

        Different options for the numerical implementation of these schemes are considered.

        Due to the efficacy of FET in the development of SP timesteppers for both the fluid and kinetic component, we hope this approach will prove effective in the future for developing SP timesteppers for the full hybrid model. We hope this will give us the opportunity to incorporate previously inaccessible kinetic effects into the highly effective, modern, finite-element MHD models.
    \end{abstract}
    
    
    \newpage
    \tableofcontents
    
    
    \newpage
    \pagenumbering{arabic}
    %\linenumbers\renewcommand\thelinenumber{\color{black!50}\arabic{linenumber}}
            \input{0 - introduction/main.tex}
        \part{Research}
            \input{1 - low-noise PiC models/main.tex}
            \input{2 - kinetic component/main.tex}
            \input{3 - fluid component/main.tex}
            \input{4 - numerical implementation/main.tex}
        \part{Project Overview}
            \input{5 - research plan/main.tex}
            \input{6 - summary/main.tex}
    
    
    %\section{}
    \newpage
    \pagenumbering{gobble}
        \printbibliography


    \newpage
    \pagenumbering{roman}
    \appendix
        \part{Appendices}
            \input{8 - Hilbert complexes/main.tex}
            \input{9 - weak conservation proofs/main.tex}
\end{document}

    
    
    %\section{}
    \newpage
    \pagenumbering{gobble}
        \printbibliography


    \newpage
    \pagenumbering{roman}
    \appendix
        \part{Appendices}
            \documentclass[12pt, a4paper]{report}

\input{template/main.tex}

\title{\BA{Title in Progress...}}
\author{Boris Andrews}
\affil{Mathematical Institute, University of Oxford}
\date{\today}


\begin{document}
    \pagenumbering{gobble}
    \maketitle
    
    
    \begin{abstract}
        Magnetic confinement reactors---in particular tokamaks---offer one of the most promising options for achieving practical nuclear fusion, with the potential to provide virtually limitless, clean energy. The theoretical and numerical modeling of tokamak plasmas is simultaneously an essential component of effective reactor design, and a great research barrier. Tokamak operational conditions exhibit comparatively low Knudsen numbers. Kinetic effects, including kinetic waves and instabilities, Landau damping, bump-on-tail instabilities and more, are therefore highly influential in tokamak plasma dynamics. Purely fluid models are inherently incapable of capturing these effects, whereas the high dimensionality in purely kinetic models render them practically intractable for most relevant purposes.

        We consider a $\delta\!f$ decomposition model, with a macroscopic fluid background and microscopic kinetic correction, both fully coupled to each other. A similar manner of discretization is proposed to that used in the recent \texttt{STRUPHY} code \cite{Holderied_Possanner_Wang_2021, Holderied_2022, Li_et_al_2023} with a finite-element model for the background and a pseudo-particle/PiC model for the correction.

        The fluid background satisfies the full, non-linear, resistive, compressible, Hall MHD equations. \cite{Laakmann_Hu_Farrell_2022} introduces finite-element(-in-space) implicit timesteppers for the incompressible analogue to this system with structure-preserving (SP) properties in the ideal case, alongside parameter-robust preconditioners. We show that these timesteppers can derive from a finite-element-in-time (FET) (and finite-element-in-space) interpretation. The benefits of this reformulation are discussed, including the derivation of timesteppers that are higher order in time, and the quantifiable dissipative SP properties in the non-ideal, resistive case.
        
        We discuss possible options for extending this FET approach to timesteppers for the compressible case.

        The kinetic corrections satisfy linearized Boltzmann equations. Using a Lénard--Bernstein collision operator, these take Fokker--Planck-like forms \cite{Fokker_1914, Planck_1917} wherein pseudo-particles in the numerical model obey the neoclassical transport equations, with particle-independent Brownian drift terms. This offers a rigorous methodology for incorporating collisions into the particle transport model, without coupling the equations of motions for each particle.
        
        Works by Chen, Chacón et al. \cite{Chen_Chacón_Barnes_2011, Chacón_Chen_Barnes_2013, Chen_Chacón_2014, Chen_Chacón_2015} have developed structure-preserving particle pushers for neoclassical transport in the Vlasov equations, derived from Crank--Nicolson integrators. We show these too can can derive from a FET interpretation, similarly offering potential extensions to higher-order-in-time particle pushers. The FET formulation is used also to consider how the stochastic drift terms can be incorporated into the pushers. Stochastic gyrokinetic expansions are also discussed.

        Different options for the numerical implementation of these schemes are considered.

        Due to the efficacy of FET in the development of SP timesteppers for both the fluid and kinetic component, we hope this approach will prove effective in the future for developing SP timesteppers for the full hybrid model. We hope this will give us the opportunity to incorporate previously inaccessible kinetic effects into the highly effective, modern, finite-element MHD models.
    \end{abstract}
    
    
    \newpage
    \tableofcontents
    
    
    \newpage
    \pagenumbering{arabic}
    %\linenumbers\renewcommand\thelinenumber{\color{black!50}\arabic{linenumber}}
            \input{0 - introduction/main.tex}
        \part{Research}
            \input{1 - low-noise PiC models/main.tex}
            \input{2 - kinetic component/main.tex}
            \input{3 - fluid component/main.tex}
            \input{4 - numerical implementation/main.tex}
        \part{Project Overview}
            \input{5 - research plan/main.tex}
            \input{6 - summary/main.tex}
    
    
    %\section{}
    \newpage
    \pagenumbering{gobble}
        \printbibliography


    \newpage
    \pagenumbering{roman}
    \appendix
        \part{Appendices}
            \input{8 - Hilbert complexes/main.tex}
            \input{9 - weak conservation proofs/main.tex}
\end{document}

            \documentclass[12pt, a4paper]{report}

\input{template/main.tex}

\title{\BA{Title in Progress...}}
\author{Boris Andrews}
\affil{Mathematical Institute, University of Oxford}
\date{\today}


\begin{document}
    \pagenumbering{gobble}
    \maketitle
    
    
    \begin{abstract}
        Magnetic confinement reactors---in particular tokamaks---offer one of the most promising options for achieving practical nuclear fusion, with the potential to provide virtually limitless, clean energy. The theoretical and numerical modeling of tokamak plasmas is simultaneously an essential component of effective reactor design, and a great research barrier. Tokamak operational conditions exhibit comparatively low Knudsen numbers. Kinetic effects, including kinetic waves and instabilities, Landau damping, bump-on-tail instabilities and more, are therefore highly influential in tokamak plasma dynamics. Purely fluid models are inherently incapable of capturing these effects, whereas the high dimensionality in purely kinetic models render them practically intractable for most relevant purposes.

        We consider a $\delta\!f$ decomposition model, with a macroscopic fluid background and microscopic kinetic correction, both fully coupled to each other. A similar manner of discretization is proposed to that used in the recent \texttt{STRUPHY} code \cite{Holderied_Possanner_Wang_2021, Holderied_2022, Li_et_al_2023} with a finite-element model for the background and a pseudo-particle/PiC model for the correction.

        The fluid background satisfies the full, non-linear, resistive, compressible, Hall MHD equations. \cite{Laakmann_Hu_Farrell_2022} introduces finite-element(-in-space) implicit timesteppers for the incompressible analogue to this system with structure-preserving (SP) properties in the ideal case, alongside parameter-robust preconditioners. We show that these timesteppers can derive from a finite-element-in-time (FET) (and finite-element-in-space) interpretation. The benefits of this reformulation are discussed, including the derivation of timesteppers that are higher order in time, and the quantifiable dissipative SP properties in the non-ideal, resistive case.
        
        We discuss possible options for extending this FET approach to timesteppers for the compressible case.

        The kinetic corrections satisfy linearized Boltzmann equations. Using a Lénard--Bernstein collision operator, these take Fokker--Planck-like forms \cite{Fokker_1914, Planck_1917} wherein pseudo-particles in the numerical model obey the neoclassical transport equations, with particle-independent Brownian drift terms. This offers a rigorous methodology for incorporating collisions into the particle transport model, without coupling the equations of motions for each particle.
        
        Works by Chen, Chacón et al. \cite{Chen_Chacón_Barnes_2011, Chacón_Chen_Barnes_2013, Chen_Chacón_2014, Chen_Chacón_2015} have developed structure-preserving particle pushers for neoclassical transport in the Vlasov equations, derived from Crank--Nicolson integrators. We show these too can can derive from a FET interpretation, similarly offering potential extensions to higher-order-in-time particle pushers. The FET formulation is used also to consider how the stochastic drift terms can be incorporated into the pushers. Stochastic gyrokinetic expansions are also discussed.

        Different options for the numerical implementation of these schemes are considered.

        Due to the efficacy of FET in the development of SP timesteppers for both the fluid and kinetic component, we hope this approach will prove effective in the future for developing SP timesteppers for the full hybrid model. We hope this will give us the opportunity to incorporate previously inaccessible kinetic effects into the highly effective, modern, finite-element MHD models.
    \end{abstract}
    
    
    \newpage
    \tableofcontents
    
    
    \newpage
    \pagenumbering{arabic}
    %\linenumbers\renewcommand\thelinenumber{\color{black!50}\arabic{linenumber}}
            \input{0 - introduction/main.tex}
        \part{Research}
            \input{1 - low-noise PiC models/main.tex}
            \input{2 - kinetic component/main.tex}
            \input{3 - fluid component/main.tex}
            \input{4 - numerical implementation/main.tex}
        \part{Project Overview}
            \input{5 - research plan/main.tex}
            \input{6 - summary/main.tex}
    
    
    %\section{}
    \newpage
    \pagenumbering{gobble}
        \printbibliography


    \newpage
    \pagenumbering{roman}
    \appendix
        \part{Appendices}
            \input{8 - Hilbert complexes/main.tex}
            \input{9 - weak conservation proofs/main.tex}
\end{document}

\end{document}

        \part{Research}
            \documentclass[12pt, a4paper]{report}

\documentclass[12pt, a4paper]{report}

\input{template/main.tex}

\title{\BA{Title in Progress...}}
\author{Boris Andrews}
\affil{Mathematical Institute, University of Oxford}
\date{\today}


\begin{document}
    \pagenumbering{gobble}
    \maketitle
    
    
    \begin{abstract}
        Magnetic confinement reactors---in particular tokamaks---offer one of the most promising options for achieving practical nuclear fusion, with the potential to provide virtually limitless, clean energy. The theoretical and numerical modeling of tokamak plasmas is simultaneously an essential component of effective reactor design, and a great research barrier. Tokamak operational conditions exhibit comparatively low Knudsen numbers. Kinetic effects, including kinetic waves and instabilities, Landau damping, bump-on-tail instabilities and more, are therefore highly influential in tokamak plasma dynamics. Purely fluid models are inherently incapable of capturing these effects, whereas the high dimensionality in purely kinetic models render them practically intractable for most relevant purposes.

        We consider a $\delta\!f$ decomposition model, with a macroscopic fluid background and microscopic kinetic correction, both fully coupled to each other. A similar manner of discretization is proposed to that used in the recent \texttt{STRUPHY} code \cite{Holderied_Possanner_Wang_2021, Holderied_2022, Li_et_al_2023} with a finite-element model for the background and a pseudo-particle/PiC model for the correction.

        The fluid background satisfies the full, non-linear, resistive, compressible, Hall MHD equations. \cite{Laakmann_Hu_Farrell_2022} introduces finite-element(-in-space) implicit timesteppers for the incompressible analogue to this system with structure-preserving (SP) properties in the ideal case, alongside parameter-robust preconditioners. We show that these timesteppers can derive from a finite-element-in-time (FET) (and finite-element-in-space) interpretation. The benefits of this reformulation are discussed, including the derivation of timesteppers that are higher order in time, and the quantifiable dissipative SP properties in the non-ideal, resistive case.
        
        We discuss possible options for extending this FET approach to timesteppers for the compressible case.

        The kinetic corrections satisfy linearized Boltzmann equations. Using a Lénard--Bernstein collision operator, these take Fokker--Planck-like forms \cite{Fokker_1914, Planck_1917} wherein pseudo-particles in the numerical model obey the neoclassical transport equations, with particle-independent Brownian drift terms. This offers a rigorous methodology for incorporating collisions into the particle transport model, without coupling the equations of motions for each particle.
        
        Works by Chen, Chacón et al. \cite{Chen_Chacón_Barnes_2011, Chacón_Chen_Barnes_2013, Chen_Chacón_2014, Chen_Chacón_2015} have developed structure-preserving particle pushers for neoclassical transport in the Vlasov equations, derived from Crank--Nicolson integrators. We show these too can can derive from a FET interpretation, similarly offering potential extensions to higher-order-in-time particle pushers. The FET formulation is used also to consider how the stochastic drift terms can be incorporated into the pushers. Stochastic gyrokinetic expansions are also discussed.

        Different options for the numerical implementation of these schemes are considered.

        Due to the efficacy of FET in the development of SP timesteppers for both the fluid and kinetic component, we hope this approach will prove effective in the future for developing SP timesteppers for the full hybrid model. We hope this will give us the opportunity to incorporate previously inaccessible kinetic effects into the highly effective, modern, finite-element MHD models.
    \end{abstract}
    
    
    \newpage
    \tableofcontents
    
    
    \newpage
    \pagenumbering{arabic}
    %\linenumbers\renewcommand\thelinenumber{\color{black!50}\arabic{linenumber}}
            \input{0 - introduction/main.tex}
        \part{Research}
            \input{1 - low-noise PiC models/main.tex}
            \input{2 - kinetic component/main.tex}
            \input{3 - fluid component/main.tex}
            \input{4 - numerical implementation/main.tex}
        \part{Project Overview}
            \input{5 - research plan/main.tex}
            \input{6 - summary/main.tex}
    
    
    %\section{}
    \newpage
    \pagenumbering{gobble}
        \printbibliography


    \newpage
    \pagenumbering{roman}
    \appendix
        \part{Appendices}
            \input{8 - Hilbert complexes/main.tex}
            \input{9 - weak conservation proofs/main.tex}
\end{document}


\title{\BA{Title in Progress...}}
\author{Boris Andrews}
\affil{Mathematical Institute, University of Oxford}
\date{\today}


\begin{document}
    \pagenumbering{gobble}
    \maketitle
    
    
    \begin{abstract}
        Magnetic confinement reactors---in particular tokamaks---offer one of the most promising options for achieving practical nuclear fusion, with the potential to provide virtually limitless, clean energy. The theoretical and numerical modeling of tokamak plasmas is simultaneously an essential component of effective reactor design, and a great research barrier. Tokamak operational conditions exhibit comparatively low Knudsen numbers. Kinetic effects, including kinetic waves and instabilities, Landau damping, bump-on-tail instabilities and more, are therefore highly influential in tokamak plasma dynamics. Purely fluid models are inherently incapable of capturing these effects, whereas the high dimensionality in purely kinetic models render them practically intractable for most relevant purposes.

        We consider a $\delta\!f$ decomposition model, with a macroscopic fluid background and microscopic kinetic correction, both fully coupled to each other. A similar manner of discretization is proposed to that used in the recent \texttt{STRUPHY} code \cite{Holderied_Possanner_Wang_2021, Holderied_2022, Li_et_al_2023} with a finite-element model for the background and a pseudo-particle/PiC model for the correction.

        The fluid background satisfies the full, non-linear, resistive, compressible, Hall MHD equations. \cite{Laakmann_Hu_Farrell_2022} introduces finite-element(-in-space) implicit timesteppers for the incompressible analogue to this system with structure-preserving (SP) properties in the ideal case, alongside parameter-robust preconditioners. We show that these timesteppers can derive from a finite-element-in-time (FET) (and finite-element-in-space) interpretation. The benefits of this reformulation are discussed, including the derivation of timesteppers that are higher order in time, and the quantifiable dissipative SP properties in the non-ideal, resistive case.
        
        We discuss possible options for extending this FET approach to timesteppers for the compressible case.

        The kinetic corrections satisfy linearized Boltzmann equations. Using a Lénard--Bernstein collision operator, these take Fokker--Planck-like forms \cite{Fokker_1914, Planck_1917} wherein pseudo-particles in the numerical model obey the neoclassical transport equations, with particle-independent Brownian drift terms. This offers a rigorous methodology for incorporating collisions into the particle transport model, without coupling the equations of motions for each particle.
        
        Works by Chen, Chacón et al. \cite{Chen_Chacón_Barnes_2011, Chacón_Chen_Barnes_2013, Chen_Chacón_2014, Chen_Chacón_2015} have developed structure-preserving particle pushers for neoclassical transport in the Vlasov equations, derived from Crank--Nicolson integrators. We show these too can can derive from a FET interpretation, similarly offering potential extensions to higher-order-in-time particle pushers. The FET formulation is used also to consider how the stochastic drift terms can be incorporated into the pushers. Stochastic gyrokinetic expansions are also discussed.

        Different options for the numerical implementation of these schemes are considered.

        Due to the efficacy of FET in the development of SP timesteppers for both the fluid and kinetic component, we hope this approach will prove effective in the future for developing SP timesteppers for the full hybrid model. We hope this will give us the opportunity to incorporate previously inaccessible kinetic effects into the highly effective, modern, finite-element MHD models.
    \end{abstract}
    
    
    \newpage
    \tableofcontents
    
    
    \newpage
    \pagenumbering{arabic}
    %\linenumbers\renewcommand\thelinenumber{\color{black!50}\arabic{linenumber}}
            \documentclass[12pt, a4paper]{report}

\input{template/main.tex}

\title{\BA{Title in Progress...}}
\author{Boris Andrews}
\affil{Mathematical Institute, University of Oxford}
\date{\today}


\begin{document}
    \pagenumbering{gobble}
    \maketitle
    
    
    \begin{abstract}
        Magnetic confinement reactors---in particular tokamaks---offer one of the most promising options for achieving practical nuclear fusion, with the potential to provide virtually limitless, clean energy. The theoretical and numerical modeling of tokamak plasmas is simultaneously an essential component of effective reactor design, and a great research barrier. Tokamak operational conditions exhibit comparatively low Knudsen numbers. Kinetic effects, including kinetic waves and instabilities, Landau damping, bump-on-tail instabilities and more, are therefore highly influential in tokamak plasma dynamics. Purely fluid models are inherently incapable of capturing these effects, whereas the high dimensionality in purely kinetic models render them practically intractable for most relevant purposes.

        We consider a $\delta\!f$ decomposition model, with a macroscopic fluid background and microscopic kinetic correction, both fully coupled to each other. A similar manner of discretization is proposed to that used in the recent \texttt{STRUPHY} code \cite{Holderied_Possanner_Wang_2021, Holderied_2022, Li_et_al_2023} with a finite-element model for the background and a pseudo-particle/PiC model for the correction.

        The fluid background satisfies the full, non-linear, resistive, compressible, Hall MHD equations. \cite{Laakmann_Hu_Farrell_2022} introduces finite-element(-in-space) implicit timesteppers for the incompressible analogue to this system with structure-preserving (SP) properties in the ideal case, alongside parameter-robust preconditioners. We show that these timesteppers can derive from a finite-element-in-time (FET) (and finite-element-in-space) interpretation. The benefits of this reformulation are discussed, including the derivation of timesteppers that are higher order in time, and the quantifiable dissipative SP properties in the non-ideal, resistive case.
        
        We discuss possible options for extending this FET approach to timesteppers for the compressible case.

        The kinetic corrections satisfy linearized Boltzmann equations. Using a Lénard--Bernstein collision operator, these take Fokker--Planck-like forms \cite{Fokker_1914, Planck_1917} wherein pseudo-particles in the numerical model obey the neoclassical transport equations, with particle-independent Brownian drift terms. This offers a rigorous methodology for incorporating collisions into the particle transport model, without coupling the equations of motions for each particle.
        
        Works by Chen, Chacón et al. \cite{Chen_Chacón_Barnes_2011, Chacón_Chen_Barnes_2013, Chen_Chacón_2014, Chen_Chacón_2015} have developed structure-preserving particle pushers for neoclassical transport in the Vlasov equations, derived from Crank--Nicolson integrators. We show these too can can derive from a FET interpretation, similarly offering potential extensions to higher-order-in-time particle pushers. The FET formulation is used also to consider how the stochastic drift terms can be incorporated into the pushers. Stochastic gyrokinetic expansions are also discussed.

        Different options for the numerical implementation of these schemes are considered.

        Due to the efficacy of FET in the development of SP timesteppers for both the fluid and kinetic component, we hope this approach will prove effective in the future for developing SP timesteppers for the full hybrid model. We hope this will give us the opportunity to incorporate previously inaccessible kinetic effects into the highly effective, modern, finite-element MHD models.
    \end{abstract}
    
    
    \newpage
    \tableofcontents
    
    
    \newpage
    \pagenumbering{arabic}
    %\linenumbers\renewcommand\thelinenumber{\color{black!50}\arabic{linenumber}}
            \input{0 - introduction/main.tex}
        \part{Research}
            \input{1 - low-noise PiC models/main.tex}
            \input{2 - kinetic component/main.tex}
            \input{3 - fluid component/main.tex}
            \input{4 - numerical implementation/main.tex}
        \part{Project Overview}
            \input{5 - research plan/main.tex}
            \input{6 - summary/main.tex}
    
    
    %\section{}
    \newpage
    \pagenumbering{gobble}
        \printbibliography


    \newpage
    \pagenumbering{roman}
    \appendix
        \part{Appendices}
            \input{8 - Hilbert complexes/main.tex}
            \input{9 - weak conservation proofs/main.tex}
\end{document}

        \part{Research}
            \documentclass[12pt, a4paper]{report}

\input{template/main.tex}

\title{\BA{Title in Progress...}}
\author{Boris Andrews}
\affil{Mathematical Institute, University of Oxford}
\date{\today}


\begin{document}
    \pagenumbering{gobble}
    \maketitle
    
    
    \begin{abstract}
        Magnetic confinement reactors---in particular tokamaks---offer one of the most promising options for achieving practical nuclear fusion, with the potential to provide virtually limitless, clean energy. The theoretical and numerical modeling of tokamak plasmas is simultaneously an essential component of effective reactor design, and a great research barrier. Tokamak operational conditions exhibit comparatively low Knudsen numbers. Kinetic effects, including kinetic waves and instabilities, Landau damping, bump-on-tail instabilities and more, are therefore highly influential in tokamak plasma dynamics. Purely fluid models are inherently incapable of capturing these effects, whereas the high dimensionality in purely kinetic models render them practically intractable for most relevant purposes.

        We consider a $\delta\!f$ decomposition model, with a macroscopic fluid background and microscopic kinetic correction, both fully coupled to each other. A similar manner of discretization is proposed to that used in the recent \texttt{STRUPHY} code \cite{Holderied_Possanner_Wang_2021, Holderied_2022, Li_et_al_2023} with a finite-element model for the background and a pseudo-particle/PiC model for the correction.

        The fluid background satisfies the full, non-linear, resistive, compressible, Hall MHD equations. \cite{Laakmann_Hu_Farrell_2022} introduces finite-element(-in-space) implicit timesteppers for the incompressible analogue to this system with structure-preserving (SP) properties in the ideal case, alongside parameter-robust preconditioners. We show that these timesteppers can derive from a finite-element-in-time (FET) (and finite-element-in-space) interpretation. The benefits of this reformulation are discussed, including the derivation of timesteppers that are higher order in time, and the quantifiable dissipative SP properties in the non-ideal, resistive case.
        
        We discuss possible options for extending this FET approach to timesteppers for the compressible case.

        The kinetic corrections satisfy linearized Boltzmann equations. Using a Lénard--Bernstein collision operator, these take Fokker--Planck-like forms \cite{Fokker_1914, Planck_1917} wherein pseudo-particles in the numerical model obey the neoclassical transport equations, with particle-independent Brownian drift terms. This offers a rigorous methodology for incorporating collisions into the particle transport model, without coupling the equations of motions for each particle.
        
        Works by Chen, Chacón et al. \cite{Chen_Chacón_Barnes_2011, Chacón_Chen_Barnes_2013, Chen_Chacón_2014, Chen_Chacón_2015} have developed structure-preserving particle pushers for neoclassical transport in the Vlasov equations, derived from Crank--Nicolson integrators. We show these too can can derive from a FET interpretation, similarly offering potential extensions to higher-order-in-time particle pushers. The FET formulation is used also to consider how the stochastic drift terms can be incorporated into the pushers. Stochastic gyrokinetic expansions are also discussed.

        Different options for the numerical implementation of these schemes are considered.

        Due to the efficacy of FET in the development of SP timesteppers for both the fluid and kinetic component, we hope this approach will prove effective in the future for developing SP timesteppers for the full hybrid model. We hope this will give us the opportunity to incorporate previously inaccessible kinetic effects into the highly effective, modern, finite-element MHD models.
    \end{abstract}
    
    
    \newpage
    \tableofcontents
    
    
    \newpage
    \pagenumbering{arabic}
    %\linenumbers\renewcommand\thelinenumber{\color{black!50}\arabic{linenumber}}
            \input{0 - introduction/main.tex}
        \part{Research}
            \input{1 - low-noise PiC models/main.tex}
            \input{2 - kinetic component/main.tex}
            \input{3 - fluid component/main.tex}
            \input{4 - numerical implementation/main.tex}
        \part{Project Overview}
            \input{5 - research plan/main.tex}
            \input{6 - summary/main.tex}
    
    
    %\section{}
    \newpage
    \pagenumbering{gobble}
        \printbibliography


    \newpage
    \pagenumbering{roman}
    \appendix
        \part{Appendices}
            \input{8 - Hilbert complexes/main.tex}
            \input{9 - weak conservation proofs/main.tex}
\end{document}

            \documentclass[12pt, a4paper]{report}

\input{template/main.tex}

\title{\BA{Title in Progress...}}
\author{Boris Andrews}
\affil{Mathematical Institute, University of Oxford}
\date{\today}


\begin{document}
    \pagenumbering{gobble}
    \maketitle
    
    
    \begin{abstract}
        Magnetic confinement reactors---in particular tokamaks---offer one of the most promising options for achieving practical nuclear fusion, with the potential to provide virtually limitless, clean energy. The theoretical and numerical modeling of tokamak plasmas is simultaneously an essential component of effective reactor design, and a great research barrier. Tokamak operational conditions exhibit comparatively low Knudsen numbers. Kinetic effects, including kinetic waves and instabilities, Landau damping, bump-on-tail instabilities and more, are therefore highly influential in tokamak plasma dynamics. Purely fluid models are inherently incapable of capturing these effects, whereas the high dimensionality in purely kinetic models render them practically intractable for most relevant purposes.

        We consider a $\delta\!f$ decomposition model, with a macroscopic fluid background and microscopic kinetic correction, both fully coupled to each other. A similar manner of discretization is proposed to that used in the recent \texttt{STRUPHY} code \cite{Holderied_Possanner_Wang_2021, Holderied_2022, Li_et_al_2023} with a finite-element model for the background and a pseudo-particle/PiC model for the correction.

        The fluid background satisfies the full, non-linear, resistive, compressible, Hall MHD equations. \cite{Laakmann_Hu_Farrell_2022} introduces finite-element(-in-space) implicit timesteppers for the incompressible analogue to this system with structure-preserving (SP) properties in the ideal case, alongside parameter-robust preconditioners. We show that these timesteppers can derive from a finite-element-in-time (FET) (and finite-element-in-space) interpretation. The benefits of this reformulation are discussed, including the derivation of timesteppers that are higher order in time, and the quantifiable dissipative SP properties in the non-ideal, resistive case.
        
        We discuss possible options for extending this FET approach to timesteppers for the compressible case.

        The kinetic corrections satisfy linearized Boltzmann equations. Using a Lénard--Bernstein collision operator, these take Fokker--Planck-like forms \cite{Fokker_1914, Planck_1917} wherein pseudo-particles in the numerical model obey the neoclassical transport equations, with particle-independent Brownian drift terms. This offers a rigorous methodology for incorporating collisions into the particle transport model, without coupling the equations of motions for each particle.
        
        Works by Chen, Chacón et al. \cite{Chen_Chacón_Barnes_2011, Chacón_Chen_Barnes_2013, Chen_Chacón_2014, Chen_Chacón_2015} have developed structure-preserving particle pushers for neoclassical transport in the Vlasov equations, derived from Crank--Nicolson integrators. We show these too can can derive from a FET interpretation, similarly offering potential extensions to higher-order-in-time particle pushers. The FET formulation is used also to consider how the stochastic drift terms can be incorporated into the pushers. Stochastic gyrokinetic expansions are also discussed.

        Different options for the numerical implementation of these schemes are considered.

        Due to the efficacy of FET in the development of SP timesteppers for both the fluid and kinetic component, we hope this approach will prove effective in the future for developing SP timesteppers for the full hybrid model. We hope this will give us the opportunity to incorporate previously inaccessible kinetic effects into the highly effective, modern, finite-element MHD models.
    \end{abstract}
    
    
    \newpage
    \tableofcontents
    
    
    \newpage
    \pagenumbering{arabic}
    %\linenumbers\renewcommand\thelinenumber{\color{black!50}\arabic{linenumber}}
            \input{0 - introduction/main.tex}
        \part{Research}
            \input{1 - low-noise PiC models/main.tex}
            \input{2 - kinetic component/main.tex}
            \input{3 - fluid component/main.tex}
            \input{4 - numerical implementation/main.tex}
        \part{Project Overview}
            \input{5 - research plan/main.tex}
            \input{6 - summary/main.tex}
    
    
    %\section{}
    \newpage
    \pagenumbering{gobble}
        \printbibliography


    \newpage
    \pagenumbering{roman}
    \appendix
        \part{Appendices}
            \input{8 - Hilbert complexes/main.tex}
            \input{9 - weak conservation proofs/main.tex}
\end{document}

            \documentclass[12pt, a4paper]{report}

\input{template/main.tex}

\title{\BA{Title in Progress...}}
\author{Boris Andrews}
\affil{Mathematical Institute, University of Oxford}
\date{\today}


\begin{document}
    \pagenumbering{gobble}
    \maketitle
    
    
    \begin{abstract}
        Magnetic confinement reactors---in particular tokamaks---offer one of the most promising options for achieving practical nuclear fusion, with the potential to provide virtually limitless, clean energy. The theoretical and numerical modeling of tokamak plasmas is simultaneously an essential component of effective reactor design, and a great research barrier. Tokamak operational conditions exhibit comparatively low Knudsen numbers. Kinetic effects, including kinetic waves and instabilities, Landau damping, bump-on-tail instabilities and more, are therefore highly influential in tokamak plasma dynamics. Purely fluid models are inherently incapable of capturing these effects, whereas the high dimensionality in purely kinetic models render them practically intractable for most relevant purposes.

        We consider a $\delta\!f$ decomposition model, with a macroscopic fluid background and microscopic kinetic correction, both fully coupled to each other. A similar manner of discretization is proposed to that used in the recent \texttt{STRUPHY} code \cite{Holderied_Possanner_Wang_2021, Holderied_2022, Li_et_al_2023} with a finite-element model for the background and a pseudo-particle/PiC model for the correction.

        The fluid background satisfies the full, non-linear, resistive, compressible, Hall MHD equations. \cite{Laakmann_Hu_Farrell_2022} introduces finite-element(-in-space) implicit timesteppers for the incompressible analogue to this system with structure-preserving (SP) properties in the ideal case, alongside parameter-robust preconditioners. We show that these timesteppers can derive from a finite-element-in-time (FET) (and finite-element-in-space) interpretation. The benefits of this reformulation are discussed, including the derivation of timesteppers that are higher order in time, and the quantifiable dissipative SP properties in the non-ideal, resistive case.
        
        We discuss possible options for extending this FET approach to timesteppers for the compressible case.

        The kinetic corrections satisfy linearized Boltzmann equations. Using a Lénard--Bernstein collision operator, these take Fokker--Planck-like forms \cite{Fokker_1914, Planck_1917} wherein pseudo-particles in the numerical model obey the neoclassical transport equations, with particle-independent Brownian drift terms. This offers a rigorous methodology for incorporating collisions into the particle transport model, without coupling the equations of motions for each particle.
        
        Works by Chen, Chacón et al. \cite{Chen_Chacón_Barnes_2011, Chacón_Chen_Barnes_2013, Chen_Chacón_2014, Chen_Chacón_2015} have developed structure-preserving particle pushers for neoclassical transport in the Vlasov equations, derived from Crank--Nicolson integrators. We show these too can can derive from a FET interpretation, similarly offering potential extensions to higher-order-in-time particle pushers. The FET formulation is used also to consider how the stochastic drift terms can be incorporated into the pushers. Stochastic gyrokinetic expansions are also discussed.

        Different options for the numerical implementation of these schemes are considered.

        Due to the efficacy of FET in the development of SP timesteppers for both the fluid and kinetic component, we hope this approach will prove effective in the future for developing SP timesteppers for the full hybrid model. We hope this will give us the opportunity to incorporate previously inaccessible kinetic effects into the highly effective, modern, finite-element MHD models.
    \end{abstract}
    
    
    \newpage
    \tableofcontents
    
    
    \newpage
    \pagenumbering{arabic}
    %\linenumbers\renewcommand\thelinenumber{\color{black!50}\arabic{linenumber}}
            \input{0 - introduction/main.tex}
        \part{Research}
            \input{1 - low-noise PiC models/main.tex}
            \input{2 - kinetic component/main.tex}
            \input{3 - fluid component/main.tex}
            \input{4 - numerical implementation/main.tex}
        \part{Project Overview}
            \input{5 - research plan/main.tex}
            \input{6 - summary/main.tex}
    
    
    %\section{}
    \newpage
    \pagenumbering{gobble}
        \printbibliography


    \newpage
    \pagenumbering{roman}
    \appendix
        \part{Appendices}
            \input{8 - Hilbert complexes/main.tex}
            \input{9 - weak conservation proofs/main.tex}
\end{document}

            \documentclass[12pt, a4paper]{report}

\input{template/main.tex}

\title{\BA{Title in Progress...}}
\author{Boris Andrews}
\affil{Mathematical Institute, University of Oxford}
\date{\today}


\begin{document}
    \pagenumbering{gobble}
    \maketitle
    
    
    \begin{abstract}
        Magnetic confinement reactors---in particular tokamaks---offer one of the most promising options for achieving practical nuclear fusion, with the potential to provide virtually limitless, clean energy. The theoretical and numerical modeling of tokamak plasmas is simultaneously an essential component of effective reactor design, and a great research barrier. Tokamak operational conditions exhibit comparatively low Knudsen numbers. Kinetic effects, including kinetic waves and instabilities, Landau damping, bump-on-tail instabilities and more, are therefore highly influential in tokamak plasma dynamics. Purely fluid models are inherently incapable of capturing these effects, whereas the high dimensionality in purely kinetic models render them practically intractable for most relevant purposes.

        We consider a $\delta\!f$ decomposition model, with a macroscopic fluid background and microscopic kinetic correction, both fully coupled to each other. A similar manner of discretization is proposed to that used in the recent \texttt{STRUPHY} code \cite{Holderied_Possanner_Wang_2021, Holderied_2022, Li_et_al_2023} with a finite-element model for the background and a pseudo-particle/PiC model for the correction.

        The fluid background satisfies the full, non-linear, resistive, compressible, Hall MHD equations. \cite{Laakmann_Hu_Farrell_2022} introduces finite-element(-in-space) implicit timesteppers for the incompressible analogue to this system with structure-preserving (SP) properties in the ideal case, alongside parameter-robust preconditioners. We show that these timesteppers can derive from a finite-element-in-time (FET) (and finite-element-in-space) interpretation. The benefits of this reformulation are discussed, including the derivation of timesteppers that are higher order in time, and the quantifiable dissipative SP properties in the non-ideal, resistive case.
        
        We discuss possible options for extending this FET approach to timesteppers for the compressible case.

        The kinetic corrections satisfy linearized Boltzmann equations. Using a Lénard--Bernstein collision operator, these take Fokker--Planck-like forms \cite{Fokker_1914, Planck_1917} wherein pseudo-particles in the numerical model obey the neoclassical transport equations, with particle-independent Brownian drift terms. This offers a rigorous methodology for incorporating collisions into the particle transport model, without coupling the equations of motions for each particle.
        
        Works by Chen, Chacón et al. \cite{Chen_Chacón_Barnes_2011, Chacón_Chen_Barnes_2013, Chen_Chacón_2014, Chen_Chacón_2015} have developed structure-preserving particle pushers for neoclassical transport in the Vlasov equations, derived from Crank--Nicolson integrators. We show these too can can derive from a FET interpretation, similarly offering potential extensions to higher-order-in-time particle pushers. The FET formulation is used also to consider how the stochastic drift terms can be incorporated into the pushers. Stochastic gyrokinetic expansions are also discussed.

        Different options for the numerical implementation of these schemes are considered.

        Due to the efficacy of FET in the development of SP timesteppers for both the fluid and kinetic component, we hope this approach will prove effective in the future for developing SP timesteppers for the full hybrid model. We hope this will give us the opportunity to incorporate previously inaccessible kinetic effects into the highly effective, modern, finite-element MHD models.
    \end{abstract}
    
    
    \newpage
    \tableofcontents
    
    
    \newpage
    \pagenumbering{arabic}
    %\linenumbers\renewcommand\thelinenumber{\color{black!50}\arabic{linenumber}}
            \input{0 - introduction/main.tex}
        \part{Research}
            \input{1 - low-noise PiC models/main.tex}
            \input{2 - kinetic component/main.tex}
            \input{3 - fluid component/main.tex}
            \input{4 - numerical implementation/main.tex}
        \part{Project Overview}
            \input{5 - research plan/main.tex}
            \input{6 - summary/main.tex}
    
    
    %\section{}
    \newpage
    \pagenumbering{gobble}
        \printbibliography


    \newpage
    \pagenumbering{roman}
    \appendix
        \part{Appendices}
            \input{8 - Hilbert complexes/main.tex}
            \input{9 - weak conservation proofs/main.tex}
\end{document}

        \part{Project Overview}
            \documentclass[12pt, a4paper]{report}

\input{template/main.tex}

\title{\BA{Title in Progress...}}
\author{Boris Andrews}
\affil{Mathematical Institute, University of Oxford}
\date{\today}


\begin{document}
    \pagenumbering{gobble}
    \maketitle
    
    
    \begin{abstract}
        Magnetic confinement reactors---in particular tokamaks---offer one of the most promising options for achieving practical nuclear fusion, with the potential to provide virtually limitless, clean energy. The theoretical and numerical modeling of tokamak plasmas is simultaneously an essential component of effective reactor design, and a great research barrier. Tokamak operational conditions exhibit comparatively low Knudsen numbers. Kinetic effects, including kinetic waves and instabilities, Landau damping, bump-on-tail instabilities and more, are therefore highly influential in tokamak plasma dynamics. Purely fluid models are inherently incapable of capturing these effects, whereas the high dimensionality in purely kinetic models render them practically intractable for most relevant purposes.

        We consider a $\delta\!f$ decomposition model, with a macroscopic fluid background and microscopic kinetic correction, both fully coupled to each other. A similar manner of discretization is proposed to that used in the recent \texttt{STRUPHY} code \cite{Holderied_Possanner_Wang_2021, Holderied_2022, Li_et_al_2023} with a finite-element model for the background and a pseudo-particle/PiC model for the correction.

        The fluid background satisfies the full, non-linear, resistive, compressible, Hall MHD equations. \cite{Laakmann_Hu_Farrell_2022} introduces finite-element(-in-space) implicit timesteppers for the incompressible analogue to this system with structure-preserving (SP) properties in the ideal case, alongside parameter-robust preconditioners. We show that these timesteppers can derive from a finite-element-in-time (FET) (and finite-element-in-space) interpretation. The benefits of this reformulation are discussed, including the derivation of timesteppers that are higher order in time, and the quantifiable dissipative SP properties in the non-ideal, resistive case.
        
        We discuss possible options for extending this FET approach to timesteppers for the compressible case.

        The kinetic corrections satisfy linearized Boltzmann equations. Using a Lénard--Bernstein collision operator, these take Fokker--Planck-like forms \cite{Fokker_1914, Planck_1917} wherein pseudo-particles in the numerical model obey the neoclassical transport equations, with particle-independent Brownian drift terms. This offers a rigorous methodology for incorporating collisions into the particle transport model, without coupling the equations of motions for each particle.
        
        Works by Chen, Chacón et al. \cite{Chen_Chacón_Barnes_2011, Chacón_Chen_Barnes_2013, Chen_Chacón_2014, Chen_Chacón_2015} have developed structure-preserving particle pushers for neoclassical transport in the Vlasov equations, derived from Crank--Nicolson integrators. We show these too can can derive from a FET interpretation, similarly offering potential extensions to higher-order-in-time particle pushers. The FET formulation is used also to consider how the stochastic drift terms can be incorporated into the pushers. Stochastic gyrokinetic expansions are also discussed.

        Different options for the numerical implementation of these schemes are considered.

        Due to the efficacy of FET in the development of SP timesteppers for both the fluid and kinetic component, we hope this approach will prove effective in the future for developing SP timesteppers for the full hybrid model. We hope this will give us the opportunity to incorporate previously inaccessible kinetic effects into the highly effective, modern, finite-element MHD models.
    \end{abstract}
    
    
    \newpage
    \tableofcontents
    
    
    \newpage
    \pagenumbering{arabic}
    %\linenumbers\renewcommand\thelinenumber{\color{black!50}\arabic{linenumber}}
            \input{0 - introduction/main.tex}
        \part{Research}
            \input{1 - low-noise PiC models/main.tex}
            \input{2 - kinetic component/main.tex}
            \input{3 - fluid component/main.tex}
            \input{4 - numerical implementation/main.tex}
        \part{Project Overview}
            \input{5 - research plan/main.tex}
            \input{6 - summary/main.tex}
    
    
    %\section{}
    \newpage
    \pagenumbering{gobble}
        \printbibliography


    \newpage
    \pagenumbering{roman}
    \appendix
        \part{Appendices}
            \input{8 - Hilbert complexes/main.tex}
            \input{9 - weak conservation proofs/main.tex}
\end{document}

            \documentclass[12pt, a4paper]{report}

\input{template/main.tex}

\title{\BA{Title in Progress...}}
\author{Boris Andrews}
\affil{Mathematical Institute, University of Oxford}
\date{\today}


\begin{document}
    \pagenumbering{gobble}
    \maketitle
    
    
    \begin{abstract}
        Magnetic confinement reactors---in particular tokamaks---offer one of the most promising options for achieving practical nuclear fusion, with the potential to provide virtually limitless, clean energy. The theoretical and numerical modeling of tokamak plasmas is simultaneously an essential component of effective reactor design, and a great research barrier. Tokamak operational conditions exhibit comparatively low Knudsen numbers. Kinetic effects, including kinetic waves and instabilities, Landau damping, bump-on-tail instabilities and more, are therefore highly influential in tokamak plasma dynamics. Purely fluid models are inherently incapable of capturing these effects, whereas the high dimensionality in purely kinetic models render them practically intractable for most relevant purposes.

        We consider a $\delta\!f$ decomposition model, with a macroscopic fluid background and microscopic kinetic correction, both fully coupled to each other. A similar manner of discretization is proposed to that used in the recent \texttt{STRUPHY} code \cite{Holderied_Possanner_Wang_2021, Holderied_2022, Li_et_al_2023} with a finite-element model for the background and a pseudo-particle/PiC model for the correction.

        The fluid background satisfies the full, non-linear, resistive, compressible, Hall MHD equations. \cite{Laakmann_Hu_Farrell_2022} introduces finite-element(-in-space) implicit timesteppers for the incompressible analogue to this system with structure-preserving (SP) properties in the ideal case, alongside parameter-robust preconditioners. We show that these timesteppers can derive from a finite-element-in-time (FET) (and finite-element-in-space) interpretation. The benefits of this reformulation are discussed, including the derivation of timesteppers that are higher order in time, and the quantifiable dissipative SP properties in the non-ideal, resistive case.
        
        We discuss possible options for extending this FET approach to timesteppers for the compressible case.

        The kinetic corrections satisfy linearized Boltzmann equations. Using a Lénard--Bernstein collision operator, these take Fokker--Planck-like forms \cite{Fokker_1914, Planck_1917} wherein pseudo-particles in the numerical model obey the neoclassical transport equations, with particle-independent Brownian drift terms. This offers a rigorous methodology for incorporating collisions into the particle transport model, without coupling the equations of motions for each particle.
        
        Works by Chen, Chacón et al. \cite{Chen_Chacón_Barnes_2011, Chacón_Chen_Barnes_2013, Chen_Chacón_2014, Chen_Chacón_2015} have developed structure-preserving particle pushers for neoclassical transport in the Vlasov equations, derived from Crank--Nicolson integrators. We show these too can can derive from a FET interpretation, similarly offering potential extensions to higher-order-in-time particle pushers. The FET formulation is used also to consider how the stochastic drift terms can be incorporated into the pushers. Stochastic gyrokinetic expansions are also discussed.

        Different options for the numerical implementation of these schemes are considered.

        Due to the efficacy of FET in the development of SP timesteppers for both the fluid and kinetic component, we hope this approach will prove effective in the future for developing SP timesteppers for the full hybrid model. We hope this will give us the opportunity to incorporate previously inaccessible kinetic effects into the highly effective, modern, finite-element MHD models.
    \end{abstract}
    
    
    \newpage
    \tableofcontents
    
    
    \newpage
    \pagenumbering{arabic}
    %\linenumbers\renewcommand\thelinenumber{\color{black!50}\arabic{linenumber}}
            \input{0 - introduction/main.tex}
        \part{Research}
            \input{1 - low-noise PiC models/main.tex}
            \input{2 - kinetic component/main.tex}
            \input{3 - fluid component/main.tex}
            \input{4 - numerical implementation/main.tex}
        \part{Project Overview}
            \input{5 - research plan/main.tex}
            \input{6 - summary/main.tex}
    
    
    %\section{}
    \newpage
    \pagenumbering{gobble}
        \printbibliography


    \newpage
    \pagenumbering{roman}
    \appendix
        \part{Appendices}
            \input{8 - Hilbert complexes/main.tex}
            \input{9 - weak conservation proofs/main.tex}
\end{document}

    
    
    %\section{}
    \newpage
    \pagenumbering{gobble}
        \printbibliography


    \newpage
    \pagenumbering{roman}
    \appendix
        \part{Appendices}
            \documentclass[12pt, a4paper]{report}

\input{template/main.tex}

\title{\BA{Title in Progress...}}
\author{Boris Andrews}
\affil{Mathematical Institute, University of Oxford}
\date{\today}


\begin{document}
    \pagenumbering{gobble}
    \maketitle
    
    
    \begin{abstract}
        Magnetic confinement reactors---in particular tokamaks---offer one of the most promising options for achieving practical nuclear fusion, with the potential to provide virtually limitless, clean energy. The theoretical and numerical modeling of tokamak plasmas is simultaneously an essential component of effective reactor design, and a great research barrier. Tokamak operational conditions exhibit comparatively low Knudsen numbers. Kinetic effects, including kinetic waves and instabilities, Landau damping, bump-on-tail instabilities and more, are therefore highly influential in tokamak plasma dynamics. Purely fluid models are inherently incapable of capturing these effects, whereas the high dimensionality in purely kinetic models render them practically intractable for most relevant purposes.

        We consider a $\delta\!f$ decomposition model, with a macroscopic fluid background and microscopic kinetic correction, both fully coupled to each other. A similar manner of discretization is proposed to that used in the recent \texttt{STRUPHY} code \cite{Holderied_Possanner_Wang_2021, Holderied_2022, Li_et_al_2023} with a finite-element model for the background and a pseudo-particle/PiC model for the correction.

        The fluid background satisfies the full, non-linear, resistive, compressible, Hall MHD equations. \cite{Laakmann_Hu_Farrell_2022} introduces finite-element(-in-space) implicit timesteppers for the incompressible analogue to this system with structure-preserving (SP) properties in the ideal case, alongside parameter-robust preconditioners. We show that these timesteppers can derive from a finite-element-in-time (FET) (and finite-element-in-space) interpretation. The benefits of this reformulation are discussed, including the derivation of timesteppers that are higher order in time, and the quantifiable dissipative SP properties in the non-ideal, resistive case.
        
        We discuss possible options for extending this FET approach to timesteppers for the compressible case.

        The kinetic corrections satisfy linearized Boltzmann equations. Using a Lénard--Bernstein collision operator, these take Fokker--Planck-like forms \cite{Fokker_1914, Planck_1917} wherein pseudo-particles in the numerical model obey the neoclassical transport equations, with particle-independent Brownian drift terms. This offers a rigorous methodology for incorporating collisions into the particle transport model, without coupling the equations of motions for each particle.
        
        Works by Chen, Chacón et al. \cite{Chen_Chacón_Barnes_2011, Chacón_Chen_Barnes_2013, Chen_Chacón_2014, Chen_Chacón_2015} have developed structure-preserving particle pushers for neoclassical transport in the Vlasov equations, derived from Crank--Nicolson integrators. We show these too can can derive from a FET interpretation, similarly offering potential extensions to higher-order-in-time particle pushers. The FET formulation is used also to consider how the stochastic drift terms can be incorporated into the pushers. Stochastic gyrokinetic expansions are also discussed.

        Different options for the numerical implementation of these schemes are considered.

        Due to the efficacy of FET in the development of SP timesteppers for both the fluid and kinetic component, we hope this approach will prove effective in the future for developing SP timesteppers for the full hybrid model. We hope this will give us the opportunity to incorporate previously inaccessible kinetic effects into the highly effective, modern, finite-element MHD models.
    \end{abstract}
    
    
    \newpage
    \tableofcontents
    
    
    \newpage
    \pagenumbering{arabic}
    %\linenumbers\renewcommand\thelinenumber{\color{black!50}\arabic{linenumber}}
            \input{0 - introduction/main.tex}
        \part{Research}
            \input{1 - low-noise PiC models/main.tex}
            \input{2 - kinetic component/main.tex}
            \input{3 - fluid component/main.tex}
            \input{4 - numerical implementation/main.tex}
        \part{Project Overview}
            \input{5 - research plan/main.tex}
            \input{6 - summary/main.tex}
    
    
    %\section{}
    \newpage
    \pagenumbering{gobble}
        \printbibliography


    \newpage
    \pagenumbering{roman}
    \appendix
        \part{Appendices}
            \input{8 - Hilbert complexes/main.tex}
            \input{9 - weak conservation proofs/main.tex}
\end{document}

            \documentclass[12pt, a4paper]{report}

\input{template/main.tex}

\title{\BA{Title in Progress...}}
\author{Boris Andrews}
\affil{Mathematical Institute, University of Oxford}
\date{\today}


\begin{document}
    \pagenumbering{gobble}
    \maketitle
    
    
    \begin{abstract}
        Magnetic confinement reactors---in particular tokamaks---offer one of the most promising options for achieving practical nuclear fusion, with the potential to provide virtually limitless, clean energy. The theoretical and numerical modeling of tokamak plasmas is simultaneously an essential component of effective reactor design, and a great research barrier. Tokamak operational conditions exhibit comparatively low Knudsen numbers. Kinetic effects, including kinetic waves and instabilities, Landau damping, bump-on-tail instabilities and more, are therefore highly influential in tokamak plasma dynamics. Purely fluid models are inherently incapable of capturing these effects, whereas the high dimensionality in purely kinetic models render them practically intractable for most relevant purposes.

        We consider a $\delta\!f$ decomposition model, with a macroscopic fluid background and microscopic kinetic correction, both fully coupled to each other. A similar manner of discretization is proposed to that used in the recent \texttt{STRUPHY} code \cite{Holderied_Possanner_Wang_2021, Holderied_2022, Li_et_al_2023} with a finite-element model for the background and a pseudo-particle/PiC model for the correction.

        The fluid background satisfies the full, non-linear, resistive, compressible, Hall MHD equations. \cite{Laakmann_Hu_Farrell_2022} introduces finite-element(-in-space) implicit timesteppers for the incompressible analogue to this system with structure-preserving (SP) properties in the ideal case, alongside parameter-robust preconditioners. We show that these timesteppers can derive from a finite-element-in-time (FET) (and finite-element-in-space) interpretation. The benefits of this reformulation are discussed, including the derivation of timesteppers that are higher order in time, and the quantifiable dissipative SP properties in the non-ideal, resistive case.
        
        We discuss possible options for extending this FET approach to timesteppers for the compressible case.

        The kinetic corrections satisfy linearized Boltzmann equations. Using a Lénard--Bernstein collision operator, these take Fokker--Planck-like forms \cite{Fokker_1914, Planck_1917} wherein pseudo-particles in the numerical model obey the neoclassical transport equations, with particle-independent Brownian drift terms. This offers a rigorous methodology for incorporating collisions into the particle transport model, without coupling the equations of motions for each particle.
        
        Works by Chen, Chacón et al. \cite{Chen_Chacón_Barnes_2011, Chacón_Chen_Barnes_2013, Chen_Chacón_2014, Chen_Chacón_2015} have developed structure-preserving particle pushers for neoclassical transport in the Vlasov equations, derived from Crank--Nicolson integrators. We show these too can can derive from a FET interpretation, similarly offering potential extensions to higher-order-in-time particle pushers. The FET formulation is used also to consider how the stochastic drift terms can be incorporated into the pushers. Stochastic gyrokinetic expansions are also discussed.

        Different options for the numerical implementation of these schemes are considered.

        Due to the efficacy of FET in the development of SP timesteppers for both the fluid and kinetic component, we hope this approach will prove effective in the future for developing SP timesteppers for the full hybrid model. We hope this will give us the opportunity to incorporate previously inaccessible kinetic effects into the highly effective, modern, finite-element MHD models.
    \end{abstract}
    
    
    \newpage
    \tableofcontents
    
    
    \newpage
    \pagenumbering{arabic}
    %\linenumbers\renewcommand\thelinenumber{\color{black!50}\arabic{linenumber}}
            \input{0 - introduction/main.tex}
        \part{Research}
            \input{1 - low-noise PiC models/main.tex}
            \input{2 - kinetic component/main.tex}
            \input{3 - fluid component/main.tex}
            \input{4 - numerical implementation/main.tex}
        \part{Project Overview}
            \input{5 - research plan/main.tex}
            \input{6 - summary/main.tex}
    
    
    %\section{}
    \newpage
    \pagenumbering{gobble}
        \printbibliography


    \newpage
    \pagenumbering{roman}
    \appendix
        \part{Appendices}
            \input{8 - Hilbert complexes/main.tex}
            \input{9 - weak conservation proofs/main.tex}
\end{document}

\end{document}

            \documentclass[12pt, a4paper]{report}

\documentclass[12pt, a4paper]{report}

\input{template/main.tex}

\title{\BA{Title in Progress...}}
\author{Boris Andrews}
\affil{Mathematical Institute, University of Oxford}
\date{\today}


\begin{document}
    \pagenumbering{gobble}
    \maketitle
    
    
    \begin{abstract}
        Magnetic confinement reactors---in particular tokamaks---offer one of the most promising options for achieving practical nuclear fusion, with the potential to provide virtually limitless, clean energy. The theoretical and numerical modeling of tokamak plasmas is simultaneously an essential component of effective reactor design, and a great research barrier. Tokamak operational conditions exhibit comparatively low Knudsen numbers. Kinetic effects, including kinetic waves and instabilities, Landau damping, bump-on-tail instabilities and more, are therefore highly influential in tokamak plasma dynamics. Purely fluid models are inherently incapable of capturing these effects, whereas the high dimensionality in purely kinetic models render them practically intractable for most relevant purposes.

        We consider a $\delta\!f$ decomposition model, with a macroscopic fluid background and microscopic kinetic correction, both fully coupled to each other. A similar manner of discretization is proposed to that used in the recent \texttt{STRUPHY} code \cite{Holderied_Possanner_Wang_2021, Holderied_2022, Li_et_al_2023} with a finite-element model for the background and a pseudo-particle/PiC model for the correction.

        The fluid background satisfies the full, non-linear, resistive, compressible, Hall MHD equations. \cite{Laakmann_Hu_Farrell_2022} introduces finite-element(-in-space) implicit timesteppers for the incompressible analogue to this system with structure-preserving (SP) properties in the ideal case, alongside parameter-robust preconditioners. We show that these timesteppers can derive from a finite-element-in-time (FET) (and finite-element-in-space) interpretation. The benefits of this reformulation are discussed, including the derivation of timesteppers that are higher order in time, and the quantifiable dissipative SP properties in the non-ideal, resistive case.
        
        We discuss possible options for extending this FET approach to timesteppers for the compressible case.

        The kinetic corrections satisfy linearized Boltzmann equations. Using a Lénard--Bernstein collision operator, these take Fokker--Planck-like forms \cite{Fokker_1914, Planck_1917} wherein pseudo-particles in the numerical model obey the neoclassical transport equations, with particle-independent Brownian drift terms. This offers a rigorous methodology for incorporating collisions into the particle transport model, without coupling the equations of motions for each particle.
        
        Works by Chen, Chacón et al. \cite{Chen_Chacón_Barnes_2011, Chacón_Chen_Barnes_2013, Chen_Chacón_2014, Chen_Chacón_2015} have developed structure-preserving particle pushers for neoclassical transport in the Vlasov equations, derived from Crank--Nicolson integrators. We show these too can can derive from a FET interpretation, similarly offering potential extensions to higher-order-in-time particle pushers. The FET formulation is used also to consider how the stochastic drift terms can be incorporated into the pushers. Stochastic gyrokinetic expansions are also discussed.

        Different options for the numerical implementation of these schemes are considered.

        Due to the efficacy of FET in the development of SP timesteppers for both the fluid and kinetic component, we hope this approach will prove effective in the future for developing SP timesteppers for the full hybrid model. We hope this will give us the opportunity to incorporate previously inaccessible kinetic effects into the highly effective, modern, finite-element MHD models.
    \end{abstract}
    
    
    \newpage
    \tableofcontents
    
    
    \newpage
    \pagenumbering{arabic}
    %\linenumbers\renewcommand\thelinenumber{\color{black!50}\arabic{linenumber}}
            \input{0 - introduction/main.tex}
        \part{Research}
            \input{1 - low-noise PiC models/main.tex}
            \input{2 - kinetic component/main.tex}
            \input{3 - fluid component/main.tex}
            \input{4 - numerical implementation/main.tex}
        \part{Project Overview}
            \input{5 - research plan/main.tex}
            \input{6 - summary/main.tex}
    
    
    %\section{}
    \newpage
    \pagenumbering{gobble}
        \printbibliography


    \newpage
    \pagenumbering{roman}
    \appendix
        \part{Appendices}
            \input{8 - Hilbert complexes/main.tex}
            \input{9 - weak conservation proofs/main.tex}
\end{document}


\title{\BA{Title in Progress...}}
\author{Boris Andrews}
\affil{Mathematical Institute, University of Oxford}
\date{\today}


\begin{document}
    \pagenumbering{gobble}
    \maketitle
    
    
    \begin{abstract}
        Magnetic confinement reactors---in particular tokamaks---offer one of the most promising options for achieving practical nuclear fusion, with the potential to provide virtually limitless, clean energy. The theoretical and numerical modeling of tokamak plasmas is simultaneously an essential component of effective reactor design, and a great research barrier. Tokamak operational conditions exhibit comparatively low Knudsen numbers. Kinetic effects, including kinetic waves and instabilities, Landau damping, bump-on-tail instabilities and more, are therefore highly influential in tokamak plasma dynamics. Purely fluid models are inherently incapable of capturing these effects, whereas the high dimensionality in purely kinetic models render them practically intractable for most relevant purposes.

        We consider a $\delta\!f$ decomposition model, with a macroscopic fluid background and microscopic kinetic correction, both fully coupled to each other. A similar manner of discretization is proposed to that used in the recent \texttt{STRUPHY} code \cite{Holderied_Possanner_Wang_2021, Holderied_2022, Li_et_al_2023} with a finite-element model for the background and a pseudo-particle/PiC model for the correction.

        The fluid background satisfies the full, non-linear, resistive, compressible, Hall MHD equations. \cite{Laakmann_Hu_Farrell_2022} introduces finite-element(-in-space) implicit timesteppers for the incompressible analogue to this system with structure-preserving (SP) properties in the ideal case, alongside parameter-robust preconditioners. We show that these timesteppers can derive from a finite-element-in-time (FET) (and finite-element-in-space) interpretation. The benefits of this reformulation are discussed, including the derivation of timesteppers that are higher order in time, and the quantifiable dissipative SP properties in the non-ideal, resistive case.
        
        We discuss possible options for extending this FET approach to timesteppers for the compressible case.

        The kinetic corrections satisfy linearized Boltzmann equations. Using a Lénard--Bernstein collision operator, these take Fokker--Planck-like forms \cite{Fokker_1914, Planck_1917} wherein pseudo-particles in the numerical model obey the neoclassical transport equations, with particle-independent Brownian drift terms. This offers a rigorous methodology for incorporating collisions into the particle transport model, without coupling the equations of motions for each particle.
        
        Works by Chen, Chacón et al. \cite{Chen_Chacón_Barnes_2011, Chacón_Chen_Barnes_2013, Chen_Chacón_2014, Chen_Chacón_2015} have developed structure-preserving particle pushers for neoclassical transport in the Vlasov equations, derived from Crank--Nicolson integrators. We show these too can can derive from a FET interpretation, similarly offering potential extensions to higher-order-in-time particle pushers. The FET formulation is used also to consider how the stochastic drift terms can be incorporated into the pushers. Stochastic gyrokinetic expansions are also discussed.

        Different options for the numerical implementation of these schemes are considered.

        Due to the efficacy of FET in the development of SP timesteppers for both the fluid and kinetic component, we hope this approach will prove effective in the future for developing SP timesteppers for the full hybrid model. We hope this will give us the opportunity to incorporate previously inaccessible kinetic effects into the highly effective, modern, finite-element MHD models.
    \end{abstract}
    
    
    \newpage
    \tableofcontents
    
    
    \newpage
    \pagenumbering{arabic}
    %\linenumbers\renewcommand\thelinenumber{\color{black!50}\arabic{linenumber}}
            \documentclass[12pt, a4paper]{report}

\input{template/main.tex}

\title{\BA{Title in Progress...}}
\author{Boris Andrews}
\affil{Mathematical Institute, University of Oxford}
\date{\today}


\begin{document}
    \pagenumbering{gobble}
    \maketitle
    
    
    \begin{abstract}
        Magnetic confinement reactors---in particular tokamaks---offer one of the most promising options for achieving practical nuclear fusion, with the potential to provide virtually limitless, clean energy. The theoretical and numerical modeling of tokamak plasmas is simultaneously an essential component of effective reactor design, and a great research barrier. Tokamak operational conditions exhibit comparatively low Knudsen numbers. Kinetic effects, including kinetic waves and instabilities, Landau damping, bump-on-tail instabilities and more, are therefore highly influential in tokamak plasma dynamics. Purely fluid models are inherently incapable of capturing these effects, whereas the high dimensionality in purely kinetic models render them practically intractable for most relevant purposes.

        We consider a $\delta\!f$ decomposition model, with a macroscopic fluid background and microscopic kinetic correction, both fully coupled to each other. A similar manner of discretization is proposed to that used in the recent \texttt{STRUPHY} code \cite{Holderied_Possanner_Wang_2021, Holderied_2022, Li_et_al_2023} with a finite-element model for the background and a pseudo-particle/PiC model for the correction.

        The fluid background satisfies the full, non-linear, resistive, compressible, Hall MHD equations. \cite{Laakmann_Hu_Farrell_2022} introduces finite-element(-in-space) implicit timesteppers for the incompressible analogue to this system with structure-preserving (SP) properties in the ideal case, alongside parameter-robust preconditioners. We show that these timesteppers can derive from a finite-element-in-time (FET) (and finite-element-in-space) interpretation. The benefits of this reformulation are discussed, including the derivation of timesteppers that are higher order in time, and the quantifiable dissipative SP properties in the non-ideal, resistive case.
        
        We discuss possible options for extending this FET approach to timesteppers for the compressible case.

        The kinetic corrections satisfy linearized Boltzmann equations. Using a Lénard--Bernstein collision operator, these take Fokker--Planck-like forms \cite{Fokker_1914, Planck_1917} wherein pseudo-particles in the numerical model obey the neoclassical transport equations, with particle-independent Brownian drift terms. This offers a rigorous methodology for incorporating collisions into the particle transport model, without coupling the equations of motions for each particle.
        
        Works by Chen, Chacón et al. \cite{Chen_Chacón_Barnes_2011, Chacón_Chen_Barnes_2013, Chen_Chacón_2014, Chen_Chacón_2015} have developed structure-preserving particle pushers for neoclassical transport in the Vlasov equations, derived from Crank--Nicolson integrators. We show these too can can derive from a FET interpretation, similarly offering potential extensions to higher-order-in-time particle pushers. The FET formulation is used also to consider how the stochastic drift terms can be incorporated into the pushers. Stochastic gyrokinetic expansions are also discussed.

        Different options for the numerical implementation of these schemes are considered.

        Due to the efficacy of FET in the development of SP timesteppers for both the fluid and kinetic component, we hope this approach will prove effective in the future for developing SP timesteppers for the full hybrid model. We hope this will give us the opportunity to incorporate previously inaccessible kinetic effects into the highly effective, modern, finite-element MHD models.
    \end{abstract}
    
    
    \newpage
    \tableofcontents
    
    
    \newpage
    \pagenumbering{arabic}
    %\linenumbers\renewcommand\thelinenumber{\color{black!50}\arabic{linenumber}}
            \input{0 - introduction/main.tex}
        \part{Research}
            \input{1 - low-noise PiC models/main.tex}
            \input{2 - kinetic component/main.tex}
            \input{3 - fluid component/main.tex}
            \input{4 - numerical implementation/main.tex}
        \part{Project Overview}
            \input{5 - research plan/main.tex}
            \input{6 - summary/main.tex}
    
    
    %\section{}
    \newpage
    \pagenumbering{gobble}
        \printbibliography


    \newpage
    \pagenumbering{roman}
    \appendix
        \part{Appendices}
            \input{8 - Hilbert complexes/main.tex}
            \input{9 - weak conservation proofs/main.tex}
\end{document}

        \part{Research}
            \documentclass[12pt, a4paper]{report}

\input{template/main.tex}

\title{\BA{Title in Progress...}}
\author{Boris Andrews}
\affil{Mathematical Institute, University of Oxford}
\date{\today}


\begin{document}
    \pagenumbering{gobble}
    \maketitle
    
    
    \begin{abstract}
        Magnetic confinement reactors---in particular tokamaks---offer one of the most promising options for achieving practical nuclear fusion, with the potential to provide virtually limitless, clean energy. The theoretical and numerical modeling of tokamak plasmas is simultaneously an essential component of effective reactor design, and a great research barrier. Tokamak operational conditions exhibit comparatively low Knudsen numbers. Kinetic effects, including kinetic waves and instabilities, Landau damping, bump-on-tail instabilities and more, are therefore highly influential in tokamak plasma dynamics. Purely fluid models are inherently incapable of capturing these effects, whereas the high dimensionality in purely kinetic models render them practically intractable for most relevant purposes.

        We consider a $\delta\!f$ decomposition model, with a macroscopic fluid background and microscopic kinetic correction, both fully coupled to each other. A similar manner of discretization is proposed to that used in the recent \texttt{STRUPHY} code \cite{Holderied_Possanner_Wang_2021, Holderied_2022, Li_et_al_2023} with a finite-element model for the background and a pseudo-particle/PiC model for the correction.

        The fluid background satisfies the full, non-linear, resistive, compressible, Hall MHD equations. \cite{Laakmann_Hu_Farrell_2022} introduces finite-element(-in-space) implicit timesteppers for the incompressible analogue to this system with structure-preserving (SP) properties in the ideal case, alongside parameter-robust preconditioners. We show that these timesteppers can derive from a finite-element-in-time (FET) (and finite-element-in-space) interpretation. The benefits of this reformulation are discussed, including the derivation of timesteppers that are higher order in time, and the quantifiable dissipative SP properties in the non-ideal, resistive case.
        
        We discuss possible options for extending this FET approach to timesteppers for the compressible case.

        The kinetic corrections satisfy linearized Boltzmann equations. Using a Lénard--Bernstein collision operator, these take Fokker--Planck-like forms \cite{Fokker_1914, Planck_1917} wherein pseudo-particles in the numerical model obey the neoclassical transport equations, with particle-independent Brownian drift terms. This offers a rigorous methodology for incorporating collisions into the particle transport model, without coupling the equations of motions for each particle.
        
        Works by Chen, Chacón et al. \cite{Chen_Chacón_Barnes_2011, Chacón_Chen_Barnes_2013, Chen_Chacón_2014, Chen_Chacón_2015} have developed structure-preserving particle pushers for neoclassical transport in the Vlasov equations, derived from Crank--Nicolson integrators. We show these too can can derive from a FET interpretation, similarly offering potential extensions to higher-order-in-time particle pushers. The FET formulation is used also to consider how the stochastic drift terms can be incorporated into the pushers. Stochastic gyrokinetic expansions are also discussed.

        Different options for the numerical implementation of these schemes are considered.

        Due to the efficacy of FET in the development of SP timesteppers for both the fluid and kinetic component, we hope this approach will prove effective in the future for developing SP timesteppers for the full hybrid model. We hope this will give us the opportunity to incorporate previously inaccessible kinetic effects into the highly effective, modern, finite-element MHD models.
    \end{abstract}
    
    
    \newpage
    \tableofcontents
    
    
    \newpage
    \pagenumbering{arabic}
    %\linenumbers\renewcommand\thelinenumber{\color{black!50}\arabic{linenumber}}
            \input{0 - introduction/main.tex}
        \part{Research}
            \input{1 - low-noise PiC models/main.tex}
            \input{2 - kinetic component/main.tex}
            \input{3 - fluid component/main.tex}
            \input{4 - numerical implementation/main.tex}
        \part{Project Overview}
            \input{5 - research plan/main.tex}
            \input{6 - summary/main.tex}
    
    
    %\section{}
    \newpage
    \pagenumbering{gobble}
        \printbibliography


    \newpage
    \pagenumbering{roman}
    \appendix
        \part{Appendices}
            \input{8 - Hilbert complexes/main.tex}
            \input{9 - weak conservation proofs/main.tex}
\end{document}

            \documentclass[12pt, a4paper]{report}

\input{template/main.tex}

\title{\BA{Title in Progress...}}
\author{Boris Andrews}
\affil{Mathematical Institute, University of Oxford}
\date{\today}


\begin{document}
    \pagenumbering{gobble}
    \maketitle
    
    
    \begin{abstract}
        Magnetic confinement reactors---in particular tokamaks---offer one of the most promising options for achieving practical nuclear fusion, with the potential to provide virtually limitless, clean energy. The theoretical and numerical modeling of tokamak plasmas is simultaneously an essential component of effective reactor design, and a great research barrier. Tokamak operational conditions exhibit comparatively low Knudsen numbers. Kinetic effects, including kinetic waves and instabilities, Landau damping, bump-on-tail instabilities and more, are therefore highly influential in tokamak plasma dynamics. Purely fluid models are inherently incapable of capturing these effects, whereas the high dimensionality in purely kinetic models render them practically intractable for most relevant purposes.

        We consider a $\delta\!f$ decomposition model, with a macroscopic fluid background and microscopic kinetic correction, both fully coupled to each other. A similar manner of discretization is proposed to that used in the recent \texttt{STRUPHY} code \cite{Holderied_Possanner_Wang_2021, Holderied_2022, Li_et_al_2023} with a finite-element model for the background and a pseudo-particle/PiC model for the correction.

        The fluid background satisfies the full, non-linear, resistive, compressible, Hall MHD equations. \cite{Laakmann_Hu_Farrell_2022} introduces finite-element(-in-space) implicit timesteppers for the incompressible analogue to this system with structure-preserving (SP) properties in the ideal case, alongside parameter-robust preconditioners. We show that these timesteppers can derive from a finite-element-in-time (FET) (and finite-element-in-space) interpretation. The benefits of this reformulation are discussed, including the derivation of timesteppers that are higher order in time, and the quantifiable dissipative SP properties in the non-ideal, resistive case.
        
        We discuss possible options for extending this FET approach to timesteppers for the compressible case.

        The kinetic corrections satisfy linearized Boltzmann equations. Using a Lénard--Bernstein collision operator, these take Fokker--Planck-like forms \cite{Fokker_1914, Planck_1917} wherein pseudo-particles in the numerical model obey the neoclassical transport equations, with particle-independent Brownian drift terms. This offers a rigorous methodology for incorporating collisions into the particle transport model, without coupling the equations of motions for each particle.
        
        Works by Chen, Chacón et al. \cite{Chen_Chacón_Barnes_2011, Chacón_Chen_Barnes_2013, Chen_Chacón_2014, Chen_Chacón_2015} have developed structure-preserving particle pushers for neoclassical transport in the Vlasov equations, derived from Crank--Nicolson integrators. We show these too can can derive from a FET interpretation, similarly offering potential extensions to higher-order-in-time particle pushers. The FET formulation is used also to consider how the stochastic drift terms can be incorporated into the pushers. Stochastic gyrokinetic expansions are also discussed.

        Different options for the numerical implementation of these schemes are considered.

        Due to the efficacy of FET in the development of SP timesteppers for both the fluid and kinetic component, we hope this approach will prove effective in the future for developing SP timesteppers for the full hybrid model. We hope this will give us the opportunity to incorporate previously inaccessible kinetic effects into the highly effective, modern, finite-element MHD models.
    \end{abstract}
    
    
    \newpage
    \tableofcontents
    
    
    \newpage
    \pagenumbering{arabic}
    %\linenumbers\renewcommand\thelinenumber{\color{black!50}\arabic{linenumber}}
            \input{0 - introduction/main.tex}
        \part{Research}
            \input{1 - low-noise PiC models/main.tex}
            \input{2 - kinetic component/main.tex}
            \input{3 - fluid component/main.tex}
            \input{4 - numerical implementation/main.tex}
        \part{Project Overview}
            \input{5 - research plan/main.tex}
            \input{6 - summary/main.tex}
    
    
    %\section{}
    \newpage
    \pagenumbering{gobble}
        \printbibliography


    \newpage
    \pagenumbering{roman}
    \appendix
        \part{Appendices}
            \input{8 - Hilbert complexes/main.tex}
            \input{9 - weak conservation proofs/main.tex}
\end{document}

            \documentclass[12pt, a4paper]{report}

\input{template/main.tex}

\title{\BA{Title in Progress...}}
\author{Boris Andrews}
\affil{Mathematical Institute, University of Oxford}
\date{\today}


\begin{document}
    \pagenumbering{gobble}
    \maketitle
    
    
    \begin{abstract}
        Magnetic confinement reactors---in particular tokamaks---offer one of the most promising options for achieving practical nuclear fusion, with the potential to provide virtually limitless, clean energy. The theoretical and numerical modeling of tokamak plasmas is simultaneously an essential component of effective reactor design, and a great research barrier. Tokamak operational conditions exhibit comparatively low Knudsen numbers. Kinetic effects, including kinetic waves and instabilities, Landau damping, bump-on-tail instabilities and more, are therefore highly influential in tokamak plasma dynamics. Purely fluid models are inherently incapable of capturing these effects, whereas the high dimensionality in purely kinetic models render them practically intractable for most relevant purposes.

        We consider a $\delta\!f$ decomposition model, with a macroscopic fluid background and microscopic kinetic correction, both fully coupled to each other. A similar manner of discretization is proposed to that used in the recent \texttt{STRUPHY} code \cite{Holderied_Possanner_Wang_2021, Holderied_2022, Li_et_al_2023} with a finite-element model for the background and a pseudo-particle/PiC model for the correction.

        The fluid background satisfies the full, non-linear, resistive, compressible, Hall MHD equations. \cite{Laakmann_Hu_Farrell_2022} introduces finite-element(-in-space) implicit timesteppers for the incompressible analogue to this system with structure-preserving (SP) properties in the ideal case, alongside parameter-robust preconditioners. We show that these timesteppers can derive from a finite-element-in-time (FET) (and finite-element-in-space) interpretation. The benefits of this reformulation are discussed, including the derivation of timesteppers that are higher order in time, and the quantifiable dissipative SP properties in the non-ideal, resistive case.
        
        We discuss possible options for extending this FET approach to timesteppers for the compressible case.

        The kinetic corrections satisfy linearized Boltzmann equations. Using a Lénard--Bernstein collision operator, these take Fokker--Planck-like forms \cite{Fokker_1914, Planck_1917} wherein pseudo-particles in the numerical model obey the neoclassical transport equations, with particle-independent Brownian drift terms. This offers a rigorous methodology for incorporating collisions into the particle transport model, without coupling the equations of motions for each particle.
        
        Works by Chen, Chacón et al. \cite{Chen_Chacón_Barnes_2011, Chacón_Chen_Barnes_2013, Chen_Chacón_2014, Chen_Chacón_2015} have developed structure-preserving particle pushers for neoclassical transport in the Vlasov equations, derived from Crank--Nicolson integrators. We show these too can can derive from a FET interpretation, similarly offering potential extensions to higher-order-in-time particle pushers. The FET formulation is used also to consider how the stochastic drift terms can be incorporated into the pushers. Stochastic gyrokinetic expansions are also discussed.

        Different options for the numerical implementation of these schemes are considered.

        Due to the efficacy of FET in the development of SP timesteppers for both the fluid and kinetic component, we hope this approach will prove effective in the future for developing SP timesteppers for the full hybrid model. We hope this will give us the opportunity to incorporate previously inaccessible kinetic effects into the highly effective, modern, finite-element MHD models.
    \end{abstract}
    
    
    \newpage
    \tableofcontents
    
    
    \newpage
    \pagenumbering{arabic}
    %\linenumbers\renewcommand\thelinenumber{\color{black!50}\arabic{linenumber}}
            \input{0 - introduction/main.tex}
        \part{Research}
            \input{1 - low-noise PiC models/main.tex}
            \input{2 - kinetic component/main.tex}
            \input{3 - fluid component/main.tex}
            \input{4 - numerical implementation/main.tex}
        \part{Project Overview}
            \input{5 - research plan/main.tex}
            \input{6 - summary/main.tex}
    
    
    %\section{}
    \newpage
    \pagenumbering{gobble}
        \printbibliography


    \newpage
    \pagenumbering{roman}
    \appendix
        \part{Appendices}
            \input{8 - Hilbert complexes/main.tex}
            \input{9 - weak conservation proofs/main.tex}
\end{document}

            \documentclass[12pt, a4paper]{report}

\input{template/main.tex}

\title{\BA{Title in Progress...}}
\author{Boris Andrews}
\affil{Mathematical Institute, University of Oxford}
\date{\today}


\begin{document}
    \pagenumbering{gobble}
    \maketitle
    
    
    \begin{abstract}
        Magnetic confinement reactors---in particular tokamaks---offer one of the most promising options for achieving practical nuclear fusion, with the potential to provide virtually limitless, clean energy. The theoretical and numerical modeling of tokamak plasmas is simultaneously an essential component of effective reactor design, and a great research barrier. Tokamak operational conditions exhibit comparatively low Knudsen numbers. Kinetic effects, including kinetic waves and instabilities, Landau damping, bump-on-tail instabilities and more, are therefore highly influential in tokamak plasma dynamics. Purely fluid models are inherently incapable of capturing these effects, whereas the high dimensionality in purely kinetic models render them practically intractable for most relevant purposes.

        We consider a $\delta\!f$ decomposition model, with a macroscopic fluid background and microscopic kinetic correction, both fully coupled to each other. A similar manner of discretization is proposed to that used in the recent \texttt{STRUPHY} code \cite{Holderied_Possanner_Wang_2021, Holderied_2022, Li_et_al_2023} with a finite-element model for the background and a pseudo-particle/PiC model for the correction.

        The fluid background satisfies the full, non-linear, resistive, compressible, Hall MHD equations. \cite{Laakmann_Hu_Farrell_2022} introduces finite-element(-in-space) implicit timesteppers for the incompressible analogue to this system with structure-preserving (SP) properties in the ideal case, alongside parameter-robust preconditioners. We show that these timesteppers can derive from a finite-element-in-time (FET) (and finite-element-in-space) interpretation. The benefits of this reformulation are discussed, including the derivation of timesteppers that are higher order in time, and the quantifiable dissipative SP properties in the non-ideal, resistive case.
        
        We discuss possible options for extending this FET approach to timesteppers for the compressible case.

        The kinetic corrections satisfy linearized Boltzmann equations. Using a Lénard--Bernstein collision operator, these take Fokker--Planck-like forms \cite{Fokker_1914, Planck_1917} wherein pseudo-particles in the numerical model obey the neoclassical transport equations, with particle-independent Brownian drift terms. This offers a rigorous methodology for incorporating collisions into the particle transport model, without coupling the equations of motions for each particle.
        
        Works by Chen, Chacón et al. \cite{Chen_Chacón_Barnes_2011, Chacón_Chen_Barnes_2013, Chen_Chacón_2014, Chen_Chacón_2015} have developed structure-preserving particle pushers for neoclassical transport in the Vlasov equations, derived from Crank--Nicolson integrators. We show these too can can derive from a FET interpretation, similarly offering potential extensions to higher-order-in-time particle pushers. The FET formulation is used also to consider how the stochastic drift terms can be incorporated into the pushers. Stochastic gyrokinetic expansions are also discussed.

        Different options for the numerical implementation of these schemes are considered.

        Due to the efficacy of FET in the development of SP timesteppers for both the fluid and kinetic component, we hope this approach will prove effective in the future for developing SP timesteppers for the full hybrid model. We hope this will give us the opportunity to incorporate previously inaccessible kinetic effects into the highly effective, modern, finite-element MHD models.
    \end{abstract}
    
    
    \newpage
    \tableofcontents
    
    
    \newpage
    \pagenumbering{arabic}
    %\linenumbers\renewcommand\thelinenumber{\color{black!50}\arabic{linenumber}}
            \input{0 - introduction/main.tex}
        \part{Research}
            \input{1 - low-noise PiC models/main.tex}
            \input{2 - kinetic component/main.tex}
            \input{3 - fluid component/main.tex}
            \input{4 - numerical implementation/main.tex}
        \part{Project Overview}
            \input{5 - research plan/main.tex}
            \input{6 - summary/main.tex}
    
    
    %\section{}
    \newpage
    \pagenumbering{gobble}
        \printbibliography


    \newpage
    \pagenumbering{roman}
    \appendix
        \part{Appendices}
            \input{8 - Hilbert complexes/main.tex}
            \input{9 - weak conservation proofs/main.tex}
\end{document}

        \part{Project Overview}
            \documentclass[12pt, a4paper]{report}

\input{template/main.tex}

\title{\BA{Title in Progress...}}
\author{Boris Andrews}
\affil{Mathematical Institute, University of Oxford}
\date{\today}


\begin{document}
    \pagenumbering{gobble}
    \maketitle
    
    
    \begin{abstract}
        Magnetic confinement reactors---in particular tokamaks---offer one of the most promising options for achieving practical nuclear fusion, with the potential to provide virtually limitless, clean energy. The theoretical and numerical modeling of tokamak plasmas is simultaneously an essential component of effective reactor design, and a great research barrier. Tokamak operational conditions exhibit comparatively low Knudsen numbers. Kinetic effects, including kinetic waves and instabilities, Landau damping, bump-on-tail instabilities and more, are therefore highly influential in tokamak plasma dynamics. Purely fluid models are inherently incapable of capturing these effects, whereas the high dimensionality in purely kinetic models render them practically intractable for most relevant purposes.

        We consider a $\delta\!f$ decomposition model, with a macroscopic fluid background and microscopic kinetic correction, both fully coupled to each other. A similar manner of discretization is proposed to that used in the recent \texttt{STRUPHY} code \cite{Holderied_Possanner_Wang_2021, Holderied_2022, Li_et_al_2023} with a finite-element model for the background and a pseudo-particle/PiC model for the correction.

        The fluid background satisfies the full, non-linear, resistive, compressible, Hall MHD equations. \cite{Laakmann_Hu_Farrell_2022} introduces finite-element(-in-space) implicit timesteppers for the incompressible analogue to this system with structure-preserving (SP) properties in the ideal case, alongside parameter-robust preconditioners. We show that these timesteppers can derive from a finite-element-in-time (FET) (and finite-element-in-space) interpretation. The benefits of this reformulation are discussed, including the derivation of timesteppers that are higher order in time, and the quantifiable dissipative SP properties in the non-ideal, resistive case.
        
        We discuss possible options for extending this FET approach to timesteppers for the compressible case.

        The kinetic corrections satisfy linearized Boltzmann equations. Using a Lénard--Bernstein collision operator, these take Fokker--Planck-like forms \cite{Fokker_1914, Planck_1917} wherein pseudo-particles in the numerical model obey the neoclassical transport equations, with particle-independent Brownian drift terms. This offers a rigorous methodology for incorporating collisions into the particle transport model, without coupling the equations of motions for each particle.
        
        Works by Chen, Chacón et al. \cite{Chen_Chacón_Barnes_2011, Chacón_Chen_Barnes_2013, Chen_Chacón_2014, Chen_Chacón_2015} have developed structure-preserving particle pushers for neoclassical transport in the Vlasov equations, derived from Crank--Nicolson integrators. We show these too can can derive from a FET interpretation, similarly offering potential extensions to higher-order-in-time particle pushers. The FET formulation is used also to consider how the stochastic drift terms can be incorporated into the pushers. Stochastic gyrokinetic expansions are also discussed.

        Different options for the numerical implementation of these schemes are considered.

        Due to the efficacy of FET in the development of SP timesteppers for both the fluid and kinetic component, we hope this approach will prove effective in the future for developing SP timesteppers for the full hybrid model. We hope this will give us the opportunity to incorporate previously inaccessible kinetic effects into the highly effective, modern, finite-element MHD models.
    \end{abstract}
    
    
    \newpage
    \tableofcontents
    
    
    \newpage
    \pagenumbering{arabic}
    %\linenumbers\renewcommand\thelinenumber{\color{black!50}\arabic{linenumber}}
            \input{0 - introduction/main.tex}
        \part{Research}
            \input{1 - low-noise PiC models/main.tex}
            \input{2 - kinetic component/main.tex}
            \input{3 - fluid component/main.tex}
            \input{4 - numerical implementation/main.tex}
        \part{Project Overview}
            \input{5 - research plan/main.tex}
            \input{6 - summary/main.tex}
    
    
    %\section{}
    \newpage
    \pagenumbering{gobble}
        \printbibliography


    \newpage
    \pagenumbering{roman}
    \appendix
        \part{Appendices}
            \input{8 - Hilbert complexes/main.tex}
            \input{9 - weak conservation proofs/main.tex}
\end{document}

            \documentclass[12pt, a4paper]{report}

\input{template/main.tex}

\title{\BA{Title in Progress...}}
\author{Boris Andrews}
\affil{Mathematical Institute, University of Oxford}
\date{\today}


\begin{document}
    \pagenumbering{gobble}
    \maketitle
    
    
    \begin{abstract}
        Magnetic confinement reactors---in particular tokamaks---offer one of the most promising options for achieving practical nuclear fusion, with the potential to provide virtually limitless, clean energy. The theoretical and numerical modeling of tokamak plasmas is simultaneously an essential component of effective reactor design, and a great research barrier. Tokamak operational conditions exhibit comparatively low Knudsen numbers. Kinetic effects, including kinetic waves and instabilities, Landau damping, bump-on-tail instabilities and more, are therefore highly influential in tokamak plasma dynamics. Purely fluid models are inherently incapable of capturing these effects, whereas the high dimensionality in purely kinetic models render them practically intractable for most relevant purposes.

        We consider a $\delta\!f$ decomposition model, with a macroscopic fluid background and microscopic kinetic correction, both fully coupled to each other. A similar manner of discretization is proposed to that used in the recent \texttt{STRUPHY} code \cite{Holderied_Possanner_Wang_2021, Holderied_2022, Li_et_al_2023} with a finite-element model for the background and a pseudo-particle/PiC model for the correction.

        The fluid background satisfies the full, non-linear, resistive, compressible, Hall MHD equations. \cite{Laakmann_Hu_Farrell_2022} introduces finite-element(-in-space) implicit timesteppers for the incompressible analogue to this system with structure-preserving (SP) properties in the ideal case, alongside parameter-robust preconditioners. We show that these timesteppers can derive from a finite-element-in-time (FET) (and finite-element-in-space) interpretation. The benefits of this reformulation are discussed, including the derivation of timesteppers that are higher order in time, and the quantifiable dissipative SP properties in the non-ideal, resistive case.
        
        We discuss possible options for extending this FET approach to timesteppers for the compressible case.

        The kinetic corrections satisfy linearized Boltzmann equations. Using a Lénard--Bernstein collision operator, these take Fokker--Planck-like forms \cite{Fokker_1914, Planck_1917} wherein pseudo-particles in the numerical model obey the neoclassical transport equations, with particle-independent Brownian drift terms. This offers a rigorous methodology for incorporating collisions into the particle transport model, without coupling the equations of motions for each particle.
        
        Works by Chen, Chacón et al. \cite{Chen_Chacón_Barnes_2011, Chacón_Chen_Barnes_2013, Chen_Chacón_2014, Chen_Chacón_2015} have developed structure-preserving particle pushers for neoclassical transport in the Vlasov equations, derived from Crank--Nicolson integrators. We show these too can can derive from a FET interpretation, similarly offering potential extensions to higher-order-in-time particle pushers. The FET formulation is used also to consider how the stochastic drift terms can be incorporated into the pushers. Stochastic gyrokinetic expansions are also discussed.

        Different options for the numerical implementation of these schemes are considered.

        Due to the efficacy of FET in the development of SP timesteppers for both the fluid and kinetic component, we hope this approach will prove effective in the future for developing SP timesteppers for the full hybrid model. We hope this will give us the opportunity to incorporate previously inaccessible kinetic effects into the highly effective, modern, finite-element MHD models.
    \end{abstract}
    
    
    \newpage
    \tableofcontents
    
    
    \newpage
    \pagenumbering{arabic}
    %\linenumbers\renewcommand\thelinenumber{\color{black!50}\arabic{linenumber}}
            \input{0 - introduction/main.tex}
        \part{Research}
            \input{1 - low-noise PiC models/main.tex}
            \input{2 - kinetic component/main.tex}
            \input{3 - fluid component/main.tex}
            \input{4 - numerical implementation/main.tex}
        \part{Project Overview}
            \input{5 - research plan/main.tex}
            \input{6 - summary/main.tex}
    
    
    %\section{}
    \newpage
    \pagenumbering{gobble}
        \printbibliography


    \newpage
    \pagenumbering{roman}
    \appendix
        \part{Appendices}
            \input{8 - Hilbert complexes/main.tex}
            \input{9 - weak conservation proofs/main.tex}
\end{document}

    
    
    %\section{}
    \newpage
    \pagenumbering{gobble}
        \printbibliography


    \newpage
    \pagenumbering{roman}
    \appendix
        \part{Appendices}
            \documentclass[12pt, a4paper]{report}

\input{template/main.tex}

\title{\BA{Title in Progress...}}
\author{Boris Andrews}
\affil{Mathematical Institute, University of Oxford}
\date{\today}


\begin{document}
    \pagenumbering{gobble}
    \maketitle
    
    
    \begin{abstract}
        Magnetic confinement reactors---in particular tokamaks---offer one of the most promising options for achieving practical nuclear fusion, with the potential to provide virtually limitless, clean energy. The theoretical and numerical modeling of tokamak plasmas is simultaneously an essential component of effective reactor design, and a great research barrier. Tokamak operational conditions exhibit comparatively low Knudsen numbers. Kinetic effects, including kinetic waves and instabilities, Landau damping, bump-on-tail instabilities and more, are therefore highly influential in tokamak plasma dynamics. Purely fluid models are inherently incapable of capturing these effects, whereas the high dimensionality in purely kinetic models render them practically intractable for most relevant purposes.

        We consider a $\delta\!f$ decomposition model, with a macroscopic fluid background and microscopic kinetic correction, both fully coupled to each other. A similar manner of discretization is proposed to that used in the recent \texttt{STRUPHY} code \cite{Holderied_Possanner_Wang_2021, Holderied_2022, Li_et_al_2023} with a finite-element model for the background and a pseudo-particle/PiC model for the correction.

        The fluid background satisfies the full, non-linear, resistive, compressible, Hall MHD equations. \cite{Laakmann_Hu_Farrell_2022} introduces finite-element(-in-space) implicit timesteppers for the incompressible analogue to this system with structure-preserving (SP) properties in the ideal case, alongside parameter-robust preconditioners. We show that these timesteppers can derive from a finite-element-in-time (FET) (and finite-element-in-space) interpretation. The benefits of this reformulation are discussed, including the derivation of timesteppers that are higher order in time, and the quantifiable dissipative SP properties in the non-ideal, resistive case.
        
        We discuss possible options for extending this FET approach to timesteppers for the compressible case.

        The kinetic corrections satisfy linearized Boltzmann equations. Using a Lénard--Bernstein collision operator, these take Fokker--Planck-like forms \cite{Fokker_1914, Planck_1917} wherein pseudo-particles in the numerical model obey the neoclassical transport equations, with particle-independent Brownian drift terms. This offers a rigorous methodology for incorporating collisions into the particle transport model, without coupling the equations of motions for each particle.
        
        Works by Chen, Chacón et al. \cite{Chen_Chacón_Barnes_2011, Chacón_Chen_Barnes_2013, Chen_Chacón_2014, Chen_Chacón_2015} have developed structure-preserving particle pushers for neoclassical transport in the Vlasov equations, derived from Crank--Nicolson integrators. We show these too can can derive from a FET interpretation, similarly offering potential extensions to higher-order-in-time particle pushers. The FET formulation is used also to consider how the stochastic drift terms can be incorporated into the pushers. Stochastic gyrokinetic expansions are also discussed.

        Different options for the numerical implementation of these schemes are considered.

        Due to the efficacy of FET in the development of SP timesteppers for both the fluid and kinetic component, we hope this approach will prove effective in the future for developing SP timesteppers for the full hybrid model. We hope this will give us the opportunity to incorporate previously inaccessible kinetic effects into the highly effective, modern, finite-element MHD models.
    \end{abstract}
    
    
    \newpage
    \tableofcontents
    
    
    \newpage
    \pagenumbering{arabic}
    %\linenumbers\renewcommand\thelinenumber{\color{black!50}\arabic{linenumber}}
            \input{0 - introduction/main.tex}
        \part{Research}
            \input{1 - low-noise PiC models/main.tex}
            \input{2 - kinetic component/main.tex}
            \input{3 - fluid component/main.tex}
            \input{4 - numerical implementation/main.tex}
        \part{Project Overview}
            \input{5 - research plan/main.tex}
            \input{6 - summary/main.tex}
    
    
    %\section{}
    \newpage
    \pagenumbering{gobble}
        \printbibliography


    \newpage
    \pagenumbering{roman}
    \appendix
        \part{Appendices}
            \input{8 - Hilbert complexes/main.tex}
            \input{9 - weak conservation proofs/main.tex}
\end{document}

            \documentclass[12pt, a4paper]{report}

\input{template/main.tex}

\title{\BA{Title in Progress...}}
\author{Boris Andrews}
\affil{Mathematical Institute, University of Oxford}
\date{\today}


\begin{document}
    \pagenumbering{gobble}
    \maketitle
    
    
    \begin{abstract}
        Magnetic confinement reactors---in particular tokamaks---offer one of the most promising options for achieving practical nuclear fusion, with the potential to provide virtually limitless, clean energy. The theoretical and numerical modeling of tokamak plasmas is simultaneously an essential component of effective reactor design, and a great research barrier. Tokamak operational conditions exhibit comparatively low Knudsen numbers. Kinetic effects, including kinetic waves and instabilities, Landau damping, bump-on-tail instabilities and more, are therefore highly influential in tokamak plasma dynamics. Purely fluid models are inherently incapable of capturing these effects, whereas the high dimensionality in purely kinetic models render them practically intractable for most relevant purposes.

        We consider a $\delta\!f$ decomposition model, with a macroscopic fluid background and microscopic kinetic correction, both fully coupled to each other. A similar manner of discretization is proposed to that used in the recent \texttt{STRUPHY} code \cite{Holderied_Possanner_Wang_2021, Holderied_2022, Li_et_al_2023} with a finite-element model for the background and a pseudo-particle/PiC model for the correction.

        The fluid background satisfies the full, non-linear, resistive, compressible, Hall MHD equations. \cite{Laakmann_Hu_Farrell_2022} introduces finite-element(-in-space) implicit timesteppers for the incompressible analogue to this system with structure-preserving (SP) properties in the ideal case, alongside parameter-robust preconditioners. We show that these timesteppers can derive from a finite-element-in-time (FET) (and finite-element-in-space) interpretation. The benefits of this reformulation are discussed, including the derivation of timesteppers that are higher order in time, and the quantifiable dissipative SP properties in the non-ideal, resistive case.
        
        We discuss possible options for extending this FET approach to timesteppers for the compressible case.

        The kinetic corrections satisfy linearized Boltzmann equations. Using a Lénard--Bernstein collision operator, these take Fokker--Planck-like forms \cite{Fokker_1914, Planck_1917} wherein pseudo-particles in the numerical model obey the neoclassical transport equations, with particle-independent Brownian drift terms. This offers a rigorous methodology for incorporating collisions into the particle transport model, without coupling the equations of motions for each particle.
        
        Works by Chen, Chacón et al. \cite{Chen_Chacón_Barnes_2011, Chacón_Chen_Barnes_2013, Chen_Chacón_2014, Chen_Chacón_2015} have developed structure-preserving particle pushers for neoclassical transport in the Vlasov equations, derived from Crank--Nicolson integrators. We show these too can can derive from a FET interpretation, similarly offering potential extensions to higher-order-in-time particle pushers. The FET formulation is used also to consider how the stochastic drift terms can be incorporated into the pushers. Stochastic gyrokinetic expansions are also discussed.

        Different options for the numerical implementation of these schemes are considered.

        Due to the efficacy of FET in the development of SP timesteppers for both the fluid and kinetic component, we hope this approach will prove effective in the future for developing SP timesteppers for the full hybrid model. We hope this will give us the opportunity to incorporate previously inaccessible kinetic effects into the highly effective, modern, finite-element MHD models.
    \end{abstract}
    
    
    \newpage
    \tableofcontents
    
    
    \newpage
    \pagenumbering{arabic}
    %\linenumbers\renewcommand\thelinenumber{\color{black!50}\arabic{linenumber}}
            \input{0 - introduction/main.tex}
        \part{Research}
            \input{1 - low-noise PiC models/main.tex}
            \input{2 - kinetic component/main.tex}
            \input{3 - fluid component/main.tex}
            \input{4 - numerical implementation/main.tex}
        \part{Project Overview}
            \input{5 - research plan/main.tex}
            \input{6 - summary/main.tex}
    
    
    %\section{}
    \newpage
    \pagenumbering{gobble}
        \printbibliography


    \newpage
    \pagenumbering{roman}
    \appendix
        \part{Appendices}
            \input{8 - Hilbert complexes/main.tex}
            \input{9 - weak conservation proofs/main.tex}
\end{document}

\end{document}

            \documentclass[12pt, a4paper]{report}

\documentclass[12pt, a4paper]{report}

\input{template/main.tex}

\title{\BA{Title in Progress...}}
\author{Boris Andrews}
\affil{Mathematical Institute, University of Oxford}
\date{\today}


\begin{document}
    \pagenumbering{gobble}
    \maketitle
    
    
    \begin{abstract}
        Magnetic confinement reactors---in particular tokamaks---offer one of the most promising options for achieving practical nuclear fusion, with the potential to provide virtually limitless, clean energy. The theoretical and numerical modeling of tokamak plasmas is simultaneously an essential component of effective reactor design, and a great research barrier. Tokamak operational conditions exhibit comparatively low Knudsen numbers. Kinetic effects, including kinetic waves and instabilities, Landau damping, bump-on-tail instabilities and more, are therefore highly influential in tokamak plasma dynamics. Purely fluid models are inherently incapable of capturing these effects, whereas the high dimensionality in purely kinetic models render them practically intractable for most relevant purposes.

        We consider a $\delta\!f$ decomposition model, with a macroscopic fluid background and microscopic kinetic correction, both fully coupled to each other. A similar manner of discretization is proposed to that used in the recent \texttt{STRUPHY} code \cite{Holderied_Possanner_Wang_2021, Holderied_2022, Li_et_al_2023} with a finite-element model for the background and a pseudo-particle/PiC model for the correction.

        The fluid background satisfies the full, non-linear, resistive, compressible, Hall MHD equations. \cite{Laakmann_Hu_Farrell_2022} introduces finite-element(-in-space) implicit timesteppers for the incompressible analogue to this system with structure-preserving (SP) properties in the ideal case, alongside parameter-robust preconditioners. We show that these timesteppers can derive from a finite-element-in-time (FET) (and finite-element-in-space) interpretation. The benefits of this reformulation are discussed, including the derivation of timesteppers that are higher order in time, and the quantifiable dissipative SP properties in the non-ideal, resistive case.
        
        We discuss possible options for extending this FET approach to timesteppers for the compressible case.

        The kinetic corrections satisfy linearized Boltzmann equations. Using a Lénard--Bernstein collision operator, these take Fokker--Planck-like forms \cite{Fokker_1914, Planck_1917} wherein pseudo-particles in the numerical model obey the neoclassical transport equations, with particle-independent Brownian drift terms. This offers a rigorous methodology for incorporating collisions into the particle transport model, without coupling the equations of motions for each particle.
        
        Works by Chen, Chacón et al. \cite{Chen_Chacón_Barnes_2011, Chacón_Chen_Barnes_2013, Chen_Chacón_2014, Chen_Chacón_2015} have developed structure-preserving particle pushers for neoclassical transport in the Vlasov equations, derived from Crank--Nicolson integrators. We show these too can can derive from a FET interpretation, similarly offering potential extensions to higher-order-in-time particle pushers. The FET formulation is used also to consider how the stochastic drift terms can be incorporated into the pushers. Stochastic gyrokinetic expansions are also discussed.

        Different options for the numerical implementation of these schemes are considered.

        Due to the efficacy of FET in the development of SP timesteppers for both the fluid and kinetic component, we hope this approach will prove effective in the future for developing SP timesteppers for the full hybrid model. We hope this will give us the opportunity to incorporate previously inaccessible kinetic effects into the highly effective, modern, finite-element MHD models.
    \end{abstract}
    
    
    \newpage
    \tableofcontents
    
    
    \newpage
    \pagenumbering{arabic}
    %\linenumbers\renewcommand\thelinenumber{\color{black!50}\arabic{linenumber}}
            \input{0 - introduction/main.tex}
        \part{Research}
            \input{1 - low-noise PiC models/main.tex}
            \input{2 - kinetic component/main.tex}
            \input{3 - fluid component/main.tex}
            \input{4 - numerical implementation/main.tex}
        \part{Project Overview}
            \input{5 - research plan/main.tex}
            \input{6 - summary/main.tex}
    
    
    %\section{}
    \newpage
    \pagenumbering{gobble}
        \printbibliography


    \newpage
    \pagenumbering{roman}
    \appendix
        \part{Appendices}
            \input{8 - Hilbert complexes/main.tex}
            \input{9 - weak conservation proofs/main.tex}
\end{document}


\title{\BA{Title in Progress...}}
\author{Boris Andrews}
\affil{Mathematical Institute, University of Oxford}
\date{\today}


\begin{document}
    \pagenumbering{gobble}
    \maketitle
    
    
    \begin{abstract}
        Magnetic confinement reactors---in particular tokamaks---offer one of the most promising options for achieving practical nuclear fusion, with the potential to provide virtually limitless, clean energy. The theoretical and numerical modeling of tokamak plasmas is simultaneously an essential component of effective reactor design, and a great research barrier. Tokamak operational conditions exhibit comparatively low Knudsen numbers. Kinetic effects, including kinetic waves and instabilities, Landau damping, bump-on-tail instabilities and more, are therefore highly influential in tokamak plasma dynamics. Purely fluid models are inherently incapable of capturing these effects, whereas the high dimensionality in purely kinetic models render them practically intractable for most relevant purposes.

        We consider a $\delta\!f$ decomposition model, with a macroscopic fluid background and microscopic kinetic correction, both fully coupled to each other. A similar manner of discretization is proposed to that used in the recent \texttt{STRUPHY} code \cite{Holderied_Possanner_Wang_2021, Holderied_2022, Li_et_al_2023} with a finite-element model for the background and a pseudo-particle/PiC model for the correction.

        The fluid background satisfies the full, non-linear, resistive, compressible, Hall MHD equations. \cite{Laakmann_Hu_Farrell_2022} introduces finite-element(-in-space) implicit timesteppers for the incompressible analogue to this system with structure-preserving (SP) properties in the ideal case, alongside parameter-robust preconditioners. We show that these timesteppers can derive from a finite-element-in-time (FET) (and finite-element-in-space) interpretation. The benefits of this reformulation are discussed, including the derivation of timesteppers that are higher order in time, and the quantifiable dissipative SP properties in the non-ideal, resistive case.
        
        We discuss possible options for extending this FET approach to timesteppers for the compressible case.

        The kinetic corrections satisfy linearized Boltzmann equations. Using a Lénard--Bernstein collision operator, these take Fokker--Planck-like forms \cite{Fokker_1914, Planck_1917} wherein pseudo-particles in the numerical model obey the neoclassical transport equations, with particle-independent Brownian drift terms. This offers a rigorous methodology for incorporating collisions into the particle transport model, without coupling the equations of motions for each particle.
        
        Works by Chen, Chacón et al. \cite{Chen_Chacón_Barnes_2011, Chacón_Chen_Barnes_2013, Chen_Chacón_2014, Chen_Chacón_2015} have developed structure-preserving particle pushers for neoclassical transport in the Vlasov equations, derived from Crank--Nicolson integrators. We show these too can can derive from a FET interpretation, similarly offering potential extensions to higher-order-in-time particle pushers. The FET formulation is used also to consider how the stochastic drift terms can be incorporated into the pushers. Stochastic gyrokinetic expansions are also discussed.

        Different options for the numerical implementation of these schemes are considered.

        Due to the efficacy of FET in the development of SP timesteppers for both the fluid and kinetic component, we hope this approach will prove effective in the future for developing SP timesteppers for the full hybrid model. We hope this will give us the opportunity to incorporate previously inaccessible kinetic effects into the highly effective, modern, finite-element MHD models.
    \end{abstract}
    
    
    \newpage
    \tableofcontents
    
    
    \newpage
    \pagenumbering{arabic}
    %\linenumbers\renewcommand\thelinenumber{\color{black!50}\arabic{linenumber}}
            \documentclass[12pt, a4paper]{report}

\input{template/main.tex}

\title{\BA{Title in Progress...}}
\author{Boris Andrews}
\affil{Mathematical Institute, University of Oxford}
\date{\today}


\begin{document}
    \pagenumbering{gobble}
    \maketitle
    
    
    \begin{abstract}
        Magnetic confinement reactors---in particular tokamaks---offer one of the most promising options for achieving practical nuclear fusion, with the potential to provide virtually limitless, clean energy. The theoretical and numerical modeling of tokamak plasmas is simultaneously an essential component of effective reactor design, and a great research barrier. Tokamak operational conditions exhibit comparatively low Knudsen numbers. Kinetic effects, including kinetic waves and instabilities, Landau damping, bump-on-tail instabilities and more, are therefore highly influential in tokamak plasma dynamics. Purely fluid models are inherently incapable of capturing these effects, whereas the high dimensionality in purely kinetic models render them practically intractable for most relevant purposes.

        We consider a $\delta\!f$ decomposition model, with a macroscopic fluid background and microscopic kinetic correction, both fully coupled to each other. A similar manner of discretization is proposed to that used in the recent \texttt{STRUPHY} code \cite{Holderied_Possanner_Wang_2021, Holderied_2022, Li_et_al_2023} with a finite-element model for the background and a pseudo-particle/PiC model for the correction.

        The fluid background satisfies the full, non-linear, resistive, compressible, Hall MHD equations. \cite{Laakmann_Hu_Farrell_2022} introduces finite-element(-in-space) implicit timesteppers for the incompressible analogue to this system with structure-preserving (SP) properties in the ideal case, alongside parameter-robust preconditioners. We show that these timesteppers can derive from a finite-element-in-time (FET) (and finite-element-in-space) interpretation. The benefits of this reformulation are discussed, including the derivation of timesteppers that are higher order in time, and the quantifiable dissipative SP properties in the non-ideal, resistive case.
        
        We discuss possible options for extending this FET approach to timesteppers for the compressible case.

        The kinetic corrections satisfy linearized Boltzmann equations. Using a Lénard--Bernstein collision operator, these take Fokker--Planck-like forms \cite{Fokker_1914, Planck_1917} wherein pseudo-particles in the numerical model obey the neoclassical transport equations, with particle-independent Brownian drift terms. This offers a rigorous methodology for incorporating collisions into the particle transport model, without coupling the equations of motions for each particle.
        
        Works by Chen, Chacón et al. \cite{Chen_Chacón_Barnes_2011, Chacón_Chen_Barnes_2013, Chen_Chacón_2014, Chen_Chacón_2015} have developed structure-preserving particle pushers for neoclassical transport in the Vlasov equations, derived from Crank--Nicolson integrators. We show these too can can derive from a FET interpretation, similarly offering potential extensions to higher-order-in-time particle pushers. The FET formulation is used also to consider how the stochastic drift terms can be incorporated into the pushers. Stochastic gyrokinetic expansions are also discussed.

        Different options for the numerical implementation of these schemes are considered.

        Due to the efficacy of FET in the development of SP timesteppers for both the fluid and kinetic component, we hope this approach will prove effective in the future for developing SP timesteppers for the full hybrid model. We hope this will give us the opportunity to incorporate previously inaccessible kinetic effects into the highly effective, modern, finite-element MHD models.
    \end{abstract}
    
    
    \newpage
    \tableofcontents
    
    
    \newpage
    \pagenumbering{arabic}
    %\linenumbers\renewcommand\thelinenumber{\color{black!50}\arabic{linenumber}}
            \input{0 - introduction/main.tex}
        \part{Research}
            \input{1 - low-noise PiC models/main.tex}
            \input{2 - kinetic component/main.tex}
            \input{3 - fluid component/main.tex}
            \input{4 - numerical implementation/main.tex}
        \part{Project Overview}
            \input{5 - research plan/main.tex}
            \input{6 - summary/main.tex}
    
    
    %\section{}
    \newpage
    \pagenumbering{gobble}
        \printbibliography


    \newpage
    \pagenumbering{roman}
    \appendix
        \part{Appendices}
            \input{8 - Hilbert complexes/main.tex}
            \input{9 - weak conservation proofs/main.tex}
\end{document}

        \part{Research}
            \documentclass[12pt, a4paper]{report}

\input{template/main.tex}

\title{\BA{Title in Progress...}}
\author{Boris Andrews}
\affil{Mathematical Institute, University of Oxford}
\date{\today}


\begin{document}
    \pagenumbering{gobble}
    \maketitle
    
    
    \begin{abstract}
        Magnetic confinement reactors---in particular tokamaks---offer one of the most promising options for achieving practical nuclear fusion, with the potential to provide virtually limitless, clean energy. The theoretical and numerical modeling of tokamak plasmas is simultaneously an essential component of effective reactor design, and a great research barrier. Tokamak operational conditions exhibit comparatively low Knudsen numbers. Kinetic effects, including kinetic waves and instabilities, Landau damping, bump-on-tail instabilities and more, are therefore highly influential in tokamak plasma dynamics. Purely fluid models are inherently incapable of capturing these effects, whereas the high dimensionality in purely kinetic models render them practically intractable for most relevant purposes.

        We consider a $\delta\!f$ decomposition model, with a macroscopic fluid background and microscopic kinetic correction, both fully coupled to each other. A similar manner of discretization is proposed to that used in the recent \texttt{STRUPHY} code \cite{Holderied_Possanner_Wang_2021, Holderied_2022, Li_et_al_2023} with a finite-element model for the background and a pseudo-particle/PiC model for the correction.

        The fluid background satisfies the full, non-linear, resistive, compressible, Hall MHD equations. \cite{Laakmann_Hu_Farrell_2022} introduces finite-element(-in-space) implicit timesteppers for the incompressible analogue to this system with structure-preserving (SP) properties in the ideal case, alongside parameter-robust preconditioners. We show that these timesteppers can derive from a finite-element-in-time (FET) (and finite-element-in-space) interpretation. The benefits of this reformulation are discussed, including the derivation of timesteppers that are higher order in time, and the quantifiable dissipative SP properties in the non-ideal, resistive case.
        
        We discuss possible options for extending this FET approach to timesteppers for the compressible case.

        The kinetic corrections satisfy linearized Boltzmann equations. Using a Lénard--Bernstein collision operator, these take Fokker--Planck-like forms \cite{Fokker_1914, Planck_1917} wherein pseudo-particles in the numerical model obey the neoclassical transport equations, with particle-independent Brownian drift terms. This offers a rigorous methodology for incorporating collisions into the particle transport model, without coupling the equations of motions for each particle.
        
        Works by Chen, Chacón et al. \cite{Chen_Chacón_Barnes_2011, Chacón_Chen_Barnes_2013, Chen_Chacón_2014, Chen_Chacón_2015} have developed structure-preserving particle pushers for neoclassical transport in the Vlasov equations, derived from Crank--Nicolson integrators. We show these too can can derive from a FET interpretation, similarly offering potential extensions to higher-order-in-time particle pushers. The FET formulation is used also to consider how the stochastic drift terms can be incorporated into the pushers. Stochastic gyrokinetic expansions are also discussed.

        Different options for the numerical implementation of these schemes are considered.

        Due to the efficacy of FET in the development of SP timesteppers for both the fluid and kinetic component, we hope this approach will prove effective in the future for developing SP timesteppers for the full hybrid model. We hope this will give us the opportunity to incorporate previously inaccessible kinetic effects into the highly effective, modern, finite-element MHD models.
    \end{abstract}
    
    
    \newpage
    \tableofcontents
    
    
    \newpage
    \pagenumbering{arabic}
    %\linenumbers\renewcommand\thelinenumber{\color{black!50}\arabic{linenumber}}
            \input{0 - introduction/main.tex}
        \part{Research}
            \input{1 - low-noise PiC models/main.tex}
            \input{2 - kinetic component/main.tex}
            \input{3 - fluid component/main.tex}
            \input{4 - numerical implementation/main.tex}
        \part{Project Overview}
            \input{5 - research plan/main.tex}
            \input{6 - summary/main.tex}
    
    
    %\section{}
    \newpage
    \pagenumbering{gobble}
        \printbibliography


    \newpage
    \pagenumbering{roman}
    \appendix
        \part{Appendices}
            \input{8 - Hilbert complexes/main.tex}
            \input{9 - weak conservation proofs/main.tex}
\end{document}

            \documentclass[12pt, a4paper]{report}

\input{template/main.tex}

\title{\BA{Title in Progress...}}
\author{Boris Andrews}
\affil{Mathematical Institute, University of Oxford}
\date{\today}


\begin{document}
    \pagenumbering{gobble}
    \maketitle
    
    
    \begin{abstract}
        Magnetic confinement reactors---in particular tokamaks---offer one of the most promising options for achieving practical nuclear fusion, with the potential to provide virtually limitless, clean energy. The theoretical and numerical modeling of tokamak plasmas is simultaneously an essential component of effective reactor design, and a great research barrier. Tokamak operational conditions exhibit comparatively low Knudsen numbers. Kinetic effects, including kinetic waves and instabilities, Landau damping, bump-on-tail instabilities and more, are therefore highly influential in tokamak plasma dynamics. Purely fluid models are inherently incapable of capturing these effects, whereas the high dimensionality in purely kinetic models render them practically intractable for most relevant purposes.

        We consider a $\delta\!f$ decomposition model, with a macroscopic fluid background and microscopic kinetic correction, both fully coupled to each other. A similar manner of discretization is proposed to that used in the recent \texttt{STRUPHY} code \cite{Holderied_Possanner_Wang_2021, Holderied_2022, Li_et_al_2023} with a finite-element model for the background and a pseudo-particle/PiC model for the correction.

        The fluid background satisfies the full, non-linear, resistive, compressible, Hall MHD equations. \cite{Laakmann_Hu_Farrell_2022} introduces finite-element(-in-space) implicit timesteppers for the incompressible analogue to this system with structure-preserving (SP) properties in the ideal case, alongside parameter-robust preconditioners. We show that these timesteppers can derive from a finite-element-in-time (FET) (and finite-element-in-space) interpretation. The benefits of this reformulation are discussed, including the derivation of timesteppers that are higher order in time, and the quantifiable dissipative SP properties in the non-ideal, resistive case.
        
        We discuss possible options for extending this FET approach to timesteppers for the compressible case.

        The kinetic corrections satisfy linearized Boltzmann equations. Using a Lénard--Bernstein collision operator, these take Fokker--Planck-like forms \cite{Fokker_1914, Planck_1917} wherein pseudo-particles in the numerical model obey the neoclassical transport equations, with particle-independent Brownian drift terms. This offers a rigorous methodology for incorporating collisions into the particle transport model, without coupling the equations of motions for each particle.
        
        Works by Chen, Chacón et al. \cite{Chen_Chacón_Barnes_2011, Chacón_Chen_Barnes_2013, Chen_Chacón_2014, Chen_Chacón_2015} have developed structure-preserving particle pushers for neoclassical transport in the Vlasov equations, derived from Crank--Nicolson integrators. We show these too can can derive from a FET interpretation, similarly offering potential extensions to higher-order-in-time particle pushers. The FET formulation is used also to consider how the stochastic drift terms can be incorporated into the pushers. Stochastic gyrokinetic expansions are also discussed.

        Different options for the numerical implementation of these schemes are considered.

        Due to the efficacy of FET in the development of SP timesteppers for both the fluid and kinetic component, we hope this approach will prove effective in the future for developing SP timesteppers for the full hybrid model. We hope this will give us the opportunity to incorporate previously inaccessible kinetic effects into the highly effective, modern, finite-element MHD models.
    \end{abstract}
    
    
    \newpage
    \tableofcontents
    
    
    \newpage
    \pagenumbering{arabic}
    %\linenumbers\renewcommand\thelinenumber{\color{black!50}\arabic{linenumber}}
            \input{0 - introduction/main.tex}
        \part{Research}
            \input{1 - low-noise PiC models/main.tex}
            \input{2 - kinetic component/main.tex}
            \input{3 - fluid component/main.tex}
            \input{4 - numerical implementation/main.tex}
        \part{Project Overview}
            \input{5 - research plan/main.tex}
            \input{6 - summary/main.tex}
    
    
    %\section{}
    \newpage
    \pagenumbering{gobble}
        \printbibliography


    \newpage
    \pagenumbering{roman}
    \appendix
        \part{Appendices}
            \input{8 - Hilbert complexes/main.tex}
            \input{9 - weak conservation proofs/main.tex}
\end{document}

            \documentclass[12pt, a4paper]{report}

\input{template/main.tex}

\title{\BA{Title in Progress...}}
\author{Boris Andrews}
\affil{Mathematical Institute, University of Oxford}
\date{\today}


\begin{document}
    \pagenumbering{gobble}
    \maketitle
    
    
    \begin{abstract}
        Magnetic confinement reactors---in particular tokamaks---offer one of the most promising options for achieving practical nuclear fusion, with the potential to provide virtually limitless, clean energy. The theoretical and numerical modeling of tokamak plasmas is simultaneously an essential component of effective reactor design, and a great research barrier. Tokamak operational conditions exhibit comparatively low Knudsen numbers. Kinetic effects, including kinetic waves and instabilities, Landau damping, bump-on-tail instabilities and more, are therefore highly influential in tokamak plasma dynamics. Purely fluid models are inherently incapable of capturing these effects, whereas the high dimensionality in purely kinetic models render them practically intractable for most relevant purposes.

        We consider a $\delta\!f$ decomposition model, with a macroscopic fluid background and microscopic kinetic correction, both fully coupled to each other. A similar manner of discretization is proposed to that used in the recent \texttt{STRUPHY} code \cite{Holderied_Possanner_Wang_2021, Holderied_2022, Li_et_al_2023} with a finite-element model for the background and a pseudo-particle/PiC model for the correction.

        The fluid background satisfies the full, non-linear, resistive, compressible, Hall MHD equations. \cite{Laakmann_Hu_Farrell_2022} introduces finite-element(-in-space) implicit timesteppers for the incompressible analogue to this system with structure-preserving (SP) properties in the ideal case, alongside parameter-robust preconditioners. We show that these timesteppers can derive from a finite-element-in-time (FET) (and finite-element-in-space) interpretation. The benefits of this reformulation are discussed, including the derivation of timesteppers that are higher order in time, and the quantifiable dissipative SP properties in the non-ideal, resistive case.
        
        We discuss possible options for extending this FET approach to timesteppers for the compressible case.

        The kinetic corrections satisfy linearized Boltzmann equations. Using a Lénard--Bernstein collision operator, these take Fokker--Planck-like forms \cite{Fokker_1914, Planck_1917} wherein pseudo-particles in the numerical model obey the neoclassical transport equations, with particle-independent Brownian drift terms. This offers a rigorous methodology for incorporating collisions into the particle transport model, without coupling the equations of motions for each particle.
        
        Works by Chen, Chacón et al. \cite{Chen_Chacón_Barnes_2011, Chacón_Chen_Barnes_2013, Chen_Chacón_2014, Chen_Chacón_2015} have developed structure-preserving particle pushers for neoclassical transport in the Vlasov equations, derived from Crank--Nicolson integrators. We show these too can can derive from a FET interpretation, similarly offering potential extensions to higher-order-in-time particle pushers. The FET formulation is used also to consider how the stochastic drift terms can be incorporated into the pushers. Stochastic gyrokinetic expansions are also discussed.

        Different options for the numerical implementation of these schemes are considered.

        Due to the efficacy of FET in the development of SP timesteppers for both the fluid and kinetic component, we hope this approach will prove effective in the future for developing SP timesteppers for the full hybrid model. We hope this will give us the opportunity to incorporate previously inaccessible kinetic effects into the highly effective, modern, finite-element MHD models.
    \end{abstract}
    
    
    \newpage
    \tableofcontents
    
    
    \newpage
    \pagenumbering{arabic}
    %\linenumbers\renewcommand\thelinenumber{\color{black!50}\arabic{linenumber}}
            \input{0 - introduction/main.tex}
        \part{Research}
            \input{1 - low-noise PiC models/main.tex}
            \input{2 - kinetic component/main.tex}
            \input{3 - fluid component/main.tex}
            \input{4 - numerical implementation/main.tex}
        \part{Project Overview}
            \input{5 - research plan/main.tex}
            \input{6 - summary/main.tex}
    
    
    %\section{}
    \newpage
    \pagenumbering{gobble}
        \printbibliography


    \newpage
    \pagenumbering{roman}
    \appendix
        \part{Appendices}
            \input{8 - Hilbert complexes/main.tex}
            \input{9 - weak conservation proofs/main.tex}
\end{document}

            \documentclass[12pt, a4paper]{report}

\input{template/main.tex}

\title{\BA{Title in Progress...}}
\author{Boris Andrews}
\affil{Mathematical Institute, University of Oxford}
\date{\today}


\begin{document}
    \pagenumbering{gobble}
    \maketitle
    
    
    \begin{abstract}
        Magnetic confinement reactors---in particular tokamaks---offer one of the most promising options for achieving practical nuclear fusion, with the potential to provide virtually limitless, clean energy. The theoretical and numerical modeling of tokamak plasmas is simultaneously an essential component of effective reactor design, and a great research barrier. Tokamak operational conditions exhibit comparatively low Knudsen numbers. Kinetic effects, including kinetic waves and instabilities, Landau damping, bump-on-tail instabilities and more, are therefore highly influential in tokamak plasma dynamics. Purely fluid models are inherently incapable of capturing these effects, whereas the high dimensionality in purely kinetic models render them practically intractable for most relevant purposes.

        We consider a $\delta\!f$ decomposition model, with a macroscopic fluid background and microscopic kinetic correction, both fully coupled to each other. A similar manner of discretization is proposed to that used in the recent \texttt{STRUPHY} code \cite{Holderied_Possanner_Wang_2021, Holderied_2022, Li_et_al_2023} with a finite-element model for the background and a pseudo-particle/PiC model for the correction.

        The fluid background satisfies the full, non-linear, resistive, compressible, Hall MHD equations. \cite{Laakmann_Hu_Farrell_2022} introduces finite-element(-in-space) implicit timesteppers for the incompressible analogue to this system with structure-preserving (SP) properties in the ideal case, alongside parameter-robust preconditioners. We show that these timesteppers can derive from a finite-element-in-time (FET) (and finite-element-in-space) interpretation. The benefits of this reformulation are discussed, including the derivation of timesteppers that are higher order in time, and the quantifiable dissipative SP properties in the non-ideal, resistive case.
        
        We discuss possible options for extending this FET approach to timesteppers for the compressible case.

        The kinetic corrections satisfy linearized Boltzmann equations. Using a Lénard--Bernstein collision operator, these take Fokker--Planck-like forms \cite{Fokker_1914, Planck_1917} wherein pseudo-particles in the numerical model obey the neoclassical transport equations, with particle-independent Brownian drift terms. This offers a rigorous methodology for incorporating collisions into the particle transport model, without coupling the equations of motions for each particle.
        
        Works by Chen, Chacón et al. \cite{Chen_Chacón_Barnes_2011, Chacón_Chen_Barnes_2013, Chen_Chacón_2014, Chen_Chacón_2015} have developed structure-preserving particle pushers for neoclassical transport in the Vlasov equations, derived from Crank--Nicolson integrators. We show these too can can derive from a FET interpretation, similarly offering potential extensions to higher-order-in-time particle pushers. The FET formulation is used also to consider how the stochastic drift terms can be incorporated into the pushers. Stochastic gyrokinetic expansions are also discussed.

        Different options for the numerical implementation of these schemes are considered.

        Due to the efficacy of FET in the development of SP timesteppers for both the fluid and kinetic component, we hope this approach will prove effective in the future for developing SP timesteppers for the full hybrid model. We hope this will give us the opportunity to incorporate previously inaccessible kinetic effects into the highly effective, modern, finite-element MHD models.
    \end{abstract}
    
    
    \newpage
    \tableofcontents
    
    
    \newpage
    \pagenumbering{arabic}
    %\linenumbers\renewcommand\thelinenumber{\color{black!50}\arabic{linenumber}}
            \input{0 - introduction/main.tex}
        \part{Research}
            \input{1 - low-noise PiC models/main.tex}
            \input{2 - kinetic component/main.tex}
            \input{3 - fluid component/main.tex}
            \input{4 - numerical implementation/main.tex}
        \part{Project Overview}
            \input{5 - research plan/main.tex}
            \input{6 - summary/main.tex}
    
    
    %\section{}
    \newpage
    \pagenumbering{gobble}
        \printbibliography


    \newpage
    \pagenumbering{roman}
    \appendix
        \part{Appendices}
            \input{8 - Hilbert complexes/main.tex}
            \input{9 - weak conservation proofs/main.tex}
\end{document}

        \part{Project Overview}
            \documentclass[12pt, a4paper]{report}

\input{template/main.tex}

\title{\BA{Title in Progress...}}
\author{Boris Andrews}
\affil{Mathematical Institute, University of Oxford}
\date{\today}


\begin{document}
    \pagenumbering{gobble}
    \maketitle
    
    
    \begin{abstract}
        Magnetic confinement reactors---in particular tokamaks---offer one of the most promising options for achieving practical nuclear fusion, with the potential to provide virtually limitless, clean energy. The theoretical and numerical modeling of tokamak plasmas is simultaneously an essential component of effective reactor design, and a great research barrier. Tokamak operational conditions exhibit comparatively low Knudsen numbers. Kinetic effects, including kinetic waves and instabilities, Landau damping, bump-on-tail instabilities and more, are therefore highly influential in tokamak plasma dynamics. Purely fluid models are inherently incapable of capturing these effects, whereas the high dimensionality in purely kinetic models render them practically intractable for most relevant purposes.

        We consider a $\delta\!f$ decomposition model, with a macroscopic fluid background and microscopic kinetic correction, both fully coupled to each other. A similar manner of discretization is proposed to that used in the recent \texttt{STRUPHY} code \cite{Holderied_Possanner_Wang_2021, Holderied_2022, Li_et_al_2023} with a finite-element model for the background and a pseudo-particle/PiC model for the correction.

        The fluid background satisfies the full, non-linear, resistive, compressible, Hall MHD equations. \cite{Laakmann_Hu_Farrell_2022} introduces finite-element(-in-space) implicit timesteppers for the incompressible analogue to this system with structure-preserving (SP) properties in the ideal case, alongside parameter-robust preconditioners. We show that these timesteppers can derive from a finite-element-in-time (FET) (and finite-element-in-space) interpretation. The benefits of this reformulation are discussed, including the derivation of timesteppers that are higher order in time, and the quantifiable dissipative SP properties in the non-ideal, resistive case.
        
        We discuss possible options for extending this FET approach to timesteppers for the compressible case.

        The kinetic corrections satisfy linearized Boltzmann equations. Using a Lénard--Bernstein collision operator, these take Fokker--Planck-like forms \cite{Fokker_1914, Planck_1917} wherein pseudo-particles in the numerical model obey the neoclassical transport equations, with particle-independent Brownian drift terms. This offers a rigorous methodology for incorporating collisions into the particle transport model, without coupling the equations of motions for each particle.
        
        Works by Chen, Chacón et al. \cite{Chen_Chacón_Barnes_2011, Chacón_Chen_Barnes_2013, Chen_Chacón_2014, Chen_Chacón_2015} have developed structure-preserving particle pushers for neoclassical transport in the Vlasov equations, derived from Crank--Nicolson integrators. We show these too can can derive from a FET interpretation, similarly offering potential extensions to higher-order-in-time particle pushers. The FET formulation is used also to consider how the stochastic drift terms can be incorporated into the pushers. Stochastic gyrokinetic expansions are also discussed.

        Different options for the numerical implementation of these schemes are considered.

        Due to the efficacy of FET in the development of SP timesteppers for both the fluid and kinetic component, we hope this approach will prove effective in the future for developing SP timesteppers for the full hybrid model. We hope this will give us the opportunity to incorporate previously inaccessible kinetic effects into the highly effective, modern, finite-element MHD models.
    \end{abstract}
    
    
    \newpage
    \tableofcontents
    
    
    \newpage
    \pagenumbering{arabic}
    %\linenumbers\renewcommand\thelinenumber{\color{black!50}\arabic{linenumber}}
            \input{0 - introduction/main.tex}
        \part{Research}
            \input{1 - low-noise PiC models/main.tex}
            \input{2 - kinetic component/main.tex}
            \input{3 - fluid component/main.tex}
            \input{4 - numerical implementation/main.tex}
        \part{Project Overview}
            \input{5 - research plan/main.tex}
            \input{6 - summary/main.tex}
    
    
    %\section{}
    \newpage
    \pagenumbering{gobble}
        \printbibliography


    \newpage
    \pagenumbering{roman}
    \appendix
        \part{Appendices}
            \input{8 - Hilbert complexes/main.tex}
            \input{9 - weak conservation proofs/main.tex}
\end{document}

            \documentclass[12pt, a4paper]{report}

\input{template/main.tex}

\title{\BA{Title in Progress...}}
\author{Boris Andrews}
\affil{Mathematical Institute, University of Oxford}
\date{\today}


\begin{document}
    \pagenumbering{gobble}
    \maketitle
    
    
    \begin{abstract}
        Magnetic confinement reactors---in particular tokamaks---offer one of the most promising options for achieving practical nuclear fusion, with the potential to provide virtually limitless, clean energy. The theoretical and numerical modeling of tokamak plasmas is simultaneously an essential component of effective reactor design, and a great research barrier. Tokamak operational conditions exhibit comparatively low Knudsen numbers. Kinetic effects, including kinetic waves and instabilities, Landau damping, bump-on-tail instabilities and more, are therefore highly influential in tokamak plasma dynamics. Purely fluid models are inherently incapable of capturing these effects, whereas the high dimensionality in purely kinetic models render them practically intractable for most relevant purposes.

        We consider a $\delta\!f$ decomposition model, with a macroscopic fluid background and microscopic kinetic correction, both fully coupled to each other. A similar manner of discretization is proposed to that used in the recent \texttt{STRUPHY} code \cite{Holderied_Possanner_Wang_2021, Holderied_2022, Li_et_al_2023} with a finite-element model for the background and a pseudo-particle/PiC model for the correction.

        The fluid background satisfies the full, non-linear, resistive, compressible, Hall MHD equations. \cite{Laakmann_Hu_Farrell_2022} introduces finite-element(-in-space) implicit timesteppers for the incompressible analogue to this system with structure-preserving (SP) properties in the ideal case, alongside parameter-robust preconditioners. We show that these timesteppers can derive from a finite-element-in-time (FET) (and finite-element-in-space) interpretation. The benefits of this reformulation are discussed, including the derivation of timesteppers that are higher order in time, and the quantifiable dissipative SP properties in the non-ideal, resistive case.
        
        We discuss possible options for extending this FET approach to timesteppers for the compressible case.

        The kinetic corrections satisfy linearized Boltzmann equations. Using a Lénard--Bernstein collision operator, these take Fokker--Planck-like forms \cite{Fokker_1914, Planck_1917} wherein pseudo-particles in the numerical model obey the neoclassical transport equations, with particle-independent Brownian drift terms. This offers a rigorous methodology for incorporating collisions into the particle transport model, without coupling the equations of motions for each particle.
        
        Works by Chen, Chacón et al. \cite{Chen_Chacón_Barnes_2011, Chacón_Chen_Barnes_2013, Chen_Chacón_2014, Chen_Chacón_2015} have developed structure-preserving particle pushers for neoclassical transport in the Vlasov equations, derived from Crank--Nicolson integrators. We show these too can can derive from a FET interpretation, similarly offering potential extensions to higher-order-in-time particle pushers. The FET formulation is used also to consider how the stochastic drift terms can be incorporated into the pushers. Stochastic gyrokinetic expansions are also discussed.

        Different options for the numerical implementation of these schemes are considered.

        Due to the efficacy of FET in the development of SP timesteppers for both the fluid and kinetic component, we hope this approach will prove effective in the future for developing SP timesteppers for the full hybrid model. We hope this will give us the opportunity to incorporate previously inaccessible kinetic effects into the highly effective, modern, finite-element MHD models.
    \end{abstract}
    
    
    \newpage
    \tableofcontents
    
    
    \newpage
    \pagenumbering{arabic}
    %\linenumbers\renewcommand\thelinenumber{\color{black!50}\arabic{linenumber}}
            \input{0 - introduction/main.tex}
        \part{Research}
            \input{1 - low-noise PiC models/main.tex}
            \input{2 - kinetic component/main.tex}
            \input{3 - fluid component/main.tex}
            \input{4 - numerical implementation/main.tex}
        \part{Project Overview}
            \input{5 - research plan/main.tex}
            \input{6 - summary/main.tex}
    
    
    %\section{}
    \newpage
    \pagenumbering{gobble}
        \printbibliography


    \newpage
    \pagenumbering{roman}
    \appendix
        \part{Appendices}
            \input{8 - Hilbert complexes/main.tex}
            \input{9 - weak conservation proofs/main.tex}
\end{document}

    
    
    %\section{}
    \newpage
    \pagenumbering{gobble}
        \printbibliography


    \newpage
    \pagenumbering{roman}
    \appendix
        \part{Appendices}
            \documentclass[12pt, a4paper]{report}

\input{template/main.tex}

\title{\BA{Title in Progress...}}
\author{Boris Andrews}
\affil{Mathematical Institute, University of Oxford}
\date{\today}


\begin{document}
    \pagenumbering{gobble}
    \maketitle
    
    
    \begin{abstract}
        Magnetic confinement reactors---in particular tokamaks---offer one of the most promising options for achieving practical nuclear fusion, with the potential to provide virtually limitless, clean energy. The theoretical and numerical modeling of tokamak plasmas is simultaneously an essential component of effective reactor design, and a great research barrier. Tokamak operational conditions exhibit comparatively low Knudsen numbers. Kinetic effects, including kinetic waves and instabilities, Landau damping, bump-on-tail instabilities and more, are therefore highly influential in tokamak plasma dynamics. Purely fluid models are inherently incapable of capturing these effects, whereas the high dimensionality in purely kinetic models render them practically intractable for most relevant purposes.

        We consider a $\delta\!f$ decomposition model, with a macroscopic fluid background and microscopic kinetic correction, both fully coupled to each other. A similar manner of discretization is proposed to that used in the recent \texttt{STRUPHY} code \cite{Holderied_Possanner_Wang_2021, Holderied_2022, Li_et_al_2023} with a finite-element model for the background and a pseudo-particle/PiC model for the correction.

        The fluid background satisfies the full, non-linear, resistive, compressible, Hall MHD equations. \cite{Laakmann_Hu_Farrell_2022} introduces finite-element(-in-space) implicit timesteppers for the incompressible analogue to this system with structure-preserving (SP) properties in the ideal case, alongside parameter-robust preconditioners. We show that these timesteppers can derive from a finite-element-in-time (FET) (and finite-element-in-space) interpretation. The benefits of this reformulation are discussed, including the derivation of timesteppers that are higher order in time, and the quantifiable dissipative SP properties in the non-ideal, resistive case.
        
        We discuss possible options for extending this FET approach to timesteppers for the compressible case.

        The kinetic corrections satisfy linearized Boltzmann equations. Using a Lénard--Bernstein collision operator, these take Fokker--Planck-like forms \cite{Fokker_1914, Planck_1917} wherein pseudo-particles in the numerical model obey the neoclassical transport equations, with particle-independent Brownian drift terms. This offers a rigorous methodology for incorporating collisions into the particle transport model, without coupling the equations of motions for each particle.
        
        Works by Chen, Chacón et al. \cite{Chen_Chacón_Barnes_2011, Chacón_Chen_Barnes_2013, Chen_Chacón_2014, Chen_Chacón_2015} have developed structure-preserving particle pushers for neoclassical transport in the Vlasov equations, derived from Crank--Nicolson integrators. We show these too can can derive from a FET interpretation, similarly offering potential extensions to higher-order-in-time particle pushers. The FET formulation is used also to consider how the stochastic drift terms can be incorporated into the pushers. Stochastic gyrokinetic expansions are also discussed.

        Different options for the numerical implementation of these schemes are considered.

        Due to the efficacy of FET in the development of SP timesteppers for both the fluid and kinetic component, we hope this approach will prove effective in the future for developing SP timesteppers for the full hybrid model. We hope this will give us the opportunity to incorporate previously inaccessible kinetic effects into the highly effective, modern, finite-element MHD models.
    \end{abstract}
    
    
    \newpage
    \tableofcontents
    
    
    \newpage
    \pagenumbering{arabic}
    %\linenumbers\renewcommand\thelinenumber{\color{black!50}\arabic{linenumber}}
            \input{0 - introduction/main.tex}
        \part{Research}
            \input{1 - low-noise PiC models/main.tex}
            \input{2 - kinetic component/main.tex}
            \input{3 - fluid component/main.tex}
            \input{4 - numerical implementation/main.tex}
        \part{Project Overview}
            \input{5 - research plan/main.tex}
            \input{6 - summary/main.tex}
    
    
    %\section{}
    \newpage
    \pagenumbering{gobble}
        \printbibliography


    \newpage
    \pagenumbering{roman}
    \appendix
        \part{Appendices}
            \input{8 - Hilbert complexes/main.tex}
            \input{9 - weak conservation proofs/main.tex}
\end{document}

            \documentclass[12pt, a4paper]{report}

\input{template/main.tex}

\title{\BA{Title in Progress...}}
\author{Boris Andrews}
\affil{Mathematical Institute, University of Oxford}
\date{\today}


\begin{document}
    \pagenumbering{gobble}
    \maketitle
    
    
    \begin{abstract}
        Magnetic confinement reactors---in particular tokamaks---offer one of the most promising options for achieving practical nuclear fusion, with the potential to provide virtually limitless, clean energy. The theoretical and numerical modeling of tokamak plasmas is simultaneously an essential component of effective reactor design, and a great research barrier. Tokamak operational conditions exhibit comparatively low Knudsen numbers. Kinetic effects, including kinetic waves and instabilities, Landau damping, bump-on-tail instabilities and more, are therefore highly influential in tokamak plasma dynamics. Purely fluid models are inherently incapable of capturing these effects, whereas the high dimensionality in purely kinetic models render them practically intractable for most relevant purposes.

        We consider a $\delta\!f$ decomposition model, with a macroscopic fluid background and microscopic kinetic correction, both fully coupled to each other. A similar manner of discretization is proposed to that used in the recent \texttt{STRUPHY} code \cite{Holderied_Possanner_Wang_2021, Holderied_2022, Li_et_al_2023} with a finite-element model for the background and a pseudo-particle/PiC model for the correction.

        The fluid background satisfies the full, non-linear, resistive, compressible, Hall MHD equations. \cite{Laakmann_Hu_Farrell_2022} introduces finite-element(-in-space) implicit timesteppers for the incompressible analogue to this system with structure-preserving (SP) properties in the ideal case, alongside parameter-robust preconditioners. We show that these timesteppers can derive from a finite-element-in-time (FET) (and finite-element-in-space) interpretation. The benefits of this reformulation are discussed, including the derivation of timesteppers that are higher order in time, and the quantifiable dissipative SP properties in the non-ideal, resistive case.
        
        We discuss possible options for extending this FET approach to timesteppers for the compressible case.

        The kinetic corrections satisfy linearized Boltzmann equations. Using a Lénard--Bernstein collision operator, these take Fokker--Planck-like forms \cite{Fokker_1914, Planck_1917} wherein pseudo-particles in the numerical model obey the neoclassical transport equations, with particle-independent Brownian drift terms. This offers a rigorous methodology for incorporating collisions into the particle transport model, without coupling the equations of motions for each particle.
        
        Works by Chen, Chacón et al. \cite{Chen_Chacón_Barnes_2011, Chacón_Chen_Barnes_2013, Chen_Chacón_2014, Chen_Chacón_2015} have developed structure-preserving particle pushers for neoclassical transport in the Vlasov equations, derived from Crank--Nicolson integrators. We show these too can can derive from a FET interpretation, similarly offering potential extensions to higher-order-in-time particle pushers. The FET formulation is used also to consider how the stochastic drift terms can be incorporated into the pushers. Stochastic gyrokinetic expansions are also discussed.

        Different options for the numerical implementation of these schemes are considered.

        Due to the efficacy of FET in the development of SP timesteppers for both the fluid and kinetic component, we hope this approach will prove effective in the future for developing SP timesteppers for the full hybrid model. We hope this will give us the opportunity to incorporate previously inaccessible kinetic effects into the highly effective, modern, finite-element MHD models.
    \end{abstract}
    
    
    \newpage
    \tableofcontents
    
    
    \newpage
    \pagenumbering{arabic}
    %\linenumbers\renewcommand\thelinenumber{\color{black!50}\arabic{linenumber}}
            \input{0 - introduction/main.tex}
        \part{Research}
            \input{1 - low-noise PiC models/main.tex}
            \input{2 - kinetic component/main.tex}
            \input{3 - fluid component/main.tex}
            \input{4 - numerical implementation/main.tex}
        \part{Project Overview}
            \input{5 - research plan/main.tex}
            \input{6 - summary/main.tex}
    
    
    %\section{}
    \newpage
    \pagenumbering{gobble}
        \printbibliography


    \newpage
    \pagenumbering{roman}
    \appendix
        \part{Appendices}
            \input{8 - Hilbert complexes/main.tex}
            \input{9 - weak conservation proofs/main.tex}
\end{document}

\end{document}

            \documentclass[12pt, a4paper]{report}

\documentclass[12pt, a4paper]{report}

\input{template/main.tex}

\title{\BA{Title in Progress...}}
\author{Boris Andrews}
\affil{Mathematical Institute, University of Oxford}
\date{\today}


\begin{document}
    \pagenumbering{gobble}
    \maketitle
    
    
    \begin{abstract}
        Magnetic confinement reactors---in particular tokamaks---offer one of the most promising options for achieving practical nuclear fusion, with the potential to provide virtually limitless, clean energy. The theoretical and numerical modeling of tokamak plasmas is simultaneously an essential component of effective reactor design, and a great research barrier. Tokamak operational conditions exhibit comparatively low Knudsen numbers. Kinetic effects, including kinetic waves and instabilities, Landau damping, bump-on-tail instabilities and more, are therefore highly influential in tokamak plasma dynamics. Purely fluid models are inherently incapable of capturing these effects, whereas the high dimensionality in purely kinetic models render them practically intractable for most relevant purposes.

        We consider a $\delta\!f$ decomposition model, with a macroscopic fluid background and microscopic kinetic correction, both fully coupled to each other. A similar manner of discretization is proposed to that used in the recent \texttt{STRUPHY} code \cite{Holderied_Possanner_Wang_2021, Holderied_2022, Li_et_al_2023} with a finite-element model for the background and a pseudo-particle/PiC model for the correction.

        The fluid background satisfies the full, non-linear, resistive, compressible, Hall MHD equations. \cite{Laakmann_Hu_Farrell_2022} introduces finite-element(-in-space) implicit timesteppers for the incompressible analogue to this system with structure-preserving (SP) properties in the ideal case, alongside parameter-robust preconditioners. We show that these timesteppers can derive from a finite-element-in-time (FET) (and finite-element-in-space) interpretation. The benefits of this reformulation are discussed, including the derivation of timesteppers that are higher order in time, and the quantifiable dissipative SP properties in the non-ideal, resistive case.
        
        We discuss possible options for extending this FET approach to timesteppers for the compressible case.

        The kinetic corrections satisfy linearized Boltzmann equations. Using a Lénard--Bernstein collision operator, these take Fokker--Planck-like forms \cite{Fokker_1914, Planck_1917} wherein pseudo-particles in the numerical model obey the neoclassical transport equations, with particle-independent Brownian drift terms. This offers a rigorous methodology for incorporating collisions into the particle transport model, without coupling the equations of motions for each particle.
        
        Works by Chen, Chacón et al. \cite{Chen_Chacón_Barnes_2011, Chacón_Chen_Barnes_2013, Chen_Chacón_2014, Chen_Chacón_2015} have developed structure-preserving particle pushers for neoclassical transport in the Vlasov equations, derived from Crank--Nicolson integrators. We show these too can can derive from a FET interpretation, similarly offering potential extensions to higher-order-in-time particle pushers. The FET formulation is used also to consider how the stochastic drift terms can be incorporated into the pushers. Stochastic gyrokinetic expansions are also discussed.

        Different options for the numerical implementation of these schemes are considered.

        Due to the efficacy of FET in the development of SP timesteppers for both the fluid and kinetic component, we hope this approach will prove effective in the future for developing SP timesteppers for the full hybrid model. We hope this will give us the opportunity to incorporate previously inaccessible kinetic effects into the highly effective, modern, finite-element MHD models.
    \end{abstract}
    
    
    \newpage
    \tableofcontents
    
    
    \newpage
    \pagenumbering{arabic}
    %\linenumbers\renewcommand\thelinenumber{\color{black!50}\arabic{linenumber}}
            \input{0 - introduction/main.tex}
        \part{Research}
            \input{1 - low-noise PiC models/main.tex}
            \input{2 - kinetic component/main.tex}
            \input{3 - fluid component/main.tex}
            \input{4 - numerical implementation/main.tex}
        \part{Project Overview}
            \input{5 - research plan/main.tex}
            \input{6 - summary/main.tex}
    
    
    %\section{}
    \newpage
    \pagenumbering{gobble}
        \printbibliography


    \newpage
    \pagenumbering{roman}
    \appendix
        \part{Appendices}
            \input{8 - Hilbert complexes/main.tex}
            \input{9 - weak conservation proofs/main.tex}
\end{document}


\title{\BA{Title in Progress...}}
\author{Boris Andrews}
\affil{Mathematical Institute, University of Oxford}
\date{\today}


\begin{document}
    \pagenumbering{gobble}
    \maketitle
    
    
    \begin{abstract}
        Magnetic confinement reactors---in particular tokamaks---offer one of the most promising options for achieving practical nuclear fusion, with the potential to provide virtually limitless, clean energy. The theoretical and numerical modeling of tokamak plasmas is simultaneously an essential component of effective reactor design, and a great research barrier. Tokamak operational conditions exhibit comparatively low Knudsen numbers. Kinetic effects, including kinetic waves and instabilities, Landau damping, bump-on-tail instabilities and more, are therefore highly influential in tokamak plasma dynamics. Purely fluid models are inherently incapable of capturing these effects, whereas the high dimensionality in purely kinetic models render them practically intractable for most relevant purposes.

        We consider a $\delta\!f$ decomposition model, with a macroscopic fluid background and microscopic kinetic correction, both fully coupled to each other. A similar manner of discretization is proposed to that used in the recent \texttt{STRUPHY} code \cite{Holderied_Possanner_Wang_2021, Holderied_2022, Li_et_al_2023} with a finite-element model for the background and a pseudo-particle/PiC model for the correction.

        The fluid background satisfies the full, non-linear, resistive, compressible, Hall MHD equations. \cite{Laakmann_Hu_Farrell_2022} introduces finite-element(-in-space) implicit timesteppers for the incompressible analogue to this system with structure-preserving (SP) properties in the ideal case, alongside parameter-robust preconditioners. We show that these timesteppers can derive from a finite-element-in-time (FET) (and finite-element-in-space) interpretation. The benefits of this reformulation are discussed, including the derivation of timesteppers that are higher order in time, and the quantifiable dissipative SP properties in the non-ideal, resistive case.
        
        We discuss possible options for extending this FET approach to timesteppers for the compressible case.

        The kinetic corrections satisfy linearized Boltzmann equations. Using a Lénard--Bernstein collision operator, these take Fokker--Planck-like forms \cite{Fokker_1914, Planck_1917} wherein pseudo-particles in the numerical model obey the neoclassical transport equations, with particle-independent Brownian drift terms. This offers a rigorous methodology for incorporating collisions into the particle transport model, without coupling the equations of motions for each particle.
        
        Works by Chen, Chacón et al. \cite{Chen_Chacón_Barnes_2011, Chacón_Chen_Barnes_2013, Chen_Chacón_2014, Chen_Chacón_2015} have developed structure-preserving particle pushers for neoclassical transport in the Vlasov equations, derived from Crank--Nicolson integrators. We show these too can can derive from a FET interpretation, similarly offering potential extensions to higher-order-in-time particle pushers. The FET formulation is used also to consider how the stochastic drift terms can be incorporated into the pushers. Stochastic gyrokinetic expansions are also discussed.

        Different options for the numerical implementation of these schemes are considered.

        Due to the efficacy of FET in the development of SP timesteppers for both the fluid and kinetic component, we hope this approach will prove effective in the future for developing SP timesteppers for the full hybrid model. We hope this will give us the opportunity to incorporate previously inaccessible kinetic effects into the highly effective, modern, finite-element MHD models.
    \end{abstract}
    
    
    \newpage
    \tableofcontents
    
    
    \newpage
    \pagenumbering{arabic}
    %\linenumbers\renewcommand\thelinenumber{\color{black!50}\arabic{linenumber}}
            \documentclass[12pt, a4paper]{report}

\input{template/main.tex}

\title{\BA{Title in Progress...}}
\author{Boris Andrews}
\affil{Mathematical Institute, University of Oxford}
\date{\today}


\begin{document}
    \pagenumbering{gobble}
    \maketitle
    
    
    \begin{abstract}
        Magnetic confinement reactors---in particular tokamaks---offer one of the most promising options for achieving practical nuclear fusion, with the potential to provide virtually limitless, clean energy. The theoretical and numerical modeling of tokamak plasmas is simultaneously an essential component of effective reactor design, and a great research barrier. Tokamak operational conditions exhibit comparatively low Knudsen numbers. Kinetic effects, including kinetic waves and instabilities, Landau damping, bump-on-tail instabilities and more, are therefore highly influential in tokamak plasma dynamics. Purely fluid models are inherently incapable of capturing these effects, whereas the high dimensionality in purely kinetic models render them practically intractable for most relevant purposes.

        We consider a $\delta\!f$ decomposition model, with a macroscopic fluid background and microscopic kinetic correction, both fully coupled to each other. A similar manner of discretization is proposed to that used in the recent \texttt{STRUPHY} code \cite{Holderied_Possanner_Wang_2021, Holderied_2022, Li_et_al_2023} with a finite-element model for the background and a pseudo-particle/PiC model for the correction.

        The fluid background satisfies the full, non-linear, resistive, compressible, Hall MHD equations. \cite{Laakmann_Hu_Farrell_2022} introduces finite-element(-in-space) implicit timesteppers for the incompressible analogue to this system with structure-preserving (SP) properties in the ideal case, alongside parameter-robust preconditioners. We show that these timesteppers can derive from a finite-element-in-time (FET) (and finite-element-in-space) interpretation. The benefits of this reformulation are discussed, including the derivation of timesteppers that are higher order in time, and the quantifiable dissipative SP properties in the non-ideal, resistive case.
        
        We discuss possible options for extending this FET approach to timesteppers for the compressible case.

        The kinetic corrections satisfy linearized Boltzmann equations. Using a Lénard--Bernstein collision operator, these take Fokker--Planck-like forms \cite{Fokker_1914, Planck_1917} wherein pseudo-particles in the numerical model obey the neoclassical transport equations, with particle-independent Brownian drift terms. This offers a rigorous methodology for incorporating collisions into the particle transport model, without coupling the equations of motions for each particle.
        
        Works by Chen, Chacón et al. \cite{Chen_Chacón_Barnes_2011, Chacón_Chen_Barnes_2013, Chen_Chacón_2014, Chen_Chacón_2015} have developed structure-preserving particle pushers for neoclassical transport in the Vlasov equations, derived from Crank--Nicolson integrators. We show these too can can derive from a FET interpretation, similarly offering potential extensions to higher-order-in-time particle pushers. The FET formulation is used also to consider how the stochastic drift terms can be incorporated into the pushers. Stochastic gyrokinetic expansions are also discussed.

        Different options for the numerical implementation of these schemes are considered.

        Due to the efficacy of FET in the development of SP timesteppers for both the fluid and kinetic component, we hope this approach will prove effective in the future for developing SP timesteppers for the full hybrid model. We hope this will give us the opportunity to incorporate previously inaccessible kinetic effects into the highly effective, modern, finite-element MHD models.
    \end{abstract}
    
    
    \newpage
    \tableofcontents
    
    
    \newpage
    \pagenumbering{arabic}
    %\linenumbers\renewcommand\thelinenumber{\color{black!50}\arabic{linenumber}}
            \input{0 - introduction/main.tex}
        \part{Research}
            \input{1 - low-noise PiC models/main.tex}
            \input{2 - kinetic component/main.tex}
            \input{3 - fluid component/main.tex}
            \input{4 - numerical implementation/main.tex}
        \part{Project Overview}
            \input{5 - research plan/main.tex}
            \input{6 - summary/main.tex}
    
    
    %\section{}
    \newpage
    \pagenumbering{gobble}
        \printbibliography


    \newpage
    \pagenumbering{roman}
    \appendix
        \part{Appendices}
            \input{8 - Hilbert complexes/main.tex}
            \input{9 - weak conservation proofs/main.tex}
\end{document}

        \part{Research}
            \documentclass[12pt, a4paper]{report}

\input{template/main.tex}

\title{\BA{Title in Progress...}}
\author{Boris Andrews}
\affil{Mathematical Institute, University of Oxford}
\date{\today}


\begin{document}
    \pagenumbering{gobble}
    \maketitle
    
    
    \begin{abstract}
        Magnetic confinement reactors---in particular tokamaks---offer one of the most promising options for achieving practical nuclear fusion, with the potential to provide virtually limitless, clean energy. The theoretical and numerical modeling of tokamak plasmas is simultaneously an essential component of effective reactor design, and a great research barrier. Tokamak operational conditions exhibit comparatively low Knudsen numbers. Kinetic effects, including kinetic waves and instabilities, Landau damping, bump-on-tail instabilities and more, are therefore highly influential in tokamak plasma dynamics. Purely fluid models are inherently incapable of capturing these effects, whereas the high dimensionality in purely kinetic models render them practically intractable for most relevant purposes.

        We consider a $\delta\!f$ decomposition model, with a macroscopic fluid background and microscopic kinetic correction, both fully coupled to each other. A similar manner of discretization is proposed to that used in the recent \texttt{STRUPHY} code \cite{Holderied_Possanner_Wang_2021, Holderied_2022, Li_et_al_2023} with a finite-element model for the background and a pseudo-particle/PiC model for the correction.

        The fluid background satisfies the full, non-linear, resistive, compressible, Hall MHD equations. \cite{Laakmann_Hu_Farrell_2022} introduces finite-element(-in-space) implicit timesteppers for the incompressible analogue to this system with structure-preserving (SP) properties in the ideal case, alongside parameter-robust preconditioners. We show that these timesteppers can derive from a finite-element-in-time (FET) (and finite-element-in-space) interpretation. The benefits of this reformulation are discussed, including the derivation of timesteppers that are higher order in time, and the quantifiable dissipative SP properties in the non-ideal, resistive case.
        
        We discuss possible options for extending this FET approach to timesteppers for the compressible case.

        The kinetic corrections satisfy linearized Boltzmann equations. Using a Lénard--Bernstein collision operator, these take Fokker--Planck-like forms \cite{Fokker_1914, Planck_1917} wherein pseudo-particles in the numerical model obey the neoclassical transport equations, with particle-independent Brownian drift terms. This offers a rigorous methodology for incorporating collisions into the particle transport model, without coupling the equations of motions for each particle.
        
        Works by Chen, Chacón et al. \cite{Chen_Chacón_Barnes_2011, Chacón_Chen_Barnes_2013, Chen_Chacón_2014, Chen_Chacón_2015} have developed structure-preserving particle pushers for neoclassical transport in the Vlasov equations, derived from Crank--Nicolson integrators. We show these too can can derive from a FET interpretation, similarly offering potential extensions to higher-order-in-time particle pushers. The FET formulation is used also to consider how the stochastic drift terms can be incorporated into the pushers. Stochastic gyrokinetic expansions are also discussed.

        Different options for the numerical implementation of these schemes are considered.

        Due to the efficacy of FET in the development of SP timesteppers for both the fluid and kinetic component, we hope this approach will prove effective in the future for developing SP timesteppers for the full hybrid model. We hope this will give us the opportunity to incorporate previously inaccessible kinetic effects into the highly effective, modern, finite-element MHD models.
    \end{abstract}
    
    
    \newpage
    \tableofcontents
    
    
    \newpage
    \pagenumbering{arabic}
    %\linenumbers\renewcommand\thelinenumber{\color{black!50}\arabic{linenumber}}
            \input{0 - introduction/main.tex}
        \part{Research}
            \input{1 - low-noise PiC models/main.tex}
            \input{2 - kinetic component/main.tex}
            \input{3 - fluid component/main.tex}
            \input{4 - numerical implementation/main.tex}
        \part{Project Overview}
            \input{5 - research plan/main.tex}
            \input{6 - summary/main.tex}
    
    
    %\section{}
    \newpage
    \pagenumbering{gobble}
        \printbibliography


    \newpage
    \pagenumbering{roman}
    \appendix
        \part{Appendices}
            \input{8 - Hilbert complexes/main.tex}
            \input{9 - weak conservation proofs/main.tex}
\end{document}

            \documentclass[12pt, a4paper]{report}

\input{template/main.tex}

\title{\BA{Title in Progress...}}
\author{Boris Andrews}
\affil{Mathematical Institute, University of Oxford}
\date{\today}


\begin{document}
    \pagenumbering{gobble}
    \maketitle
    
    
    \begin{abstract}
        Magnetic confinement reactors---in particular tokamaks---offer one of the most promising options for achieving practical nuclear fusion, with the potential to provide virtually limitless, clean energy. The theoretical and numerical modeling of tokamak plasmas is simultaneously an essential component of effective reactor design, and a great research barrier. Tokamak operational conditions exhibit comparatively low Knudsen numbers. Kinetic effects, including kinetic waves and instabilities, Landau damping, bump-on-tail instabilities and more, are therefore highly influential in tokamak plasma dynamics. Purely fluid models are inherently incapable of capturing these effects, whereas the high dimensionality in purely kinetic models render them practically intractable for most relevant purposes.

        We consider a $\delta\!f$ decomposition model, with a macroscopic fluid background and microscopic kinetic correction, both fully coupled to each other. A similar manner of discretization is proposed to that used in the recent \texttt{STRUPHY} code \cite{Holderied_Possanner_Wang_2021, Holderied_2022, Li_et_al_2023} with a finite-element model for the background and a pseudo-particle/PiC model for the correction.

        The fluid background satisfies the full, non-linear, resistive, compressible, Hall MHD equations. \cite{Laakmann_Hu_Farrell_2022} introduces finite-element(-in-space) implicit timesteppers for the incompressible analogue to this system with structure-preserving (SP) properties in the ideal case, alongside parameter-robust preconditioners. We show that these timesteppers can derive from a finite-element-in-time (FET) (and finite-element-in-space) interpretation. The benefits of this reformulation are discussed, including the derivation of timesteppers that are higher order in time, and the quantifiable dissipative SP properties in the non-ideal, resistive case.
        
        We discuss possible options for extending this FET approach to timesteppers for the compressible case.

        The kinetic corrections satisfy linearized Boltzmann equations. Using a Lénard--Bernstein collision operator, these take Fokker--Planck-like forms \cite{Fokker_1914, Planck_1917} wherein pseudo-particles in the numerical model obey the neoclassical transport equations, with particle-independent Brownian drift terms. This offers a rigorous methodology for incorporating collisions into the particle transport model, without coupling the equations of motions for each particle.
        
        Works by Chen, Chacón et al. \cite{Chen_Chacón_Barnes_2011, Chacón_Chen_Barnes_2013, Chen_Chacón_2014, Chen_Chacón_2015} have developed structure-preserving particle pushers for neoclassical transport in the Vlasov equations, derived from Crank--Nicolson integrators. We show these too can can derive from a FET interpretation, similarly offering potential extensions to higher-order-in-time particle pushers. The FET formulation is used also to consider how the stochastic drift terms can be incorporated into the pushers. Stochastic gyrokinetic expansions are also discussed.

        Different options for the numerical implementation of these schemes are considered.

        Due to the efficacy of FET in the development of SP timesteppers for both the fluid and kinetic component, we hope this approach will prove effective in the future for developing SP timesteppers for the full hybrid model. We hope this will give us the opportunity to incorporate previously inaccessible kinetic effects into the highly effective, modern, finite-element MHD models.
    \end{abstract}
    
    
    \newpage
    \tableofcontents
    
    
    \newpage
    \pagenumbering{arabic}
    %\linenumbers\renewcommand\thelinenumber{\color{black!50}\arabic{linenumber}}
            \input{0 - introduction/main.tex}
        \part{Research}
            \input{1 - low-noise PiC models/main.tex}
            \input{2 - kinetic component/main.tex}
            \input{3 - fluid component/main.tex}
            \input{4 - numerical implementation/main.tex}
        \part{Project Overview}
            \input{5 - research plan/main.tex}
            \input{6 - summary/main.tex}
    
    
    %\section{}
    \newpage
    \pagenumbering{gobble}
        \printbibliography


    \newpage
    \pagenumbering{roman}
    \appendix
        \part{Appendices}
            \input{8 - Hilbert complexes/main.tex}
            \input{9 - weak conservation proofs/main.tex}
\end{document}

            \documentclass[12pt, a4paper]{report}

\input{template/main.tex}

\title{\BA{Title in Progress...}}
\author{Boris Andrews}
\affil{Mathematical Institute, University of Oxford}
\date{\today}


\begin{document}
    \pagenumbering{gobble}
    \maketitle
    
    
    \begin{abstract}
        Magnetic confinement reactors---in particular tokamaks---offer one of the most promising options for achieving practical nuclear fusion, with the potential to provide virtually limitless, clean energy. The theoretical and numerical modeling of tokamak plasmas is simultaneously an essential component of effective reactor design, and a great research barrier. Tokamak operational conditions exhibit comparatively low Knudsen numbers. Kinetic effects, including kinetic waves and instabilities, Landau damping, bump-on-tail instabilities and more, are therefore highly influential in tokamak plasma dynamics. Purely fluid models are inherently incapable of capturing these effects, whereas the high dimensionality in purely kinetic models render them practically intractable for most relevant purposes.

        We consider a $\delta\!f$ decomposition model, with a macroscopic fluid background and microscopic kinetic correction, both fully coupled to each other. A similar manner of discretization is proposed to that used in the recent \texttt{STRUPHY} code \cite{Holderied_Possanner_Wang_2021, Holderied_2022, Li_et_al_2023} with a finite-element model for the background and a pseudo-particle/PiC model for the correction.

        The fluid background satisfies the full, non-linear, resistive, compressible, Hall MHD equations. \cite{Laakmann_Hu_Farrell_2022} introduces finite-element(-in-space) implicit timesteppers for the incompressible analogue to this system with structure-preserving (SP) properties in the ideal case, alongside parameter-robust preconditioners. We show that these timesteppers can derive from a finite-element-in-time (FET) (and finite-element-in-space) interpretation. The benefits of this reformulation are discussed, including the derivation of timesteppers that are higher order in time, and the quantifiable dissipative SP properties in the non-ideal, resistive case.
        
        We discuss possible options for extending this FET approach to timesteppers for the compressible case.

        The kinetic corrections satisfy linearized Boltzmann equations. Using a Lénard--Bernstein collision operator, these take Fokker--Planck-like forms \cite{Fokker_1914, Planck_1917} wherein pseudo-particles in the numerical model obey the neoclassical transport equations, with particle-independent Brownian drift terms. This offers a rigorous methodology for incorporating collisions into the particle transport model, without coupling the equations of motions for each particle.
        
        Works by Chen, Chacón et al. \cite{Chen_Chacón_Barnes_2011, Chacón_Chen_Barnes_2013, Chen_Chacón_2014, Chen_Chacón_2015} have developed structure-preserving particle pushers for neoclassical transport in the Vlasov equations, derived from Crank--Nicolson integrators. We show these too can can derive from a FET interpretation, similarly offering potential extensions to higher-order-in-time particle pushers. The FET formulation is used also to consider how the stochastic drift terms can be incorporated into the pushers. Stochastic gyrokinetic expansions are also discussed.

        Different options for the numerical implementation of these schemes are considered.

        Due to the efficacy of FET in the development of SP timesteppers for both the fluid and kinetic component, we hope this approach will prove effective in the future for developing SP timesteppers for the full hybrid model. We hope this will give us the opportunity to incorporate previously inaccessible kinetic effects into the highly effective, modern, finite-element MHD models.
    \end{abstract}
    
    
    \newpage
    \tableofcontents
    
    
    \newpage
    \pagenumbering{arabic}
    %\linenumbers\renewcommand\thelinenumber{\color{black!50}\arabic{linenumber}}
            \input{0 - introduction/main.tex}
        \part{Research}
            \input{1 - low-noise PiC models/main.tex}
            \input{2 - kinetic component/main.tex}
            \input{3 - fluid component/main.tex}
            \input{4 - numerical implementation/main.tex}
        \part{Project Overview}
            \input{5 - research plan/main.tex}
            \input{6 - summary/main.tex}
    
    
    %\section{}
    \newpage
    \pagenumbering{gobble}
        \printbibliography


    \newpage
    \pagenumbering{roman}
    \appendix
        \part{Appendices}
            \input{8 - Hilbert complexes/main.tex}
            \input{9 - weak conservation proofs/main.tex}
\end{document}

            \documentclass[12pt, a4paper]{report}

\input{template/main.tex}

\title{\BA{Title in Progress...}}
\author{Boris Andrews}
\affil{Mathematical Institute, University of Oxford}
\date{\today}


\begin{document}
    \pagenumbering{gobble}
    \maketitle
    
    
    \begin{abstract}
        Magnetic confinement reactors---in particular tokamaks---offer one of the most promising options for achieving practical nuclear fusion, with the potential to provide virtually limitless, clean energy. The theoretical and numerical modeling of tokamak plasmas is simultaneously an essential component of effective reactor design, and a great research barrier. Tokamak operational conditions exhibit comparatively low Knudsen numbers. Kinetic effects, including kinetic waves and instabilities, Landau damping, bump-on-tail instabilities and more, are therefore highly influential in tokamak plasma dynamics. Purely fluid models are inherently incapable of capturing these effects, whereas the high dimensionality in purely kinetic models render them practically intractable for most relevant purposes.

        We consider a $\delta\!f$ decomposition model, with a macroscopic fluid background and microscopic kinetic correction, both fully coupled to each other. A similar manner of discretization is proposed to that used in the recent \texttt{STRUPHY} code \cite{Holderied_Possanner_Wang_2021, Holderied_2022, Li_et_al_2023} with a finite-element model for the background and a pseudo-particle/PiC model for the correction.

        The fluid background satisfies the full, non-linear, resistive, compressible, Hall MHD equations. \cite{Laakmann_Hu_Farrell_2022} introduces finite-element(-in-space) implicit timesteppers for the incompressible analogue to this system with structure-preserving (SP) properties in the ideal case, alongside parameter-robust preconditioners. We show that these timesteppers can derive from a finite-element-in-time (FET) (and finite-element-in-space) interpretation. The benefits of this reformulation are discussed, including the derivation of timesteppers that are higher order in time, and the quantifiable dissipative SP properties in the non-ideal, resistive case.
        
        We discuss possible options for extending this FET approach to timesteppers for the compressible case.

        The kinetic corrections satisfy linearized Boltzmann equations. Using a Lénard--Bernstein collision operator, these take Fokker--Planck-like forms \cite{Fokker_1914, Planck_1917} wherein pseudo-particles in the numerical model obey the neoclassical transport equations, with particle-independent Brownian drift terms. This offers a rigorous methodology for incorporating collisions into the particle transport model, without coupling the equations of motions for each particle.
        
        Works by Chen, Chacón et al. \cite{Chen_Chacón_Barnes_2011, Chacón_Chen_Barnes_2013, Chen_Chacón_2014, Chen_Chacón_2015} have developed structure-preserving particle pushers for neoclassical transport in the Vlasov equations, derived from Crank--Nicolson integrators. We show these too can can derive from a FET interpretation, similarly offering potential extensions to higher-order-in-time particle pushers. The FET formulation is used also to consider how the stochastic drift terms can be incorporated into the pushers. Stochastic gyrokinetic expansions are also discussed.

        Different options for the numerical implementation of these schemes are considered.

        Due to the efficacy of FET in the development of SP timesteppers for both the fluid and kinetic component, we hope this approach will prove effective in the future for developing SP timesteppers for the full hybrid model. We hope this will give us the opportunity to incorporate previously inaccessible kinetic effects into the highly effective, modern, finite-element MHD models.
    \end{abstract}
    
    
    \newpage
    \tableofcontents
    
    
    \newpage
    \pagenumbering{arabic}
    %\linenumbers\renewcommand\thelinenumber{\color{black!50}\arabic{linenumber}}
            \input{0 - introduction/main.tex}
        \part{Research}
            \input{1 - low-noise PiC models/main.tex}
            \input{2 - kinetic component/main.tex}
            \input{3 - fluid component/main.tex}
            \input{4 - numerical implementation/main.tex}
        \part{Project Overview}
            \input{5 - research plan/main.tex}
            \input{6 - summary/main.tex}
    
    
    %\section{}
    \newpage
    \pagenumbering{gobble}
        \printbibliography


    \newpage
    \pagenumbering{roman}
    \appendix
        \part{Appendices}
            \input{8 - Hilbert complexes/main.tex}
            \input{9 - weak conservation proofs/main.tex}
\end{document}

        \part{Project Overview}
            \documentclass[12pt, a4paper]{report}

\input{template/main.tex}

\title{\BA{Title in Progress...}}
\author{Boris Andrews}
\affil{Mathematical Institute, University of Oxford}
\date{\today}


\begin{document}
    \pagenumbering{gobble}
    \maketitle
    
    
    \begin{abstract}
        Magnetic confinement reactors---in particular tokamaks---offer one of the most promising options for achieving practical nuclear fusion, with the potential to provide virtually limitless, clean energy. The theoretical and numerical modeling of tokamak plasmas is simultaneously an essential component of effective reactor design, and a great research barrier. Tokamak operational conditions exhibit comparatively low Knudsen numbers. Kinetic effects, including kinetic waves and instabilities, Landau damping, bump-on-tail instabilities and more, are therefore highly influential in tokamak plasma dynamics. Purely fluid models are inherently incapable of capturing these effects, whereas the high dimensionality in purely kinetic models render them practically intractable for most relevant purposes.

        We consider a $\delta\!f$ decomposition model, with a macroscopic fluid background and microscopic kinetic correction, both fully coupled to each other. A similar manner of discretization is proposed to that used in the recent \texttt{STRUPHY} code \cite{Holderied_Possanner_Wang_2021, Holderied_2022, Li_et_al_2023} with a finite-element model for the background and a pseudo-particle/PiC model for the correction.

        The fluid background satisfies the full, non-linear, resistive, compressible, Hall MHD equations. \cite{Laakmann_Hu_Farrell_2022} introduces finite-element(-in-space) implicit timesteppers for the incompressible analogue to this system with structure-preserving (SP) properties in the ideal case, alongside parameter-robust preconditioners. We show that these timesteppers can derive from a finite-element-in-time (FET) (and finite-element-in-space) interpretation. The benefits of this reformulation are discussed, including the derivation of timesteppers that are higher order in time, and the quantifiable dissipative SP properties in the non-ideal, resistive case.
        
        We discuss possible options for extending this FET approach to timesteppers for the compressible case.

        The kinetic corrections satisfy linearized Boltzmann equations. Using a Lénard--Bernstein collision operator, these take Fokker--Planck-like forms \cite{Fokker_1914, Planck_1917} wherein pseudo-particles in the numerical model obey the neoclassical transport equations, with particle-independent Brownian drift terms. This offers a rigorous methodology for incorporating collisions into the particle transport model, without coupling the equations of motions for each particle.
        
        Works by Chen, Chacón et al. \cite{Chen_Chacón_Barnes_2011, Chacón_Chen_Barnes_2013, Chen_Chacón_2014, Chen_Chacón_2015} have developed structure-preserving particle pushers for neoclassical transport in the Vlasov equations, derived from Crank--Nicolson integrators. We show these too can can derive from a FET interpretation, similarly offering potential extensions to higher-order-in-time particle pushers. The FET formulation is used also to consider how the stochastic drift terms can be incorporated into the pushers. Stochastic gyrokinetic expansions are also discussed.

        Different options for the numerical implementation of these schemes are considered.

        Due to the efficacy of FET in the development of SP timesteppers for both the fluid and kinetic component, we hope this approach will prove effective in the future for developing SP timesteppers for the full hybrid model. We hope this will give us the opportunity to incorporate previously inaccessible kinetic effects into the highly effective, modern, finite-element MHD models.
    \end{abstract}
    
    
    \newpage
    \tableofcontents
    
    
    \newpage
    \pagenumbering{arabic}
    %\linenumbers\renewcommand\thelinenumber{\color{black!50}\arabic{linenumber}}
            \input{0 - introduction/main.tex}
        \part{Research}
            \input{1 - low-noise PiC models/main.tex}
            \input{2 - kinetic component/main.tex}
            \input{3 - fluid component/main.tex}
            \input{4 - numerical implementation/main.tex}
        \part{Project Overview}
            \input{5 - research plan/main.tex}
            \input{6 - summary/main.tex}
    
    
    %\section{}
    \newpage
    \pagenumbering{gobble}
        \printbibliography


    \newpage
    \pagenumbering{roman}
    \appendix
        \part{Appendices}
            \input{8 - Hilbert complexes/main.tex}
            \input{9 - weak conservation proofs/main.tex}
\end{document}

            \documentclass[12pt, a4paper]{report}

\input{template/main.tex}

\title{\BA{Title in Progress...}}
\author{Boris Andrews}
\affil{Mathematical Institute, University of Oxford}
\date{\today}


\begin{document}
    \pagenumbering{gobble}
    \maketitle
    
    
    \begin{abstract}
        Magnetic confinement reactors---in particular tokamaks---offer one of the most promising options for achieving practical nuclear fusion, with the potential to provide virtually limitless, clean energy. The theoretical and numerical modeling of tokamak plasmas is simultaneously an essential component of effective reactor design, and a great research barrier. Tokamak operational conditions exhibit comparatively low Knudsen numbers. Kinetic effects, including kinetic waves and instabilities, Landau damping, bump-on-tail instabilities and more, are therefore highly influential in tokamak plasma dynamics. Purely fluid models are inherently incapable of capturing these effects, whereas the high dimensionality in purely kinetic models render them practically intractable for most relevant purposes.

        We consider a $\delta\!f$ decomposition model, with a macroscopic fluid background and microscopic kinetic correction, both fully coupled to each other. A similar manner of discretization is proposed to that used in the recent \texttt{STRUPHY} code \cite{Holderied_Possanner_Wang_2021, Holderied_2022, Li_et_al_2023} with a finite-element model for the background and a pseudo-particle/PiC model for the correction.

        The fluid background satisfies the full, non-linear, resistive, compressible, Hall MHD equations. \cite{Laakmann_Hu_Farrell_2022} introduces finite-element(-in-space) implicit timesteppers for the incompressible analogue to this system with structure-preserving (SP) properties in the ideal case, alongside parameter-robust preconditioners. We show that these timesteppers can derive from a finite-element-in-time (FET) (and finite-element-in-space) interpretation. The benefits of this reformulation are discussed, including the derivation of timesteppers that are higher order in time, and the quantifiable dissipative SP properties in the non-ideal, resistive case.
        
        We discuss possible options for extending this FET approach to timesteppers for the compressible case.

        The kinetic corrections satisfy linearized Boltzmann equations. Using a Lénard--Bernstein collision operator, these take Fokker--Planck-like forms \cite{Fokker_1914, Planck_1917} wherein pseudo-particles in the numerical model obey the neoclassical transport equations, with particle-independent Brownian drift terms. This offers a rigorous methodology for incorporating collisions into the particle transport model, without coupling the equations of motions for each particle.
        
        Works by Chen, Chacón et al. \cite{Chen_Chacón_Barnes_2011, Chacón_Chen_Barnes_2013, Chen_Chacón_2014, Chen_Chacón_2015} have developed structure-preserving particle pushers for neoclassical transport in the Vlasov equations, derived from Crank--Nicolson integrators. We show these too can can derive from a FET interpretation, similarly offering potential extensions to higher-order-in-time particle pushers. The FET formulation is used also to consider how the stochastic drift terms can be incorporated into the pushers. Stochastic gyrokinetic expansions are also discussed.

        Different options for the numerical implementation of these schemes are considered.

        Due to the efficacy of FET in the development of SP timesteppers for both the fluid and kinetic component, we hope this approach will prove effective in the future for developing SP timesteppers for the full hybrid model. We hope this will give us the opportunity to incorporate previously inaccessible kinetic effects into the highly effective, modern, finite-element MHD models.
    \end{abstract}
    
    
    \newpage
    \tableofcontents
    
    
    \newpage
    \pagenumbering{arabic}
    %\linenumbers\renewcommand\thelinenumber{\color{black!50}\arabic{linenumber}}
            \input{0 - introduction/main.tex}
        \part{Research}
            \input{1 - low-noise PiC models/main.tex}
            \input{2 - kinetic component/main.tex}
            \input{3 - fluid component/main.tex}
            \input{4 - numerical implementation/main.tex}
        \part{Project Overview}
            \input{5 - research plan/main.tex}
            \input{6 - summary/main.tex}
    
    
    %\section{}
    \newpage
    \pagenumbering{gobble}
        \printbibliography


    \newpage
    \pagenumbering{roman}
    \appendix
        \part{Appendices}
            \input{8 - Hilbert complexes/main.tex}
            \input{9 - weak conservation proofs/main.tex}
\end{document}

    
    
    %\section{}
    \newpage
    \pagenumbering{gobble}
        \printbibliography


    \newpage
    \pagenumbering{roman}
    \appendix
        \part{Appendices}
            \documentclass[12pt, a4paper]{report}

\input{template/main.tex}

\title{\BA{Title in Progress...}}
\author{Boris Andrews}
\affil{Mathematical Institute, University of Oxford}
\date{\today}


\begin{document}
    \pagenumbering{gobble}
    \maketitle
    
    
    \begin{abstract}
        Magnetic confinement reactors---in particular tokamaks---offer one of the most promising options for achieving practical nuclear fusion, with the potential to provide virtually limitless, clean energy. The theoretical and numerical modeling of tokamak plasmas is simultaneously an essential component of effective reactor design, and a great research barrier. Tokamak operational conditions exhibit comparatively low Knudsen numbers. Kinetic effects, including kinetic waves and instabilities, Landau damping, bump-on-tail instabilities and more, are therefore highly influential in tokamak plasma dynamics. Purely fluid models are inherently incapable of capturing these effects, whereas the high dimensionality in purely kinetic models render them practically intractable for most relevant purposes.

        We consider a $\delta\!f$ decomposition model, with a macroscopic fluid background and microscopic kinetic correction, both fully coupled to each other. A similar manner of discretization is proposed to that used in the recent \texttt{STRUPHY} code \cite{Holderied_Possanner_Wang_2021, Holderied_2022, Li_et_al_2023} with a finite-element model for the background and a pseudo-particle/PiC model for the correction.

        The fluid background satisfies the full, non-linear, resistive, compressible, Hall MHD equations. \cite{Laakmann_Hu_Farrell_2022} introduces finite-element(-in-space) implicit timesteppers for the incompressible analogue to this system with structure-preserving (SP) properties in the ideal case, alongside parameter-robust preconditioners. We show that these timesteppers can derive from a finite-element-in-time (FET) (and finite-element-in-space) interpretation. The benefits of this reformulation are discussed, including the derivation of timesteppers that are higher order in time, and the quantifiable dissipative SP properties in the non-ideal, resistive case.
        
        We discuss possible options for extending this FET approach to timesteppers for the compressible case.

        The kinetic corrections satisfy linearized Boltzmann equations. Using a Lénard--Bernstein collision operator, these take Fokker--Planck-like forms \cite{Fokker_1914, Planck_1917} wherein pseudo-particles in the numerical model obey the neoclassical transport equations, with particle-independent Brownian drift terms. This offers a rigorous methodology for incorporating collisions into the particle transport model, without coupling the equations of motions for each particle.
        
        Works by Chen, Chacón et al. \cite{Chen_Chacón_Barnes_2011, Chacón_Chen_Barnes_2013, Chen_Chacón_2014, Chen_Chacón_2015} have developed structure-preserving particle pushers for neoclassical transport in the Vlasov equations, derived from Crank--Nicolson integrators. We show these too can can derive from a FET interpretation, similarly offering potential extensions to higher-order-in-time particle pushers. The FET formulation is used also to consider how the stochastic drift terms can be incorporated into the pushers. Stochastic gyrokinetic expansions are also discussed.

        Different options for the numerical implementation of these schemes are considered.

        Due to the efficacy of FET in the development of SP timesteppers for both the fluid and kinetic component, we hope this approach will prove effective in the future for developing SP timesteppers for the full hybrid model. We hope this will give us the opportunity to incorporate previously inaccessible kinetic effects into the highly effective, modern, finite-element MHD models.
    \end{abstract}
    
    
    \newpage
    \tableofcontents
    
    
    \newpage
    \pagenumbering{arabic}
    %\linenumbers\renewcommand\thelinenumber{\color{black!50}\arabic{linenumber}}
            \input{0 - introduction/main.tex}
        \part{Research}
            \input{1 - low-noise PiC models/main.tex}
            \input{2 - kinetic component/main.tex}
            \input{3 - fluid component/main.tex}
            \input{4 - numerical implementation/main.tex}
        \part{Project Overview}
            \input{5 - research plan/main.tex}
            \input{6 - summary/main.tex}
    
    
    %\section{}
    \newpage
    \pagenumbering{gobble}
        \printbibliography


    \newpage
    \pagenumbering{roman}
    \appendix
        \part{Appendices}
            \input{8 - Hilbert complexes/main.tex}
            \input{9 - weak conservation proofs/main.tex}
\end{document}

            \documentclass[12pt, a4paper]{report}

\input{template/main.tex}

\title{\BA{Title in Progress...}}
\author{Boris Andrews}
\affil{Mathematical Institute, University of Oxford}
\date{\today}


\begin{document}
    \pagenumbering{gobble}
    \maketitle
    
    
    \begin{abstract}
        Magnetic confinement reactors---in particular tokamaks---offer one of the most promising options for achieving practical nuclear fusion, with the potential to provide virtually limitless, clean energy. The theoretical and numerical modeling of tokamak plasmas is simultaneously an essential component of effective reactor design, and a great research barrier. Tokamak operational conditions exhibit comparatively low Knudsen numbers. Kinetic effects, including kinetic waves and instabilities, Landau damping, bump-on-tail instabilities and more, are therefore highly influential in tokamak plasma dynamics. Purely fluid models are inherently incapable of capturing these effects, whereas the high dimensionality in purely kinetic models render them practically intractable for most relevant purposes.

        We consider a $\delta\!f$ decomposition model, with a macroscopic fluid background and microscopic kinetic correction, both fully coupled to each other. A similar manner of discretization is proposed to that used in the recent \texttt{STRUPHY} code \cite{Holderied_Possanner_Wang_2021, Holderied_2022, Li_et_al_2023} with a finite-element model for the background and a pseudo-particle/PiC model for the correction.

        The fluid background satisfies the full, non-linear, resistive, compressible, Hall MHD equations. \cite{Laakmann_Hu_Farrell_2022} introduces finite-element(-in-space) implicit timesteppers for the incompressible analogue to this system with structure-preserving (SP) properties in the ideal case, alongside parameter-robust preconditioners. We show that these timesteppers can derive from a finite-element-in-time (FET) (and finite-element-in-space) interpretation. The benefits of this reformulation are discussed, including the derivation of timesteppers that are higher order in time, and the quantifiable dissipative SP properties in the non-ideal, resistive case.
        
        We discuss possible options for extending this FET approach to timesteppers for the compressible case.

        The kinetic corrections satisfy linearized Boltzmann equations. Using a Lénard--Bernstein collision operator, these take Fokker--Planck-like forms \cite{Fokker_1914, Planck_1917} wherein pseudo-particles in the numerical model obey the neoclassical transport equations, with particle-independent Brownian drift terms. This offers a rigorous methodology for incorporating collisions into the particle transport model, without coupling the equations of motions for each particle.
        
        Works by Chen, Chacón et al. \cite{Chen_Chacón_Barnes_2011, Chacón_Chen_Barnes_2013, Chen_Chacón_2014, Chen_Chacón_2015} have developed structure-preserving particle pushers for neoclassical transport in the Vlasov equations, derived from Crank--Nicolson integrators. We show these too can can derive from a FET interpretation, similarly offering potential extensions to higher-order-in-time particle pushers. The FET formulation is used also to consider how the stochastic drift terms can be incorporated into the pushers. Stochastic gyrokinetic expansions are also discussed.

        Different options for the numerical implementation of these schemes are considered.

        Due to the efficacy of FET in the development of SP timesteppers for both the fluid and kinetic component, we hope this approach will prove effective in the future for developing SP timesteppers for the full hybrid model. We hope this will give us the opportunity to incorporate previously inaccessible kinetic effects into the highly effective, modern, finite-element MHD models.
    \end{abstract}
    
    
    \newpage
    \tableofcontents
    
    
    \newpage
    \pagenumbering{arabic}
    %\linenumbers\renewcommand\thelinenumber{\color{black!50}\arabic{linenumber}}
            \input{0 - introduction/main.tex}
        \part{Research}
            \input{1 - low-noise PiC models/main.tex}
            \input{2 - kinetic component/main.tex}
            \input{3 - fluid component/main.tex}
            \input{4 - numerical implementation/main.tex}
        \part{Project Overview}
            \input{5 - research plan/main.tex}
            \input{6 - summary/main.tex}
    
    
    %\section{}
    \newpage
    \pagenumbering{gobble}
        \printbibliography


    \newpage
    \pagenumbering{roman}
    \appendix
        \part{Appendices}
            \input{8 - Hilbert complexes/main.tex}
            \input{9 - weak conservation proofs/main.tex}
\end{document}

\end{document}

        \part{Project Overview}
            \documentclass[12pt, a4paper]{report}

\documentclass[12pt, a4paper]{report}

\input{template/main.tex}

\title{\BA{Title in Progress...}}
\author{Boris Andrews}
\affil{Mathematical Institute, University of Oxford}
\date{\today}


\begin{document}
    \pagenumbering{gobble}
    \maketitle
    
    
    \begin{abstract}
        Magnetic confinement reactors---in particular tokamaks---offer one of the most promising options for achieving practical nuclear fusion, with the potential to provide virtually limitless, clean energy. The theoretical and numerical modeling of tokamak plasmas is simultaneously an essential component of effective reactor design, and a great research barrier. Tokamak operational conditions exhibit comparatively low Knudsen numbers. Kinetic effects, including kinetic waves and instabilities, Landau damping, bump-on-tail instabilities and more, are therefore highly influential in tokamak plasma dynamics. Purely fluid models are inherently incapable of capturing these effects, whereas the high dimensionality in purely kinetic models render them practically intractable for most relevant purposes.

        We consider a $\delta\!f$ decomposition model, with a macroscopic fluid background and microscopic kinetic correction, both fully coupled to each other. A similar manner of discretization is proposed to that used in the recent \texttt{STRUPHY} code \cite{Holderied_Possanner_Wang_2021, Holderied_2022, Li_et_al_2023} with a finite-element model for the background and a pseudo-particle/PiC model for the correction.

        The fluid background satisfies the full, non-linear, resistive, compressible, Hall MHD equations. \cite{Laakmann_Hu_Farrell_2022} introduces finite-element(-in-space) implicit timesteppers for the incompressible analogue to this system with structure-preserving (SP) properties in the ideal case, alongside parameter-robust preconditioners. We show that these timesteppers can derive from a finite-element-in-time (FET) (and finite-element-in-space) interpretation. The benefits of this reformulation are discussed, including the derivation of timesteppers that are higher order in time, and the quantifiable dissipative SP properties in the non-ideal, resistive case.
        
        We discuss possible options for extending this FET approach to timesteppers for the compressible case.

        The kinetic corrections satisfy linearized Boltzmann equations. Using a Lénard--Bernstein collision operator, these take Fokker--Planck-like forms \cite{Fokker_1914, Planck_1917} wherein pseudo-particles in the numerical model obey the neoclassical transport equations, with particle-independent Brownian drift terms. This offers a rigorous methodology for incorporating collisions into the particle transport model, without coupling the equations of motions for each particle.
        
        Works by Chen, Chacón et al. \cite{Chen_Chacón_Barnes_2011, Chacón_Chen_Barnes_2013, Chen_Chacón_2014, Chen_Chacón_2015} have developed structure-preserving particle pushers for neoclassical transport in the Vlasov equations, derived from Crank--Nicolson integrators. We show these too can can derive from a FET interpretation, similarly offering potential extensions to higher-order-in-time particle pushers. The FET formulation is used also to consider how the stochastic drift terms can be incorporated into the pushers. Stochastic gyrokinetic expansions are also discussed.

        Different options for the numerical implementation of these schemes are considered.

        Due to the efficacy of FET in the development of SP timesteppers for both the fluid and kinetic component, we hope this approach will prove effective in the future for developing SP timesteppers for the full hybrid model. We hope this will give us the opportunity to incorporate previously inaccessible kinetic effects into the highly effective, modern, finite-element MHD models.
    \end{abstract}
    
    
    \newpage
    \tableofcontents
    
    
    \newpage
    \pagenumbering{arabic}
    %\linenumbers\renewcommand\thelinenumber{\color{black!50}\arabic{linenumber}}
            \input{0 - introduction/main.tex}
        \part{Research}
            \input{1 - low-noise PiC models/main.tex}
            \input{2 - kinetic component/main.tex}
            \input{3 - fluid component/main.tex}
            \input{4 - numerical implementation/main.tex}
        \part{Project Overview}
            \input{5 - research plan/main.tex}
            \input{6 - summary/main.tex}
    
    
    %\section{}
    \newpage
    \pagenumbering{gobble}
        \printbibliography


    \newpage
    \pagenumbering{roman}
    \appendix
        \part{Appendices}
            \input{8 - Hilbert complexes/main.tex}
            \input{9 - weak conservation proofs/main.tex}
\end{document}


\title{\BA{Title in Progress...}}
\author{Boris Andrews}
\affil{Mathematical Institute, University of Oxford}
\date{\today}


\begin{document}
    \pagenumbering{gobble}
    \maketitle
    
    
    \begin{abstract}
        Magnetic confinement reactors---in particular tokamaks---offer one of the most promising options for achieving practical nuclear fusion, with the potential to provide virtually limitless, clean energy. The theoretical and numerical modeling of tokamak plasmas is simultaneously an essential component of effective reactor design, and a great research barrier. Tokamak operational conditions exhibit comparatively low Knudsen numbers. Kinetic effects, including kinetic waves and instabilities, Landau damping, bump-on-tail instabilities and more, are therefore highly influential in tokamak plasma dynamics. Purely fluid models are inherently incapable of capturing these effects, whereas the high dimensionality in purely kinetic models render them practically intractable for most relevant purposes.

        We consider a $\delta\!f$ decomposition model, with a macroscopic fluid background and microscopic kinetic correction, both fully coupled to each other. A similar manner of discretization is proposed to that used in the recent \texttt{STRUPHY} code \cite{Holderied_Possanner_Wang_2021, Holderied_2022, Li_et_al_2023} with a finite-element model for the background and a pseudo-particle/PiC model for the correction.

        The fluid background satisfies the full, non-linear, resistive, compressible, Hall MHD equations. \cite{Laakmann_Hu_Farrell_2022} introduces finite-element(-in-space) implicit timesteppers for the incompressible analogue to this system with structure-preserving (SP) properties in the ideal case, alongside parameter-robust preconditioners. We show that these timesteppers can derive from a finite-element-in-time (FET) (and finite-element-in-space) interpretation. The benefits of this reformulation are discussed, including the derivation of timesteppers that are higher order in time, and the quantifiable dissipative SP properties in the non-ideal, resistive case.
        
        We discuss possible options for extending this FET approach to timesteppers for the compressible case.

        The kinetic corrections satisfy linearized Boltzmann equations. Using a Lénard--Bernstein collision operator, these take Fokker--Planck-like forms \cite{Fokker_1914, Planck_1917} wherein pseudo-particles in the numerical model obey the neoclassical transport equations, with particle-independent Brownian drift terms. This offers a rigorous methodology for incorporating collisions into the particle transport model, without coupling the equations of motions for each particle.
        
        Works by Chen, Chacón et al. \cite{Chen_Chacón_Barnes_2011, Chacón_Chen_Barnes_2013, Chen_Chacón_2014, Chen_Chacón_2015} have developed structure-preserving particle pushers for neoclassical transport in the Vlasov equations, derived from Crank--Nicolson integrators. We show these too can can derive from a FET interpretation, similarly offering potential extensions to higher-order-in-time particle pushers. The FET formulation is used also to consider how the stochastic drift terms can be incorporated into the pushers. Stochastic gyrokinetic expansions are also discussed.

        Different options for the numerical implementation of these schemes are considered.

        Due to the efficacy of FET in the development of SP timesteppers for both the fluid and kinetic component, we hope this approach will prove effective in the future for developing SP timesteppers for the full hybrid model. We hope this will give us the opportunity to incorporate previously inaccessible kinetic effects into the highly effective, modern, finite-element MHD models.
    \end{abstract}
    
    
    \newpage
    \tableofcontents
    
    
    \newpage
    \pagenumbering{arabic}
    %\linenumbers\renewcommand\thelinenumber{\color{black!50}\arabic{linenumber}}
            \documentclass[12pt, a4paper]{report}

\input{template/main.tex}

\title{\BA{Title in Progress...}}
\author{Boris Andrews}
\affil{Mathematical Institute, University of Oxford}
\date{\today}


\begin{document}
    \pagenumbering{gobble}
    \maketitle
    
    
    \begin{abstract}
        Magnetic confinement reactors---in particular tokamaks---offer one of the most promising options for achieving practical nuclear fusion, with the potential to provide virtually limitless, clean energy. The theoretical and numerical modeling of tokamak plasmas is simultaneously an essential component of effective reactor design, and a great research barrier. Tokamak operational conditions exhibit comparatively low Knudsen numbers. Kinetic effects, including kinetic waves and instabilities, Landau damping, bump-on-tail instabilities and more, are therefore highly influential in tokamak plasma dynamics. Purely fluid models are inherently incapable of capturing these effects, whereas the high dimensionality in purely kinetic models render them practically intractable for most relevant purposes.

        We consider a $\delta\!f$ decomposition model, with a macroscopic fluid background and microscopic kinetic correction, both fully coupled to each other. A similar manner of discretization is proposed to that used in the recent \texttt{STRUPHY} code \cite{Holderied_Possanner_Wang_2021, Holderied_2022, Li_et_al_2023} with a finite-element model for the background and a pseudo-particle/PiC model for the correction.

        The fluid background satisfies the full, non-linear, resistive, compressible, Hall MHD equations. \cite{Laakmann_Hu_Farrell_2022} introduces finite-element(-in-space) implicit timesteppers for the incompressible analogue to this system with structure-preserving (SP) properties in the ideal case, alongside parameter-robust preconditioners. We show that these timesteppers can derive from a finite-element-in-time (FET) (and finite-element-in-space) interpretation. The benefits of this reformulation are discussed, including the derivation of timesteppers that are higher order in time, and the quantifiable dissipative SP properties in the non-ideal, resistive case.
        
        We discuss possible options for extending this FET approach to timesteppers for the compressible case.

        The kinetic corrections satisfy linearized Boltzmann equations. Using a Lénard--Bernstein collision operator, these take Fokker--Planck-like forms \cite{Fokker_1914, Planck_1917} wherein pseudo-particles in the numerical model obey the neoclassical transport equations, with particle-independent Brownian drift terms. This offers a rigorous methodology for incorporating collisions into the particle transport model, without coupling the equations of motions for each particle.
        
        Works by Chen, Chacón et al. \cite{Chen_Chacón_Barnes_2011, Chacón_Chen_Barnes_2013, Chen_Chacón_2014, Chen_Chacón_2015} have developed structure-preserving particle pushers for neoclassical transport in the Vlasov equations, derived from Crank--Nicolson integrators. We show these too can can derive from a FET interpretation, similarly offering potential extensions to higher-order-in-time particle pushers. The FET formulation is used also to consider how the stochastic drift terms can be incorporated into the pushers. Stochastic gyrokinetic expansions are also discussed.

        Different options for the numerical implementation of these schemes are considered.

        Due to the efficacy of FET in the development of SP timesteppers for both the fluid and kinetic component, we hope this approach will prove effective in the future for developing SP timesteppers for the full hybrid model. We hope this will give us the opportunity to incorporate previously inaccessible kinetic effects into the highly effective, modern, finite-element MHD models.
    \end{abstract}
    
    
    \newpage
    \tableofcontents
    
    
    \newpage
    \pagenumbering{arabic}
    %\linenumbers\renewcommand\thelinenumber{\color{black!50}\arabic{linenumber}}
            \input{0 - introduction/main.tex}
        \part{Research}
            \input{1 - low-noise PiC models/main.tex}
            \input{2 - kinetic component/main.tex}
            \input{3 - fluid component/main.tex}
            \input{4 - numerical implementation/main.tex}
        \part{Project Overview}
            \input{5 - research plan/main.tex}
            \input{6 - summary/main.tex}
    
    
    %\section{}
    \newpage
    \pagenumbering{gobble}
        \printbibliography


    \newpage
    \pagenumbering{roman}
    \appendix
        \part{Appendices}
            \input{8 - Hilbert complexes/main.tex}
            \input{9 - weak conservation proofs/main.tex}
\end{document}

        \part{Research}
            \documentclass[12pt, a4paper]{report}

\input{template/main.tex}

\title{\BA{Title in Progress...}}
\author{Boris Andrews}
\affil{Mathematical Institute, University of Oxford}
\date{\today}


\begin{document}
    \pagenumbering{gobble}
    \maketitle
    
    
    \begin{abstract}
        Magnetic confinement reactors---in particular tokamaks---offer one of the most promising options for achieving practical nuclear fusion, with the potential to provide virtually limitless, clean energy. The theoretical and numerical modeling of tokamak plasmas is simultaneously an essential component of effective reactor design, and a great research barrier. Tokamak operational conditions exhibit comparatively low Knudsen numbers. Kinetic effects, including kinetic waves and instabilities, Landau damping, bump-on-tail instabilities and more, are therefore highly influential in tokamak plasma dynamics. Purely fluid models are inherently incapable of capturing these effects, whereas the high dimensionality in purely kinetic models render them practically intractable for most relevant purposes.

        We consider a $\delta\!f$ decomposition model, with a macroscopic fluid background and microscopic kinetic correction, both fully coupled to each other. A similar manner of discretization is proposed to that used in the recent \texttt{STRUPHY} code \cite{Holderied_Possanner_Wang_2021, Holderied_2022, Li_et_al_2023} with a finite-element model for the background and a pseudo-particle/PiC model for the correction.

        The fluid background satisfies the full, non-linear, resistive, compressible, Hall MHD equations. \cite{Laakmann_Hu_Farrell_2022} introduces finite-element(-in-space) implicit timesteppers for the incompressible analogue to this system with structure-preserving (SP) properties in the ideal case, alongside parameter-robust preconditioners. We show that these timesteppers can derive from a finite-element-in-time (FET) (and finite-element-in-space) interpretation. The benefits of this reformulation are discussed, including the derivation of timesteppers that are higher order in time, and the quantifiable dissipative SP properties in the non-ideal, resistive case.
        
        We discuss possible options for extending this FET approach to timesteppers for the compressible case.

        The kinetic corrections satisfy linearized Boltzmann equations. Using a Lénard--Bernstein collision operator, these take Fokker--Planck-like forms \cite{Fokker_1914, Planck_1917} wherein pseudo-particles in the numerical model obey the neoclassical transport equations, with particle-independent Brownian drift terms. This offers a rigorous methodology for incorporating collisions into the particle transport model, without coupling the equations of motions for each particle.
        
        Works by Chen, Chacón et al. \cite{Chen_Chacón_Barnes_2011, Chacón_Chen_Barnes_2013, Chen_Chacón_2014, Chen_Chacón_2015} have developed structure-preserving particle pushers for neoclassical transport in the Vlasov equations, derived from Crank--Nicolson integrators. We show these too can can derive from a FET interpretation, similarly offering potential extensions to higher-order-in-time particle pushers. The FET formulation is used also to consider how the stochastic drift terms can be incorporated into the pushers. Stochastic gyrokinetic expansions are also discussed.

        Different options for the numerical implementation of these schemes are considered.

        Due to the efficacy of FET in the development of SP timesteppers for both the fluid and kinetic component, we hope this approach will prove effective in the future for developing SP timesteppers for the full hybrid model. We hope this will give us the opportunity to incorporate previously inaccessible kinetic effects into the highly effective, modern, finite-element MHD models.
    \end{abstract}
    
    
    \newpage
    \tableofcontents
    
    
    \newpage
    \pagenumbering{arabic}
    %\linenumbers\renewcommand\thelinenumber{\color{black!50}\arabic{linenumber}}
            \input{0 - introduction/main.tex}
        \part{Research}
            \input{1 - low-noise PiC models/main.tex}
            \input{2 - kinetic component/main.tex}
            \input{3 - fluid component/main.tex}
            \input{4 - numerical implementation/main.tex}
        \part{Project Overview}
            \input{5 - research plan/main.tex}
            \input{6 - summary/main.tex}
    
    
    %\section{}
    \newpage
    \pagenumbering{gobble}
        \printbibliography


    \newpage
    \pagenumbering{roman}
    \appendix
        \part{Appendices}
            \input{8 - Hilbert complexes/main.tex}
            \input{9 - weak conservation proofs/main.tex}
\end{document}

            \documentclass[12pt, a4paper]{report}

\input{template/main.tex}

\title{\BA{Title in Progress...}}
\author{Boris Andrews}
\affil{Mathematical Institute, University of Oxford}
\date{\today}


\begin{document}
    \pagenumbering{gobble}
    \maketitle
    
    
    \begin{abstract}
        Magnetic confinement reactors---in particular tokamaks---offer one of the most promising options for achieving practical nuclear fusion, with the potential to provide virtually limitless, clean energy. The theoretical and numerical modeling of tokamak plasmas is simultaneously an essential component of effective reactor design, and a great research barrier. Tokamak operational conditions exhibit comparatively low Knudsen numbers. Kinetic effects, including kinetic waves and instabilities, Landau damping, bump-on-tail instabilities and more, are therefore highly influential in tokamak plasma dynamics. Purely fluid models are inherently incapable of capturing these effects, whereas the high dimensionality in purely kinetic models render them practically intractable for most relevant purposes.

        We consider a $\delta\!f$ decomposition model, with a macroscopic fluid background and microscopic kinetic correction, both fully coupled to each other. A similar manner of discretization is proposed to that used in the recent \texttt{STRUPHY} code \cite{Holderied_Possanner_Wang_2021, Holderied_2022, Li_et_al_2023} with a finite-element model for the background and a pseudo-particle/PiC model for the correction.

        The fluid background satisfies the full, non-linear, resistive, compressible, Hall MHD equations. \cite{Laakmann_Hu_Farrell_2022} introduces finite-element(-in-space) implicit timesteppers for the incompressible analogue to this system with structure-preserving (SP) properties in the ideal case, alongside parameter-robust preconditioners. We show that these timesteppers can derive from a finite-element-in-time (FET) (and finite-element-in-space) interpretation. The benefits of this reformulation are discussed, including the derivation of timesteppers that are higher order in time, and the quantifiable dissipative SP properties in the non-ideal, resistive case.
        
        We discuss possible options for extending this FET approach to timesteppers for the compressible case.

        The kinetic corrections satisfy linearized Boltzmann equations. Using a Lénard--Bernstein collision operator, these take Fokker--Planck-like forms \cite{Fokker_1914, Planck_1917} wherein pseudo-particles in the numerical model obey the neoclassical transport equations, with particle-independent Brownian drift terms. This offers a rigorous methodology for incorporating collisions into the particle transport model, without coupling the equations of motions for each particle.
        
        Works by Chen, Chacón et al. \cite{Chen_Chacón_Barnes_2011, Chacón_Chen_Barnes_2013, Chen_Chacón_2014, Chen_Chacón_2015} have developed structure-preserving particle pushers for neoclassical transport in the Vlasov equations, derived from Crank--Nicolson integrators. We show these too can can derive from a FET interpretation, similarly offering potential extensions to higher-order-in-time particle pushers. The FET formulation is used also to consider how the stochastic drift terms can be incorporated into the pushers. Stochastic gyrokinetic expansions are also discussed.

        Different options for the numerical implementation of these schemes are considered.

        Due to the efficacy of FET in the development of SP timesteppers for both the fluid and kinetic component, we hope this approach will prove effective in the future for developing SP timesteppers for the full hybrid model. We hope this will give us the opportunity to incorporate previously inaccessible kinetic effects into the highly effective, modern, finite-element MHD models.
    \end{abstract}
    
    
    \newpage
    \tableofcontents
    
    
    \newpage
    \pagenumbering{arabic}
    %\linenumbers\renewcommand\thelinenumber{\color{black!50}\arabic{linenumber}}
            \input{0 - introduction/main.tex}
        \part{Research}
            \input{1 - low-noise PiC models/main.tex}
            \input{2 - kinetic component/main.tex}
            \input{3 - fluid component/main.tex}
            \input{4 - numerical implementation/main.tex}
        \part{Project Overview}
            \input{5 - research plan/main.tex}
            \input{6 - summary/main.tex}
    
    
    %\section{}
    \newpage
    \pagenumbering{gobble}
        \printbibliography


    \newpage
    \pagenumbering{roman}
    \appendix
        \part{Appendices}
            \input{8 - Hilbert complexes/main.tex}
            \input{9 - weak conservation proofs/main.tex}
\end{document}

            \documentclass[12pt, a4paper]{report}

\input{template/main.tex}

\title{\BA{Title in Progress...}}
\author{Boris Andrews}
\affil{Mathematical Institute, University of Oxford}
\date{\today}


\begin{document}
    \pagenumbering{gobble}
    \maketitle
    
    
    \begin{abstract}
        Magnetic confinement reactors---in particular tokamaks---offer one of the most promising options for achieving practical nuclear fusion, with the potential to provide virtually limitless, clean energy. The theoretical and numerical modeling of tokamak plasmas is simultaneously an essential component of effective reactor design, and a great research barrier. Tokamak operational conditions exhibit comparatively low Knudsen numbers. Kinetic effects, including kinetic waves and instabilities, Landau damping, bump-on-tail instabilities and more, are therefore highly influential in tokamak plasma dynamics. Purely fluid models are inherently incapable of capturing these effects, whereas the high dimensionality in purely kinetic models render them practically intractable for most relevant purposes.

        We consider a $\delta\!f$ decomposition model, with a macroscopic fluid background and microscopic kinetic correction, both fully coupled to each other. A similar manner of discretization is proposed to that used in the recent \texttt{STRUPHY} code \cite{Holderied_Possanner_Wang_2021, Holderied_2022, Li_et_al_2023} with a finite-element model for the background and a pseudo-particle/PiC model for the correction.

        The fluid background satisfies the full, non-linear, resistive, compressible, Hall MHD equations. \cite{Laakmann_Hu_Farrell_2022} introduces finite-element(-in-space) implicit timesteppers for the incompressible analogue to this system with structure-preserving (SP) properties in the ideal case, alongside parameter-robust preconditioners. We show that these timesteppers can derive from a finite-element-in-time (FET) (and finite-element-in-space) interpretation. The benefits of this reformulation are discussed, including the derivation of timesteppers that are higher order in time, and the quantifiable dissipative SP properties in the non-ideal, resistive case.
        
        We discuss possible options for extending this FET approach to timesteppers for the compressible case.

        The kinetic corrections satisfy linearized Boltzmann equations. Using a Lénard--Bernstein collision operator, these take Fokker--Planck-like forms \cite{Fokker_1914, Planck_1917} wherein pseudo-particles in the numerical model obey the neoclassical transport equations, with particle-independent Brownian drift terms. This offers a rigorous methodology for incorporating collisions into the particle transport model, without coupling the equations of motions for each particle.
        
        Works by Chen, Chacón et al. \cite{Chen_Chacón_Barnes_2011, Chacón_Chen_Barnes_2013, Chen_Chacón_2014, Chen_Chacón_2015} have developed structure-preserving particle pushers for neoclassical transport in the Vlasov equations, derived from Crank--Nicolson integrators. We show these too can can derive from a FET interpretation, similarly offering potential extensions to higher-order-in-time particle pushers. The FET formulation is used also to consider how the stochastic drift terms can be incorporated into the pushers. Stochastic gyrokinetic expansions are also discussed.

        Different options for the numerical implementation of these schemes are considered.

        Due to the efficacy of FET in the development of SP timesteppers for both the fluid and kinetic component, we hope this approach will prove effective in the future for developing SP timesteppers for the full hybrid model. We hope this will give us the opportunity to incorporate previously inaccessible kinetic effects into the highly effective, modern, finite-element MHD models.
    \end{abstract}
    
    
    \newpage
    \tableofcontents
    
    
    \newpage
    \pagenumbering{arabic}
    %\linenumbers\renewcommand\thelinenumber{\color{black!50}\arabic{linenumber}}
            \input{0 - introduction/main.tex}
        \part{Research}
            \input{1 - low-noise PiC models/main.tex}
            \input{2 - kinetic component/main.tex}
            \input{3 - fluid component/main.tex}
            \input{4 - numerical implementation/main.tex}
        \part{Project Overview}
            \input{5 - research plan/main.tex}
            \input{6 - summary/main.tex}
    
    
    %\section{}
    \newpage
    \pagenumbering{gobble}
        \printbibliography


    \newpage
    \pagenumbering{roman}
    \appendix
        \part{Appendices}
            \input{8 - Hilbert complexes/main.tex}
            \input{9 - weak conservation proofs/main.tex}
\end{document}

            \documentclass[12pt, a4paper]{report}

\input{template/main.tex}

\title{\BA{Title in Progress...}}
\author{Boris Andrews}
\affil{Mathematical Institute, University of Oxford}
\date{\today}


\begin{document}
    \pagenumbering{gobble}
    \maketitle
    
    
    \begin{abstract}
        Magnetic confinement reactors---in particular tokamaks---offer one of the most promising options for achieving practical nuclear fusion, with the potential to provide virtually limitless, clean energy. The theoretical and numerical modeling of tokamak plasmas is simultaneously an essential component of effective reactor design, and a great research barrier. Tokamak operational conditions exhibit comparatively low Knudsen numbers. Kinetic effects, including kinetic waves and instabilities, Landau damping, bump-on-tail instabilities and more, are therefore highly influential in tokamak plasma dynamics. Purely fluid models are inherently incapable of capturing these effects, whereas the high dimensionality in purely kinetic models render them practically intractable for most relevant purposes.

        We consider a $\delta\!f$ decomposition model, with a macroscopic fluid background and microscopic kinetic correction, both fully coupled to each other. A similar manner of discretization is proposed to that used in the recent \texttt{STRUPHY} code \cite{Holderied_Possanner_Wang_2021, Holderied_2022, Li_et_al_2023} with a finite-element model for the background and a pseudo-particle/PiC model for the correction.

        The fluid background satisfies the full, non-linear, resistive, compressible, Hall MHD equations. \cite{Laakmann_Hu_Farrell_2022} introduces finite-element(-in-space) implicit timesteppers for the incompressible analogue to this system with structure-preserving (SP) properties in the ideal case, alongside parameter-robust preconditioners. We show that these timesteppers can derive from a finite-element-in-time (FET) (and finite-element-in-space) interpretation. The benefits of this reformulation are discussed, including the derivation of timesteppers that are higher order in time, and the quantifiable dissipative SP properties in the non-ideal, resistive case.
        
        We discuss possible options for extending this FET approach to timesteppers for the compressible case.

        The kinetic corrections satisfy linearized Boltzmann equations. Using a Lénard--Bernstein collision operator, these take Fokker--Planck-like forms \cite{Fokker_1914, Planck_1917} wherein pseudo-particles in the numerical model obey the neoclassical transport equations, with particle-independent Brownian drift terms. This offers a rigorous methodology for incorporating collisions into the particle transport model, without coupling the equations of motions for each particle.
        
        Works by Chen, Chacón et al. \cite{Chen_Chacón_Barnes_2011, Chacón_Chen_Barnes_2013, Chen_Chacón_2014, Chen_Chacón_2015} have developed structure-preserving particle pushers for neoclassical transport in the Vlasov equations, derived from Crank--Nicolson integrators. We show these too can can derive from a FET interpretation, similarly offering potential extensions to higher-order-in-time particle pushers. The FET formulation is used also to consider how the stochastic drift terms can be incorporated into the pushers. Stochastic gyrokinetic expansions are also discussed.

        Different options for the numerical implementation of these schemes are considered.

        Due to the efficacy of FET in the development of SP timesteppers for both the fluid and kinetic component, we hope this approach will prove effective in the future for developing SP timesteppers for the full hybrid model. We hope this will give us the opportunity to incorporate previously inaccessible kinetic effects into the highly effective, modern, finite-element MHD models.
    \end{abstract}
    
    
    \newpage
    \tableofcontents
    
    
    \newpage
    \pagenumbering{arabic}
    %\linenumbers\renewcommand\thelinenumber{\color{black!50}\arabic{linenumber}}
            \input{0 - introduction/main.tex}
        \part{Research}
            \input{1 - low-noise PiC models/main.tex}
            \input{2 - kinetic component/main.tex}
            \input{3 - fluid component/main.tex}
            \input{4 - numerical implementation/main.tex}
        \part{Project Overview}
            \input{5 - research plan/main.tex}
            \input{6 - summary/main.tex}
    
    
    %\section{}
    \newpage
    \pagenumbering{gobble}
        \printbibliography


    \newpage
    \pagenumbering{roman}
    \appendix
        \part{Appendices}
            \input{8 - Hilbert complexes/main.tex}
            \input{9 - weak conservation proofs/main.tex}
\end{document}

        \part{Project Overview}
            \documentclass[12pt, a4paper]{report}

\input{template/main.tex}

\title{\BA{Title in Progress...}}
\author{Boris Andrews}
\affil{Mathematical Institute, University of Oxford}
\date{\today}


\begin{document}
    \pagenumbering{gobble}
    \maketitle
    
    
    \begin{abstract}
        Magnetic confinement reactors---in particular tokamaks---offer one of the most promising options for achieving practical nuclear fusion, with the potential to provide virtually limitless, clean energy. The theoretical and numerical modeling of tokamak plasmas is simultaneously an essential component of effective reactor design, and a great research barrier. Tokamak operational conditions exhibit comparatively low Knudsen numbers. Kinetic effects, including kinetic waves and instabilities, Landau damping, bump-on-tail instabilities and more, are therefore highly influential in tokamak plasma dynamics. Purely fluid models are inherently incapable of capturing these effects, whereas the high dimensionality in purely kinetic models render them practically intractable for most relevant purposes.

        We consider a $\delta\!f$ decomposition model, with a macroscopic fluid background and microscopic kinetic correction, both fully coupled to each other. A similar manner of discretization is proposed to that used in the recent \texttt{STRUPHY} code \cite{Holderied_Possanner_Wang_2021, Holderied_2022, Li_et_al_2023} with a finite-element model for the background and a pseudo-particle/PiC model for the correction.

        The fluid background satisfies the full, non-linear, resistive, compressible, Hall MHD equations. \cite{Laakmann_Hu_Farrell_2022} introduces finite-element(-in-space) implicit timesteppers for the incompressible analogue to this system with structure-preserving (SP) properties in the ideal case, alongside parameter-robust preconditioners. We show that these timesteppers can derive from a finite-element-in-time (FET) (and finite-element-in-space) interpretation. The benefits of this reformulation are discussed, including the derivation of timesteppers that are higher order in time, and the quantifiable dissipative SP properties in the non-ideal, resistive case.
        
        We discuss possible options for extending this FET approach to timesteppers for the compressible case.

        The kinetic corrections satisfy linearized Boltzmann equations. Using a Lénard--Bernstein collision operator, these take Fokker--Planck-like forms \cite{Fokker_1914, Planck_1917} wherein pseudo-particles in the numerical model obey the neoclassical transport equations, with particle-independent Brownian drift terms. This offers a rigorous methodology for incorporating collisions into the particle transport model, without coupling the equations of motions for each particle.
        
        Works by Chen, Chacón et al. \cite{Chen_Chacón_Barnes_2011, Chacón_Chen_Barnes_2013, Chen_Chacón_2014, Chen_Chacón_2015} have developed structure-preserving particle pushers for neoclassical transport in the Vlasov equations, derived from Crank--Nicolson integrators. We show these too can can derive from a FET interpretation, similarly offering potential extensions to higher-order-in-time particle pushers. The FET formulation is used also to consider how the stochastic drift terms can be incorporated into the pushers. Stochastic gyrokinetic expansions are also discussed.

        Different options for the numerical implementation of these schemes are considered.

        Due to the efficacy of FET in the development of SP timesteppers for both the fluid and kinetic component, we hope this approach will prove effective in the future for developing SP timesteppers for the full hybrid model. We hope this will give us the opportunity to incorporate previously inaccessible kinetic effects into the highly effective, modern, finite-element MHD models.
    \end{abstract}
    
    
    \newpage
    \tableofcontents
    
    
    \newpage
    \pagenumbering{arabic}
    %\linenumbers\renewcommand\thelinenumber{\color{black!50}\arabic{linenumber}}
            \input{0 - introduction/main.tex}
        \part{Research}
            \input{1 - low-noise PiC models/main.tex}
            \input{2 - kinetic component/main.tex}
            \input{3 - fluid component/main.tex}
            \input{4 - numerical implementation/main.tex}
        \part{Project Overview}
            \input{5 - research plan/main.tex}
            \input{6 - summary/main.tex}
    
    
    %\section{}
    \newpage
    \pagenumbering{gobble}
        \printbibliography


    \newpage
    \pagenumbering{roman}
    \appendix
        \part{Appendices}
            \input{8 - Hilbert complexes/main.tex}
            \input{9 - weak conservation proofs/main.tex}
\end{document}

            \documentclass[12pt, a4paper]{report}

\input{template/main.tex}

\title{\BA{Title in Progress...}}
\author{Boris Andrews}
\affil{Mathematical Institute, University of Oxford}
\date{\today}


\begin{document}
    \pagenumbering{gobble}
    \maketitle
    
    
    \begin{abstract}
        Magnetic confinement reactors---in particular tokamaks---offer one of the most promising options for achieving practical nuclear fusion, with the potential to provide virtually limitless, clean energy. The theoretical and numerical modeling of tokamak plasmas is simultaneously an essential component of effective reactor design, and a great research barrier. Tokamak operational conditions exhibit comparatively low Knudsen numbers. Kinetic effects, including kinetic waves and instabilities, Landau damping, bump-on-tail instabilities and more, are therefore highly influential in tokamak plasma dynamics. Purely fluid models are inherently incapable of capturing these effects, whereas the high dimensionality in purely kinetic models render them practically intractable for most relevant purposes.

        We consider a $\delta\!f$ decomposition model, with a macroscopic fluid background and microscopic kinetic correction, both fully coupled to each other. A similar manner of discretization is proposed to that used in the recent \texttt{STRUPHY} code \cite{Holderied_Possanner_Wang_2021, Holderied_2022, Li_et_al_2023} with a finite-element model for the background and a pseudo-particle/PiC model for the correction.

        The fluid background satisfies the full, non-linear, resistive, compressible, Hall MHD equations. \cite{Laakmann_Hu_Farrell_2022} introduces finite-element(-in-space) implicit timesteppers for the incompressible analogue to this system with structure-preserving (SP) properties in the ideal case, alongside parameter-robust preconditioners. We show that these timesteppers can derive from a finite-element-in-time (FET) (and finite-element-in-space) interpretation. The benefits of this reformulation are discussed, including the derivation of timesteppers that are higher order in time, and the quantifiable dissipative SP properties in the non-ideal, resistive case.
        
        We discuss possible options for extending this FET approach to timesteppers for the compressible case.

        The kinetic corrections satisfy linearized Boltzmann equations. Using a Lénard--Bernstein collision operator, these take Fokker--Planck-like forms \cite{Fokker_1914, Planck_1917} wherein pseudo-particles in the numerical model obey the neoclassical transport equations, with particle-independent Brownian drift terms. This offers a rigorous methodology for incorporating collisions into the particle transport model, without coupling the equations of motions for each particle.
        
        Works by Chen, Chacón et al. \cite{Chen_Chacón_Barnes_2011, Chacón_Chen_Barnes_2013, Chen_Chacón_2014, Chen_Chacón_2015} have developed structure-preserving particle pushers for neoclassical transport in the Vlasov equations, derived from Crank--Nicolson integrators. We show these too can can derive from a FET interpretation, similarly offering potential extensions to higher-order-in-time particle pushers. The FET formulation is used also to consider how the stochastic drift terms can be incorporated into the pushers. Stochastic gyrokinetic expansions are also discussed.

        Different options for the numerical implementation of these schemes are considered.

        Due to the efficacy of FET in the development of SP timesteppers for both the fluid and kinetic component, we hope this approach will prove effective in the future for developing SP timesteppers for the full hybrid model. We hope this will give us the opportunity to incorporate previously inaccessible kinetic effects into the highly effective, modern, finite-element MHD models.
    \end{abstract}
    
    
    \newpage
    \tableofcontents
    
    
    \newpage
    \pagenumbering{arabic}
    %\linenumbers\renewcommand\thelinenumber{\color{black!50}\arabic{linenumber}}
            \input{0 - introduction/main.tex}
        \part{Research}
            \input{1 - low-noise PiC models/main.tex}
            \input{2 - kinetic component/main.tex}
            \input{3 - fluid component/main.tex}
            \input{4 - numerical implementation/main.tex}
        \part{Project Overview}
            \input{5 - research plan/main.tex}
            \input{6 - summary/main.tex}
    
    
    %\section{}
    \newpage
    \pagenumbering{gobble}
        \printbibliography


    \newpage
    \pagenumbering{roman}
    \appendix
        \part{Appendices}
            \input{8 - Hilbert complexes/main.tex}
            \input{9 - weak conservation proofs/main.tex}
\end{document}

    
    
    %\section{}
    \newpage
    \pagenumbering{gobble}
        \printbibliography


    \newpage
    \pagenumbering{roman}
    \appendix
        \part{Appendices}
            \documentclass[12pt, a4paper]{report}

\input{template/main.tex}

\title{\BA{Title in Progress...}}
\author{Boris Andrews}
\affil{Mathematical Institute, University of Oxford}
\date{\today}


\begin{document}
    \pagenumbering{gobble}
    \maketitle
    
    
    \begin{abstract}
        Magnetic confinement reactors---in particular tokamaks---offer one of the most promising options for achieving practical nuclear fusion, with the potential to provide virtually limitless, clean energy. The theoretical and numerical modeling of tokamak plasmas is simultaneously an essential component of effective reactor design, and a great research barrier. Tokamak operational conditions exhibit comparatively low Knudsen numbers. Kinetic effects, including kinetic waves and instabilities, Landau damping, bump-on-tail instabilities and more, are therefore highly influential in tokamak plasma dynamics. Purely fluid models are inherently incapable of capturing these effects, whereas the high dimensionality in purely kinetic models render them practically intractable for most relevant purposes.

        We consider a $\delta\!f$ decomposition model, with a macroscopic fluid background and microscopic kinetic correction, both fully coupled to each other. A similar manner of discretization is proposed to that used in the recent \texttt{STRUPHY} code \cite{Holderied_Possanner_Wang_2021, Holderied_2022, Li_et_al_2023} with a finite-element model for the background and a pseudo-particle/PiC model for the correction.

        The fluid background satisfies the full, non-linear, resistive, compressible, Hall MHD equations. \cite{Laakmann_Hu_Farrell_2022} introduces finite-element(-in-space) implicit timesteppers for the incompressible analogue to this system with structure-preserving (SP) properties in the ideal case, alongside parameter-robust preconditioners. We show that these timesteppers can derive from a finite-element-in-time (FET) (and finite-element-in-space) interpretation. The benefits of this reformulation are discussed, including the derivation of timesteppers that are higher order in time, and the quantifiable dissipative SP properties in the non-ideal, resistive case.
        
        We discuss possible options for extending this FET approach to timesteppers for the compressible case.

        The kinetic corrections satisfy linearized Boltzmann equations. Using a Lénard--Bernstein collision operator, these take Fokker--Planck-like forms \cite{Fokker_1914, Planck_1917} wherein pseudo-particles in the numerical model obey the neoclassical transport equations, with particle-independent Brownian drift terms. This offers a rigorous methodology for incorporating collisions into the particle transport model, without coupling the equations of motions for each particle.
        
        Works by Chen, Chacón et al. \cite{Chen_Chacón_Barnes_2011, Chacón_Chen_Barnes_2013, Chen_Chacón_2014, Chen_Chacón_2015} have developed structure-preserving particle pushers for neoclassical transport in the Vlasov equations, derived from Crank--Nicolson integrators. We show these too can can derive from a FET interpretation, similarly offering potential extensions to higher-order-in-time particle pushers. The FET formulation is used also to consider how the stochastic drift terms can be incorporated into the pushers. Stochastic gyrokinetic expansions are also discussed.

        Different options for the numerical implementation of these schemes are considered.

        Due to the efficacy of FET in the development of SP timesteppers for both the fluid and kinetic component, we hope this approach will prove effective in the future for developing SP timesteppers for the full hybrid model. We hope this will give us the opportunity to incorporate previously inaccessible kinetic effects into the highly effective, modern, finite-element MHD models.
    \end{abstract}
    
    
    \newpage
    \tableofcontents
    
    
    \newpage
    \pagenumbering{arabic}
    %\linenumbers\renewcommand\thelinenumber{\color{black!50}\arabic{linenumber}}
            \input{0 - introduction/main.tex}
        \part{Research}
            \input{1 - low-noise PiC models/main.tex}
            \input{2 - kinetic component/main.tex}
            \input{3 - fluid component/main.tex}
            \input{4 - numerical implementation/main.tex}
        \part{Project Overview}
            \input{5 - research plan/main.tex}
            \input{6 - summary/main.tex}
    
    
    %\section{}
    \newpage
    \pagenumbering{gobble}
        \printbibliography


    \newpage
    \pagenumbering{roman}
    \appendix
        \part{Appendices}
            \input{8 - Hilbert complexes/main.tex}
            \input{9 - weak conservation proofs/main.tex}
\end{document}

            \documentclass[12pt, a4paper]{report}

\input{template/main.tex}

\title{\BA{Title in Progress...}}
\author{Boris Andrews}
\affil{Mathematical Institute, University of Oxford}
\date{\today}


\begin{document}
    \pagenumbering{gobble}
    \maketitle
    
    
    \begin{abstract}
        Magnetic confinement reactors---in particular tokamaks---offer one of the most promising options for achieving practical nuclear fusion, with the potential to provide virtually limitless, clean energy. The theoretical and numerical modeling of tokamak plasmas is simultaneously an essential component of effective reactor design, and a great research barrier. Tokamak operational conditions exhibit comparatively low Knudsen numbers. Kinetic effects, including kinetic waves and instabilities, Landau damping, bump-on-tail instabilities and more, are therefore highly influential in tokamak plasma dynamics. Purely fluid models are inherently incapable of capturing these effects, whereas the high dimensionality in purely kinetic models render them practically intractable for most relevant purposes.

        We consider a $\delta\!f$ decomposition model, with a macroscopic fluid background and microscopic kinetic correction, both fully coupled to each other. A similar manner of discretization is proposed to that used in the recent \texttt{STRUPHY} code \cite{Holderied_Possanner_Wang_2021, Holderied_2022, Li_et_al_2023} with a finite-element model for the background and a pseudo-particle/PiC model for the correction.

        The fluid background satisfies the full, non-linear, resistive, compressible, Hall MHD equations. \cite{Laakmann_Hu_Farrell_2022} introduces finite-element(-in-space) implicit timesteppers for the incompressible analogue to this system with structure-preserving (SP) properties in the ideal case, alongside parameter-robust preconditioners. We show that these timesteppers can derive from a finite-element-in-time (FET) (and finite-element-in-space) interpretation. The benefits of this reformulation are discussed, including the derivation of timesteppers that are higher order in time, and the quantifiable dissipative SP properties in the non-ideal, resistive case.
        
        We discuss possible options for extending this FET approach to timesteppers for the compressible case.

        The kinetic corrections satisfy linearized Boltzmann equations. Using a Lénard--Bernstein collision operator, these take Fokker--Planck-like forms \cite{Fokker_1914, Planck_1917} wherein pseudo-particles in the numerical model obey the neoclassical transport equations, with particle-independent Brownian drift terms. This offers a rigorous methodology for incorporating collisions into the particle transport model, without coupling the equations of motions for each particle.
        
        Works by Chen, Chacón et al. \cite{Chen_Chacón_Barnes_2011, Chacón_Chen_Barnes_2013, Chen_Chacón_2014, Chen_Chacón_2015} have developed structure-preserving particle pushers for neoclassical transport in the Vlasov equations, derived from Crank--Nicolson integrators. We show these too can can derive from a FET interpretation, similarly offering potential extensions to higher-order-in-time particle pushers. The FET formulation is used also to consider how the stochastic drift terms can be incorporated into the pushers. Stochastic gyrokinetic expansions are also discussed.

        Different options for the numerical implementation of these schemes are considered.

        Due to the efficacy of FET in the development of SP timesteppers for both the fluid and kinetic component, we hope this approach will prove effective in the future for developing SP timesteppers for the full hybrid model. We hope this will give us the opportunity to incorporate previously inaccessible kinetic effects into the highly effective, modern, finite-element MHD models.
    \end{abstract}
    
    
    \newpage
    \tableofcontents
    
    
    \newpage
    \pagenumbering{arabic}
    %\linenumbers\renewcommand\thelinenumber{\color{black!50}\arabic{linenumber}}
            \input{0 - introduction/main.tex}
        \part{Research}
            \input{1 - low-noise PiC models/main.tex}
            \input{2 - kinetic component/main.tex}
            \input{3 - fluid component/main.tex}
            \input{4 - numerical implementation/main.tex}
        \part{Project Overview}
            \input{5 - research plan/main.tex}
            \input{6 - summary/main.tex}
    
    
    %\section{}
    \newpage
    \pagenumbering{gobble}
        \printbibliography


    \newpage
    \pagenumbering{roman}
    \appendix
        \part{Appendices}
            \input{8 - Hilbert complexes/main.tex}
            \input{9 - weak conservation proofs/main.tex}
\end{document}

\end{document}

            \documentclass[12pt, a4paper]{report}

\documentclass[12pt, a4paper]{report}

\input{template/main.tex}

\title{\BA{Title in Progress...}}
\author{Boris Andrews}
\affil{Mathematical Institute, University of Oxford}
\date{\today}


\begin{document}
    \pagenumbering{gobble}
    \maketitle
    
    
    \begin{abstract}
        Magnetic confinement reactors---in particular tokamaks---offer one of the most promising options for achieving practical nuclear fusion, with the potential to provide virtually limitless, clean energy. The theoretical and numerical modeling of tokamak plasmas is simultaneously an essential component of effective reactor design, and a great research barrier. Tokamak operational conditions exhibit comparatively low Knudsen numbers. Kinetic effects, including kinetic waves and instabilities, Landau damping, bump-on-tail instabilities and more, are therefore highly influential in tokamak plasma dynamics. Purely fluid models are inherently incapable of capturing these effects, whereas the high dimensionality in purely kinetic models render them practically intractable for most relevant purposes.

        We consider a $\delta\!f$ decomposition model, with a macroscopic fluid background and microscopic kinetic correction, both fully coupled to each other. A similar manner of discretization is proposed to that used in the recent \texttt{STRUPHY} code \cite{Holderied_Possanner_Wang_2021, Holderied_2022, Li_et_al_2023} with a finite-element model for the background and a pseudo-particle/PiC model for the correction.

        The fluid background satisfies the full, non-linear, resistive, compressible, Hall MHD equations. \cite{Laakmann_Hu_Farrell_2022} introduces finite-element(-in-space) implicit timesteppers for the incompressible analogue to this system with structure-preserving (SP) properties in the ideal case, alongside parameter-robust preconditioners. We show that these timesteppers can derive from a finite-element-in-time (FET) (and finite-element-in-space) interpretation. The benefits of this reformulation are discussed, including the derivation of timesteppers that are higher order in time, and the quantifiable dissipative SP properties in the non-ideal, resistive case.
        
        We discuss possible options for extending this FET approach to timesteppers for the compressible case.

        The kinetic corrections satisfy linearized Boltzmann equations. Using a Lénard--Bernstein collision operator, these take Fokker--Planck-like forms \cite{Fokker_1914, Planck_1917} wherein pseudo-particles in the numerical model obey the neoclassical transport equations, with particle-independent Brownian drift terms. This offers a rigorous methodology for incorporating collisions into the particle transport model, without coupling the equations of motions for each particle.
        
        Works by Chen, Chacón et al. \cite{Chen_Chacón_Barnes_2011, Chacón_Chen_Barnes_2013, Chen_Chacón_2014, Chen_Chacón_2015} have developed structure-preserving particle pushers for neoclassical transport in the Vlasov equations, derived from Crank--Nicolson integrators. We show these too can can derive from a FET interpretation, similarly offering potential extensions to higher-order-in-time particle pushers. The FET formulation is used also to consider how the stochastic drift terms can be incorporated into the pushers. Stochastic gyrokinetic expansions are also discussed.

        Different options for the numerical implementation of these schemes are considered.

        Due to the efficacy of FET in the development of SP timesteppers for both the fluid and kinetic component, we hope this approach will prove effective in the future for developing SP timesteppers for the full hybrid model. We hope this will give us the opportunity to incorporate previously inaccessible kinetic effects into the highly effective, modern, finite-element MHD models.
    \end{abstract}
    
    
    \newpage
    \tableofcontents
    
    
    \newpage
    \pagenumbering{arabic}
    %\linenumbers\renewcommand\thelinenumber{\color{black!50}\arabic{linenumber}}
            \input{0 - introduction/main.tex}
        \part{Research}
            \input{1 - low-noise PiC models/main.tex}
            \input{2 - kinetic component/main.tex}
            \input{3 - fluid component/main.tex}
            \input{4 - numerical implementation/main.tex}
        \part{Project Overview}
            \input{5 - research plan/main.tex}
            \input{6 - summary/main.tex}
    
    
    %\section{}
    \newpage
    \pagenumbering{gobble}
        \printbibliography


    \newpage
    \pagenumbering{roman}
    \appendix
        \part{Appendices}
            \input{8 - Hilbert complexes/main.tex}
            \input{9 - weak conservation proofs/main.tex}
\end{document}


\title{\BA{Title in Progress...}}
\author{Boris Andrews}
\affil{Mathematical Institute, University of Oxford}
\date{\today}


\begin{document}
    \pagenumbering{gobble}
    \maketitle
    
    
    \begin{abstract}
        Magnetic confinement reactors---in particular tokamaks---offer one of the most promising options for achieving practical nuclear fusion, with the potential to provide virtually limitless, clean energy. The theoretical and numerical modeling of tokamak plasmas is simultaneously an essential component of effective reactor design, and a great research barrier. Tokamak operational conditions exhibit comparatively low Knudsen numbers. Kinetic effects, including kinetic waves and instabilities, Landau damping, bump-on-tail instabilities and more, are therefore highly influential in tokamak plasma dynamics. Purely fluid models are inherently incapable of capturing these effects, whereas the high dimensionality in purely kinetic models render them practically intractable for most relevant purposes.

        We consider a $\delta\!f$ decomposition model, with a macroscopic fluid background and microscopic kinetic correction, both fully coupled to each other. A similar manner of discretization is proposed to that used in the recent \texttt{STRUPHY} code \cite{Holderied_Possanner_Wang_2021, Holderied_2022, Li_et_al_2023} with a finite-element model for the background and a pseudo-particle/PiC model for the correction.

        The fluid background satisfies the full, non-linear, resistive, compressible, Hall MHD equations. \cite{Laakmann_Hu_Farrell_2022} introduces finite-element(-in-space) implicit timesteppers for the incompressible analogue to this system with structure-preserving (SP) properties in the ideal case, alongside parameter-robust preconditioners. We show that these timesteppers can derive from a finite-element-in-time (FET) (and finite-element-in-space) interpretation. The benefits of this reformulation are discussed, including the derivation of timesteppers that are higher order in time, and the quantifiable dissipative SP properties in the non-ideal, resistive case.
        
        We discuss possible options for extending this FET approach to timesteppers for the compressible case.

        The kinetic corrections satisfy linearized Boltzmann equations. Using a Lénard--Bernstein collision operator, these take Fokker--Planck-like forms \cite{Fokker_1914, Planck_1917} wherein pseudo-particles in the numerical model obey the neoclassical transport equations, with particle-independent Brownian drift terms. This offers a rigorous methodology for incorporating collisions into the particle transport model, without coupling the equations of motions for each particle.
        
        Works by Chen, Chacón et al. \cite{Chen_Chacón_Barnes_2011, Chacón_Chen_Barnes_2013, Chen_Chacón_2014, Chen_Chacón_2015} have developed structure-preserving particle pushers for neoclassical transport in the Vlasov equations, derived from Crank--Nicolson integrators. We show these too can can derive from a FET interpretation, similarly offering potential extensions to higher-order-in-time particle pushers. The FET formulation is used also to consider how the stochastic drift terms can be incorporated into the pushers. Stochastic gyrokinetic expansions are also discussed.

        Different options for the numerical implementation of these schemes are considered.

        Due to the efficacy of FET in the development of SP timesteppers for both the fluid and kinetic component, we hope this approach will prove effective in the future for developing SP timesteppers for the full hybrid model. We hope this will give us the opportunity to incorporate previously inaccessible kinetic effects into the highly effective, modern, finite-element MHD models.
    \end{abstract}
    
    
    \newpage
    \tableofcontents
    
    
    \newpage
    \pagenumbering{arabic}
    %\linenumbers\renewcommand\thelinenumber{\color{black!50}\arabic{linenumber}}
            \documentclass[12pt, a4paper]{report}

\input{template/main.tex}

\title{\BA{Title in Progress...}}
\author{Boris Andrews}
\affil{Mathematical Institute, University of Oxford}
\date{\today}


\begin{document}
    \pagenumbering{gobble}
    \maketitle
    
    
    \begin{abstract}
        Magnetic confinement reactors---in particular tokamaks---offer one of the most promising options for achieving practical nuclear fusion, with the potential to provide virtually limitless, clean energy. The theoretical and numerical modeling of tokamak plasmas is simultaneously an essential component of effective reactor design, and a great research barrier. Tokamak operational conditions exhibit comparatively low Knudsen numbers. Kinetic effects, including kinetic waves and instabilities, Landau damping, bump-on-tail instabilities and more, are therefore highly influential in tokamak plasma dynamics. Purely fluid models are inherently incapable of capturing these effects, whereas the high dimensionality in purely kinetic models render them practically intractable for most relevant purposes.

        We consider a $\delta\!f$ decomposition model, with a macroscopic fluid background and microscopic kinetic correction, both fully coupled to each other. A similar manner of discretization is proposed to that used in the recent \texttt{STRUPHY} code \cite{Holderied_Possanner_Wang_2021, Holderied_2022, Li_et_al_2023} with a finite-element model for the background and a pseudo-particle/PiC model for the correction.

        The fluid background satisfies the full, non-linear, resistive, compressible, Hall MHD equations. \cite{Laakmann_Hu_Farrell_2022} introduces finite-element(-in-space) implicit timesteppers for the incompressible analogue to this system with structure-preserving (SP) properties in the ideal case, alongside parameter-robust preconditioners. We show that these timesteppers can derive from a finite-element-in-time (FET) (and finite-element-in-space) interpretation. The benefits of this reformulation are discussed, including the derivation of timesteppers that are higher order in time, and the quantifiable dissipative SP properties in the non-ideal, resistive case.
        
        We discuss possible options for extending this FET approach to timesteppers for the compressible case.

        The kinetic corrections satisfy linearized Boltzmann equations. Using a Lénard--Bernstein collision operator, these take Fokker--Planck-like forms \cite{Fokker_1914, Planck_1917} wherein pseudo-particles in the numerical model obey the neoclassical transport equations, with particle-independent Brownian drift terms. This offers a rigorous methodology for incorporating collisions into the particle transport model, without coupling the equations of motions for each particle.
        
        Works by Chen, Chacón et al. \cite{Chen_Chacón_Barnes_2011, Chacón_Chen_Barnes_2013, Chen_Chacón_2014, Chen_Chacón_2015} have developed structure-preserving particle pushers for neoclassical transport in the Vlasov equations, derived from Crank--Nicolson integrators. We show these too can can derive from a FET interpretation, similarly offering potential extensions to higher-order-in-time particle pushers. The FET formulation is used also to consider how the stochastic drift terms can be incorporated into the pushers. Stochastic gyrokinetic expansions are also discussed.

        Different options for the numerical implementation of these schemes are considered.

        Due to the efficacy of FET in the development of SP timesteppers for both the fluid and kinetic component, we hope this approach will prove effective in the future for developing SP timesteppers for the full hybrid model. We hope this will give us the opportunity to incorporate previously inaccessible kinetic effects into the highly effective, modern, finite-element MHD models.
    \end{abstract}
    
    
    \newpage
    \tableofcontents
    
    
    \newpage
    \pagenumbering{arabic}
    %\linenumbers\renewcommand\thelinenumber{\color{black!50}\arabic{linenumber}}
            \input{0 - introduction/main.tex}
        \part{Research}
            \input{1 - low-noise PiC models/main.tex}
            \input{2 - kinetic component/main.tex}
            \input{3 - fluid component/main.tex}
            \input{4 - numerical implementation/main.tex}
        \part{Project Overview}
            \input{5 - research plan/main.tex}
            \input{6 - summary/main.tex}
    
    
    %\section{}
    \newpage
    \pagenumbering{gobble}
        \printbibliography


    \newpage
    \pagenumbering{roman}
    \appendix
        \part{Appendices}
            \input{8 - Hilbert complexes/main.tex}
            \input{9 - weak conservation proofs/main.tex}
\end{document}

        \part{Research}
            \documentclass[12pt, a4paper]{report}

\input{template/main.tex}

\title{\BA{Title in Progress...}}
\author{Boris Andrews}
\affil{Mathematical Institute, University of Oxford}
\date{\today}


\begin{document}
    \pagenumbering{gobble}
    \maketitle
    
    
    \begin{abstract}
        Magnetic confinement reactors---in particular tokamaks---offer one of the most promising options for achieving practical nuclear fusion, with the potential to provide virtually limitless, clean energy. The theoretical and numerical modeling of tokamak plasmas is simultaneously an essential component of effective reactor design, and a great research barrier. Tokamak operational conditions exhibit comparatively low Knudsen numbers. Kinetic effects, including kinetic waves and instabilities, Landau damping, bump-on-tail instabilities and more, are therefore highly influential in tokamak plasma dynamics. Purely fluid models are inherently incapable of capturing these effects, whereas the high dimensionality in purely kinetic models render them practically intractable for most relevant purposes.

        We consider a $\delta\!f$ decomposition model, with a macroscopic fluid background and microscopic kinetic correction, both fully coupled to each other. A similar manner of discretization is proposed to that used in the recent \texttt{STRUPHY} code \cite{Holderied_Possanner_Wang_2021, Holderied_2022, Li_et_al_2023} with a finite-element model for the background and a pseudo-particle/PiC model for the correction.

        The fluid background satisfies the full, non-linear, resistive, compressible, Hall MHD equations. \cite{Laakmann_Hu_Farrell_2022} introduces finite-element(-in-space) implicit timesteppers for the incompressible analogue to this system with structure-preserving (SP) properties in the ideal case, alongside parameter-robust preconditioners. We show that these timesteppers can derive from a finite-element-in-time (FET) (and finite-element-in-space) interpretation. The benefits of this reformulation are discussed, including the derivation of timesteppers that are higher order in time, and the quantifiable dissipative SP properties in the non-ideal, resistive case.
        
        We discuss possible options for extending this FET approach to timesteppers for the compressible case.

        The kinetic corrections satisfy linearized Boltzmann equations. Using a Lénard--Bernstein collision operator, these take Fokker--Planck-like forms \cite{Fokker_1914, Planck_1917} wherein pseudo-particles in the numerical model obey the neoclassical transport equations, with particle-independent Brownian drift terms. This offers a rigorous methodology for incorporating collisions into the particle transport model, without coupling the equations of motions for each particle.
        
        Works by Chen, Chacón et al. \cite{Chen_Chacón_Barnes_2011, Chacón_Chen_Barnes_2013, Chen_Chacón_2014, Chen_Chacón_2015} have developed structure-preserving particle pushers for neoclassical transport in the Vlasov equations, derived from Crank--Nicolson integrators. We show these too can can derive from a FET interpretation, similarly offering potential extensions to higher-order-in-time particle pushers. The FET formulation is used also to consider how the stochastic drift terms can be incorporated into the pushers. Stochastic gyrokinetic expansions are also discussed.

        Different options for the numerical implementation of these schemes are considered.

        Due to the efficacy of FET in the development of SP timesteppers for both the fluid and kinetic component, we hope this approach will prove effective in the future for developing SP timesteppers for the full hybrid model. We hope this will give us the opportunity to incorporate previously inaccessible kinetic effects into the highly effective, modern, finite-element MHD models.
    \end{abstract}
    
    
    \newpage
    \tableofcontents
    
    
    \newpage
    \pagenumbering{arabic}
    %\linenumbers\renewcommand\thelinenumber{\color{black!50}\arabic{linenumber}}
            \input{0 - introduction/main.tex}
        \part{Research}
            \input{1 - low-noise PiC models/main.tex}
            \input{2 - kinetic component/main.tex}
            \input{3 - fluid component/main.tex}
            \input{4 - numerical implementation/main.tex}
        \part{Project Overview}
            \input{5 - research plan/main.tex}
            \input{6 - summary/main.tex}
    
    
    %\section{}
    \newpage
    \pagenumbering{gobble}
        \printbibliography


    \newpage
    \pagenumbering{roman}
    \appendix
        \part{Appendices}
            \input{8 - Hilbert complexes/main.tex}
            \input{9 - weak conservation proofs/main.tex}
\end{document}

            \documentclass[12pt, a4paper]{report}

\input{template/main.tex}

\title{\BA{Title in Progress...}}
\author{Boris Andrews}
\affil{Mathematical Institute, University of Oxford}
\date{\today}


\begin{document}
    \pagenumbering{gobble}
    \maketitle
    
    
    \begin{abstract}
        Magnetic confinement reactors---in particular tokamaks---offer one of the most promising options for achieving practical nuclear fusion, with the potential to provide virtually limitless, clean energy. The theoretical and numerical modeling of tokamak plasmas is simultaneously an essential component of effective reactor design, and a great research barrier. Tokamak operational conditions exhibit comparatively low Knudsen numbers. Kinetic effects, including kinetic waves and instabilities, Landau damping, bump-on-tail instabilities and more, are therefore highly influential in tokamak plasma dynamics. Purely fluid models are inherently incapable of capturing these effects, whereas the high dimensionality in purely kinetic models render them practically intractable for most relevant purposes.

        We consider a $\delta\!f$ decomposition model, with a macroscopic fluid background and microscopic kinetic correction, both fully coupled to each other. A similar manner of discretization is proposed to that used in the recent \texttt{STRUPHY} code \cite{Holderied_Possanner_Wang_2021, Holderied_2022, Li_et_al_2023} with a finite-element model for the background and a pseudo-particle/PiC model for the correction.

        The fluid background satisfies the full, non-linear, resistive, compressible, Hall MHD equations. \cite{Laakmann_Hu_Farrell_2022} introduces finite-element(-in-space) implicit timesteppers for the incompressible analogue to this system with structure-preserving (SP) properties in the ideal case, alongside parameter-robust preconditioners. We show that these timesteppers can derive from a finite-element-in-time (FET) (and finite-element-in-space) interpretation. The benefits of this reformulation are discussed, including the derivation of timesteppers that are higher order in time, and the quantifiable dissipative SP properties in the non-ideal, resistive case.
        
        We discuss possible options for extending this FET approach to timesteppers for the compressible case.

        The kinetic corrections satisfy linearized Boltzmann equations. Using a Lénard--Bernstein collision operator, these take Fokker--Planck-like forms \cite{Fokker_1914, Planck_1917} wherein pseudo-particles in the numerical model obey the neoclassical transport equations, with particle-independent Brownian drift terms. This offers a rigorous methodology for incorporating collisions into the particle transport model, without coupling the equations of motions for each particle.
        
        Works by Chen, Chacón et al. \cite{Chen_Chacón_Barnes_2011, Chacón_Chen_Barnes_2013, Chen_Chacón_2014, Chen_Chacón_2015} have developed structure-preserving particle pushers for neoclassical transport in the Vlasov equations, derived from Crank--Nicolson integrators. We show these too can can derive from a FET interpretation, similarly offering potential extensions to higher-order-in-time particle pushers. The FET formulation is used also to consider how the stochastic drift terms can be incorporated into the pushers. Stochastic gyrokinetic expansions are also discussed.

        Different options for the numerical implementation of these schemes are considered.

        Due to the efficacy of FET in the development of SP timesteppers for both the fluid and kinetic component, we hope this approach will prove effective in the future for developing SP timesteppers for the full hybrid model. We hope this will give us the opportunity to incorporate previously inaccessible kinetic effects into the highly effective, modern, finite-element MHD models.
    \end{abstract}
    
    
    \newpage
    \tableofcontents
    
    
    \newpage
    \pagenumbering{arabic}
    %\linenumbers\renewcommand\thelinenumber{\color{black!50}\arabic{linenumber}}
            \input{0 - introduction/main.tex}
        \part{Research}
            \input{1 - low-noise PiC models/main.tex}
            \input{2 - kinetic component/main.tex}
            \input{3 - fluid component/main.tex}
            \input{4 - numerical implementation/main.tex}
        \part{Project Overview}
            \input{5 - research plan/main.tex}
            \input{6 - summary/main.tex}
    
    
    %\section{}
    \newpage
    \pagenumbering{gobble}
        \printbibliography


    \newpage
    \pagenumbering{roman}
    \appendix
        \part{Appendices}
            \input{8 - Hilbert complexes/main.tex}
            \input{9 - weak conservation proofs/main.tex}
\end{document}

            \documentclass[12pt, a4paper]{report}

\input{template/main.tex}

\title{\BA{Title in Progress...}}
\author{Boris Andrews}
\affil{Mathematical Institute, University of Oxford}
\date{\today}


\begin{document}
    \pagenumbering{gobble}
    \maketitle
    
    
    \begin{abstract}
        Magnetic confinement reactors---in particular tokamaks---offer one of the most promising options for achieving practical nuclear fusion, with the potential to provide virtually limitless, clean energy. The theoretical and numerical modeling of tokamak plasmas is simultaneously an essential component of effective reactor design, and a great research barrier. Tokamak operational conditions exhibit comparatively low Knudsen numbers. Kinetic effects, including kinetic waves and instabilities, Landau damping, bump-on-tail instabilities and more, are therefore highly influential in tokamak plasma dynamics. Purely fluid models are inherently incapable of capturing these effects, whereas the high dimensionality in purely kinetic models render them practically intractable for most relevant purposes.

        We consider a $\delta\!f$ decomposition model, with a macroscopic fluid background and microscopic kinetic correction, both fully coupled to each other. A similar manner of discretization is proposed to that used in the recent \texttt{STRUPHY} code \cite{Holderied_Possanner_Wang_2021, Holderied_2022, Li_et_al_2023} with a finite-element model for the background and a pseudo-particle/PiC model for the correction.

        The fluid background satisfies the full, non-linear, resistive, compressible, Hall MHD equations. \cite{Laakmann_Hu_Farrell_2022} introduces finite-element(-in-space) implicit timesteppers for the incompressible analogue to this system with structure-preserving (SP) properties in the ideal case, alongside parameter-robust preconditioners. We show that these timesteppers can derive from a finite-element-in-time (FET) (and finite-element-in-space) interpretation. The benefits of this reformulation are discussed, including the derivation of timesteppers that are higher order in time, and the quantifiable dissipative SP properties in the non-ideal, resistive case.
        
        We discuss possible options for extending this FET approach to timesteppers for the compressible case.

        The kinetic corrections satisfy linearized Boltzmann equations. Using a Lénard--Bernstein collision operator, these take Fokker--Planck-like forms \cite{Fokker_1914, Planck_1917} wherein pseudo-particles in the numerical model obey the neoclassical transport equations, with particle-independent Brownian drift terms. This offers a rigorous methodology for incorporating collisions into the particle transport model, without coupling the equations of motions for each particle.
        
        Works by Chen, Chacón et al. \cite{Chen_Chacón_Barnes_2011, Chacón_Chen_Barnes_2013, Chen_Chacón_2014, Chen_Chacón_2015} have developed structure-preserving particle pushers for neoclassical transport in the Vlasov equations, derived from Crank--Nicolson integrators. We show these too can can derive from a FET interpretation, similarly offering potential extensions to higher-order-in-time particle pushers. The FET formulation is used also to consider how the stochastic drift terms can be incorporated into the pushers. Stochastic gyrokinetic expansions are also discussed.

        Different options for the numerical implementation of these schemes are considered.

        Due to the efficacy of FET in the development of SP timesteppers for both the fluid and kinetic component, we hope this approach will prove effective in the future for developing SP timesteppers for the full hybrid model. We hope this will give us the opportunity to incorporate previously inaccessible kinetic effects into the highly effective, modern, finite-element MHD models.
    \end{abstract}
    
    
    \newpage
    \tableofcontents
    
    
    \newpage
    \pagenumbering{arabic}
    %\linenumbers\renewcommand\thelinenumber{\color{black!50}\arabic{linenumber}}
            \input{0 - introduction/main.tex}
        \part{Research}
            \input{1 - low-noise PiC models/main.tex}
            \input{2 - kinetic component/main.tex}
            \input{3 - fluid component/main.tex}
            \input{4 - numerical implementation/main.tex}
        \part{Project Overview}
            \input{5 - research plan/main.tex}
            \input{6 - summary/main.tex}
    
    
    %\section{}
    \newpage
    \pagenumbering{gobble}
        \printbibliography


    \newpage
    \pagenumbering{roman}
    \appendix
        \part{Appendices}
            \input{8 - Hilbert complexes/main.tex}
            \input{9 - weak conservation proofs/main.tex}
\end{document}

            \documentclass[12pt, a4paper]{report}

\input{template/main.tex}

\title{\BA{Title in Progress...}}
\author{Boris Andrews}
\affil{Mathematical Institute, University of Oxford}
\date{\today}


\begin{document}
    \pagenumbering{gobble}
    \maketitle
    
    
    \begin{abstract}
        Magnetic confinement reactors---in particular tokamaks---offer one of the most promising options for achieving practical nuclear fusion, with the potential to provide virtually limitless, clean energy. The theoretical and numerical modeling of tokamak plasmas is simultaneously an essential component of effective reactor design, and a great research barrier. Tokamak operational conditions exhibit comparatively low Knudsen numbers. Kinetic effects, including kinetic waves and instabilities, Landau damping, bump-on-tail instabilities and more, are therefore highly influential in tokamak plasma dynamics. Purely fluid models are inherently incapable of capturing these effects, whereas the high dimensionality in purely kinetic models render them practically intractable for most relevant purposes.

        We consider a $\delta\!f$ decomposition model, with a macroscopic fluid background and microscopic kinetic correction, both fully coupled to each other. A similar manner of discretization is proposed to that used in the recent \texttt{STRUPHY} code \cite{Holderied_Possanner_Wang_2021, Holderied_2022, Li_et_al_2023} with a finite-element model for the background and a pseudo-particle/PiC model for the correction.

        The fluid background satisfies the full, non-linear, resistive, compressible, Hall MHD equations. \cite{Laakmann_Hu_Farrell_2022} introduces finite-element(-in-space) implicit timesteppers for the incompressible analogue to this system with structure-preserving (SP) properties in the ideal case, alongside parameter-robust preconditioners. We show that these timesteppers can derive from a finite-element-in-time (FET) (and finite-element-in-space) interpretation. The benefits of this reformulation are discussed, including the derivation of timesteppers that are higher order in time, and the quantifiable dissipative SP properties in the non-ideal, resistive case.
        
        We discuss possible options for extending this FET approach to timesteppers for the compressible case.

        The kinetic corrections satisfy linearized Boltzmann equations. Using a Lénard--Bernstein collision operator, these take Fokker--Planck-like forms \cite{Fokker_1914, Planck_1917} wherein pseudo-particles in the numerical model obey the neoclassical transport equations, with particle-independent Brownian drift terms. This offers a rigorous methodology for incorporating collisions into the particle transport model, without coupling the equations of motions for each particle.
        
        Works by Chen, Chacón et al. \cite{Chen_Chacón_Barnes_2011, Chacón_Chen_Barnes_2013, Chen_Chacón_2014, Chen_Chacón_2015} have developed structure-preserving particle pushers for neoclassical transport in the Vlasov equations, derived from Crank--Nicolson integrators. We show these too can can derive from a FET interpretation, similarly offering potential extensions to higher-order-in-time particle pushers. The FET formulation is used also to consider how the stochastic drift terms can be incorporated into the pushers. Stochastic gyrokinetic expansions are also discussed.

        Different options for the numerical implementation of these schemes are considered.

        Due to the efficacy of FET in the development of SP timesteppers for both the fluid and kinetic component, we hope this approach will prove effective in the future for developing SP timesteppers for the full hybrid model. We hope this will give us the opportunity to incorporate previously inaccessible kinetic effects into the highly effective, modern, finite-element MHD models.
    \end{abstract}
    
    
    \newpage
    \tableofcontents
    
    
    \newpage
    \pagenumbering{arabic}
    %\linenumbers\renewcommand\thelinenumber{\color{black!50}\arabic{linenumber}}
            \input{0 - introduction/main.tex}
        \part{Research}
            \input{1 - low-noise PiC models/main.tex}
            \input{2 - kinetic component/main.tex}
            \input{3 - fluid component/main.tex}
            \input{4 - numerical implementation/main.tex}
        \part{Project Overview}
            \input{5 - research plan/main.tex}
            \input{6 - summary/main.tex}
    
    
    %\section{}
    \newpage
    \pagenumbering{gobble}
        \printbibliography


    \newpage
    \pagenumbering{roman}
    \appendix
        \part{Appendices}
            \input{8 - Hilbert complexes/main.tex}
            \input{9 - weak conservation proofs/main.tex}
\end{document}

        \part{Project Overview}
            \documentclass[12pt, a4paper]{report}

\input{template/main.tex}

\title{\BA{Title in Progress...}}
\author{Boris Andrews}
\affil{Mathematical Institute, University of Oxford}
\date{\today}


\begin{document}
    \pagenumbering{gobble}
    \maketitle
    
    
    \begin{abstract}
        Magnetic confinement reactors---in particular tokamaks---offer one of the most promising options for achieving practical nuclear fusion, with the potential to provide virtually limitless, clean energy. The theoretical and numerical modeling of tokamak plasmas is simultaneously an essential component of effective reactor design, and a great research barrier. Tokamak operational conditions exhibit comparatively low Knudsen numbers. Kinetic effects, including kinetic waves and instabilities, Landau damping, bump-on-tail instabilities and more, are therefore highly influential in tokamak plasma dynamics. Purely fluid models are inherently incapable of capturing these effects, whereas the high dimensionality in purely kinetic models render them practically intractable for most relevant purposes.

        We consider a $\delta\!f$ decomposition model, with a macroscopic fluid background and microscopic kinetic correction, both fully coupled to each other. A similar manner of discretization is proposed to that used in the recent \texttt{STRUPHY} code \cite{Holderied_Possanner_Wang_2021, Holderied_2022, Li_et_al_2023} with a finite-element model for the background and a pseudo-particle/PiC model for the correction.

        The fluid background satisfies the full, non-linear, resistive, compressible, Hall MHD equations. \cite{Laakmann_Hu_Farrell_2022} introduces finite-element(-in-space) implicit timesteppers for the incompressible analogue to this system with structure-preserving (SP) properties in the ideal case, alongside parameter-robust preconditioners. We show that these timesteppers can derive from a finite-element-in-time (FET) (and finite-element-in-space) interpretation. The benefits of this reformulation are discussed, including the derivation of timesteppers that are higher order in time, and the quantifiable dissipative SP properties in the non-ideal, resistive case.
        
        We discuss possible options for extending this FET approach to timesteppers for the compressible case.

        The kinetic corrections satisfy linearized Boltzmann equations. Using a Lénard--Bernstein collision operator, these take Fokker--Planck-like forms \cite{Fokker_1914, Planck_1917} wherein pseudo-particles in the numerical model obey the neoclassical transport equations, with particle-independent Brownian drift terms. This offers a rigorous methodology for incorporating collisions into the particle transport model, without coupling the equations of motions for each particle.
        
        Works by Chen, Chacón et al. \cite{Chen_Chacón_Barnes_2011, Chacón_Chen_Barnes_2013, Chen_Chacón_2014, Chen_Chacón_2015} have developed structure-preserving particle pushers for neoclassical transport in the Vlasov equations, derived from Crank--Nicolson integrators. We show these too can can derive from a FET interpretation, similarly offering potential extensions to higher-order-in-time particle pushers. The FET formulation is used also to consider how the stochastic drift terms can be incorporated into the pushers. Stochastic gyrokinetic expansions are also discussed.

        Different options for the numerical implementation of these schemes are considered.

        Due to the efficacy of FET in the development of SP timesteppers for both the fluid and kinetic component, we hope this approach will prove effective in the future for developing SP timesteppers for the full hybrid model. We hope this will give us the opportunity to incorporate previously inaccessible kinetic effects into the highly effective, modern, finite-element MHD models.
    \end{abstract}
    
    
    \newpage
    \tableofcontents
    
    
    \newpage
    \pagenumbering{arabic}
    %\linenumbers\renewcommand\thelinenumber{\color{black!50}\arabic{linenumber}}
            \input{0 - introduction/main.tex}
        \part{Research}
            \input{1 - low-noise PiC models/main.tex}
            \input{2 - kinetic component/main.tex}
            \input{3 - fluid component/main.tex}
            \input{4 - numerical implementation/main.tex}
        \part{Project Overview}
            \input{5 - research plan/main.tex}
            \input{6 - summary/main.tex}
    
    
    %\section{}
    \newpage
    \pagenumbering{gobble}
        \printbibliography


    \newpage
    \pagenumbering{roman}
    \appendix
        \part{Appendices}
            \input{8 - Hilbert complexes/main.tex}
            \input{9 - weak conservation proofs/main.tex}
\end{document}

            \documentclass[12pt, a4paper]{report}

\input{template/main.tex}

\title{\BA{Title in Progress...}}
\author{Boris Andrews}
\affil{Mathematical Institute, University of Oxford}
\date{\today}


\begin{document}
    \pagenumbering{gobble}
    \maketitle
    
    
    \begin{abstract}
        Magnetic confinement reactors---in particular tokamaks---offer one of the most promising options for achieving practical nuclear fusion, with the potential to provide virtually limitless, clean energy. The theoretical and numerical modeling of tokamak plasmas is simultaneously an essential component of effective reactor design, and a great research barrier. Tokamak operational conditions exhibit comparatively low Knudsen numbers. Kinetic effects, including kinetic waves and instabilities, Landau damping, bump-on-tail instabilities and more, are therefore highly influential in tokamak plasma dynamics. Purely fluid models are inherently incapable of capturing these effects, whereas the high dimensionality in purely kinetic models render them practically intractable for most relevant purposes.

        We consider a $\delta\!f$ decomposition model, with a macroscopic fluid background and microscopic kinetic correction, both fully coupled to each other. A similar manner of discretization is proposed to that used in the recent \texttt{STRUPHY} code \cite{Holderied_Possanner_Wang_2021, Holderied_2022, Li_et_al_2023} with a finite-element model for the background and a pseudo-particle/PiC model for the correction.

        The fluid background satisfies the full, non-linear, resistive, compressible, Hall MHD equations. \cite{Laakmann_Hu_Farrell_2022} introduces finite-element(-in-space) implicit timesteppers for the incompressible analogue to this system with structure-preserving (SP) properties in the ideal case, alongside parameter-robust preconditioners. We show that these timesteppers can derive from a finite-element-in-time (FET) (and finite-element-in-space) interpretation. The benefits of this reformulation are discussed, including the derivation of timesteppers that are higher order in time, and the quantifiable dissipative SP properties in the non-ideal, resistive case.
        
        We discuss possible options for extending this FET approach to timesteppers for the compressible case.

        The kinetic corrections satisfy linearized Boltzmann equations. Using a Lénard--Bernstein collision operator, these take Fokker--Planck-like forms \cite{Fokker_1914, Planck_1917} wherein pseudo-particles in the numerical model obey the neoclassical transport equations, with particle-independent Brownian drift terms. This offers a rigorous methodology for incorporating collisions into the particle transport model, without coupling the equations of motions for each particle.
        
        Works by Chen, Chacón et al. \cite{Chen_Chacón_Barnes_2011, Chacón_Chen_Barnes_2013, Chen_Chacón_2014, Chen_Chacón_2015} have developed structure-preserving particle pushers for neoclassical transport in the Vlasov equations, derived from Crank--Nicolson integrators. We show these too can can derive from a FET interpretation, similarly offering potential extensions to higher-order-in-time particle pushers. The FET formulation is used also to consider how the stochastic drift terms can be incorporated into the pushers. Stochastic gyrokinetic expansions are also discussed.

        Different options for the numerical implementation of these schemes are considered.

        Due to the efficacy of FET in the development of SP timesteppers for both the fluid and kinetic component, we hope this approach will prove effective in the future for developing SP timesteppers for the full hybrid model. We hope this will give us the opportunity to incorporate previously inaccessible kinetic effects into the highly effective, modern, finite-element MHD models.
    \end{abstract}
    
    
    \newpage
    \tableofcontents
    
    
    \newpage
    \pagenumbering{arabic}
    %\linenumbers\renewcommand\thelinenumber{\color{black!50}\arabic{linenumber}}
            \input{0 - introduction/main.tex}
        \part{Research}
            \input{1 - low-noise PiC models/main.tex}
            \input{2 - kinetic component/main.tex}
            \input{3 - fluid component/main.tex}
            \input{4 - numerical implementation/main.tex}
        \part{Project Overview}
            \input{5 - research plan/main.tex}
            \input{6 - summary/main.tex}
    
    
    %\section{}
    \newpage
    \pagenumbering{gobble}
        \printbibliography


    \newpage
    \pagenumbering{roman}
    \appendix
        \part{Appendices}
            \input{8 - Hilbert complexes/main.tex}
            \input{9 - weak conservation proofs/main.tex}
\end{document}

    
    
    %\section{}
    \newpage
    \pagenumbering{gobble}
        \printbibliography


    \newpage
    \pagenumbering{roman}
    \appendix
        \part{Appendices}
            \documentclass[12pt, a4paper]{report}

\input{template/main.tex}

\title{\BA{Title in Progress...}}
\author{Boris Andrews}
\affil{Mathematical Institute, University of Oxford}
\date{\today}


\begin{document}
    \pagenumbering{gobble}
    \maketitle
    
    
    \begin{abstract}
        Magnetic confinement reactors---in particular tokamaks---offer one of the most promising options for achieving practical nuclear fusion, with the potential to provide virtually limitless, clean energy. The theoretical and numerical modeling of tokamak plasmas is simultaneously an essential component of effective reactor design, and a great research barrier. Tokamak operational conditions exhibit comparatively low Knudsen numbers. Kinetic effects, including kinetic waves and instabilities, Landau damping, bump-on-tail instabilities and more, are therefore highly influential in tokamak plasma dynamics. Purely fluid models are inherently incapable of capturing these effects, whereas the high dimensionality in purely kinetic models render them practically intractable for most relevant purposes.

        We consider a $\delta\!f$ decomposition model, with a macroscopic fluid background and microscopic kinetic correction, both fully coupled to each other. A similar manner of discretization is proposed to that used in the recent \texttt{STRUPHY} code \cite{Holderied_Possanner_Wang_2021, Holderied_2022, Li_et_al_2023} with a finite-element model for the background and a pseudo-particle/PiC model for the correction.

        The fluid background satisfies the full, non-linear, resistive, compressible, Hall MHD equations. \cite{Laakmann_Hu_Farrell_2022} introduces finite-element(-in-space) implicit timesteppers for the incompressible analogue to this system with structure-preserving (SP) properties in the ideal case, alongside parameter-robust preconditioners. We show that these timesteppers can derive from a finite-element-in-time (FET) (and finite-element-in-space) interpretation. The benefits of this reformulation are discussed, including the derivation of timesteppers that are higher order in time, and the quantifiable dissipative SP properties in the non-ideal, resistive case.
        
        We discuss possible options for extending this FET approach to timesteppers for the compressible case.

        The kinetic corrections satisfy linearized Boltzmann equations. Using a Lénard--Bernstein collision operator, these take Fokker--Planck-like forms \cite{Fokker_1914, Planck_1917} wherein pseudo-particles in the numerical model obey the neoclassical transport equations, with particle-independent Brownian drift terms. This offers a rigorous methodology for incorporating collisions into the particle transport model, without coupling the equations of motions for each particle.
        
        Works by Chen, Chacón et al. \cite{Chen_Chacón_Barnes_2011, Chacón_Chen_Barnes_2013, Chen_Chacón_2014, Chen_Chacón_2015} have developed structure-preserving particle pushers for neoclassical transport in the Vlasov equations, derived from Crank--Nicolson integrators. We show these too can can derive from a FET interpretation, similarly offering potential extensions to higher-order-in-time particle pushers. The FET formulation is used also to consider how the stochastic drift terms can be incorporated into the pushers. Stochastic gyrokinetic expansions are also discussed.

        Different options for the numerical implementation of these schemes are considered.

        Due to the efficacy of FET in the development of SP timesteppers for both the fluid and kinetic component, we hope this approach will prove effective in the future for developing SP timesteppers for the full hybrid model. We hope this will give us the opportunity to incorporate previously inaccessible kinetic effects into the highly effective, modern, finite-element MHD models.
    \end{abstract}
    
    
    \newpage
    \tableofcontents
    
    
    \newpage
    \pagenumbering{arabic}
    %\linenumbers\renewcommand\thelinenumber{\color{black!50}\arabic{linenumber}}
            \input{0 - introduction/main.tex}
        \part{Research}
            \input{1 - low-noise PiC models/main.tex}
            \input{2 - kinetic component/main.tex}
            \input{3 - fluid component/main.tex}
            \input{4 - numerical implementation/main.tex}
        \part{Project Overview}
            \input{5 - research plan/main.tex}
            \input{6 - summary/main.tex}
    
    
    %\section{}
    \newpage
    \pagenumbering{gobble}
        \printbibliography


    \newpage
    \pagenumbering{roman}
    \appendix
        \part{Appendices}
            \input{8 - Hilbert complexes/main.tex}
            \input{9 - weak conservation proofs/main.tex}
\end{document}

            \documentclass[12pt, a4paper]{report}

\input{template/main.tex}

\title{\BA{Title in Progress...}}
\author{Boris Andrews}
\affil{Mathematical Institute, University of Oxford}
\date{\today}


\begin{document}
    \pagenumbering{gobble}
    \maketitle
    
    
    \begin{abstract}
        Magnetic confinement reactors---in particular tokamaks---offer one of the most promising options for achieving practical nuclear fusion, with the potential to provide virtually limitless, clean energy. The theoretical and numerical modeling of tokamak plasmas is simultaneously an essential component of effective reactor design, and a great research barrier. Tokamak operational conditions exhibit comparatively low Knudsen numbers. Kinetic effects, including kinetic waves and instabilities, Landau damping, bump-on-tail instabilities and more, are therefore highly influential in tokamak plasma dynamics. Purely fluid models are inherently incapable of capturing these effects, whereas the high dimensionality in purely kinetic models render them practically intractable for most relevant purposes.

        We consider a $\delta\!f$ decomposition model, with a macroscopic fluid background and microscopic kinetic correction, both fully coupled to each other. A similar manner of discretization is proposed to that used in the recent \texttt{STRUPHY} code \cite{Holderied_Possanner_Wang_2021, Holderied_2022, Li_et_al_2023} with a finite-element model for the background and a pseudo-particle/PiC model for the correction.

        The fluid background satisfies the full, non-linear, resistive, compressible, Hall MHD equations. \cite{Laakmann_Hu_Farrell_2022} introduces finite-element(-in-space) implicit timesteppers for the incompressible analogue to this system with structure-preserving (SP) properties in the ideal case, alongside parameter-robust preconditioners. We show that these timesteppers can derive from a finite-element-in-time (FET) (and finite-element-in-space) interpretation. The benefits of this reformulation are discussed, including the derivation of timesteppers that are higher order in time, and the quantifiable dissipative SP properties in the non-ideal, resistive case.
        
        We discuss possible options for extending this FET approach to timesteppers for the compressible case.

        The kinetic corrections satisfy linearized Boltzmann equations. Using a Lénard--Bernstein collision operator, these take Fokker--Planck-like forms \cite{Fokker_1914, Planck_1917} wherein pseudo-particles in the numerical model obey the neoclassical transport equations, with particle-independent Brownian drift terms. This offers a rigorous methodology for incorporating collisions into the particle transport model, without coupling the equations of motions for each particle.
        
        Works by Chen, Chacón et al. \cite{Chen_Chacón_Barnes_2011, Chacón_Chen_Barnes_2013, Chen_Chacón_2014, Chen_Chacón_2015} have developed structure-preserving particle pushers for neoclassical transport in the Vlasov equations, derived from Crank--Nicolson integrators. We show these too can can derive from a FET interpretation, similarly offering potential extensions to higher-order-in-time particle pushers. The FET formulation is used also to consider how the stochastic drift terms can be incorporated into the pushers. Stochastic gyrokinetic expansions are also discussed.

        Different options for the numerical implementation of these schemes are considered.

        Due to the efficacy of FET in the development of SP timesteppers for both the fluid and kinetic component, we hope this approach will prove effective in the future for developing SP timesteppers for the full hybrid model. We hope this will give us the opportunity to incorporate previously inaccessible kinetic effects into the highly effective, modern, finite-element MHD models.
    \end{abstract}
    
    
    \newpage
    \tableofcontents
    
    
    \newpage
    \pagenumbering{arabic}
    %\linenumbers\renewcommand\thelinenumber{\color{black!50}\arabic{linenumber}}
            \input{0 - introduction/main.tex}
        \part{Research}
            \input{1 - low-noise PiC models/main.tex}
            \input{2 - kinetic component/main.tex}
            \input{3 - fluid component/main.tex}
            \input{4 - numerical implementation/main.tex}
        \part{Project Overview}
            \input{5 - research plan/main.tex}
            \input{6 - summary/main.tex}
    
    
    %\section{}
    \newpage
    \pagenumbering{gobble}
        \printbibliography


    \newpage
    \pagenumbering{roman}
    \appendix
        \part{Appendices}
            \input{8 - Hilbert complexes/main.tex}
            \input{9 - weak conservation proofs/main.tex}
\end{document}

\end{document}

    
    
    %\section{}
    \newpage
    \pagenumbering{gobble}
        \printbibliography


    \newpage
    \pagenumbering{roman}
    \appendix
        \part{Appendices}
            \documentclass[12pt, a4paper]{report}

\documentclass[12pt, a4paper]{report}

\input{template/main.tex}

\title{\BA{Title in Progress...}}
\author{Boris Andrews}
\affil{Mathematical Institute, University of Oxford}
\date{\today}


\begin{document}
    \pagenumbering{gobble}
    \maketitle
    
    
    \begin{abstract}
        Magnetic confinement reactors---in particular tokamaks---offer one of the most promising options for achieving practical nuclear fusion, with the potential to provide virtually limitless, clean energy. The theoretical and numerical modeling of tokamak plasmas is simultaneously an essential component of effective reactor design, and a great research barrier. Tokamak operational conditions exhibit comparatively low Knudsen numbers. Kinetic effects, including kinetic waves and instabilities, Landau damping, bump-on-tail instabilities and more, are therefore highly influential in tokamak plasma dynamics. Purely fluid models are inherently incapable of capturing these effects, whereas the high dimensionality in purely kinetic models render them practically intractable for most relevant purposes.

        We consider a $\delta\!f$ decomposition model, with a macroscopic fluid background and microscopic kinetic correction, both fully coupled to each other. A similar manner of discretization is proposed to that used in the recent \texttt{STRUPHY} code \cite{Holderied_Possanner_Wang_2021, Holderied_2022, Li_et_al_2023} with a finite-element model for the background and a pseudo-particle/PiC model for the correction.

        The fluid background satisfies the full, non-linear, resistive, compressible, Hall MHD equations. \cite{Laakmann_Hu_Farrell_2022} introduces finite-element(-in-space) implicit timesteppers for the incompressible analogue to this system with structure-preserving (SP) properties in the ideal case, alongside parameter-robust preconditioners. We show that these timesteppers can derive from a finite-element-in-time (FET) (and finite-element-in-space) interpretation. The benefits of this reformulation are discussed, including the derivation of timesteppers that are higher order in time, and the quantifiable dissipative SP properties in the non-ideal, resistive case.
        
        We discuss possible options for extending this FET approach to timesteppers for the compressible case.

        The kinetic corrections satisfy linearized Boltzmann equations. Using a Lénard--Bernstein collision operator, these take Fokker--Planck-like forms \cite{Fokker_1914, Planck_1917} wherein pseudo-particles in the numerical model obey the neoclassical transport equations, with particle-independent Brownian drift terms. This offers a rigorous methodology for incorporating collisions into the particle transport model, without coupling the equations of motions for each particle.
        
        Works by Chen, Chacón et al. \cite{Chen_Chacón_Barnes_2011, Chacón_Chen_Barnes_2013, Chen_Chacón_2014, Chen_Chacón_2015} have developed structure-preserving particle pushers for neoclassical transport in the Vlasov equations, derived from Crank--Nicolson integrators. We show these too can can derive from a FET interpretation, similarly offering potential extensions to higher-order-in-time particle pushers. The FET formulation is used also to consider how the stochastic drift terms can be incorporated into the pushers. Stochastic gyrokinetic expansions are also discussed.

        Different options for the numerical implementation of these schemes are considered.

        Due to the efficacy of FET in the development of SP timesteppers for both the fluid and kinetic component, we hope this approach will prove effective in the future for developing SP timesteppers for the full hybrid model. We hope this will give us the opportunity to incorporate previously inaccessible kinetic effects into the highly effective, modern, finite-element MHD models.
    \end{abstract}
    
    
    \newpage
    \tableofcontents
    
    
    \newpage
    \pagenumbering{arabic}
    %\linenumbers\renewcommand\thelinenumber{\color{black!50}\arabic{linenumber}}
            \input{0 - introduction/main.tex}
        \part{Research}
            \input{1 - low-noise PiC models/main.tex}
            \input{2 - kinetic component/main.tex}
            \input{3 - fluid component/main.tex}
            \input{4 - numerical implementation/main.tex}
        \part{Project Overview}
            \input{5 - research plan/main.tex}
            \input{6 - summary/main.tex}
    
    
    %\section{}
    \newpage
    \pagenumbering{gobble}
        \printbibliography


    \newpage
    \pagenumbering{roman}
    \appendix
        \part{Appendices}
            \input{8 - Hilbert complexes/main.tex}
            \input{9 - weak conservation proofs/main.tex}
\end{document}


\title{\BA{Title in Progress...}}
\author{Boris Andrews}
\affil{Mathematical Institute, University of Oxford}
\date{\today}


\begin{document}
    \pagenumbering{gobble}
    \maketitle
    
    
    \begin{abstract}
        Magnetic confinement reactors---in particular tokamaks---offer one of the most promising options for achieving practical nuclear fusion, with the potential to provide virtually limitless, clean energy. The theoretical and numerical modeling of tokamak plasmas is simultaneously an essential component of effective reactor design, and a great research barrier. Tokamak operational conditions exhibit comparatively low Knudsen numbers. Kinetic effects, including kinetic waves and instabilities, Landau damping, bump-on-tail instabilities and more, are therefore highly influential in tokamak plasma dynamics. Purely fluid models are inherently incapable of capturing these effects, whereas the high dimensionality in purely kinetic models render them practically intractable for most relevant purposes.

        We consider a $\delta\!f$ decomposition model, with a macroscopic fluid background and microscopic kinetic correction, both fully coupled to each other. A similar manner of discretization is proposed to that used in the recent \texttt{STRUPHY} code \cite{Holderied_Possanner_Wang_2021, Holderied_2022, Li_et_al_2023} with a finite-element model for the background and a pseudo-particle/PiC model for the correction.

        The fluid background satisfies the full, non-linear, resistive, compressible, Hall MHD equations. \cite{Laakmann_Hu_Farrell_2022} introduces finite-element(-in-space) implicit timesteppers for the incompressible analogue to this system with structure-preserving (SP) properties in the ideal case, alongside parameter-robust preconditioners. We show that these timesteppers can derive from a finite-element-in-time (FET) (and finite-element-in-space) interpretation. The benefits of this reformulation are discussed, including the derivation of timesteppers that are higher order in time, and the quantifiable dissipative SP properties in the non-ideal, resistive case.
        
        We discuss possible options for extending this FET approach to timesteppers for the compressible case.

        The kinetic corrections satisfy linearized Boltzmann equations. Using a Lénard--Bernstein collision operator, these take Fokker--Planck-like forms \cite{Fokker_1914, Planck_1917} wherein pseudo-particles in the numerical model obey the neoclassical transport equations, with particle-independent Brownian drift terms. This offers a rigorous methodology for incorporating collisions into the particle transport model, without coupling the equations of motions for each particle.
        
        Works by Chen, Chacón et al. \cite{Chen_Chacón_Barnes_2011, Chacón_Chen_Barnes_2013, Chen_Chacón_2014, Chen_Chacón_2015} have developed structure-preserving particle pushers for neoclassical transport in the Vlasov equations, derived from Crank--Nicolson integrators. We show these too can can derive from a FET interpretation, similarly offering potential extensions to higher-order-in-time particle pushers. The FET formulation is used also to consider how the stochastic drift terms can be incorporated into the pushers. Stochastic gyrokinetic expansions are also discussed.

        Different options for the numerical implementation of these schemes are considered.

        Due to the efficacy of FET in the development of SP timesteppers for both the fluid and kinetic component, we hope this approach will prove effective in the future for developing SP timesteppers for the full hybrid model. We hope this will give us the opportunity to incorporate previously inaccessible kinetic effects into the highly effective, modern, finite-element MHD models.
    \end{abstract}
    
    
    \newpage
    \tableofcontents
    
    
    \newpage
    \pagenumbering{arabic}
    %\linenumbers\renewcommand\thelinenumber{\color{black!50}\arabic{linenumber}}
            \documentclass[12pt, a4paper]{report}

\input{template/main.tex}

\title{\BA{Title in Progress...}}
\author{Boris Andrews}
\affil{Mathematical Institute, University of Oxford}
\date{\today}


\begin{document}
    \pagenumbering{gobble}
    \maketitle
    
    
    \begin{abstract}
        Magnetic confinement reactors---in particular tokamaks---offer one of the most promising options for achieving practical nuclear fusion, with the potential to provide virtually limitless, clean energy. The theoretical and numerical modeling of tokamak plasmas is simultaneously an essential component of effective reactor design, and a great research barrier. Tokamak operational conditions exhibit comparatively low Knudsen numbers. Kinetic effects, including kinetic waves and instabilities, Landau damping, bump-on-tail instabilities and more, are therefore highly influential in tokamak plasma dynamics. Purely fluid models are inherently incapable of capturing these effects, whereas the high dimensionality in purely kinetic models render them practically intractable for most relevant purposes.

        We consider a $\delta\!f$ decomposition model, with a macroscopic fluid background and microscopic kinetic correction, both fully coupled to each other. A similar manner of discretization is proposed to that used in the recent \texttt{STRUPHY} code \cite{Holderied_Possanner_Wang_2021, Holderied_2022, Li_et_al_2023} with a finite-element model for the background and a pseudo-particle/PiC model for the correction.

        The fluid background satisfies the full, non-linear, resistive, compressible, Hall MHD equations. \cite{Laakmann_Hu_Farrell_2022} introduces finite-element(-in-space) implicit timesteppers for the incompressible analogue to this system with structure-preserving (SP) properties in the ideal case, alongside parameter-robust preconditioners. We show that these timesteppers can derive from a finite-element-in-time (FET) (and finite-element-in-space) interpretation. The benefits of this reformulation are discussed, including the derivation of timesteppers that are higher order in time, and the quantifiable dissipative SP properties in the non-ideal, resistive case.
        
        We discuss possible options for extending this FET approach to timesteppers for the compressible case.

        The kinetic corrections satisfy linearized Boltzmann equations. Using a Lénard--Bernstein collision operator, these take Fokker--Planck-like forms \cite{Fokker_1914, Planck_1917} wherein pseudo-particles in the numerical model obey the neoclassical transport equations, with particle-independent Brownian drift terms. This offers a rigorous methodology for incorporating collisions into the particle transport model, without coupling the equations of motions for each particle.
        
        Works by Chen, Chacón et al. \cite{Chen_Chacón_Barnes_2011, Chacón_Chen_Barnes_2013, Chen_Chacón_2014, Chen_Chacón_2015} have developed structure-preserving particle pushers for neoclassical transport in the Vlasov equations, derived from Crank--Nicolson integrators. We show these too can can derive from a FET interpretation, similarly offering potential extensions to higher-order-in-time particle pushers. The FET formulation is used also to consider how the stochastic drift terms can be incorporated into the pushers. Stochastic gyrokinetic expansions are also discussed.

        Different options for the numerical implementation of these schemes are considered.

        Due to the efficacy of FET in the development of SP timesteppers for both the fluid and kinetic component, we hope this approach will prove effective in the future for developing SP timesteppers for the full hybrid model. We hope this will give us the opportunity to incorporate previously inaccessible kinetic effects into the highly effective, modern, finite-element MHD models.
    \end{abstract}
    
    
    \newpage
    \tableofcontents
    
    
    \newpage
    \pagenumbering{arabic}
    %\linenumbers\renewcommand\thelinenumber{\color{black!50}\arabic{linenumber}}
            \input{0 - introduction/main.tex}
        \part{Research}
            \input{1 - low-noise PiC models/main.tex}
            \input{2 - kinetic component/main.tex}
            \input{3 - fluid component/main.tex}
            \input{4 - numerical implementation/main.tex}
        \part{Project Overview}
            \input{5 - research plan/main.tex}
            \input{6 - summary/main.tex}
    
    
    %\section{}
    \newpage
    \pagenumbering{gobble}
        \printbibliography


    \newpage
    \pagenumbering{roman}
    \appendix
        \part{Appendices}
            \input{8 - Hilbert complexes/main.tex}
            \input{9 - weak conservation proofs/main.tex}
\end{document}

        \part{Research}
            \documentclass[12pt, a4paper]{report}

\input{template/main.tex}

\title{\BA{Title in Progress...}}
\author{Boris Andrews}
\affil{Mathematical Institute, University of Oxford}
\date{\today}


\begin{document}
    \pagenumbering{gobble}
    \maketitle
    
    
    \begin{abstract}
        Magnetic confinement reactors---in particular tokamaks---offer one of the most promising options for achieving practical nuclear fusion, with the potential to provide virtually limitless, clean energy. The theoretical and numerical modeling of tokamak plasmas is simultaneously an essential component of effective reactor design, and a great research barrier. Tokamak operational conditions exhibit comparatively low Knudsen numbers. Kinetic effects, including kinetic waves and instabilities, Landau damping, bump-on-tail instabilities and more, are therefore highly influential in tokamak plasma dynamics. Purely fluid models are inherently incapable of capturing these effects, whereas the high dimensionality in purely kinetic models render them practically intractable for most relevant purposes.

        We consider a $\delta\!f$ decomposition model, with a macroscopic fluid background and microscopic kinetic correction, both fully coupled to each other. A similar manner of discretization is proposed to that used in the recent \texttt{STRUPHY} code \cite{Holderied_Possanner_Wang_2021, Holderied_2022, Li_et_al_2023} with a finite-element model for the background and a pseudo-particle/PiC model for the correction.

        The fluid background satisfies the full, non-linear, resistive, compressible, Hall MHD equations. \cite{Laakmann_Hu_Farrell_2022} introduces finite-element(-in-space) implicit timesteppers for the incompressible analogue to this system with structure-preserving (SP) properties in the ideal case, alongside parameter-robust preconditioners. We show that these timesteppers can derive from a finite-element-in-time (FET) (and finite-element-in-space) interpretation. The benefits of this reformulation are discussed, including the derivation of timesteppers that are higher order in time, and the quantifiable dissipative SP properties in the non-ideal, resistive case.
        
        We discuss possible options for extending this FET approach to timesteppers for the compressible case.

        The kinetic corrections satisfy linearized Boltzmann equations. Using a Lénard--Bernstein collision operator, these take Fokker--Planck-like forms \cite{Fokker_1914, Planck_1917} wherein pseudo-particles in the numerical model obey the neoclassical transport equations, with particle-independent Brownian drift terms. This offers a rigorous methodology for incorporating collisions into the particle transport model, without coupling the equations of motions for each particle.
        
        Works by Chen, Chacón et al. \cite{Chen_Chacón_Barnes_2011, Chacón_Chen_Barnes_2013, Chen_Chacón_2014, Chen_Chacón_2015} have developed structure-preserving particle pushers for neoclassical transport in the Vlasov equations, derived from Crank--Nicolson integrators. We show these too can can derive from a FET interpretation, similarly offering potential extensions to higher-order-in-time particle pushers. The FET formulation is used also to consider how the stochastic drift terms can be incorporated into the pushers. Stochastic gyrokinetic expansions are also discussed.

        Different options for the numerical implementation of these schemes are considered.

        Due to the efficacy of FET in the development of SP timesteppers for both the fluid and kinetic component, we hope this approach will prove effective in the future for developing SP timesteppers for the full hybrid model. We hope this will give us the opportunity to incorporate previously inaccessible kinetic effects into the highly effective, modern, finite-element MHD models.
    \end{abstract}
    
    
    \newpage
    \tableofcontents
    
    
    \newpage
    \pagenumbering{arabic}
    %\linenumbers\renewcommand\thelinenumber{\color{black!50}\arabic{linenumber}}
            \input{0 - introduction/main.tex}
        \part{Research}
            \input{1 - low-noise PiC models/main.tex}
            \input{2 - kinetic component/main.tex}
            \input{3 - fluid component/main.tex}
            \input{4 - numerical implementation/main.tex}
        \part{Project Overview}
            \input{5 - research plan/main.tex}
            \input{6 - summary/main.tex}
    
    
    %\section{}
    \newpage
    \pagenumbering{gobble}
        \printbibliography


    \newpage
    \pagenumbering{roman}
    \appendix
        \part{Appendices}
            \input{8 - Hilbert complexes/main.tex}
            \input{9 - weak conservation proofs/main.tex}
\end{document}

            \documentclass[12pt, a4paper]{report}

\input{template/main.tex}

\title{\BA{Title in Progress...}}
\author{Boris Andrews}
\affil{Mathematical Institute, University of Oxford}
\date{\today}


\begin{document}
    \pagenumbering{gobble}
    \maketitle
    
    
    \begin{abstract}
        Magnetic confinement reactors---in particular tokamaks---offer one of the most promising options for achieving practical nuclear fusion, with the potential to provide virtually limitless, clean energy. The theoretical and numerical modeling of tokamak plasmas is simultaneously an essential component of effective reactor design, and a great research barrier. Tokamak operational conditions exhibit comparatively low Knudsen numbers. Kinetic effects, including kinetic waves and instabilities, Landau damping, bump-on-tail instabilities and more, are therefore highly influential in tokamak plasma dynamics. Purely fluid models are inherently incapable of capturing these effects, whereas the high dimensionality in purely kinetic models render them practically intractable for most relevant purposes.

        We consider a $\delta\!f$ decomposition model, with a macroscopic fluid background and microscopic kinetic correction, both fully coupled to each other. A similar manner of discretization is proposed to that used in the recent \texttt{STRUPHY} code \cite{Holderied_Possanner_Wang_2021, Holderied_2022, Li_et_al_2023} with a finite-element model for the background and a pseudo-particle/PiC model for the correction.

        The fluid background satisfies the full, non-linear, resistive, compressible, Hall MHD equations. \cite{Laakmann_Hu_Farrell_2022} introduces finite-element(-in-space) implicit timesteppers for the incompressible analogue to this system with structure-preserving (SP) properties in the ideal case, alongside parameter-robust preconditioners. We show that these timesteppers can derive from a finite-element-in-time (FET) (and finite-element-in-space) interpretation. The benefits of this reformulation are discussed, including the derivation of timesteppers that are higher order in time, and the quantifiable dissipative SP properties in the non-ideal, resistive case.
        
        We discuss possible options for extending this FET approach to timesteppers for the compressible case.

        The kinetic corrections satisfy linearized Boltzmann equations. Using a Lénard--Bernstein collision operator, these take Fokker--Planck-like forms \cite{Fokker_1914, Planck_1917} wherein pseudo-particles in the numerical model obey the neoclassical transport equations, with particle-independent Brownian drift terms. This offers a rigorous methodology for incorporating collisions into the particle transport model, without coupling the equations of motions for each particle.
        
        Works by Chen, Chacón et al. \cite{Chen_Chacón_Barnes_2011, Chacón_Chen_Barnes_2013, Chen_Chacón_2014, Chen_Chacón_2015} have developed structure-preserving particle pushers for neoclassical transport in the Vlasov equations, derived from Crank--Nicolson integrators. We show these too can can derive from a FET interpretation, similarly offering potential extensions to higher-order-in-time particle pushers. The FET formulation is used also to consider how the stochastic drift terms can be incorporated into the pushers. Stochastic gyrokinetic expansions are also discussed.

        Different options for the numerical implementation of these schemes are considered.

        Due to the efficacy of FET in the development of SP timesteppers for both the fluid and kinetic component, we hope this approach will prove effective in the future for developing SP timesteppers for the full hybrid model. We hope this will give us the opportunity to incorporate previously inaccessible kinetic effects into the highly effective, modern, finite-element MHD models.
    \end{abstract}
    
    
    \newpage
    \tableofcontents
    
    
    \newpage
    \pagenumbering{arabic}
    %\linenumbers\renewcommand\thelinenumber{\color{black!50}\arabic{linenumber}}
            \input{0 - introduction/main.tex}
        \part{Research}
            \input{1 - low-noise PiC models/main.tex}
            \input{2 - kinetic component/main.tex}
            \input{3 - fluid component/main.tex}
            \input{4 - numerical implementation/main.tex}
        \part{Project Overview}
            \input{5 - research plan/main.tex}
            \input{6 - summary/main.tex}
    
    
    %\section{}
    \newpage
    \pagenumbering{gobble}
        \printbibliography


    \newpage
    \pagenumbering{roman}
    \appendix
        \part{Appendices}
            \input{8 - Hilbert complexes/main.tex}
            \input{9 - weak conservation proofs/main.tex}
\end{document}

            \documentclass[12pt, a4paper]{report}

\input{template/main.tex}

\title{\BA{Title in Progress...}}
\author{Boris Andrews}
\affil{Mathematical Institute, University of Oxford}
\date{\today}


\begin{document}
    \pagenumbering{gobble}
    \maketitle
    
    
    \begin{abstract}
        Magnetic confinement reactors---in particular tokamaks---offer one of the most promising options for achieving practical nuclear fusion, with the potential to provide virtually limitless, clean energy. The theoretical and numerical modeling of tokamak plasmas is simultaneously an essential component of effective reactor design, and a great research barrier. Tokamak operational conditions exhibit comparatively low Knudsen numbers. Kinetic effects, including kinetic waves and instabilities, Landau damping, bump-on-tail instabilities and more, are therefore highly influential in tokamak plasma dynamics. Purely fluid models are inherently incapable of capturing these effects, whereas the high dimensionality in purely kinetic models render them practically intractable for most relevant purposes.

        We consider a $\delta\!f$ decomposition model, with a macroscopic fluid background and microscopic kinetic correction, both fully coupled to each other. A similar manner of discretization is proposed to that used in the recent \texttt{STRUPHY} code \cite{Holderied_Possanner_Wang_2021, Holderied_2022, Li_et_al_2023} with a finite-element model for the background and a pseudo-particle/PiC model for the correction.

        The fluid background satisfies the full, non-linear, resistive, compressible, Hall MHD equations. \cite{Laakmann_Hu_Farrell_2022} introduces finite-element(-in-space) implicit timesteppers for the incompressible analogue to this system with structure-preserving (SP) properties in the ideal case, alongside parameter-robust preconditioners. We show that these timesteppers can derive from a finite-element-in-time (FET) (and finite-element-in-space) interpretation. The benefits of this reformulation are discussed, including the derivation of timesteppers that are higher order in time, and the quantifiable dissipative SP properties in the non-ideal, resistive case.
        
        We discuss possible options for extending this FET approach to timesteppers for the compressible case.

        The kinetic corrections satisfy linearized Boltzmann equations. Using a Lénard--Bernstein collision operator, these take Fokker--Planck-like forms \cite{Fokker_1914, Planck_1917} wherein pseudo-particles in the numerical model obey the neoclassical transport equations, with particle-independent Brownian drift terms. This offers a rigorous methodology for incorporating collisions into the particle transport model, without coupling the equations of motions for each particle.
        
        Works by Chen, Chacón et al. \cite{Chen_Chacón_Barnes_2011, Chacón_Chen_Barnes_2013, Chen_Chacón_2014, Chen_Chacón_2015} have developed structure-preserving particle pushers for neoclassical transport in the Vlasov equations, derived from Crank--Nicolson integrators. We show these too can can derive from a FET interpretation, similarly offering potential extensions to higher-order-in-time particle pushers. The FET formulation is used also to consider how the stochastic drift terms can be incorporated into the pushers. Stochastic gyrokinetic expansions are also discussed.

        Different options for the numerical implementation of these schemes are considered.

        Due to the efficacy of FET in the development of SP timesteppers for both the fluid and kinetic component, we hope this approach will prove effective in the future for developing SP timesteppers for the full hybrid model. We hope this will give us the opportunity to incorporate previously inaccessible kinetic effects into the highly effective, modern, finite-element MHD models.
    \end{abstract}
    
    
    \newpage
    \tableofcontents
    
    
    \newpage
    \pagenumbering{arabic}
    %\linenumbers\renewcommand\thelinenumber{\color{black!50}\arabic{linenumber}}
            \input{0 - introduction/main.tex}
        \part{Research}
            \input{1 - low-noise PiC models/main.tex}
            \input{2 - kinetic component/main.tex}
            \input{3 - fluid component/main.tex}
            \input{4 - numerical implementation/main.tex}
        \part{Project Overview}
            \input{5 - research plan/main.tex}
            \input{6 - summary/main.tex}
    
    
    %\section{}
    \newpage
    \pagenumbering{gobble}
        \printbibliography


    \newpage
    \pagenumbering{roman}
    \appendix
        \part{Appendices}
            \input{8 - Hilbert complexes/main.tex}
            \input{9 - weak conservation proofs/main.tex}
\end{document}

            \documentclass[12pt, a4paper]{report}

\input{template/main.tex}

\title{\BA{Title in Progress...}}
\author{Boris Andrews}
\affil{Mathematical Institute, University of Oxford}
\date{\today}


\begin{document}
    \pagenumbering{gobble}
    \maketitle
    
    
    \begin{abstract}
        Magnetic confinement reactors---in particular tokamaks---offer one of the most promising options for achieving practical nuclear fusion, with the potential to provide virtually limitless, clean energy. The theoretical and numerical modeling of tokamak plasmas is simultaneously an essential component of effective reactor design, and a great research barrier. Tokamak operational conditions exhibit comparatively low Knudsen numbers. Kinetic effects, including kinetic waves and instabilities, Landau damping, bump-on-tail instabilities and more, are therefore highly influential in tokamak plasma dynamics. Purely fluid models are inherently incapable of capturing these effects, whereas the high dimensionality in purely kinetic models render them practically intractable for most relevant purposes.

        We consider a $\delta\!f$ decomposition model, with a macroscopic fluid background and microscopic kinetic correction, both fully coupled to each other. A similar manner of discretization is proposed to that used in the recent \texttt{STRUPHY} code \cite{Holderied_Possanner_Wang_2021, Holderied_2022, Li_et_al_2023} with a finite-element model for the background and a pseudo-particle/PiC model for the correction.

        The fluid background satisfies the full, non-linear, resistive, compressible, Hall MHD equations. \cite{Laakmann_Hu_Farrell_2022} introduces finite-element(-in-space) implicit timesteppers for the incompressible analogue to this system with structure-preserving (SP) properties in the ideal case, alongside parameter-robust preconditioners. We show that these timesteppers can derive from a finite-element-in-time (FET) (and finite-element-in-space) interpretation. The benefits of this reformulation are discussed, including the derivation of timesteppers that are higher order in time, and the quantifiable dissipative SP properties in the non-ideal, resistive case.
        
        We discuss possible options for extending this FET approach to timesteppers for the compressible case.

        The kinetic corrections satisfy linearized Boltzmann equations. Using a Lénard--Bernstein collision operator, these take Fokker--Planck-like forms \cite{Fokker_1914, Planck_1917} wherein pseudo-particles in the numerical model obey the neoclassical transport equations, with particle-independent Brownian drift terms. This offers a rigorous methodology for incorporating collisions into the particle transport model, without coupling the equations of motions for each particle.
        
        Works by Chen, Chacón et al. \cite{Chen_Chacón_Barnes_2011, Chacón_Chen_Barnes_2013, Chen_Chacón_2014, Chen_Chacón_2015} have developed structure-preserving particle pushers for neoclassical transport in the Vlasov equations, derived from Crank--Nicolson integrators. We show these too can can derive from a FET interpretation, similarly offering potential extensions to higher-order-in-time particle pushers. The FET formulation is used also to consider how the stochastic drift terms can be incorporated into the pushers. Stochastic gyrokinetic expansions are also discussed.

        Different options for the numerical implementation of these schemes are considered.

        Due to the efficacy of FET in the development of SP timesteppers for both the fluid and kinetic component, we hope this approach will prove effective in the future for developing SP timesteppers for the full hybrid model. We hope this will give us the opportunity to incorporate previously inaccessible kinetic effects into the highly effective, modern, finite-element MHD models.
    \end{abstract}
    
    
    \newpage
    \tableofcontents
    
    
    \newpage
    \pagenumbering{arabic}
    %\linenumbers\renewcommand\thelinenumber{\color{black!50}\arabic{linenumber}}
            \input{0 - introduction/main.tex}
        \part{Research}
            \input{1 - low-noise PiC models/main.tex}
            \input{2 - kinetic component/main.tex}
            \input{3 - fluid component/main.tex}
            \input{4 - numerical implementation/main.tex}
        \part{Project Overview}
            \input{5 - research plan/main.tex}
            \input{6 - summary/main.tex}
    
    
    %\section{}
    \newpage
    \pagenumbering{gobble}
        \printbibliography


    \newpage
    \pagenumbering{roman}
    \appendix
        \part{Appendices}
            \input{8 - Hilbert complexes/main.tex}
            \input{9 - weak conservation proofs/main.tex}
\end{document}

        \part{Project Overview}
            \documentclass[12pt, a4paper]{report}

\input{template/main.tex}

\title{\BA{Title in Progress...}}
\author{Boris Andrews}
\affil{Mathematical Institute, University of Oxford}
\date{\today}


\begin{document}
    \pagenumbering{gobble}
    \maketitle
    
    
    \begin{abstract}
        Magnetic confinement reactors---in particular tokamaks---offer one of the most promising options for achieving practical nuclear fusion, with the potential to provide virtually limitless, clean energy. The theoretical and numerical modeling of tokamak plasmas is simultaneously an essential component of effective reactor design, and a great research barrier. Tokamak operational conditions exhibit comparatively low Knudsen numbers. Kinetic effects, including kinetic waves and instabilities, Landau damping, bump-on-tail instabilities and more, are therefore highly influential in tokamak plasma dynamics. Purely fluid models are inherently incapable of capturing these effects, whereas the high dimensionality in purely kinetic models render them practically intractable for most relevant purposes.

        We consider a $\delta\!f$ decomposition model, with a macroscopic fluid background and microscopic kinetic correction, both fully coupled to each other. A similar manner of discretization is proposed to that used in the recent \texttt{STRUPHY} code \cite{Holderied_Possanner_Wang_2021, Holderied_2022, Li_et_al_2023} with a finite-element model for the background and a pseudo-particle/PiC model for the correction.

        The fluid background satisfies the full, non-linear, resistive, compressible, Hall MHD equations. \cite{Laakmann_Hu_Farrell_2022} introduces finite-element(-in-space) implicit timesteppers for the incompressible analogue to this system with structure-preserving (SP) properties in the ideal case, alongside parameter-robust preconditioners. We show that these timesteppers can derive from a finite-element-in-time (FET) (and finite-element-in-space) interpretation. The benefits of this reformulation are discussed, including the derivation of timesteppers that are higher order in time, and the quantifiable dissipative SP properties in the non-ideal, resistive case.
        
        We discuss possible options for extending this FET approach to timesteppers for the compressible case.

        The kinetic corrections satisfy linearized Boltzmann equations. Using a Lénard--Bernstein collision operator, these take Fokker--Planck-like forms \cite{Fokker_1914, Planck_1917} wherein pseudo-particles in the numerical model obey the neoclassical transport equations, with particle-independent Brownian drift terms. This offers a rigorous methodology for incorporating collisions into the particle transport model, without coupling the equations of motions for each particle.
        
        Works by Chen, Chacón et al. \cite{Chen_Chacón_Barnes_2011, Chacón_Chen_Barnes_2013, Chen_Chacón_2014, Chen_Chacón_2015} have developed structure-preserving particle pushers for neoclassical transport in the Vlasov equations, derived from Crank--Nicolson integrators. We show these too can can derive from a FET interpretation, similarly offering potential extensions to higher-order-in-time particle pushers. The FET formulation is used also to consider how the stochastic drift terms can be incorporated into the pushers. Stochastic gyrokinetic expansions are also discussed.

        Different options for the numerical implementation of these schemes are considered.

        Due to the efficacy of FET in the development of SP timesteppers for both the fluid and kinetic component, we hope this approach will prove effective in the future for developing SP timesteppers for the full hybrid model. We hope this will give us the opportunity to incorporate previously inaccessible kinetic effects into the highly effective, modern, finite-element MHD models.
    \end{abstract}
    
    
    \newpage
    \tableofcontents
    
    
    \newpage
    \pagenumbering{arabic}
    %\linenumbers\renewcommand\thelinenumber{\color{black!50}\arabic{linenumber}}
            \input{0 - introduction/main.tex}
        \part{Research}
            \input{1 - low-noise PiC models/main.tex}
            \input{2 - kinetic component/main.tex}
            \input{3 - fluid component/main.tex}
            \input{4 - numerical implementation/main.tex}
        \part{Project Overview}
            \input{5 - research plan/main.tex}
            \input{6 - summary/main.tex}
    
    
    %\section{}
    \newpage
    \pagenumbering{gobble}
        \printbibliography


    \newpage
    \pagenumbering{roman}
    \appendix
        \part{Appendices}
            \input{8 - Hilbert complexes/main.tex}
            \input{9 - weak conservation proofs/main.tex}
\end{document}

            \documentclass[12pt, a4paper]{report}

\input{template/main.tex}

\title{\BA{Title in Progress...}}
\author{Boris Andrews}
\affil{Mathematical Institute, University of Oxford}
\date{\today}


\begin{document}
    \pagenumbering{gobble}
    \maketitle
    
    
    \begin{abstract}
        Magnetic confinement reactors---in particular tokamaks---offer one of the most promising options for achieving practical nuclear fusion, with the potential to provide virtually limitless, clean energy. The theoretical and numerical modeling of tokamak plasmas is simultaneously an essential component of effective reactor design, and a great research barrier. Tokamak operational conditions exhibit comparatively low Knudsen numbers. Kinetic effects, including kinetic waves and instabilities, Landau damping, bump-on-tail instabilities and more, are therefore highly influential in tokamak plasma dynamics. Purely fluid models are inherently incapable of capturing these effects, whereas the high dimensionality in purely kinetic models render them practically intractable for most relevant purposes.

        We consider a $\delta\!f$ decomposition model, with a macroscopic fluid background and microscopic kinetic correction, both fully coupled to each other. A similar manner of discretization is proposed to that used in the recent \texttt{STRUPHY} code \cite{Holderied_Possanner_Wang_2021, Holderied_2022, Li_et_al_2023} with a finite-element model for the background and a pseudo-particle/PiC model for the correction.

        The fluid background satisfies the full, non-linear, resistive, compressible, Hall MHD equations. \cite{Laakmann_Hu_Farrell_2022} introduces finite-element(-in-space) implicit timesteppers for the incompressible analogue to this system with structure-preserving (SP) properties in the ideal case, alongside parameter-robust preconditioners. We show that these timesteppers can derive from a finite-element-in-time (FET) (and finite-element-in-space) interpretation. The benefits of this reformulation are discussed, including the derivation of timesteppers that are higher order in time, and the quantifiable dissipative SP properties in the non-ideal, resistive case.
        
        We discuss possible options for extending this FET approach to timesteppers for the compressible case.

        The kinetic corrections satisfy linearized Boltzmann equations. Using a Lénard--Bernstein collision operator, these take Fokker--Planck-like forms \cite{Fokker_1914, Planck_1917} wherein pseudo-particles in the numerical model obey the neoclassical transport equations, with particle-independent Brownian drift terms. This offers a rigorous methodology for incorporating collisions into the particle transport model, without coupling the equations of motions for each particle.
        
        Works by Chen, Chacón et al. \cite{Chen_Chacón_Barnes_2011, Chacón_Chen_Barnes_2013, Chen_Chacón_2014, Chen_Chacón_2015} have developed structure-preserving particle pushers for neoclassical transport in the Vlasov equations, derived from Crank--Nicolson integrators. We show these too can can derive from a FET interpretation, similarly offering potential extensions to higher-order-in-time particle pushers. The FET formulation is used also to consider how the stochastic drift terms can be incorporated into the pushers. Stochastic gyrokinetic expansions are also discussed.

        Different options for the numerical implementation of these schemes are considered.

        Due to the efficacy of FET in the development of SP timesteppers for both the fluid and kinetic component, we hope this approach will prove effective in the future for developing SP timesteppers for the full hybrid model. We hope this will give us the opportunity to incorporate previously inaccessible kinetic effects into the highly effective, modern, finite-element MHD models.
    \end{abstract}
    
    
    \newpage
    \tableofcontents
    
    
    \newpage
    \pagenumbering{arabic}
    %\linenumbers\renewcommand\thelinenumber{\color{black!50}\arabic{linenumber}}
            \input{0 - introduction/main.tex}
        \part{Research}
            \input{1 - low-noise PiC models/main.tex}
            \input{2 - kinetic component/main.tex}
            \input{3 - fluid component/main.tex}
            \input{4 - numerical implementation/main.tex}
        \part{Project Overview}
            \input{5 - research plan/main.tex}
            \input{6 - summary/main.tex}
    
    
    %\section{}
    \newpage
    \pagenumbering{gobble}
        \printbibliography


    \newpage
    \pagenumbering{roman}
    \appendix
        \part{Appendices}
            \input{8 - Hilbert complexes/main.tex}
            \input{9 - weak conservation proofs/main.tex}
\end{document}

    
    
    %\section{}
    \newpage
    \pagenumbering{gobble}
        \printbibliography


    \newpage
    \pagenumbering{roman}
    \appendix
        \part{Appendices}
            \documentclass[12pt, a4paper]{report}

\input{template/main.tex}

\title{\BA{Title in Progress...}}
\author{Boris Andrews}
\affil{Mathematical Institute, University of Oxford}
\date{\today}


\begin{document}
    \pagenumbering{gobble}
    \maketitle
    
    
    \begin{abstract}
        Magnetic confinement reactors---in particular tokamaks---offer one of the most promising options for achieving practical nuclear fusion, with the potential to provide virtually limitless, clean energy. The theoretical and numerical modeling of tokamak plasmas is simultaneously an essential component of effective reactor design, and a great research barrier. Tokamak operational conditions exhibit comparatively low Knudsen numbers. Kinetic effects, including kinetic waves and instabilities, Landau damping, bump-on-tail instabilities and more, are therefore highly influential in tokamak plasma dynamics. Purely fluid models are inherently incapable of capturing these effects, whereas the high dimensionality in purely kinetic models render them practically intractable for most relevant purposes.

        We consider a $\delta\!f$ decomposition model, with a macroscopic fluid background and microscopic kinetic correction, both fully coupled to each other. A similar manner of discretization is proposed to that used in the recent \texttt{STRUPHY} code \cite{Holderied_Possanner_Wang_2021, Holderied_2022, Li_et_al_2023} with a finite-element model for the background and a pseudo-particle/PiC model for the correction.

        The fluid background satisfies the full, non-linear, resistive, compressible, Hall MHD equations. \cite{Laakmann_Hu_Farrell_2022} introduces finite-element(-in-space) implicit timesteppers for the incompressible analogue to this system with structure-preserving (SP) properties in the ideal case, alongside parameter-robust preconditioners. We show that these timesteppers can derive from a finite-element-in-time (FET) (and finite-element-in-space) interpretation. The benefits of this reformulation are discussed, including the derivation of timesteppers that are higher order in time, and the quantifiable dissipative SP properties in the non-ideal, resistive case.
        
        We discuss possible options for extending this FET approach to timesteppers for the compressible case.

        The kinetic corrections satisfy linearized Boltzmann equations. Using a Lénard--Bernstein collision operator, these take Fokker--Planck-like forms \cite{Fokker_1914, Planck_1917} wherein pseudo-particles in the numerical model obey the neoclassical transport equations, with particle-independent Brownian drift terms. This offers a rigorous methodology for incorporating collisions into the particle transport model, without coupling the equations of motions for each particle.
        
        Works by Chen, Chacón et al. \cite{Chen_Chacón_Barnes_2011, Chacón_Chen_Barnes_2013, Chen_Chacón_2014, Chen_Chacón_2015} have developed structure-preserving particle pushers for neoclassical transport in the Vlasov equations, derived from Crank--Nicolson integrators. We show these too can can derive from a FET interpretation, similarly offering potential extensions to higher-order-in-time particle pushers. The FET formulation is used also to consider how the stochastic drift terms can be incorporated into the pushers. Stochastic gyrokinetic expansions are also discussed.

        Different options for the numerical implementation of these schemes are considered.

        Due to the efficacy of FET in the development of SP timesteppers for both the fluid and kinetic component, we hope this approach will prove effective in the future for developing SP timesteppers for the full hybrid model. We hope this will give us the opportunity to incorporate previously inaccessible kinetic effects into the highly effective, modern, finite-element MHD models.
    \end{abstract}
    
    
    \newpage
    \tableofcontents
    
    
    \newpage
    \pagenumbering{arabic}
    %\linenumbers\renewcommand\thelinenumber{\color{black!50}\arabic{linenumber}}
            \input{0 - introduction/main.tex}
        \part{Research}
            \input{1 - low-noise PiC models/main.tex}
            \input{2 - kinetic component/main.tex}
            \input{3 - fluid component/main.tex}
            \input{4 - numerical implementation/main.tex}
        \part{Project Overview}
            \input{5 - research plan/main.tex}
            \input{6 - summary/main.tex}
    
    
    %\section{}
    \newpage
    \pagenumbering{gobble}
        \printbibliography


    \newpage
    \pagenumbering{roman}
    \appendix
        \part{Appendices}
            \input{8 - Hilbert complexes/main.tex}
            \input{9 - weak conservation proofs/main.tex}
\end{document}

            \documentclass[12pt, a4paper]{report}

\input{template/main.tex}

\title{\BA{Title in Progress...}}
\author{Boris Andrews}
\affil{Mathematical Institute, University of Oxford}
\date{\today}


\begin{document}
    \pagenumbering{gobble}
    \maketitle
    
    
    \begin{abstract}
        Magnetic confinement reactors---in particular tokamaks---offer one of the most promising options for achieving practical nuclear fusion, with the potential to provide virtually limitless, clean energy. The theoretical and numerical modeling of tokamak plasmas is simultaneously an essential component of effective reactor design, and a great research barrier. Tokamak operational conditions exhibit comparatively low Knudsen numbers. Kinetic effects, including kinetic waves and instabilities, Landau damping, bump-on-tail instabilities and more, are therefore highly influential in tokamak plasma dynamics. Purely fluid models are inherently incapable of capturing these effects, whereas the high dimensionality in purely kinetic models render them practically intractable for most relevant purposes.

        We consider a $\delta\!f$ decomposition model, with a macroscopic fluid background and microscopic kinetic correction, both fully coupled to each other. A similar manner of discretization is proposed to that used in the recent \texttt{STRUPHY} code \cite{Holderied_Possanner_Wang_2021, Holderied_2022, Li_et_al_2023} with a finite-element model for the background and a pseudo-particle/PiC model for the correction.

        The fluid background satisfies the full, non-linear, resistive, compressible, Hall MHD equations. \cite{Laakmann_Hu_Farrell_2022} introduces finite-element(-in-space) implicit timesteppers for the incompressible analogue to this system with structure-preserving (SP) properties in the ideal case, alongside parameter-robust preconditioners. We show that these timesteppers can derive from a finite-element-in-time (FET) (and finite-element-in-space) interpretation. The benefits of this reformulation are discussed, including the derivation of timesteppers that are higher order in time, and the quantifiable dissipative SP properties in the non-ideal, resistive case.
        
        We discuss possible options for extending this FET approach to timesteppers for the compressible case.

        The kinetic corrections satisfy linearized Boltzmann equations. Using a Lénard--Bernstein collision operator, these take Fokker--Planck-like forms \cite{Fokker_1914, Planck_1917} wherein pseudo-particles in the numerical model obey the neoclassical transport equations, with particle-independent Brownian drift terms. This offers a rigorous methodology for incorporating collisions into the particle transport model, without coupling the equations of motions for each particle.
        
        Works by Chen, Chacón et al. \cite{Chen_Chacón_Barnes_2011, Chacón_Chen_Barnes_2013, Chen_Chacón_2014, Chen_Chacón_2015} have developed structure-preserving particle pushers for neoclassical transport in the Vlasov equations, derived from Crank--Nicolson integrators. We show these too can can derive from a FET interpretation, similarly offering potential extensions to higher-order-in-time particle pushers. The FET formulation is used also to consider how the stochastic drift terms can be incorporated into the pushers. Stochastic gyrokinetic expansions are also discussed.

        Different options for the numerical implementation of these schemes are considered.

        Due to the efficacy of FET in the development of SP timesteppers for both the fluid and kinetic component, we hope this approach will prove effective in the future for developing SP timesteppers for the full hybrid model. We hope this will give us the opportunity to incorporate previously inaccessible kinetic effects into the highly effective, modern, finite-element MHD models.
    \end{abstract}
    
    
    \newpage
    \tableofcontents
    
    
    \newpage
    \pagenumbering{arabic}
    %\linenumbers\renewcommand\thelinenumber{\color{black!50}\arabic{linenumber}}
            \input{0 - introduction/main.tex}
        \part{Research}
            \input{1 - low-noise PiC models/main.tex}
            \input{2 - kinetic component/main.tex}
            \input{3 - fluid component/main.tex}
            \input{4 - numerical implementation/main.tex}
        \part{Project Overview}
            \input{5 - research plan/main.tex}
            \input{6 - summary/main.tex}
    
    
    %\section{}
    \newpage
    \pagenumbering{gobble}
        \printbibliography


    \newpage
    \pagenumbering{roman}
    \appendix
        \part{Appendices}
            \input{8 - Hilbert complexes/main.tex}
            \input{9 - weak conservation proofs/main.tex}
\end{document}

\end{document}

            \documentclass[12pt, a4paper]{report}

\documentclass[12pt, a4paper]{report}

\input{template/main.tex}

\title{\BA{Title in Progress...}}
\author{Boris Andrews}
\affil{Mathematical Institute, University of Oxford}
\date{\today}


\begin{document}
    \pagenumbering{gobble}
    \maketitle
    
    
    \begin{abstract}
        Magnetic confinement reactors---in particular tokamaks---offer one of the most promising options for achieving practical nuclear fusion, with the potential to provide virtually limitless, clean energy. The theoretical and numerical modeling of tokamak plasmas is simultaneously an essential component of effective reactor design, and a great research barrier. Tokamak operational conditions exhibit comparatively low Knudsen numbers. Kinetic effects, including kinetic waves and instabilities, Landau damping, bump-on-tail instabilities and more, are therefore highly influential in tokamak plasma dynamics. Purely fluid models are inherently incapable of capturing these effects, whereas the high dimensionality in purely kinetic models render them practically intractable for most relevant purposes.

        We consider a $\delta\!f$ decomposition model, with a macroscopic fluid background and microscopic kinetic correction, both fully coupled to each other. A similar manner of discretization is proposed to that used in the recent \texttt{STRUPHY} code \cite{Holderied_Possanner_Wang_2021, Holderied_2022, Li_et_al_2023} with a finite-element model for the background and a pseudo-particle/PiC model for the correction.

        The fluid background satisfies the full, non-linear, resistive, compressible, Hall MHD equations. \cite{Laakmann_Hu_Farrell_2022} introduces finite-element(-in-space) implicit timesteppers for the incompressible analogue to this system with structure-preserving (SP) properties in the ideal case, alongside parameter-robust preconditioners. We show that these timesteppers can derive from a finite-element-in-time (FET) (and finite-element-in-space) interpretation. The benefits of this reformulation are discussed, including the derivation of timesteppers that are higher order in time, and the quantifiable dissipative SP properties in the non-ideal, resistive case.
        
        We discuss possible options for extending this FET approach to timesteppers for the compressible case.

        The kinetic corrections satisfy linearized Boltzmann equations. Using a Lénard--Bernstein collision operator, these take Fokker--Planck-like forms \cite{Fokker_1914, Planck_1917} wherein pseudo-particles in the numerical model obey the neoclassical transport equations, with particle-independent Brownian drift terms. This offers a rigorous methodology for incorporating collisions into the particle transport model, without coupling the equations of motions for each particle.
        
        Works by Chen, Chacón et al. \cite{Chen_Chacón_Barnes_2011, Chacón_Chen_Barnes_2013, Chen_Chacón_2014, Chen_Chacón_2015} have developed structure-preserving particle pushers for neoclassical transport in the Vlasov equations, derived from Crank--Nicolson integrators. We show these too can can derive from a FET interpretation, similarly offering potential extensions to higher-order-in-time particle pushers. The FET formulation is used also to consider how the stochastic drift terms can be incorporated into the pushers. Stochastic gyrokinetic expansions are also discussed.

        Different options for the numerical implementation of these schemes are considered.

        Due to the efficacy of FET in the development of SP timesteppers for both the fluid and kinetic component, we hope this approach will prove effective in the future for developing SP timesteppers for the full hybrid model. We hope this will give us the opportunity to incorporate previously inaccessible kinetic effects into the highly effective, modern, finite-element MHD models.
    \end{abstract}
    
    
    \newpage
    \tableofcontents
    
    
    \newpage
    \pagenumbering{arabic}
    %\linenumbers\renewcommand\thelinenumber{\color{black!50}\arabic{linenumber}}
            \input{0 - introduction/main.tex}
        \part{Research}
            \input{1 - low-noise PiC models/main.tex}
            \input{2 - kinetic component/main.tex}
            \input{3 - fluid component/main.tex}
            \input{4 - numerical implementation/main.tex}
        \part{Project Overview}
            \input{5 - research plan/main.tex}
            \input{6 - summary/main.tex}
    
    
    %\section{}
    \newpage
    \pagenumbering{gobble}
        \printbibliography


    \newpage
    \pagenumbering{roman}
    \appendix
        \part{Appendices}
            \input{8 - Hilbert complexes/main.tex}
            \input{9 - weak conservation proofs/main.tex}
\end{document}


\title{\BA{Title in Progress...}}
\author{Boris Andrews}
\affil{Mathematical Institute, University of Oxford}
\date{\today}


\begin{document}
    \pagenumbering{gobble}
    \maketitle
    
    
    \begin{abstract}
        Magnetic confinement reactors---in particular tokamaks---offer one of the most promising options for achieving practical nuclear fusion, with the potential to provide virtually limitless, clean energy. The theoretical and numerical modeling of tokamak plasmas is simultaneously an essential component of effective reactor design, and a great research barrier. Tokamak operational conditions exhibit comparatively low Knudsen numbers. Kinetic effects, including kinetic waves and instabilities, Landau damping, bump-on-tail instabilities and more, are therefore highly influential in tokamak plasma dynamics. Purely fluid models are inherently incapable of capturing these effects, whereas the high dimensionality in purely kinetic models render them practically intractable for most relevant purposes.

        We consider a $\delta\!f$ decomposition model, with a macroscopic fluid background and microscopic kinetic correction, both fully coupled to each other. A similar manner of discretization is proposed to that used in the recent \texttt{STRUPHY} code \cite{Holderied_Possanner_Wang_2021, Holderied_2022, Li_et_al_2023} with a finite-element model for the background and a pseudo-particle/PiC model for the correction.

        The fluid background satisfies the full, non-linear, resistive, compressible, Hall MHD equations. \cite{Laakmann_Hu_Farrell_2022} introduces finite-element(-in-space) implicit timesteppers for the incompressible analogue to this system with structure-preserving (SP) properties in the ideal case, alongside parameter-robust preconditioners. We show that these timesteppers can derive from a finite-element-in-time (FET) (and finite-element-in-space) interpretation. The benefits of this reformulation are discussed, including the derivation of timesteppers that are higher order in time, and the quantifiable dissipative SP properties in the non-ideal, resistive case.
        
        We discuss possible options for extending this FET approach to timesteppers for the compressible case.

        The kinetic corrections satisfy linearized Boltzmann equations. Using a Lénard--Bernstein collision operator, these take Fokker--Planck-like forms \cite{Fokker_1914, Planck_1917} wherein pseudo-particles in the numerical model obey the neoclassical transport equations, with particle-independent Brownian drift terms. This offers a rigorous methodology for incorporating collisions into the particle transport model, without coupling the equations of motions for each particle.
        
        Works by Chen, Chacón et al. \cite{Chen_Chacón_Barnes_2011, Chacón_Chen_Barnes_2013, Chen_Chacón_2014, Chen_Chacón_2015} have developed structure-preserving particle pushers for neoclassical transport in the Vlasov equations, derived from Crank--Nicolson integrators. We show these too can can derive from a FET interpretation, similarly offering potential extensions to higher-order-in-time particle pushers. The FET formulation is used also to consider how the stochastic drift terms can be incorporated into the pushers. Stochastic gyrokinetic expansions are also discussed.

        Different options for the numerical implementation of these schemes are considered.

        Due to the efficacy of FET in the development of SP timesteppers for both the fluid and kinetic component, we hope this approach will prove effective in the future for developing SP timesteppers for the full hybrid model. We hope this will give us the opportunity to incorporate previously inaccessible kinetic effects into the highly effective, modern, finite-element MHD models.
    \end{abstract}
    
    
    \newpage
    \tableofcontents
    
    
    \newpage
    \pagenumbering{arabic}
    %\linenumbers\renewcommand\thelinenumber{\color{black!50}\arabic{linenumber}}
            \documentclass[12pt, a4paper]{report}

\input{template/main.tex}

\title{\BA{Title in Progress...}}
\author{Boris Andrews}
\affil{Mathematical Institute, University of Oxford}
\date{\today}


\begin{document}
    \pagenumbering{gobble}
    \maketitle
    
    
    \begin{abstract}
        Magnetic confinement reactors---in particular tokamaks---offer one of the most promising options for achieving practical nuclear fusion, with the potential to provide virtually limitless, clean energy. The theoretical and numerical modeling of tokamak plasmas is simultaneously an essential component of effective reactor design, and a great research barrier. Tokamak operational conditions exhibit comparatively low Knudsen numbers. Kinetic effects, including kinetic waves and instabilities, Landau damping, bump-on-tail instabilities and more, are therefore highly influential in tokamak plasma dynamics. Purely fluid models are inherently incapable of capturing these effects, whereas the high dimensionality in purely kinetic models render them practically intractable for most relevant purposes.

        We consider a $\delta\!f$ decomposition model, with a macroscopic fluid background and microscopic kinetic correction, both fully coupled to each other. A similar manner of discretization is proposed to that used in the recent \texttt{STRUPHY} code \cite{Holderied_Possanner_Wang_2021, Holderied_2022, Li_et_al_2023} with a finite-element model for the background and a pseudo-particle/PiC model for the correction.

        The fluid background satisfies the full, non-linear, resistive, compressible, Hall MHD equations. \cite{Laakmann_Hu_Farrell_2022} introduces finite-element(-in-space) implicit timesteppers for the incompressible analogue to this system with structure-preserving (SP) properties in the ideal case, alongside parameter-robust preconditioners. We show that these timesteppers can derive from a finite-element-in-time (FET) (and finite-element-in-space) interpretation. The benefits of this reformulation are discussed, including the derivation of timesteppers that are higher order in time, and the quantifiable dissipative SP properties in the non-ideal, resistive case.
        
        We discuss possible options for extending this FET approach to timesteppers for the compressible case.

        The kinetic corrections satisfy linearized Boltzmann equations. Using a Lénard--Bernstein collision operator, these take Fokker--Planck-like forms \cite{Fokker_1914, Planck_1917} wherein pseudo-particles in the numerical model obey the neoclassical transport equations, with particle-independent Brownian drift terms. This offers a rigorous methodology for incorporating collisions into the particle transport model, without coupling the equations of motions for each particle.
        
        Works by Chen, Chacón et al. \cite{Chen_Chacón_Barnes_2011, Chacón_Chen_Barnes_2013, Chen_Chacón_2014, Chen_Chacón_2015} have developed structure-preserving particle pushers for neoclassical transport in the Vlasov equations, derived from Crank--Nicolson integrators. We show these too can can derive from a FET interpretation, similarly offering potential extensions to higher-order-in-time particle pushers. The FET formulation is used also to consider how the stochastic drift terms can be incorporated into the pushers. Stochastic gyrokinetic expansions are also discussed.

        Different options for the numerical implementation of these schemes are considered.

        Due to the efficacy of FET in the development of SP timesteppers for both the fluid and kinetic component, we hope this approach will prove effective in the future for developing SP timesteppers for the full hybrid model. We hope this will give us the opportunity to incorporate previously inaccessible kinetic effects into the highly effective, modern, finite-element MHD models.
    \end{abstract}
    
    
    \newpage
    \tableofcontents
    
    
    \newpage
    \pagenumbering{arabic}
    %\linenumbers\renewcommand\thelinenumber{\color{black!50}\arabic{linenumber}}
            \input{0 - introduction/main.tex}
        \part{Research}
            \input{1 - low-noise PiC models/main.tex}
            \input{2 - kinetic component/main.tex}
            \input{3 - fluid component/main.tex}
            \input{4 - numerical implementation/main.tex}
        \part{Project Overview}
            \input{5 - research plan/main.tex}
            \input{6 - summary/main.tex}
    
    
    %\section{}
    \newpage
    \pagenumbering{gobble}
        \printbibliography


    \newpage
    \pagenumbering{roman}
    \appendix
        \part{Appendices}
            \input{8 - Hilbert complexes/main.tex}
            \input{9 - weak conservation proofs/main.tex}
\end{document}

        \part{Research}
            \documentclass[12pt, a4paper]{report}

\input{template/main.tex}

\title{\BA{Title in Progress...}}
\author{Boris Andrews}
\affil{Mathematical Institute, University of Oxford}
\date{\today}


\begin{document}
    \pagenumbering{gobble}
    \maketitle
    
    
    \begin{abstract}
        Magnetic confinement reactors---in particular tokamaks---offer one of the most promising options for achieving practical nuclear fusion, with the potential to provide virtually limitless, clean energy. The theoretical and numerical modeling of tokamak plasmas is simultaneously an essential component of effective reactor design, and a great research barrier. Tokamak operational conditions exhibit comparatively low Knudsen numbers. Kinetic effects, including kinetic waves and instabilities, Landau damping, bump-on-tail instabilities and more, are therefore highly influential in tokamak plasma dynamics. Purely fluid models are inherently incapable of capturing these effects, whereas the high dimensionality in purely kinetic models render them practically intractable for most relevant purposes.

        We consider a $\delta\!f$ decomposition model, with a macroscopic fluid background and microscopic kinetic correction, both fully coupled to each other. A similar manner of discretization is proposed to that used in the recent \texttt{STRUPHY} code \cite{Holderied_Possanner_Wang_2021, Holderied_2022, Li_et_al_2023} with a finite-element model for the background and a pseudo-particle/PiC model for the correction.

        The fluid background satisfies the full, non-linear, resistive, compressible, Hall MHD equations. \cite{Laakmann_Hu_Farrell_2022} introduces finite-element(-in-space) implicit timesteppers for the incompressible analogue to this system with structure-preserving (SP) properties in the ideal case, alongside parameter-robust preconditioners. We show that these timesteppers can derive from a finite-element-in-time (FET) (and finite-element-in-space) interpretation. The benefits of this reformulation are discussed, including the derivation of timesteppers that are higher order in time, and the quantifiable dissipative SP properties in the non-ideal, resistive case.
        
        We discuss possible options for extending this FET approach to timesteppers for the compressible case.

        The kinetic corrections satisfy linearized Boltzmann equations. Using a Lénard--Bernstein collision operator, these take Fokker--Planck-like forms \cite{Fokker_1914, Planck_1917} wherein pseudo-particles in the numerical model obey the neoclassical transport equations, with particle-independent Brownian drift terms. This offers a rigorous methodology for incorporating collisions into the particle transport model, without coupling the equations of motions for each particle.
        
        Works by Chen, Chacón et al. \cite{Chen_Chacón_Barnes_2011, Chacón_Chen_Barnes_2013, Chen_Chacón_2014, Chen_Chacón_2015} have developed structure-preserving particle pushers for neoclassical transport in the Vlasov equations, derived from Crank--Nicolson integrators. We show these too can can derive from a FET interpretation, similarly offering potential extensions to higher-order-in-time particle pushers. The FET formulation is used also to consider how the stochastic drift terms can be incorporated into the pushers. Stochastic gyrokinetic expansions are also discussed.

        Different options for the numerical implementation of these schemes are considered.

        Due to the efficacy of FET in the development of SP timesteppers for both the fluid and kinetic component, we hope this approach will prove effective in the future for developing SP timesteppers for the full hybrid model. We hope this will give us the opportunity to incorporate previously inaccessible kinetic effects into the highly effective, modern, finite-element MHD models.
    \end{abstract}
    
    
    \newpage
    \tableofcontents
    
    
    \newpage
    \pagenumbering{arabic}
    %\linenumbers\renewcommand\thelinenumber{\color{black!50}\arabic{linenumber}}
            \input{0 - introduction/main.tex}
        \part{Research}
            \input{1 - low-noise PiC models/main.tex}
            \input{2 - kinetic component/main.tex}
            \input{3 - fluid component/main.tex}
            \input{4 - numerical implementation/main.tex}
        \part{Project Overview}
            \input{5 - research plan/main.tex}
            \input{6 - summary/main.tex}
    
    
    %\section{}
    \newpage
    \pagenumbering{gobble}
        \printbibliography


    \newpage
    \pagenumbering{roman}
    \appendix
        \part{Appendices}
            \input{8 - Hilbert complexes/main.tex}
            \input{9 - weak conservation proofs/main.tex}
\end{document}

            \documentclass[12pt, a4paper]{report}

\input{template/main.tex}

\title{\BA{Title in Progress...}}
\author{Boris Andrews}
\affil{Mathematical Institute, University of Oxford}
\date{\today}


\begin{document}
    \pagenumbering{gobble}
    \maketitle
    
    
    \begin{abstract}
        Magnetic confinement reactors---in particular tokamaks---offer one of the most promising options for achieving practical nuclear fusion, with the potential to provide virtually limitless, clean energy. The theoretical and numerical modeling of tokamak plasmas is simultaneously an essential component of effective reactor design, and a great research barrier. Tokamak operational conditions exhibit comparatively low Knudsen numbers. Kinetic effects, including kinetic waves and instabilities, Landau damping, bump-on-tail instabilities and more, are therefore highly influential in tokamak plasma dynamics. Purely fluid models are inherently incapable of capturing these effects, whereas the high dimensionality in purely kinetic models render them practically intractable for most relevant purposes.

        We consider a $\delta\!f$ decomposition model, with a macroscopic fluid background and microscopic kinetic correction, both fully coupled to each other. A similar manner of discretization is proposed to that used in the recent \texttt{STRUPHY} code \cite{Holderied_Possanner_Wang_2021, Holderied_2022, Li_et_al_2023} with a finite-element model for the background and a pseudo-particle/PiC model for the correction.

        The fluid background satisfies the full, non-linear, resistive, compressible, Hall MHD equations. \cite{Laakmann_Hu_Farrell_2022} introduces finite-element(-in-space) implicit timesteppers for the incompressible analogue to this system with structure-preserving (SP) properties in the ideal case, alongside parameter-robust preconditioners. We show that these timesteppers can derive from a finite-element-in-time (FET) (and finite-element-in-space) interpretation. The benefits of this reformulation are discussed, including the derivation of timesteppers that are higher order in time, and the quantifiable dissipative SP properties in the non-ideal, resistive case.
        
        We discuss possible options for extending this FET approach to timesteppers for the compressible case.

        The kinetic corrections satisfy linearized Boltzmann equations. Using a Lénard--Bernstein collision operator, these take Fokker--Planck-like forms \cite{Fokker_1914, Planck_1917} wherein pseudo-particles in the numerical model obey the neoclassical transport equations, with particle-independent Brownian drift terms. This offers a rigorous methodology for incorporating collisions into the particle transport model, without coupling the equations of motions for each particle.
        
        Works by Chen, Chacón et al. \cite{Chen_Chacón_Barnes_2011, Chacón_Chen_Barnes_2013, Chen_Chacón_2014, Chen_Chacón_2015} have developed structure-preserving particle pushers for neoclassical transport in the Vlasov equations, derived from Crank--Nicolson integrators. We show these too can can derive from a FET interpretation, similarly offering potential extensions to higher-order-in-time particle pushers. The FET formulation is used also to consider how the stochastic drift terms can be incorporated into the pushers. Stochastic gyrokinetic expansions are also discussed.

        Different options for the numerical implementation of these schemes are considered.

        Due to the efficacy of FET in the development of SP timesteppers for both the fluid and kinetic component, we hope this approach will prove effective in the future for developing SP timesteppers for the full hybrid model. We hope this will give us the opportunity to incorporate previously inaccessible kinetic effects into the highly effective, modern, finite-element MHD models.
    \end{abstract}
    
    
    \newpage
    \tableofcontents
    
    
    \newpage
    \pagenumbering{arabic}
    %\linenumbers\renewcommand\thelinenumber{\color{black!50}\arabic{linenumber}}
            \input{0 - introduction/main.tex}
        \part{Research}
            \input{1 - low-noise PiC models/main.tex}
            \input{2 - kinetic component/main.tex}
            \input{3 - fluid component/main.tex}
            \input{4 - numerical implementation/main.tex}
        \part{Project Overview}
            \input{5 - research plan/main.tex}
            \input{6 - summary/main.tex}
    
    
    %\section{}
    \newpage
    \pagenumbering{gobble}
        \printbibliography


    \newpage
    \pagenumbering{roman}
    \appendix
        \part{Appendices}
            \input{8 - Hilbert complexes/main.tex}
            \input{9 - weak conservation proofs/main.tex}
\end{document}

            \documentclass[12pt, a4paper]{report}

\input{template/main.tex}

\title{\BA{Title in Progress...}}
\author{Boris Andrews}
\affil{Mathematical Institute, University of Oxford}
\date{\today}


\begin{document}
    \pagenumbering{gobble}
    \maketitle
    
    
    \begin{abstract}
        Magnetic confinement reactors---in particular tokamaks---offer one of the most promising options for achieving practical nuclear fusion, with the potential to provide virtually limitless, clean energy. The theoretical and numerical modeling of tokamak plasmas is simultaneously an essential component of effective reactor design, and a great research barrier. Tokamak operational conditions exhibit comparatively low Knudsen numbers. Kinetic effects, including kinetic waves and instabilities, Landau damping, bump-on-tail instabilities and more, are therefore highly influential in tokamak plasma dynamics. Purely fluid models are inherently incapable of capturing these effects, whereas the high dimensionality in purely kinetic models render them practically intractable for most relevant purposes.

        We consider a $\delta\!f$ decomposition model, with a macroscopic fluid background and microscopic kinetic correction, both fully coupled to each other. A similar manner of discretization is proposed to that used in the recent \texttt{STRUPHY} code \cite{Holderied_Possanner_Wang_2021, Holderied_2022, Li_et_al_2023} with a finite-element model for the background and a pseudo-particle/PiC model for the correction.

        The fluid background satisfies the full, non-linear, resistive, compressible, Hall MHD equations. \cite{Laakmann_Hu_Farrell_2022} introduces finite-element(-in-space) implicit timesteppers for the incompressible analogue to this system with structure-preserving (SP) properties in the ideal case, alongside parameter-robust preconditioners. We show that these timesteppers can derive from a finite-element-in-time (FET) (and finite-element-in-space) interpretation. The benefits of this reformulation are discussed, including the derivation of timesteppers that are higher order in time, and the quantifiable dissipative SP properties in the non-ideal, resistive case.
        
        We discuss possible options for extending this FET approach to timesteppers for the compressible case.

        The kinetic corrections satisfy linearized Boltzmann equations. Using a Lénard--Bernstein collision operator, these take Fokker--Planck-like forms \cite{Fokker_1914, Planck_1917} wherein pseudo-particles in the numerical model obey the neoclassical transport equations, with particle-independent Brownian drift terms. This offers a rigorous methodology for incorporating collisions into the particle transport model, without coupling the equations of motions for each particle.
        
        Works by Chen, Chacón et al. \cite{Chen_Chacón_Barnes_2011, Chacón_Chen_Barnes_2013, Chen_Chacón_2014, Chen_Chacón_2015} have developed structure-preserving particle pushers for neoclassical transport in the Vlasov equations, derived from Crank--Nicolson integrators. We show these too can can derive from a FET interpretation, similarly offering potential extensions to higher-order-in-time particle pushers. The FET formulation is used also to consider how the stochastic drift terms can be incorporated into the pushers. Stochastic gyrokinetic expansions are also discussed.

        Different options for the numerical implementation of these schemes are considered.

        Due to the efficacy of FET in the development of SP timesteppers for both the fluid and kinetic component, we hope this approach will prove effective in the future for developing SP timesteppers for the full hybrid model. We hope this will give us the opportunity to incorporate previously inaccessible kinetic effects into the highly effective, modern, finite-element MHD models.
    \end{abstract}
    
    
    \newpage
    \tableofcontents
    
    
    \newpage
    \pagenumbering{arabic}
    %\linenumbers\renewcommand\thelinenumber{\color{black!50}\arabic{linenumber}}
            \input{0 - introduction/main.tex}
        \part{Research}
            \input{1 - low-noise PiC models/main.tex}
            \input{2 - kinetic component/main.tex}
            \input{3 - fluid component/main.tex}
            \input{4 - numerical implementation/main.tex}
        \part{Project Overview}
            \input{5 - research plan/main.tex}
            \input{6 - summary/main.tex}
    
    
    %\section{}
    \newpage
    \pagenumbering{gobble}
        \printbibliography


    \newpage
    \pagenumbering{roman}
    \appendix
        \part{Appendices}
            \input{8 - Hilbert complexes/main.tex}
            \input{9 - weak conservation proofs/main.tex}
\end{document}

            \documentclass[12pt, a4paper]{report}

\input{template/main.tex}

\title{\BA{Title in Progress...}}
\author{Boris Andrews}
\affil{Mathematical Institute, University of Oxford}
\date{\today}


\begin{document}
    \pagenumbering{gobble}
    \maketitle
    
    
    \begin{abstract}
        Magnetic confinement reactors---in particular tokamaks---offer one of the most promising options for achieving practical nuclear fusion, with the potential to provide virtually limitless, clean energy. The theoretical and numerical modeling of tokamak plasmas is simultaneously an essential component of effective reactor design, and a great research barrier. Tokamak operational conditions exhibit comparatively low Knudsen numbers. Kinetic effects, including kinetic waves and instabilities, Landau damping, bump-on-tail instabilities and more, are therefore highly influential in tokamak plasma dynamics. Purely fluid models are inherently incapable of capturing these effects, whereas the high dimensionality in purely kinetic models render them practically intractable for most relevant purposes.

        We consider a $\delta\!f$ decomposition model, with a macroscopic fluid background and microscopic kinetic correction, both fully coupled to each other. A similar manner of discretization is proposed to that used in the recent \texttt{STRUPHY} code \cite{Holderied_Possanner_Wang_2021, Holderied_2022, Li_et_al_2023} with a finite-element model for the background and a pseudo-particle/PiC model for the correction.

        The fluid background satisfies the full, non-linear, resistive, compressible, Hall MHD equations. \cite{Laakmann_Hu_Farrell_2022} introduces finite-element(-in-space) implicit timesteppers for the incompressible analogue to this system with structure-preserving (SP) properties in the ideal case, alongside parameter-robust preconditioners. We show that these timesteppers can derive from a finite-element-in-time (FET) (and finite-element-in-space) interpretation. The benefits of this reformulation are discussed, including the derivation of timesteppers that are higher order in time, and the quantifiable dissipative SP properties in the non-ideal, resistive case.
        
        We discuss possible options for extending this FET approach to timesteppers for the compressible case.

        The kinetic corrections satisfy linearized Boltzmann equations. Using a Lénard--Bernstein collision operator, these take Fokker--Planck-like forms \cite{Fokker_1914, Planck_1917} wherein pseudo-particles in the numerical model obey the neoclassical transport equations, with particle-independent Brownian drift terms. This offers a rigorous methodology for incorporating collisions into the particle transport model, without coupling the equations of motions for each particle.
        
        Works by Chen, Chacón et al. \cite{Chen_Chacón_Barnes_2011, Chacón_Chen_Barnes_2013, Chen_Chacón_2014, Chen_Chacón_2015} have developed structure-preserving particle pushers for neoclassical transport in the Vlasov equations, derived from Crank--Nicolson integrators. We show these too can can derive from a FET interpretation, similarly offering potential extensions to higher-order-in-time particle pushers. The FET formulation is used also to consider how the stochastic drift terms can be incorporated into the pushers. Stochastic gyrokinetic expansions are also discussed.

        Different options for the numerical implementation of these schemes are considered.

        Due to the efficacy of FET in the development of SP timesteppers for both the fluid and kinetic component, we hope this approach will prove effective in the future for developing SP timesteppers for the full hybrid model. We hope this will give us the opportunity to incorporate previously inaccessible kinetic effects into the highly effective, modern, finite-element MHD models.
    \end{abstract}
    
    
    \newpage
    \tableofcontents
    
    
    \newpage
    \pagenumbering{arabic}
    %\linenumbers\renewcommand\thelinenumber{\color{black!50}\arabic{linenumber}}
            \input{0 - introduction/main.tex}
        \part{Research}
            \input{1 - low-noise PiC models/main.tex}
            \input{2 - kinetic component/main.tex}
            \input{3 - fluid component/main.tex}
            \input{4 - numerical implementation/main.tex}
        \part{Project Overview}
            \input{5 - research plan/main.tex}
            \input{6 - summary/main.tex}
    
    
    %\section{}
    \newpage
    \pagenumbering{gobble}
        \printbibliography


    \newpage
    \pagenumbering{roman}
    \appendix
        \part{Appendices}
            \input{8 - Hilbert complexes/main.tex}
            \input{9 - weak conservation proofs/main.tex}
\end{document}

        \part{Project Overview}
            \documentclass[12pt, a4paper]{report}

\input{template/main.tex}

\title{\BA{Title in Progress...}}
\author{Boris Andrews}
\affil{Mathematical Institute, University of Oxford}
\date{\today}


\begin{document}
    \pagenumbering{gobble}
    \maketitle
    
    
    \begin{abstract}
        Magnetic confinement reactors---in particular tokamaks---offer one of the most promising options for achieving practical nuclear fusion, with the potential to provide virtually limitless, clean energy. The theoretical and numerical modeling of tokamak plasmas is simultaneously an essential component of effective reactor design, and a great research barrier. Tokamak operational conditions exhibit comparatively low Knudsen numbers. Kinetic effects, including kinetic waves and instabilities, Landau damping, bump-on-tail instabilities and more, are therefore highly influential in tokamak plasma dynamics. Purely fluid models are inherently incapable of capturing these effects, whereas the high dimensionality in purely kinetic models render them practically intractable for most relevant purposes.

        We consider a $\delta\!f$ decomposition model, with a macroscopic fluid background and microscopic kinetic correction, both fully coupled to each other. A similar manner of discretization is proposed to that used in the recent \texttt{STRUPHY} code \cite{Holderied_Possanner_Wang_2021, Holderied_2022, Li_et_al_2023} with a finite-element model for the background and a pseudo-particle/PiC model for the correction.

        The fluid background satisfies the full, non-linear, resistive, compressible, Hall MHD equations. \cite{Laakmann_Hu_Farrell_2022} introduces finite-element(-in-space) implicit timesteppers for the incompressible analogue to this system with structure-preserving (SP) properties in the ideal case, alongside parameter-robust preconditioners. We show that these timesteppers can derive from a finite-element-in-time (FET) (and finite-element-in-space) interpretation. The benefits of this reformulation are discussed, including the derivation of timesteppers that are higher order in time, and the quantifiable dissipative SP properties in the non-ideal, resistive case.
        
        We discuss possible options for extending this FET approach to timesteppers for the compressible case.

        The kinetic corrections satisfy linearized Boltzmann equations. Using a Lénard--Bernstein collision operator, these take Fokker--Planck-like forms \cite{Fokker_1914, Planck_1917} wherein pseudo-particles in the numerical model obey the neoclassical transport equations, with particle-independent Brownian drift terms. This offers a rigorous methodology for incorporating collisions into the particle transport model, without coupling the equations of motions for each particle.
        
        Works by Chen, Chacón et al. \cite{Chen_Chacón_Barnes_2011, Chacón_Chen_Barnes_2013, Chen_Chacón_2014, Chen_Chacón_2015} have developed structure-preserving particle pushers for neoclassical transport in the Vlasov equations, derived from Crank--Nicolson integrators. We show these too can can derive from a FET interpretation, similarly offering potential extensions to higher-order-in-time particle pushers. The FET formulation is used also to consider how the stochastic drift terms can be incorporated into the pushers. Stochastic gyrokinetic expansions are also discussed.

        Different options for the numerical implementation of these schemes are considered.

        Due to the efficacy of FET in the development of SP timesteppers for both the fluid and kinetic component, we hope this approach will prove effective in the future for developing SP timesteppers for the full hybrid model. We hope this will give us the opportunity to incorporate previously inaccessible kinetic effects into the highly effective, modern, finite-element MHD models.
    \end{abstract}
    
    
    \newpage
    \tableofcontents
    
    
    \newpage
    \pagenumbering{arabic}
    %\linenumbers\renewcommand\thelinenumber{\color{black!50}\arabic{linenumber}}
            \input{0 - introduction/main.tex}
        \part{Research}
            \input{1 - low-noise PiC models/main.tex}
            \input{2 - kinetic component/main.tex}
            \input{3 - fluid component/main.tex}
            \input{4 - numerical implementation/main.tex}
        \part{Project Overview}
            \input{5 - research plan/main.tex}
            \input{6 - summary/main.tex}
    
    
    %\section{}
    \newpage
    \pagenumbering{gobble}
        \printbibliography


    \newpage
    \pagenumbering{roman}
    \appendix
        \part{Appendices}
            \input{8 - Hilbert complexes/main.tex}
            \input{9 - weak conservation proofs/main.tex}
\end{document}

            \documentclass[12pt, a4paper]{report}

\input{template/main.tex}

\title{\BA{Title in Progress...}}
\author{Boris Andrews}
\affil{Mathematical Institute, University of Oxford}
\date{\today}


\begin{document}
    \pagenumbering{gobble}
    \maketitle
    
    
    \begin{abstract}
        Magnetic confinement reactors---in particular tokamaks---offer one of the most promising options for achieving practical nuclear fusion, with the potential to provide virtually limitless, clean energy. The theoretical and numerical modeling of tokamak plasmas is simultaneously an essential component of effective reactor design, and a great research barrier. Tokamak operational conditions exhibit comparatively low Knudsen numbers. Kinetic effects, including kinetic waves and instabilities, Landau damping, bump-on-tail instabilities and more, are therefore highly influential in tokamak plasma dynamics. Purely fluid models are inherently incapable of capturing these effects, whereas the high dimensionality in purely kinetic models render them practically intractable for most relevant purposes.

        We consider a $\delta\!f$ decomposition model, with a macroscopic fluid background and microscopic kinetic correction, both fully coupled to each other. A similar manner of discretization is proposed to that used in the recent \texttt{STRUPHY} code \cite{Holderied_Possanner_Wang_2021, Holderied_2022, Li_et_al_2023} with a finite-element model for the background and a pseudo-particle/PiC model for the correction.

        The fluid background satisfies the full, non-linear, resistive, compressible, Hall MHD equations. \cite{Laakmann_Hu_Farrell_2022} introduces finite-element(-in-space) implicit timesteppers for the incompressible analogue to this system with structure-preserving (SP) properties in the ideal case, alongside parameter-robust preconditioners. We show that these timesteppers can derive from a finite-element-in-time (FET) (and finite-element-in-space) interpretation. The benefits of this reformulation are discussed, including the derivation of timesteppers that are higher order in time, and the quantifiable dissipative SP properties in the non-ideal, resistive case.
        
        We discuss possible options for extending this FET approach to timesteppers for the compressible case.

        The kinetic corrections satisfy linearized Boltzmann equations. Using a Lénard--Bernstein collision operator, these take Fokker--Planck-like forms \cite{Fokker_1914, Planck_1917} wherein pseudo-particles in the numerical model obey the neoclassical transport equations, with particle-independent Brownian drift terms. This offers a rigorous methodology for incorporating collisions into the particle transport model, without coupling the equations of motions for each particle.
        
        Works by Chen, Chacón et al. \cite{Chen_Chacón_Barnes_2011, Chacón_Chen_Barnes_2013, Chen_Chacón_2014, Chen_Chacón_2015} have developed structure-preserving particle pushers for neoclassical transport in the Vlasov equations, derived from Crank--Nicolson integrators. We show these too can can derive from a FET interpretation, similarly offering potential extensions to higher-order-in-time particle pushers. The FET formulation is used also to consider how the stochastic drift terms can be incorporated into the pushers. Stochastic gyrokinetic expansions are also discussed.

        Different options for the numerical implementation of these schemes are considered.

        Due to the efficacy of FET in the development of SP timesteppers for both the fluid and kinetic component, we hope this approach will prove effective in the future for developing SP timesteppers for the full hybrid model. We hope this will give us the opportunity to incorporate previously inaccessible kinetic effects into the highly effective, modern, finite-element MHD models.
    \end{abstract}
    
    
    \newpage
    \tableofcontents
    
    
    \newpage
    \pagenumbering{arabic}
    %\linenumbers\renewcommand\thelinenumber{\color{black!50}\arabic{linenumber}}
            \input{0 - introduction/main.tex}
        \part{Research}
            \input{1 - low-noise PiC models/main.tex}
            \input{2 - kinetic component/main.tex}
            \input{3 - fluid component/main.tex}
            \input{4 - numerical implementation/main.tex}
        \part{Project Overview}
            \input{5 - research plan/main.tex}
            \input{6 - summary/main.tex}
    
    
    %\section{}
    \newpage
    \pagenumbering{gobble}
        \printbibliography


    \newpage
    \pagenumbering{roman}
    \appendix
        \part{Appendices}
            \input{8 - Hilbert complexes/main.tex}
            \input{9 - weak conservation proofs/main.tex}
\end{document}

    
    
    %\section{}
    \newpage
    \pagenumbering{gobble}
        \printbibliography


    \newpage
    \pagenumbering{roman}
    \appendix
        \part{Appendices}
            \documentclass[12pt, a4paper]{report}

\input{template/main.tex}

\title{\BA{Title in Progress...}}
\author{Boris Andrews}
\affil{Mathematical Institute, University of Oxford}
\date{\today}


\begin{document}
    \pagenumbering{gobble}
    \maketitle
    
    
    \begin{abstract}
        Magnetic confinement reactors---in particular tokamaks---offer one of the most promising options for achieving practical nuclear fusion, with the potential to provide virtually limitless, clean energy. The theoretical and numerical modeling of tokamak plasmas is simultaneously an essential component of effective reactor design, and a great research barrier. Tokamak operational conditions exhibit comparatively low Knudsen numbers. Kinetic effects, including kinetic waves and instabilities, Landau damping, bump-on-tail instabilities and more, are therefore highly influential in tokamak plasma dynamics. Purely fluid models are inherently incapable of capturing these effects, whereas the high dimensionality in purely kinetic models render them practically intractable for most relevant purposes.

        We consider a $\delta\!f$ decomposition model, with a macroscopic fluid background and microscopic kinetic correction, both fully coupled to each other. A similar manner of discretization is proposed to that used in the recent \texttt{STRUPHY} code \cite{Holderied_Possanner_Wang_2021, Holderied_2022, Li_et_al_2023} with a finite-element model for the background and a pseudo-particle/PiC model for the correction.

        The fluid background satisfies the full, non-linear, resistive, compressible, Hall MHD equations. \cite{Laakmann_Hu_Farrell_2022} introduces finite-element(-in-space) implicit timesteppers for the incompressible analogue to this system with structure-preserving (SP) properties in the ideal case, alongside parameter-robust preconditioners. We show that these timesteppers can derive from a finite-element-in-time (FET) (and finite-element-in-space) interpretation. The benefits of this reformulation are discussed, including the derivation of timesteppers that are higher order in time, and the quantifiable dissipative SP properties in the non-ideal, resistive case.
        
        We discuss possible options for extending this FET approach to timesteppers for the compressible case.

        The kinetic corrections satisfy linearized Boltzmann equations. Using a Lénard--Bernstein collision operator, these take Fokker--Planck-like forms \cite{Fokker_1914, Planck_1917} wherein pseudo-particles in the numerical model obey the neoclassical transport equations, with particle-independent Brownian drift terms. This offers a rigorous methodology for incorporating collisions into the particle transport model, without coupling the equations of motions for each particle.
        
        Works by Chen, Chacón et al. \cite{Chen_Chacón_Barnes_2011, Chacón_Chen_Barnes_2013, Chen_Chacón_2014, Chen_Chacón_2015} have developed structure-preserving particle pushers for neoclassical transport in the Vlasov equations, derived from Crank--Nicolson integrators. We show these too can can derive from a FET interpretation, similarly offering potential extensions to higher-order-in-time particle pushers. The FET formulation is used also to consider how the stochastic drift terms can be incorporated into the pushers. Stochastic gyrokinetic expansions are also discussed.

        Different options for the numerical implementation of these schemes are considered.

        Due to the efficacy of FET in the development of SP timesteppers for both the fluid and kinetic component, we hope this approach will prove effective in the future for developing SP timesteppers for the full hybrid model. We hope this will give us the opportunity to incorporate previously inaccessible kinetic effects into the highly effective, modern, finite-element MHD models.
    \end{abstract}
    
    
    \newpage
    \tableofcontents
    
    
    \newpage
    \pagenumbering{arabic}
    %\linenumbers\renewcommand\thelinenumber{\color{black!50}\arabic{linenumber}}
            \input{0 - introduction/main.tex}
        \part{Research}
            \input{1 - low-noise PiC models/main.tex}
            \input{2 - kinetic component/main.tex}
            \input{3 - fluid component/main.tex}
            \input{4 - numerical implementation/main.tex}
        \part{Project Overview}
            \input{5 - research plan/main.tex}
            \input{6 - summary/main.tex}
    
    
    %\section{}
    \newpage
    \pagenumbering{gobble}
        \printbibliography


    \newpage
    \pagenumbering{roman}
    \appendix
        \part{Appendices}
            \input{8 - Hilbert complexes/main.tex}
            \input{9 - weak conservation proofs/main.tex}
\end{document}

            \documentclass[12pt, a4paper]{report}

\input{template/main.tex}

\title{\BA{Title in Progress...}}
\author{Boris Andrews}
\affil{Mathematical Institute, University of Oxford}
\date{\today}


\begin{document}
    \pagenumbering{gobble}
    \maketitle
    
    
    \begin{abstract}
        Magnetic confinement reactors---in particular tokamaks---offer one of the most promising options for achieving practical nuclear fusion, with the potential to provide virtually limitless, clean energy. The theoretical and numerical modeling of tokamak plasmas is simultaneously an essential component of effective reactor design, and a great research barrier. Tokamak operational conditions exhibit comparatively low Knudsen numbers. Kinetic effects, including kinetic waves and instabilities, Landau damping, bump-on-tail instabilities and more, are therefore highly influential in tokamak plasma dynamics. Purely fluid models are inherently incapable of capturing these effects, whereas the high dimensionality in purely kinetic models render them practically intractable for most relevant purposes.

        We consider a $\delta\!f$ decomposition model, with a macroscopic fluid background and microscopic kinetic correction, both fully coupled to each other. A similar manner of discretization is proposed to that used in the recent \texttt{STRUPHY} code \cite{Holderied_Possanner_Wang_2021, Holderied_2022, Li_et_al_2023} with a finite-element model for the background and a pseudo-particle/PiC model for the correction.

        The fluid background satisfies the full, non-linear, resistive, compressible, Hall MHD equations. \cite{Laakmann_Hu_Farrell_2022} introduces finite-element(-in-space) implicit timesteppers for the incompressible analogue to this system with structure-preserving (SP) properties in the ideal case, alongside parameter-robust preconditioners. We show that these timesteppers can derive from a finite-element-in-time (FET) (and finite-element-in-space) interpretation. The benefits of this reformulation are discussed, including the derivation of timesteppers that are higher order in time, and the quantifiable dissipative SP properties in the non-ideal, resistive case.
        
        We discuss possible options for extending this FET approach to timesteppers for the compressible case.

        The kinetic corrections satisfy linearized Boltzmann equations. Using a Lénard--Bernstein collision operator, these take Fokker--Planck-like forms \cite{Fokker_1914, Planck_1917} wherein pseudo-particles in the numerical model obey the neoclassical transport equations, with particle-independent Brownian drift terms. This offers a rigorous methodology for incorporating collisions into the particle transport model, without coupling the equations of motions for each particle.
        
        Works by Chen, Chacón et al. \cite{Chen_Chacón_Barnes_2011, Chacón_Chen_Barnes_2013, Chen_Chacón_2014, Chen_Chacón_2015} have developed structure-preserving particle pushers for neoclassical transport in the Vlasov equations, derived from Crank--Nicolson integrators. We show these too can can derive from a FET interpretation, similarly offering potential extensions to higher-order-in-time particle pushers. The FET formulation is used also to consider how the stochastic drift terms can be incorporated into the pushers. Stochastic gyrokinetic expansions are also discussed.

        Different options for the numerical implementation of these schemes are considered.

        Due to the efficacy of FET in the development of SP timesteppers for both the fluid and kinetic component, we hope this approach will prove effective in the future for developing SP timesteppers for the full hybrid model. We hope this will give us the opportunity to incorporate previously inaccessible kinetic effects into the highly effective, modern, finite-element MHD models.
    \end{abstract}
    
    
    \newpage
    \tableofcontents
    
    
    \newpage
    \pagenumbering{arabic}
    %\linenumbers\renewcommand\thelinenumber{\color{black!50}\arabic{linenumber}}
            \input{0 - introduction/main.tex}
        \part{Research}
            \input{1 - low-noise PiC models/main.tex}
            \input{2 - kinetic component/main.tex}
            \input{3 - fluid component/main.tex}
            \input{4 - numerical implementation/main.tex}
        \part{Project Overview}
            \input{5 - research plan/main.tex}
            \input{6 - summary/main.tex}
    
    
    %\section{}
    \newpage
    \pagenumbering{gobble}
        \printbibliography


    \newpage
    \pagenumbering{roman}
    \appendix
        \part{Appendices}
            \input{8 - Hilbert complexes/main.tex}
            \input{9 - weak conservation proofs/main.tex}
\end{document}

\end{document}

\end{document}


    \line

    In addition to the individual motivations for a FEM model for the fluid component (\ref{eqn:PiC-coupled mass conservation}--\ref{eqn:PiC-coupled Maxwell's equations steady-state}) and a pseudo-particle method for the kinetic component (\ref{eqn:linearized Boltzmann equation}) the two techniques have notable advantages when used together for the full $\delta\!f$ decomposition:
    \begin{itemize}
        \item  {\bf Weak formulations and pseudo-particle distribution functions:} When using a pseudo-particle method for the $\delta\!f_{\pm}$, suppose the pseudo-particles are indexed via $*_{\pm}^{(i)}$, with position, $\bfX_{\pm}^{(i)}(t)$, velocity, $\bfV_{\pm}^{(i)}(t)$, evolving according to a coupled system of SDEs. In a classical physical sense, once can interpret this as assuming the weak approximations
        \begin{equation}\label{eqn:delta function approximation}
            f_{\pm}(\bfx, \bfv; t)  \approx  \sum_{i}w_{\pm}^{(i)}\delta^{3}\left[\bfx - \bfX_{\pm}^{(i)}(t)\right]\delta^{3}\left[\bfv - \bfV_{\pm}^{(i)}(t)\right],
        \end{equation}
        where $\delta^{3}$ again denotes the 3D $\delta$ function, and $w_{\pm}^{(i)}$ denotes the (potentially either positive or negative) weighting in $\delta\!f_{\pm}$ of the particle indexed by $*_{\pm}^{(i)}$. As such, when incorporating this pseudo-particle approximation to the $\delta\!f_{\pm}$ corrections in the fluid model, those equations featuring $\delta\!f_{\pm}$ consequently feature $\delta$ functions in space. By way of example, the charge conservation equation (\ref{eqn:PiC-coupled charge conservation}) can be written exactly as
        \begin{equation}
            \rmM^{2}\partial_{t}\rho_{\rmC} + \nabla_{\bfx}\cdot\bfj  =  - \frac{\beta}{2}\nabla_{\bfx}\cdot\left[\int_{\bfv}[\delta\!f_{+} - \delta\!f_{-}]\bfv\right]
        \end{equation}
        Making the $\delta$ function approximation (\ref{eqn:delta function approximation}),
        \begin{multline}\label{eqn:PiC-coupled charge conservation with pseudo-particles}
            \rmM^{2}\partial_{t}\rho_{\rmC} + \nabla_{\bfx}\cdot\bfj  \\
            \approx  - \frac{\beta}{2}\nabla_{\bfx}\cdot\left[\sum_{i}w_{+}^{(i)}\delta^{3}\left[\bfx - \bfX_{+}^{(i)}\right]\bfV_{+}^{(i)} - \sum_{i}w_{-}^{(i)}\delta^{3}\left[\bfx - \bfX_{-}^{(i)}\right]\bfV_{-}^{(i)}\right]
        \end{multline}        
        This is apparently problematic, as these $\delta$ functions exist only in a weak sense, as distributions in the dual space $\left(C^{\infty}_{\rmc}\right)^{*}$.
        
        FEM models however derive from weak formulations of the corresponding PDEs, with the result that the weak nature of the definition of the $\delta$ functions is not so much a problem. ``Testing'' (in $L^{2}$) against some test function $\phi(\bfx)$ in some chosen test space defined over the spatial domain, the example (\ref{eqn:PiC-coupled charge conservation with pseudo-particles}) can take the weak form,
        \begin{multline}
            \int_{\bfx}\phi\partial_{t}\rho_{\rmC} + \int_{\bfx}\phi\nabla_{\bfx}\cdot\bfj  \\
            \approx  \frac{\beta}{2}\left(\sum_{i}w_{+}^{(i)}\nabla_{\bfx}\phi\left(\bfX_{+}^{(i)}\right)\cdot\bfV_{+}^{(i)} - \sum_{i}w_{-}^{(i)}\nabla_{\bfx}\phi\left(\bfX_{-}^{(i)}\right)\cdot\bfV_{-}^{(i)}\right).
        \end{multline}
        The other fluid equations in (\ref{eqn:PiC-coupled mass conservation}--\ref{eqn:PiC-coupled Maxwell's equations steady-state}) featuring $\delta\!f_{\pm}$ follow in a similar suit.

        This can be taken further when using finite elements in time (FET)---discussed further in Subsection \ref{cha:FET}---when testing (in $L^{2}$) against test functions $\varphi(\bfx; t)$ in test spaces defined over the whole space-\emph{time} domain, as
        \begin{multline}
            \int_{\bfx; t}\phi\partial_{t}\rho_{\rmC} + \int_{\bfx; t}\phi\nabla_{\bfx}\cdot\bfj \\
            \approx  \frac{\beta}{2}\int_{t}\left[\sum_{i}w_{+}^{(i)}\nabla_{\bfx}\phi\left(\bfX_{+}^{(i)}(t)\right)\cdot\bfV_{+}^{(i)}(t) - \sum_{i}w_{-}^{(i)}\nabla_{\bfx}\phi\left(\bfX_{-}^{(i)}(t)\right)\cdot\bfV_{-}^{(i)}(t)\right].
        \end{multline}
        Since $\bfX_{\pm}^{(i)}(t)$ and $\bfV_{\pm}^{(i)}(t)$ are complex stochastic processes, further care must be taken regarding the continuity of the test spaces to ensure these RHS particle terms are well-defined. \BA{(Said here that care must be taken, but haven't really taken that care yet.)}

        \item  {\bf FET for quantifiable time derivatives:} It was noted in Section \ref{cha:delta f decompositions} that the RHS of the kinetic equations in the $\delta\!f_{\pm}$ corrections (\ref{eqn:linearized Boltzmann equation}) features terms corresponding to time derivatives of the fluid parameters, $\partial_{t}\rho_{\rmM}$, $\partial_{t}\rho_{\rmC}$, $\partial_{t}\bfp$, $\partial_{t}p$, that we would like not to have to substitute for their given values from the fluid equations (\ref{eqn:PiC-coupled mass conservation}--\ref{eqn:PiC-coupled pressure conservation}).
        
        Using a FET approach for the fluid equations means these time derivatives are exactly quantified at all positions in the space-time domain, avoiding this problem. 
    \end{itemize}

    We refer to this FEM--PiC discretization of the fluid (\ref{eqn:PiC-coupled mass conservation}--\ref{eqn:PiC-coupled Maxwell's equations steady-state}) and kinetic component (\ref{eqn:linearized Boltzmann equation}) of the $\delta\!f$ decomposed Boltzmann equations as a coupled, hybrid FEM--PiC model.

    The recent \texttt{STRUPHY} code \cite{Holderied_Possanner_Wang_2021, Holderied_2022, Li_et_al_2023} implements a similar coupled, hybrid FEM--PiC model in a bespoke \texttt{Python} implementation, with:
    \begin{itemize}
      \item  A reduced linear, ideal MHD background model, discretized using similar SP FEM timesteppers as those developed by Hu et al. \cite{Hu_Xu_2015, Hu_Ma_Yu_2017, Hu_Lee_Xu_2021, Green_et_al_2022, LFM22, Laakmann_Hu_Farrell_2022} as above, on which we would like our timesteppers to be based.
      \item  A collision-free Vlasov model for the correction.
    \end{itemize}
    The efficacy of \texttt{STRUPHY} provides motivation for the efficacy of the discretization presented here.
    