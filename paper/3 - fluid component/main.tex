\chapter{Fluid Component}
    \BA{Introduction.}

    \BA{Have barely touched on preconditioning. I think I'll end up just leaving this as a remark/area for further work really. There are \emph{ideas} there, but I've really not had the chance to investigate them fully yet.}

    \BA{I have some inconsistency in notation here with the rest of the paper in how I donate integrals. I should make this consistent, by denoting $\int_{\bfx}  \mapsto  \int_{\bfOmega}$ and, say, $\int_{\bfv}  \mapsto  \int_{\bfXi}$.}

    \BA{This is a little rough. I will give a full derivation of this model, but that'll be in the previous section. For now, I just want to have the system written down.}

    \BA{Just realised, I've got some $\nabla$'s earlier that should be $\nabla_{\bfx}$'s.}
    
    We consider now the fluid component of the $\delta\!f$ model (\ref{eqn:PiC-coupled mass conservation}--\ref{eqn:PiC-coupled Maxwell's equations steady-state}) neglecting terms deriving from the $\delta\!f_{\pm}$ corrections. Since we are considering a system over space, $\bfx$ (and time, $t$) only, $\nabla_{\bfx}$ shall be denoted as simply $\nabla$. Terms of negligible order as specificed in (\ref{eqn:PiC-coupled mass conservation}--\ref{eqn:PiC-coupled Maxwell's equations steady-state}) shall also be ignored.
    
    To complete the system, make the following approximations for the spatially dissipative terms deriving from the collision operators:
    \begin{align}
        \int_{\bfv}\left(\frac{1}{{\rmRef}_{++}}\bfdelta\bfC_{++} + \frac{1}{{\rmRef}_{+-}}\bfdelta\bfC_{+-}\right)\!\left[f_{+}^{(0)}, f_{-}^{(0)}\right]                    &\approx  \frac{1}{\rmRef}\nabla\cdot[\rho_{\rmM}\bftau[\bfu]]  \\
        \int_{\bfv}\left(\frac{1}{{\rmRef}_{++}}\bfdelta\bfC_{++} + \frac{1}{{\rmRef}_{+-}}\bfdelta\bfC_{+-}\right)\!\left[f_{+}^{(0)}, f_{-}^{(0)}\right]\cdot(\bfv - \bfu)  &\approx  \frac{2}{\rmRef}\rho_{\rmM}\|\bftau[\bfu]\|^{2} + \frac{1}{\rmPe}\nabla\cdot[\rho_{\rmM}\nabla\theta]
    \end{align}
    where the Péclet number, $\rmPe$, is a dimensionless constant, and $\bftau[\bfu]$ denotes the (deviatoric) strain,
    \begin{equation}\label{eqn:strain equation}
        \bftau[\bfu]  :=  \left(\nabla\bfu + \nabla\bfu^{\rmT}\right) - \frac{2}{3}(\nabla\cdot\bfu)\bfI,
    \end{equation}
    such that $\tr[\bftau[\bfu]]  =  0$. The assumption that the stress tensor here is predominantly trace-free is referred to as the Stokes hypothesis. \cite{Stokes_1845} Its validity is discussed in \cite{Vincenti_Kruger_1975}.

    Under the assumption of non-relativistic velocity scales, $\rmM  \ll  1$, the material fluid equations (\ref{eqn:PiC-coupled mass conservation}--\ref{eqn:PiC-coupled pressure conservation}) then take the forms:
    \begin{align}
         \partial_{t}\rho_{\rmM}  &=  - \nabla\cdot\bfp  \\
                               0  &=  - \nabla\cdot\bfj  \\
                \partial_{t}\bfp  &=  - \nabla\cdot\left[\rho_{\rmM}\bfu^{\otimes 2}\right] - \nabla p + \frac{2}{\beta}\bfj\wedge\bfB + \frac{1}{\rmRef}\nabla\cdot[\rho_{\rmM}\bftau[\bfu]]  \\
        \frac{3}{2}\partial_{t}p  &=  - \frac{3}{2}\nabla\cdot[p\bfu] - p\nabla\cdot\bfu + \frac{2}{\beta}\bfj\cdot(\bfE + \bfu\wedge\bfB) + \frac{2}{\rmRef}\rho_{\rmM}\|\bftau[\bfu]\|^{2} + \frac{1}{\rmPe}\nabla\cdot[\rho_{\rmM}\nabla\theta]
    \end{align}
    Due to the non-relativistic assumption, $\rho_{\rmC}$ is no longer present in these equation, i.e. the plasma is functionally \emph{``quasineutral''}.

    These equations are coupled with Maxwell's equations (\ref{eqn:PiC-coupled Maxwell's equations transient}--\ref{eqn:PiC-coupled Maxwell's equations steady-state}) whereby under the same non-relativistic assumption, $\rmM  \ll  1$:
    \begin{align}
                 \bfzero  &=  \nabla\wedge\bfB - \bfj,  &
        \partial_{t}\bfB  &=  - \nabla\wedge\bfE,  \\
         \nabla\cdot\bfE  &=  \rho_{\rmC},  &
         \nabla\cdot\bfB  &=  0.
    \end{align}
    
    This system is completed by the thermal equilibrium identity (\ref{eqn:thermal equilibrium identity}), which can be interpreted here as an identity for $\bfj$. Again with the non-relativistic assumption, $\rmM  \ll  1$, we can assume the collision operators are such that this gives the identity:
    \begin{equation}
        \frac{1}{\rmRem}\rho_{\rmM}\bfj  =  \bfE + \bfu\wedge\bfB - \rmRH\frac{1}{\rho_{\rmM}}\bfj\wedge\bfB
    \end{equation}
    where the magnetic Reynolds number, $\rmRem$, and Hall number, $\rmRH$, are dimensionless constants also.

    This system can be reduced by solving for $\rho_{\rmC}$, $\bfj$, $\bfE$ as:
    \begin{align}
        \rho_{\rmC}  &=  \nabla\cdot\bfE  \\
               \bfj  &=  \nabla\wedge\bfB  \\
               \bfE  &=  \frac{1}{\rmRem}\rho_{\rmM}\bfj - \bfu\wedge\bfB + \rmRH\frac{1}{\rho_{\rmM}}\bfj\wedge\bfB
    \end{align}
    to give the final system in Figure \ref{fig:compressible strong form}.

    \begin{figure}
        \centering
        \line
        \begin{align}
            \partial_{t}\rho_{\rmM}  &=  - \nabla\cdot\bfp  \\
            \partial_{t}\bfp  &=  -  \nabla\cdot\left[\rho_{\rmM}\bfu^{\otimes 2}\right] - \nabla p + \frac{2}{\beta}(\nabla\wedge\bfB)\wedge\bfB + \frac{1}{\rmRef}\nabla\cdot[\rho_{\rmM}\bftau[\bfu]]  \\
            \frac{3}{2}\partial_{t}p  &=  - \frac{3}{2}\nabla\cdot[p\bfu] - p\nabla\cdot\bfu + \frac{2}{\beta\rmRef}\rho_{\rmM}\|\bftau[\bfu]\|^{2} + \frac{2}{\beta\rmRem}\rho_{\rmM}\|\nabla\wedge\bfB\|^{2} + \frac{1}{\rmPe}\nabla\cdot[\rho_{\rmM}\nabla\theta]  \label{eqn:energy conservation}  \\
            \partial_{t}\bfB  &=  - \nabla\wedge\left[\frac{1}{\rmRem}\rho_{\rmM}\nabla\wedge\bfB - \bfu\wedge\bfB + \rmRH\frac{1}{\rho_{\rmM}}(\nabla\wedge\bfB)\wedge\bfB\right]
        \end{align}
        \shortline
        \begin{equation}
            0  =  \nabla\cdot\bfB|_{t = 0}
        \end{equation}
        \line
        \caption{Compressible strong formulation}
        \label{fig:compressible strong form}
    \end{figure}

    In the ideal limit $\rmRef, \rmRem  =  \infty$, the viscous and Ohmic heating terms, $\frac{2}{\beta\rmRef}\rho_{\rmM}\|\bftau[\bfu]\|^{2}$ and $\frac{2}{\beta\rmRem}\rho_{\rmM}\|\nabla\wedge\bfB\|^{2}$ respectively, are negligible. Despite considering turbulent, high-Reynolds systems, they are left in the system, as they give exact energy conservation for general $\rmRef$, $\rmRem$, giving a more accurate and rich structure for the analysis and discretization.

    This system resembles the \emph{incompressible} Hall MHD system presented in \cite{LHF22}, incorporating compressibility, a necessary factor in some form for kinetic effects, through the inclusion of the energy equation (\ref{eqn:energy conservation}). (See Figure \ref{fig:incompressible strong form}) This work therefore offers a good starting point therefore for the analysis, discretization and preconditioning of the compressible system.

    \begin{figure}
        \centering
        \line
        \begin{align}
            0  &=  - \nabla\cdot\bfu  \\
            \partial_{t}\bfu  &=  -  \nabla\cdot\left[\bfu^{\otimes 2}\right] - \nabla p + \frac{2}{\beta}(\nabla\wedge\bfB)\wedge\bfB + \frac{1}{\rmRef}\Delta\bfu  \\
            \partial_{t}\bfB  &=  - \nabla\wedge\left[\frac{1}{\rmRem}\nabla\wedge\bfB - \bfu\wedge\bfB + \rmRH(\nabla\wedge\bfB)\wedge\bfB\right]
        \end{align}
        \shortline
        \begin{equation}
            0  =  \nabla\cdot\bfB|_{t = 0}
        \end{equation}
        \line
        \caption{Incompressible strong formulation}
        \label{fig:incompressible strong form}
    \end{figure}

    Since $\rho_{\rmC}$ is eliminated from the system, we shall henceforth write $\rho_{\rmM}$ as simply $\rho$.

    
    \section{Preserved Structures}
    \BA{Introduction.}
    
    Consider first those quantities that are conserved by the transient system, so as to seek discretisations which better represent the physical behaviour of the system by \emph{also} conserved these quantities. 
    
    \cite{LHF22} considers conservation of the following 3 quantities, which the authors define in the incompressible case as: \BA{(Oops I've never defined $\bfA$! That should probably be in the introduction...)}
    \begin{center}\begin{tabular}{ c c c }
        Properties  &  Symbol  &  Definition  \\
        \hline\hline
        Energy  &  $\rmE$  &  $\int_{\bfOmega}\left[\frac{1}{\rmEu\rho}\|\bfp\|^{2} + p + \frac{1}{\beta}\|\bfB\|^{2}\right]$  \\
        Magnetic helicity  &  $\rmH_{\rmM}$  &  $\int_{\bfOmega}\bfA\cdot\bfB$  \\
        Hybrid helicity  &  $\rmH_{\rmH}$  &  $\int_{\bfOmega}(a\bfA + \bfp)\cdot(b\bfB + \nabla\wedge\bfp)$
    \end{tabular}\end{center}
    where $a$, $b$ satisfy the relation $a + b  =  \frac{4}{\beta\rmRH}$. \BA{(What do these represent \emph{physically}? Diagrams!)} Taking the derivatives of these quantities over time (still in the incompressible system) gives
    \begin{align}
        \frac{d\rmE}{dt}  &=  \BA{\cdots}  \\
        \frac{d\rmH_{\rmM}}{dt}  &=  \int_{\bfGamma}(- \varphi\bfB + \bfA\wedge\bfE)\cdot\bfn - \frac{2}{\rmRem}\int_{\bfOmega}\bfB\cdot\bfj  \\
        \frac{d\rmH_{\rmH}}{dt}  &=  \BA{\cdots} \\
    \end{align}

    \BA{Proven that in the \emph{compressible} case, $\frac{d\rmE}{dt}$ evaluates as
    {\small \begin{equation}
        \frac{d\rmE}{dt}  =  \int_{\bfGamma}\left[- \frac{1}{2\rmEu\rho}\|\bfp\|^{2}\bfp - \frac{p}{2\rho}\bfp + \frac{1}{\rmEu\rmRe_{f}}\nabla\left[\frac{1}{\rho}\bfp\right]\cdot\frac{1}{\rho}\bfp - \frac{p}{2\rho}\bfp + \frac{1}{2\rmPe}\nabla\left[\frac{p}{\rho} + \frac{1}{\beta}\bfB\wedge\bfE\right]\right]\cdot\bfn
    \end{equation}}}
    
    \section{Preserved Structures}
    \BA{Introduction.}
    
    Consider first those quantities that are conserved by the transient system, so as to seek discretisations which better represent the physical behaviour of the system by \emph{also} conserved these quantities. 
    
    \cite{LHF22} considers conservation of the following 3 quantities, which the authors define in the incompressible case as: \BA{(Oops I've never defined $\bfA$! That should probably be in the introduction...)}
    \begin{center}\begin{tabular}{ c c c }
        Properties  &  Symbol  &  Definition  \\
        \hline\hline
        Energy  &  $\rmE$  &  $\int_{\bfOmega}\left[\frac{1}{\rmEu\rho}\|\bfp\|^{2} + p + \frac{1}{\beta}\|\bfB\|^{2}\right]$  \\
        Magnetic helicity  &  $\rmH_{\rmM}$  &  $\int_{\bfOmega}\bfA\cdot\bfB$  \\
        Hybrid helicity  &  $\rmH_{\rmH}$  &  $\int_{\bfOmega}(a\bfA + \bfp)\cdot(b\bfB + \nabla\wedge\bfp)$
    \end{tabular}\end{center}
    where $a$, $b$ satisfy the relation $a + b  =  \frac{4}{\beta\rmRH}$. \BA{(What do these represent \emph{physically}? Diagrams!)} Taking the derivatives of these quantities over time (still in the incompressible system) gives
    \begin{align}
        \frac{d\rmE}{dt}  &=  \BA{\cdots}  \\
        \frac{d\rmH_{\rmM}}{dt}  &=  \int_{\bfGamma}(- \varphi\bfB + \bfA\wedge\bfE)\cdot\bfn - \frac{2}{\rmRem}\int_{\bfOmega}\bfB\cdot\bfj  \\
        \frac{d\rmH_{\rmH}}{dt}  &=  \BA{\cdots} \\
    \end{align}

    \BA{Proven that in the \emph{compressible} case, $\frac{d\rmE}{dt}$ evaluates as
    {\small \begin{equation}
        \frac{d\rmE}{dt}  =  \int_{\bfGamma}\left[- \frac{1}{2\rmEu\rho}\|\bfp\|^{2}\bfp - \frac{p}{2\rho}\bfp + \frac{1}{\rmEu\rmRe_{f}}\nabla\left[\frac{1}{\rho}\bfp\right]\cdot\frac{1}{\rho}\bfp - \frac{p}{2\rho}\bfp + \frac{1}{2\rmPe}\nabla\left[\frac{p}{\rho} + \frac{1}{\beta}\bfB\wedge\bfE\right]\right]\cdot\bfn
    \end{equation}}}
    
    \section{Preserved Structures}
    \BA{Introduction.}
    
    Consider first those quantities that are conserved by the transient system, so as to seek discretisations which better represent the physical behaviour of the system by \emph{also} conserved these quantities. 
    
    \cite{LHF22} considers conservation of the following 3 quantities, which the authors define in the incompressible case as: \BA{(Oops I've never defined $\bfA$! That should probably be in the introduction...)}
    \begin{center}\begin{tabular}{ c c c }
        Properties  &  Symbol  &  Definition  \\
        \hline\hline
        Energy  &  $\rmE$  &  $\int_{\bfOmega}\left[\frac{1}{\rmEu\rho}\|\bfp\|^{2} + p + \frac{1}{\beta}\|\bfB\|^{2}\right]$  \\
        Magnetic helicity  &  $\rmH_{\rmM}$  &  $\int_{\bfOmega}\bfA\cdot\bfB$  \\
        Hybrid helicity  &  $\rmH_{\rmH}$  &  $\int_{\bfOmega}(a\bfA + \bfp)\cdot(b\bfB + \nabla\wedge\bfp)$
    \end{tabular}\end{center}
    where $a$, $b$ satisfy the relation $a + b  =  \frac{4}{\beta\rmRH}$. \BA{(What do these represent \emph{physically}? Diagrams!)} Taking the derivatives of these quantities over time (still in the incompressible system) gives
    \begin{align}
        \frac{d\rmE}{dt}  &=  \BA{\cdots}  \\
        \frac{d\rmH_{\rmM}}{dt}  &=  \int_{\bfGamma}(- \varphi\bfB + \bfA\wedge\bfE)\cdot\bfn - \frac{2}{\rmRem}\int_{\bfOmega}\bfB\cdot\bfj  \\
        \frac{d\rmH_{\rmH}}{dt}  &=  \BA{\cdots} \\
    \end{align}

    \BA{Proven that in the \emph{compressible} case, $\frac{d\rmE}{dt}$ evaluates as
    {\small \begin{equation}
        \frac{d\rmE}{dt}  =  \int_{\bfGamma}\left[- \frac{1}{2\rmEu\rho}\|\bfp\|^{2}\bfp - \frac{p}{2\rho}\bfp + \frac{1}{\rmEu\rmRe_{f}}\nabla\left[\frac{1}{\rho}\bfp\right]\cdot\frac{1}{\rho}\bfp - \frac{p}{2\rho}\bfp + \frac{1}{2\rmPe}\nabla\left[\frac{p}{\rho} + \frac{1}{\beta}\bfB\wedge\bfE\right]\right]\cdot\bfn
    \end{equation}}}
    
    \section{Preserved Structures}
    \BA{Introduction.}
    
    Consider first those quantities that are conserved by the transient system, so as to seek discretisations which better represent the physical behaviour of the system by \emph{also} conserved these quantities. 
    
    \cite{LHF22} considers conservation of the following 3 quantities, which the authors define in the incompressible case as: \BA{(Oops I've never defined $\bfA$! That should probably be in the introduction...)}
    \begin{center}\begin{tabular}{ c c c }
        Properties  &  Symbol  &  Definition  \\
        \hline\hline
        Energy  &  $\rmE$  &  $\int_{\bfOmega}\left[\frac{1}{\rmEu\rho}\|\bfp\|^{2} + p + \frac{1}{\beta}\|\bfB\|^{2}\right]$  \\
        Magnetic helicity  &  $\rmH_{\rmM}$  &  $\int_{\bfOmega}\bfA\cdot\bfB$  \\
        Hybrid helicity  &  $\rmH_{\rmH}$  &  $\int_{\bfOmega}(a\bfA + \bfp)\cdot(b\bfB + \nabla\wedge\bfp)$
    \end{tabular}\end{center}
    where $a$, $b$ satisfy the relation $a + b  =  \frac{4}{\beta\rmRH}$. \BA{(What do these represent \emph{physically}? Diagrams!)} Taking the derivatives of these quantities over time (still in the incompressible system) gives
    \begin{align}
        \frac{d\rmE}{dt}  &=  \BA{\cdots}  \\
        \frac{d\rmH_{\rmM}}{dt}  &=  \int_{\bfGamma}(- \varphi\bfB + \bfA\wedge\bfE)\cdot\bfn - \frac{2}{\rmRem}\int_{\bfOmega}\bfB\cdot\bfj  \\
        \frac{d\rmH_{\rmH}}{dt}  &=  \BA{\cdots} \\
    \end{align}

    \BA{Proven that in the \emph{compressible} case, $\frac{d\rmE}{dt}$ evaluates as
    {\small \begin{equation}
        \frac{d\rmE}{dt}  =  \int_{\bfGamma}\left[- \frac{1}{2\rmEu\rho}\|\bfp\|^{2}\bfp - \frac{p}{2\rho}\bfp + \frac{1}{\rmEu\rmRe_{f}}\nabla\left[\frac{1}{\rho}\bfp\right]\cdot\frac{1}{\rho}\bfp - \frac{p}{2\rho}\bfp + \frac{1}{2\rmPe}\nabla\left[\frac{p}{\rho} + \frac{1}{\beta}\bfB\wedge\bfE\right]\right]\cdot\bfn
    \end{equation}}}
    
    \section{Preserved Structures}
    \BA{Introduction.}
    
    Consider first those quantities that are conserved by the transient system, so as to seek discretisations which better represent the physical behaviour of the system by \emph{also} conserved these quantities. 
    
    \cite{LHF22} considers conservation of the following 3 quantities, which the authors define in the incompressible case as: \BA{(Oops I've never defined $\bfA$! That should probably be in the introduction...)}
    \begin{center}\begin{tabular}{ c c c }
        Properties  &  Symbol  &  Definition  \\
        \hline\hline
        Energy  &  $\rmE$  &  $\int_{\bfOmega}\left[\frac{1}{\rmEu\rho}\|\bfp\|^{2} + p + \frac{1}{\beta}\|\bfB\|^{2}\right]$  \\
        Magnetic helicity  &  $\rmH_{\rmM}$  &  $\int_{\bfOmega}\bfA\cdot\bfB$  \\
        Hybrid helicity  &  $\rmH_{\rmH}$  &  $\int_{\bfOmega}(a\bfA + \bfp)\cdot(b\bfB + \nabla\wedge\bfp)$
    \end{tabular}\end{center}
    where $a$, $b$ satisfy the relation $a + b  =  \frac{4}{\beta\rmRH}$. \BA{(What do these represent \emph{physically}? Diagrams!)} Taking the derivatives of these quantities over time (still in the incompressible system) gives
    \begin{align}
        \frac{d\rmE}{dt}  &=  \BA{\cdots}  \\
        \frac{d\rmH_{\rmM}}{dt}  &=  \int_{\bfGamma}(- \varphi\bfB + \bfA\wedge\bfE)\cdot\bfn - \frac{2}{\rmRem}\int_{\bfOmega}\bfB\cdot\bfj  \\
        \frac{d\rmH_{\rmH}}{dt}  &=  \BA{\cdots} \\
    \end{align}

    \BA{Proven that in the \emph{compressible} case, $\frac{d\rmE}{dt}$ evaluates as
    {\small \begin{equation}
        \frac{d\rmE}{dt}  =  \int_{\bfGamma}\left[- \frac{1}{2\rmEu\rho}\|\bfp\|^{2}\bfp - \frac{p}{2\rho}\bfp + \frac{1}{\rmEu\rmRe_{f}}\nabla\left[\frac{1}{\rho}\bfp\right]\cdot\frac{1}{\rho}\bfp - \frac{p}{2\rho}\bfp + \frac{1}{2\rmPe}\nabla\left[\frac{p}{\rho} + \frac{1}{\beta}\bfB\wedge\bfE\right]\right]\cdot\bfn
    \end{equation}}}
    
    \section{Preserved Structures}
    \BA{Introduction.}
    
    Consider first those quantities that are conserved by the transient system, so as to seek discretisations which better represent the physical behaviour of the system by \emph{also} conserved these quantities. 
    
    \cite{LHF22} considers conservation of the following 3 quantities, which the authors define in the incompressible case as: \BA{(Oops I've never defined $\bfA$! That should probably be in the introduction...)}
    \begin{center}\begin{tabular}{ c c c }
        Properties  &  Symbol  &  Definition  \\
        \hline\hline
        Energy  &  $\rmE$  &  $\int_{\bfOmega}\left[\frac{1}{\rmEu\rho}\|\bfp\|^{2} + p + \frac{1}{\beta}\|\bfB\|^{2}\right]$  \\
        Magnetic helicity  &  $\rmH_{\rmM}$  &  $\int_{\bfOmega}\bfA\cdot\bfB$  \\
        Hybrid helicity  &  $\rmH_{\rmH}$  &  $\int_{\bfOmega}(a\bfA + \bfp)\cdot(b\bfB + \nabla\wedge\bfp)$
    \end{tabular}\end{center}
    where $a$, $b$ satisfy the relation $a + b  =  \frac{4}{\beta\rmRH}$. \BA{(What do these represent \emph{physically}? Diagrams!)} Taking the derivatives of these quantities over time (still in the incompressible system) gives
    \begin{align}
        \frac{d\rmE}{dt}  &=  \BA{\cdots}  \\
        \frac{d\rmH_{\rmM}}{dt}  &=  \int_{\bfGamma}(- \varphi\bfB + \bfA\wedge\bfE)\cdot\bfn - \frac{2}{\rmRem}\int_{\bfOmega}\bfB\cdot\bfj  \\
        \frac{d\rmH_{\rmH}}{dt}  &=  \BA{\cdots} \\
    \end{align}

    \BA{Proven that in the \emph{compressible} case, $\frac{d\rmE}{dt}$ evaluates as
    {\small \begin{equation}
        \frac{d\rmE}{dt}  =  \int_{\bfGamma}\left[- \frac{1}{2\rmEu\rho}\|\bfp\|^{2}\bfp - \frac{p}{2\rho}\bfp + \frac{1}{\rmEu\rmRe_{f}}\nabla\left[\frac{1}{\rho}\bfp\right]\cdot\frac{1}{\rho}\bfp - \frac{p}{2\rho}\bfp + \frac{1}{2\rmPe}\nabla\left[\frac{p}{\rho} + \frac{1}{\beta}\bfB\wedge\bfE\right]\right]\cdot\bfn
    \end{equation}}}
    

    
    \section*{Summary}
        \BA{Summary.}
    