\subsubsection{Example 1: The Heat Equation (Revisited)}
    Consider a new coupled system, identical to the original heat equation (\ref{eqn:heat equation}) in the continuous case:
    \begin{align}
                    0  &=  \widetilde{u} - u  \\
        \partial_{t}u  &=  \frac{1}{\rmPe}\Delta\widetilde{u}
    \end{align}
    with homogeneous Dirichlet or Neumann BCs on \emph{both} $u$ and $\widetilde{u}$. This can be cast into a weak form as follows, wherein one seeks $u_{-}  \in  \calU_{-}$, $\widetilde{u}  \in  \widetilde{\calU}$ such that:
    \begin{align}
        &\forall  v  \in  \calV,  &  0  &=  \left\langle v, \widetilde{u}\rangle_{\bfOmega\otimes T} - \langle v, u\right\rangle_{\bfOmega\otimes T}  \label{eqn:heat equation auxiliary weak form 1}  \\
        &\forall  \widetilde{v}  \in  \widetilde{\calV},  &  \left\langle\widetilde{v}, \partial_{t}u\right\rangle_{\bfOmega\otimes T}  &=  - \frac{1}{\rmPe}\left\langle\nabla\widetilde{v}, \nabla\widetilde{u}\right\rangle_{\bfOmega\otimes T}  \label{eqn:heat equation auxiliary weak form 2}
    \end{align}
    where similarly for conciseness, $u$ is equal to $u_{-}$ with the addition of a lifting of an extension, $u_{0}$, of the ICs, $\widehat{u}_{0}$. Note, crucially there is no lifting of ICs on $\widetilde{u}$.

    The following subspace conditions (likely with equality for well-posedness) can then be safely assumed, since no ICs are being enforced on $\partial_{t}\calU$, $\widetilde{\calU}$:
    \begin{align}
        \partial_{t}\calU  \leqslant  \calV,  &&
        \widetilde{\calU}  \leqslant  \widetilde{\calV},  
    \end{align}
    Taking then:
    \begin{align}
        v  =  \partial_{t}u &\text{ in (\ref{eqn:heat equation auxiliary weak form 1})}  &&\implies  &  0  &=  \left\langle\partial_{t}u, \widetilde{u}\rangle_{\bfOmega\otimes T} - \langle\partial_{t}u, u\right\rangle_{\bfOmega\otimes T}  \\
        \widetilde{v}  = \widetilde{u} &\text{ in (\ref{eqn:heat equation auxiliary weak form 2})}  &&\implies  &  \left\langle\widetilde{u}, \partial_{t}u\right\rangle_{\bfOmega\otimes T}  &=  - \frac{1}{\rmPe}\left\langle\nabla\widetilde{u}, \nabla\widetilde{u}\right\rangle_{\bfOmega\otimes T}
    \end{align}
    Summing these:
    \begin{align}
        \left(\langle\partial_{t}u, u\rangle_{\bfOmega\otimes T} - \mst{\langle\partial_{t}u, \widetilde{u}\rangle_{\bfOmega\otimes T}}\tall\right) + \mst{\langle\widetilde{u}, \partial_{t}u\rangle_{\bfOmega\otimes T}} 
        &=  - \frac{1}{\rmPe}\langle\nabla\widetilde{u}, \nabla\widetilde{u}\rangle_{\bfOmega\otimes T}  \\
        \partial_{t}\left[\frac{1}{2}\int_{\bfOmega\otimes T}u^{2}\right]  &=  - \frac{1}{\rmPe}\int_{\bfOmega\otimes T}\left\|\nabla\widetilde{u}\right\|^{2}  \\
        \rmE\left(t^{N}\right)  &\leq  \rmE(0)
    \end{align}
    
    Since (\ref{eqn:heat equation auxiliary weak form 1}) is spatially homogeneous, if $\partial_{t}\calU$, $\widetilde{\calU}$ are tensor product spaces with $\partial_{t}\calU  =  \widetilde{\calU}  \left(=  \calV  =  \widetilde{\calV}\right)$, then (\ref{eqn:heat equation auxiliary weak form 1}) implies that
    \begin{align}
        &\forall  v  \in  \calV,  &  0  &=  \left\langle\nabla v, \nabla\widetilde{u}\rangle_{\bfOmega\otimes T} - \langle\nabla v, \nabla u\right\rangle_{\bfOmega\otimes T}.
    \end{align}
    This can be substituted into (\ref{eqn:heat equation auxiliary weak form 2}) to give,
    \begin{align}
        &\forall  v  \in  \calV,  &  \langle v, \partial_{t}u\rangle_{\bfOmega\otimes T}  &=  - \frac{1}{\rmPe}\langle\nabla v, \nabla u\rangle_{\bfOmega\otimes T},
    \end{align}
    decoupling the auxiliary equation. This is identical to the weak form that was initially considered for the heat equation. If $\bbU^{\rmh}  <  \calU$ is taken to be CG-in-time, then $\bbV^{\rmh}  =  \partial_{t}\bbU^{\rmh}$ must be DG-in-time, giving the CG-DG-in-time discretization considered above. As such, deriving this original weak form indirectly via the auxiliary space $\widetilde{\calU}$, one can observe why both:
    \begin{itemize}
        \item  The GL timesteppers derived from a CG-DG discretization of the heat equation necessarily satisfy dissipativity.
        \item  The dissipation in $\rmE$ is \emph{not} equal to $\frac{1}{\rmPe}\int_{\bfOmega\otimes T}\|\nabla u\|^{2}$, as it is in fact equal to the similar $\frac{1}{\rmPe}\int_{\bfOmega\otimes T}\left\|\nabla\widetilde{u}\right\|^{2}$, the norm on the \emph{auxiliary} space.
    \end{itemize}
    