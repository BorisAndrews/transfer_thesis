\subsection{Valid Collision Operators Models}\label{cha:collision operators}
    For a model for the (local) collision operators $\bfC_{\pm_{1}\pm_{2}}^{(0)}$ to be valid for use in the pseudo-particle model, it must satisfy 2 sets of validity conditions:
    \begin{enumerate}
        \item  The first set of validity conditions on $\bfC_{\pm_{1}\pm_{2}}^{(0)}$ are the momentum and energy conservation conditions from Corollary \ref{cor:phase-restricted conservation on local collision operators}. Again, the asymptotic equalities, $sim$, shall be dropped for equalities, $=$.

        \item  The second set of validity conditions on $\bfC_{\pm_{1}\pm_{2}}^{(0)}$ are the non-negativity conditions: $\forall \bfx, \bfv, t$ and $\forall \delta t$ sufficiently small,
        \begin{equation}\label{eqn:linearized collision operator non-negativity}
            \widehat{\exp\left[- \delta t\frac{\rme^{2}}{\rmm_{\pm}}i\bfomega\cdot\widehat{\calC}_{\pm}\right]}  \geq  0.
        \end{equation}
        Since $\widehat{\calC}_{\pm}$ are functions of $\bfC_{\pm_{1}\pm_{2}}^{(0)}$ (\ref{eqn:linearized collision operator}) this is in fact a condition on $\bfC_{\pm_{1}\pm_{2}}^{(0)}$. This is a simplification of the PDF non-negativity condition (\ref{eqn:PDF non-negativity condition}) as noted in the previous subsection, simplified from the final form (\ref{eqn:Levy process characteristic function}).
    \end{enumerate}
    
    We consider in this subsection examples of such collision operator models that satisfy both of these conditions.

    \line

    \begin{definition}[Lénard--Bernstein operators \BA{(Should ask Paul Dellar for a reference here.)}]
        In the notation of this thesis, Lénard--Bernstein collision operator models $\bfC_{ss'}^{\rm LB}$ are those composed of a drift and a diffusion term,
        \begin{equation}\label{eqn:Lénard--Bernstein operator}
            \bfC_{ss'}^{\rm LB}  =  \nu_{ss'}\frac{\rho_{s'}}{\rmm_{s'}}f_{s}(\bfv - \bfu_{s'}) + \nabla_{\bfv}\left[\nu_{ss'}\frac{\rho_{s'}}{\rmm_{s'}}D_{ss'}f_{s}\right],
        \end{equation}
        for drift and diffusion constants $\nu_{ss'}$, $D_{ss'}$ respectively, potentially functions of the fluid parameters, where $(\bfu_{s})_{s}$ are the phase-specific velocities,
        \begin{equation}
            \bfu_{s}  :=  \frac{1}{\rho_{s}}\int_{\bfv}f_{s}\rmm_{s}\bfv.
        \end{equation}
    \end{definition}

    For these to satisfy the first validity condition (Corollary \ref{cor:phase-restricted conservation on local collision operators}) one requires $\forall s, s'$:
    \begin{align}
        \nu_{ss'}          =  \nu_{s's},  &&
        D_{ss'} + D_{s's}  =  3(\theta_{s} + \theta_{s'}) + \|\bfu_{s} - \bfu_{s'}\|^{2},
    \end{align}
    where similarly $(\theta_{s})_{s}$ are the phase-specific temperatures,
    \begin{equation}
        \theta_{s}  :=  \frac{1}{\rho_{s}}\int_{\bfv}f_{s}\frac{1}{3}\rmm_{s}\|\bfv\|^{2}.
    \end{equation}
    
    Consider then $\calC_{\pm}^{\rm LB}$ in the 2-phase ion--electron $\delta\!f$ model, mapping $(\nu_{+-} = \nu_{-+})  \mapsto  (- \nu_{+-} = - \nu_{-+})$ to match physical intuition. Note first that in the $\delta\!f$ model, these identities can be written up to leading order in terms of the \emph{thermalized} phase-specific velocity and temperature, $(\bfu_{s})_{s}  \mapsto  (\bfu_{s}^{(0)})_{s}$ and $(\theta_{s})_{s}  \mapsto  (\theta_{s}^{(0)})_{s}$, defined akin to the thermalized current density (\ref{eqn:thermalized current density}) with $(f_{s})_{s}  \mapsto  (f_{s}^{(0)})_{s}$ as:
    \begin{align}
        \bfu_{s}    :=  \frac{1}{\rho_{s}}\int_{\bfv}f_{s}\rmm_{s}\bfv,  &&
        \theta_{s}  :=  \frac{1}{\rho_{s}}\int_{\bfv}f_{s}\frac{1}{3}\rmm_{s}\|\bfv\|^{2}.
    \end{align}
    The only terms in the Lénard--Bernstein operators (\ref{eqn:Lénard--Bernstein operator}) that are \emph{not} determined by the fluid moments $(\rho_{s})_{s}$, $\bfp$, $p$ up to leading order are then the distribution functions $(f_{s})_{s}$. Thus, when taking the derivatives $\partial\bfC_{\pm_{1}\pm_{2}}^{\rm LB}$ in (\ref{eqn:linearized collision operator}) against $\delta\!f$ corrections with 0 such moments, $\calC_{\pm}^{\rm LB}$ evaluate up to leading-order as simply
    \begin{equation}
        \partial\bfC_{\pm_{1}\pm_{2}}^{\rm LB}  \sim  \pm_{1}\pm_{2}\left(\nu_{\pm_{1}\pm_{2}}\frac{\rho_{\pm_{2}}}{\rmm_{\pm_{2}}}\delta\!f_{\pm_{1}}\left(\bfv - \bfu_{\pm_{2}}^{(0)}\right) + \nabla_{\bfv}\left[\nu_{\pm_{1}\pm_{2}}\frac{\rho_{\pm_{2}}}{\rmm_{\pm_{2}}}D_{\pm_{1}\pm_{2}}\delta\!f_{\pm_{1}}\right]\right).
    \end{equation}
    $\calC_{\pm}^{\rm LB}$ then evaluate as $\calC_{\pm}^{\rm LB}  =  \partial\bfC_{\pm\pm}^{\rm LB} - \partial\bfC_{\pm\mp}^{\rm LB}$.
    
    Presuming $\nu_{\pm_{1}\pm_{2}}$ and $D_{\pm_{1}\pm_{2}}$ are all positive, these satisfy the second set of validity conditions, the non-negativity conditions (\ref{eqn:linearized collision operator non-negativity}). With the LHS of the linearized Boltzmann equation (\ref{eqn:linearized Boltzmann equation simplified}) taking a Fokker--Planck-like form \cite{Fokker_1914, Planck_1917}, pseudo-particles under the Lénard--Bernstein collision operators move up to leading order according to the Ornstein--Uhlenbeck-like \cite{Gardiner_1985, Karatzas_Shreve_1991, Gard_1998} SDE system:
    \begin{align}
        \rmd\bfX  &=  \bfV\rmd t  \label{eqn:particle motion SDE}  \\
        \rmd\bfV  &=  \pm\frac{\rme}{\rmm_{\pm}}(\bfE + \bfV\wedge\bfB)\rmd t + \lambda_{\pm}(\bfu - \bfV)\rmd t + \sigma_{\pm}\rmd\bfW  \label{eqn:particle forcing SDE}
    \end{align}
    where:
    \begin{align}
        \lambda_{\pm}                &:=  \frac{\rme^{2}}{\rmm_{\pm}}\left(\nu_{\pm\pm}\frac{\rho_{\pm}}{\rmm_{\pm}} + \nu_{\pm\mp}\frac{\rho_{\mp}}{\rmm_{\mp}}\right)  \\
        \frac{1}{2}\sigma_{\pm}^{2}  &:=  \frac{\rme^{2}}{\rmm_{\pm}}\left(\nu_{\pm\pm}\frac{\rho_{\pm}}{\rmm_{\pm}}D_{\pm\pm} + \nu_{\pm\mp}\frac{\rho_{\mp}}{\rmm_{\mp}}D_{\pm\mp}\right)
    \end{align}
    and $\bfW$ is a standard Wiener process, independent for each pseudo-particle.

    \begin{remark}[Non-local collision operators]
        I'd love to talk now about non-local collision operators including fractional derivatives as discussed in Section \ref{cha:FEM vs. PiC}, but I just can't get them to work with the energy conservation conditions from Corollary \ref{cor:phase-restricted conservation on local collision operators}. I've got reason to expect them not to do so either, as Lévy flights and Lévy stable distributions with stability parameter $< 2$ have unbounded second moments, which naturally correspond to energy. It's a shame, as I think the high-amplitude jumps would give a much greater physical behavior as discussed earlier, however reconciling that with the energy balance conditions is just a problem for another time I guess! I think this just requires a sit pack and a rethink really.

        For the rest of this thesis, I'll just assume the Lénard--Bernstein collision operator model, and leave the consideration of further collision operator models for another time.
    \end{remark}
    