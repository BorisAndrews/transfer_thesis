\section{Pseudo-Particle Generation/Elimination}
    The PiC algorithm here depends on modeling pseudo-particles only when needed, i.e. only when kinetic effects are large and the distibution functions, $f_{\pm}$, are far from thermalization, $f_{\pm}^{(0)}$. This requires pseudo-particles---in a numerical sense, objects from a class of pseudo-particles---to be:
    \begin{itemize}
        \item  Generated when required, i.e. constructed as computational objects.
        \item  Eliminated when not required, i.e. destructed as computational objects.
    \end{itemize}
    The latter, pseudo-particle elimination, shall be discussed here first.
    
    
    \subsection*{Elimination}
    \begin{definition}[Bhatnagar--Gross--Krook (BGK) collision operator]
        In \cite{Bhatnagar_Gross_Krook_1954} Bhatnagar, Gross and Krook propose a linearized approximation to the collision operator. In the notation used in this thesis, this states, for each phase $s$,
        \begin{equation}
            \frac{{\rmq_{s}}^{2}}{\rmm_{s}}\nabla_{\bfv}\cdot\bfC_{ss}^{(0)}  =  \nu_{s}\left(f_{s}^{(0)} - f_{s}\right),
        \end{equation}
        for a (potentially) phase-dependent decay parameter, $\nu_{s}$.
    \end{definition}
    This can naturally be written in the $\delta\!f$ model in the form
    \begin{equation}
        \frac{{\rmq_{s}}^{2}}{\rmm_{s}}\nabla_{\bfv}\cdot\bfC_{ss}^{(0)}  =  - \nu_{s}\delta\!f_{s}.
     \end{equation}
    Introducing a BGK-like component to the intra-phase (local) collision operators, $\bfC_{\pm\pm}^{(0)}$, therefore modifies the linearized Boltzmann equations (\ref{eqn:linearized Boltzmann equation}) via the introduction of similar $- \nu_{\pm}\delta\!f_{\pm}$ terms on the RHS:
    \begin{equation}
        \partial_{t}\delta\!f_{\pm} + \cdots  =  \calF - \nu_{\pm}\delta\!f_{\pm}
    \end{equation}

    \BA{How does such a term manifest in the pseudo-particle model?}

    \BA{A particle model---in the physical sense of Section \ref{cha:particle models}---where particles cease to exist after periods of time sampled from an exponential distribution ${\rm Exp}[\nu_{s}]$---essentially with half-life ${\nu_{s}}^{- 1}$---would result in the introduction of a RHS term in the Boltzmann equation (\ref{eqn:Boltzmann equation}) for phase $s$ of the form $- \nu_{s}f_{s}$.}
    
    \subsection*{Generation}
    The PiC algorithm here relies on ``generating particles''---i.e. in a computational sense, creating new particle objects in the simulation---with chosen weights during the numerical timestepper, according to the size of $\calF$.

    \BA{Number of particles limited by computational capacity/parameters.}
    
    