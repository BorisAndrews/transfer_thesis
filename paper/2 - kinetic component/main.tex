\chapter{Kinetic Component}\label{cha:kinetic component}
    Recall the kinetic component of the $\delta\!f$ model, the linearized Boltzmann equations (\ref{eqn:linearized Boltzmann equation}):
    \begin{multline}\label{eqn:linearized Boltzmann equation simplified}
        \partial_{t}\delta\!f_{\pm} + \nabla_{\bfx}\cdot[\delta\!f_{\pm}\bfv] \pm \frac{\rme}{\rmm_{\pm}}\nabla_{\bfv}\cdot[\delta\!f_{\pm}(\bfE + \bfv\wedge\bfB)]  \\
        \mp \frac{\rme^{2}}{\rmm_{\pm}}\nabla_{\bfv}\cdot\left[\partial\bfC_{\pm+}^{(0)}\left[\left(f_{\pm}^{(0)}, f_{+}^{(0)}\right); \left({\color{white} f_{+}^{(0)}}\!\!\!\!\!\!\!\!\!\delta\!f_{\pm}, \delta\!f_{+}\right)\right]\right.  \\
        \left.- \partial\bfC_{\pm-}^{(0)}\left[\left(f_{\pm}^{(0)}, f_{-}^{(0)}\right); \left({\color{white} f_{+}^{(0)}}\!\!\!\!\!\!\!\!\!\delta\!f_{\pm}, \delta\!f_{-}\right)\right]\right]  =  \calF,
    \end{multline}
    where $\calF$ denotes the inhomogeneous RHS,
    \begin{multline}
        \calF[\rho_{\pm}, \bfp, p](\bfx, \bfv; t)  \\
        :=  - \partial f_{\pm}^{(0)}[(\rho_{\pm}, \bfp, p); \partial_{t}[(\rho_{\pm}, \bfp, p)]] - \partial f_{\pm}^{(0)}[(\rho_{\pm}, \bfp, p); \nabla_{\bfx}\cdot[\bfv\otimes(\rho_{\pm}, \bfp, p)]]  \\
        - \frac{\rho_{\rmC}}{\rho_{\rmM}}\nabla_{\bfv}\cdot\left[f_{\pm}^{(0)}\left(\bfE + \frac{1}{\rho_{\rmC}}\bfj^{(0)}\wedge\bfB\right) - \nabla_{\bfv}\left[f_{\pm}^{(0)}\left(\frac{1}{\rho_{\rmC}}\bfj^{(0)} - \bfu\right)\cdot\left(\bfE + \bfu\wedge\bfB\right)\right]\right],
    \end{multline}
    and asymptotic equalities, $\sim$, has been substituted for equality, $=$, for sake of presentation.

    For the numerical solution of (\ref{eqn:linearized Boltzmann equation simplified}) a Monte Carlo pseudo-particle approximation is proposed, whereby as the simulation progresses in time, particles with (crucially) positive \emph{or} negative weights:
    \begin{enumerate}
        \item  Are generated/eliminated stochastically with initial positions, $\bfx$, and velocities, $\bfv$, according to the RHS of (\ref{eqn:linearized Boltzmann equation simplified}), $\calF$.
        \item  Move through the simulation domain according to SDEs determined by the LHS of (\ref{eqn:linearized Boltzmann equation simplified}).
    \end{enumerate}
    We discuss each of these 2 phases of the simulation algorithm in turn:


    \section{Preserved Structures}
    \BA{Introduction.}
    
    Consider first those quantities that are conserved by the transient system, so as to seek discretisations which better represent the physical behaviour of the system by \emph{also} conserved these quantities. 
    
    \cite{LHF22} considers conservation of the following 3 quantities, which the authors define in the incompressible case as: \BA{(Oops I've never defined $\bfA$! That should probably be in the introduction...)}
    \begin{center}\begin{tabular}{ c c c }
        Properties  &  Symbol  &  Definition  \\
        \hline\hline
        Energy  &  $\rmE$  &  $\int_{\bfOmega}\left[\frac{1}{\rmEu\rho}\|\bfp\|^{2} + p + \frac{1}{\beta}\|\bfB\|^{2}\right]$  \\
        Magnetic helicity  &  $\rmH_{\rmM}$  &  $\int_{\bfOmega}\bfA\cdot\bfB$  \\
        Hybrid helicity  &  $\rmH_{\rmH}$  &  $\int_{\bfOmega}(a\bfA + \bfp)\cdot(b\bfB + \nabla\wedge\bfp)$
    \end{tabular}\end{center}
    where $a$, $b$ satisfy the relation $a + b  =  \frac{4}{\beta\rmRH}$. \BA{(What do these represent \emph{physically}? Diagrams!)} Taking the derivatives of these quantities over time (still in the incompressible system) gives
    \begin{align}
        \frac{d\rmE}{dt}  &=  \BA{\cdots}  \\
        \frac{d\rmH_{\rmM}}{dt}  &=  \int_{\bfGamma}(- \varphi\bfB + \bfA\wedge\bfE)\cdot\bfn - \frac{2}{\rmRem}\int_{\bfOmega}\bfB\cdot\bfj  \\
        \frac{d\rmH_{\rmH}}{dt}  &=  \BA{\cdots} \\
    \end{align}

    \BA{Proven that in the \emph{compressible} case, $\frac{d\rmE}{dt}$ evaluates as
    {\small \begin{equation}
        \frac{d\rmE}{dt}  =  \int_{\bfGamma}\left[- \frac{1}{2\rmEu\rho}\|\bfp\|^{2}\bfp - \frac{p}{2\rho}\bfp + \frac{1}{\rmEu\rmRe_{f}}\nabla\left[\frac{1}{\rho}\bfp\right]\cdot\frac{1}{\rho}\bfp - \frac{p}{2\rho}\bfp + \frac{1}{2\rmPe}\nabla\left[\frac{p}{\rho} + \frac{1}{\beta}\bfB\wedge\bfE\right]\right]\cdot\bfn
    \end{equation}}}
    
    \section{Preserved Structures}
    \BA{Introduction.}
    
    Consider first those quantities that are conserved by the transient system, so as to seek discretisations which better represent the physical behaviour of the system by \emph{also} conserved these quantities. 
    
    \cite{LHF22} considers conservation of the following 3 quantities, which the authors define in the incompressible case as: \BA{(Oops I've never defined $\bfA$! That should probably be in the introduction...)}
    \begin{center}\begin{tabular}{ c c c }
        Properties  &  Symbol  &  Definition  \\
        \hline\hline
        Energy  &  $\rmE$  &  $\int_{\bfOmega}\left[\frac{1}{\rmEu\rho}\|\bfp\|^{2} + p + \frac{1}{\beta}\|\bfB\|^{2}\right]$  \\
        Magnetic helicity  &  $\rmH_{\rmM}$  &  $\int_{\bfOmega}\bfA\cdot\bfB$  \\
        Hybrid helicity  &  $\rmH_{\rmH}$  &  $\int_{\bfOmega}(a\bfA + \bfp)\cdot(b\bfB + \nabla\wedge\bfp)$
    \end{tabular}\end{center}
    where $a$, $b$ satisfy the relation $a + b  =  \frac{4}{\beta\rmRH}$. \BA{(What do these represent \emph{physically}? Diagrams!)} Taking the derivatives of these quantities over time (still in the incompressible system) gives
    \begin{align}
        \frac{d\rmE}{dt}  &=  \BA{\cdots}  \\
        \frac{d\rmH_{\rmM}}{dt}  &=  \int_{\bfGamma}(- \varphi\bfB + \bfA\wedge\bfE)\cdot\bfn - \frac{2}{\rmRem}\int_{\bfOmega}\bfB\cdot\bfj  \\
        \frac{d\rmH_{\rmH}}{dt}  &=  \BA{\cdots} \\
    \end{align}

    \BA{Proven that in the \emph{compressible} case, $\frac{d\rmE}{dt}$ evaluates as
    {\small \begin{equation}
        \frac{d\rmE}{dt}  =  \int_{\bfGamma}\left[- \frac{1}{2\rmEu\rho}\|\bfp\|^{2}\bfp - \frac{p}{2\rho}\bfp + \frac{1}{\rmEu\rmRe_{f}}\nabla\left[\frac{1}{\rho}\bfp\right]\cdot\frac{1}{\rho}\bfp - \frac{p}{2\rho}\bfp + \frac{1}{2\rmPe}\nabla\left[\frac{p}{\rho} + \frac{1}{\beta}\bfB\wedge\bfE\right]\right]\cdot\bfn
    \end{equation}}}
    



    \section*{Summary}
        \BA{Summary.}
    