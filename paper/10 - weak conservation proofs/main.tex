\chapter{Proofs of Conservation Law in the Weak Formulation}\label{cha:weak conservation proofs}
    \section*{Stationary-state case}
        \begin{theorem*}[Energy Balance in the Stationary-State Weak Formulation]
            Provided $\calU  \leqslant  \calM$ and $1  \in  \calD$, influx and outflux \BA{(Don't like that phrasing...)} of energy on the boundary, $\bfGamma$, are in exact balance, i.e.
            \begin{equation}
                0  =  \oint_{\bfGamma}\left[- \frac{1}{2}\|\bfu\|^{2}\bfp - 2p\bfu + \frac{2}{\beta}\bfB\wedge\bfE + \frac{1}{\rmRef}\rho\bftau\cdot\bfu + \frac{1}{\rmPe}\rho\nabla\theta\right]\cdot\bfn
            \end{equation}
        \end{theorem*}
        \begin{proof}
            With the given subspace criteria, consider identities from the following test functions:
            \begin{align}
                \begin{split}
                    \bfu  \in  \calM  &\implies
                    0  =  \int_{\bfOmega}\left[- (\bfu\otimes\bfp):\nabla\bfu + p\nabla\cdot\bfu + \frac{2}{\beta}(\bfj\wedge\bfB)\cdot\bfu - \frac{1}{\rmRef}\rho\bftau:\nabla_{\rms}\bfu\right]  \\
                    &\;\;\;\;\;\;\;\;\;\;\;\;\;\;\;\;\;\;\;\;\;\;\;\;\;\;\;\;\;\;\;\;\;\;\;\;\;\;\;\;\;\;\;\;\;\;\;\;\;\;\;\;\;\;\;\;\;\;\;\;\;\;\;\;\;\;\;\;\;\;\;\;\;\;\;\;\;+ \oint_{\bfGamma}\left[- p\bfu + \frac{1}{\rmRef}\rho\bftau\cdot\bfu\right]\cdot\bfn
                \end{split}  \\
                1  \in  \calD  &\implies
                0  =  \int_{\bfOmega}\left[- p\nabla\cdot\bfu + \frac{1}{\rmRef}\rho\bftau:\nabla_{\rms}\bfu + \frac{2}{\beta\rmRem}\|\bfj\|^{2}\right] + \oint_{\bfGamma}\left[- p\bfu + \frac{1}{\rmPe}\rho\nabla\theta\right]\cdot\bfn  \\
                - \frac{2}{\beta}\bfj  \in  \calJ  &\implies
                0  =  \frac{2}{\beta}\int_{\bfOmega}\left[- \frac{1}{\rmRem}\|\bfj\|^{2} + \bfE\cdot\bfj + (\bfu\wedge\bfB)\cdot\bfj - \rmRH(\bfj\wedge\bfB)\cdot\bfj\right]  \\
                \frac{2}{\beta}\bfE  \in  \calE  &\implies
                0  =  \frac{2}{\beta}\int_{\bfOmega}\left[\bfB\cdot(\nabla\wedge\bfE) - \bfj\cdot\bfE\right] + \oint_{\bfGamma}(\bfB\wedge\bfE)\cdot\bfn
            \end{align}
            Taking the sum of these 4 identities,
            \begin{multline}
                0  =  \int_{\bfOmega}\left[- (\bfu\otimes\bfp):\nabla\bfu - \frac{2\rmRH}{\beta}(\bfj\wedge\bfB)\cdot\bfj + \frac{2}{\beta}\bfB\cdot(\nabla\wedge\bfE)\right]  \\
                \oint_{\bfGamma}\left[- 2p\bfu + \frac{2}{\beta}\bfB\wedge\bfE + \frac{1}{\rmRef}\rho\bftau\cdot\bfu + \frac{1}{\rmPe}\rho\nabla\theta\right]\cdot\bfn
            \end{multline}
            For each remaining integral on $\bfOmega$:
            \begin{itemize}
                \item  By integration by parts, $\int_{\bfOmega}(\bfu\otimes\bfp):\nabla\bfu  =  - \frac{1}{2}\int_{\bfOmega}\|\bfu\|^{2}\nabla\cdot\bfp + \frac{1}{2}\oint_{\bfGamma}\|\bfu\|^{2}\bfp\cdot\bfn$. By strong incompressibility, $\|\bfu\|^{2}\nabla\cdot\bfp  =  0$.
                \item  By orthogonality, $(\bfj\wedge\bfB)\cdot\bfj  =  0$.
                \item  By the strong solution of Faraday's law, $\bfB\cdot(\nabla\wedge\bfE)  =  0$.
            \end{itemize}
            Thus,
            \begin{equation}
                0  =  \oint_{\bfGamma}\left[- \frac{1}{2}\|\bfu\|^{2}\bfp - 2p\bfu + \frac{2}{\beta}\bfB\wedge\bfE + \frac{1}{\rmRef}\rho\bftau\cdot\bfu + \frac{1}{\rmPe}\rho\nabla\theta\right]\cdot\bfn
            \end{equation}
        \end{proof}
        
        \begin{theorem*}[Helicity Balance in the Stationary-State Weak Formulation]
            Provided $\calB  \leqslant  \calJ$, and there exists a magnetic potential $\bfA$ such that $\bfB  =  \nabla\wedge\bfA$, influx and dissipation \BA{(Don't like that phrasing...)} of magnetic helicity are in exact balance, i.e.
            \begin{equation}
                0  =  \frac{2}{\rmRem}\int_{\bfOmega}\bfB\cdot\bfj + 2\oint_{\bfGamma}(\bfE\wedge\bfA)\cdot\bfn
            \end{equation}
        \end{theorem*}
        \begin{proof}
            With $2\bfB  \in  (\calB  \leqslant)  \calJ$,
            \begin{align}
                0  =  2\int_{\bfOmega}\left[\frac{1}{\rmRem}\bfj\cdot\bfB - \bfE\cdot\bfB - (\bfu\wedge\bfB)\cdot\bfB + \rmRH(\bfj\wedge\bfB)\cdot\bfB\right]
            \end{align}
            The last two of these terms immediately vanish due to orthogonality.
            
            For $\int_{\bfOmega}\bfE\cdot\bfB$, writing $\bfB  =  \nabla\wedge\bfA$,
            \begin{align}
                \int_{\bfOmega}\bfE\cdot\bfB  &=  \int_{\bfOmega}\bfE\cdot(\nabla\wedge\bfA)  \\
                &=  \int_{\bfOmega}(\nabla\wedge\bfE)\cdot\bfA + \oint_{\bfGamma}(\bfA\wedge\bfE)\cdot\bfn  \\
                &=  \oint_{\bfGamma}(\bfA\wedge\bfE)\cdot\bfn
            \end{align}
            where the last identity holds due to exact satisfaction of Faraday's law, $\nabla\wedge\bfE  =  \bfzero$.

            $(\bfu\wedge\bfB)\cdot\bfB$ and $\rmR_{\rmH}(\bfj\wedge\bfB)\cdot\bfB$, similarly vanish by orthogonality. 
            
            Thus,
            \begin{equation}
                0  =  \frac{2}{\rmRem}\int_{\bfOmega}\bfB\cdot\bfj + 2\oint_{\bfGamma}(\bfE\wedge\bfA)\cdot\bfn
            \end{equation}
        \end{proof}
        
    
    \section*{Periodic case}
        \BA{Again, I've moved this from what was \emph{previously} what I originally had as the timestepper, but I now realise is more valid for the periodic problem. THe phrasing etc. needs to be modified, but the proof is very similar ().}
        
        \begin{theorem*}[Energy Conservation in the Periodic Weak Formulation]
            Defining the energy,
            \begin{equation}
                \rmE(t)  :=  \int_{\bfOmega\otimes\{t\}}\left(\frac{1}{2}\rho\|\bfu\|^{2} + \frac{1}{\beta}\|\bfB\|^{2} + p\right)
            \end{equation}
            provided $\calU  \leqslant  \calM$, $\calE 
             \leqslant  \calF$, and $1  \in  \calD$,
            \begin{equation}
                \rmE\left(t^{k + 1}\right) - \rmE\left(t^{k}\right)  =  \oint_{\bfGamma\otimes T^{k}}\left(- \frac{1}{2}\|\bfu\|^{2}\bfp - 2p\bfu + \frac{2}{\beta}\bfB\wedge\bfE + \frac{1}{\rmRef}\rho\bftau\cdot\bfu + \frac{1}{\rmPe}\rho\nabla\theta\right)\cdot\bfn
            \end{equation}
        \end{theorem*}
        \begin{proof}
            \begin{align}
                \begin{split}
                    \rmE\left(t^{k + 1}\right) - \rmE\left(t^{k}\right)  &=  \int_{\bfOmega\otimes\left\{t^{k + 1}\right\}}\left(\frac{1}{2}\rho\|\bfu\|^{2} + \frac{1}{\beta}\|\bfB\|^{2} + p\right)  \\
                    &\;\;\;\;\;\;\;\;\;\;\;\;\;\;\;\;\;\;\;\;\;\;\;\;\;\;\;\;\;\;\;\;\;\;\;\;\;\;\;\;\;\;\;\;- \int_{\bfOmega\otimes\left\{t^{k}\right\}}\left(\frac{1}{2}\rho\|\bfu\|^{2} + \frac{1}{\beta}\|\bfB\|^{2} + p\right)
                \end{split}  \\
                &=  \int_{T^{k}}\partial_{t}\left[\int_{\bfOmega}\left(\frac{1}{2}\rho\|\bfu\|^{2} + \frac{1}{\beta}\|\bfB\|^{2}\right)\right] + \left(\int_{\bfOmega\otimes\left\{t^{k + 1}\right\}}p - \int_{\bfOmega\otimes\left\{t^{k}\right\}}p\right)  \\
                \begin{split}
                    &=  \int_{\bfOmega\otimes T^{k}}\left(\left(\frac{1}{2}\partial_{t}\rho\|\bfu\|^{2} + \rho\bfu\cdot\partial_{t}\bfu\right) + \frac{2}{\beta}\bfB\cdot\partial_{t}\bfB\right)  \\
                    &\;\;\;\;\;\;\;\;\;\;\;\;\;\;\;\;\;\;\;\;\;\;\;\;\;\;\;\;\;\;\;\;\;\;\;\;\;\;\;\;\;\;\;\;\;\;\;\;\;\;\;\;\;+ \left(\int_{\bfOmega\otimes\left\{t^{k + 1}\right\}}p - \int_{\bfOmega\otimes\left\{t^{k}\right\}}p\right)
                \end{split}
            \end{align}
            By the strong satisfaction of mass conservation and Faraday's law, $\partial_{t}\rho$, $\partial_{t}\bfB$ can immediately be substituted for $- \nabla\cdot\bfp$, $- \nabla\wedge\bfE$ respectively.
            
            For the other two terms, one can exploit the weak formulation. For $\int_{\bfOmega\otimes T^{k}}\rho\bfu\cdot\partial_{t}\bfu$, with $\bfu  \in  \calU  \leqslant  \calM$,
            \begin{align}
                \int_{\bfOmega\otimes T^{k}}\rho\bfu\cdot\partial_{t}\bfu  &=  \langle\rho\partial_{t}\bfu, \bfu\rangle_{\bfOmega\otimes T^{k}}  \\
                \begin{split}
                    &=  \left.\left(- \langle\bfp\cdot\nabla\bfu, \bfu\rangle + \langle p, \nabla\cdot\bfu\rangle + \frac{2}{\beta}\langle\bfj\wedge\bfB, \bfu\rangle - \frac{1}{\rmRef}\langle\rho\bftau, \nabla_{s}\bfu\rangle\right)\right|_{\bfOmega\otimes T^{k}}  \\
                    &\;\;\;\;\;\;\;\;\;\;\;\;\;\;\;\;\;\;\;\;\;\;\;\;  + \left.\left(- \langle p, \bfu\cdot\bfn\rangle + \frac{1}{\rmRef}\langle\rho\bftau, \sym(\bfu\otimes\bfn)\rangle\right)\right|_{\bfGamma\otimes T^{k}}
                \end{split}  \\
                \begin{split}
                    &=  \int_{\bfOmega\otimes T^{k}}\left(- (\bfu\otimes\bfp):\nabla\bfu + p\nabla\cdot\bfu + \frac{2}{\beta}(\bfj\wedge\bfB)\cdot\bfu - \frac{1}{\rmRef}\rho\bftau:\nabla_{s}\bfu\right)  \\
                    &\;\;\;\;\;\;\;\;\;\;\;\;\;\;\;\;\;\;\;\;\;\;\;\;  + \oint_{\bfGamma\otimes T^{k}}\left(- p\bfu + \frac{1}{\rmRef}\rho\bftau\cdot\bfu\right)\cdot\bfn
                \end{split}
            \end{align}
            For $\int_{\bfOmega\otimes\left\{t^{k + 1}\right\}}p - \int_{\bfOmega\otimes\left\{t^{k}\right\}}p$, with $1  \in  \calD$,
            \begin{align}
                \int_{\bfOmega\otimes\left\{t^{k + 1}\right\}}p - \int_{\bfOmega\otimes\left\{t^{k}\right\}}p  &=  \langle p, 1\rangle_{\bfOmega\otimes\partial T^{k}}  \\
                \begin{split}
                    &=  \left.\left(- \langle p\nabla\cdot\bfu, 1\rangle + \frac{1}{\rmRef}\langle\rho\bftau:\nabla_{\rms}\bfu, 1\rangle + \frac{2}{\beta\rmRem}\left\langle\|\bfj\|^{2}, 1\right\rangle\right)\right|_{\bfOmega\otimes T^{k}}  \\
                    &\;\;\;\;\;\;\;\;\;\;\;\;\;\;\;\;\;\;\;\;\;\;\;\;+ \left.\left(- \langle p\bfu\cdot\bfn, 1\rangle + \frac{1}{\rmPe}\langle\rho\nabla\theta\cdot\bfn, 1\rangle\right)\right|_{\bfGamma\otimes T^{k}}
                \end{split}  \\
                \begin{split}
                    &=  \int_{\bfOmega\otimes T^{k}}\left(- p\nabla\cdot\bfu + \frac{1}{\rmRef}\rho\bftau:\nabla_{\rms}\bfu + \frac{2}{\beta\rmRem}\|\bfj\|^{2}\right)  \\
                    &\;\;\;\;\;\;\;\;\;\;\;\;\;\;\;\;\;\;\;\;\;\;\;\;+ \oint_{\bfGamma\otimes T^{k}}\left(- p\bfu + \frac{1}{\rmPe}\rho\nabla\theta\right)\cdot\bfn
                \end{split}
            \end{align}
    
            Discarding immediately canceling terms, this evaluates as
            \begin{multline}
                \rmE\left(t^{k + 1}\right) - \rmE\left(t^{k}\right)  =  \int_{\bfOmega\otimes T^{k}}\left( - \frac{1}{2}\|\bfu\|^{2}\nabla\cdot\bfp + \left(- (\bfu\otimes\bfp):\nabla\bfu + \frac{2}{\beta}(\bfj\wedge\bfB)\cdot\bfu\right)\right.  \\
                + \left.\frac{2}{\beta\rmRem}\|\bfj\|^{2} - \frac{2}{\beta}\bfB\cdot(\nabla\wedge\bfE)\right)  \\
                + \oint_{\bfGamma\otimes T^{k}}\left(- 2p\bfu + \frac{1}{\rmRef}\rho\bftau\cdot\bfu + \frac{1}{\rmPe}\rho\nabla\theta\right)\cdot\bfn
            \end{multline}
            
            Similar to the case in the strong formulation,
            \begin{equation}
                \int_{\bfOmega\otimes T^{k}}\left( - \frac{1}{2}\|\bfu\|^{2}\nabla\cdot\bfp + - (\bfu\otimes\bfp):\nabla\bfu\right)  =  - \frac{1}{2}\oint_{\bfGamma\otimes T^{k}}\|\bfu\|^{2}\bfp\cdot\bfn
            \end{equation}
            
            For the remaining EM integrals over $\bfOmega\otimes T^{k}$, with $\bfE  \in  \calE  \leqslant  \calF$, from the weak formulation of Ampère's law,
            \begin{equation}
                \int_{\bfOmega\otimes T^{k}}\bfB\cdot(\nabla\wedge\bfE)  =  \int_{\bfOmega\otimes T^{k}}\bfj\cdot\bfE - \oint_{\bfGamma\otimes T^{k}}(\bfB\wedge\bfE)\cdot\bfn
            \end{equation}
            From the weak formulation of the current identity then, since $\bfj  \in  \calJ$
            \begin{align}
                \int_{\bfOmega\otimes T^{k}}\bfE\cdot\bfj  &=  \int_{\bfOmega\otimes T^{k}}\left(\frac{1}{\rmRem}\|\bfj\|^{2} - (\bfu\wedge\bfB)\cdot\bfj + \rmRH(\bfj\wedge\bfB)\cdot\bfj\right)  \\
                &=  \int_{\bfOmega\otimes T^{k}}\left(\frac{1}{\rmRem}\|\bfj\|^{2} - (\bfu\wedge\bfB)\cdot\bfj\right)
            \end{align}
            Collecting these terms therefore,
            \begin{equation}
                \int_{\bfOmega\otimes T^{k}}\left((\bfj\wedge\bfB)\cdot\bfu + \frac{1}{\rmRem}\|\bfj\|^{2} - \bfB\cdot\nabla\wedge\bfE\tall\right)  =  \oint_{\bfGamma\otimes T^{k}}(\bfB\wedge\bfE)\cdot\bfn
            \end{equation}
    
            Thus,
            \begin{equation}
                \rmE\left(t^{k + 1}\right) - \rmE\left(t^{k}\right)  =  \oint_{\bfGamma\otimes T^{k}}\left(- \frac{1}{2}\|\bfu\|^{2}\bfp - 2p\bfu + \frac{2}{\beta}\bfB\wedge\bfE + \frac{1}{\rmRef}\rho\bftau\cdot\bfu + \frac{1}{\rmPe}\rho\nabla\theta\right)\cdot\bfn
            \end{equation}
        \end{proof}
    
        \begin{theorem*}[Helicity Conservation in the Periodic Weak Formulation]
            Defining the magnetic helicity,
            \begin{equation}
                \rmH_{\rmM}(t)  :=  \int_{\bfOmega\otimes\{t\}}\bfA\cdot\bfB  \left(=  \calH(\bfA|_{t})\tall\right)
            \end{equation}
            provided $\calB  \leqslant  \calJ$ and $\calB  \leqslant  \calF$,
            \begin{align}
                \rmH_{\rmM}\left(t^{k + 1}\right) - \rmH_{\rmM}\left(t^{k}\right)  &=  - \frac{2}{\rmRem}\int_{\bfOmega\otimes\left\{T^{k}\right\}}\bfB\cdot(\nabla\wedge\bfB) + \oint_{\bfGamma\otimes\left\{T^{k}\right\}}[\bfA\wedge(\bfE - \nabla\varphi)]\cdot\bfn  \\
                \left(\tall\right.&=  \left.- \frac{2}{\rmRem}\int_{T^{k}}\calH(\bfB) + \oint_{\bfGamma\otimes\left\{T^{k}\right\}}[\bfA\wedge(\bfE - \nabla\varphi)]\cdot\bfn\tall\right)
            \end{align}
        \end{theorem*}
        \begin{proof}
            \begin{align}
                \rmH_{\rmM}\left(t^{k + 1}\right) - \rmH_{\rmM}\left(t^{k}\right)  &=  \int_{\bfOmega\otimes\left\{t^{k + 1}\right\}}\bfA\cdot\bfB - \int_{\bfOmega\otimes\left\{t^{k}\right\}}\bfA\cdot\bfB  \\
                &=  \int_{\bfOmega\otimes\left\{T^{k}\right\}}\partial_{t}[\bfA\cdot\bfB]  \\
                &=  \int_{\bfOmega\otimes\left\{T^{k}\right\}}[\partial_{t}\bfA\cdot\bfB + \bfA\cdot\partial_{t}\bfB]  \\
                &=  2\int_{\bfOmega\otimes\left\{T^{k}\right\}}\bfA\cdot\partial_{t}\bfB + \oint_{\bfGamma\otimes\left\{T^{k}\right\}}[\bfA\wedge\partial_{t}\bfA]\cdot\bfn
            \end{align}
            By the exact solution of Faraday's law, $\partial_{t}\bfB$ can be substituted for $- \nabla\wedge\bfE$; by the definition of the EM potentials, $\partial_{t}\bfA$ can be substituted for $- \bfE - \nabla\varphi$.
            \begin{align}
                \rmH_{\rmM}\left(t^{k + 1}\right) - \rmH_{\rmM}\left(t^{k}\right)  &=  - 2\int_{\bfOmega\otimes\left\{T^{k}\right\}}\bfA\cdot(\nabla\wedge\bfE) + \oint_{\bfGamma\otimes\left\{T^{k}\right\}}[(\bfE + \nabla\varphi)\wedge\bfA]\cdot\bfn  \\
                &=  - 2\int_{\bfOmega\otimes\left\{T^{k}\right\}}\bfE\cdot\bfB + \oint_{\bfGamma\otimes\left\{T^{k}\right\}}[\bfA\wedge(\bfE - \nabla\varphi)]\cdot\bfn
            \end{align}
            With $\bfB  \in  \calB  \leqslant  \calJ$, from the weak formulation of the current identity,
            \begin{multline}
                \rmH_{\rmM}\left(t^{k + 1}\right) - \rmH_{\rmM}\left(t^{k}\right)  =  - 2\int_{\bfOmega\otimes\left\{T^{k}\right\}}\left(\frac{1}{\rmRem}\bfj\cdot\bfB - (\bfu\wedge\bfB)\cdot\bfB + \rmRH(\bfj\wedge\bfB)\cdot\bfB\right)  \\
                + \oint_{\bfGamma\otimes\left\{T^{k}\right\}}[\bfA\wedge(\bfE - \nabla\varphi)]\cdot\bfn
            \end{multline}
            Finally, with $\bfB  \in  \calB  \leqslant  \calF$,
            \begin{align}
                \rmH_{\rmM}\left(t^{k + 1}\right) - \rmH_{\rmM}\left(t^{k}\right)  &=  - \frac{2}{\rmRem}\int_{\bfOmega\otimes\left\{T^{k}\right\}}\bfB\cdot(\nabla\wedge\bfB) + \oint_{\bfGamma\otimes\left\{T^{k}\right\}}[\bfA\wedge(\bfE - \nabla\varphi)]\cdot\bfn  \\
                \left(\tall\right.&=  \left.- \frac{2}{\rmRem}\int_{T^{k}}\calH(\bfB) + \oint_{\bfGamma\otimes\left\{T^{k}\right\}}[\bfA\wedge(\bfE - \nabla\varphi)]\cdot\bfn\tall\right)
            \end{align}
        \end{proof}

        
    \section*{Timestepping Case}