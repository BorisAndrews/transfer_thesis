\section{Kinetic Models: The General Theory}
    \BA{What is a ``kinetic'' model?}
    
    \begin{definition}[Kinetic model]
        Here, ``kinetic'' model refers to a model wherein a fluid is modelled via a particle density function of position and velocity (and time).
    \end{definition}
    
    This is typically written as $f(\bfx, \bfv; t)$. Obviously, with the high dimensionality, this can be computationally prohibitive, so we'd like to avoid such a model if possible.
    
    \BA{Boltzmann equation definition here.}
    
    \begin{definition}[Fluid model]
        Here, ``fluid'' model refers to a reduced kinetic model, where a collection of functions of position (and time) only are modeled.
    \end{definition}
    
    These typically define certain properties of the kinetic model density function, $f$, at each position, $\bfx$, to in some way capture the physics of the full kinetic model. The Navier–Stokes equations, for example, can be found from the assumption that collisions between particles dominate the behavior of each particle, such that at each position, $\bfx$, the kinetic model density function, $f|_{\bfx}$, converges to a high entropy state (often a scaled normal distribution: a Maxwellian) characterized by 3 conserved variables:
    \begin{itemize}
        \item  Mass (per unit volume, i.e. density)
        \item  Momentum (per unit volume)
        \item  Energy (per unit volume, i.e. temperature after conservation of mass of momentum)
    \end{itemize}
    When collisions between particles do \emph{not} dominate the particle behavior—as is the case in the very low-density, high-temperature \BA{(Does this increase or decrease the size of the Coulomb collision term?)} tokamak plasmas—the assumptions that give rise to these fluid approximations often break down, implying that, without modification, these models fail to capture so-called ``kinetic effects'' \BA{(Such as?)}.
    
    \begin{figure}[!ht]
        \centering
        \begin{tikzpicture}[align = center, node distance = 4cm, auto]
            \node[1] (11) at (0, 0) {{\bf Fluid} Single-Phase Model (Navier–Stokes)};
            \node[2] (12) at (0.5, -1.75) {{\bf Kinetic} Single-Phase Model};
            \node[2] (21) at (5.5, 0) {{\bf Fluid} Multi-phase Model};
            \node[3] (22) at (6, -1.75) {{\bf Kinetic} Multi-phase model};
            \node[3] (31) at (11, 0) {{\bf Fluid} Plasma Model (MHD)};
            \node[4] (32) at (11.5, -1.75) {{\bf Kinetic} \\ Plasma Model};
    
            \path[arrow] (11) -- (12);
            \path[arrow] (11) -- (21);
            \path[arrow] (12) -- (22);
            \path[arrow] (21) -- (22);
            \path[arrow] (21) -- (31);
            \path[arrow] (22) -- (32);
            \path[arrow] (31) -- (32);
        \end{tikzpicture}
        \caption{\BA{Diagram of workflow for creating a kinetic plasma model to account for pressure anisotropy.}}
    \end{figure}
    
    \documentclass[12pt, a4paper]{report}

\documentclass[12pt, a4paper]{report}

\documentclass[12pt, a4paper]{report}

\input{template/main.tex}

\title{\BA{Title in Progress...}}
\author{Boris Andrews}
\affil{Mathematical Institute, University of Oxford}
\date{\today}


\begin{document}
    \pagenumbering{gobble}
    \maketitle
    
    
    \begin{abstract}
        Magnetic confinement reactors---in particular tokamaks---offer one of the most promising options for achieving practical nuclear fusion, with the potential to provide virtually limitless, clean energy. The theoretical and numerical modeling of tokamak plasmas is simultaneously an essential component of effective reactor design, and a great research barrier. Tokamak operational conditions exhibit comparatively low Knudsen numbers. Kinetic effects, including kinetic waves and instabilities, Landau damping, bump-on-tail instabilities and more, are therefore highly influential in tokamak plasma dynamics. Purely fluid models are inherently incapable of capturing these effects, whereas the high dimensionality in purely kinetic models render them practically intractable for most relevant purposes.

        We consider a $\delta\!f$ decomposition model, with a macroscopic fluid background and microscopic kinetic correction, both fully coupled to each other. A similar manner of discretization is proposed to that used in the recent \texttt{STRUPHY} code \cite{Holderied_Possanner_Wang_2021, Holderied_2022, Li_et_al_2023} with a finite-element model for the background and a pseudo-particle/PiC model for the correction.

        The fluid background satisfies the full, non-linear, resistive, compressible, Hall MHD equations. \cite{Laakmann_Hu_Farrell_2022} introduces finite-element(-in-space) implicit timesteppers for the incompressible analogue to this system with structure-preserving (SP) properties in the ideal case, alongside parameter-robust preconditioners. We show that these timesteppers can derive from a finite-element-in-time (FET) (and finite-element-in-space) interpretation. The benefits of this reformulation are discussed, including the derivation of timesteppers that are higher order in time, and the quantifiable dissipative SP properties in the non-ideal, resistive case.
        
        We discuss possible options for extending this FET approach to timesteppers for the compressible case.

        The kinetic corrections satisfy linearized Boltzmann equations. Using a Lénard--Bernstein collision operator, these take Fokker--Planck-like forms \cite{Fokker_1914, Planck_1917} wherein pseudo-particles in the numerical model obey the neoclassical transport equations, with particle-independent Brownian drift terms. This offers a rigorous methodology for incorporating collisions into the particle transport model, without coupling the equations of motions for each particle.
        
        Works by Chen, Chacón et al. \cite{Chen_Chacón_Barnes_2011, Chacón_Chen_Barnes_2013, Chen_Chacón_2014, Chen_Chacón_2015} have developed structure-preserving particle pushers for neoclassical transport in the Vlasov equations, derived from Crank--Nicolson integrators. We show these too can can derive from a FET interpretation, similarly offering potential extensions to higher-order-in-time particle pushers. The FET formulation is used also to consider how the stochastic drift terms can be incorporated into the pushers. Stochastic gyrokinetic expansions are also discussed.

        Different options for the numerical implementation of these schemes are considered.

        Due to the efficacy of FET in the development of SP timesteppers for both the fluid and kinetic component, we hope this approach will prove effective in the future for developing SP timesteppers for the full hybrid model. We hope this will give us the opportunity to incorporate previously inaccessible kinetic effects into the highly effective, modern, finite-element MHD models.
    \end{abstract}
    
    
    \newpage
    \tableofcontents
    
    
    \newpage
    \pagenumbering{arabic}
    %\linenumbers\renewcommand\thelinenumber{\color{black!50}\arabic{linenumber}}
            \input{0 - introduction/main.tex}
        \part{Research}
            \input{1 - low-noise PiC models/main.tex}
            \input{2 - kinetic component/main.tex}
            \input{3 - fluid component/main.tex}
            \input{4 - numerical implementation/main.tex}
        \part{Project Overview}
            \input{5 - research plan/main.tex}
            \input{6 - summary/main.tex}
    
    
    %\section{}
    \newpage
    \pagenumbering{gobble}
        \printbibliography


    \newpage
    \pagenumbering{roman}
    \appendix
        \part{Appendices}
            \input{8 - Hilbert complexes/main.tex}
            \input{9 - weak conservation proofs/main.tex}
\end{document}


\title{\BA{Title in Progress...}}
\author{Boris Andrews}
\affil{Mathematical Institute, University of Oxford}
\date{\today}


\begin{document}
    \pagenumbering{gobble}
    \maketitle
    
    
    \begin{abstract}
        Magnetic confinement reactors---in particular tokamaks---offer one of the most promising options for achieving practical nuclear fusion, with the potential to provide virtually limitless, clean energy. The theoretical and numerical modeling of tokamak plasmas is simultaneously an essential component of effective reactor design, and a great research barrier. Tokamak operational conditions exhibit comparatively low Knudsen numbers. Kinetic effects, including kinetic waves and instabilities, Landau damping, bump-on-tail instabilities and more, are therefore highly influential in tokamak plasma dynamics. Purely fluid models are inherently incapable of capturing these effects, whereas the high dimensionality in purely kinetic models render them practically intractable for most relevant purposes.

        We consider a $\delta\!f$ decomposition model, with a macroscopic fluid background and microscopic kinetic correction, both fully coupled to each other. A similar manner of discretization is proposed to that used in the recent \texttt{STRUPHY} code \cite{Holderied_Possanner_Wang_2021, Holderied_2022, Li_et_al_2023} with a finite-element model for the background and a pseudo-particle/PiC model for the correction.

        The fluid background satisfies the full, non-linear, resistive, compressible, Hall MHD equations. \cite{Laakmann_Hu_Farrell_2022} introduces finite-element(-in-space) implicit timesteppers for the incompressible analogue to this system with structure-preserving (SP) properties in the ideal case, alongside parameter-robust preconditioners. We show that these timesteppers can derive from a finite-element-in-time (FET) (and finite-element-in-space) interpretation. The benefits of this reformulation are discussed, including the derivation of timesteppers that are higher order in time, and the quantifiable dissipative SP properties in the non-ideal, resistive case.
        
        We discuss possible options for extending this FET approach to timesteppers for the compressible case.

        The kinetic corrections satisfy linearized Boltzmann equations. Using a Lénard--Bernstein collision operator, these take Fokker--Planck-like forms \cite{Fokker_1914, Planck_1917} wherein pseudo-particles in the numerical model obey the neoclassical transport equations, with particle-independent Brownian drift terms. This offers a rigorous methodology for incorporating collisions into the particle transport model, without coupling the equations of motions for each particle.
        
        Works by Chen, Chacón et al. \cite{Chen_Chacón_Barnes_2011, Chacón_Chen_Barnes_2013, Chen_Chacón_2014, Chen_Chacón_2015} have developed structure-preserving particle pushers for neoclassical transport in the Vlasov equations, derived from Crank--Nicolson integrators. We show these too can can derive from a FET interpretation, similarly offering potential extensions to higher-order-in-time particle pushers. The FET formulation is used also to consider how the stochastic drift terms can be incorporated into the pushers. Stochastic gyrokinetic expansions are also discussed.

        Different options for the numerical implementation of these schemes are considered.

        Due to the efficacy of FET in the development of SP timesteppers for both the fluid and kinetic component, we hope this approach will prove effective in the future for developing SP timesteppers for the full hybrid model. We hope this will give us the opportunity to incorporate previously inaccessible kinetic effects into the highly effective, modern, finite-element MHD models.
    \end{abstract}
    
    
    \newpage
    \tableofcontents
    
    
    \newpage
    \pagenumbering{arabic}
    %\linenumbers\renewcommand\thelinenumber{\color{black!50}\arabic{linenumber}}
            \documentclass[12pt, a4paper]{report}

\input{template/main.tex}

\title{\BA{Title in Progress...}}
\author{Boris Andrews}
\affil{Mathematical Institute, University of Oxford}
\date{\today}


\begin{document}
    \pagenumbering{gobble}
    \maketitle
    
    
    \begin{abstract}
        Magnetic confinement reactors---in particular tokamaks---offer one of the most promising options for achieving practical nuclear fusion, with the potential to provide virtually limitless, clean energy. The theoretical and numerical modeling of tokamak plasmas is simultaneously an essential component of effective reactor design, and a great research barrier. Tokamak operational conditions exhibit comparatively low Knudsen numbers. Kinetic effects, including kinetic waves and instabilities, Landau damping, bump-on-tail instabilities and more, are therefore highly influential in tokamak plasma dynamics. Purely fluid models are inherently incapable of capturing these effects, whereas the high dimensionality in purely kinetic models render them practically intractable for most relevant purposes.

        We consider a $\delta\!f$ decomposition model, with a macroscopic fluid background and microscopic kinetic correction, both fully coupled to each other. A similar manner of discretization is proposed to that used in the recent \texttt{STRUPHY} code \cite{Holderied_Possanner_Wang_2021, Holderied_2022, Li_et_al_2023} with a finite-element model for the background and a pseudo-particle/PiC model for the correction.

        The fluid background satisfies the full, non-linear, resistive, compressible, Hall MHD equations. \cite{Laakmann_Hu_Farrell_2022} introduces finite-element(-in-space) implicit timesteppers for the incompressible analogue to this system with structure-preserving (SP) properties in the ideal case, alongside parameter-robust preconditioners. We show that these timesteppers can derive from a finite-element-in-time (FET) (and finite-element-in-space) interpretation. The benefits of this reformulation are discussed, including the derivation of timesteppers that are higher order in time, and the quantifiable dissipative SP properties in the non-ideal, resistive case.
        
        We discuss possible options for extending this FET approach to timesteppers for the compressible case.

        The kinetic corrections satisfy linearized Boltzmann equations. Using a Lénard--Bernstein collision operator, these take Fokker--Planck-like forms \cite{Fokker_1914, Planck_1917} wherein pseudo-particles in the numerical model obey the neoclassical transport equations, with particle-independent Brownian drift terms. This offers a rigorous methodology for incorporating collisions into the particle transport model, without coupling the equations of motions for each particle.
        
        Works by Chen, Chacón et al. \cite{Chen_Chacón_Barnes_2011, Chacón_Chen_Barnes_2013, Chen_Chacón_2014, Chen_Chacón_2015} have developed structure-preserving particle pushers for neoclassical transport in the Vlasov equations, derived from Crank--Nicolson integrators. We show these too can can derive from a FET interpretation, similarly offering potential extensions to higher-order-in-time particle pushers. The FET formulation is used also to consider how the stochastic drift terms can be incorporated into the pushers. Stochastic gyrokinetic expansions are also discussed.

        Different options for the numerical implementation of these schemes are considered.

        Due to the efficacy of FET in the development of SP timesteppers for both the fluid and kinetic component, we hope this approach will prove effective in the future for developing SP timesteppers for the full hybrid model. We hope this will give us the opportunity to incorporate previously inaccessible kinetic effects into the highly effective, modern, finite-element MHD models.
    \end{abstract}
    
    
    \newpage
    \tableofcontents
    
    
    \newpage
    \pagenumbering{arabic}
    %\linenumbers\renewcommand\thelinenumber{\color{black!50}\arabic{linenumber}}
            \input{0 - introduction/main.tex}
        \part{Research}
            \input{1 - low-noise PiC models/main.tex}
            \input{2 - kinetic component/main.tex}
            \input{3 - fluid component/main.tex}
            \input{4 - numerical implementation/main.tex}
        \part{Project Overview}
            \input{5 - research plan/main.tex}
            \input{6 - summary/main.tex}
    
    
    %\section{}
    \newpage
    \pagenumbering{gobble}
        \printbibliography


    \newpage
    \pagenumbering{roman}
    \appendix
        \part{Appendices}
            \input{8 - Hilbert complexes/main.tex}
            \input{9 - weak conservation proofs/main.tex}
\end{document}

        \part{Research}
            \documentclass[12pt, a4paper]{report}

\input{template/main.tex}

\title{\BA{Title in Progress...}}
\author{Boris Andrews}
\affil{Mathematical Institute, University of Oxford}
\date{\today}


\begin{document}
    \pagenumbering{gobble}
    \maketitle
    
    
    \begin{abstract}
        Magnetic confinement reactors---in particular tokamaks---offer one of the most promising options for achieving practical nuclear fusion, with the potential to provide virtually limitless, clean energy. The theoretical and numerical modeling of tokamak plasmas is simultaneously an essential component of effective reactor design, and a great research barrier. Tokamak operational conditions exhibit comparatively low Knudsen numbers. Kinetic effects, including kinetic waves and instabilities, Landau damping, bump-on-tail instabilities and more, are therefore highly influential in tokamak plasma dynamics. Purely fluid models are inherently incapable of capturing these effects, whereas the high dimensionality in purely kinetic models render them practically intractable for most relevant purposes.

        We consider a $\delta\!f$ decomposition model, with a macroscopic fluid background and microscopic kinetic correction, both fully coupled to each other. A similar manner of discretization is proposed to that used in the recent \texttt{STRUPHY} code \cite{Holderied_Possanner_Wang_2021, Holderied_2022, Li_et_al_2023} with a finite-element model for the background and a pseudo-particle/PiC model for the correction.

        The fluid background satisfies the full, non-linear, resistive, compressible, Hall MHD equations. \cite{Laakmann_Hu_Farrell_2022} introduces finite-element(-in-space) implicit timesteppers for the incompressible analogue to this system with structure-preserving (SP) properties in the ideal case, alongside parameter-robust preconditioners. We show that these timesteppers can derive from a finite-element-in-time (FET) (and finite-element-in-space) interpretation. The benefits of this reformulation are discussed, including the derivation of timesteppers that are higher order in time, and the quantifiable dissipative SP properties in the non-ideal, resistive case.
        
        We discuss possible options for extending this FET approach to timesteppers for the compressible case.

        The kinetic corrections satisfy linearized Boltzmann equations. Using a Lénard--Bernstein collision operator, these take Fokker--Planck-like forms \cite{Fokker_1914, Planck_1917} wherein pseudo-particles in the numerical model obey the neoclassical transport equations, with particle-independent Brownian drift terms. This offers a rigorous methodology for incorporating collisions into the particle transport model, without coupling the equations of motions for each particle.
        
        Works by Chen, Chacón et al. \cite{Chen_Chacón_Barnes_2011, Chacón_Chen_Barnes_2013, Chen_Chacón_2014, Chen_Chacón_2015} have developed structure-preserving particle pushers for neoclassical transport in the Vlasov equations, derived from Crank--Nicolson integrators. We show these too can can derive from a FET interpretation, similarly offering potential extensions to higher-order-in-time particle pushers. The FET formulation is used also to consider how the stochastic drift terms can be incorporated into the pushers. Stochastic gyrokinetic expansions are also discussed.

        Different options for the numerical implementation of these schemes are considered.

        Due to the efficacy of FET in the development of SP timesteppers for both the fluid and kinetic component, we hope this approach will prove effective in the future for developing SP timesteppers for the full hybrid model. We hope this will give us the opportunity to incorporate previously inaccessible kinetic effects into the highly effective, modern, finite-element MHD models.
    \end{abstract}
    
    
    \newpage
    \tableofcontents
    
    
    \newpage
    \pagenumbering{arabic}
    %\linenumbers\renewcommand\thelinenumber{\color{black!50}\arabic{linenumber}}
            \input{0 - introduction/main.tex}
        \part{Research}
            \input{1 - low-noise PiC models/main.tex}
            \input{2 - kinetic component/main.tex}
            \input{3 - fluid component/main.tex}
            \input{4 - numerical implementation/main.tex}
        \part{Project Overview}
            \input{5 - research plan/main.tex}
            \input{6 - summary/main.tex}
    
    
    %\section{}
    \newpage
    \pagenumbering{gobble}
        \printbibliography


    \newpage
    \pagenumbering{roman}
    \appendix
        \part{Appendices}
            \input{8 - Hilbert complexes/main.tex}
            \input{9 - weak conservation proofs/main.tex}
\end{document}

            \documentclass[12pt, a4paper]{report}

\input{template/main.tex}

\title{\BA{Title in Progress...}}
\author{Boris Andrews}
\affil{Mathematical Institute, University of Oxford}
\date{\today}


\begin{document}
    \pagenumbering{gobble}
    \maketitle
    
    
    \begin{abstract}
        Magnetic confinement reactors---in particular tokamaks---offer one of the most promising options for achieving practical nuclear fusion, with the potential to provide virtually limitless, clean energy. The theoretical and numerical modeling of tokamak plasmas is simultaneously an essential component of effective reactor design, and a great research barrier. Tokamak operational conditions exhibit comparatively low Knudsen numbers. Kinetic effects, including kinetic waves and instabilities, Landau damping, bump-on-tail instabilities and more, are therefore highly influential in tokamak plasma dynamics. Purely fluid models are inherently incapable of capturing these effects, whereas the high dimensionality in purely kinetic models render them practically intractable for most relevant purposes.

        We consider a $\delta\!f$ decomposition model, with a macroscopic fluid background and microscopic kinetic correction, both fully coupled to each other. A similar manner of discretization is proposed to that used in the recent \texttt{STRUPHY} code \cite{Holderied_Possanner_Wang_2021, Holderied_2022, Li_et_al_2023} with a finite-element model for the background and a pseudo-particle/PiC model for the correction.

        The fluid background satisfies the full, non-linear, resistive, compressible, Hall MHD equations. \cite{Laakmann_Hu_Farrell_2022} introduces finite-element(-in-space) implicit timesteppers for the incompressible analogue to this system with structure-preserving (SP) properties in the ideal case, alongside parameter-robust preconditioners. We show that these timesteppers can derive from a finite-element-in-time (FET) (and finite-element-in-space) interpretation. The benefits of this reformulation are discussed, including the derivation of timesteppers that are higher order in time, and the quantifiable dissipative SP properties in the non-ideal, resistive case.
        
        We discuss possible options for extending this FET approach to timesteppers for the compressible case.

        The kinetic corrections satisfy linearized Boltzmann equations. Using a Lénard--Bernstein collision operator, these take Fokker--Planck-like forms \cite{Fokker_1914, Planck_1917} wherein pseudo-particles in the numerical model obey the neoclassical transport equations, with particle-independent Brownian drift terms. This offers a rigorous methodology for incorporating collisions into the particle transport model, without coupling the equations of motions for each particle.
        
        Works by Chen, Chacón et al. \cite{Chen_Chacón_Barnes_2011, Chacón_Chen_Barnes_2013, Chen_Chacón_2014, Chen_Chacón_2015} have developed structure-preserving particle pushers for neoclassical transport in the Vlasov equations, derived from Crank--Nicolson integrators. We show these too can can derive from a FET interpretation, similarly offering potential extensions to higher-order-in-time particle pushers. The FET formulation is used also to consider how the stochastic drift terms can be incorporated into the pushers. Stochastic gyrokinetic expansions are also discussed.

        Different options for the numerical implementation of these schemes are considered.

        Due to the efficacy of FET in the development of SP timesteppers for both the fluid and kinetic component, we hope this approach will prove effective in the future for developing SP timesteppers for the full hybrid model. We hope this will give us the opportunity to incorporate previously inaccessible kinetic effects into the highly effective, modern, finite-element MHD models.
    \end{abstract}
    
    
    \newpage
    \tableofcontents
    
    
    \newpage
    \pagenumbering{arabic}
    %\linenumbers\renewcommand\thelinenumber{\color{black!50}\arabic{linenumber}}
            \input{0 - introduction/main.tex}
        \part{Research}
            \input{1 - low-noise PiC models/main.tex}
            \input{2 - kinetic component/main.tex}
            \input{3 - fluid component/main.tex}
            \input{4 - numerical implementation/main.tex}
        \part{Project Overview}
            \input{5 - research plan/main.tex}
            \input{6 - summary/main.tex}
    
    
    %\section{}
    \newpage
    \pagenumbering{gobble}
        \printbibliography


    \newpage
    \pagenumbering{roman}
    \appendix
        \part{Appendices}
            \input{8 - Hilbert complexes/main.tex}
            \input{9 - weak conservation proofs/main.tex}
\end{document}

            \documentclass[12pt, a4paper]{report}

\input{template/main.tex}

\title{\BA{Title in Progress...}}
\author{Boris Andrews}
\affil{Mathematical Institute, University of Oxford}
\date{\today}


\begin{document}
    \pagenumbering{gobble}
    \maketitle
    
    
    \begin{abstract}
        Magnetic confinement reactors---in particular tokamaks---offer one of the most promising options for achieving practical nuclear fusion, with the potential to provide virtually limitless, clean energy. The theoretical and numerical modeling of tokamak plasmas is simultaneously an essential component of effective reactor design, and a great research barrier. Tokamak operational conditions exhibit comparatively low Knudsen numbers. Kinetic effects, including kinetic waves and instabilities, Landau damping, bump-on-tail instabilities and more, are therefore highly influential in tokamak plasma dynamics. Purely fluid models are inherently incapable of capturing these effects, whereas the high dimensionality in purely kinetic models render them practically intractable for most relevant purposes.

        We consider a $\delta\!f$ decomposition model, with a macroscopic fluid background and microscopic kinetic correction, both fully coupled to each other. A similar manner of discretization is proposed to that used in the recent \texttt{STRUPHY} code \cite{Holderied_Possanner_Wang_2021, Holderied_2022, Li_et_al_2023} with a finite-element model for the background and a pseudo-particle/PiC model for the correction.

        The fluid background satisfies the full, non-linear, resistive, compressible, Hall MHD equations. \cite{Laakmann_Hu_Farrell_2022} introduces finite-element(-in-space) implicit timesteppers for the incompressible analogue to this system with structure-preserving (SP) properties in the ideal case, alongside parameter-robust preconditioners. We show that these timesteppers can derive from a finite-element-in-time (FET) (and finite-element-in-space) interpretation. The benefits of this reformulation are discussed, including the derivation of timesteppers that are higher order in time, and the quantifiable dissipative SP properties in the non-ideal, resistive case.
        
        We discuss possible options for extending this FET approach to timesteppers for the compressible case.

        The kinetic corrections satisfy linearized Boltzmann equations. Using a Lénard--Bernstein collision operator, these take Fokker--Planck-like forms \cite{Fokker_1914, Planck_1917} wherein pseudo-particles in the numerical model obey the neoclassical transport equations, with particle-independent Brownian drift terms. This offers a rigorous methodology for incorporating collisions into the particle transport model, without coupling the equations of motions for each particle.
        
        Works by Chen, Chacón et al. \cite{Chen_Chacón_Barnes_2011, Chacón_Chen_Barnes_2013, Chen_Chacón_2014, Chen_Chacón_2015} have developed structure-preserving particle pushers for neoclassical transport in the Vlasov equations, derived from Crank--Nicolson integrators. We show these too can can derive from a FET interpretation, similarly offering potential extensions to higher-order-in-time particle pushers. The FET formulation is used also to consider how the stochastic drift terms can be incorporated into the pushers. Stochastic gyrokinetic expansions are also discussed.

        Different options for the numerical implementation of these schemes are considered.

        Due to the efficacy of FET in the development of SP timesteppers for both the fluid and kinetic component, we hope this approach will prove effective in the future for developing SP timesteppers for the full hybrid model. We hope this will give us the opportunity to incorporate previously inaccessible kinetic effects into the highly effective, modern, finite-element MHD models.
    \end{abstract}
    
    
    \newpage
    \tableofcontents
    
    
    \newpage
    \pagenumbering{arabic}
    %\linenumbers\renewcommand\thelinenumber{\color{black!50}\arabic{linenumber}}
            \input{0 - introduction/main.tex}
        \part{Research}
            \input{1 - low-noise PiC models/main.tex}
            \input{2 - kinetic component/main.tex}
            \input{3 - fluid component/main.tex}
            \input{4 - numerical implementation/main.tex}
        \part{Project Overview}
            \input{5 - research plan/main.tex}
            \input{6 - summary/main.tex}
    
    
    %\section{}
    \newpage
    \pagenumbering{gobble}
        \printbibliography


    \newpage
    \pagenumbering{roman}
    \appendix
        \part{Appendices}
            \input{8 - Hilbert complexes/main.tex}
            \input{9 - weak conservation proofs/main.tex}
\end{document}

            \documentclass[12pt, a4paper]{report}

\input{template/main.tex}

\title{\BA{Title in Progress...}}
\author{Boris Andrews}
\affil{Mathematical Institute, University of Oxford}
\date{\today}


\begin{document}
    \pagenumbering{gobble}
    \maketitle
    
    
    \begin{abstract}
        Magnetic confinement reactors---in particular tokamaks---offer one of the most promising options for achieving practical nuclear fusion, with the potential to provide virtually limitless, clean energy. The theoretical and numerical modeling of tokamak plasmas is simultaneously an essential component of effective reactor design, and a great research barrier. Tokamak operational conditions exhibit comparatively low Knudsen numbers. Kinetic effects, including kinetic waves and instabilities, Landau damping, bump-on-tail instabilities and more, are therefore highly influential in tokamak plasma dynamics. Purely fluid models are inherently incapable of capturing these effects, whereas the high dimensionality in purely kinetic models render them practically intractable for most relevant purposes.

        We consider a $\delta\!f$ decomposition model, with a macroscopic fluid background and microscopic kinetic correction, both fully coupled to each other. A similar manner of discretization is proposed to that used in the recent \texttt{STRUPHY} code \cite{Holderied_Possanner_Wang_2021, Holderied_2022, Li_et_al_2023} with a finite-element model for the background and a pseudo-particle/PiC model for the correction.

        The fluid background satisfies the full, non-linear, resistive, compressible, Hall MHD equations. \cite{Laakmann_Hu_Farrell_2022} introduces finite-element(-in-space) implicit timesteppers for the incompressible analogue to this system with structure-preserving (SP) properties in the ideal case, alongside parameter-robust preconditioners. We show that these timesteppers can derive from a finite-element-in-time (FET) (and finite-element-in-space) interpretation. The benefits of this reformulation are discussed, including the derivation of timesteppers that are higher order in time, and the quantifiable dissipative SP properties in the non-ideal, resistive case.
        
        We discuss possible options for extending this FET approach to timesteppers for the compressible case.

        The kinetic corrections satisfy linearized Boltzmann equations. Using a Lénard--Bernstein collision operator, these take Fokker--Planck-like forms \cite{Fokker_1914, Planck_1917} wherein pseudo-particles in the numerical model obey the neoclassical transport equations, with particle-independent Brownian drift terms. This offers a rigorous methodology for incorporating collisions into the particle transport model, without coupling the equations of motions for each particle.
        
        Works by Chen, Chacón et al. \cite{Chen_Chacón_Barnes_2011, Chacón_Chen_Barnes_2013, Chen_Chacón_2014, Chen_Chacón_2015} have developed structure-preserving particle pushers for neoclassical transport in the Vlasov equations, derived from Crank--Nicolson integrators. We show these too can can derive from a FET interpretation, similarly offering potential extensions to higher-order-in-time particle pushers. The FET formulation is used also to consider how the stochastic drift terms can be incorporated into the pushers. Stochastic gyrokinetic expansions are also discussed.

        Different options for the numerical implementation of these schemes are considered.

        Due to the efficacy of FET in the development of SP timesteppers for both the fluid and kinetic component, we hope this approach will prove effective in the future for developing SP timesteppers for the full hybrid model. We hope this will give us the opportunity to incorporate previously inaccessible kinetic effects into the highly effective, modern, finite-element MHD models.
    \end{abstract}
    
    
    \newpage
    \tableofcontents
    
    
    \newpage
    \pagenumbering{arabic}
    %\linenumbers\renewcommand\thelinenumber{\color{black!50}\arabic{linenumber}}
            \input{0 - introduction/main.tex}
        \part{Research}
            \input{1 - low-noise PiC models/main.tex}
            \input{2 - kinetic component/main.tex}
            \input{3 - fluid component/main.tex}
            \input{4 - numerical implementation/main.tex}
        \part{Project Overview}
            \input{5 - research plan/main.tex}
            \input{6 - summary/main.tex}
    
    
    %\section{}
    \newpage
    \pagenumbering{gobble}
        \printbibliography


    \newpage
    \pagenumbering{roman}
    \appendix
        \part{Appendices}
            \input{8 - Hilbert complexes/main.tex}
            \input{9 - weak conservation proofs/main.tex}
\end{document}

        \part{Project Overview}
            \documentclass[12pt, a4paper]{report}

\input{template/main.tex}

\title{\BA{Title in Progress...}}
\author{Boris Andrews}
\affil{Mathematical Institute, University of Oxford}
\date{\today}


\begin{document}
    \pagenumbering{gobble}
    \maketitle
    
    
    \begin{abstract}
        Magnetic confinement reactors---in particular tokamaks---offer one of the most promising options for achieving practical nuclear fusion, with the potential to provide virtually limitless, clean energy. The theoretical and numerical modeling of tokamak plasmas is simultaneously an essential component of effective reactor design, and a great research barrier. Tokamak operational conditions exhibit comparatively low Knudsen numbers. Kinetic effects, including kinetic waves and instabilities, Landau damping, bump-on-tail instabilities and more, are therefore highly influential in tokamak plasma dynamics. Purely fluid models are inherently incapable of capturing these effects, whereas the high dimensionality in purely kinetic models render them practically intractable for most relevant purposes.

        We consider a $\delta\!f$ decomposition model, with a macroscopic fluid background and microscopic kinetic correction, both fully coupled to each other. A similar manner of discretization is proposed to that used in the recent \texttt{STRUPHY} code \cite{Holderied_Possanner_Wang_2021, Holderied_2022, Li_et_al_2023} with a finite-element model for the background and a pseudo-particle/PiC model for the correction.

        The fluid background satisfies the full, non-linear, resistive, compressible, Hall MHD equations. \cite{Laakmann_Hu_Farrell_2022} introduces finite-element(-in-space) implicit timesteppers for the incompressible analogue to this system with structure-preserving (SP) properties in the ideal case, alongside parameter-robust preconditioners. We show that these timesteppers can derive from a finite-element-in-time (FET) (and finite-element-in-space) interpretation. The benefits of this reformulation are discussed, including the derivation of timesteppers that are higher order in time, and the quantifiable dissipative SP properties in the non-ideal, resistive case.
        
        We discuss possible options for extending this FET approach to timesteppers for the compressible case.

        The kinetic corrections satisfy linearized Boltzmann equations. Using a Lénard--Bernstein collision operator, these take Fokker--Planck-like forms \cite{Fokker_1914, Planck_1917} wherein pseudo-particles in the numerical model obey the neoclassical transport equations, with particle-independent Brownian drift terms. This offers a rigorous methodology for incorporating collisions into the particle transport model, without coupling the equations of motions for each particle.
        
        Works by Chen, Chacón et al. \cite{Chen_Chacón_Barnes_2011, Chacón_Chen_Barnes_2013, Chen_Chacón_2014, Chen_Chacón_2015} have developed structure-preserving particle pushers for neoclassical transport in the Vlasov equations, derived from Crank--Nicolson integrators. We show these too can can derive from a FET interpretation, similarly offering potential extensions to higher-order-in-time particle pushers. The FET formulation is used also to consider how the stochastic drift terms can be incorporated into the pushers. Stochastic gyrokinetic expansions are also discussed.

        Different options for the numerical implementation of these schemes are considered.

        Due to the efficacy of FET in the development of SP timesteppers for both the fluid and kinetic component, we hope this approach will prove effective in the future for developing SP timesteppers for the full hybrid model. We hope this will give us the opportunity to incorporate previously inaccessible kinetic effects into the highly effective, modern, finite-element MHD models.
    \end{abstract}
    
    
    \newpage
    \tableofcontents
    
    
    \newpage
    \pagenumbering{arabic}
    %\linenumbers\renewcommand\thelinenumber{\color{black!50}\arabic{linenumber}}
            \input{0 - introduction/main.tex}
        \part{Research}
            \input{1 - low-noise PiC models/main.tex}
            \input{2 - kinetic component/main.tex}
            \input{3 - fluid component/main.tex}
            \input{4 - numerical implementation/main.tex}
        \part{Project Overview}
            \input{5 - research plan/main.tex}
            \input{6 - summary/main.tex}
    
    
    %\section{}
    \newpage
    \pagenumbering{gobble}
        \printbibliography


    \newpage
    \pagenumbering{roman}
    \appendix
        \part{Appendices}
            \input{8 - Hilbert complexes/main.tex}
            \input{9 - weak conservation proofs/main.tex}
\end{document}

            \documentclass[12pt, a4paper]{report}

\input{template/main.tex}

\title{\BA{Title in Progress...}}
\author{Boris Andrews}
\affil{Mathematical Institute, University of Oxford}
\date{\today}


\begin{document}
    \pagenumbering{gobble}
    \maketitle
    
    
    \begin{abstract}
        Magnetic confinement reactors---in particular tokamaks---offer one of the most promising options for achieving practical nuclear fusion, with the potential to provide virtually limitless, clean energy. The theoretical and numerical modeling of tokamak plasmas is simultaneously an essential component of effective reactor design, and a great research barrier. Tokamak operational conditions exhibit comparatively low Knudsen numbers. Kinetic effects, including kinetic waves and instabilities, Landau damping, bump-on-tail instabilities and more, are therefore highly influential in tokamak plasma dynamics. Purely fluid models are inherently incapable of capturing these effects, whereas the high dimensionality in purely kinetic models render them practically intractable for most relevant purposes.

        We consider a $\delta\!f$ decomposition model, with a macroscopic fluid background and microscopic kinetic correction, both fully coupled to each other. A similar manner of discretization is proposed to that used in the recent \texttt{STRUPHY} code \cite{Holderied_Possanner_Wang_2021, Holderied_2022, Li_et_al_2023} with a finite-element model for the background and a pseudo-particle/PiC model for the correction.

        The fluid background satisfies the full, non-linear, resistive, compressible, Hall MHD equations. \cite{Laakmann_Hu_Farrell_2022} introduces finite-element(-in-space) implicit timesteppers for the incompressible analogue to this system with structure-preserving (SP) properties in the ideal case, alongside parameter-robust preconditioners. We show that these timesteppers can derive from a finite-element-in-time (FET) (and finite-element-in-space) interpretation. The benefits of this reformulation are discussed, including the derivation of timesteppers that are higher order in time, and the quantifiable dissipative SP properties in the non-ideal, resistive case.
        
        We discuss possible options for extending this FET approach to timesteppers for the compressible case.

        The kinetic corrections satisfy linearized Boltzmann equations. Using a Lénard--Bernstein collision operator, these take Fokker--Planck-like forms \cite{Fokker_1914, Planck_1917} wherein pseudo-particles in the numerical model obey the neoclassical transport equations, with particle-independent Brownian drift terms. This offers a rigorous methodology for incorporating collisions into the particle transport model, without coupling the equations of motions for each particle.
        
        Works by Chen, Chacón et al. \cite{Chen_Chacón_Barnes_2011, Chacón_Chen_Barnes_2013, Chen_Chacón_2014, Chen_Chacón_2015} have developed structure-preserving particle pushers for neoclassical transport in the Vlasov equations, derived from Crank--Nicolson integrators. We show these too can can derive from a FET interpretation, similarly offering potential extensions to higher-order-in-time particle pushers. The FET formulation is used also to consider how the stochastic drift terms can be incorporated into the pushers. Stochastic gyrokinetic expansions are also discussed.

        Different options for the numerical implementation of these schemes are considered.

        Due to the efficacy of FET in the development of SP timesteppers for both the fluid and kinetic component, we hope this approach will prove effective in the future for developing SP timesteppers for the full hybrid model. We hope this will give us the opportunity to incorporate previously inaccessible kinetic effects into the highly effective, modern, finite-element MHD models.
    \end{abstract}
    
    
    \newpage
    \tableofcontents
    
    
    \newpage
    \pagenumbering{arabic}
    %\linenumbers\renewcommand\thelinenumber{\color{black!50}\arabic{linenumber}}
            \input{0 - introduction/main.tex}
        \part{Research}
            \input{1 - low-noise PiC models/main.tex}
            \input{2 - kinetic component/main.tex}
            \input{3 - fluid component/main.tex}
            \input{4 - numerical implementation/main.tex}
        \part{Project Overview}
            \input{5 - research plan/main.tex}
            \input{6 - summary/main.tex}
    
    
    %\section{}
    \newpage
    \pagenumbering{gobble}
        \printbibliography


    \newpage
    \pagenumbering{roman}
    \appendix
        \part{Appendices}
            \input{8 - Hilbert complexes/main.tex}
            \input{9 - weak conservation proofs/main.tex}
\end{document}

    
    
    %\section{}
    \newpage
    \pagenumbering{gobble}
        \printbibliography


    \newpage
    \pagenumbering{roman}
    \appendix
        \part{Appendices}
            \documentclass[12pt, a4paper]{report}

\input{template/main.tex}

\title{\BA{Title in Progress...}}
\author{Boris Andrews}
\affil{Mathematical Institute, University of Oxford}
\date{\today}


\begin{document}
    \pagenumbering{gobble}
    \maketitle
    
    
    \begin{abstract}
        Magnetic confinement reactors---in particular tokamaks---offer one of the most promising options for achieving practical nuclear fusion, with the potential to provide virtually limitless, clean energy. The theoretical and numerical modeling of tokamak plasmas is simultaneously an essential component of effective reactor design, and a great research barrier. Tokamak operational conditions exhibit comparatively low Knudsen numbers. Kinetic effects, including kinetic waves and instabilities, Landau damping, bump-on-tail instabilities and more, are therefore highly influential in tokamak plasma dynamics. Purely fluid models are inherently incapable of capturing these effects, whereas the high dimensionality in purely kinetic models render them practically intractable for most relevant purposes.

        We consider a $\delta\!f$ decomposition model, with a macroscopic fluid background and microscopic kinetic correction, both fully coupled to each other. A similar manner of discretization is proposed to that used in the recent \texttt{STRUPHY} code \cite{Holderied_Possanner_Wang_2021, Holderied_2022, Li_et_al_2023} with a finite-element model for the background and a pseudo-particle/PiC model for the correction.

        The fluid background satisfies the full, non-linear, resistive, compressible, Hall MHD equations. \cite{Laakmann_Hu_Farrell_2022} introduces finite-element(-in-space) implicit timesteppers for the incompressible analogue to this system with structure-preserving (SP) properties in the ideal case, alongside parameter-robust preconditioners. We show that these timesteppers can derive from a finite-element-in-time (FET) (and finite-element-in-space) interpretation. The benefits of this reformulation are discussed, including the derivation of timesteppers that are higher order in time, and the quantifiable dissipative SP properties in the non-ideal, resistive case.
        
        We discuss possible options for extending this FET approach to timesteppers for the compressible case.

        The kinetic corrections satisfy linearized Boltzmann equations. Using a Lénard--Bernstein collision operator, these take Fokker--Planck-like forms \cite{Fokker_1914, Planck_1917} wherein pseudo-particles in the numerical model obey the neoclassical transport equations, with particle-independent Brownian drift terms. This offers a rigorous methodology for incorporating collisions into the particle transport model, without coupling the equations of motions for each particle.
        
        Works by Chen, Chacón et al. \cite{Chen_Chacón_Barnes_2011, Chacón_Chen_Barnes_2013, Chen_Chacón_2014, Chen_Chacón_2015} have developed structure-preserving particle pushers for neoclassical transport in the Vlasov equations, derived from Crank--Nicolson integrators. We show these too can can derive from a FET interpretation, similarly offering potential extensions to higher-order-in-time particle pushers. The FET formulation is used also to consider how the stochastic drift terms can be incorporated into the pushers. Stochastic gyrokinetic expansions are also discussed.

        Different options for the numerical implementation of these schemes are considered.

        Due to the efficacy of FET in the development of SP timesteppers for both the fluid and kinetic component, we hope this approach will prove effective in the future for developing SP timesteppers for the full hybrid model. We hope this will give us the opportunity to incorporate previously inaccessible kinetic effects into the highly effective, modern, finite-element MHD models.
    \end{abstract}
    
    
    \newpage
    \tableofcontents
    
    
    \newpage
    \pagenumbering{arabic}
    %\linenumbers\renewcommand\thelinenumber{\color{black!50}\arabic{linenumber}}
            \input{0 - introduction/main.tex}
        \part{Research}
            \input{1 - low-noise PiC models/main.tex}
            \input{2 - kinetic component/main.tex}
            \input{3 - fluid component/main.tex}
            \input{4 - numerical implementation/main.tex}
        \part{Project Overview}
            \input{5 - research plan/main.tex}
            \input{6 - summary/main.tex}
    
    
    %\section{}
    \newpage
    \pagenumbering{gobble}
        \printbibliography


    \newpage
    \pagenumbering{roman}
    \appendix
        \part{Appendices}
            \input{8 - Hilbert complexes/main.tex}
            \input{9 - weak conservation proofs/main.tex}
\end{document}

            \documentclass[12pt, a4paper]{report}

\input{template/main.tex}

\title{\BA{Title in Progress...}}
\author{Boris Andrews}
\affil{Mathematical Institute, University of Oxford}
\date{\today}


\begin{document}
    \pagenumbering{gobble}
    \maketitle
    
    
    \begin{abstract}
        Magnetic confinement reactors---in particular tokamaks---offer one of the most promising options for achieving practical nuclear fusion, with the potential to provide virtually limitless, clean energy. The theoretical and numerical modeling of tokamak plasmas is simultaneously an essential component of effective reactor design, and a great research barrier. Tokamak operational conditions exhibit comparatively low Knudsen numbers. Kinetic effects, including kinetic waves and instabilities, Landau damping, bump-on-tail instabilities and more, are therefore highly influential in tokamak plasma dynamics. Purely fluid models are inherently incapable of capturing these effects, whereas the high dimensionality in purely kinetic models render them practically intractable for most relevant purposes.

        We consider a $\delta\!f$ decomposition model, with a macroscopic fluid background and microscopic kinetic correction, both fully coupled to each other. A similar manner of discretization is proposed to that used in the recent \texttt{STRUPHY} code \cite{Holderied_Possanner_Wang_2021, Holderied_2022, Li_et_al_2023} with a finite-element model for the background and a pseudo-particle/PiC model for the correction.

        The fluid background satisfies the full, non-linear, resistive, compressible, Hall MHD equations. \cite{Laakmann_Hu_Farrell_2022} introduces finite-element(-in-space) implicit timesteppers for the incompressible analogue to this system with structure-preserving (SP) properties in the ideal case, alongside parameter-robust preconditioners. We show that these timesteppers can derive from a finite-element-in-time (FET) (and finite-element-in-space) interpretation. The benefits of this reformulation are discussed, including the derivation of timesteppers that are higher order in time, and the quantifiable dissipative SP properties in the non-ideal, resistive case.
        
        We discuss possible options for extending this FET approach to timesteppers for the compressible case.

        The kinetic corrections satisfy linearized Boltzmann equations. Using a Lénard--Bernstein collision operator, these take Fokker--Planck-like forms \cite{Fokker_1914, Planck_1917} wherein pseudo-particles in the numerical model obey the neoclassical transport equations, with particle-independent Brownian drift terms. This offers a rigorous methodology for incorporating collisions into the particle transport model, without coupling the equations of motions for each particle.
        
        Works by Chen, Chacón et al. \cite{Chen_Chacón_Barnes_2011, Chacón_Chen_Barnes_2013, Chen_Chacón_2014, Chen_Chacón_2015} have developed structure-preserving particle pushers for neoclassical transport in the Vlasov equations, derived from Crank--Nicolson integrators. We show these too can can derive from a FET interpretation, similarly offering potential extensions to higher-order-in-time particle pushers. The FET formulation is used also to consider how the stochastic drift terms can be incorporated into the pushers. Stochastic gyrokinetic expansions are also discussed.

        Different options for the numerical implementation of these schemes are considered.

        Due to the efficacy of FET in the development of SP timesteppers for both the fluid and kinetic component, we hope this approach will prove effective in the future for developing SP timesteppers for the full hybrid model. We hope this will give us the opportunity to incorporate previously inaccessible kinetic effects into the highly effective, modern, finite-element MHD models.
    \end{abstract}
    
    
    \newpage
    \tableofcontents
    
    
    \newpage
    \pagenumbering{arabic}
    %\linenumbers\renewcommand\thelinenumber{\color{black!50}\arabic{linenumber}}
            \input{0 - introduction/main.tex}
        \part{Research}
            \input{1 - low-noise PiC models/main.tex}
            \input{2 - kinetic component/main.tex}
            \input{3 - fluid component/main.tex}
            \input{4 - numerical implementation/main.tex}
        \part{Project Overview}
            \input{5 - research plan/main.tex}
            \input{6 - summary/main.tex}
    
    
    %\section{}
    \newpage
    \pagenumbering{gobble}
        \printbibliography


    \newpage
    \pagenumbering{roman}
    \appendix
        \part{Appendices}
            \input{8 - Hilbert complexes/main.tex}
            \input{9 - weak conservation proofs/main.tex}
\end{document}

\end{document}


\title{\BA{Title in Progress...}}
\author{Boris Andrews}
\affil{Mathematical Institute, University of Oxford}
\date{\today}


\begin{document}
    \pagenumbering{gobble}
    \maketitle
    
    
    \begin{abstract}
        Magnetic confinement reactors---in particular tokamaks---offer one of the most promising options for achieving practical nuclear fusion, with the potential to provide virtually limitless, clean energy. The theoretical and numerical modeling of tokamak plasmas is simultaneously an essential component of effective reactor design, and a great research barrier. Tokamak operational conditions exhibit comparatively low Knudsen numbers. Kinetic effects, including kinetic waves and instabilities, Landau damping, bump-on-tail instabilities and more, are therefore highly influential in tokamak plasma dynamics. Purely fluid models are inherently incapable of capturing these effects, whereas the high dimensionality in purely kinetic models render them practically intractable for most relevant purposes.

        We consider a $\delta\!f$ decomposition model, with a macroscopic fluid background and microscopic kinetic correction, both fully coupled to each other. A similar manner of discretization is proposed to that used in the recent \texttt{STRUPHY} code \cite{Holderied_Possanner_Wang_2021, Holderied_2022, Li_et_al_2023} with a finite-element model for the background and a pseudo-particle/PiC model for the correction.

        The fluid background satisfies the full, non-linear, resistive, compressible, Hall MHD equations. \cite{Laakmann_Hu_Farrell_2022} introduces finite-element(-in-space) implicit timesteppers for the incompressible analogue to this system with structure-preserving (SP) properties in the ideal case, alongside parameter-robust preconditioners. We show that these timesteppers can derive from a finite-element-in-time (FET) (and finite-element-in-space) interpretation. The benefits of this reformulation are discussed, including the derivation of timesteppers that are higher order in time, and the quantifiable dissipative SP properties in the non-ideal, resistive case.
        
        We discuss possible options for extending this FET approach to timesteppers for the compressible case.

        The kinetic corrections satisfy linearized Boltzmann equations. Using a Lénard--Bernstein collision operator, these take Fokker--Planck-like forms \cite{Fokker_1914, Planck_1917} wherein pseudo-particles in the numerical model obey the neoclassical transport equations, with particle-independent Brownian drift terms. This offers a rigorous methodology for incorporating collisions into the particle transport model, without coupling the equations of motions for each particle.
        
        Works by Chen, Chacón et al. \cite{Chen_Chacón_Barnes_2011, Chacón_Chen_Barnes_2013, Chen_Chacón_2014, Chen_Chacón_2015} have developed structure-preserving particle pushers for neoclassical transport in the Vlasov equations, derived from Crank--Nicolson integrators. We show these too can can derive from a FET interpretation, similarly offering potential extensions to higher-order-in-time particle pushers. The FET formulation is used also to consider how the stochastic drift terms can be incorporated into the pushers. Stochastic gyrokinetic expansions are also discussed.

        Different options for the numerical implementation of these schemes are considered.

        Due to the efficacy of FET in the development of SP timesteppers for both the fluid and kinetic component, we hope this approach will prove effective in the future for developing SP timesteppers for the full hybrid model. We hope this will give us the opportunity to incorporate previously inaccessible kinetic effects into the highly effective, modern, finite-element MHD models.
    \end{abstract}
    
    
    \newpage
    \tableofcontents
    
    
    \newpage
    \pagenumbering{arabic}
    %\linenumbers\renewcommand\thelinenumber{\color{black!50}\arabic{linenumber}}
            \documentclass[12pt, a4paper]{report}

\documentclass[12pt, a4paper]{report}

\input{template/main.tex}

\title{\BA{Title in Progress...}}
\author{Boris Andrews}
\affil{Mathematical Institute, University of Oxford}
\date{\today}


\begin{document}
    \pagenumbering{gobble}
    \maketitle
    
    
    \begin{abstract}
        Magnetic confinement reactors---in particular tokamaks---offer one of the most promising options for achieving practical nuclear fusion, with the potential to provide virtually limitless, clean energy. The theoretical and numerical modeling of tokamak plasmas is simultaneously an essential component of effective reactor design, and a great research barrier. Tokamak operational conditions exhibit comparatively low Knudsen numbers. Kinetic effects, including kinetic waves and instabilities, Landau damping, bump-on-tail instabilities and more, are therefore highly influential in tokamak plasma dynamics. Purely fluid models are inherently incapable of capturing these effects, whereas the high dimensionality in purely kinetic models render them practically intractable for most relevant purposes.

        We consider a $\delta\!f$ decomposition model, with a macroscopic fluid background and microscopic kinetic correction, both fully coupled to each other. A similar manner of discretization is proposed to that used in the recent \texttt{STRUPHY} code \cite{Holderied_Possanner_Wang_2021, Holderied_2022, Li_et_al_2023} with a finite-element model for the background and a pseudo-particle/PiC model for the correction.

        The fluid background satisfies the full, non-linear, resistive, compressible, Hall MHD equations. \cite{Laakmann_Hu_Farrell_2022} introduces finite-element(-in-space) implicit timesteppers for the incompressible analogue to this system with structure-preserving (SP) properties in the ideal case, alongside parameter-robust preconditioners. We show that these timesteppers can derive from a finite-element-in-time (FET) (and finite-element-in-space) interpretation. The benefits of this reformulation are discussed, including the derivation of timesteppers that are higher order in time, and the quantifiable dissipative SP properties in the non-ideal, resistive case.
        
        We discuss possible options for extending this FET approach to timesteppers for the compressible case.

        The kinetic corrections satisfy linearized Boltzmann equations. Using a Lénard--Bernstein collision operator, these take Fokker--Planck-like forms \cite{Fokker_1914, Planck_1917} wherein pseudo-particles in the numerical model obey the neoclassical transport equations, with particle-independent Brownian drift terms. This offers a rigorous methodology for incorporating collisions into the particle transport model, without coupling the equations of motions for each particle.
        
        Works by Chen, Chacón et al. \cite{Chen_Chacón_Barnes_2011, Chacón_Chen_Barnes_2013, Chen_Chacón_2014, Chen_Chacón_2015} have developed structure-preserving particle pushers for neoclassical transport in the Vlasov equations, derived from Crank--Nicolson integrators. We show these too can can derive from a FET interpretation, similarly offering potential extensions to higher-order-in-time particle pushers. The FET formulation is used also to consider how the stochastic drift terms can be incorporated into the pushers. Stochastic gyrokinetic expansions are also discussed.

        Different options for the numerical implementation of these schemes are considered.

        Due to the efficacy of FET in the development of SP timesteppers for both the fluid and kinetic component, we hope this approach will prove effective in the future for developing SP timesteppers for the full hybrid model. We hope this will give us the opportunity to incorporate previously inaccessible kinetic effects into the highly effective, modern, finite-element MHD models.
    \end{abstract}
    
    
    \newpage
    \tableofcontents
    
    
    \newpage
    \pagenumbering{arabic}
    %\linenumbers\renewcommand\thelinenumber{\color{black!50}\arabic{linenumber}}
            \input{0 - introduction/main.tex}
        \part{Research}
            \input{1 - low-noise PiC models/main.tex}
            \input{2 - kinetic component/main.tex}
            \input{3 - fluid component/main.tex}
            \input{4 - numerical implementation/main.tex}
        \part{Project Overview}
            \input{5 - research plan/main.tex}
            \input{6 - summary/main.tex}
    
    
    %\section{}
    \newpage
    \pagenumbering{gobble}
        \printbibliography


    \newpage
    \pagenumbering{roman}
    \appendix
        \part{Appendices}
            \input{8 - Hilbert complexes/main.tex}
            \input{9 - weak conservation proofs/main.tex}
\end{document}


\title{\BA{Title in Progress...}}
\author{Boris Andrews}
\affil{Mathematical Institute, University of Oxford}
\date{\today}


\begin{document}
    \pagenumbering{gobble}
    \maketitle
    
    
    \begin{abstract}
        Magnetic confinement reactors---in particular tokamaks---offer one of the most promising options for achieving practical nuclear fusion, with the potential to provide virtually limitless, clean energy. The theoretical and numerical modeling of tokamak plasmas is simultaneously an essential component of effective reactor design, and a great research barrier. Tokamak operational conditions exhibit comparatively low Knudsen numbers. Kinetic effects, including kinetic waves and instabilities, Landau damping, bump-on-tail instabilities and more, are therefore highly influential in tokamak plasma dynamics. Purely fluid models are inherently incapable of capturing these effects, whereas the high dimensionality in purely kinetic models render them practically intractable for most relevant purposes.

        We consider a $\delta\!f$ decomposition model, with a macroscopic fluid background and microscopic kinetic correction, both fully coupled to each other. A similar manner of discretization is proposed to that used in the recent \texttt{STRUPHY} code \cite{Holderied_Possanner_Wang_2021, Holderied_2022, Li_et_al_2023} with a finite-element model for the background and a pseudo-particle/PiC model for the correction.

        The fluid background satisfies the full, non-linear, resistive, compressible, Hall MHD equations. \cite{Laakmann_Hu_Farrell_2022} introduces finite-element(-in-space) implicit timesteppers for the incompressible analogue to this system with structure-preserving (SP) properties in the ideal case, alongside parameter-robust preconditioners. We show that these timesteppers can derive from a finite-element-in-time (FET) (and finite-element-in-space) interpretation. The benefits of this reformulation are discussed, including the derivation of timesteppers that are higher order in time, and the quantifiable dissipative SP properties in the non-ideal, resistive case.
        
        We discuss possible options for extending this FET approach to timesteppers for the compressible case.

        The kinetic corrections satisfy linearized Boltzmann equations. Using a Lénard--Bernstein collision operator, these take Fokker--Planck-like forms \cite{Fokker_1914, Planck_1917} wherein pseudo-particles in the numerical model obey the neoclassical transport equations, with particle-independent Brownian drift terms. This offers a rigorous methodology for incorporating collisions into the particle transport model, without coupling the equations of motions for each particle.
        
        Works by Chen, Chacón et al. \cite{Chen_Chacón_Barnes_2011, Chacón_Chen_Barnes_2013, Chen_Chacón_2014, Chen_Chacón_2015} have developed structure-preserving particle pushers for neoclassical transport in the Vlasov equations, derived from Crank--Nicolson integrators. We show these too can can derive from a FET interpretation, similarly offering potential extensions to higher-order-in-time particle pushers. The FET formulation is used also to consider how the stochastic drift terms can be incorporated into the pushers. Stochastic gyrokinetic expansions are also discussed.

        Different options for the numerical implementation of these schemes are considered.

        Due to the efficacy of FET in the development of SP timesteppers for both the fluid and kinetic component, we hope this approach will prove effective in the future for developing SP timesteppers for the full hybrid model. We hope this will give us the opportunity to incorporate previously inaccessible kinetic effects into the highly effective, modern, finite-element MHD models.
    \end{abstract}
    
    
    \newpage
    \tableofcontents
    
    
    \newpage
    \pagenumbering{arabic}
    %\linenumbers\renewcommand\thelinenumber{\color{black!50}\arabic{linenumber}}
            \documentclass[12pt, a4paper]{report}

\input{template/main.tex}

\title{\BA{Title in Progress...}}
\author{Boris Andrews}
\affil{Mathematical Institute, University of Oxford}
\date{\today}


\begin{document}
    \pagenumbering{gobble}
    \maketitle
    
    
    \begin{abstract}
        Magnetic confinement reactors---in particular tokamaks---offer one of the most promising options for achieving practical nuclear fusion, with the potential to provide virtually limitless, clean energy. The theoretical and numerical modeling of tokamak plasmas is simultaneously an essential component of effective reactor design, and a great research barrier. Tokamak operational conditions exhibit comparatively low Knudsen numbers. Kinetic effects, including kinetic waves and instabilities, Landau damping, bump-on-tail instabilities and more, are therefore highly influential in tokamak plasma dynamics. Purely fluid models are inherently incapable of capturing these effects, whereas the high dimensionality in purely kinetic models render them practically intractable for most relevant purposes.

        We consider a $\delta\!f$ decomposition model, with a macroscopic fluid background and microscopic kinetic correction, both fully coupled to each other. A similar manner of discretization is proposed to that used in the recent \texttt{STRUPHY} code \cite{Holderied_Possanner_Wang_2021, Holderied_2022, Li_et_al_2023} with a finite-element model for the background and a pseudo-particle/PiC model for the correction.

        The fluid background satisfies the full, non-linear, resistive, compressible, Hall MHD equations. \cite{Laakmann_Hu_Farrell_2022} introduces finite-element(-in-space) implicit timesteppers for the incompressible analogue to this system with structure-preserving (SP) properties in the ideal case, alongside parameter-robust preconditioners. We show that these timesteppers can derive from a finite-element-in-time (FET) (and finite-element-in-space) interpretation. The benefits of this reformulation are discussed, including the derivation of timesteppers that are higher order in time, and the quantifiable dissipative SP properties in the non-ideal, resistive case.
        
        We discuss possible options for extending this FET approach to timesteppers for the compressible case.

        The kinetic corrections satisfy linearized Boltzmann equations. Using a Lénard--Bernstein collision operator, these take Fokker--Planck-like forms \cite{Fokker_1914, Planck_1917} wherein pseudo-particles in the numerical model obey the neoclassical transport equations, with particle-independent Brownian drift terms. This offers a rigorous methodology for incorporating collisions into the particle transport model, without coupling the equations of motions for each particle.
        
        Works by Chen, Chacón et al. \cite{Chen_Chacón_Barnes_2011, Chacón_Chen_Barnes_2013, Chen_Chacón_2014, Chen_Chacón_2015} have developed structure-preserving particle pushers for neoclassical transport in the Vlasov equations, derived from Crank--Nicolson integrators. We show these too can can derive from a FET interpretation, similarly offering potential extensions to higher-order-in-time particle pushers. The FET formulation is used also to consider how the stochastic drift terms can be incorporated into the pushers. Stochastic gyrokinetic expansions are also discussed.

        Different options for the numerical implementation of these schemes are considered.

        Due to the efficacy of FET in the development of SP timesteppers for both the fluid and kinetic component, we hope this approach will prove effective in the future for developing SP timesteppers for the full hybrid model. We hope this will give us the opportunity to incorporate previously inaccessible kinetic effects into the highly effective, modern, finite-element MHD models.
    \end{abstract}
    
    
    \newpage
    \tableofcontents
    
    
    \newpage
    \pagenumbering{arabic}
    %\linenumbers\renewcommand\thelinenumber{\color{black!50}\arabic{linenumber}}
            \input{0 - introduction/main.tex}
        \part{Research}
            \input{1 - low-noise PiC models/main.tex}
            \input{2 - kinetic component/main.tex}
            \input{3 - fluid component/main.tex}
            \input{4 - numerical implementation/main.tex}
        \part{Project Overview}
            \input{5 - research plan/main.tex}
            \input{6 - summary/main.tex}
    
    
    %\section{}
    \newpage
    \pagenumbering{gobble}
        \printbibliography


    \newpage
    \pagenumbering{roman}
    \appendix
        \part{Appendices}
            \input{8 - Hilbert complexes/main.tex}
            \input{9 - weak conservation proofs/main.tex}
\end{document}

        \part{Research}
            \documentclass[12pt, a4paper]{report}

\input{template/main.tex}

\title{\BA{Title in Progress...}}
\author{Boris Andrews}
\affil{Mathematical Institute, University of Oxford}
\date{\today}


\begin{document}
    \pagenumbering{gobble}
    \maketitle
    
    
    \begin{abstract}
        Magnetic confinement reactors---in particular tokamaks---offer one of the most promising options for achieving practical nuclear fusion, with the potential to provide virtually limitless, clean energy. The theoretical and numerical modeling of tokamak plasmas is simultaneously an essential component of effective reactor design, and a great research barrier. Tokamak operational conditions exhibit comparatively low Knudsen numbers. Kinetic effects, including kinetic waves and instabilities, Landau damping, bump-on-tail instabilities and more, are therefore highly influential in tokamak plasma dynamics. Purely fluid models are inherently incapable of capturing these effects, whereas the high dimensionality in purely kinetic models render them practically intractable for most relevant purposes.

        We consider a $\delta\!f$ decomposition model, with a macroscopic fluid background and microscopic kinetic correction, both fully coupled to each other. A similar manner of discretization is proposed to that used in the recent \texttt{STRUPHY} code \cite{Holderied_Possanner_Wang_2021, Holderied_2022, Li_et_al_2023} with a finite-element model for the background and a pseudo-particle/PiC model for the correction.

        The fluid background satisfies the full, non-linear, resistive, compressible, Hall MHD equations. \cite{Laakmann_Hu_Farrell_2022} introduces finite-element(-in-space) implicit timesteppers for the incompressible analogue to this system with structure-preserving (SP) properties in the ideal case, alongside parameter-robust preconditioners. We show that these timesteppers can derive from a finite-element-in-time (FET) (and finite-element-in-space) interpretation. The benefits of this reformulation are discussed, including the derivation of timesteppers that are higher order in time, and the quantifiable dissipative SP properties in the non-ideal, resistive case.
        
        We discuss possible options for extending this FET approach to timesteppers for the compressible case.

        The kinetic corrections satisfy linearized Boltzmann equations. Using a Lénard--Bernstein collision operator, these take Fokker--Planck-like forms \cite{Fokker_1914, Planck_1917} wherein pseudo-particles in the numerical model obey the neoclassical transport equations, with particle-independent Brownian drift terms. This offers a rigorous methodology for incorporating collisions into the particle transport model, without coupling the equations of motions for each particle.
        
        Works by Chen, Chacón et al. \cite{Chen_Chacón_Barnes_2011, Chacón_Chen_Barnes_2013, Chen_Chacón_2014, Chen_Chacón_2015} have developed structure-preserving particle pushers for neoclassical transport in the Vlasov equations, derived from Crank--Nicolson integrators. We show these too can can derive from a FET interpretation, similarly offering potential extensions to higher-order-in-time particle pushers. The FET formulation is used also to consider how the stochastic drift terms can be incorporated into the pushers. Stochastic gyrokinetic expansions are also discussed.

        Different options for the numerical implementation of these schemes are considered.

        Due to the efficacy of FET in the development of SP timesteppers for both the fluid and kinetic component, we hope this approach will prove effective in the future for developing SP timesteppers for the full hybrid model. We hope this will give us the opportunity to incorporate previously inaccessible kinetic effects into the highly effective, modern, finite-element MHD models.
    \end{abstract}
    
    
    \newpage
    \tableofcontents
    
    
    \newpage
    \pagenumbering{arabic}
    %\linenumbers\renewcommand\thelinenumber{\color{black!50}\arabic{linenumber}}
            \input{0 - introduction/main.tex}
        \part{Research}
            \input{1 - low-noise PiC models/main.tex}
            \input{2 - kinetic component/main.tex}
            \input{3 - fluid component/main.tex}
            \input{4 - numerical implementation/main.tex}
        \part{Project Overview}
            \input{5 - research plan/main.tex}
            \input{6 - summary/main.tex}
    
    
    %\section{}
    \newpage
    \pagenumbering{gobble}
        \printbibliography


    \newpage
    \pagenumbering{roman}
    \appendix
        \part{Appendices}
            \input{8 - Hilbert complexes/main.tex}
            \input{9 - weak conservation proofs/main.tex}
\end{document}

            \documentclass[12pt, a4paper]{report}

\input{template/main.tex}

\title{\BA{Title in Progress...}}
\author{Boris Andrews}
\affil{Mathematical Institute, University of Oxford}
\date{\today}


\begin{document}
    \pagenumbering{gobble}
    \maketitle
    
    
    \begin{abstract}
        Magnetic confinement reactors---in particular tokamaks---offer one of the most promising options for achieving practical nuclear fusion, with the potential to provide virtually limitless, clean energy. The theoretical and numerical modeling of tokamak plasmas is simultaneously an essential component of effective reactor design, and a great research barrier. Tokamak operational conditions exhibit comparatively low Knudsen numbers. Kinetic effects, including kinetic waves and instabilities, Landau damping, bump-on-tail instabilities and more, are therefore highly influential in tokamak plasma dynamics. Purely fluid models are inherently incapable of capturing these effects, whereas the high dimensionality in purely kinetic models render them practically intractable for most relevant purposes.

        We consider a $\delta\!f$ decomposition model, with a macroscopic fluid background and microscopic kinetic correction, both fully coupled to each other. A similar manner of discretization is proposed to that used in the recent \texttt{STRUPHY} code \cite{Holderied_Possanner_Wang_2021, Holderied_2022, Li_et_al_2023} with a finite-element model for the background and a pseudo-particle/PiC model for the correction.

        The fluid background satisfies the full, non-linear, resistive, compressible, Hall MHD equations. \cite{Laakmann_Hu_Farrell_2022} introduces finite-element(-in-space) implicit timesteppers for the incompressible analogue to this system with structure-preserving (SP) properties in the ideal case, alongside parameter-robust preconditioners. We show that these timesteppers can derive from a finite-element-in-time (FET) (and finite-element-in-space) interpretation. The benefits of this reformulation are discussed, including the derivation of timesteppers that are higher order in time, and the quantifiable dissipative SP properties in the non-ideal, resistive case.
        
        We discuss possible options for extending this FET approach to timesteppers for the compressible case.

        The kinetic corrections satisfy linearized Boltzmann equations. Using a Lénard--Bernstein collision operator, these take Fokker--Planck-like forms \cite{Fokker_1914, Planck_1917} wherein pseudo-particles in the numerical model obey the neoclassical transport equations, with particle-independent Brownian drift terms. This offers a rigorous methodology for incorporating collisions into the particle transport model, without coupling the equations of motions for each particle.
        
        Works by Chen, Chacón et al. \cite{Chen_Chacón_Barnes_2011, Chacón_Chen_Barnes_2013, Chen_Chacón_2014, Chen_Chacón_2015} have developed structure-preserving particle pushers for neoclassical transport in the Vlasov equations, derived from Crank--Nicolson integrators. We show these too can can derive from a FET interpretation, similarly offering potential extensions to higher-order-in-time particle pushers. The FET formulation is used also to consider how the stochastic drift terms can be incorporated into the pushers. Stochastic gyrokinetic expansions are also discussed.

        Different options for the numerical implementation of these schemes are considered.

        Due to the efficacy of FET in the development of SP timesteppers for both the fluid and kinetic component, we hope this approach will prove effective in the future for developing SP timesteppers for the full hybrid model. We hope this will give us the opportunity to incorporate previously inaccessible kinetic effects into the highly effective, modern, finite-element MHD models.
    \end{abstract}
    
    
    \newpage
    \tableofcontents
    
    
    \newpage
    \pagenumbering{arabic}
    %\linenumbers\renewcommand\thelinenumber{\color{black!50}\arabic{linenumber}}
            \input{0 - introduction/main.tex}
        \part{Research}
            \input{1 - low-noise PiC models/main.tex}
            \input{2 - kinetic component/main.tex}
            \input{3 - fluid component/main.tex}
            \input{4 - numerical implementation/main.tex}
        \part{Project Overview}
            \input{5 - research plan/main.tex}
            \input{6 - summary/main.tex}
    
    
    %\section{}
    \newpage
    \pagenumbering{gobble}
        \printbibliography


    \newpage
    \pagenumbering{roman}
    \appendix
        \part{Appendices}
            \input{8 - Hilbert complexes/main.tex}
            \input{9 - weak conservation proofs/main.tex}
\end{document}

            \documentclass[12pt, a4paper]{report}

\input{template/main.tex}

\title{\BA{Title in Progress...}}
\author{Boris Andrews}
\affil{Mathematical Institute, University of Oxford}
\date{\today}


\begin{document}
    \pagenumbering{gobble}
    \maketitle
    
    
    \begin{abstract}
        Magnetic confinement reactors---in particular tokamaks---offer one of the most promising options for achieving practical nuclear fusion, with the potential to provide virtually limitless, clean energy. The theoretical and numerical modeling of tokamak plasmas is simultaneously an essential component of effective reactor design, and a great research barrier. Tokamak operational conditions exhibit comparatively low Knudsen numbers. Kinetic effects, including kinetic waves and instabilities, Landau damping, bump-on-tail instabilities and more, are therefore highly influential in tokamak plasma dynamics. Purely fluid models are inherently incapable of capturing these effects, whereas the high dimensionality in purely kinetic models render them practically intractable for most relevant purposes.

        We consider a $\delta\!f$ decomposition model, with a macroscopic fluid background and microscopic kinetic correction, both fully coupled to each other. A similar manner of discretization is proposed to that used in the recent \texttt{STRUPHY} code \cite{Holderied_Possanner_Wang_2021, Holderied_2022, Li_et_al_2023} with a finite-element model for the background and a pseudo-particle/PiC model for the correction.

        The fluid background satisfies the full, non-linear, resistive, compressible, Hall MHD equations. \cite{Laakmann_Hu_Farrell_2022} introduces finite-element(-in-space) implicit timesteppers for the incompressible analogue to this system with structure-preserving (SP) properties in the ideal case, alongside parameter-robust preconditioners. We show that these timesteppers can derive from a finite-element-in-time (FET) (and finite-element-in-space) interpretation. The benefits of this reformulation are discussed, including the derivation of timesteppers that are higher order in time, and the quantifiable dissipative SP properties in the non-ideal, resistive case.
        
        We discuss possible options for extending this FET approach to timesteppers for the compressible case.

        The kinetic corrections satisfy linearized Boltzmann equations. Using a Lénard--Bernstein collision operator, these take Fokker--Planck-like forms \cite{Fokker_1914, Planck_1917} wherein pseudo-particles in the numerical model obey the neoclassical transport equations, with particle-independent Brownian drift terms. This offers a rigorous methodology for incorporating collisions into the particle transport model, without coupling the equations of motions for each particle.
        
        Works by Chen, Chacón et al. \cite{Chen_Chacón_Barnes_2011, Chacón_Chen_Barnes_2013, Chen_Chacón_2014, Chen_Chacón_2015} have developed structure-preserving particle pushers for neoclassical transport in the Vlasov equations, derived from Crank--Nicolson integrators. We show these too can can derive from a FET interpretation, similarly offering potential extensions to higher-order-in-time particle pushers. The FET formulation is used also to consider how the stochastic drift terms can be incorporated into the pushers. Stochastic gyrokinetic expansions are also discussed.

        Different options for the numerical implementation of these schemes are considered.

        Due to the efficacy of FET in the development of SP timesteppers for both the fluid and kinetic component, we hope this approach will prove effective in the future for developing SP timesteppers for the full hybrid model. We hope this will give us the opportunity to incorporate previously inaccessible kinetic effects into the highly effective, modern, finite-element MHD models.
    \end{abstract}
    
    
    \newpage
    \tableofcontents
    
    
    \newpage
    \pagenumbering{arabic}
    %\linenumbers\renewcommand\thelinenumber{\color{black!50}\arabic{linenumber}}
            \input{0 - introduction/main.tex}
        \part{Research}
            \input{1 - low-noise PiC models/main.tex}
            \input{2 - kinetic component/main.tex}
            \input{3 - fluid component/main.tex}
            \input{4 - numerical implementation/main.tex}
        \part{Project Overview}
            \input{5 - research plan/main.tex}
            \input{6 - summary/main.tex}
    
    
    %\section{}
    \newpage
    \pagenumbering{gobble}
        \printbibliography


    \newpage
    \pagenumbering{roman}
    \appendix
        \part{Appendices}
            \input{8 - Hilbert complexes/main.tex}
            \input{9 - weak conservation proofs/main.tex}
\end{document}

            \documentclass[12pt, a4paper]{report}

\input{template/main.tex}

\title{\BA{Title in Progress...}}
\author{Boris Andrews}
\affil{Mathematical Institute, University of Oxford}
\date{\today}


\begin{document}
    \pagenumbering{gobble}
    \maketitle
    
    
    \begin{abstract}
        Magnetic confinement reactors---in particular tokamaks---offer one of the most promising options for achieving practical nuclear fusion, with the potential to provide virtually limitless, clean energy. The theoretical and numerical modeling of tokamak plasmas is simultaneously an essential component of effective reactor design, and a great research barrier. Tokamak operational conditions exhibit comparatively low Knudsen numbers. Kinetic effects, including kinetic waves and instabilities, Landau damping, bump-on-tail instabilities and more, are therefore highly influential in tokamak plasma dynamics. Purely fluid models are inherently incapable of capturing these effects, whereas the high dimensionality in purely kinetic models render them practically intractable for most relevant purposes.

        We consider a $\delta\!f$ decomposition model, with a macroscopic fluid background and microscopic kinetic correction, both fully coupled to each other. A similar manner of discretization is proposed to that used in the recent \texttt{STRUPHY} code \cite{Holderied_Possanner_Wang_2021, Holderied_2022, Li_et_al_2023} with a finite-element model for the background and a pseudo-particle/PiC model for the correction.

        The fluid background satisfies the full, non-linear, resistive, compressible, Hall MHD equations. \cite{Laakmann_Hu_Farrell_2022} introduces finite-element(-in-space) implicit timesteppers for the incompressible analogue to this system with structure-preserving (SP) properties in the ideal case, alongside parameter-robust preconditioners. We show that these timesteppers can derive from a finite-element-in-time (FET) (and finite-element-in-space) interpretation. The benefits of this reformulation are discussed, including the derivation of timesteppers that are higher order in time, and the quantifiable dissipative SP properties in the non-ideal, resistive case.
        
        We discuss possible options for extending this FET approach to timesteppers for the compressible case.

        The kinetic corrections satisfy linearized Boltzmann equations. Using a Lénard--Bernstein collision operator, these take Fokker--Planck-like forms \cite{Fokker_1914, Planck_1917} wherein pseudo-particles in the numerical model obey the neoclassical transport equations, with particle-independent Brownian drift terms. This offers a rigorous methodology for incorporating collisions into the particle transport model, without coupling the equations of motions for each particle.
        
        Works by Chen, Chacón et al. \cite{Chen_Chacón_Barnes_2011, Chacón_Chen_Barnes_2013, Chen_Chacón_2014, Chen_Chacón_2015} have developed structure-preserving particle pushers for neoclassical transport in the Vlasov equations, derived from Crank--Nicolson integrators. We show these too can can derive from a FET interpretation, similarly offering potential extensions to higher-order-in-time particle pushers. The FET formulation is used also to consider how the stochastic drift terms can be incorporated into the pushers. Stochastic gyrokinetic expansions are also discussed.

        Different options for the numerical implementation of these schemes are considered.

        Due to the efficacy of FET in the development of SP timesteppers for both the fluid and kinetic component, we hope this approach will prove effective in the future for developing SP timesteppers for the full hybrid model. We hope this will give us the opportunity to incorporate previously inaccessible kinetic effects into the highly effective, modern, finite-element MHD models.
    \end{abstract}
    
    
    \newpage
    \tableofcontents
    
    
    \newpage
    \pagenumbering{arabic}
    %\linenumbers\renewcommand\thelinenumber{\color{black!50}\arabic{linenumber}}
            \input{0 - introduction/main.tex}
        \part{Research}
            \input{1 - low-noise PiC models/main.tex}
            \input{2 - kinetic component/main.tex}
            \input{3 - fluid component/main.tex}
            \input{4 - numerical implementation/main.tex}
        \part{Project Overview}
            \input{5 - research plan/main.tex}
            \input{6 - summary/main.tex}
    
    
    %\section{}
    \newpage
    \pagenumbering{gobble}
        \printbibliography


    \newpage
    \pagenumbering{roman}
    \appendix
        \part{Appendices}
            \input{8 - Hilbert complexes/main.tex}
            \input{9 - weak conservation proofs/main.tex}
\end{document}

        \part{Project Overview}
            \documentclass[12pt, a4paper]{report}

\input{template/main.tex}

\title{\BA{Title in Progress...}}
\author{Boris Andrews}
\affil{Mathematical Institute, University of Oxford}
\date{\today}


\begin{document}
    \pagenumbering{gobble}
    \maketitle
    
    
    \begin{abstract}
        Magnetic confinement reactors---in particular tokamaks---offer one of the most promising options for achieving practical nuclear fusion, with the potential to provide virtually limitless, clean energy. The theoretical and numerical modeling of tokamak plasmas is simultaneously an essential component of effective reactor design, and a great research barrier. Tokamak operational conditions exhibit comparatively low Knudsen numbers. Kinetic effects, including kinetic waves and instabilities, Landau damping, bump-on-tail instabilities and more, are therefore highly influential in tokamak plasma dynamics. Purely fluid models are inherently incapable of capturing these effects, whereas the high dimensionality in purely kinetic models render them practically intractable for most relevant purposes.

        We consider a $\delta\!f$ decomposition model, with a macroscopic fluid background and microscopic kinetic correction, both fully coupled to each other. A similar manner of discretization is proposed to that used in the recent \texttt{STRUPHY} code \cite{Holderied_Possanner_Wang_2021, Holderied_2022, Li_et_al_2023} with a finite-element model for the background and a pseudo-particle/PiC model for the correction.

        The fluid background satisfies the full, non-linear, resistive, compressible, Hall MHD equations. \cite{Laakmann_Hu_Farrell_2022} introduces finite-element(-in-space) implicit timesteppers for the incompressible analogue to this system with structure-preserving (SP) properties in the ideal case, alongside parameter-robust preconditioners. We show that these timesteppers can derive from a finite-element-in-time (FET) (and finite-element-in-space) interpretation. The benefits of this reformulation are discussed, including the derivation of timesteppers that are higher order in time, and the quantifiable dissipative SP properties in the non-ideal, resistive case.
        
        We discuss possible options for extending this FET approach to timesteppers for the compressible case.

        The kinetic corrections satisfy linearized Boltzmann equations. Using a Lénard--Bernstein collision operator, these take Fokker--Planck-like forms \cite{Fokker_1914, Planck_1917} wherein pseudo-particles in the numerical model obey the neoclassical transport equations, with particle-independent Brownian drift terms. This offers a rigorous methodology for incorporating collisions into the particle transport model, without coupling the equations of motions for each particle.
        
        Works by Chen, Chacón et al. \cite{Chen_Chacón_Barnes_2011, Chacón_Chen_Barnes_2013, Chen_Chacón_2014, Chen_Chacón_2015} have developed structure-preserving particle pushers for neoclassical transport in the Vlasov equations, derived from Crank--Nicolson integrators. We show these too can can derive from a FET interpretation, similarly offering potential extensions to higher-order-in-time particle pushers. The FET formulation is used also to consider how the stochastic drift terms can be incorporated into the pushers. Stochastic gyrokinetic expansions are also discussed.

        Different options for the numerical implementation of these schemes are considered.

        Due to the efficacy of FET in the development of SP timesteppers for both the fluid and kinetic component, we hope this approach will prove effective in the future for developing SP timesteppers for the full hybrid model. We hope this will give us the opportunity to incorporate previously inaccessible kinetic effects into the highly effective, modern, finite-element MHD models.
    \end{abstract}
    
    
    \newpage
    \tableofcontents
    
    
    \newpage
    \pagenumbering{arabic}
    %\linenumbers\renewcommand\thelinenumber{\color{black!50}\arabic{linenumber}}
            \input{0 - introduction/main.tex}
        \part{Research}
            \input{1 - low-noise PiC models/main.tex}
            \input{2 - kinetic component/main.tex}
            \input{3 - fluid component/main.tex}
            \input{4 - numerical implementation/main.tex}
        \part{Project Overview}
            \input{5 - research plan/main.tex}
            \input{6 - summary/main.tex}
    
    
    %\section{}
    \newpage
    \pagenumbering{gobble}
        \printbibliography


    \newpage
    \pagenumbering{roman}
    \appendix
        \part{Appendices}
            \input{8 - Hilbert complexes/main.tex}
            \input{9 - weak conservation proofs/main.tex}
\end{document}

            \documentclass[12pt, a4paper]{report}

\input{template/main.tex}

\title{\BA{Title in Progress...}}
\author{Boris Andrews}
\affil{Mathematical Institute, University of Oxford}
\date{\today}


\begin{document}
    \pagenumbering{gobble}
    \maketitle
    
    
    \begin{abstract}
        Magnetic confinement reactors---in particular tokamaks---offer one of the most promising options for achieving practical nuclear fusion, with the potential to provide virtually limitless, clean energy. The theoretical and numerical modeling of tokamak plasmas is simultaneously an essential component of effective reactor design, and a great research barrier. Tokamak operational conditions exhibit comparatively low Knudsen numbers. Kinetic effects, including kinetic waves and instabilities, Landau damping, bump-on-tail instabilities and more, are therefore highly influential in tokamak plasma dynamics. Purely fluid models are inherently incapable of capturing these effects, whereas the high dimensionality in purely kinetic models render them practically intractable for most relevant purposes.

        We consider a $\delta\!f$ decomposition model, with a macroscopic fluid background and microscopic kinetic correction, both fully coupled to each other. A similar manner of discretization is proposed to that used in the recent \texttt{STRUPHY} code \cite{Holderied_Possanner_Wang_2021, Holderied_2022, Li_et_al_2023} with a finite-element model for the background and a pseudo-particle/PiC model for the correction.

        The fluid background satisfies the full, non-linear, resistive, compressible, Hall MHD equations. \cite{Laakmann_Hu_Farrell_2022} introduces finite-element(-in-space) implicit timesteppers for the incompressible analogue to this system with structure-preserving (SP) properties in the ideal case, alongside parameter-robust preconditioners. We show that these timesteppers can derive from a finite-element-in-time (FET) (and finite-element-in-space) interpretation. The benefits of this reformulation are discussed, including the derivation of timesteppers that are higher order in time, and the quantifiable dissipative SP properties in the non-ideal, resistive case.
        
        We discuss possible options for extending this FET approach to timesteppers for the compressible case.

        The kinetic corrections satisfy linearized Boltzmann equations. Using a Lénard--Bernstein collision operator, these take Fokker--Planck-like forms \cite{Fokker_1914, Planck_1917} wherein pseudo-particles in the numerical model obey the neoclassical transport equations, with particle-independent Brownian drift terms. This offers a rigorous methodology for incorporating collisions into the particle transport model, without coupling the equations of motions for each particle.
        
        Works by Chen, Chacón et al. \cite{Chen_Chacón_Barnes_2011, Chacón_Chen_Barnes_2013, Chen_Chacón_2014, Chen_Chacón_2015} have developed structure-preserving particle pushers for neoclassical transport in the Vlasov equations, derived from Crank--Nicolson integrators. We show these too can can derive from a FET interpretation, similarly offering potential extensions to higher-order-in-time particle pushers. The FET formulation is used also to consider how the stochastic drift terms can be incorporated into the pushers. Stochastic gyrokinetic expansions are also discussed.

        Different options for the numerical implementation of these schemes are considered.

        Due to the efficacy of FET in the development of SP timesteppers for both the fluid and kinetic component, we hope this approach will prove effective in the future for developing SP timesteppers for the full hybrid model. We hope this will give us the opportunity to incorporate previously inaccessible kinetic effects into the highly effective, modern, finite-element MHD models.
    \end{abstract}
    
    
    \newpage
    \tableofcontents
    
    
    \newpage
    \pagenumbering{arabic}
    %\linenumbers\renewcommand\thelinenumber{\color{black!50}\arabic{linenumber}}
            \input{0 - introduction/main.tex}
        \part{Research}
            \input{1 - low-noise PiC models/main.tex}
            \input{2 - kinetic component/main.tex}
            \input{3 - fluid component/main.tex}
            \input{4 - numerical implementation/main.tex}
        \part{Project Overview}
            \input{5 - research plan/main.tex}
            \input{6 - summary/main.tex}
    
    
    %\section{}
    \newpage
    \pagenumbering{gobble}
        \printbibliography


    \newpage
    \pagenumbering{roman}
    \appendix
        \part{Appendices}
            \input{8 - Hilbert complexes/main.tex}
            \input{9 - weak conservation proofs/main.tex}
\end{document}

    
    
    %\section{}
    \newpage
    \pagenumbering{gobble}
        \printbibliography


    \newpage
    \pagenumbering{roman}
    \appendix
        \part{Appendices}
            \documentclass[12pt, a4paper]{report}

\input{template/main.tex}

\title{\BA{Title in Progress...}}
\author{Boris Andrews}
\affil{Mathematical Institute, University of Oxford}
\date{\today}


\begin{document}
    \pagenumbering{gobble}
    \maketitle
    
    
    \begin{abstract}
        Magnetic confinement reactors---in particular tokamaks---offer one of the most promising options for achieving practical nuclear fusion, with the potential to provide virtually limitless, clean energy. The theoretical and numerical modeling of tokamak plasmas is simultaneously an essential component of effective reactor design, and a great research barrier. Tokamak operational conditions exhibit comparatively low Knudsen numbers. Kinetic effects, including kinetic waves and instabilities, Landau damping, bump-on-tail instabilities and more, are therefore highly influential in tokamak plasma dynamics. Purely fluid models are inherently incapable of capturing these effects, whereas the high dimensionality in purely kinetic models render them practically intractable for most relevant purposes.

        We consider a $\delta\!f$ decomposition model, with a macroscopic fluid background and microscopic kinetic correction, both fully coupled to each other. A similar manner of discretization is proposed to that used in the recent \texttt{STRUPHY} code \cite{Holderied_Possanner_Wang_2021, Holderied_2022, Li_et_al_2023} with a finite-element model for the background and a pseudo-particle/PiC model for the correction.

        The fluid background satisfies the full, non-linear, resistive, compressible, Hall MHD equations. \cite{Laakmann_Hu_Farrell_2022} introduces finite-element(-in-space) implicit timesteppers for the incompressible analogue to this system with structure-preserving (SP) properties in the ideal case, alongside parameter-robust preconditioners. We show that these timesteppers can derive from a finite-element-in-time (FET) (and finite-element-in-space) interpretation. The benefits of this reformulation are discussed, including the derivation of timesteppers that are higher order in time, and the quantifiable dissipative SP properties in the non-ideal, resistive case.
        
        We discuss possible options for extending this FET approach to timesteppers for the compressible case.

        The kinetic corrections satisfy linearized Boltzmann equations. Using a Lénard--Bernstein collision operator, these take Fokker--Planck-like forms \cite{Fokker_1914, Planck_1917} wherein pseudo-particles in the numerical model obey the neoclassical transport equations, with particle-independent Brownian drift terms. This offers a rigorous methodology for incorporating collisions into the particle transport model, without coupling the equations of motions for each particle.
        
        Works by Chen, Chacón et al. \cite{Chen_Chacón_Barnes_2011, Chacón_Chen_Barnes_2013, Chen_Chacón_2014, Chen_Chacón_2015} have developed structure-preserving particle pushers for neoclassical transport in the Vlasov equations, derived from Crank--Nicolson integrators. We show these too can can derive from a FET interpretation, similarly offering potential extensions to higher-order-in-time particle pushers. The FET formulation is used also to consider how the stochastic drift terms can be incorporated into the pushers. Stochastic gyrokinetic expansions are also discussed.

        Different options for the numerical implementation of these schemes are considered.

        Due to the efficacy of FET in the development of SP timesteppers for both the fluid and kinetic component, we hope this approach will prove effective in the future for developing SP timesteppers for the full hybrid model. We hope this will give us the opportunity to incorporate previously inaccessible kinetic effects into the highly effective, modern, finite-element MHD models.
    \end{abstract}
    
    
    \newpage
    \tableofcontents
    
    
    \newpage
    \pagenumbering{arabic}
    %\linenumbers\renewcommand\thelinenumber{\color{black!50}\arabic{linenumber}}
            \input{0 - introduction/main.tex}
        \part{Research}
            \input{1 - low-noise PiC models/main.tex}
            \input{2 - kinetic component/main.tex}
            \input{3 - fluid component/main.tex}
            \input{4 - numerical implementation/main.tex}
        \part{Project Overview}
            \input{5 - research plan/main.tex}
            \input{6 - summary/main.tex}
    
    
    %\section{}
    \newpage
    \pagenumbering{gobble}
        \printbibliography


    \newpage
    \pagenumbering{roman}
    \appendix
        \part{Appendices}
            \input{8 - Hilbert complexes/main.tex}
            \input{9 - weak conservation proofs/main.tex}
\end{document}

            \documentclass[12pt, a4paper]{report}

\input{template/main.tex}

\title{\BA{Title in Progress...}}
\author{Boris Andrews}
\affil{Mathematical Institute, University of Oxford}
\date{\today}


\begin{document}
    \pagenumbering{gobble}
    \maketitle
    
    
    \begin{abstract}
        Magnetic confinement reactors---in particular tokamaks---offer one of the most promising options for achieving practical nuclear fusion, with the potential to provide virtually limitless, clean energy. The theoretical and numerical modeling of tokamak plasmas is simultaneously an essential component of effective reactor design, and a great research barrier. Tokamak operational conditions exhibit comparatively low Knudsen numbers. Kinetic effects, including kinetic waves and instabilities, Landau damping, bump-on-tail instabilities and more, are therefore highly influential in tokamak plasma dynamics. Purely fluid models are inherently incapable of capturing these effects, whereas the high dimensionality in purely kinetic models render them practically intractable for most relevant purposes.

        We consider a $\delta\!f$ decomposition model, with a macroscopic fluid background and microscopic kinetic correction, both fully coupled to each other. A similar manner of discretization is proposed to that used in the recent \texttt{STRUPHY} code \cite{Holderied_Possanner_Wang_2021, Holderied_2022, Li_et_al_2023} with a finite-element model for the background and a pseudo-particle/PiC model for the correction.

        The fluid background satisfies the full, non-linear, resistive, compressible, Hall MHD equations. \cite{Laakmann_Hu_Farrell_2022} introduces finite-element(-in-space) implicit timesteppers for the incompressible analogue to this system with structure-preserving (SP) properties in the ideal case, alongside parameter-robust preconditioners. We show that these timesteppers can derive from a finite-element-in-time (FET) (and finite-element-in-space) interpretation. The benefits of this reformulation are discussed, including the derivation of timesteppers that are higher order in time, and the quantifiable dissipative SP properties in the non-ideal, resistive case.
        
        We discuss possible options for extending this FET approach to timesteppers for the compressible case.

        The kinetic corrections satisfy linearized Boltzmann equations. Using a Lénard--Bernstein collision operator, these take Fokker--Planck-like forms \cite{Fokker_1914, Planck_1917} wherein pseudo-particles in the numerical model obey the neoclassical transport equations, with particle-independent Brownian drift terms. This offers a rigorous methodology for incorporating collisions into the particle transport model, without coupling the equations of motions for each particle.
        
        Works by Chen, Chacón et al. \cite{Chen_Chacón_Barnes_2011, Chacón_Chen_Barnes_2013, Chen_Chacón_2014, Chen_Chacón_2015} have developed structure-preserving particle pushers for neoclassical transport in the Vlasov equations, derived from Crank--Nicolson integrators. We show these too can can derive from a FET interpretation, similarly offering potential extensions to higher-order-in-time particle pushers. The FET formulation is used also to consider how the stochastic drift terms can be incorporated into the pushers. Stochastic gyrokinetic expansions are also discussed.

        Different options for the numerical implementation of these schemes are considered.

        Due to the efficacy of FET in the development of SP timesteppers for both the fluid and kinetic component, we hope this approach will prove effective in the future for developing SP timesteppers for the full hybrid model. We hope this will give us the opportunity to incorporate previously inaccessible kinetic effects into the highly effective, modern, finite-element MHD models.
    \end{abstract}
    
    
    \newpage
    \tableofcontents
    
    
    \newpage
    \pagenumbering{arabic}
    %\linenumbers\renewcommand\thelinenumber{\color{black!50}\arabic{linenumber}}
            \input{0 - introduction/main.tex}
        \part{Research}
            \input{1 - low-noise PiC models/main.tex}
            \input{2 - kinetic component/main.tex}
            \input{3 - fluid component/main.tex}
            \input{4 - numerical implementation/main.tex}
        \part{Project Overview}
            \input{5 - research plan/main.tex}
            \input{6 - summary/main.tex}
    
    
    %\section{}
    \newpage
    \pagenumbering{gobble}
        \printbibliography


    \newpage
    \pagenumbering{roman}
    \appendix
        \part{Appendices}
            \input{8 - Hilbert complexes/main.tex}
            \input{9 - weak conservation proofs/main.tex}
\end{document}

\end{document}

        \part{Research}
            \documentclass[12pt, a4paper]{report}

\documentclass[12pt, a4paper]{report}

\input{template/main.tex}

\title{\BA{Title in Progress...}}
\author{Boris Andrews}
\affil{Mathematical Institute, University of Oxford}
\date{\today}


\begin{document}
    \pagenumbering{gobble}
    \maketitle
    
    
    \begin{abstract}
        Magnetic confinement reactors---in particular tokamaks---offer one of the most promising options for achieving practical nuclear fusion, with the potential to provide virtually limitless, clean energy. The theoretical and numerical modeling of tokamak plasmas is simultaneously an essential component of effective reactor design, and a great research barrier. Tokamak operational conditions exhibit comparatively low Knudsen numbers. Kinetic effects, including kinetic waves and instabilities, Landau damping, bump-on-tail instabilities and more, are therefore highly influential in tokamak plasma dynamics. Purely fluid models are inherently incapable of capturing these effects, whereas the high dimensionality in purely kinetic models render them practically intractable for most relevant purposes.

        We consider a $\delta\!f$ decomposition model, with a macroscopic fluid background and microscopic kinetic correction, both fully coupled to each other. A similar manner of discretization is proposed to that used in the recent \texttt{STRUPHY} code \cite{Holderied_Possanner_Wang_2021, Holderied_2022, Li_et_al_2023} with a finite-element model for the background and a pseudo-particle/PiC model for the correction.

        The fluid background satisfies the full, non-linear, resistive, compressible, Hall MHD equations. \cite{Laakmann_Hu_Farrell_2022} introduces finite-element(-in-space) implicit timesteppers for the incompressible analogue to this system with structure-preserving (SP) properties in the ideal case, alongside parameter-robust preconditioners. We show that these timesteppers can derive from a finite-element-in-time (FET) (and finite-element-in-space) interpretation. The benefits of this reformulation are discussed, including the derivation of timesteppers that are higher order in time, and the quantifiable dissipative SP properties in the non-ideal, resistive case.
        
        We discuss possible options for extending this FET approach to timesteppers for the compressible case.

        The kinetic corrections satisfy linearized Boltzmann equations. Using a Lénard--Bernstein collision operator, these take Fokker--Planck-like forms \cite{Fokker_1914, Planck_1917} wherein pseudo-particles in the numerical model obey the neoclassical transport equations, with particle-independent Brownian drift terms. This offers a rigorous methodology for incorporating collisions into the particle transport model, without coupling the equations of motions for each particle.
        
        Works by Chen, Chacón et al. \cite{Chen_Chacón_Barnes_2011, Chacón_Chen_Barnes_2013, Chen_Chacón_2014, Chen_Chacón_2015} have developed structure-preserving particle pushers for neoclassical transport in the Vlasov equations, derived from Crank--Nicolson integrators. We show these too can can derive from a FET interpretation, similarly offering potential extensions to higher-order-in-time particle pushers. The FET formulation is used also to consider how the stochastic drift terms can be incorporated into the pushers. Stochastic gyrokinetic expansions are also discussed.

        Different options for the numerical implementation of these schemes are considered.

        Due to the efficacy of FET in the development of SP timesteppers for both the fluid and kinetic component, we hope this approach will prove effective in the future for developing SP timesteppers for the full hybrid model. We hope this will give us the opportunity to incorporate previously inaccessible kinetic effects into the highly effective, modern, finite-element MHD models.
    \end{abstract}
    
    
    \newpage
    \tableofcontents
    
    
    \newpage
    \pagenumbering{arabic}
    %\linenumbers\renewcommand\thelinenumber{\color{black!50}\arabic{linenumber}}
            \input{0 - introduction/main.tex}
        \part{Research}
            \input{1 - low-noise PiC models/main.tex}
            \input{2 - kinetic component/main.tex}
            \input{3 - fluid component/main.tex}
            \input{4 - numerical implementation/main.tex}
        \part{Project Overview}
            \input{5 - research plan/main.tex}
            \input{6 - summary/main.tex}
    
    
    %\section{}
    \newpage
    \pagenumbering{gobble}
        \printbibliography


    \newpage
    \pagenumbering{roman}
    \appendix
        \part{Appendices}
            \input{8 - Hilbert complexes/main.tex}
            \input{9 - weak conservation proofs/main.tex}
\end{document}


\title{\BA{Title in Progress...}}
\author{Boris Andrews}
\affil{Mathematical Institute, University of Oxford}
\date{\today}


\begin{document}
    \pagenumbering{gobble}
    \maketitle
    
    
    \begin{abstract}
        Magnetic confinement reactors---in particular tokamaks---offer one of the most promising options for achieving practical nuclear fusion, with the potential to provide virtually limitless, clean energy. The theoretical and numerical modeling of tokamak plasmas is simultaneously an essential component of effective reactor design, and a great research barrier. Tokamak operational conditions exhibit comparatively low Knudsen numbers. Kinetic effects, including kinetic waves and instabilities, Landau damping, bump-on-tail instabilities and more, are therefore highly influential in tokamak plasma dynamics. Purely fluid models are inherently incapable of capturing these effects, whereas the high dimensionality in purely kinetic models render them practically intractable for most relevant purposes.

        We consider a $\delta\!f$ decomposition model, with a macroscopic fluid background and microscopic kinetic correction, both fully coupled to each other. A similar manner of discretization is proposed to that used in the recent \texttt{STRUPHY} code \cite{Holderied_Possanner_Wang_2021, Holderied_2022, Li_et_al_2023} with a finite-element model for the background and a pseudo-particle/PiC model for the correction.

        The fluid background satisfies the full, non-linear, resistive, compressible, Hall MHD equations. \cite{Laakmann_Hu_Farrell_2022} introduces finite-element(-in-space) implicit timesteppers for the incompressible analogue to this system with structure-preserving (SP) properties in the ideal case, alongside parameter-robust preconditioners. We show that these timesteppers can derive from a finite-element-in-time (FET) (and finite-element-in-space) interpretation. The benefits of this reformulation are discussed, including the derivation of timesteppers that are higher order in time, and the quantifiable dissipative SP properties in the non-ideal, resistive case.
        
        We discuss possible options for extending this FET approach to timesteppers for the compressible case.

        The kinetic corrections satisfy linearized Boltzmann equations. Using a Lénard--Bernstein collision operator, these take Fokker--Planck-like forms \cite{Fokker_1914, Planck_1917} wherein pseudo-particles in the numerical model obey the neoclassical transport equations, with particle-independent Brownian drift terms. This offers a rigorous methodology for incorporating collisions into the particle transport model, without coupling the equations of motions for each particle.
        
        Works by Chen, Chacón et al. \cite{Chen_Chacón_Barnes_2011, Chacón_Chen_Barnes_2013, Chen_Chacón_2014, Chen_Chacón_2015} have developed structure-preserving particle pushers for neoclassical transport in the Vlasov equations, derived from Crank--Nicolson integrators. We show these too can can derive from a FET interpretation, similarly offering potential extensions to higher-order-in-time particle pushers. The FET formulation is used also to consider how the stochastic drift terms can be incorporated into the pushers. Stochastic gyrokinetic expansions are also discussed.

        Different options for the numerical implementation of these schemes are considered.

        Due to the efficacy of FET in the development of SP timesteppers for both the fluid and kinetic component, we hope this approach will prove effective in the future for developing SP timesteppers for the full hybrid model. We hope this will give us the opportunity to incorporate previously inaccessible kinetic effects into the highly effective, modern, finite-element MHD models.
    \end{abstract}
    
    
    \newpage
    \tableofcontents
    
    
    \newpage
    \pagenumbering{arabic}
    %\linenumbers\renewcommand\thelinenumber{\color{black!50}\arabic{linenumber}}
            \documentclass[12pt, a4paper]{report}

\input{template/main.tex}

\title{\BA{Title in Progress...}}
\author{Boris Andrews}
\affil{Mathematical Institute, University of Oxford}
\date{\today}


\begin{document}
    \pagenumbering{gobble}
    \maketitle
    
    
    \begin{abstract}
        Magnetic confinement reactors---in particular tokamaks---offer one of the most promising options for achieving practical nuclear fusion, with the potential to provide virtually limitless, clean energy. The theoretical and numerical modeling of tokamak plasmas is simultaneously an essential component of effective reactor design, and a great research barrier. Tokamak operational conditions exhibit comparatively low Knudsen numbers. Kinetic effects, including kinetic waves and instabilities, Landau damping, bump-on-tail instabilities and more, are therefore highly influential in tokamak plasma dynamics. Purely fluid models are inherently incapable of capturing these effects, whereas the high dimensionality in purely kinetic models render them practically intractable for most relevant purposes.

        We consider a $\delta\!f$ decomposition model, with a macroscopic fluid background and microscopic kinetic correction, both fully coupled to each other. A similar manner of discretization is proposed to that used in the recent \texttt{STRUPHY} code \cite{Holderied_Possanner_Wang_2021, Holderied_2022, Li_et_al_2023} with a finite-element model for the background and a pseudo-particle/PiC model for the correction.

        The fluid background satisfies the full, non-linear, resistive, compressible, Hall MHD equations. \cite{Laakmann_Hu_Farrell_2022} introduces finite-element(-in-space) implicit timesteppers for the incompressible analogue to this system with structure-preserving (SP) properties in the ideal case, alongside parameter-robust preconditioners. We show that these timesteppers can derive from a finite-element-in-time (FET) (and finite-element-in-space) interpretation. The benefits of this reformulation are discussed, including the derivation of timesteppers that are higher order in time, and the quantifiable dissipative SP properties in the non-ideal, resistive case.
        
        We discuss possible options for extending this FET approach to timesteppers for the compressible case.

        The kinetic corrections satisfy linearized Boltzmann equations. Using a Lénard--Bernstein collision operator, these take Fokker--Planck-like forms \cite{Fokker_1914, Planck_1917} wherein pseudo-particles in the numerical model obey the neoclassical transport equations, with particle-independent Brownian drift terms. This offers a rigorous methodology for incorporating collisions into the particle transport model, without coupling the equations of motions for each particle.
        
        Works by Chen, Chacón et al. \cite{Chen_Chacón_Barnes_2011, Chacón_Chen_Barnes_2013, Chen_Chacón_2014, Chen_Chacón_2015} have developed structure-preserving particle pushers for neoclassical transport in the Vlasov equations, derived from Crank--Nicolson integrators. We show these too can can derive from a FET interpretation, similarly offering potential extensions to higher-order-in-time particle pushers. The FET formulation is used also to consider how the stochastic drift terms can be incorporated into the pushers. Stochastic gyrokinetic expansions are also discussed.

        Different options for the numerical implementation of these schemes are considered.

        Due to the efficacy of FET in the development of SP timesteppers for both the fluid and kinetic component, we hope this approach will prove effective in the future for developing SP timesteppers for the full hybrid model. We hope this will give us the opportunity to incorporate previously inaccessible kinetic effects into the highly effective, modern, finite-element MHD models.
    \end{abstract}
    
    
    \newpage
    \tableofcontents
    
    
    \newpage
    \pagenumbering{arabic}
    %\linenumbers\renewcommand\thelinenumber{\color{black!50}\arabic{linenumber}}
            \input{0 - introduction/main.tex}
        \part{Research}
            \input{1 - low-noise PiC models/main.tex}
            \input{2 - kinetic component/main.tex}
            \input{3 - fluid component/main.tex}
            \input{4 - numerical implementation/main.tex}
        \part{Project Overview}
            \input{5 - research plan/main.tex}
            \input{6 - summary/main.tex}
    
    
    %\section{}
    \newpage
    \pagenumbering{gobble}
        \printbibliography


    \newpage
    \pagenumbering{roman}
    \appendix
        \part{Appendices}
            \input{8 - Hilbert complexes/main.tex}
            \input{9 - weak conservation proofs/main.tex}
\end{document}

        \part{Research}
            \documentclass[12pt, a4paper]{report}

\input{template/main.tex}

\title{\BA{Title in Progress...}}
\author{Boris Andrews}
\affil{Mathematical Institute, University of Oxford}
\date{\today}


\begin{document}
    \pagenumbering{gobble}
    \maketitle
    
    
    \begin{abstract}
        Magnetic confinement reactors---in particular tokamaks---offer one of the most promising options for achieving practical nuclear fusion, with the potential to provide virtually limitless, clean energy. The theoretical and numerical modeling of tokamak plasmas is simultaneously an essential component of effective reactor design, and a great research barrier. Tokamak operational conditions exhibit comparatively low Knudsen numbers. Kinetic effects, including kinetic waves and instabilities, Landau damping, bump-on-tail instabilities and more, are therefore highly influential in tokamak plasma dynamics. Purely fluid models are inherently incapable of capturing these effects, whereas the high dimensionality in purely kinetic models render them practically intractable for most relevant purposes.

        We consider a $\delta\!f$ decomposition model, with a macroscopic fluid background and microscopic kinetic correction, both fully coupled to each other. A similar manner of discretization is proposed to that used in the recent \texttt{STRUPHY} code \cite{Holderied_Possanner_Wang_2021, Holderied_2022, Li_et_al_2023} with a finite-element model for the background and a pseudo-particle/PiC model for the correction.

        The fluid background satisfies the full, non-linear, resistive, compressible, Hall MHD equations. \cite{Laakmann_Hu_Farrell_2022} introduces finite-element(-in-space) implicit timesteppers for the incompressible analogue to this system with structure-preserving (SP) properties in the ideal case, alongside parameter-robust preconditioners. We show that these timesteppers can derive from a finite-element-in-time (FET) (and finite-element-in-space) interpretation. The benefits of this reformulation are discussed, including the derivation of timesteppers that are higher order in time, and the quantifiable dissipative SP properties in the non-ideal, resistive case.
        
        We discuss possible options for extending this FET approach to timesteppers for the compressible case.

        The kinetic corrections satisfy linearized Boltzmann equations. Using a Lénard--Bernstein collision operator, these take Fokker--Planck-like forms \cite{Fokker_1914, Planck_1917} wherein pseudo-particles in the numerical model obey the neoclassical transport equations, with particle-independent Brownian drift terms. This offers a rigorous methodology for incorporating collisions into the particle transport model, without coupling the equations of motions for each particle.
        
        Works by Chen, Chacón et al. \cite{Chen_Chacón_Barnes_2011, Chacón_Chen_Barnes_2013, Chen_Chacón_2014, Chen_Chacón_2015} have developed structure-preserving particle pushers for neoclassical transport in the Vlasov equations, derived from Crank--Nicolson integrators. We show these too can can derive from a FET interpretation, similarly offering potential extensions to higher-order-in-time particle pushers. The FET formulation is used also to consider how the stochastic drift terms can be incorporated into the pushers. Stochastic gyrokinetic expansions are also discussed.

        Different options for the numerical implementation of these schemes are considered.

        Due to the efficacy of FET in the development of SP timesteppers for both the fluid and kinetic component, we hope this approach will prove effective in the future for developing SP timesteppers for the full hybrid model. We hope this will give us the opportunity to incorporate previously inaccessible kinetic effects into the highly effective, modern, finite-element MHD models.
    \end{abstract}
    
    
    \newpage
    \tableofcontents
    
    
    \newpage
    \pagenumbering{arabic}
    %\linenumbers\renewcommand\thelinenumber{\color{black!50}\arabic{linenumber}}
            \input{0 - introduction/main.tex}
        \part{Research}
            \input{1 - low-noise PiC models/main.tex}
            \input{2 - kinetic component/main.tex}
            \input{3 - fluid component/main.tex}
            \input{4 - numerical implementation/main.tex}
        \part{Project Overview}
            \input{5 - research plan/main.tex}
            \input{6 - summary/main.tex}
    
    
    %\section{}
    \newpage
    \pagenumbering{gobble}
        \printbibliography


    \newpage
    \pagenumbering{roman}
    \appendix
        \part{Appendices}
            \input{8 - Hilbert complexes/main.tex}
            \input{9 - weak conservation proofs/main.tex}
\end{document}

            \documentclass[12pt, a4paper]{report}

\input{template/main.tex}

\title{\BA{Title in Progress...}}
\author{Boris Andrews}
\affil{Mathematical Institute, University of Oxford}
\date{\today}


\begin{document}
    \pagenumbering{gobble}
    \maketitle
    
    
    \begin{abstract}
        Magnetic confinement reactors---in particular tokamaks---offer one of the most promising options for achieving practical nuclear fusion, with the potential to provide virtually limitless, clean energy. The theoretical and numerical modeling of tokamak plasmas is simultaneously an essential component of effective reactor design, and a great research barrier. Tokamak operational conditions exhibit comparatively low Knudsen numbers. Kinetic effects, including kinetic waves and instabilities, Landau damping, bump-on-tail instabilities and more, are therefore highly influential in tokamak plasma dynamics. Purely fluid models are inherently incapable of capturing these effects, whereas the high dimensionality in purely kinetic models render them practically intractable for most relevant purposes.

        We consider a $\delta\!f$ decomposition model, with a macroscopic fluid background and microscopic kinetic correction, both fully coupled to each other. A similar manner of discretization is proposed to that used in the recent \texttt{STRUPHY} code \cite{Holderied_Possanner_Wang_2021, Holderied_2022, Li_et_al_2023} with a finite-element model for the background and a pseudo-particle/PiC model for the correction.

        The fluid background satisfies the full, non-linear, resistive, compressible, Hall MHD equations. \cite{Laakmann_Hu_Farrell_2022} introduces finite-element(-in-space) implicit timesteppers for the incompressible analogue to this system with structure-preserving (SP) properties in the ideal case, alongside parameter-robust preconditioners. We show that these timesteppers can derive from a finite-element-in-time (FET) (and finite-element-in-space) interpretation. The benefits of this reformulation are discussed, including the derivation of timesteppers that are higher order in time, and the quantifiable dissipative SP properties in the non-ideal, resistive case.
        
        We discuss possible options for extending this FET approach to timesteppers for the compressible case.

        The kinetic corrections satisfy linearized Boltzmann equations. Using a Lénard--Bernstein collision operator, these take Fokker--Planck-like forms \cite{Fokker_1914, Planck_1917} wherein pseudo-particles in the numerical model obey the neoclassical transport equations, with particle-independent Brownian drift terms. This offers a rigorous methodology for incorporating collisions into the particle transport model, without coupling the equations of motions for each particle.
        
        Works by Chen, Chacón et al. \cite{Chen_Chacón_Barnes_2011, Chacón_Chen_Barnes_2013, Chen_Chacón_2014, Chen_Chacón_2015} have developed structure-preserving particle pushers for neoclassical transport in the Vlasov equations, derived from Crank--Nicolson integrators. We show these too can can derive from a FET interpretation, similarly offering potential extensions to higher-order-in-time particle pushers. The FET formulation is used also to consider how the stochastic drift terms can be incorporated into the pushers. Stochastic gyrokinetic expansions are also discussed.

        Different options for the numerical implementation of these schemes are considered.

        Due to the efficacy of FET in the development of SP timesteppers for both the fluid and kinetic component, we hope this approach will prove effective in the future for developing SP timesteppers for the full hybrid model. We hope this will give us the opportunity to incorporate previously inaccessible kinetic effects into the highly effective, modern, finite-element MHD models.
    \end{abstract}
    
    
    \newpage
    \tableofcontents
    
    
    \newpage
    \pagenumbering{arabic}
    %\linenumbers\renewcommand\thelinenumber{\color{black!50}\arabic{linenumber}}
            \input{0 - introduction/main.tex}
        \part{Research}
            \input{1 - low-noise PiC models/main.tex}
            \input{2 - kinetic component/main.tex}
            \input{3 - fluid component/main.tex}
            \input{4 - numerical implementation/main.tex}
        \part{Project Overview}
            \input{5 - research plan/main.tex}
            \input{6 - summary/main.tex}
    
    
    %\section{}
    \newpage
    \pagenumbering{gobble}
        \printbibliography


    \newpage
    \pagenumbering{roman}
    \appendix
        \part{Appendices}
            \input{8 - Hilbert complexes/main.tex}
            \input{9 - weak conservation proofs/main.tex}
\end{document}

            \documentclass[12pt, a4paper]{report}

\input{template/main.tex}

\title{\BA{Title in Progress...}}
\author{Boris Andrews}
\affil{Mathematical Institute, University of Oxford}
\date{\today}


\begin{document}
    \pagenumbering{gobble}
    \maketitle
    
    
    \begin{abstract}
        Magnetic confinement reactors---in particular tokamaks---offer one of the most promising options for achieving practical nuclear fusion, with the potential to provide virtually limitless, clean energy. The theoretical and numerical modeling of tokamak plasmas is simultaneously an essential component of effective reactor design, and a great research barrier. Tokamak operational conditions exhibit comparatively low Knudsen numbers. Kinetic effects, including kinetic waves and instabilities, Landau damping, bump-on-tail instabilities and more, are therefore highly influential in tokamak plasma dynamics. Purely fluid models are inherently incapable of capturing these effects, whereas the high dimensionality in purely kinetic models render them practically intractable for most relevant purposes.

        We consider a $\delta\!f$ decomposition model, with a macroscopic fluid background and microscopic kinetic correction, both fully coupled to each other. A similar manner of discretization is proposed to that used in the recent \texttt{STRUPHY} code \cite{Holderied_Possanner_Wang_2021, Holderied_2022, Li_et_al_2023} with a finite-element model for the background and a pseudo-particle/PiC model for the correction.

        The fluid background satisfies the full, non-linear, resistive, compressible, Hall MHD equations. \cite{Laakmann_Hu_Farrell_2022} introduces finite-element(-in-space) implicit timesteppers for the incompressible analogue to this system with structure-preserving (SP) properties in the ideal case, alongside parameter-robust preconditioners. We show that these timesteppers can derive from a finite-element-in-time (FET) (and finite-element-in-space) interpretation. The benefits of this reformulation are discussed, including the derivation of timesteppers that are higher order in time, and the quantifiable dissipative SP properties in the non-ideal, resistive case.
        
        We discuss possible options for extending this FET approach to timesteppers for the compressible case.

        The kinetic corrections satisfy linearized Boltzmann equations. Using a Lénard--Bernstein collision operator, these take Fokker--Planck-like forms \cite{Fokker_1914, Planck_1917} wherein pseudo-particles in the numerical model obey the neoclassical transport equations, with particle-independent Brownian drift terms. This offers a rigorous methodology for incorporating collisions into the particle transport model, without coupling the equations of motions for each particle.
        
        Works by Chen, Chacón et al. \cite{Chen_Chacón_Barnes_2011, Chacón_Chen_Barnes_2013, Chen_Chacón_2014, Chen_Chacón_2015} have developed structure-preserving particle pushers for neoclassical transport in the Vlasov equations, derived from Crank--Nicolson integrators. We show these too can can derive from a FET interpretation, similarly offering potential extensions to higher-order-in-time particle pushers. The FET formulation is used also to consider how the stochastic drift terms can be incorporated into the pushers. Stochastic gyrokinetic expansions are also discussed.

        Different options for the numerical implementation of these schemes are considered.

        Due to the efficacy of FET in the development of SP timesteppers for both the fluid and kinetic component, we hope this approach will prove effective in the future for developing SP timesteppers for the full hybrid model. We hope this will give us the opportunity to incorporate previously inaccessible kinetic effects into the highly effective, modern, finite-element MHD models.
    \end{abstract}
    
    
    \newpage
    \tableofcontents
    
    
    \newpage
    \pagenumbering{arabic}
    %\linenumbers\renewcommand\thelinenumber{\color{black!50}\arabic{linenumber}}
            \input{0 - introduction/main.tex}
        \part{Research}
            \input{1 - low-noise PiC models/main.tex}
            \input{2 - kinetic component/main.tex}
            \input{3 - fluid component/main.tex}
            \input{4 - numerical implementation/main.tex}
        \part{Project Overview}
            \input{5 - research plan/main.tex}
            \input{6 - summary/main.tex}
    
    
    %\section{}
    \newpage
    \pagenumbering{gobble}
        \printbibliography


    \newpage
    \pagenumbering{roman}
    \appendix
        \part{Appendices}
            \input{8 - Hilbert complexes/main.tex}
            \input{9 - weak conservation proofs/main.tex}
\end{document}

            \documentclass[12pt, a4paper]{report}

\input{template/main.tex}

\title{\BA{Title in Progress...}}
\author{Boris Andrews}
\affil{Mathematical Institute, University of Oxford}
\date{\today}


\begin{document}
    \pagenumbering{gobble}
    \maketitle
    
    
    \begin{abstract}
        Magnetic confinement reactors---in particular tokamaks---offer one of the most promising options for achieving practical nuclear fusion, with the potential to provide virtually limitless, clean energy. The theoretical and numerical modeling of tokamak plasmas is simultaneously an essential component of effective reactor design, and a great research barrier. Tokamak operational conditions exhibit comparatively low Knudsen numbers. Kinetic effects, including kinetic waves and instabilities, Landau damping, bump-on-tail instabilities and more, are therefore highly influential in tokamak plasma dynamics. Purely fluid models are inherently incapable of capturing these effects, whereas the high dimensionality in purely kinetic models render them practically intractable for most relevant purposes.

        We consider a $\delta\!f$ decomposition model, with a macroscopic fluid background and microscopic kinetic correction, both fully coupled to each other. A similar manner of discretization is proposed to that used in the recent \texttt{STRUPHY} code \cite{Holderied_Possanner_Wang_2021, Holderied_2022, Li_et_al_2023} with a finite-element model for the background and a pseudo-particle/PiC model for the correction.

        The fluid background satisfies the full, non-linear, resistive, compressible, Hall MHD equations. \cite{Laakmann_Hu_Farrell_2022} introduces finite-element(-in-space) implicit timesteppers for the incompressible analogue to this system with structure-preserving (SP) properties in the ideal case, alongside parameter-robust preconditioners. We show that these timesteppers can derive from a finite-element-in-time (FET) (and finite-element-in-space) interpretation. The benefits of this reformulation are discussed, including the derivation of timesteppers that are higher order in time, and the quantifiable dissipative SP properties in the non-ideal, resistive case.
        
        We discuss possible options for extending this FET approach to timesteppers for the compressible case.

        The kinetic corrections satisfy linearized Boltzmann equations. Using a Lénard--Bernstein collision operator, these take Fokker--Planck-like forms \cite{Fokker_1914, Planck_1917} wherein pseudo-particles in the numerical model obey the neoclassical transport equations, with particle-independent Brownian drift terms. This offers a rigorous methodology for incorporating collisions into the particle transport model, without coupling the equations of motions for each particle.
        
        Works by Chen, Chacón et al. \cite{Chen_Chacón_Barnes_2011, Chacón_Chen_Barnes_2013, Chen_Chacón_2014, Chen_Chacón_2015} have developed structure-preserving particle pushers for neoclassical transport in the Vlasov equations, derived from Crank--Nicolson integrators. We show these too can can derive from a FET interpretation, similarly offering potential extensions to higher-order-in-time particle pushers. The FET formulation is used also to consider how the stochastic drift terms can be incorporated into the pushers. Stochastic gyrokinetic expansions are also discussed.

        Different options for the numerical implementation of these schemes are considered.

        Due to the efficacy of FET in the development of SP timesteppers for both the fluid and kinetic component, we hope this approach will prove effective in the future for developing SP timesteppers for the full hybrid model. We hope this will give us the opportunity to incorporate previously inaccessible kinetic effects into the highly effective, modern, finite-element MHD models.
    \end{abstract}
    
    
    \newpage
    \tableofcontents
    
    
    \newpage
    \pagenumbering{arabic}
    %\linenumbers\renewcommand\thelinenumber{\color{black!50}\arabic{linenumber}}
            \input{0 - introduction/main.tex}
        \part{Research}
            \input{1 - low-noise PiC models/main.tex}
            \input{2 - kinetic component/main.tex}
            \input{3 - fluid component/main.tex}
            \input{4 - numerical implementation/main.tex}
        \part{Project Overview}
            \input{5 - research plan/main.tex}
            \input{6 - summary/main.tex}
    
    
    %\section{}
    \newpage
    \pagenumbering{gobble}
        \printbibliography


    \newpage
    \pagenumbering{roman}
    \appendix
        \part{Appendices}
            \input{8 - Hilbert complexes/main.tex}
            \input{9 - weak conservation proofs/main.tex}
\end{document}

        \part{Project Overview}
            \documentclass[12pt, a4paper]{report}

\input{template/main.tex}

\title{\BA{Title in Progress...}}
\author{Boris Andrews}
\affil{Mathematical Institute, University of Oxford}
\date{\today}


\begin{document}
    \pagenumbering{gobble}
    \maketitle
    
    
    \begin{abstract}
        Magnetic confinement reactors---in particular tokamaks---offer one of the most promising options for achieving practical nuclear fusion, with the potential to provide virtually limitless, clean energy. The theoretical and numerical modeling of tokamak plasmas is simultaneously an essential component of effective reactor design, and a great research barrier. Tokamak operational conditions exhibit comparatively low Knudsen numbers. Kinetic effects, including kinetic waves and instabilities, Landau damping, bump-on-tail instabilities and more, are therefore highly influential in tokamak plasma dynamics. Purely fluid models are inherently incapable of capturing these effects, whereas the high dimensionality in purely kinetic models render them practically intractable for most relevant purposes.

        We consider a $\delta\!f$ decomposition model, with a macroscopic fluid background and microscopic kinetic correction, both fully coupled to each other. A similar manner of discretization is proposed to that used in the recent \texttt{STRUPHY} code \cite{Holderied_Possanner_Wang_2021, Holderied_2022, Li_et_al_2023} with a finite-element model for the background and a pseudo-particle/PiC model for the correction.

        The fluid background satisfies the full, non-linear, resistive, compressible, Hall MHD equations. \cite{Laakmann_Hu_Farrell_2022} introduces finite-element(-in-space) implicit timesteppers for the incompressible analogue to this system with structure-preserving (SP) properties in the ideal case, alongside parameter-robust preconditioners. We show that these timesteppers can derive from a finite-element-in-time (FET) (and finite-element-in-space) interpretation. The benefits of this reformulation are discussed, including the derivation of timesteppers that are higher order in time, and the quantifiable dissipative SP properties in the non-ideal, resistive case.
        
        We discuss possible options for extending this FET approach to timesteppers for the compressible case.

        The kinetic corrections satisfy linearized Boltzmann equations. Using a Lénard--Bernstein collision operator, these take Fokker--Planck-like forms \cite{Fokker_1914, Planck_1917} wherein pseudo-particles in the numerical model obey the neoclassical transport equations, with particle-independent Brownian drift terms. This offers a rigorous methodology for incorporating collisions into the particle transport model, without coupling the equations of motions for each particle.
        
        Works by Chen, Chacón et al. \cite{Chen_Chacón_Barnes_2011, Chacón_Chen_Barnes_2013, Chen_Chacón_2014, Chen_Chacón_2015} have developed structure-preserving particle pushers for neoclassical transport in the Vlasov equations, derived from Crank--Nicolson integrators. We show these too can can derive from a FET interpretation, similarly offering potential extensions to higher-order-in-time particle pushers. The FET formulation is used also to consider how the stochastic drift terms can be incorporated into the pushers. Stochastic gyrokinetic expansions are also discussed.

        Different options for the numerical implementation of these schemes are considered.

        Due to the efficacy of FET in the development of SP timesteppers for both the fluid and kinetic component, we hope this approach will prove effective in the future for developing SP timesteppers for the full hybrid model. We hope this will give us the opportunity to incorporate previously inaccessible kinetic effects into the highly effective, modern, finite-element MHD models.
    \end{abstract}
    
    
    \newpage
    \tableofcontents
    
    
    \newpage
    \pagenumbering{arabic}
    %\linenumbers\renewcommand\thelinenumber{\color{black!50}\arabic{linenumber}}
            \input{0 - introduction/main.tex}
        \part{Research}
            \input{1 - low-noise PiC models/main.tex}
            \input{2 - kinetic component/main.tex}
            \input{3 - fluid component/main.tex}
            \input{4 - numerical implementation/main.tex}
        \part{Project Overview}
            \input{5 - research plan/main.tex}
            \input{6 - summary/main.tex}
    
    
    %\section{}
    \newpage
    \pagenumbering{gobble}
        \printbibliography


    \newpage
    \pagenumbering{roman}
    \appendix
        \part{Appendices}
            \input{8 - Hilbert complexes/main.tex}
            \input{9 - weak conservation proofs/main.tex}
\end{document}

            \documentclass[12pt, a4paper]{report}

\input{template/main.tex}

\title{\BA{Title in Progress...}}
\author{Boris Andrews}
\affil{Mathematical Institute, University of Oxford}
\date{\today}


\begin{document}
    \pagenumbering{gobble}
    \maketitle
    
    
    \begin{abstract}
        Magnetic confinement reactors---in particular tokamaks---offer one of the most promising options for achieving practical nuclear fusion, with the potential to provide virtually limitless, clean energy. The theoretical and numerical modeling of tokamak plasmas is simultaneously an essential component of effective reactor design, and a great research barrier. Tokamak operational conditions exhibit comparatively low Knudsen numbers. Kinetic effects, including kinetic waves and instabilities, Landau damping, bump-on-tail instabilities and more, are therefore highly influential in tokamak plasma dynamics. Purely fluid models are inherently incapable of capturing these effects, whereas the high dimensionality in purely kinetic models render them practically intractable for most relevant purposes.

        We consider a $\delta\!f$ decomposition model, with a macroscopic fluid background and microscopic kinetic correction, both fully coupled to each other. A similar manner of discretization is proposed to that used in the recent \texttt{STRUPHY} code \cite{Holderied_Possanner_Wang_2021, Holderied_2022, Li_et_al_2023} with a finite-element model for the background and a pseudo-particle/PiC model for the correction.

        The fluid background satisfies the full, non-linear, resistive, compressible, Hall MHD equations. \cite{Laakmann_Hu_Farrell_2022} introduces finite-element(-in-space) implicit timesteppers for the incompressible analogue to this system with structure-preserving (SP) properties in the ideal case, alongside parameter-robust preconditioners. We show that these timesteppers can derive from a finite-element-in-time (FET) (and finite-element-in-space) interpretation. The benefits of this reformulation are discussed, including the derivation of timesteppers that are higher order in time, and the quantifiable dissipative SP properties in the non-ideal, resistive case.
        
        We discuss possible options for extending this FET approach to timesteppers for the compressible case.

        The kinetic corrections satisfy linearized Boltzmann equations. Using a Lénard--Bernstein collision operator, these take Fokker--Planck-like forms \cite{Fokker_1914, Planck_1917} wherein pseudo-particles in the numerical model obey the neoclassical transport equations, with particle-independent Brownian drift terms. This offers a rigorous methodology for incorporating collisions into the particle transport model, without coupling the equations of motions for each particle.
        
        Works by Chen, Chacón et al. \cite{Chen_Chacón_Barnes_2011, Chacón_Chen_Barnes_2013, Chen_Chacón_2014, Chen_Chacón_2015} have developed structure-preserving particle pushers for neoclassical transport in the Vlasov equations, derived from Crank--Nicolson integrators. We show these too can can derive from a FET interpretation, similarly offering potential extensions to higher-order-in-time particle pushers. The FET formulation is used also to consider how the stochastic drift terms can be incorporated into the pushers. Stochastic gyrokinetic expansions are also discussed.

        Different options for the numerical implementation of these schemes are considered.

        Due to the efficacy of FET in the development of SP timesteppers for both the fluid and kinetic component, we hope this approach will prove effective in the future for developing SP timesteppers for the full hybrid model. We hope this will give us the opportunity to incorporate previously inaccessible kinetic effects into the highly effective, modern, finite-element MHD models.
    \end{abstract}
    
    
    \newpage
    \tableofcontents
    
    
    \newpage
    \pagenumbering{arabic}
    %\linenumbers\renewcommand\thelinenumber{\color{black!50}\arabic{linenumber}}
            \input{0 - introduction/main.tex}
        \part{Research}
            \input{1 - low-noise PiC models/main.tex}
            \input{2 - kinetic component/main.tex}
            \input{3 - fluid component/main.tex}
            \input{4 - numerical implementation/main.tex}
        \part{Project Overview}
            \input{5 - research plan/main.tex}
            \input{6 - summary/main.tex}
    
    
    %\section{}
    \newpage
    \pagenumbering{gobble}
        \printbibliography


    \newpage
    \pagenumbering{roman}
    \appendix
        \part{Appendices}
            \input{8 - Hilbert complexes/main.tex}
            \input{9 - weak conservation proofs/main.tex}
\end{document}

    
    
    %\section{}
    \newpage
    \pagenumbering{gobble}
        \printbibliography


    \newpage
    \pagenumbering{roman}
    \appendix
        \part{Appendices}
            \documentclass[12pt, a4paper]{report}

\input{template/main.tex}

\title{\BA{Title in Progress...}}
\author{Boris Andrews}
\affil{Mathematical Institute, University of Oxford}
\date{\today}


\begin{document}
    \pagenumbering{gobble}
    \maketitle
    
    
    \begin{abstract}
        Magnetic confinement reactors---in particular tokamaks---offer one of the most promising options for achieving practical nuclear fusion, with the potential to provide virtually limitless, clean energy. The theoretical and numerical modeling of tokamak plasmas is simultaneously an essential component of effective reactor design, and a great research barrier. Tokamak operational conditions exhibit comparatively low Knudsen numbers. Kinetic effects, including kinetic waves and instabilities, Landau damping, bump-on-tail instabilities and more, are therefore highly influential in tokamak plasma dynamics. Purely fluid models are inherently incapable of capturing these effects, whereas the high dimensionality in purely kinetic models render them practically intractable for most relevant purposes.

        We consider a $\delta\!f$ decomposition model, with a macroscopic fluid background and microscopic kinetic correction, both fully coupled to each other. A similar manner of discretization is proposed to that used in the recent \texttt{STRUPHY} code \cite{Holderied_Possanner_Wang_2021, Holderied_2022, Li_et_al_2023} with a finite-element model for the background and a pseudo-particle/PiC model for the correction.

        The fluid background satisfies the full, non-linear, resistive, compressible, Hall MHD equations. \cite{Laakmann_Hu_Farrell_2022} introduces finite-element(-in-space) implicit timesteppers for the incompressible analogue to this system with structure-preserving (SP) properties in the ideal case, alongside parameter-robust preconditioners. We show that these timesteppers can derive from a finite-element-in-time (FET) (and finite-element-in-space) interpretation. The benefits of this reformulation are discussed, including the derivation of timesteppers that are higher order in time, and the quantifiable dissipative SP properties in the non-ideal, resistive case.
        
        We discuss possible options for extending this FET approach to timesteppers for the compressible case.

        The kinetic corrections satisfy linearized Boltzmann equations. Using a Lénard--Bernstein collision operator, these take Fokker--Planck-like forms \cite{Fokker_1914, Planck_1917} wherein pseudo-particles in the numerical model obey the neoclassical transport equations, with particle-independent Brownian drift terms. This offers a rigorous methodology for incorporating collisions into the particle transport model, without coupling the equations of motions for each particle.
        
        Works by Chen, Chacón et al. \cite{Chen_Chacón_Barnes_2011, Chacón_Chen_Barnes_2013, Chen_Chacón_2014, Chen_Chacón_2015} have developed structure-preserving particle pushers for neoclassical transport in the Vlasov equations, derived from Crank--Nicolson integrators. We show these too can can derive from a FET interpretation, similarly offering potential extensions to higher-order-in-time particle pushers. The FET formulation is used also to consider how the stochastic drift terms can be incorporated into the pushers. Stochastic gyrokinetic expansions are also discussed.

        Different options for the numerical implementation of these schemes are considered.

        Due to the efficacy of FET in the development of SP timesteppers for both the fluid and kinetic component, we hope this approach will prove effective in the future for developing SP timesteppers for the full hybrid model. We hope this will give us the opportunity to incorporate previously inaccessible kinetic effects into the highly effective, modern, finite-element MHD models.
    \end{abstract}
    
    
    \newpage
    \tableofcontents
    
    
    \newpage
    \pagenumbering{arabic}
    %\linenumbers\renewcommand\thelinenumber{\color{black!50}\arabic{linenumber}}
            \input{0 - introduction/main.tex}
        \part{Research}
            \input{1 - low-noise PiC models/main.tex}
            \input{2 - kinetic component/main.tex}
            \input{3 - fluid component/main.tex}
            \input{4 - numerical implementation/main.tex}
        \part{Project Overview}
            \input{5 - research plan/main.tex}
            \input{6 - summary/main.tex}
    
    
    %\section{}
    \newpage
    \pagenumbering{gobble}
        \printbibliography


    \newpage
    \pagenumbering{roman}
    \appendix
        \part{Appendices}
            \input{8 - Hilbert complexes/main.tex}
            \input{9 - weak conservation proofs/main.tex}
\end{document}

            \documentclass[12pt, a4paper]{report}

\input{template/main.tex}

\title{\BA{Title in Progress...}}
\author{Boris Andrews}
\affil{Mathematical Institute, University of Oxford}
\date{\today}


\begin{document}
    \pagenumbering{gobble}
    \maketitle
    
    
    \begin{abstract}
        Magnetic confinement reactors---in particular tokamaks---offer one of the most promising options for achieving practical nuclear fusion, with the potential to provide virtually limitless, clean energy. The theoretical and numerical modeling of tokamak plasmas is simultaneously an essential component of effective reactor design, and a great research barrier. Tokamak operational conditions exhibit comparatively low Knudsen numbers. Kinetic effects, including kinetic waves and instabilities, Landau damping, bump-on-tail instabilities and more, are therefore highly influential in tokamak plasma dynamics. Purely fluid models are inherently incapable of capturing these effects, whereas the high dimensionality in purely kinetic models render them practically intractable for most relevant purposes.

        We consider a $\delta\!f$ decomposition model, with a macroscopic fluid background and microscopic kinetic correction, both fully coupled to each other. A similar manner of discretization is proposed to that used in the recent \texttt{STRUPHY} code \cite{Holderied_Possanner_Wang_2021, Holderied_2022, Li_et_al_2023} with a finite-element model for the background and a pseudo-particle/PiC model for the correction.

        The fluid background satisfies the full, non-linear, resistive, compressible, Hall MHD equations. \cite{Laakmann_Hu_Farrell_2022} introduces finite-element(-in-space) implicit timesteppers for the incompressible analogue to this system with structure-preserving (SP) properties in the ideal case, alongside parameter-robust preconditioners. We show that these timesteppers can derive from a finite-element-in-time (FET) (and finite-element-in-space) interpretation. The benefits of this reformulation are discussed, including the derivation of timesteppers that are higher order in time, and the quantifiable dissipative SP properties in the non-ideal, resistive case.
        
        We discuss possible options for extending this FET approach to timesteppers for the compressible case.

        The kinetic corrections satisfy linearized Boltzmann equations. Using a Lénard--Bernstein collision operator, these take Fokker--Planck-like forms \cite{Fokker_1914, Planck_1917} wherein pseudo-particles in the numerical model obey the neoclassical transport equations, with particle-independent Brownian drift terms. This offers a rigorous methodology for incorporating collisions into the particle transport model, without coupling the equations of motions for each particle.
        
        Works by Chen, Chacón et al. \cite{Chen_Chacón_Barnes_2011, Chacón_Chen_Barnes_2013, Chen_Chacón_2014, Chen_Chacón_2015} have developed structure-preserving particle pushers for neoclassical transport in the Vlasov equations, derived from Crank--Nicolson integrators. We show these too can can derive from a FET interpretation, similarly offering potential extensions to higher-order-in-time particle pushers. The FET formulation is used also to consider how the stochastic drift terms can be incorporated into the pushers. Stochastic gyrokinetic expansions are also discussed.

        Different options for the numerical implementation of these schemes are considered.

        Due to the efficacy of FET in the development of SP timesteppers for both the fluid and kinetic component, we hope this approach will prove effective in the future for developing SP timesteppers for the full hybrid model. We hope this will give us the opportunity to incorporate previously inaccessible kinetic effects into the highly effective, modern, finite-element MHD models.
    \end{abstract}
    
    
    \newpage
    \tableofcontents
    
    
    \newpage
    \pagenumbering{arabic}
    %\linenumbers\renewcommand\thelinenumber{\color{black!50}\arabic{linenumber}}
            \input{0 - introduction/main.tex}
        \part{Research}
            \input{1 - low-noise PiC models/main.tex}
            \input{2 - kinetic component/main.tex}
            \input{3 - fluid component/main.tex}
            \input{4 - numerical implementation/main.tex}
        \part{Project Overview}
            \input{5 - research plan/main.tex}
            \input{6 - summary/main.tex}
    
    
    %\section{}
    \newpage
    \pagenumbering{gobble}
        \printbibliography


    \newpage
    \pagenumbering{roman}
    \appendix
        \part{Appendices}
            \input{8 - Hilbert complexes/main.tex}
            \input{9 - weak conservation proofs/main.tex}
\end{document}

\end{document}

            \documentclass[12pt, a4paper]{report}

\documentclass[12pt, a4paper]{report}

\input{template/main.tex}

\title{\BA{Title in Progress...}}
\author{Boris Andrews}
\affil{Mathematical Institute, University of Oxford}
\date{\today}


\begin{document}
    \pagenumbering{gobble}
    \maketitle
    
    
    \begin{abstract}
        Magnetic confinement reactors---in particular tokamaks---offer one of the most promising options for achieving practical nuclear fusion, with the potential to provide virtually limitless, clean energy. The theoretical and numerical modeling of tokamak plasmas is simultaneously an essential component of effective reactor design, and a great research barrier. Tokamak operational conditions exhibit comparatively low Knudsen numbers. Kinetic effects, including kinetic waves and instabilities, Landau damping, bump-on-tail instabilities and more, are therefore highly influential in tokamak plasma dynamics. Purely fluid models are inherently incapable of capturing these effects, whereas the high dimensionality in purely kinetic models render them practically intractable for most relevant purposes.

        We consider a $\delta\!f$ decomposition model, with a macroscopic fluid background and microscopic kinetic correction, both fully coupled to each other. A similar manner of discretization is proposed to that used in the recent \texttt{STRUPHY} code \cite{Holderied_Possanner_Wang_2021, Holderied_2022, Li_et_al_2023} with a finite-element model for the background and a pseudo-particle/PiC model for the correction.

        The fluid background satisfies the full, non-linear, resistive, compressible, Hall MHD equations. \cite{Laakmann_Hu_Farrell_2022} introduces finite-element(-in-space) implicit timesteppers for the incompressible analogue to this system with structure-preserving (SP) properties in the ideal case, alongside parameter-robust preconditioners. We show that these timesteppers can derive from a finite-element-in-time (FET) (and finite-element-in-space) interpretation. The benefits of this reformulation are discussed, including the derivation of timesteppers that are higher order in time, and the quantifiable dissipative SP properties in the non-ideal, resistive case.
        
        We discuss possible options for extending this FET approach to timesteppers for the compressible case.

        The kinetic corrections satisfy linearized Boltzmann equations. Using a Lénard--Bernstein collision operator, these take Fokker--Planck-like forms \cite{Fokker_1914, Planck_1917} wherein pseudo-particles in the numerical model obey the neoclassical transport equations, with particle-independent Brownian drift terms. This offers a rigorous methodology for incorporating collisions into the particle transport model, without coupling the equations of motions for each particle.
        
        Works by Chen, Chacón et al. \cite{Chen_Chacón_Barnes_2011, Chacón_Chen_Barnes_2013, Chen_Chacón_2014, Chen_Chacón_2015} have developed structure-preserving particle pushers for neoclassical transport in the Vlasov equations, derived from Crank--Nicolson integrators. We show these too can can derive from a FET interpretation, similarly offering potential extensions to higher-order-in-time particle pushers. The FET formulation is used also to consider how the stochastic drift terms can be incorporated into the pushers. Stochastic gyrokinetic expansions are also discussed.

        Different options for the numerical implementation of these schemes are considered.

        Due to the efficacy of FET in the development of SP timesteppers for both the fluid and kinetic component, we hope this approach will prove effective in the future for developing SP timesteppers for the full hybrid model. We hope this will give us the opportunity to incorporate previously inaccessible kinetic effects into the highly effective, modern, finite-element MHD models.
    \end{abstract}
    
    
    \newpage
    \tableofcontents
    
    
    \newpage
    \pagenumbering{arabic}
    %\linenumbers\renewcommand\thelinenumber{\color{black!50}\arabic{linenumber}}
            \input{0 - introduction/main.tex}
        \part{Research}
            \input{1 - low-noise PiC models/main.tex}
            \input{2 - kinetic component/main.tex}
            \input{3 - fluid component/main.tex}
            \input{4 - numerical implementation/main.tex}
        \part{Project Overview}
            \input{5 - research plan/main.tex}
            \input{6 - summary/main.tex}
    
    
    %\section{}
    \newpage
    \pagenumbering{gobble}
        \printbibliography


    \newpage
    \pagenumbering{roman}
    \appendix
        \part{Appendices}
            \input{8 - Hilbert complexes/main.tex}
            \input{9 - weak conservation proofs/main.tex}
\end{document}


\title{\BA{Title in Progress...}}
\author{Boris Andrews}
\affil{Mathematical Institute, University of Oxford}
\date{\today}


\begin{document}
    \pagenumbering{gobble}
    \maketitle
    
    
    \begin{abstract}
        Magnetic confinement reactors---in particular tokamaks---offer one of the most promising options for achieving practical nuclear fusion, with the potential to provide virtually limitless, clean energy. The theoretical and numerical modeling of tokamak plasmas is simultaneously an essential component of effective reactor design, and a great research barrier. Tokamak operational conditions exhibit comparatively low Knudsen numbers. Kinetic effects, including kinetic waves and instabilities, Landau damping, bump-on-tail instabilities and more, are therefore highly influential in tokamak plasma dynamics. Purely fluid models are inherently incapable of capturing these effects, whereas the high dimensionality in purely kinetic models render them practically intractable for most relevant purposes.

        We consider a $\delta\!f$ decomposition model, with a macroscopic fluid background and microscopic kinetic correction, both fully coupled to each other. A similar manner of discretization is proposed to that used in the recent \texttt{STRUPHY} code \cite{Holderied_Possanner_Wang_2021, Holderied_2022, Li_et_al_2023} with a finite-element model for the background and a pseudo-particle/PiC model for the correction.

        The fluid background satisfies the full, non-linear, resistive, compressible, Hall MHD equations. \cite{Laakmann_Hu_Farrell_2022} introduces finite-element(-in-space) implicit timesteppers for the incompressible analogue to this system with structure-preserving (SP) properties in the ideal case, alongside parameter-robust preconditioners. We show that these timesteppers can derive from a finite-element-in-time (FET) (and finite-element-in-space) interpretation. The benefits of this reformulation are discussed, including the derivation of timesteppers that are higher order in time, and the quantifiable dissipative SP properties in the non-ideal, resistive case.
        
        We discuss possible options for extending this FET approach to timesteppers for the compressible case.

        The kinetic corrections satisfy linearized Boltzmann equations. Using a Lénard--Bernstein collision operator, these take Fokker--Planck-like forms \cite{Fokker_1914, Planck_1917} wherein pseudo-particles in the numerical model obey the neoclassical transport equations, with particle-independent Brownian drift terms. This offers a rigorous methodology for incorporating collisions into the particle transport model, without coupling the equations of motions for each particle.
        
        Works by Chen, Chacón et al. \cite{Chen_Chacón_Barnes_2011, Chacón_Chen_Barnes_2013, Chen_Chacón_2014, Chen_Chacón_2015} have developed structure-preserving particle pushers for neoclassical transport in the Vlasov equations, derived from Crank--Nicolson integrators. We show these too can can derive from a FET interpretation, similarly offering potential extensions to higher-order-in-time particle pushers. The FET formulation is used also to consider how the stochastic drift terms can be incorporated into the pushers. Stochastic gyrokinetic expansions are also discussed.

        Different options for the numerical implementation of these schemes are considered.

        Due to the efficacy of FET in the development of SP timesteppers for both the fluid and kinetic component, we hope this approach will prove effective in the future for developing SP timesteppers for the full hybrid model. We hope this will give us the opportunity to incorporate previously inaccessible kinetic effects into the highly effective, modern, finite-element MHD models.
    \end{abstract}
    
    
    \newpage
    \tableofcontents
    
    
    \newpage
    \pagenumbering{arabic}
    %\linenumbers\renewcommand\thelinenumber{\color{black!50}\arabic{linenumber}}
            \documentclass[12pt, a4paper]{report}

\input{template/main.tex}

\title{\BA{Title in Progress...}}
\author{Boris Andrews}
\affil{Mathematical Institute, University of Oxford}
\date{\today}


\begin{document}
    \pagenumbering{gobble}
    \maketitle
    
    
    \begin{abstract}
        Magnetic confinement reactors---in particular tokamaks---offer one of the most promising options for achieving practical nuclear fusion, with the potential to provide virtually limitless, clean energy. The theoretical and numerical modeling of tokamak plasmas is simultaneously an essential component of effective reactor design, and a great research barrier. Tokamak operational conditions exhibit comparatively low Knudsen numbers. Kinetic effects, including kinetic waves and instabilities, Landau damping, bump-on-tail instabilities and more, are therefore highly influential in tokamak plasma dynamics. Purely fluid models are inherently incapable of capturing these effects, whereas the high dimensionality in purely kinetic models render them practically intractable for most relevant purposes.

        We consider a $\delta\!f$ decomposition model, with a macroscopic fluid background and microscopic kinetic correction, both fully coupled to each other. A similar manner of discretization is proposed to that used in the recent \texttt{STRUPHY} code \cite{Holderied_Possanner_Wang_2021, Holderied_2022, Li_et_al_2023} with a finite-element model for the background and a pseudo-particle/PiC model for the correction.

        The fluid background satisfies the full, non-linear, resistive, compressible, Hall MHD equations. \cite{Laakmann_Hu_Farrell_2022} introduces finite-element(-in-space) implicit timesteppers for the incompressible analogue to this system with structure-preserving (SP) properties in the ideal case, alongside parameter-robust preconditioners. We show that these timesteppers can derive from a finite-element-in-time (FET) (and finite-element-in-space) interpretation. The benefits of this reformulation are discussed, including the derivation of timesteppers that are higher order in time, and the quantifiable dissipative SP properties in the non-ideal, resistive case.
        
        We discuss possible options for extending this FET approach to timesteppers for the compressible case.

        The kinetic corrections satisfy linearized Boltzmann equations. Using a Lénard--Bernstein collision operator, these take Fokker--Planck-like forms \cite{Fokker_1914, Planck_1917} wherein pseudo-particles in the numerical model obey the neoclassical transport equations, with particle-independent Brownian drift terms. This offers a rigorous methodology for incorporating collisions into the particle transport model, without coupling the equations of motions for each particle.
        
        Works by Chen, Chacón et al. \cite{Chen_Chacón_Barnes_2011, Chacón_Chen_Barnes_2013, Chen_Chacón_2014, Chen_Chacón_2015} have developed structure-preserving particle pushers for neoclassical transport in the Vlasov equations, derived from Crank--Nicolson integrators. We show these too can can derive from a FET interpretation, similarly offering potential extensions to higher-order-in-time particle pushers. The FET formulation is used also to consider how the stochastic drift terms can be incorporated into the pushers. Stochastic gyrokinetic expansions are also discussed.

        Different options for the numerical implementation of these schemes are considered.

        Due to the efficacy of FET in the development of SP timesteppers for both the fluid and kinetic component, we hope this approach will prove effective in the future for developing SP timesteppers for the full hybrid model. We hope this will give us the opportunity to incorporate previously inaccessible kinetic effects into the highly effective, modern, finite-element MHD models.
    \end{abstract}
    
    
    \newpage
    \tableofcontents
    
    
    \newpage
    \pagenumbering{arabic}
    %\linenumbers\renewcommand\thelinenumber{\color{black!50}\arabic{linenumber}}
            \input{0 - introduction/main.tex}
        \part{Research}
            \input{1 - low-noise PiC models/main.tex}
            \input{2 - kinetic component/main.tex}
            \input{3 - fluid component/main.tex}
            \input{4 - numerical implementation/main.tex}
        \part{Project Overview}
            \input{5 - research plan/main.tex}
            \input{6 - summary/main.tex}
    
    
    %\section{}
    \newpage
    \pagenumbering{gobble}
        \printbibliography


    \newpage
    \pagenumbering{roman}
    \appendix
        \part{Appendices}
            \input{8 - Hilbert complexes/main.tex}
            \input{9 - weak conservation proofs/main.tex}
\end{document}

        \part{Research}
            \documentclass[12pt, a4paper]{report}

\input{template/main.tex}

\title{\BA{Title in Progress...}}
\author{Boris Andrews}
\affil{Mathematical Institute, University of Oxford}
\date{\today}


\begin{document}
    \pagenumbering{gobble}
    \maketitle
    
    
    \begin{abstract}
        Magnetic confinement reactors---in particular tokamaks---offer one of the most promising options for achieving practical nuclear fusion, with the potential to provide virtually limitless, clean energy. The theoretical and numerical modeling of tokamak plasmas is simultaneously an essential component of effective reactor design, and a great research barrier. Tokamak operational conditions exhibit comparatively low Knudsen numbers. Kinetic effects, including kinetic waves and instabilities, Landau damping, bump-on-tail instabilities and more, are therefore highly influential in tokamak plasma dynamics. Purely fluid models are inherently incapable of capturing these effects, whereas the high dimensionality in purely kinetic models render them practically intractable for most relevant purposes.

        We consider a $\delta\!f$ decomposition model, with a macroscopic fluid background and microscopic kinetic correction, both fully coupled to each other. A similar manner of discretization is proposed to that used in the recent \texttt{STRUPHY} code \cite{Holderied_Possanner_Wang_2021, Holderied_2022, Li_et_al_2023} with a finite-element model for the background and a pseudo-particle/PiC model for the correction.

        The fluid background satisfies the full, non-linear, resistive, compressible, Hall MHD equations. \cite{Laakmann_Hu_Farrell_2022} introduces finite-element(-in-space) implicit timesteppers for the incompressible analogue to this system with structure-preserving (SP) properties in the ideal case, alongside parameter-robust preconditioners. We show that these timesteppers can derive from a finite-element-in-time (FET) (and finite-element-in-space) interpretation. The benefits of this reformulation are discussed, including the derivation of timesteppers that are higher order in time, and the quantifiable dissipative SP properties in the non-ideal, resistive case.
        
        We discuss possible options for extending this FET approach to timesteppers for the compressible case.

        The kinetic corrections satisfy linearized Boltzmann equations. Using a Lénard--Bernstein collision operator, these take Fokker--Planck-like forms \cite{Fokker_1914, Planck_1917} wherein pseudo-particles in the numerical model obey the neoclassical transport equations, with particle-independent Brownian drift terms. This offers a rigorous methodology for incorporating collisions into the particle transport model, without coupling the equations of motions for each particle.
        
        Works by Chen, Chacón et al. \cite{Chen_Chacón_Barnes_2011, Chacón_Chen_Barnes_2013, Chen_Chacón_2014, Chen_Chacón_2015} have developed structure-preserving particle pushers for neoclassical transport in the Vlasov equations, derived from Crank--Nicolson integrators. We show these too can can derive from a FET interpretation, similarly offering potential extensions to higher-order-in-time particle pushers. The FET formulation is used also to consider how the stochastic drift terms can be incorporated into the pushers. Stochastic gyrokinetic expansions are also discussed.

        Different options for the numerical implementation of these schemes are considered.

        Due to the efficacy of FET in the development of SP timesteppers for both the fluid and kinetic component, we hope this approach will prove effective in the future for developing SP timesteppers for the full hybrid model. We hope this will give us the opportunity to incorporate previously inaccessible kinetic effects into the highly effective, modern, finite-element MHD models.
    \end{abstract}
    
    
    \newpage
    \tableofcontents
    
    
    \newpage
    \pagenumbering{arabic}
    %\linenumbers\renewcommand\thelinenumber{\color{black!50}\arabic{linenumber}}
            \input{0 - introduction/main.tex}
        \part{Research}
            \input{1 - low-noise PiC models/main.tex}
            \input{2 - kinetic component/main.tex}
            \input{3 - fluid component/main.tex}
            \input{4 - numerical implementation/main.tex}
        \part{Project Overview}
            \input{5 - research plan/main.tex}
            \input{6 - summary/main.tex}
    
    
    %\section{}
    \newpage
    \pagenumbering{gobble}
        \printbibliography


    \newpage
    \pagenumbering{roman}
    \appendix
        \part{Appendices}
            \input{8 - Hilbert complexes/main.tex}
            \input{9 - weak conservation proofs/main.tex}
\end{document}

            \documentclass[12pt, a4paper]{report}

\input{template/main.tex}

\title{\BA{Title in Progress...}}
\author{Boris Andrews}
\affil{Mathematical Institute, University of Oxford}
\date{\today}


\begin{document}
    \pagenumbering{gobble}
    \maketitle
    
    
    \begin{abstract}
        Magnetic confinement reactors---in particular tokamaks---offer one of the most promising options for achieving practical nuclear fusion, with the potential to provide virtually limitless, clean energy. The theoretical and numerical modeling of tokamak plasmas is simultaneously an essential component of effective reactor design, and a great research barrier. Tokamak operational conditions exhibit comparatively low Knudsen numbers. Kinetic effects, including kinetic waves and instabilities, Landau damping, bump-on-tail instabilities and more, are therefore highly influential in tokamak plasma dynamics. Purely fluid models are inherently incapable of capturing these effects, whereas the high dimensionality in purely kinetic models render them practically intractable for most relevant purposes.

        We consider a $\delta\!f$ decomposition model, with a macroscopic fluid background and microscopic kinetic correction, both fully coupled to each other. A similar manner of discretization is proposed to that used in the recent \texttt{STRUPHY} code \cite{Holderied_Possanner_Wang_2021, Holderied_2022, Li_et_al_2023} with a finite-element model for the background and a pseudo-particle/PiC model for the correction.

        The fluid background satisfies the full, non-linear, resistive, compressible, Hall MHD equations. \cite{Laakmann_Hu_Farrell_2022} introduces finite-element(-in-space) implicit timesteppers for the incompressible analogue to this system with structure-preserving (SP) properties in the ideal case, alongside parameter-robust preconditioners. We show that these timesteppers can derive from a finite-element-in-time (FET) (and finite-element-in-space) interpretation. The benefits of this reformulation are discussed, including the derivation of timesteppers that are higher order in time, and the quantifiable dissipative SP properties in the non-ideal, resistive case.
        
        We discuss possible options for extending this FET approach to timesteppers for the compressible case.

        The kinetic corrections satisfy linearized Boltzmann equations. Using a Lénard--Bernstein collision operator, these take Fokker--Planck-like forms \cite{Fokker_1914, Planck_1917} wherein pseudo-particles in the numerical model obey the neoclassical transport equations, with particle-independent Brownian drift terms. This offers a rigorous methodology for incorporating collisions into the particle transport model, without coupling the equations of motions for each particle.
        
        Works by Chen, Chacón et al. \cite{Chen_Chacón_Barnes_2011, Chacón_Chen_Barnes_2013, Chen_Chacón_2014, Chen_Chacón_2015} have developed structure-preserving particle pushers for neoclassical transport in the Vlasov equations, derived from Crank--Nicolson integrators. We show these too can can derive from a FET interpretation, similarly offering potential extensions to higher-order-in-time particle pushers. The FET formulation is used also to consider how the stochastic drift terms can be incorporated into the pushers. Stochastic gyrokinetic expansions are also discussed.

        Different options for the numerical implementation of these schemes are considered.

        Due to the efficacy of FET in the development of SP timesteppers for both the fluid and kinetic component, we hope this approach will prove effective in the future for developing SP timesteppers for the full hybrid model. We hope this will give us the opportunity to incorporate previously inaccessible kinetic effects into the highly effective, modern, finite-element MHD models.
    \end{abstract}
    
    
    \newpage
    \tableofcontents
    
    
    \newpage
    \pagenumbering{arabic}
    %\linenumbers\renewcommand\thelinenumber{\color{black!50}\arabic{linenumber}}
            \input{0 - introduction/main.tex}
        \part{Research}
            \input{1 - low-noise PiC models/main.tex}
            \input{2 - kinetic component/main.tex}
            \input{3 - fluid component/main.tex}
            \input{4 - numerical implementation/main.tex}
        \part{Project Overview}
            \input{5 - research plan/main.tex}
            \input{6 - summary/main.tex}
    
    
    %\section{}
    \newpage
    \pagenumbering{gobble}
        \printbibliography


    \newpage
    \pagenumbering{roman}
    \appendix
        \part{Appendices}
            \input{8 - Hilbert complexes/main.tex}
            \input{9 - weak conservation proofs/main.tex}
\end{document}

            \documentclass[12pt, a4paper]{report}

\input{template/main.tex}

\title{\BA{Title in Progress...}}
\author{Boris Andrews}
\affil{Mathematical Institute, University of Oxford}
\date{\today}


\begin{document}
    \pagenumbering{gobble}
    \maketitle
    
    
    \begin{abstract}
        Magnetic confinement reactors---in particular tokamaks---offer one of the most promising options for achieving practical nuclear fusion, with the potential to provide virtually limitless, clean energy. The theoretical and numerical modeling of tokamak plasmas is simultaneously an essential component of effective reactor design, and a great research barrier. Tokamak operational conditions exhibit comparatively low Knudsen numbers. Kinetic effects, including kinetic waves and instabilities, Landau damping, bump-on-tail instabilities and more, are therefore highly influential in tokamak plasma dynamics. Purely fluid models are inherently incapable of capturing these effects, whereas the high dimensionality in purely kinetic models render them practically intractable for most relevant purposes.

        We consider a $\delta\!f$ decomposition model, with a macroscopic fluid background and microscopic kinetic correction, both fully coupled to each other. A similar manner of discretization is proposed to that used in the recent \texttt{STRUPHY} code \cite{Holderied_Possanner_Wang_2021, Holderied_2022, Li_et_al_2023} with a finite-element model for the background and a pseudo-particle/PiC model for the correction.

        The fluid background satisfies the full, non-linear, resistive, compressible, Hall MHD equations. \cite{Laakmann_Hu_Farrell_2022} introduces finite-element(-in-space) implicit timesteppers for the incompressible analogue to this system with structure-preserving (SP) properties in the ideal case, alongside parameter-robust preconditioners. We show that these timesteppers can derive from a finite-element-in-time (FET) (and finite-element-in-space) interpretation. The benefits of this reformulation are discussed, including the derivation of timesteppers that are higher order in time, and the quantifiable dissipative SP properties in the non-ideal, resistive case.
        
        We discuss possible options for extending this FET approach to timesteppers for the compressible case.

        The kinetic corrections satisfy linearized Boltzmann equations. Using a Lénard--Bernstein collision operator, these take Fokker--Planck-like forms \cite{Fokker_1914, Planck_1917} wherein pseudo-particles in the numerical model obey the neoclassical transport equations, with particle-independent Brownian drift terms. This offers a rigorous methodology for incorporating collisions into the particle transport model, without coupling the equations of motions for each particle.
        
        Works by Chen, Chacón et al. \cite{Chen_Chacón_Barnes_2011, Chacón_Chen_Barnes_2013, Chen_Chacón_2014, Chen_Chacón_2015} have developed structure-preserving particle pushers for neoclassical transport in the Vlasov equations, derived from Crank--Nicolson integrators. We show these too can can derive from a FET interpretation, similarly offering potential extensions to higher-order-in-time particle pushers. The FET formulation is used also to consider how the stochastic drift terms can be incorporated into the pushers. Stochastic gyrokinetic expansions are also discussed.

        Different options for the numerical implementation of these schemes are considered.

        Due to the efficacy of FET in the development of SP timesteppers for both the fluid and kinetic component, we hope this approach will prove effective in the future for developing SP timesteppers for the full hybrid model. We hope this will give us the opportunity to incorporate previously inaccessible kinetic effects into the highly effective, modern, finite-element MHD models.
    \end{abstract}
    
    
    \newpage
    \tableofcontents
    
    
    \newpage
    \pagenumbering{arabic}
    %\linenumbers\renewcommand\thelinenumber{\color{black!50}\arabic{linenumber}}
            \input{0 - introduction/main.tex}
        \part{Research}
            \input{1 - low-noise PiC models/main.tex}
            \input{2 - kinetic component/main.tex}
            \input{3 - fluid component/main.tex}
            \input{4 - numerical implementation/main.tex}
        \part{Project Overview}
            \input{5 - research plan/main.tex}
            \input{6 - summary/main.tex}
    
    
    %\section{}
    \newpage
    \pagenumbering{gobble}
        \printbibliography


    \newpage
    \pagenumbering{roman}
    \appendix
        \part{Appendices}
            \input{8 - Hilbert complexes/main.tex}
            \input{9 - weak conservation proofs/main.tex}
\end{document}

            \documentclass[12pt, a4paper]{report}

\input{template/main.tex}

\title{\BA{Title in Progress...}}
\author{Boris Andrews}
\affil{Mathematical Institute, University of Oxford}
\date{\today}


\begin{document}
    \pagenumbering{gobble}
    \maketitle
    
    
    \begin{abstract}
        Magnetic confinement reactors---in particular tokamaks---offer one of the most promising options for achieving practical nuclear fusion, with the potential to provide virtually limitless, clean energy. The theoretical and numerical modeling of tokamak plasmas is simultaneously an essential component of effective reactor design, and a great research barrier. Tokamak operational conditions exhibit comparatively low Knudsen numbers. Kinetic effects, including kinetic waves and instabilities, Landau damping, bump-on-tail instabilities and more, are therefore highly influential in tokamak plasma dynamics. Purely fluid models are inherently incapable of capturing these effects, whereas the high dimensionality in purely kinetic models render them practically intractable for most relevant purposes.

        We consider a $\delta\!f$ decomposition model, with a macroscopic fluid background and microscopic kinetic correction, both fully coupled to each other. A similar manner of discretization is proposed to that used in the recent \texttt{STRUPHY} code \cite{Holderied_Possanner_Wang_2021, Holderied_2022, Li_et_al_2023} with a finite-element model for the background and a pseudo-particle/PiC model for the correction.

        The fluid background satisfies the full, non-linear, resistive, compressible, Hall MHD equations. \cite{Laakmann_Hu_Farrell_2022} introduces finite-element(-in-space) implicit timesteppers for the incompressible analogue to this system with structure-preserving (SP) properties in the ideal case, alongside parameter-robust preconditioners. We show that these timesteppers can derive from a finite-element-in-time (FET) (and finite-element-in-space) interpretation. The benefits of this reformulation are discussed, including the derivation of timesteppers that are higher order in time, and the quantifiable dissipative SP properties in the non-ideal, resistive case.
        
        We discuss possible options for extending this FET approach to timesteppers for the compressible case.

        The kinetic corrections satisfy linearized Boltzmann equations. Using a Lénard--Bernstein collision operator, these take Fokker--Planck-like forms \cite{Fokker_1914, Planck_1917} wherein pseudo-particles in the numerical model obey the neoclassical transport equations, with particle-independent Brownian drift terms. This offers a rigorous methodology for incorporating collisions into the particle transport model, without coupling the equations of motions for each particle.
        
        Works by Chen, Chacón et al. \cite{Chen_Chacón_Barnes_2011, Chacón_Chen_Barnes_2013, Chen_Chacón_2014, Chen_Chacón_2015} have developed structure-preserving particle pushers for neoclassical transport in the Vlasov equations, derived from Crank--Nicolson integrators. We show these too can can derive from a FET interpretation, similarly offering potential extensions to higher-order-in-time particle pushers. The FET formulation is used also to consider how the stochastic drift terms can be incorporated into the pushers. Stochastic gyrokinetic expansions are also discussed.

        Different options for the numerical implementation of these schemes are considered.

        Due to the efficacy of FET in the development of SP timesteppers for both the fluid and kinetic component, we hope this approach will prove effective in the future for developing SP timesteppers for the full hybrid model. We hope this will give us the opportunity to incorporate previously inaccessible kinetic effects into the highly effective, modern, finite-element MHD models.
    \end{abstract}
    
    
    \newpage
    \tableofcontents
    
    
    \newpage
    \pagenumbering{arabic}
    %\linenumbers\renewcommand\thelinenumber{\color{black!50}\arabic{linenumber}}
            \input{0 - introduction/main.tex}
        \part{Research}
            \input{1 - low-noise PiC models/main.tex}
            \input{2 - kinetic component/main.tex}
            \input{3 - fluid component/main.tex}
            \input{4 - numerical implementation/main.tex}
        \part{Project Overview}
            \input{5 - research plan/main.tex}
            \input{6 - summary/main.tex}
    
    
    %\section{}
    \newpage
    \pagenumbering{gobble}
        \printbibliography


    \newpage
    \pagenumbering{roman}
    \appendix
        \part{Appendices}
            \input{8 - Hilbert complexes/main.tex}
            \input{9 - weak conservation proofs/main.tex}
\end{document}

        \part{Project Overview}
            \documentclass[12pt, a4paper]{report}

\input{template/main.tex}

\title{\BA{Title in Progress...}}
\author{Boris Andrews}
\affil{Mathematical Institute, University of Oxford}
\date{\today}


\begin{document}
    \pagenumbering{gobble}
    \maketitle
    
    
    \begin{abstract}
        Magnetic confinement reactors---in particular tokamaks---offer one of the most promising options for achieving practical nuclear fusion, with the potential to provide virtually limitless, clean energy. The theoretical and numerical modeling of tokamak plasmas is simultaneously an essential component of effective reactor design, and a great research barrier. Tokamak operational conditions exhibit comparatively low Knudsen numbers. Kinetic effects, including kinetic waves and instabilities, Landau damping, bump-on-tail instabilities and more, are therefore highly influential in tokamak plasma dynamics. Purely fluid models are inherently incapable of capturing these effects, whereas the high dimensionality in purely kinetic models render them practically intractable for most relevant purposes.

        We consider a $\delta\!f$ decomposition model, with a macroscopic fluid background and microscopic kinetic correction, both fully coupled to each other. A similar manner of discretization is proposed to that used in the recent \texttt{STRUPHY} code \cite{Holderied_Possanner_Wang_2021, Holderied_2022, Li_et_al_2023} with a finite-element model for the background and a pseudo-particle/PiC model for the correction.

        The fluid background satisfies the full, non-linear, resistive, compressible, Hall MHD equations. \cite{Laakmann_Hu_Farrell_2022} introduces finite-element(-in-space) implicit timesteppers for the incompressible analogue to this system with structure-preserving (SP) properties in the ideal case, alongside parameter-robust preconditioners. We show that these timesteppers can derive from a finite-element-in-time (FET) (and finite-element-in-space) interpretation. The benefits of this reformulation are discussed, including the derivation of timesteppers that are higher order in time, and the quantifiable dissipative SP properties in the non-ideal, resistive case.
        
        We discuss possible options for extending this FET approach to timesteppers for the compressible case.

        The kinetic corrections satisfy linearized Boltzmann equations. Using a Lénard--Bernstein collision operator, these take Fokker--Planck-like forms \cite{Fokker_1914, Planck_1917} wherein pseudo-particles in the numerical model obey the neoclassical transport equations, with particle-independent Brownian drift terms. This offers a rigorous methodology for incorporating collisions into the particle transport model, without coupling the equations of motions for each particle.
        
        Works by Chen, Chacón et al. \cite{Chen_Chacón_Barnes_2011, Chacón_Chen_Barnes_2013, Chen_Chacón_2014, Chen_Chacón_2015} have developed structure-preserving particle pushers for neoclassical transport in the Vlasov equations, derived from Crank--Nicolson integrators. We show these too can can derive from a FET interpretation, similarly offering potential extensions to higher-order-in-time particle pushers. The FET formulation is used also to consider how the stochastic drift terms can be incorporated into the pushers. Stochastic gyrokinetic expansions are also discussed.

        Different options for the numerical implementation of these schemes are considered.

        Due to the efficacy of FET in the development of SP timesteppers for both the fluid and kinetic component, we hope this approach will prove effective in the future for developing SP timesteppers for the full hybrid model. We hope this will give us the opportunity to incorporate previously inaccessible kinetic effects into the highly effective, modern, finite-element MHD models.
    \end{abstract}
    
    
    \newpage
    \tableofcontents
    
    
    \newpage
    \pagenumbering{arabic}
    %\linenumbers\renewcommand\thelinenumber{\color{black!50}\arabic{linenumber}}
            \input{0 - introduction/main.tex}
        \part{Research}
            \input{1 - low-noise PiC models/main.tex}
            \input{2 - kinetic component/main.tex}
            \input{3 - fluid component/main.tex}
            \input{4 - numerical implementation/main.tex}
        \part{Project Overview}
            \input{5 - research plan/main.tex}
            \input{6 - summary/main.tex}
    
    
    %\section{}
    \newpage
    \pagenumbering{gobble}
        \printbibliography


    \newpage
    \pagenumbering{roman}
    \appendix
        \part{Appendices}
            \input{8 - Hilbert complexes/main.tex}
            \input{9 - weak conservation proofs/main.tex}
\end{document}

            \documentclass[12pt, a4paper]{report}

\input{template/main.tex}

\title{\BA{Title in Progress...}}
\author{Boris Andrews}
\affil{Mathematical Institute, University of Oxford}
\date{\today}


\begin{document}
    \pagenumbering{gobble}
    \maketitle
    
    
    \begin{abstract}
        Magnetic confinement reactors---in particular tokamaks---offer one of the most promising options for achieving practical nuclear fusion, with the potential to provide virtually limitless, clean energy. The theoretical and numerical modeling of tokamak plasmas is simultaneously an essential component of effective reactor design, and a great research barrier. Tokamak operational conditions exhibit comparatively low Knudsen numbers. Kinetic effects, including kinetic waves and instabilities, Landau damping, bump-on-tail instabilities and more, are therefore highly influential in tokamak plasma dynamics. Purely fluid models are inherently incapable of capturing these effects, whereas the high dimensionality in purely kinetic models render them practically intractable for most relevant purposes.

        We consider a $\delta\!f$ decomposition model, with a macroscopic fluid background and microscopic kinetic correction, both fully coupled to each other. A similar manner of discretization is proposed to that used in the recent \texttt{STRUPHY} code \cite{Holderied_Possanner_Wang_2021, Holderied_2022, Li_et_al_2023} with a finite-element model for the background and a pseudo-particle/PiC model for the correction.

        The fluid background satisfies the full, non-linear, resistive, compressible, Hall MHD equations. \cite{Laakmann_Hu_Farrell_2022} introduces finite-element(-in-space) implicit timesteppers for the incompressible analogue to this system with structure-preserving (SP) properties in the ideal case, alongside parameter-robust preconditioners. We show that these timesteppers can derive from a finite-element-in-time (FET) (and finite-element-in-space) interpretation. The benefits of this reformulation are discussed, including the derivation of timesteppers that are higher order in time, and the quantifiable dissipative SP properties in the non-ideal, resistive case.
        
        We discuss possible options for extending this FET approach to timesteppers for the compressible case.

        The kinetic corrections satisfy linearized Boltzmann equations. Using a Lénard--Bernstein collision operator, these take Fokker--Planck-like forms \cite{Fokker_1914, Planck_1917} wherein pseudo-particles in the numerical model obey the neoclassical transport equations, with particle-independent Brownian drift terms. This offers a rigorous methodology for incorporating collisions into the particle transport model, without coupling the equations of motions for each particle.
        
        Works by Chen, Chacón et al. \cite{Chen_Chacón_Barnes_2011, Chacón_Chen_Barnes_2013, Chen_Chacón_2014, Chen_Chacón_2015} have developed structure-preserving particle pushers for neoclassical transport in the Vlasov equations, derived from Crank--Nicolson integrators. We show these too can can derive from a FET interpretation, similarly offering potential extensions to higher-order-in-time particle pushers. The FET formulation is used also to consider how the stochastic drift terms can be incorporated into the pushers. Stochastic gyrokinetic expansions are also discussed.

        Different options for the numerical implementation of these schemes are considered.

        Due to the efficacy of FET in the development of SP timesteppers for both the fluid and kinetic component, we hope this approach will prove effective in the future for developing SP timesteppers for the full hybrid model. We hope this will give us the opportunity to incorporate previously inaccessible kinetic effects into the highly effective, modern, finite-element MHD models.
    \end{abstract}
    
    
    \newpage
    \tableofcontents
    
    
    \newpage
    \pagenumbering{arabic}
    %\linenumbers\renewcommand\thelinenumber{\color{black!50}\arabic{linenumber}}
            \input{0 - introduction/main.tex}
        \part{Research}
            \input{1 - low-noise PiC models/main.tex}
            \input{2 - kinetic component/main.tex}
            \input{3 - fluid component/main.tex}
            \input{4 - numerical implementation/main.tex}
        \part{Project Overview}
            \input{5 - research plan/main.tex}
            \input{6 - summary/main.tex}
    
    
    %\section{}
    \newpage
    \pagenumbering{gobble}
        \printbibliography


    \newpage
    \pagenumbering{roman}
    \appendix
        \part{Appendices}
            \input{8 - Hilbert complexes/main.tex}
            \input{9 - weak conservation proofs/main.tex}
\end{document}

    
    
    %\section{}
    \newpage
    \pagenumbering{gobble}
        \printbibliography


    \newpage
    \pagenumbering{roman}
    \appendix
        \part{Appendices}
            \documentclass[12pt, a4paper]{report}

\input{template/main.tex}

\title{\BA{Title in Progress...}}
\author{Boris Andrews}
\affil{Mathematical Institute, University of Oxford}
\date{\today}


\begin{document}
    \pagenumbering{gobble}
    \maketitle
    
    
    \begin{abstract}
        Magnetic confinement reactors---in particular tokamaks---offer one of the most promising options for achieving practical nuclear fusion, with the potential to provide virtually limitless, clean energy. The theoretical and numerical modeling of tokamak plasmas is simultaneously an essential component of effective reactor design, and a great research barrier. Tokamak operational conditions exhibit comparatively low Knudsen numbers. Kinetic effects, including kinetic waves and instabilities, Landau damping, bump-on-tail instabilities and more, are therefore highly influential in tokamak plasma dynamics. Purely fluid models are inherently incapable of capturing these effects, whereas the high dimensionality in purely kinetic models render them practically intractable for most relevant purposes.

        We consider a $\delta\!f$ decomposition model, with a macroscopic fluid background and microscopic kinetic correction, both fully coupled to each other. A similar manner of discretization is proposed to that used in the recent \texttt{STRUPHY} code \cite{Holderied_Possanner_Wang_2021, Holderied_2022, Li_et_al_2023} with a finite-element model for the background and a pseudo-particle/PiC model for the correction.

        The fluid background satisfies the full, non-linear, resistive, compressible, Hall MHD equations. \cite{Laakmann_Hu_Farrell_2022} introduces finite-element(-in-space) implicit timesteppers for the incompressible analogue to this system with structure-preserving (SP) properties in the ideal case, alongside parameter-robust preconditioners. We show that these timesteppers can derive from a finite-element-in-time (FET) (and finite-element-in-space) interpretation. The benefits of this reformulation are discussed, including the derivation of timesteppers that are higher order in time, and the quantifiable dissipative SP properties in the non-ideal, resistive case.
        
        We discuss possible options for extending this FET approach to timesteppers for the compressible case.

        The kinetic corrections satisfy linearized Boltzmann equations. Using a Lénard--Bernstein collision operator, these take Fokker--Planck-like forms \cite{Fokker_1914, Planck_1917} wherein pseudo-particles in the numerical model obey the neoclassical transport equations, with particle-independent Brownian drift terms. This offers a rigorous methodology for incorporating collisions into the particle transport model, without coupling the equations of motions for each particle.
        
        Works by Chen, Chacón et al. \cite{Chen_Chacón_Barnes_2011, Chacón_Chen_Barnes_2013, Chen_Chacón_2014, Chen_Chacón_2015} have developed structure-preserving particle pushers for neoclassical transport in the Vlasov equations, derived from Crank--Nicolson integrators. We show these too can can derive from a FET interpretation, similarly offering potential extensions to higher-order-in-time particle pushers. The FET formulation is used also to consider how the stochastic drift terms can be incorporated into the pushers. Stochastic gyrokinetic expansions are also discussed.

        Different options for the numerical implementation of these schemes are considered.

        Due to the efficacy of FET in the development of SP timesteppers for both the fluid and kinetic component, we hope this approach will prove effective in the future for developing SP timesteppers for the full hybrid model. We hope this will give us the opportunity to incorporate previously inaccessible kinetic effects into the highly effective, modern, finite-element MHD models.
    \end{abstract}
    
    
    \newpage
    \tableofcontents
    
    
    \newpage
    \pagenumbering{arabic}
    %\linenumbers\renewcommand\thelinenumber{\color{black!50}\arabic{linenumber}}
            \input{0 - introduction/main.tex}
        \part{Research}
            \input{1 - low-noise PiC models/main.tex}
            \input{2 - kinetic component/main.tex}
            \input{3 - fluid component/main.tex}
            \input{4 - numerical implementation/main.tex}
        \part{Project Overview}
            \input{5 - research plan/main.tex}
            \input{6 - summary/main.tex}
    
    
    %\section{}
    \newpage
    \pagenumbering{gobble}
        \printbibliography


    \newpage
    \pagenumbering{roman}
    \appendix
        \part{Appendices}
            \input{8 - Hilbert complexes/main.tex}
            \input{9 - weak conservation proofs/main.tex}
\end{document}

            \documentclass[12pt, a4paper]{report}

\input{template/main.tex}

\title{\BA{Title in Progress...}}
\author{Boris Andrews}
\affil{Mathematical Institute, University of Oxford}
\date{\today}


\begin{document}
    \pagenumbering{gobble}
    \maketitle
    
    
    \begin{abstract}
        Magnetic confinement reactors---in particular tokamaks---offer one of the most promising options for achieving practical nuclear fusion, with the potential to provide virtually limitless, clean energy. The theoretical and numerical modeling of tokamak plasmas is simultaneously an essential component of effective reactor design, and a great research barrier. Tokamak operational conditions exhibit comparatively low Knudsen numbers. Kinetic effects, including kinetic waves and instabilities, Landau damping, bump-on-tail instabilities and more, are therefore highly influential in tokamak plasma dynamics. Purely fluid models are inherently incapable of capturing these effects, whereas the high dimensionality in purely kinetic models render them practically intractable for most relevant purposes.

        We consider a $\delta\!f$ decomposition model, with a macroscopic fluid background and microscopic kinetic correction, both fully coupled to each other. A similar manner of discretization is proposed to that used in the recent \texttt{STRUPHY} code \cite{Holderied_Possanner_Wang_2021, Holderied_2022, Li_et_al_2023} with a finite-element model for the background and a pseudo-particle/PiC model for the correction.

        The fluid background satisfies the full, non-linear, resistive, compressible, Hall MHD equations. \cite{Laakmann_Hu_Farrell_2022} introduces finite-element(-in-space) implicit timesteppers for the incompressible analogue to this system with structure-preserving (SP) properties in the ideal case, alongside parameter-robust preconditioners. We show that these timesteppers can derive from a finite-element-in-time (FET) (and finite-element-in-space) interpretation. The benefits of this reformulation are discussed, including the derivation of timesteppers that are higher order in time, and the quantifiable dissipative SP properties in the non-ideal, resistive case.
        
        We discuss possible options for extending this FET approach to timesteppers for the compressible case.

        The kinetic corrections satisfy linearized Boltzmann equations. Using a Lénard--Bernstein collision operator, these take Fokker--Planck-like forms \cite{Fokker_1914, Planck_1917} wherein pseudo-particles in the numerical model obey the neoclassical transport equations, with particle-independent Brownian drift terms. This offers a rigorous methodology for incorporating collisions into the particle transport model, without coupling the equations of motions for each particle.
        
        Works by Chen, Chacón et al. \cite{Chen_Chacón_Barnes_2011, Chacón_Chen_Barnes_2013, Chen_Chacón_2014, Chen_Chacón_2015} have developed structure-preserving particle pushers for neoclassical transport in the Vlasov equations, derived from Crank--Nicolson integrators. We show these too can can derive from a FET interpretation, similarly offering potential extensions to higher-order-in-time particle pushers. The FET formulation is used also to consider how the stochastic drift terms can be incorporated into the pushers. Stochastic gyrokinetic expansions are also discussed.

        Different options for the numerical implementation of these schemes are considered.

        Due to the efficacy of FET in the development of SP timesteppers for both the fluid and kinetic component, we hope this approach will prove effective in the future for developing SP timesteppers for the full hybrid model. We hope this will give us the opportunity to incorporate previously inaccessible kinetic effects into the highly effective, modern, finite-element MHD models.
    \end{abstract}
    
    
    \newpage
    \tableofcontents
    
    
    \newpage
    \pagenumbering{arabic}
    %\linenumbers\renewcommand\thelinenumber{\color{black!50}\arabic{linenumber}}
            \input{0 - introduction/main.tex}
        \part{Research}
            \input{1 - low-noise PiC models/main.tex}
            \input{2 - kinetic component/main.tex}
            \input{3 - fluid component/main.tex}
            \input{4 - numerical implementation/main.tex}
        \part{Project Overview}
            \input{5 - research plan/main.tex}
            \input{6 - summary/main.tex}
    
    
    %\section{}
    \newpage
    \pagenumbering{gobble}
        \printbibliography


    \newpage
    \pagenumbering{roman}
    \appendix
        \part{Appendices}
            \input{8 - Hilbert complexes/main.tex}
            \input{9 - weak conservation proofs/main.tex}
\end{document}

\end{document}

            \documentclass[12pt, a4paper]{report}

\documentclass[12pt, a4paper]{report}

\input{template/main.tex}

\title{\BA{Title in Progress...}}
\author{Boris Andrews}
\affil{Mathematical Institute, University of Oxford}
\date{\today}


\begin{document}
    \pagenumbering{gobble}
    \maketitle
    
    
    \begin{abstract}
        Magnetic confinement reactors---in particular tokamaks---offer one of the most promising options for achieving practical nuclear fusion, with the potential to provide virtually limitless, clean energy. The theoretical and numerical modeling of tokamak plasmas is simultaneously an essential component of effective reactor design, and a great research barrier. Tokamak operational conditions exhibit comparatively low Knudsen numbers. Kinetic effects, including kinetic waves and instabilities, Landau damping, bump-on-tail instabilities and more, are therefore highly influential in tokamak plasma dynamics. Purely fluid models are inherently incapable of capturing these effects, whereas the high dimensionality in purely kinetic models render them practically intractable for most relevant purposes.

        We consider a $\delta\!f$ decomposition model, with a macroscopic fluid background and microscopic kinetic correction, both fully coupled to each other. A similar manner of discretization is proposed to that used in the recent \texttt{STRUPHY} code \cite{Holderied_Possanner_Wang_2021, Holderied_2022, Li_et_al_2023} with a finite-element model for the background and a pseudo-particle/PiC model for the correction.

        The fluid background satisfies the full, non-linear, resistive, compressible, Hall MHD equations. \cite{Laakmann_Hu_Farrell_2022} introduces finite-element(-in-space) implicit timesteppers for the incompressible analogue to this system with structure-preserving (SP) properties in the ideal case, alongside parameter-robust preconditioners. We show that these timesteppers can derive from a finite-element-in-time (FET) (and finite-element-in-space) interpretation. The benefits of this reformulation are discussed, including the derivation of timesteppers that are higher order in time, and the quantifiable dissipative SP properties in the non-ideal, resistive case.
        
        We discuss possible options for extending this FET approach to timesteppers for the compressible case.

        The kinetic corrections satisfy linearized Boltzmann equations. Using a Lénard--Bernstein collision operator, these take Fokker--Planck-like forms \cite{Fokker_1914, Planck_1917} wherein pseudo-particles in the numerical model obey the neoclassical transport equations, with particle-independent Brownian drift terms. This offers a rigorous methodology for incorporating collisions into the particle transport model, without coupling the equations of motions for each particle.
        
        Works by Chen, Chacón et al. \cite{Chen_Chacón_Barnes_2011, Chacón_Chen_Barnes_2013, Chen_Chacón_2014, Chen_Chacón_2015} have developed structure-preserving particle pushers for neoclassical transport in the Vlasov equations, derived from Crank--Nicolson integrators. We show these too can can derive from a FET interpretation, similarly offering potential extensions to higher-order-in-time particle pushers. The FET formulation is used also to consider how the stochastic drift terms can be incorporated into the pushers. Stochastic gyrokinetic expansions are also discussed.

        Different options for the numerical implementation of these schemes are considered.

        Due to the efficacy of FET in the development of SP timesteppers for both the fluid and kinetic component, we hope this approach will prove effective in the future for developing SP timesteppers for the full hybrid model. We hope this will give us the opportunity to incorporate previously inaccessible kinetic effects into the highly effective, modern, finite-element MHD models.
    \end{abstract}
    
    
    \newpage
    \tableofcontents
    
    
    \newpage
    \pagenumbering{arabic}
    %\linenumbers\renewcommand\thelinenumber{\color{black!50}\arabic{linenumber}}
            \input{0 - introduction/main.tex}
        \part{Research}
            \input{1 - low-noise PiC models/main.tex}
            \input{2 - kinetic component/main.tex}
            \input{3 - fluid component/main.tex}
            \input{4 - numerical implementation/main.tex}
        \part{Project Overview}
            \input{5 - research plan/main.tex}
            \input{6 - summary/main.tex}
    
    
    %\section{}
    \newpage
    \pagenumbering{gobble}
        \printbibliography


    \newpage
    \pagenumbering{roman}
    \appendix
        \part{Appendices}
            \input{8 - Hilbert complexes/main.tex}
            \input{9 - weak conservation proofs/main.tex}
\end{document}


\title{\BA{Title in Progress...}}
\author{Boris Andrews}
\affil{Mathematical Institute, University of Oxford}
\date{\today}


\begin{document}
    \pagenumbering{gobble}
    \maketitle
    
    
    \begin{abstract}
        Magnetic confinement reactors---in particular tokamaks---offer one of the most promising options for achieving practical nuclear fusion, with the potential to provide virtually limitless, clean energy. The theoretical and numerical modeling of tokamak plasmas is simultaneously an essential component of effective reactor design, and a great research barrier. Tokamak operational conditions exhibit comparatively low Knudsen numbers. Kinetic effects, including kinetic waves and instabilities, Landau damping, bump-on-tail instabilities and more, are therefore highly influential in tokamak plasma dynamics. Purely fluid models are inherently incapable of capturing these effects, whereas the high dimensionality in purely kinetic models render them practically intractable for most relevant purposes.

        We consider a $\delta\!f$ decomposition model, with a macroscopic fluid background and microscopic kinetic correction, both fully coupled to each other. A similar manner of discretization is proposed to that used in the recent \texttt{STRUPHY} code \cite{Holderied_Possanner_Wang_2021, Holderied_2022, Li_et_al_2023} with a finite-element model for the background and a pseudo-particle/PiC model for the correction.

        The fluid background satisfies the full, non-linear, resistive, compressible, Hall MHD equations. \cite{Laakmann_Hu_Farrell_2022} introduces finite-element(-in-space) implicit timesteppers for the incompressible analogue to this system with structure-preserving (SP) properties in the ideal case, alongside parameter-robust preconditioners. We show that these timesteppers can derive from a finite-element-in-time (FET) (and finite-element-in-space) interpretation. The benefits of this reformulation are discussed, including the derivation of timesteppers that are higher order in time, and the quantifiable dissipative SP properties in the non-ideal, resistive case.
        
        We discuss possible options for extending this FET approach to timesteppers for the compressible case.

        The kinetic corrections satisfy linearized Boltzmann equations. Using a Lénard--Bernstein collision operator, these take Fokker--Planck-like forms \cite{Fokker_1914, Planck_1917} wherein pseudo-particles in the numerical model obey the neoclassical transport equations, with particle-independent Brownian drift terms. This offers a rigorous methodology for incorporating collisions into the particle transport model, without coupling the equations of motions for each particle.
        
        Works by Chen, Chacón et al. \cite{Chen_Chacón_Barnes_2011, Chacón_Chen_Barnes_2013, Chen_Chacón_2014, Chen_Chacón_2015} have developed structure-preserving particle pushers for neoclassical transport in the Vlasov equations, derived from Crank--Nicolson integrators. We show these too can can derive from a FET interpretation, similarly offering potential extensions to higher-order-in-time particle pushers. The FET formulation is used also to consider how the stochastic drift terms can be incorporated into the pushers. Stochastic gyrokinetic expansions are also discussed.

        Different options for the numerical implementation of these schemes are considered.

        Due to the efficacy of FET in the development of SP timesteppers for both the fluid and kinetic component, we hope this approach will prove effective in the future for developing SP timesteppers for the full hybrid model. We hope this will give us the opportunity to incorporate previously inaccessible kinetic effects into the highly effective, modern, finite-element MHD models.
    \end{abstract}
    
    
    \newpage
    \tableofcontents
    
    
    \newpage
    \pagenumbering{arabic}
    %\linenumbers\renewcommand\thelinenumber{\color{black!50}\arabic{linenumber}}
            \documentclass[12pt, a4paper]{report}

\input{template/main.tex}

\title{\BA{Title in Progress...}}
\author{Boris Andrews}
\affil{Mathematical Institute, University of Oxford}
\date{\today}


\begin{document}
    \pagenumbering{gobble}
    \maketitle
    
    
    \begin{abstract}
        Magnetic confinement reactors---in particular tokamaks---offer one of the most promising options for achieving practical nuclear fusion, with the potential to provide virtually limitless, clean energy. The theoretical and numerical modeling of tokamak plasmas is simultaneously an essential component of effective reactor design, and a great research barrier. Tokamak operational conditions exhibit comparatively low Knudsen numbers. Kinetic effects, including kinetic waves and instabilities, Landau damping, bump-on-tail instabilities and more, are therefore highly influential in tokamak plasma dynamics. Purely fluid models are inherently incapable of capturing these effects, whereas the high dimensionality in purely kinetic models render them practically intractable for most relevant purposes.

        We consider a $\delta\!f$ decomposition model, with a macroscopic fluid background and microscopic kinetic correction, both fully coupled to each other. A similar manner of discretization is proposed to that used in the recent \texttt{STRUPHY} code \cite{Holderied_Possanner_Wang_2021, Holderied_2022, Li_et_al_2023} with a finite-element model for the background and a pseudo-particle/PiC model for the correction.

        The fluid background satisfies the full, non-linear, resistive, compressible, Hall MHD equations. \cite{Laakmann_Hu_Farrell_2022} introduces finite-element(-in-space) implicit timesteppers for the incompressible analogue to this system with structure-preserving (SP) properties in the ideal case, alongside parameter-robust preconditioners. We show that these timesteppers can derive from a finite-element-in-time (FET) (and finite-element-in-space) interpretation. The benefits of this reformulation are discussed, including the derivation of timesteppers that are higher order in time, and the quantifiable dissipative SP properties in the non-ideal, resistive case.
        
        We discuss possible options for extending this FET approach to timesteppers for the compressible case.

        The kinetic corrections satisfy linearized Boltzmann equations. Using a Lénard--Bernstein collision operator, these take Fokker--Planck-like forms \cite{Fokker_1914, Planck_1917} wherein pseudo-particles in the numerical model obey the neoclassical transport equations, with particle-independent Brownian drift terms. This offers a rigorous methodology for incorporating collisions into the particle transport model, without coupling the equations of motions for each particle.
        
        Works by Chen, Chacón et al. \cite{Chen_Chacón_Barnes_2011, Chacón_Chen_Barnes_2013, Chen_Chacón_2014, Chen_Chacón_2015} have developed structure-preserving particle pushers for neoclassical transport in the Vlasov equations, derived from Crank--Nicolson integrators. We show these too can can derive from a FET interpretation, similarly offering potential extensions to higher-order-in-time particle pushers. The FET formulation is used also to consider how the stochastic drift terms can be incorporated into the pushers. Stochastic gyrokinetic expansions are also discussed.

        Different options for the numerical implementation of these schemes are considered.

        Due to the efficacy of FET in the development of SP timesteppers for both the fluid and kinetic component, we hope this approach will prove effective in the future for developing SP timesteppers for the full hybrid model. We hope this will give us the opportunity to incorporate previously inaccessible kinetic effects into the highly effective, modern, finite-element MHD models.
    \end{abstract}
    
    
    \newpage
    \tableofcontents
    
    
    \newpage
    \pagenumbering{arabic}
    %\linenumbers\renewcommand\thelinenumber{\color{black!50}\arabic{linenumber}}
            \input{0 - introduction/main.tex}
        \part{Research}
            \input{1 - low-noise PiC models/main.tex}
            \input{2 - kinetic component/main.tex}
            \input{3 - fluid component/main.tex}
            \input{4 - numerical implementation/main.tex}
        \part{Project Overview}
            \input{5 - research plan/main.tex}
            \input{6 - summary/main.tex}
    
    
    %\section{}
    \newpage
    \pagenumbering{gobble}
        \printbibliography


    \newpage
    \pagenumbering{roman}
    \appendix
        \part{Appendices}
            \input{8 - Hilbert complexes/main.tex}
            \input{9 - weak conservation proofs/main.tex}
\end{document}

        \part{Research}
            \documentclass[12pt, a4paper]{report}

\input{template/main.tex}

\title{\BA{Title in Progress...}}
\author{Boris Andrews}
\affil{Mathematical Institute, University of Oxford}
\date{\today}


\begin{document}
    \pagenumbering{gobble}
    \maketitle
    
    
    \begin{abstract}
        Magnetic confinement reactors---in particular tokamaks---offer one of the most promising options for achieving practical nuclear fusion, with the potential to provide virtually limitless, clean energy. The theoretical and numerical modeling of tokamak plasmas is simultaneously an essential component of effective reactor design, and a great research barrier. Tokamak operational conditions exhibit comparatively low Knudsen numbers. Kinetic effects, including kinetic waves and instabilities, Landau damping, bump-on-tail instabilities and more, are therefore highly influential in tokamak plasma dynamics. Purely fluid models are inherently incapable of capturing these effects, whereas the high dimensionality in purely kinetic models render them practically intractable for most relevant purposes.

        We consider a $\delta\!f$ decomposition model, with a macroscopic fluid background and microscopic kinetic correction, both fully coupled to each other. A similar manner of discretization is proposed to that used in the recent \texttt{STRUPHY} code \cite{Holderied_Possanner_Wang_2021, Holderied_2022, Li_et_al_2023} with a finite-element model for the background and a pseudo-particle/PiC model for the correction.

        The fluid background satisfies the full, non-linear, resistive, compressible, Hall MHD equations. \cite{Laakmann_Hu_Farrell_2022} introduces finite-element(-in-space) implicit timesteppers for the incompressible analogue to this system with structure-preserving (SP) properties in the ideal case, alongside parameter-robust preconditioners. We show that these timesteppers can derive from a finite-element-in-time (FET) (and finite-element-in-space) interpretation. The benefits of this reformulation are discussed, including the derivation of timesteppers that are higher order in time, and the quantifiable dissipative SP properties in the non-ideal, resistive case.
        
        We discuss possible options for extending this FET approach to timesteppers for the compressible case.

        The kinetic corrections satisfy linearized Boltzmann equations. Using a Lénard--Bernstein collision operator, these take Fokker--Planck-like forms \cite{Fokker_1914, Planck_1917} wherein pseudo-particles in the numerical model obey the neoclassical transport equations, with particle-independent Brownian drift terms. This offers a rigorous methodology for incorporating collisions into the particle transport model, without coupling the equations of motions for each particle.
        
        Works by Chen, Chacón et al. \cite{Chen_Chacón_Barnes_2011, Chacón_Chen_Barnes_2013, Chen_Chacón_2014, Chen_Chacón_2015} have developed structure-preserving particle pushers for neoclassical transport in the Vlasov equations, derived from Crank--Nicolson integrators. We show these too can can derive from a FET interpretation, similarly offering potential extensions to higher-order-in-time particle pushers. The FET formulation is used also to consider how the stochastic drift terms can be incorporated into the pushers. Stochastic gyrokinetic expansions are also discussed.

        Different options for the numerical implementation of these schemes are considered.

        Due to the efficacy of FET in the development of SP timesteppers for both the fluid and kinetic component, we hope this approach will prove effective in the future for developing SP timesteppers for the full hybrid model. We hope this will give us the opportunity to incorporate previously inaccessible kinetic effects into the highly effective, modern, finite-element MHD models.
    \end{abstract}
    
    
    \newpage
    \tableofcontents
    
    
    \newpage
    \pagenumbering{arabic}
    %\linenumbers\renewcommand\thelinenumber{\color{black!50}\arabic{linenumber}}
            \input{0 - introduction/main.tex}
        \part{Research}
            \input{1 - low-noise PiC models/main.tex}
            \input{2 - kinetic component/main.tex}
            \input{3 - fluid component/main.tex}
            \input{4 - numerical implementation/main.tex}
        \part{Project Overview}
            \input{5 - research plan/main.tex}
            \input{6 - summary/main.tex}
    
    
    %\section{}
    \newpage
    \pagenumbering{gobble}
        \printbibliography


    \newpage
    \pagenumbering{roman}
    \appendix
        \part{Appendices}
            \input{8 - Hilbert complexes/main.tex}
            \input{9 - weak conservation proofs/main.tex}
\end{document}

            \documentclass[12pt, a4paper]{report}

\input{template/main.tex}

\title{\BA{Title in Progress...}}
\author{Boris Andrews}
\affil{Mathematical Institute, University of Oxford}
\date{\today}


\begin{document}
    \pagenumbering{gobble}
    \maketitle
    
    
    \begin{abstract}
        Magnetic confinement reactors---in particular tokamaks---offer one of the most promising options for achieving practical nuclear fusion, with the potential to provide virtually limitless, clean energy. The theoretical and numerical modeling of tokamak plasmas is simultaneously an essential component of effective reactor design, and a great research barrier. Tokamak operational conditions exhibit comparatively low Knudsen numbers. Kinetic effects, including kinetic waves and instabilities, Landau damping, bump-on-tail instabilities and more, are therefore highly influential in tokamak plasma dynamics. Purely fluid models are inherently incapable of capturing these effects, whereas the high dimensionality in purely kinetic models render them practically intractable for most relevant purposes.

        We consider a $\delta\!f$ decomposition model, with a macroscopic fluid background and microscopic kinetic correction, both fully coupled to each other. A similar manner of discretization is proposed to that used in the recent \texttt{STRUPHY} code \cite{Holderied_Possanner_Wang_2021, Holderied_2022, Li_et_al_2023} with a finite-element model for the background and a pseudo-particle/PiC model for the correction.

        The fluid background satisfies the full, non-linear, resistive, compressible, Hall MHD equations. \cite{Laakmann_Hu_Farrell_2022} introduces finite-element(-in-space) implicit timesteppers for the incompressible analogue to this system with structure-preserving (SP) properties in the ideal case, alongside parameter-robust preconditioners. We show that these timesteppers can derive from a finite-element-in-time (FET) (and finite-element-in-space) interpretation. The benefits of this reformulation are discussed, including the derivation of timesteppers that are higher order in time, and the quantifiable dissipative SP properties in the non-ideal, resistive case.
        
        We discuss possible options for extending this FET approach to timesteppers for the compressible case.

        The kinetic corrections satisfy linearized Boltzmann equations. Using a Lénard--Bernstein collision operator, these take Fokker--Planck-like forms \cite{Fokker_1914, Planck_1917} wherein pseudo-particles in the numerical model obey the neoclassical transport equations, with particle-independent Brownian drift terms. This offers a rigorous methodology for incorporating collisions into the particle transport model, without coupling the equations of motions for each particle.
        
        Works by Chen, Chacón et al. \cite{Chen_Chacón_Barnes_2011, Chacón_Chen_Barnes_2013, Chen_Chacón_2014, Chen_Chacón_2015} have developed structure-preserving particle pushers for neoclassical transport in the Vlasov equations, derived from Crank--Nicolson integrators. We show these too can can derive from a FET interpretation, similarly offering potential extensions to higher-order-in-time particle pushers. The FET formulation is used also to consider how the stochastic drift terms can be incorporated into the pushers. Stochastic gyrokinetic expansions are also discussed.

        Different options for the numerical implementation of these schemes are considered.

        Due to the efficacy of FET in the development of SP timesteppers for both the fluid and kinetic component, we hope this approach will prove effective in the future for developing SP timesteppers for the full hybrid model. We hope this will give us the opportunity to incorporate previously inaccessible kinetic effects into the highly effective, modern, finite-element MHD models.
    \end{abstract}
    
    
    \newpage
    \tableofcontents
    
    
    \newpage
    \pagenumbering{arabic}
    %\linenumbers\renewcommand\thelinenumber{\color{black!50}\arabic{linenumber}}
            \input{0 - introduction/main.tex}
        \part{Research}
            \input{1 - low-noise PiC models/main.tex}
            \input{2 - kinetic component/main.tex}
            \input{3 - fluid component/main.tex}
            \input{4 - numerical implementation/main.tex}
        \part{Project Overview}
            \input{5 - research plan/main.tex}
            \input{6 - summary/main.tex}
    
    
    %\section{}
    \newpage
    \pagenumbering{gobble}
        \printbibliography


    \newpage
    \pagenumbering{roman}
    \appendix
        \part{Appendices}
            \input{8 - Hilbert complexes/main.tex}
            \input{9 - weak conservation proofs/main.tex}
\end{document}

            \documentclass[12pt, a4paper]{report}

\input{template/main.tex}

\title{\BA{Title in Progress...}}
\author{Boris Andrews}
\affil{Mathematical Institute, University of Oxford}
\date{\today}


\begin{document}
    \pagenumbering{gobble}
    \maketitle
    
    
    \begin{abstract}
        Magnetic confinement reactors---in particular tokamaks---offer one of the most promising options for achieving practical nuclear fusion, with the potential to provide virtually limitless, clean energy. The theoretical and numerical modeling of tokamak plasmas is simultaneously an essential component of effective reactor design, and a great research barrier. Tokamak operational conditions exhibit comparatively low Knudsen numbers. Kinetic effects, including kinetic waves and instabilities, Landau damping, bump-on-tail instabilities and more, are therefore highly influential in tokamak plasma dynamics. Purely fluid models are inherently incapable of capturing these effects, whereas the high dimensionality in purely kinetic models render them practically intractable for most relevant purposes.

        We consider a $\delta\!f$ decomposition model, with a macroscopic fluid background and microscopic kinetic correction, both fully coupled to each other. A similar manner of discretization is proposed to that used in the recent \texttt{STRUPHY} code \cite{Holderied_Possanner_Wang_2021, Holderied_2022, Li_et_al_2023} with a finite-element model for the background and a pseudo-particle/PiC model for the correction.

        The fluid background satisfies the full, non-linear, resistive, compressible, Hall MHD equations. \cite{Laakmann_Hu_Farrell_2022} introduces finite-element(-in-space) implicit timesteppers for the incompressible analogue to this system with structure-preserving (SP) properties in the ideal case, alongside parameter-robust preconditioners. We show that these timesteppers can derive from a finite-element-in-time (FET) (and finite-element-in-space) interpretation. The benefits of this reformulation are discussed, including the derivation of timesteppers that are higher order in time, and the quantifiable dissipative SP properties in the non-ideal, resistive case.
        
        We discuss possible options for extending this FET approach to timesteppers for the compressible case.

        The kinetic corrections satisfy linearized Boltzmann equations. Using a Lénard--Bernstein collision operator, these take Fokker--Planck-like forms \cite{Fokker_1914, Planck_1917} wherein pseudo-particles in the numerical model obey the neoclassical transport equations, with particle-independent Brownian drift terms. This offers a rigorous methodology for incorporating collisions into the particle transport model, without coupling the equations of motions for each particle.
        
        Works by Chen, Chacón et al. \cite{Chen_Chacón_Barnes_2011, Chacón_Chen_Barnes_2013, Chen_Chacón_2014, Chen_Chacón_2015} have developed structure-preserving particle pushers for neoclassical transport in the Vlasov equations, derived from Crank--Nicolson integrators. We show these too can can derive from a FET interpretation, similarly offering potential extensions to higher-order-in-time particle pushers. The FET formulation is used also to consider how the stochastic drift terms can be incorporated into the pushers. Stochastic gyrokinetic expansions are also discussed.

        Different options for the numerical implementation of these schemes are considered.

        Due to the efficacy of FET in the development of SP timesteppers for both the fluid and kinetic component, we hope this approach will prove effective in the future for developing SP timesteppers for the full hybrid model. We hope this will give us the opportunity to incorporate previously inaccessible kinetic effects into the highly effective, modern, finite-element MHD models.
    \end{abstract}
    
    
    \newpage
    \tableofcontents
    
    
    \newpage
    \pagenumbering{arabic}
    %\linenumbers\renewcommand\thelinenumber{\color{black!50}\arabic{linenumber}}
            \input{0 - introduction/main.tex}
        \part{Research}
            \input{1 - low-noise PiC models/main.tex}
            \input{2 - kinetic component/main.tex}
            \input{3 - fluid component/main.tex}
            \input{4 - numerical implementation/main.tex}
        \part{Project Overview}
            \input{5 - research plan/main.tex}
            \input{6 - summary/main.tex}
    
    
    %\section{}
    \newpage
    \pagenumbering{gobble}
        \printbibliography


    \newpage
    \pagenumbering{roman}
    \appendix
        \part{Appendices}
            \input{8 - Hilbert complexes/main.tex}
            \input{9 - weak conservation proofs/main.tex}
\end{document}

            \documentclass[12pt, a4paper]{report}

\input{template/main.tex}

\title{\BA{Title in Progress...}}
\author{Boris Andrews}
\affil{Mathematical Institute, University of Oxford}
\date{\today}


\begin{document}
    \pagenumbering{gobble}
    \maketitle
    
    
    \begin{abstract}
        Magnetic confinement reactors---in particular tokamaks---offer one of the most promising options for achieving practical nuclear fusion, with the potential to provide virtually limitless, clean energy. The theoretical and numerical modeling of tokamak plasmas is simultaneously an essential component of effective reactor design, and a great research barrier. Tokamak operational conditions exhibit comparatively low Knudsen numbers. Kinetic effects, including kinetic waves and instabilities, Landau damping, bump-on-tail instabilities and more, are therefore highly influential in tokamak plasma dynamics. Purely fluid models are inherently incapable of capturing these effects, whereas the high dimensionality in purely kinetic models render them practically intractable for most relevant purposes.

        We consider a $\delta\!f$ decomposition model, with a macroscopic fluid background and microscopic kinetic correction, both fully coupled to each other. A similar manner of discretization is proposed to that used in the recent \texttt{STRUPHY} code \cite{Holderied_Possanner_Wang_2021, Holderied_2022, Li_et_al_2023} with a finite-element model for the background and a pseudo-particle/PiC model for the correction.

        The fluid background satisfies the full, non-linear, resistive, compressible, Hall MHD equations. \cite{Laakmann_Hu_Farrell_2022} introduces finite-element(-in-space) implicit timesteppers for the incompressible analogue to this system with structure-preserving (SP) properties in the ideal case, alongside parameter-robust preconditioners. We show that these timesteppers can derive from a finite-element-in-time (FET) (and finite-element-in-space) interpretation. The benefits of this reformulation are discussed, including the derivation of timesteppers that are higher order in time, and the quantifiable dissipative SP properties in the non-ideal, resistive case.
        
        We discuss possible options for extending this FET approach to timesteppers for the compressible case.

        The kinetic corrections satisfy linearized Boltzmann equations. Using a Lénard--Bernstein collision operator, these take Fokker--Planck-like forms \cite{Fokker_1914, Planck_1917} wherein pseudo-particles in the numerical model obey the neoclassical transport equations, with particle-independent Brownian drift terms. This offers a rigorous methodology for incorporating collisions into the particle transport model, without coupling the equations of motions for each particle.
        
        Works by Chen, Chacón et al. \cite{Chen_Chacón_Barnes_2011, Chacón_Chen_Barnes_2013, Chen_Chacón_2014, Chen_Chacón_2015} have developed structure-preserving particle pushers for neoclassical transport in the Vlasov equations, derived from Crank--Nicolson integrators. We show these too can can derive from a FET interpretation, similarly offering potential extensions to higher-order-in-time particle pushers. The FET formulation is used also to consider how the stochastic drift terms can be incorporated into the pushers. Stochastic gyrokinetic expansions are also discussed.

        Different options for the numerical implementation of these schemes are considered.

        Due to the efficacy of FET in the development of SP timesteppers for both the fluid and kinetic component, we hope this approach will prove effective in the future for developing SP timesteppers for the full hybrid model. We hope this will give us the opportunity to incorporate previously inaccessible kinetic effects into the highly effective, modern, finite-element MHD models.
    \end{abstract}
    
    
    \newpage
    \tableofcontents
    
    
    \newpage
    \pagenumbering{arabic}
    %\linenumbers\renewcommand\thelinenumber{\color{black!50}\arabic{linenumber}}
            \input{0 - introduction/main.tex}
        \part{Research}
            \input{1 - low-noise PiC models/main.tex}
            \input{2 - kinetic component/main.tex}
            \input{3 - fluid component/main.tex}
            \input{4 - numerical implementation/main.tex}
        \part{Project Overview}
            \input{5 - research plan/main.tex}
            \input{6 - summary/main.tex}
    
    
    %\section{}
    \newpage
    \pagenumbering{gobble}
        \printbibliography


    \newpage
    \pagenumbering{roman}
    \appendix
        \part{Appendices}
            \input{8 - Hilbert complexes/main.tex}
            \input{9 - weak conservation proofs/main.tex}
\end{document}

        \part{Project Overview}
            \documentclass[12pt, a4paper]{report}

\input{template/main.tex}

\title{\BA{Title in Progress...}}
\author{Boris Andrews}
\affil{Mathematical Institute, University of Oxford}
\date{\today}


\begin{document}
    \pagenumbering{gobble}
    \maketitle
    
    
    \begin{abstract}
        Magnetic confinement reactors---in particular tokamaks---offer one of the most promising options for achieving practical nuclear fusion, with the potential to provide virtually limitless, clean energy. The theoretical and numerical modeling of tokamak plasmas is simultaneously an essential component of effective reactor design, and a great research barrier. Tokamak operational conditions exhibit comparatively low Knudsen numbers. Kinetic effects, including kinetic waves and instabilities, Landau damping, bump-on-tail instabilities and more, are therefore highly influential in tokamak plasma dynamics. Purely fluid models are inherently incapable of capturing these effects, whereas the high dimensionality in purely kinetic models render them practically intractable for most relevant purposes.

        We consider a $\delta\!f$ decomposition model, with a macroscopic fluid background and microscopic kinetic correction, both fully coupled to each other. A similar manner of discretization is proposed to that used in the recent \texttt{STRUPHY} code \cite{Holderied_Possanner_Wang_2021, Holderied_2022, Li_et_al_2023} with a finite-element model for the background and a pseudo-particle/PiC model for the correction.

        The fluid background satisfies the full, non-linear, resistive, compressible, Hall MHD equations. \cite{Laakmann_Hu_Farrell_2022} introduces finite-element(-in-space) implicit timesteppers for the incompressible analogue to this system with structure-preserving (SP) properties in the ideal case, alongside parameter-robust preconditioners. We show that these timesteppers can derive from a finite-element-in-time (FET) (and finite-element-in-space) interpretation. The benefits of this reformulation are discussed, including the derivation of timesteppers that are higher order in time, and the quantifiable dissipative SP properties in the non-ideal, resistive case.
        
        We discuss possible options for extending this FET approach to timesteppers for the compressible case.

        The kinetic corrections satisfy linearized Boltzmann equations. Using a Lénard--Bernstein collision operator, these take Fokker--Planck-like forms \cite{Fokker_1914, Planck_1917} wherein pseudo-particles in the numerical model obey the neoclassical transport equations, with particle-independent Brownian drift terms. This offers a rigorous methodology for incorporating collisions into the particle transport model, without coupling the equations of motions for each particle.
        
        Works by Chen, Chacón et al. \cite{Chen_Chacón_Barnes_2011, Chacón_Chen_Barnes_2013, Chen_Chacón_2014, Chen_Chacón_2015} have developed structure-preserving particle pushers for neoclassical transport in the Vlasov equations, derived from Crank--Nicolson integrators. We show these too can can derive from a FET interpretation, similarly offering potential extensions to higher-order-in-time particle pushers. The FET formulation is used also to consider how the stochastic drift terms can be incorporated into the pushers. Stochastic gyrokinetic expansions are also discussed.

        Different options for the numerical implementation of these schemes are considered.

        Due to the efficacy of FET in the development of SP timesteppers for both the fluid and kinetic component, we hope this approach will prove effective in the future for developing SP timesteppers for the full hybrid model. We hope this will give us the opportunity to incorporate previously inaccessible kinetic effects into the highly effective, modern, finite-element MHD models.
    \end{abstract}
    
    
    \newpage
    \tableofcontents
    
    
    \newpage
    \pagenumbering{arabic}
    %\linenumbers\renewcommand\thelinenumber{\color{black!50}\arabic{linenumber}}
            \input{0 - introduction/main.tex}
        \part{Research}
            \input{1 - low-noise PiC models/main.tex}
            \input{2 - kinetic component/main.tex}
            \input{3 - fluid component/main.tex}
            \input{4 - numerical implementation/main.tex}
        \part{Project Overview}
            \input{5 - research plan/main.tex}
            \input{6 - summary/main.tex}
    
    
    %\section{}
    \newpage
    \pagenumbering{gobble}
        \printbibliography


    \newpage
    \pagenumbering{roman}
    \appendix
        \part{Appendices}
            \input{8 - Hilbert complexes/main.tex}
            \input{9 - weak conservation proofs/main.tex}
\end{document}

            \documentclass[12pt, a4paper]{report}

\input{template/main.tex}

\title{\BA{Title in Progress...}}
\author{Boris Andrews}
\affil{Mathematical Institute, University of Oxford}
\date{\today}


\begin{document}
    \pagenumbering{gobble}
    \maketitle
    
    
    \begin{abstract}
        Magnetic confinement reactors---in particular tokamaks---offer one of the most promising options for achieving practical nuclear fusion, with the potential to provide virtually limitless, clean energy. The theoretical and numerical modeling of tokamak plasmas is simultaneously an essential component of effective reactor design, and a great research barrier. Tokamak operational conditions exhibit comparatively low Knudsen numbers. Kinetic effects, including kinetic waves and instabilities, Landau damping, bump-on-tail instabilities and more, are therefore highly influential in tokamak plasma dynamics. Purely fluid models are inherently incapable of capturing these effects, whereas the high dimensionality in purely kinetic models render them practically intractable for most relevant purposes.

        We consider a $\delta\!f$ decomposition model, with a macroscopic fluid background and microscopic kinetic correction, both fully coupled to each other. A similar manner of discretization is proposed to that used in the recent \texttt{STRUPHY} code \cite{Holderied_Possanner_Wang_2021, Holderied_2022, Li_et_al_2023} with a finite-element model for the background and a pseudo-particle/PiC model for the correction.

        The fluid background satisfies the full, non-linear, resistive, compressible, Hall MHD equations. \cite{Laakmann_Hu_Farrell_2022} introduces finite-element(-in-space) implicit timesteppers for the incompressible analogue to this system with structure-preserving (SP) properties in the ideal case, alongside parameter-robust preconditioners. We show that these timesteppers can derive from a finite-element-in-time (FET) (and finite-element-in-space) interpretation. The benefits of this reformulation are discussed, including the derivation of timesteppers that are higher order in time, and the quantifiable dissipative SP properties in the non-ideal, resistive case.
        
        We discuss possible options for extending this FET approach to timesteppers for the compressible case.

        The kinetic corrections satisfy linearized Boltzmann equations. Using a Lénard--Bernstein collision operator, these take Fokker--Planck-like forms \cite{Fokker_1914, Planck_1917} wherein pseudo-particles in the numerical model obey the neoclassical transport equations, with particle-independent Brownian drift terms. This offers a rigorous methodology for incorporating collisions into the particle transport model, without coupling the equations of motions for each particle.
        
        Works by Chen, Chacón et al. \cite{Chen_Chacón_Barnes_2011, Chacón_Chen_Barnes_2013, Chen_Chacón_2014, Chen_Chacón_2015} have developed structure-preserving particle pushers for neoclassical transport in the Vlasov equations, derived from Crank--Nicolson integrators. We show these too can can derive from a FET interpretation, similarly offering potential extensions to higher-order-in-time particle pushers. The FET formulation is used also to consider how the stochastic drift terms can be incorporated into the pushers. Stochastic gyrokinetic expansions are also discussed.

        Different options for the numerical implementation of these schemes are considered.

        Due to the efficacy of FET in the development of SP timesteppers for both the fluid and kinetic component, we hope this approach will prove effective in the future for developing SP timesteppers for the full hybrid model. We hope this will give us the opportunity to incorporate previously inaccessible kinetic effects into the highly effective, modern, finite-element MHD models.
    \end{abstract}
    
    
    \newpage
    \tableofcontents
    
    
    \newpage
    \pagenumbering{arabic}
    %\linenumbers\renewcommand\thelinenumber{\color{black!50}\arabic{linenumber}}
            \input{0 - introduction/main.tex}
        \part{Research}
            \input{1 - low-noise PiC models/main.tex}
            \input{2 - kinetic component/main.tex}
            \input{3 - fluid component/main.tex}
            \input{4 - numerical implementation/main.tex}
        \part{Project Overview}
            \input{5 - research plan/main.tex}
            \input{6 - summary/main.tex}
    
    
    %\section{}
    \newpage
    \pagenumbering{gobble}
        \printbibliography


    \newpage
    \pagenumbering{roman}
    \appendix
        \part{Appendices}
            \input{8 - Hilbert complexes/main.tex}
            \input{9 - weak conservation proofs/main.tex}
\end{document}

    
    
    %\section{}
    \newpage
    \pagenumbering{gobble}
        \printbibliography


    \newpage
    \pagenumbering{roman}
    \appendix
        \part{Appendices}
            \documentclass[12pt, a4paper]{report}

\input{template/main.tex}

\title{\BA{Title in Progress...}}
\author{Boris Andrews}
\affil{Mathematical Institute, University of Oxford}
\date{\today}


\begin{document}
    \pagenumbering{gobble}
    \maketitle
    
    
    \begin{abstract}
        Magnetic confinement reactors---in particular tokamaks---offer one of the most promising options for achieving practical nuclear fusion, with the potential to provide virtually limitless, clean energy. The theoretical and numerical modeling of tokamak plasmas is simultaneously an essential component of effective reactor design, and a great research barrier. Tokamak operational conditions exhibit comparatively low Knudsen numbers. Kinetic effects, including kinetic waves and instabilities, Landau damping, bump-on-tail instabilities and more, are therefore highly influential in tokamak plasma dynamics. Purely fluid models are inherently incapable of capturing these effects, whereas the high dimensionality in purely kinetic models render them practically intractable for most relevant purposes.

        We consider a $\delta\!f$ decomposition model, with a macroscopic fluid background and microscopic kinetic correction, both fully coupled to each other. A similar manner of discretization is proposed to that used in the recent \texttt{STRUPHY} code \cite{Holderied_Possanner_Wang_2021, Holderied_2022, Li_et_al_2023} with a finite-element model for the background and a pseudo-particle/PiC model for the correction.

        The fluid background satisfies the full, non-linear, resistive, compressible, Hall MHD equations. \cite{Laakmann_Hu_Farrell_2022} introduces finite-element(-in-space) implicit timesteppers for the incompressible analogue to this system with structure-preserving (SP) properties in the ideal case, alongside parameter-robust preconditioners. We show that these timesteppers can derive from a finite-element-in-time (FET) (and finite-element-in-space) interpretation. The benefits of this reformulation are discussed, including the derivation of timesteppers that are higher order in time, and the quantifiable dissipative SP properties in the non-ideal, resistive case.
        
        We discuss possible options for extending this FET approach to timesteppers for the compressible case.

        The kinetic corrections satisfy linearized Boltzmann equations. Using a Lénard--Bernstein collision operator, these take Fokker--Planck-like forms \cite{Fokker_1914, Planck_1917} wherein pseudo-particles in the numerical model obey the neoclassical transport equations, with particle-independent Brownian drift terms. This offers a rigorous methodology for incorporating collisions into the particle transport model, without coupling the equations of motions for each particle.
        
        Works by Chen, Chacón et al. \cite{Chen_Chacón_Barnes_2011, Chacón_Chen_Barnes_2013, Chen_Chacón_2014, Chen_Chacón_2015} have developed structure-preserving particle pushers for neoclassical transport in the Vlasov equations, derived from Crank--Nicolson integrators. We show these too can can derive from a FET interpretation, similarly offering potential extensions to higher-order-in-time particle pushers. The FET formulation is used also to consider how the stochastic drift terms can be incorporated into the pushers. Stochastic gyrokinetic expansions are also discussed.

        Different options for the numerical implementation of these schemes are considered.

        Due to the efficacy of FET in the development of SP timesteppers for both the fluid and kinetic component, we hope this approach will prove effective in the future for developing SP timesteppers for the full hybrid model. We hope this will give us the opportunity to incorporate previously inaccessible kinetic effects into the highly effective, modern, finite-element MHD models.
    \end{abstract}
    
    
    \newpage
    \tableofcontents
    
    
    \newpage
    \pagenumbering{arabic}
    %\linenumbers\renewcommand\thelinenumber{\color{black!50}\arabic{linenumber}}
            \input{0 - introduction/main.tex}
        \part{Research}
            \input{1 - low-noise PiC models/main.tex}
            \input{2 - kinetic component/main.tex}
            \input{3 - fluid component/main.tex}
            \input{4 - numerical implementation/main.tex}
        \part{Project Overview}
            \input{5 - research plan/main.tex}
            \input{6 - summary/main.tex}
    
    
    %\section{}
    \newpage
    \pagenumbering{gobble}
        \printbibliography


    \newpage
    \pagenumbering{roman}
    \appendix
        \part{Appendices}
            \input{8 - Hilbert complexes/main.tex}
            \input{9 - weak conservation proofs/main.tex}
\end{document}

            \documentclass[12pt, a4paper]{report}

\input{template/main.tex}

\title{\BA{Title in Progress...}}
\author{Boris Andrews}
\affil{Mathematical Institute, University of Oxford}
\date{\today}


\begin{document}
    \pagenumbering{gobble}
    \maketitle
    
    
    \begin{abstract}
        Magnetic confinement reactors---in particular tokamaks---offer one of the most promising options for achieving practical nuclear fusion, with the potential to provide virtually limitless, clean energy. The theoretical and numerical modeling of tokamak plasmas is simultaneously an essential component of effective reactor design, and a great research barrier. Tokamak operational conditions exhibit comparatively low Knudsen numbers. Kinetic effects, including kinetic waves and instabilities, Landau damping, bump-on-tail instabilities and more, are therefore highly influential in tokamak plasma dynamics. Purely fluid models are inherently incapable of capturing these effects, whereas the high dimensionality in purely kinetic models render them practically intractable for most relevant purposes.

        We consider a $\delta\!f$ decomposition model, with a macroscopic fluid background and microscopic kinetic correction, both fully coupled to each other. A similar manner of discretization is proposed to that used in the recent \texttt{STRUPHY} code \cite{Holderied_Possanner_Wang_2021, Holderied_2022, Li_et_al_2023} with a finite-element model for the background and a pseudo-particle/PiC model for the correction.

        The fluid background satisfies the full, non-linear, resistive, compressible, Hall MHD equations. \cite{Laakmann_Hu_Farrell_2022} introduces finite-element(-in-space) implicit timesteppers for the incompressible analogue to this system with structure-preserving (SP) properties in the ideal case, alongside parameter-robust preconditioners. We show that these timesteppers can derive from a finite-element-in-time (FET) (and finite-element-in-space) interpretation. The benefits of this reformulation are discussed, including the derivation of timesteppers that are higher order in time, and the quantifiable dissipative SP properties in the non-ideal, resistive case.
        
        We discuss possible options for extending this FET approach to timesteppers for the compressible case.

        The kinetic corrections satisfy linearized Boltzmann equations. Using a Lénard--Bernstein collision operator, these take Fokker--Planck-like forms \cite{Fokker_1914, Planck_1917} wherein pseudo-particles in the numerical model obey the neoclassical transport equations, with particle-independent Brownian drift terms. This offers a rigorous methodology for incorporating collisions into the particle transport model, without coupling the equations of motions for each particle.
        
        Works by Chen, Chacón et al. \cite{Chen_Chacón_Barnes_2011, Chacón_Chen_Barnes_2013, Chen_Chacón_2014, Chen_Chacón_2015} have developed structure-preserving particle pushers for neoclassical transport in the Vlasov equations, derived from Crank--Nicolson integrators. We show these too can can derive from a FET interpretation, similarly offering potential extensions to higher-order-in-time particle pushers. The FET formulation is used also to consider how the stochastic drift terms can be incorporated into the pushers. Stochastic gyrokinetic expansions are also discussed.

        Different options for the numerical implementation of these schemes are considered.

        Due to the efficacy of FET in the development of SP timesteppers for both the fluid and kinetic component, we hope this approach will prove effective in the future for developing SP timesteppers for the full hybrid model. We hope this will give us the opportunity to incorporate previously inaccessible kinetic effects into the highly effective, modern, finite-element MHD models.
    \end{abstract}
    
    
    \newpage
    \tableofcontents
    
    
    \newpage
    \pagenumbering{arabic}
    %\linenumbers\renewcommand\thelinenumber{\color{black!50}\arabic{linenumber}}
            \input{0 - introduction/main.tex}
        \part{Research}
            \input{1 - low-noise PiC models/main.tex}
            \input{2 - kinetic component/main.tex}
            \input{3 - fluid component/main.tex}
            \input{4 - numerical implementation/main.tex}
        \part{Project Overview}
            \input{5 - research plan/main.tex}
            \input{6 - summary/main.tex}
    
    
    %\section{}
    \newpage
    \pagenumbering{gobble}
        \printbibliography


    \newpage
    \pagenumbering{roman}
    \appendix
        \part{Appendices}
            \input{8 - Hilbert complexes/main.tex}
            \input{9 - weak conservation proofs/main.tex}
\end{document}

\end{document}

            \documentclass[12pt, a4paper]{report}

\documentclass[12pt, a4paper]{report}

\input{template/main.tex}

\title{\BA{Title in Progress...}}
\author{Boris Andrews}
\affil{Mathematical Institute, University of Oxford}
\date{\today}


\begin{document}
    \pagenumbering{gobble}
    \maketitle
    
    
    \begin{abstract}
        Magnetic confinement reactors---in particular tokamaks---offer one of the most promising options for achieving practical nuclear fusion, with the potential to provide virtually limitless, clean energy. The theoretical and numerical modeling of tokamak plasmas is simultaneously an essential component of effective reactor design, and a great research barrier. Tokamak operational conditions exhibit comparatively low Knudsen numbers. Kinetic effects, including kinetic waves and instabilities, Landau damping, bump-on-tail instabilities and more, are therefore highly influential in tokamak plasma dynamics. Purely fluid models are inherently incapable of capturing these effects, whereas the high dimensionality in purely kinetic models render them practically intractable for most relevant purposes.

        We consider a $\delta\!f$ decomposition model, with a macroscopic fluid background and microscopic kinetic correction, both fully coupled to each other. A similar manner of discretization is proposed to that used in the recent \texttt{STRUPHY} code \cite{Holderied_Possanner_Wang_2021, Holderied_2022, Li_et_al_2023} with a finite-element model for the background and a pseudo-particle/PiC model for the correction.

        The fluid background satisfies the full, non-linear, resistive, compressible, Hall MHD equations. \cite{Laakmann_Hu_Farrell_2022} introduces finite-element(-in-space) implicit timesteppers for the incompressible analogue to this system with structure-preserving (SP) properties in the ideal case, alongside parameter-robust preconditioners. We show that these timesteppers can derive from a finite-element-in-time (FET) (and finite-element-in-space) interpretation. The benefits of this reformulation are discussed, including the derivation of timesteppers that are higher order in time, and the quantifiable dissipative SP properties in the non-ideal, resistive case.
        
        We discuss possible options for extending this FET approach to timesteppers for the compressible case.

        The kinetic corrections satisfy linearized Boltzmann equations. Using a Lénard--Bernstein collision operator, these take Fokker--Planck-like forms \cite{Fokker_1914, Planck_1917} wherein pseudo-particles in the numerical model obey the neoclassical transport equations, with particle-independent Brownian drift terms. This offers a rigorous methodology for incorporating collisions into the particle transport model, without coupling the equations of motions for each particle.
        
        Works by Chen, Chacón et al. \cite{Chen_Chacón_Barnes_2011, Chacón_Chen_Barnes_2013, Chen_Chacón_2014, Chen_Chacón_2015} have developed structure-preserving particle pushers for neoclassical transport in the Vlasov equations, derived from Crank--Nicolson integrators. We show these too can can derive from a FET interpretation, similarly offering potential extensions to higher-order-in-time particle pushers. The FET formulation is used also to consider how the stochastic drift terms can be incorporated into the pushers. Stochastic gyrokinetic expansions are also discussed.

        Different options for the numerical implementation of these schemes are considered.

        Due to the efficacy of FET in the development of SP timesteppers for both the fluid and kinetic component, we hope this approach will prove effective in the future for developing SP timesteppers for the full hybrid model. We hope this will give us the opportunity to incorporate previously inaccessible kinetic effects into the highly effective, modern, finite-element MHD models.
    \end{abstract}
    
    
    \newpage
    \tableofcontents
    
    
    \newpage
    \pagenumbering{arabic}
    %\linenumbers\renewcommand\thelinenumber{\color{black!50}\arabic{linenumber}}
            \input{0 - introduction/main.tex}
        \part{Research}
            \input{1 - low-noise PiC models/main.tex}
            \input{2 - kinetic component/main.tex}
            \input{3 - fluid component/main.tex}
            \input{4 - numerical implementation/main.tex}
        \part{Project Overview}
            \input{5 - research plan/main.tex}
            \input{6 - summary/main.tex}
    
    
    %\section{}
    \newpage
    \pagenumbering{gobble}
        \printbibliography


    \newpage
    \pagenumbering{roman}
    \appendix
        \part{Appendices}
            \input{8 - Hilbert complexes/main.tex}
            \input{9 - weak conservation proofs/main.tex}
\end{document}


\title{\BA{Title in Progress...}}
\author{Boris Andrews}
\affil{Mathematical Institute, University of Oxford}
\date{\today}


\begin{document}
    \pagenumbering{gobble}
    \maketitle
    
    
    \begin{abstract}
        Magnetic confinement reactors---in particular tokamaks---offer one of the most promising options for achieving practical nuclear fusion, with the potential to provide virtually limitless, clean energy. The theoretical and numerical modeling of tokamak plasmas is simultaneously an essential component of effective reactor design, and a great research barrier. Tokamak operational conditions exhibit comparatively low Knudsen numbers. Kinetic effects, including kinetic waves and instabilities, Landau damping, bump-on-tail instabilities and more, are therefore highly influential in tokamak plasma dynamics. Purely fluid models are inherently incapable of capturing these effects, whereas the high dimensionality in purely kinetic models render them practically intractable for most relevant purposes.

        We consider a $\delta\!f$ decomposition model, with a macroscopic fluid background and microscopic kinetic correction, both fully coupled to each other. A similar manner of discretization is proposed to that used in the recent \texttt{STRUPHY} code \cite{Holderied_Possanner_Wang_2021, Holderied_2022, Li_et_al_2023} with a finite-element model for the background and a pseudo-particle/PiC model for the correction.

        The fluid background satisfies the full, non-linear, resistive, compressible, Hall MHD equations. \cite{Laakmann_Hu_Farrell_2022} introduces finite-element(-in-space) implicit timesteppers for the incompressible analogue to this system with structure-preserving (SP) properties in the ideal case, alongside parameter-robust preconditioners. We show that these timesteppers can derive from a finite-element-in-time (FET) (and finite-element-in-space) interpretation. The benefits of this reformulation are discussed, including the derivation of timesteppers that are higher order in time, and the quantifiable dissipative SP properties in the non-ideal, resistive case.
        
        We discuss possible options for extending this FET approach to timesteppers for the compressible case.

        The kinetic corrections satisfy linearized Boltzmann equations. Using a Lénard--Bernstein collision operator, these take Fokker--Planck-like forms \cite{Fokker_1914, Planck_1917} wherein pseudo-particles in the numerical model obey the neoclassical transport equations, with particle-independent Brownian drift terms. This offers a rigorous methodology for incorporating collisions into the particle transport model, without coupling the equations of motions for each particle.
        
        Works by Chen, Chacón et al. \cite{Chen_Chacón_Barnes_2011, Chacón_Chen_Barnes_2013, Chen_Chacón_2014, Chen_Chacón_2015} have developed structure-preserving particle pushers for neoclassical transport in the Vlasov equations, derived from Crank--Nicolson integrators. We show these too can can derive from a FET interpretation, similarly offering potential extensions to higher-order-in-time particle pushers. The FET formulation is used also to consider how the stochastic drift terms can be incorporated into the pushers. Stochastic gyrokinetic expansions are also discussed.

        Different options for the numerical implementation of these schemes are considered.

        Due to the efficacy of FET in the development of SP timesteppers for both the fluid and kinetic component, we hope this approach will prove effective in the future for developing SP timesteppers for the full hybrid model. We hope this will give us the opportunity to incorporate previously inaccessible kinetic effects into the highly effective, modern, finite-element MHD models.
    \end{abstract}
    
    
    \newpage
    \tableofcontents
    
    
    \newpage
    \pagenumbering{arabic}
    %\linenumbers\renewcommand\thelinenumber{\color{black!50}\arabic{linenumber}}
            \documentclass[12pt, a4paper]{report}

\input{template/main.tex}

\title{\BA{Title in Progress...}}
\author{Boris Andrews}
\affil{Mathematical Institute, University of Oxford}
\date{\today}


\begin{document}
    \pagenumbering{gobble}
    \maketitle
    
    
    \begin{abstract}
        Magnetic confinement reactors---in particular tokamaks---offer one of the most promising options for achieving practical nuclear fusion, with the potential to provide virtually limitless, clean energy. The theoretical and numerical modeling of tokamak plasmas is simultaneously an essential component of effective reactor design, and a great research barrier. Tokamak operational conditions exhibit comparatively low Knudsen numbers. Kinetic effects, including kinetic waves and instabilities, Landau damping, bump-on-tail instabilities and more, are therefore highly influential in tokamak plasma dynamics. Purely fluid models are inherently incapable of capturing these effects, whereas the high dimensionality in purely kinetic models render them practically intractable for most relevant purposes.

        We consider a $\delta\!f$ decomposition model, with a macroscopic fluid background and microscopic kinetic correction, both fully coupled to each other. A similar manner of discretization is proposed to that used in the recent \texttt{STRUPHY} code \cite{Holderied_Possanner_Wang_2021, Holderied_2022, Li_et_al_2023} with a finite-element model for the background and a pseudo-particle/PiC model for the correction.

        The fluid background satisfies the full, non-linear, resistive, compressible, Hall MHD equations. \cite{Laakmann_Hu_Farrell_2022} introduces finite-element(-in-space) implicit timesteppers for the incompressible analogue to this system with structure-preserving (SP) properties in the ideal case, alongside parameter-robust preconditioners. We show that these timesteppers can derive from a finite-element-in-time (FET) (and finite-element-in-space) interpretation. The benefits of this reformulation are discussed, including the derivation of timesteppers that are higher order in time, and the quantifiable dissipative SP properties in the non-ideal, resistive case.
        
        We discuss possible options for extending this FET approach to timesteppers for the compressible case.

        The kinetic corrections satisfy linearized Boltzmann equations. Using a Lénard--Bernstein collision operator, these take Fokker--Planck-like forms \cite{Fokker_1914, Planck_1917} wherein pseudo-particles in the numerical model obey the neoclassical transport equations, with particle-independent Brownian drift terms. This offers a rigorous methodology for incorporating collisions into the particle transport model, without coupling the equations of motions for each particle.
        
        Works by Chen, Chacón et al. \cite{Chen_Chacón_Barnes_2011, Chacón_Chen_Barnes_2013, Chen_Chacón_2014, Chen_Chacón_2015} have developed structure-preserving particle pushers for neoclassical transport in the Vlasov equations, derived from Crank--Nicolson integrators. We show these too can can derive from a FET interpretation, similarly offering potential extensions to higher-order-in-time particle pushers. The FET formulation is used also to consider how the stochastic drift terms can be incorporated into the pushers. Stochastic gyrokinetic expansions are also discussed.

        Different options for the numerical implementation of these schemes are considered.

        Due to the efficacy of FET in the development of SP timesteppers for both the fluid and kinetic component, we hope this approach will prove effective in the future for developing SP timesteppers for the full hybrid model. We hope this will give us the opportunity to incorporate previously inaccessible kinetic effects into the highly effective, modern, finite-element MHD models.
    \end{abstract}
    
    
    \newpage
    \tableofcontents
    
    
    \newpage
    \pagenumbering{arabic}
    %\linenumbers\renewcommand\thelinenumber{\color{black!50}\arabic{linenumber}}
            \input{0 - introduction/main.tex}
        \part{Research}
            \input{1 - low-noise PiC models/main.tex}
            \input{2 - kinetic component/main.tex}
            \input{3 - fluid component/main.tex}
            \input{4 - numerical implementation/main.tex}
        \part{Project Overview}
            \input{5 - research plan/main.tex}
            \input{6 - summary/main.tex}
    
    
    %\section{}
    \newpage
    \pagenumbering{gobble}
        \printbibliography


    \newpage
    \pagenumbering{roman}
    \appendix
        \part{Appendices}
            \input{8 - Hilbert complexes/main.tex}
            \input{9 - weak conservation proofs/main.tex}
\end{document}

        \part{Research}
            \documentclass[12pt, a4paper]{report}

\input{template/main.tex}

\title{\BA{Title in Progress...}}
\author{Boris Andrews}
\affil{Mathematical Institute, University of Oxford}
\date{\today}


\begin{document}
    \pagenumbering{gobble}
    \maketitle
    
    
    \begin{abstract}
        Magnetic confinement reactors---in particular tokamaks---offer one of the most promising options for achieving practical nuclear fusion, with the potential to provide virtually limitless, clean energy. The theoretical and numerical modeling of tokamak plasmas is simultaneously an essential component of effective reactor design, and a great research barrier. Tokamak operational conditions exhibit comparatively low Knudsen numbers. Kinetic effects, including kinetic waves and instabilities, Landau damping, bump-on-tail instabilities and more, are therefore highly influential in tokamak plasma dynamics. Purely fluid models are inherently incapable of capturing these effects, whereas the high dimensionality in purely kinetic models render them practically intractable for most relevant purposes.

        We consider a $\delta\!f$ decomposition model, with a macroscopic fluid background and microscopic kinetic correction, both fully coupled to each other. A similar manner of discretization is proposed to that used in the recent \texttt{STRUPHY} code \cite{Holderied_Possanner_Wang_2021, Holderied_2022, Li_et_al_2023} with a finite-element model for the background and a pseudo-particle/PiC model for the correction.

        The fluid background satisfies the full, non-linear, resistive, compressible, Hall MHD equations. \cite{Laakmann_Hu_Farrell_2022} introduces finite-element(-in-space) implicit timesteppers for the incompressible analogue to this system with structure-preserving (SP) properties in the ideal case, alongside parameter-robust preconditioners. We show that these timesteppers can derive from a finite-element-in-time (FET) (and finite-element-in-space) interpretation. The benefits of this reformulation are discussed, including the derivation of timesteppers that are higher order in time, and the quantifiable dissipative SP properties in the non-ideal, resistive case.
        
        We discuss possible options for extending this FET approach to timesteppers for the compressible case.

        The kinetic corrections satisfy linearized Boltzmann equations. Using a Lénard--Bernstein collision operator, these take Fokker--Planck-like forms \cite{Fokker_1914, Planck_1917} wherein pseudo-particles in the numerical model obey the neoclassical transport equations, with particle-independent Brownian drift terms. This offers a rigorous methodology for incorporating collisions into the particle transport model, without coupling the equations of motions for each particle.
        
        Works by Chen, Chacón et al. \cite{Chen_Chacón_Barnes_2011, Chacón_Chen_Barnes_2013, Chen_Chacón_2014, Chen_Chacón_2015} have developed structure-preserving particle pushers for neoclassical transport in the Vlasov equations, derived from Crank--Nicolson integrators. We show these too can can derive from a FET interpretation, similarly offering potential extensions to higher-order-in-time particle pushers. The FET formulation is used also to consider how the stochastic drift terms can be incorporated into the pushers. Stochastic gyrokinetic expansions are also discussed.

        Different options for the numerical implementation of these schemes are considered.

        Due to the efficacy of FET in the development of SP timesteppers for both the fluid and kinetic component, we hope this approach will prove effective in the future for developing SP timesteppers for the full hybrid model. We hope this will give us the opportunity to incorporate previously inaccessible kinetic effects into the highly effective, modern, finite-element MHD models.
    \end{abstract}
    
    
    \newpage
    \tableofcontents
    
    
    \newpage
    \pagenumbering{arabic}
    %\linenumbers\renewcommand\thelinenumber{\color{black!50}\arabic{linenumber}}
            \input{0 - introduction/main.tex}
        \part{Research}
            \input{1 - low-noise PiC models/main.tex}
            \input{2 - kinetic component/main.tex}
            \input{3 - fluid component/main.tex}
            \input{4 - numerical implementation/main.tex}
        \part{Project Overview}
            \input{5 - research plan/main.tex}
            \input{6 - summary/main.tex}
    
    
    %\section{}
    \newpage
    \pagenumbering{gobble}
        \printbibliography


    \newpage
    \pagenumbering{roman}
    \appendix
        \part{Appendices}
            \input{8 - Hilbert complexes/main.tex}
            \input{9 - weak conservation proofs/main.tex}
\end{document}

            \documentclass[12pt, a4paper]{report}

\input{template/main.tex}

\title{\BA{Title in Progress...}}
\author{Boris Andrews}
\affil{Mathematical Institute, University of Oxford}
\date{\today}


\begin{document}
    \pagenumbering{gobble}
    \maketitle
    
    
    \begin{abstract}
        Magnetic confinement reactors---in particular tokamaks---offer one of the most promising options for achieving practical nuclear fusion, with the potential to provide virtually limitless, clean energy. The theoretical and numerical modeling of tokamak plasmas is simultaneously an essential component of effective reactor design, and a great research barrier. Tokamak operational conditions exhibit comparatively low Knudsen numbers. Kinetic effects, including kinetic waves and instabilities, Landau damping, bump-on-tail instabilities and more, are therefore highly influential in tokamak plasma dynamics. Purely fluid models are inherently incapable of capturing these effects, whereas the high dimensionality in purely kinetic models render them practically intractable for most relevant purposes.

        We consider a $\delta\!f$ decomposition model, with a macroscopic fluid background and microscopic kinetic correction, both fully coupled to each other. A similar manner of discretization is proposed to that used in the recent \texttt{STRUPHY} code \cite{Holderied_Possanner_Wang_2021, Holderied_2022, Li_et_al_2023} with a finite-element model for the background and a pseudo-particle/PiC model for the correction.

        The fluid background satisfies the full, non-linear, resistive, compressible, Hall MHD equations. \cite{Laakmann_Hu_Farrell_2022} introduces finite-element(-in-space) implicit timesteppers for the incompressible analogue to this system with structure-preserving (SP) properties in the ideal case, alongside parameter-robust preconditioners. We show that these timesteppers can derive from a finite-element-in-time (FET) (and finite-element-in-space) interpretation. The benefits of this reformulation are discussed, including the derivation of timesteppers that are higher order in time, and the quantifiable dissipative SP properties in the non-ideal, resistive case.
        
        We discuss possible options for extending this FET approach to timesteppers for the compressible case.

        The kinetic corrections satisfy linearized Boltzmann equations. Using a Lénard--Bernstein collision operator, these take Fokker--Planck-like forms \cite{Fokker_1914, Planck_1917} wherein pseudo-particles in the numerical model obey the neoclassical transport equations, with particle-independent Brownian drift terms. This offers a rigorous methodology for incorporating collisions into the particle transport model, without coupling the equations of motions for each particle.
        
        Works by Chen, Chacón et al. \cite{Chen_Chacón_Barnes_2011, Chacón_Chen_Barnes_2013, Chen_Chacón_2014, Chen_Chacón_2015} have developed structure-preserving particle pushers for neoclassical transport in the Vlasov equations, derived from Crank--Nicolson integrators. We show these too can can derive from a FET interpretation, similarly offering potential extensions to higher-order-in-time particle pushers. The FET formulation is used also to consider how the stochastic drift terms can be incorporated into the pushers. Stochastic gyrokinetic expansions are also discussed.

        Different options for the numerical implementation of these schemes are considered.

        Due to the efficacy of FET in the development of SP timesteppers for both the fluid and kinetic component, we hope this approach will prove effective in the future for developing SP timesteppers for the full hybrid model. We hope this will give us the opportunity to incorporate previously inaccessible kinetic effects into the highly effective, modern, finite-element MHD models.
    \end{abstract}
    
    
    \newpage
    \tableofcontents
    
    
    \newpage
    \pagenumbering{arabic}
    %\linenumbers\renewcommand\thelinenumber{\color{black!50}\arabic{linenumber}}
            \input{0 - introduction/main.tex}
        \part{Research}
            \input{1 - low-noise PiC models/main.tex}
            \input{2 - kinetic component/main.tex}
            \input{3 - fluid component/main.tex}
            \input{4 - numerical implementation/main.tex}
        \part{Project Overview}
            \input{5 - research plan/main.tex}
            \input{6 - summary/main.tex}
    
    
    %\section{}
    \newpage
    \pagenumbering{gobble}
        \printbibliography


    \newpage
    \pagenumbering{roman}
    \appendix
        \part{Appendices}
            \input{8 - Hilbert complexes/main.tex}
            \input{9 - weak conservation proofs/main.tex}
\end{document}

            \documentclass[12pt, a4paper]{report}

\input{template/main.tex}

\title{\BA{Title in Progress...}}
\author{Boris Andrews}
\affil{Mathematical Institute, University of Oxford}
\date{\today}


\begin{document}
    \pagenumbering{gobble}
    \maketitle
    
    
    \begin{abstract}
        Magnetic confinement reactors---in particular tokamaks---offer one of the most promising options for achieving practical nuclear fusion, with the potential to provide virtually limitless, clean energy. The theoretical and numerical modeling of tokamak plasmas is simultaneously an essential component of effective reactor design, and a great research barrier. Tokamak operational conditions exhibit comparatively low Knudsen numbers. Kinetic effects, including kinetic waves and instabilities, Landau damping, bump-on-tail instabilities and more, are therefore highly influential in tokamak plasma dynamics. Purely fluid models are inherently incapable of capturing these effects, whereas the high dimensionality in purely kinetic models render them practically intractable for most relevant purposes.

        We consider a $\delta\!f$ decomposition model, with a macroscopic fluid background and microscopic kinetic correction, both fully coupled to each other. A similar manner of discretization is proposed to that used in the recent \texttt{STRUPHY} code \cite{Holderied_Possanner_Wang_2021, Holderied_2022, Li_et_al_2023} with a finite-element model for the background and a pseudo-particle/PiC model for the correction.

        The fluid background satisfies the full, non-linear, resistive, compressible, Hall MHD equations. \cite{Laakmann_Hu_Farrell_2022} introduces finite-element(-in-space) implicit timesteppers for the incompressible analogue to this system with structure-preserving (SP) properties in the ideal case, alongside parameter-robust preconditioners. We show that these timesteppers can derive from a finite-element-in-time (FET) (and finite-element-in-space) interpretation. The benefits of this reformulation are discussed, including the derivation of timesteppers that are higher order in time, and the quantifiable dissipative SP properties in the non-ideal, resistive case.
        
        We discuss possible options for extending this FET approach to timesteppers for the compressible case.

        The kinetic corrections satisfy linearized Boltzmann equations. Using a Lénard--Bernstein collision operator, these take Fokker--Planck-like forms \cite{Fokker_1914, Planck_1917} wherein pseudo-particles in the numerical model obey the neoclassical transport equations, with particle-independent Brownian drift terms. This offers a rigorous methodology for incorporating collisions into the particle transport model, without coupling the equations of motions for each particle.
        
        Works by Chen, Chacón et al. \cite{Chen_Chacón_Barnes_2011, Chacón_Chen_Barnes_2013, Chen_Chacón_2014, Chen_Chacón_2015} have developed structure-preserving particle pushers for neoclassical transport in the Vlasov equations, derived from Crank--Nicolson integrators. We show these too can can derive from a FET interpretation, similarly offering potential extensions to higher-order-in-time particle pushers. The FET formulation is used also to consider how the stochastic drift terms can be incorporated into the pushers. Stochastic gyrokinetic expansions are also discussed.

        Different options for the numerical implementation of these schemes are considered.

        Due to the efficacy of FET in the development of SP timesteppers for both the fluid and kinetic component, we hope this approach will prove effective in the future for developing SP timesteppers for the full hybrid model. We hope this will give us the opportunity to incorporate previously inaccessible kinetic effects into the highly effective, modern, finite-element MHD models.
    \end{abstract}
    
    
    \newpage
    \tableofcontents
    
    
    \newpage
    \pagenumbering{arabic}
    %\linenumbers\renewcommand\thelinenumber{\color{black!50}\arabic{linenumber}}
            \input{0 - introduction/main.tex}
        \part{Research}
            \input{1 - low-noise PiC models/main.tex}
            \input{2 - kinetic component/main.tex}
            \input{3 - fluid component/main.tex}
            \input{4 - numerical implementation/main.tex}
        \part{Project Overview}
            \input{5 - research plan/main.tex}
            \input{6 - summary/main.tex}
    
    
    %\section{}
    \newpage
    \pagenumbering{gobble}
        \printbibliography


    \newpage
    \pagenumbering{roman}
    \appendix
        \part{Appendices}
            \input{8 - Hilbert complexes/main.tex}
            \input{9 - weak conservation proofs/main.tex}
\end{document}

            \documentclass[12pt, a4paper]{report}

\input{template/main.tex}

\title{\BA{Title in Progress...}}
\author{Boris Andrews}
\affil{Mathematical Institute, University of Oxford}
\date{\today}


\begin{document}
    \pagenumbering{gobble}
    \maketitle
    
    
    \begin{abstract}
        Magnetic confinement reactors---in particular tokamaks---offer one of the most promising options for achieving practical nuclear fusion, with the potential to provide virtually limitless, clean energy. The theoretical and numerical modeling of tokamak plasmas is simultaneously an essential component of effective reactor design, and a great research barrier. Tokamak operational conditions exhibit comparatively low Knudsen numbers. Kinetic effects, including kinetic waves and instabilities, Landau damping, bump-on-tail instabilities and more, are therefore highly influential in tokamak plasma dynamics. Purely fluid models are inherently incapable of capturing these effects, whereas the high dimensionality in purely kinetic models render them practically intractable for most relevant purposes.

        We consider a $\delta\!f$ decomposition model, with a macroscopic fluid background and microscopic kinetic correction, both fully coupled to each other. A similar manner of discretization is proposed to that used in the recent \texttt{STRUPHY} code \cite{Holderied_Possanner_Wang_2021, Holderied_2022, Li_et_al_2023} with a finite-element model for the background and a pseudo-particle/PiC model for the correction.

        The fluid background satisfies the full, non-linear, resistive, compressible, Hall MHD equations. \cite{Laakmann_Hu_Farrell_2022} introduces finite-element(-in-space) implicit timesteppers for the incompressible analogue to this system with structure-preserving (SP) properties in the ideal case, alongside parameter-robust preconditioners. We show that these timesteppers can derive from a finite-element-in-time (FET) (and finite-element-in-space) interpretation. The benefits of this reformulation are discussed, including the derivation of timesteppers that are higher order in time, and the quantifiable dissipative SP properties in the non-ideal, resistive case.
        
        We discuss possible options for extending this FET approach to timesteppers for the compressible case.

        The kinetic corrections satisfy linearized Boltzmann equations. Using a Lénard--Bernstein collision operator, these take Fokker--Planck-like forms \cite{Fokker_1914, Planck_1917} wherein pseudo-particles in the numerical model obey the neoclassical transport equations, with particle-independent Brownian drift terms. This offers a rigorous methodology for incorporating collisions into the particle transport model, without coupling the equations of motions for each particle.
        
        Works by Chen, Chacón et al. \cite{Chen_Chacón_Barnes_2011, Chacón_Chen_Barnes_2013, Chen_Chacón_2014, Chen_Chacón_2015} have developed structure-preserving particle pushers for neoclassical transport in the Vlasov equations, derived from Crank--Nicolson integrators. We show these too can can derive from a FET interpretation, similarly offering potential extensions to higher-order-in-time particle pushers. The FET formulation is used also to consider how the stochastic drift terms can be incorporated into the pushers. Stochastic gyrokinetic expansions are also discussed.

        Different options for the numerical implementation of these schemes are considered.

        Due to the efficacy of FET in the development of SP timesteppers for both the fluid and kinetic component, we hope this approach will prove effective in the future for developing SP timesteppers for the full hybrid model. We hope this will give us the opportunity to incorporate previously inaccessible kinetic effects into the highly effective, modern, finite-element MHD models.
    \end{abstract}
    
    
    \newpage
    \tableofcontents
    
    
    \newpage
    \pagenumbering{arabic}
    %\linenumbers\renewcommand\thelinenumber{\color{black!50}\arabic{linenumber}}
            \input{0 - introduction/main.tex}
        \part{Research}
            \input{1 - low-noise PiC models/main.tex}
            \input{2 - kinetic component/main.tex}
            \input{3 - fluid component/main.tex}
            \input{4 - numerical implementation/main.tex}
        \part{Project Overview}
            \input{5 - research plan/main.tex}
            \input{6 - summary/main.tex}
    
    
    %\section{}
    \newpage
    \pagenumbering{gobble}
        \printbibliography


    \newpage
    \pagenumbering{roman}
    \appendix
        \part{Appendices}
            \input{8 - Hilbert complexes/main.tex}
            \input{9 - weak conservation proofs/main.tex}
\end{document}

        \part{Project Overview}
            \documentclass[12pt, a4paper]{report}

\input{template/main.tex}

\title{\BA{Title in Progress...}}
\author{Boris Andrews}
\affil{Mathematical Institute, University of Oxford}
\date{\today}


\begin{document}
    \pagenumbering{gobble}
    \maketitle
    
    
    \begin{abstract}
        Magnetic confinement reactors---in particular tokamaks---offer one of the most promising options for achieving practical nuclear fusion, with the potential to provide virtually limitless, clean energy. The theoretical and numerical modeling of tokamak plasmas is simultaneously an essential component of effective reactor design, and a great research barrier. Tokamak operational conditions exhibit comparatively low Knudsen numbers. Kinetic effects, including kinetic waves and instabilities, Landau damping, bump-on-tail instabilities and more, are therefore highly influential in tokamak plasma dynamics. Purely fluid models are inherently incapable of capturing these effects, whereas the high dimensionality in purely kinetic models render them practically intractable for most relevant purposes.

        We consider a $\delta\!f$ decomposition model, with a macroscopic fluid background and microscopic kinetic correction, both fully coupled to each other. A similar manner of discretization is proposed to that used in the recent \texttt{STRUPHY} code \cite{Holderied_Possanner_Wang_2021, Holderied_2022, Li_et_al_2023} with a finite-element model for the background and a pseudo-particle/PiC model for the correction.

        The fluid background satisfies the full, non-linear, resistive, compressible, Hall MHD equations. \cite{Laakmann_Hu_Farrell_2022} introduces finite-element(-in-space) implicit timesteppers for the incompressible analogue to this system with structure-preserving (SP) properties in the ideal case, alongside parameter-robust preconditioners. We show that these timesteppers can derive from a finite-element-in-time (FET) (and finite-element-in-space) interpretation. The benefits of this reformulation are discussed, including the derivation of timesteppers that are higher order in time, and the quantifiable dissipative SP properties in the non-ideal, resistive case.
        
        We discuss possible options for extending this FET approach to timesteppers for the compressible case.

        The kinetic corrections satisfy linearized Boltzmann equations. Using a Lénard--Bernstein collision operator, these take Fokker--Planck-like forms \cite{Fokker_1914, Planck_1917} wherein pseudo-particles in the numerical model obey the neoclassical transport equations, with particle-independent Brownian drift terms. This offers a rigorous methodology for incorporating collisions into the particle transport model, without coupling the equations of motions for each particle.
        
        Works by Chen, Chacón et al. \cite{Chen_Chacón_Barnes_2011, Chacón_Chen_Barnes_2013, Chen_Chacón_2014, Chen_Chacón_2015} have developed structure-preserving particle pushers for neoclassical transport in the Vlasov equations, derived from Crank--Nicolson integrators. We show these too can can derive from a FET interpretation, similarly offering potential extensions to higher-order-in-time particle pushers. The FET formulation is used also to consider how the stochastic drift terms can be incorporated into the pushers. Stochastic gyrokinetic expansions are also discussed.

        Different options for the numerical implementation of these schemes are considered.

        Due to the efficacy of FET in the development of SP timesteppers for both the fluid and kinetic component, we hope this approach will prove effective in the future for developing SP timesteppers for the full hybrid model. We hope this will give us the opportunity to incorporate previously inaccessible kinetic effects into the highly effective, modern, finite-element MHD models.
    \end{abstract}
    
    
    \newpage
    \tableofcontents
    
    
    \newpage
    \pagenumbering{arabic}
    %\linenumbers\renewcommand\thelinenumber{\color{black!50}\arabic{linenumber}}
            \input{0 - introduction/main.tex}
        \part{Research}
            \input{1 - low-noise PiC models/main.tex}
            \input{2 - kinetic component/main.tex}
            \input{3 - fluid component/main.tex}
            \input{4 - numerical implementation/main.tex}
        \part{Project Overview}
            \input{5 - research plan/main.tex}
            \input{6 - summary/main.tex}
    
    
    %\section{}
    \newpage
    \pagenumbering{gobble}
        \printbibliography


    \newpage
    \pagenumbering{roman}
    \appendix
        \part{Appendices}
            \input{8 - Hilbert complexes/main.tex}
            \input{9 - weak conservation proofs/main.tex}
\end{document}

            \documentclass[12pt, a4paper]{report}

\input{template/main.tex}

\title{\BA{Title in Progress...}}
\author{Boris Andrews}
\affil{Mathematical Institute, University of Oxford}
\date{\today}


\begin{document}
    \pagenumbering{gobble}
    \maketitle
    
    
    \begin{abstract}
        Magnetic confinement reactors---in particular tokamaks---offer one of the most promising options for achieving practical nuclear fusion, with the potential to provide virtually limitless, clean energy. The theoretical and numerical modeling of tokamak plasmas is simultaneously an essential component of effective reactor design, and a great research barrier. Tokamak operational conditions exhibit comparatively low Knudsen numbers. Kinetic effects, including kinetic waves and instabilities, Landau damping, bump-on-tail instabilities and more, are therefore highly influential in tokamak plasma dynamics. Purely fluid models are inherently incapable of capturing these effects, whereas the high dimensionality in purely kinetic models render them practically intractable for most relevant purposes.

        We consider a $\delta\!f$ decomposition model, with a macroscopic fluid background and microscopic kinetic correction, both fully coupled to each other. A similar manner of discretization is proposed to that used in the recent \texttt{STRUPHY} code \cite{Holderied_Possanner_Wang_2021, Holderied_2022, Li_et_al_2023} with a finite-element model for the background and a pseudo-particle/PiC model for the correction.

        The fluid background satisfies the full, non-linear, resistive, compressible, Hall MHD equations. \cite{Laakmann_Hu_Farrell_2022} introduces finite-element(-in-space) implicit timesteppers for the incompressible analogue to this system with structure-preserving (SP) properties in the ideal case, alongside parameter-robust preconditioners. We show that these timesteppers can derive from a finite-element-in-time (FET) (and finite-element-in-space) interpretation. The benefits of this reformulation are discussed, including the derivation of timesteppers that are higher order in time, and the quantifiable dissipative SP properties in the non-ideal, resistive case.
        
        We discuss possible options for extending this FET approach to timesteppers for the compressible case.

        The kinetic corrections satisfy linearized Boltzmann equations. Using a Lénard--Bernstein collision operator, these take Fokker--Planck-like forms \cite{Fokker_1914, Planck_1917} wherein pseudo-particles in the numerical model obey the neoclassical transport equations, with particle-independent Brownian drift terms. This offers a rigorous methodology for incorporating collisions into the particle transport model, without coupling the equations of motions for each particle.
        
        Works by Chen, Chacón et al. \cite{Chen_Chacón_Barnes_2011, Chacón_Chen_Barnes_2013, Chen_Chacón_2014, Chen_Chacón_2015} have developed structure-preserving particle pushers for neoclassical transport in the Vlasov equations, derived from Crank--Nicolson integrators. We show these too can can derive from a FET interpretation, similarly offering potential extensions to higher-order-in-time particle pushers. The FET formulation is used also to consider how the stochastic drift terms can be incorporated into the pushers. Stochastic gyrokinetic expansions are also discussed.

        Different options for the numerical implementation of these schemes are considered.

        Due to the efficacy of FET in the development of SP timesteppers for both the fluid and kinetic component, we hope this approach will prove effective in the future for developing SP timesteppers for the full hybrid model. We hope this will give us the opportunity to incorporate previously inaccessible kinetic effects into the highly effective, modern, finite-element MHD models.
    \end{abstract}
    
    
    \newpage
    \tableofcontents
    
    
    \newpage
    \pagenumbering{arabic}
    %\linenumbers\renewcommand\thelinenumber{\color{black!50}\arabic{linenumber}}
            \input{0 - introduction/main.tex}
        \part{Research}
            \input{1 - low-noise PiC models/main.tex}
            \input{2 - kinetic component/main.tex}
            \input{3 - fluid component/main.tex}
            \input{4 - numerical implementation/main.tex}
        \part{Project Overview}
            \input{5 - research plan/main.tex}
            \input{6 - summary/main.tex}
    
    
    %\section{}
    \newpage
    \pagenumbering{gobble}
        \printbibliography


    \newpage
    \pagenumbering{roman}
    \appendix
        \part{Appendices}
            \input{8 - Hilbert complexes/main.tex}
            \input{9 - weak conservation proofs/main.tex}
\end{document}

    
    
    %\section{}
    \newpage
    \pagenumbering{gobble}
        \printbibliography


    \newpage
    \pagenumbering{roman}
    \appendix
        \part{Appendices}
            \documentclass[12pt, a4paper]{report}

\input{template/main.tex}

\title{\BA{Title in Progress...}}
\author{Boris Andrews}
\affil{Mathematical Institute, University of Oxford}
\date{\today}


\begin{document}
    \pagenumbering{gobble}
    \maketitle
    
    
    \begin{abstract}
        Magnetic confinement reactors---in particular tokamaks---offer one of the most promising options for achieving practical nuclear fusion, with the potential to provide virtually limitless, clean energy. The theoretical and numerical modeling of tokamak plasmas is simultaneously an essential component of effective reactor design, and a great research barrier. Tokamak operational conditions exhibit comparatively low Knudsen numbers. Kinetic effects, including kinetic waves and instabilities, Landau damping, bump-on-tail instabilities and more, are therefore highly influential in tokamak plasma dynamics. Purely fluid models are inherently incapable of capturing these effects, whereas the high dimensionality in purely kinetic models render them practically intractable for most relevant purposes.

        We consider a $\delta\!f$ decomposition model, with a macroscopic fluid background and microscopic kinetic correction, both fully coupled to each other. A similar manner of discretization is proposed to that used in the recent \texttt{STRUPHY} code \cite{Holderied_Possanner_Wang_2021, Holderied_2022, Li_et_al_2023} with a finite-element model for the background and a pseudo-particle/PiC model for the correction.

        The fluid background satisfies the full, non-linear, resistive, compressible, Hall MHD equations. \cite{Laakmann_Hu_Farrell_2022} introduces finite-element(-in-space) implicit timesteppers for the incompressible analogue to this system with structure-preserving (SP) properties in the ideal case, alongside parameter-robust preconditioners. We show that these timesteppers can derive from a finite-element-in-time (FET) (and finite-element-in-space) interpretation. The benefits of this reformulation are discussed, including the derivation of timesteppers that are higher order in time, and the quantifiable dissipative SP properties in the non-ideal, resistive case.
        
        We discuss possible options for extending this FET approach to timesteppers for the compressible case.

        The kinetic corrections satisfy linearized Boltzmann equations. Using a Lénard--Bernstein collision operator, these take Fokker--Planck-like forms \cite{Fokker_1914, Planck_1917} wherein pseudo-particles in the numerical model obey the neoclassical transport equations, with particle-independent Brownian drift terms. This offers a rigorous methodology for incorporating collisions into the particle transport model, without coupling the equations of motions for each particle.
        
        Works by Chen, Chacón et al. \cite{Chen_Chacón_Barnes_2011, Chacón_Chen_Barnes_2013, Chen_Chacón_2014, Chen_Chacón_2015} have developed structure-preserving particle pushers for neoclassical transport in the Vlasov equations, derived from Crank--Nicolson integrators. We show these too can can derive from a FET interpretation, similarly offering potential extensions to higher-order-in-time particle pushers. The FET formulation is used also to consider how the stochastic drift terms can be incorporated into the pushers. Stochastic gyrokinetic expansions are also discussed.

        Different options for the numerical implementation of these schemes are considered.

        Due to the efficacy of FET in the development of SP timesteppers for both the fluid and kinetic component, we hope this approach will prove effective in the future for developing SP timesteppers for the full hybrid model. We hope this will give us the opportunity to incorporate previously inaccessible kinetic effects into the highly effective, modern, finite-element MHD models.
    \end{abstract}
    
    
    \newpage
    \tableofcontents
    
    
    \newpage
    \pagenumbering{arabic}
    %\linenumbers\renewcommand\thelinenumber{\color{black!50}\arabic{linenumber}}
            \input{0 - introduction/main.tex}
        \part{Research}
            \input{1 - low-noise PiC models/main.tex}
            \input{2 - kinetic component/main.tex}
            \input{3 - fluid component/main.tex}
            \input{4 - numerical implementation/main.tex}
        \part{Project Overview}
            \input{5 - research plan/main.tex}
            \input{6 - summary/main.tex}
    
    
    %\section{}
    \newpage
    \pagenumbering{gobble}
        \printbibliography


    \newpage
    \pagenumbering{roman}
    \appendix
        \part{Appendices}
            \input{8 - Hilbert complexes/main.tex}
            \input{9 - weak conservation proofs/main.tex}
\end{document}

            \documentclass[12pt, a4paper]{report}

\input{template/main.tex}

\title{\BA{Title in Progress...}}
\author{Boris Andrews}
\affil{Mathematical Institute, University of Oxford}
\date{\today}


\begin{document}
    \pagenumbering{gobble}
    \maketitle
    
    
    \begin{abstract}
        Magnetic confinement reactors---in particular tokamaks---offer one of the most promising options for achieving practical nuclear fusion, with the potential to provide virtually limitless, clean energy. The theoretical and numerical modeling of tokamak plasmas is simultaneously an essential component of effective reactor design, and a great research barrier. Tokamak operational conditions exhibit comparatively low Knudsen numbers. Kinetic effects, including kinetic waves and instabilities, Landau damping, bump-on-tail instabilities and more, are therefore highly influential in tokamak plasma dynamics. Purely fluid models are inherently incapable of capturing these effects, whereas the high dimensionality in purely kinetic models render them practically intractable for most relevant purposes.

        We consider a $\delta\!f$ decomposition model, with a macroscopic fluid background and microscopic kinetic correction, both fully coupled to each other. A similar manner of discretization is proposed to that used in the recent \texttt{STRUPHY} code \cite{Holderied_Possanner_Wang_2021, Holderied_2022, Li_et_al_2023} with a finite-element model for the background and a pseudo-particle/PiC model for the correction.

        The fluid background satisfies the full, non-linear, resistive, compressible, Hall MHD equations. \cite{Laakmann_Hu_Farrell_2022} introduces finite-element(-in-space) implicit timesteppers for the incompressible analogue to this system with structure-preserving (SP) properties in the ideal case, alongside parameter-robust preconditioners. We show that these timesteppers can derive from a finite-element-in-time (FET) (and finite-element-in-space) interpretation. The benefits of this reformulation are discussed, including the derivation of timesteppers that are higher order in time, and the quantifiable dissipative SP properties in the non-ideal, resistive case.
        
        We discuss possible options for extending this FET approach to timesteppers for the compressible case.

        The kinetic corrections satisfy linearized Boltzmann equations. Using a Lénard--Bernstein collision operator, these take Fokker--Planck-like forms \cite{Fokker_1914, Planck_1917} wherein pseudo-particles in the numerical model obey the neoclassical transport equations, with particle-independent Brownian drift terms. This offers a rigorous methodology for incorporating collisions into the particle transport model, without coupling the equations of motions for each particle.
        
        Works by Chen, Chacón et al. \cite{Chen_Chacón_Barnes_2011, Chacón_Chen_Barnes_2013, Chen_Chacón_2014, Chen_Chacón_2015} have developed structure-preserving particle pushers for neoclassical transport in the Vlasov equations, derived from Crank--Nicolson integrators. We show these too can can derive from a FET interpretation, similarly offering potential extensions to higher-order-in-time particle pushers. The FET formulation is used also to consider how the stochastic drift terms can be incorporated into the pushers. Stochastic gyrokinetic expansions are also discussed.

        Different options for the numerical implementation of these schemes are considered.

        Due to the efficacy of FET in the development of SP timesteppers for both the fluid and kinetic component, we hope this approach will prove effective in the future for developing SP timesteppers for the full hybrid model. We hope this will give us the opportunity to incorporate previously inaccessible kinetic effects into the highly effective, modern, finite-element MHD models.
    \end{abstract}
    
    
    \newpage
    \tableofcontents
    
    
    \newpage
    \pagenumbering{arabic}
    %\linenumbers\renewcommand\thelinenumber{\color{black!50}\arabic{linenumber}}
            \input{0 - introduction/main.tex}
        \part{Research}
            \input{1 - low-noise PiC models/main.tex}
            \input{2 - kinetic component/main.tex}
            \input{3 - fluid component/main.tex}
            \input{4 - numerical implementation/main.tex}
        \part{Project Overview}
            \input{5 - research plan/main.tex}
            \input{6 - summary/main.tex}
    
    
    %\section{}
    \newpage
    \pagenumbering{gobble}
        \printbibliography


    \newpage
    \pagenumbering{roman}
    \appendix
        \part{Appendices}
            \input{8 - Hilbert complexes/main.tex}
            \input{9 - weak conservation proofs/main.tex}
\end{document}

\end{document}

        \part{Project Overview}
            \documentclass[12pt, a4paper]{report}

\documentclass[12pt, a4paper]{report}

\input{template/main.tex}

\title{\BA{Title in Progress...}}
\author{Boris Andrews}
\affil{Mathematical Institute, University of Oxford}
\date{\today}


\begin{document}
    \pagenumbering{gobble}
    \maketitle
    
    
    \begin{abstract}
        Magnetic confinement reactors---in particular tokamaks---offer one of the most promising options for achieving practical nuclear fusion, with the potential to provide virtually limitless, clean energy. The theoretical and numerical modeling of tokamak plasmas is simultaneously an essential component of effective reactor design, and a great research barrier. Tokamak operational conditions exhibit comparatively low Knudsen numbers. Kinetic effects, including kinetic waves and instabilities, Landau damping, bump-on-tail instabilities and more, are therefore highly influential in tokamak plasma dynamics. Purely fluid models are inherently incapable of capturing these effects, whereas the high dimensionality in purely kinetic models render them practically intractable for most relevant purposes.

        We consider a $\delta\!f$ decomposition model, with a macroscopic fluid background and microscopic kinetic correction, both fully coupled to each other. A similar manner of discretization is proposed to that used in the recent \texttt{STRUPHY} code \cite{Holderied_Possanner_Wang_2021, Holderied_2022, Li_et_al_2023} with a finite-element model for the background and a pseudo-particle/PiC model for the correction.

        The fluid background satisfies the full, non-linear, resistive, compressible, Hall MHD equations. \cite{Laakmann_Hu_Farrell_2022} introduces finite-element(-in-space) implicit timesteppers for the incompressible analogue to this system with structure-preserving (SP) properties in the ideal case, alongside parameter-robust preconditioners. We show that these timesteppers can derive from a finite-element-in-time (FET) (and finite-element-in-space) interpretation. The benefits of this reformulation are discussed, including the derivation of timesteppers that are higher order in time, and the quantifiable dissipative SP properties in the non-ideal, resistive case.
        
        We discuss possible options for extending this FET approach to timesteppers for the compressible case.

        The kinetic corrections satisfy linearized Boltzmann equations. Using a Lénard--Bernstein collision operator, these take Fokker--Planck-like forms \cite{Fokker_1914, Planck_1917} wherein pseudo-particles in the numerical model obey the neoclassical transport equations, with particle-independent Brownian drift terms. This offers a rigorous methodology for incorporating collisions into the particle transport model, without coupling the equations of motions for each particle.
        
        Works by Chen, Chacón et al. \cite{Chen_Chacón_Barnes_2011, Chacón_Chen_Barnes_2013, Chen_Chacón_2014, Chen_Chacón_2015} have developed structure-preserving particle pushers for neoclassical transport in the Vlasov equations, derived from Crank--Nicolson integrators. We show these too can can derive from a FET interpretation, similarly offering potential extensions to higher-order-in-time particle pushers. The FET formulation is used also to consider how the stochastic drift terms can be incorporated into the pushers. Stochastic gyrokinetic expansions are also discussed.

        Different options for the numerical implementation of these schemes are considered.

        Due to the efficacy of FET in the development of SP timesteppers for both the fluid and kinetic component, we hope this approach will prove effective in the future for developing SP timesteppers for the full hybrid model. We hope this will give us the opportunity to incorporate previously inaccessible kinetic effects into the highly effective, modern, finite-element MHD models.
    \end{abstract}
    
    
    \newpage
    \tableofcontents
    
    
    \newpage
    \pagenumbering{arabic}
    %\linenumbers\renewcommand\thelinenumber{\color{black!50}\arabic{linenumber}}
            \input{0 - introduction/main.tex}
        \part{Research}
            \input{1 - low-noise PiC models/main.tex}
            \input{2 - kinetic component/main.tex}
            \input{3 - fluid component/main.tex}
            \input{4 - numerical implementation/main.tex}
        \part{Project Overview}
            \input{5 - research plan/main.tex}
            \input{6 - summary/main.tex}
    
    
    %\section{}
    \newpage
    \pagenumbering{gobble}
        \printbibliography


    \newpage
    \pagenumbering{roman}
    \appendix
        \part{Appendices}
            \input{8 - Hilbert complexes/main.tex}
            \input{9 - weak conservation proofs/main.tex}
\end{document}


\title{\BA{Title in Progress...}}
\author{Boris Andrews}
\affil{Mathematical Institute, University of Oxford}
\date{\today}


\begin{document}
    \pagenumbering{gobble}
    \maketitle
    
    
    \begin{abstract}
        Magnetic confinement reactors---in particular tokamaks---offer one of the most promising options for achieving practical nuclear fusion, with the potential to provide virtually limitless, clean energy. The theoretical and numerical modeling of tokamak plasmas is simultaneously an essential component of effective reactor design, and a great research barrier. Tokamak operational conditions exhibit comparatively low Knudsen numbers. Kinetic effects, including kinetic waves and instabilities, Landau damping, bump-on-tail instabilities and more, are therefore highly influential in tokamak plasma dynamics. Purely fluid models are inherently incapable of capturing these effects, whereas the high dimensionality in purely kinetic models render them practically intractable for most relevant purposes.

        We consider a $\delta\!f$ decomposition model, with a macroscopic fluid background and microscopic kinetic correction, both fully coupled to each other. A similar manner of discretization is proposed to that used in the recent \texttt{STRUPHY} code \cite{Holderied_Possanner_Wang_2021, Holderied_2022, Li_et_al_2023} with a finite-element model for the background and a pseudo-particle/PiC model for the correction.

        The fluid background satisfies the full, non-linear, resistive, compressible, Hall MHD equations. \cite{Laakmann_Hu_Farrell_2022} introduces finite-element(-in-space) implicit timesteppers for the incompressible analogue to this system with structure-preserving (SP) properties in the ideal case, alongside parameter-robust preconditioners. We show that these timesteppers can derive from a finite-element-in-time (FET) (and finite-element-in-space) interpretation. The benefits of this reformulation are discussed, including the derivation of timesteppers that are higher order in time, and the quantifiable dissipative SP properties in the non-ideal, resistive case.
        
        We discuss possible options for extending this FET approach to timesteppers for the compressible case.

        The kinetic corrections satisfy linearized Boltzmann equations. Using a Lénard--Bernstein collision operator, these take Fokker--Planck-like forms \cite{Fokker_1914, Planck_1917} wherein pseudo-particles in the numerical model obey the neoclassical transport equations, with particle-independent Brownian drift terms. This offers a rigorous methodology for incorporating collisions into the particle transport model, without coupling the equations of motions for each particle.
        
        Works by Chen, Chacón et al. \cite{Chen_Chacón_Barnes_2011, Chacón_Chen_Barnes_2013, Chen_Chacón_2014, Chen_Chacón_2015} have developed structure-preserving particle pushers for neoclassical transport in the Vlasov equations, derived from Crank--Nicolson integrators. We show these too can can derive from a FET interpretation, similarly offering potential extensions to higher-order-in-time particle pushers. The FET formulation is used also to consider how the stochastic drift terms can be incorporated into the pushers. Stochastic gyrokinetic expansions are also discussed.

        Different options for the numerical implementation of these schemes are considered.

        Due to the efficacy of FET in the development of SP timesteppers for both the fluid and kinetic component, we hope this approach will prove effective in the future for developing SP timesteppers for the full hybrid model. We hope this will give us the opportunity to incorporate previously inaccessible kinetic effects into the highly effective, modern, finite-element MHD models.
    \end{abstract}
    
    
    \newpage
    \tableofcontents
    
    
    \newpage
    \pagenumbering{arabic}
    %\linenumbers\renewcommand\thelinenumber{\color{black!50}\arabic{linenumber}}
            \documentclass[12pt, a4paper]{report}

\input{template/main.tex}

\title{\BA{Title in Progress...}}
\author{Boris Andrews}
\affil{Mathematical Institute, University of Oxford}
\date{\today}


\begin{document}
    \pagenumbering{gobble}
    \maketitle
    
    
    \begin{abstract}
        Magnetic confinement reactors---in particular tokamaks---offer one of the most promising options for achieving practical nuclear fusion, with the potential to provide virtually limitless, clean energy. The theoretical and numerical modeling of tokamak plasmas is simultaneously an essential component of effective reactor design, and a great research barrier. Tokamak operational conditions exhibit comparatively low Knudsen numbers. Kinetic effects, including kinetic waves and instabilities, Landau damping, bump-on-tail instabilities and more, are therefore highly influential in tokamak plasma dynamics. Purely fluid models are inherently incapable of capturing these effects, whereas the high dimensionality in purely kinetic models render them practically intractable for most relevant purposes.

        We consider a $\delta\!f$ decomposition model, with a macroscopic fluid background and microscopic kinetic correction, both fully coupled to each other. A similar manner of discretization is proposed to that used in the recent \texttt{STRUPHY} code \cite{Holderied_Possanner_Wang_2021, Holderied_2022, Li_et_al_2023} with a finite-element model for the background and a pseudo-particle/PiC model for the correction.

        The fluid background satisfies the full, non-linear, resistive, compressible, Hall MHD equations. \cite{Laakmann_Hu_Farrell_2022} introduces finite-element(-in-space) implicit timesteppers for the incompressible analogue to this system with structure-preserving (SP) properties in the ideal case, alongside parameter-robust preconditioners. We show that these timesteppers can derive from a finite-element-in-time (FET) (and finite-element-in-space) interpretation. The benefits of this reformulation are discussed, including the derivation of timesteppers that are higher order in time, and the quantifiable dissipative SP properties in the non-ideal, resistive case.
        
        We discuss possible options for extending this FET approach to timesteppers for the compressible case.

        The kinetic corrections satisfy linearized Boltzmann equations. Using a Lénard--Bernstein collision operator, these take Fokker--Planck-like forms \cite{Fokker_1914, Planck_1917} wherein pseudo-particles in the numerical model obey the neoclassical transport equations, with particle-independent Brownian drift terms. This offers a rigorous methodology for incorporating collisions into the particle transport model, without coupling the equations of motions for each particle.
        
        Works by Chen, Chacón et al. \cite{Chen_Chacón_Barnes_2011, Chacón_Chen_Barnes_2013, Chen_Chacón_2014, Chen_Chacón_2015} have developed structure-preserving particle pushers for neoclassical transport in the Vlasov equations, derived from Crank--Nicolson integrators. We show these too can can derive from a FET interpretation, similarly offering potential extensions to higher-order-in-time particle pushers. The FET formulation is used also to consider how the stochastic drift terms can be incorporated into the pushers. Stochastic gyrokinetic expansions are also discussed.

        Different options for the numerical implementation of these schemes are considered.

        Due to the efficacy of FET in the development of SP timesteppers for both the fluid and kinetic component, we hope this approach will prove effective in the future for developing SP timesteppers for the full hybrid model. We hope this will give us the opportunity to incorporate previously inaccessible kinetic effects into the highly effective, modern, finite-element MHD models.
    \end{abstract}
    
    
    \newpage
    \tableofcontents
    
    
    \newpage
    \pagenumbering{arabic}
    %\linenumbers\renewcommand\thelinenumber{\color{black!50}\arabic{linenumber}}
            \input{0 - introduction/main.tex}
        \part{Research}
            \input{1 - low-noise PiC models/main.tex}
            \input{2 - kinetic component/main.tex}
            \input{3 - fluid component/main.tex}
            \input{4 - numerical implementation/main.tex}
        \part{Project Overview}
            \input{5 - research plan/main.tex}
            \input{6 - summary/main.tex}
    
    
    %\section{}
    \newpage
    \pagenumbering{gobble}
        \printbibliography


    \newpage
    \pagenumbering{roman}
    \appendix
        \part{Appendices}
            \input{8 - Hilbert complexes/main.tex}
            \input{9 - weak conservation proofs/main.tex}
\end{document}

        \part{Research}
            \documentclass[12pt, a4paper]{report}

\input{template/main.tex}

\title{\BA{Title in Progress...}}
\author{Boris Andrews}
\affil{Mathematical Institute, University of Oxford}
\date{\today}


\begin{document}
    \pagenumbering{gobble}
    \maketitle
    
    
    \begin{abstract}
        Magnetic confinement reactors---in particular tokamaks---offer one of the most promising options for achieving practical nuclear fusion, with the potential to provide virtually limitless, clean energy. The theoretical and numerical modeling of tokamak plasmas is simultaneously an essential component of effective reactor design, and a great research barrier. Tokamak operational conditions exhibit comparatively low Knudsen numbers. Kinetic effects, including kinetic waves and instabilities, Landau damping, bump-on-tail instabilities and more, are therefore highly influential in tokamak plasma dynamics. Purely fluid models are inherently incapable of capturing these effects, whereas the high dimensionality in purely kinetic models render them practically intractable for most relevant purposes.

        We consider a $\delta\!f$ decomposition model, with a macroscopic fluid background and microscopic kinetic correction, both fully coupled to each other. A similar manner of discretization is proposed to that used in the recent \texttt{STRUPHY} code \cite{Holderied_Possanner_Wang_2021, Holderied_2022, Li_et_al_2023} with a finite-element model for the background and a pseudo-particle/PiC model for the correction.

        The fluid background satisfies the full, non-linear, resistive, compressible, Hall MHD equations. \cite{Laakmann_Hu_Farrell_2022} introduces finite-element(-in-space) implicit timesteppers for the incompressible analogue to this system with structure-preserving (SP) properties in the ideal case, alongside parameter-robust preconditioners. We show that these timesteppers can derive from a finite-element-in-time (FET) (and finite-element-in-space) interpretation. The benefits of this reformulation are discussed, including the derivation of timesteppers that are higher order in time, and the quantifiable dissipative SP properties in the non-ideal, resistive case.
        
        We discuss possible options for extending this FET approach to timesteppers for the compressible case.

        The kinetic corrections satisfy linearized Boltzmann equations. Using a Lénard--Bernstein collision operator, these take Fokker--Planck-like forms \cite{Fokker_1914, Planck_1917} wherein pseudo-particles in the numerical model obey the neoclassical transport equations, with particle-independent Brownian drift terms. This offers a rigorous methodology for incorporating collisions into the particle transport model, without coupling the equations of motions for each particle.
        
        Works by Chen, Chacón et al. \cite{Chen_Chacón_Barnes_2011, Chacón_Chen_Barnes_2013, Chen_Chacón_2014, Chen_Chacón_2015} have developed structure-preserving particle pushers for neoclassical transport in the Vlasov equations, derived from Crank--Nicolson integrators. We show these too can can derive from a FET interpretation, similarly offering potential extensions to higher-order-in-time particle pushers. The FET formulation is used also to consider how the stochastic drift terms can be incorporated into the pushers. Stochastic gyrokinetic expansions are also discussed.

        Different options for the numerical implementation of these schemes are considered.

        Due to the efficacy of FET in the development of SP timesteppers for both the fluid and kinetic component, we hope this approach will prove effective in the future for developing SP timesteppers for the full hybrid model. We hope this will give us the opportunity to incorporate previously inaccessible kinetic effects into the highly effective, modern, finite-element MHD models.
    \end{abstract}
    
    
    \newpage
    \tableofcontents
    
    
    \newpage
    \pagenumbering{arabic}
    %\linenumbers\renewcommand\thelinenumber{\color{black!50}\arabic{linenumber}}
            \input{0 - introduction/main.tex}
        \part{Research}
            \input{1 - low-noise PiC models/main.tex}
            \input{2 - kinetic component/main.tex}
            \input{3 - fluid component/main.tex}
            \input{4 - numerical implementation/main.tex}
        \part{Project Overview}
            \input{5 - research plan/main.tex}
            \input{6 - summary/main.tex}
    
    
    %\section{}
    \newpage
    \pagenumbering{gobble}
        \printbibliography


    \newpage
    \pagenumbering{roman}
    \appendix
        \part{Appendices}
            \input{8 - Hilbert complexes/main.tex}
            \input{9 - weak conservation proofs/main.tex}
\end{document}

            \documentclass[12pt, a4paper]{report}

\input{template/main.tex}

\title{\BA{Title in Progress...}}
\author{Boris Andrews}
\affil{Mathematical Institute, University of Oxford}
\date{\today}


\begin{document}
    \pagenumbering{gobble}
    \maketitle
    
    
    \begin{abstract}
        Magnetic confinement reactors---in particular tokamaks---offer one of the most promising options for achieving practical nuclear fusion, with the potential to provide virtually limitless, clean energy. The theoretical and numerical modeling of tokamak plasmas is simultaneously an essential component of effective reactor design, and a great research barrier. Tokamak operational conditions exhibit comparatively low Knudsen numbers. Kinetic effects, including kinetic waves and instabilities, Landau damping, bump-on-tail instabilities and more, are therefore highly influential in tokamak plasma dynamics. Purely fluid models are inherently incapable of capturing these effects, whereas the high dimensionality in purely kinetic models render them practically intractable for most relevant purposes.

        We consider a $\delta\!f$ decomposition model, with a macroscopic fluid background and microscopic kinetic correction, both fully coupled to each other. A similar manner of discretization is proposed to that used in the recent \texttt{STRUPHY} code \cite{Holderied_Possanner_Wang_2021, Holderied_2022, Li_et_al_2023} with a finite-element model for the background and a pseudo-particle/PiC model for the correction.

        The fluid background satisfies the full, non-linear, resistive, compressible, Hall MHD equations. \cite{Laakmann_Hu_Farrell_2022} introduces finite-element(-in-space) implicit timesteppers for the incompressible analogue to this system with structure-preserving (SP) properties in the ideal case, alongside parameter-robust preconditioners. We show that these timesteppers can derive from a finite-element-in-time (FET) (and finite-element-in-space) interpretation. The benefits of this reformulation are discussed, including the derivation of timesteppers that are higher order in time, and the quantifiable dissipative SP properties in the non-ideal, resistive case.
        
        We discuss possible options for extending this FET approach to timesteppers for the compressible case.

        The kinetic corrections satisfy linearized Boltzmann equations. Using a Lénard--Bernstein collision operator, these take Fokker--Planck-like forms \cite{Fokker_1914, Planck_1917} wherein pseudo-particles in the numerical model obey the neoclassical transport equations, with particle-independent Brownian drift terms. This offers a rigorous methodology for incorporating collisions into the particle transport model, without coupling the equations of motions for each particle.
        
        Works by Chen, Chacón et al. \cite{Chen_Chacón_Barnes_2011, Chacón_Chen_Barnes_2013, Chen_Chacón_2014, Chen_Chacón_2015} have developed structure-preserving particle pushers for neoclassical transport in the Vlasov equations, derived from Crank--Nicolson integrators. We show these too can can derive from a FET interpretation, similarly offering potential extensions to higher-order-in-time particle pushers. The FET formulation is used also to consider how the stochastic drift terms can be incorporated into the pushers. Stochastic gyrokinetic expansions are also discussed.

        Different options for the numerical implementation of these schemes are considered.

        Due to the efficacy of FET in the development of SP timesteppers for both the fluid and kinetic component, we hope this approach will prove effective in the future for developing SP timesteppers for the full hybrid model. We hope this will give us the opportunity to incorporate previously inaccessible kinetic effects into the highly effective, modern, finite-element MHD models.
    \end{abstract}
    
    
    \newpage
    \tableofcontents
    
    
    \newpage
    \pagenumbering{arabic}
    %\linenumbers\renewcommand\thelinenumber{\color{black!50}\arabic{linenumber}}
            \input{0 - introduction/main.tex}
        \part{Research}
            \input{1 - low-noise PiC models/main.tex}
            \input{2 - kinetic component/main.tex}
            \input{3 - fluid component/main.tex}
            \input{4 - numerical implementation/main.tex}
        \part{Project Overview}
            \input{5 - research plan/main.tex}
            \input{6 - summary/main.tex}
    
    
    %\section{}
    \newpage
    \pagenumbering{gobble}
        \printbibliography


    \newpage
    \pagenumbering{roman}
    \appendix
        \part{Appendices}
            \input{8 - Hilbert complexes/main.tex}
            \input{9 - weak conservation proofs/main.tex}
\end{document}

            \documentclass[12pt, a4paper]{report}

\input{template/main.tex}

\title{\BA{Title in Progress...}}
\author{Boris Andrews}
\affil{Mathematical Institute, University of Oxford}
\date{\today}


\begin{document}
    \pagenumbering{gobble}
    \maketitle
    
    
    \begin{abstract}
        Magnetic confinement reactors---in particular tokamaks---offer one of the most promising options for achieving practical nuclear fusion, with the potential to provide virtually limitless, clean energy. The theoretical and numerical modeling of tokamak plasmas is simultaneously an essential component of effective reactor design, and a great research barrier. Tokamak operational conditions exhibit comparatively low Knudsen numbers. Kinetic effects, including kinetic waves and instabilities, Landau damping, bump-on-tail instabilities and more, are therefore highly influential in tokamak plasma dynamics. Purely fluid models are inherently incapable of capturing these effects, whereas the high dimensionality in purely kinetic models render them practically intractable for most relevant purposes.

        We consider a $\delta\!f$ decomposition model, with a macroscopic fluid background and microscopic kinetic correction, both fully coupled to each other. A similar manner of discretization is proposed to that used in the recent \texttt{STRUPHY} code \cite{Holderied_Possanner_Wang_2021, Holderied_2022, Li_et_al_2023} with a finite-element model for the background and a pseudo-particle/PiC model for the correction.

        The fluid background satisfies the full, non-linear, resistive, compressible, Hall MHD equations. \cite{Laakmann_Hu_Farrell_2022} introduces finite-element(-in-space) implicit timesteppers for the incompressible analogue to this system with structure-preserving (SP) properties in the ideal case, alongside parameter-robust preconditioners. We show that these timesteppers can derive from a finite-element-in-time (FET) (and finite-element-in-space) interpretation. The benefits of this reformulation are discussed, including the derivation of timesteppers that are higher order in time, and the quantifiable dissipative SP properties in the non-ideal, resistive case.
        
        We discuss possible options for extending this FET approach to timesteppers for the compressible case.

        The kinetic corrections satisfy linearized Boltzmann equations. Using a Lénard--Bernstein collision operator, these take Fokker--Planck-like forms \cite{Fokker_1914, Planck_1917} wherein pseudo-particles in the numerical model obey the neoclassical transport equations, with particle-independent Brownian drift terms. This offers a rigorous methodology for incorporating collisions into the particle transport model, without coupling the equations of motions for each particle.
        
        Works by Chen, Chacón et al. \cite{Chen_Chacón_Barnes_2011, Chacón_Chen_Barnes_2013, Chen_Chacón_2014, Chen_Chacón_2015} have developed structure-preserving particle pushers for neoclassical transport in the Vlasov equations, derived from Crank--Nicolson integrators. We show these too can can derive from a FET interpretation, similarly offering potential extensions to higher-order-in-time particle pushers. The FET formulation is used also to consider how the stochastic drift terms can be incorporated into the pushers. Stochastic gyrokinetic expansions are also discussed.

        Different options for the numerical implementation of these schemes are considered.

        Due to the efficacy of FET in the development of SP timesteppers for both the fluid and kinetic component, we hope this approach will prove effective in the future for developing SP timesteppers for the full hybrid model. We hope this will give us the opportunity to incorporate previously inaccessible kinetic effects into the highly effective, modern, finite-element MHD models.
    \end{abstract}
    
    
    \newpage
    \tableofcontents
    
    
    \newpage
    \pagenumbering{arabic}
    %\linenumbers\renewcommand\thelinenumber{\color{black!50}\arabic{linenumber}}
            \input{0 - introduction/main.tex}
        \part{Research}
            \input{1 - low-noise PiC models/main.tex}
            \input{2 - kinetic component/main.tex}
            \input{3 - fluid component/main.tex}
            \input{4 - numerical implementation/main.tex}
        \part{Project Overview}
            \input{5 - research plan/main.tex}
            \input{6 - summary/main.tex}
    
    
    %\section{}
    \newpage
    \pagenumbering{gobble}
        \printbibliography


    \newpage
    \pagenumbering{roman}
    \appendix
        \part{Appendices}
            \input{8 - Hilbert complexes/main.tex}
            \input{9 - weak conservation proofs/main.tex}
\end{document}

            \documentclass[12pt, a4paper]{report}

\input{template/main.tex}

\title{\BA{Title in Progress...}}
\author{Boris Andrews}
\affil{Mathematical Institute, University of Oxford}
\date{\today}


\begin{document}
    \pagenumbering{gobble}
    \maketitle
    
    
    \begin{abstract}
        Magnetic confinement reactors---in particular tokamaks---offer one of the most promising options for achieving practical nuclear fusion, with the potential to provide virtually limitless, clean energy. The theoretical and numerical modeling of tokamak plasmas is simultaneously an essential component of effective reactor design, and a great research barrier. Tokamak operational conditions exhibit comparatively low Knudsen numbers. Kinetic effects, including kinetic waves and instabilities, Landau damping, bump-on-tail instabilities and more, are therefore highly influential in tokamak plasma dynamics. Purely fluid models are inherently incapable of capturing these effects, whereas the high dimensionality in purely kinetic models render them practically intractable for most relevant purposes.

        We consider a $\delta\!f$ decomposition model, with a macroscopic fluid background and microscopic kinetic correction, both fully coupled to each other. A similar manner of discretization is proposed to that used in the recent \texttt{STRUPHY} code \cite{Holderied_Possanner_Wang_2021, Holderied_2022, Li_et_al_2023} with a finite-element model for the background and a pseudo-particle/PiC model for the correction.

        The fluid background satisfies the full, non-linear, resistive, compressible, Hall MHD equations. \cite{Laakmann_Hu_Farrell_2022} introduces finite-element(-in-space) implicit timesteppers for the incompressible analogue to this system with structure-preserving (SP) properties in the ideal case, alongside parameter-robust preconditioners. We show that these timesteppers can derive from a finite-element-in-time (FET) (and finite-element-in-space) interpretation. The benefits of this reformulation are discussed, including the derivation of timesteppers that are higher order in time, and the quantifiable dissipative SP properties in the non-ideal, resistive case.
        
        We discuss possible options for extending this FET approach to timesteppers for the compressible case.

        The kinetic corrections satisfy linearized Boltzmann equations. Using a Lénard--Bernstein collision operator, these take Fokker--Planck-like forms \cite{Fokker_1914, Planck_1917} wherein pseudo-particles in the numerical model obey the neoclassical transport equations, with particle-independent Brownian drift terms. This offers a rigorous methodology for incorporating collisions into the particle transport model, without coupling the equations of motions for each particle.
        
        Works by Chen, Chacón et al. \cite{Chen_Chacón_Barnes_2011, Chacón_Chen_Barnes_2013, Chen_Chacón_2014, Chen_Chacón_2015} have developed structure-preserving particle pushers for neoclassical transport in the Vlasov equations, derived from Crank--Nicolson integrators. We show these too can can derive from a FET interpretation, similarly offering potential extensions to higher-order-in-time particle pushers. The FET formulation is used also to consider how the stochastic drift terms can be incorporated into the pushers. Stochastic gyrokinetic expansions are also discussed.

        Different options for the numerical implementation of these schemes are considered.

        Due to the efficacy of FET in the development of SP timesteppers for both the fluid and kinetic component, we hope this approach will prove effective in the future for developing SP timesteppers for the full hybrid model. We hope this will give us the opportunity to incorporate previously inaccessible kinetic effects into the highly effective, modern, finite-element MHD models.
    \end{abstract}
    
    
    \newpage
    \tableofcontents
    
    
    \newpage
    \pagenumbering{arabic}
    %\linenumbers\renewcommand\thelinenumber{\color{black!50}\arabic{linenumber}}
            \input{0 - introduction/main.tex}
        \part{Research}
            \input{1 - low-noise PiC models/main.tex}
            \input{2 - kinetic component/main.tex}
            \input{3 - fluid component/main.tex}
            \input{4 - numerical implementation/main.tex}
        \part{Project Overview}
            \input{5 - research plan/main.tex}
            \input{6 - summary/main.tex}
    
    
    %\section{}
    \newpage
    \pagenumbering{gobble}
        \printbibliography


    \newpage
    \pagenumbering{roman}
    \appendix
        \part{Appendices}
            \input{8 - Hilbert complexes/main.tex}
            \input{9 - weak conservation proofs/main.tex}
\end{document}

        \part{Project Overview}
            \documentclass[12pt, a4paper]{report}

\input{template/main.tex}

\title{\BA{Title in Progress...}}
\author{Boris Andrews}
\affil{Mathematical Institute, University of Oxford}
\date{\today}


\begin{document}
    \pagenumbering{gobble}
    \maketitle
    
    
    \begin{abstract}
        Magnetic confinement reactors---in particular tokamaks---offer one of the most promising options for achieving practical nuclear fusion, with the potential to provide virtually limitless, clean energy. The theoretical and numerical modeling of tokamak plasmas is simultaneously an essential component of effective reactor design, and a great research barrier. Tokamak operational conditions exhibit comparatively low Knudsen numbers. Kinetic effects, including kinetic waves and instabilities, Landau damping, bump-on-tail instabilities and more, are therefore highly influential in tokamak plasma dynamics. Purely fluid models are inherently incapable of capturing these effects, whereas the high dimensionality in purely kinetic models render them practically intractable for most relevant purposes.

        We consider a $\delta\!f$ decomposition model, with a macroscopic fluid background and microscopic kinetic correction, both fully coupled to each other. A similar manner of discretization is proposed to that used in the recent \texttt{STRUPHY} code \cite{Holderied_Possanner_Wang_2021, Holderied_2022, Li_et_al_2023} with a finite-element model for the background and a pseudo-particle/PiC model for the correction.

        The fluid background satisfies the full, non-linear, resistive, compressible, Hall MHD equations. \cite{Laakmann_Hu_Farrell_2022} introduces finite-element(-in-space) implicit timesteppers for the incompressible analogue to this system with structure-preserving (SP) properties in the ideal case, alongside parameter-robust preconditioners. We show that these timesteppers can derive from a finite-element-in-time (FET) (and finite-element-in-space) interpretation. The benefits of this reformulation are discussed, including the derivation of timesteppers that are higher order in time, and the quantifiable dissipative SP properties in the non-ideal, resistive case.
        
        We discuss possible options for extending this FET approach to timesteppers for the compressible case.

        The kinetic corrections satisfy linearized Boltzmann equations. Using a Lénard--Bernstein collision operator, these take Fokker--Planck-like forms \cite{Fokker_1914, Planck_1917} wherein pseudo-particles in the numerical model obey the neoclassical transport equations, with particle-independent Brownian drift terms. This offers a rigorous methodology for incorporating collisions into the particle transport model, without coupling the equations of motions for each particle.
        
        Works by Chen, Chacón et al. \cite{Chen_Chacón_Barnes_2011, Chacón_Chen_Barnes_2013, Chen_Chacón_2014, Chen_Chacón_2015} have developed structure-preserving particle pushers for neoclassical transport in the Vlasov equations, derived from Crank--Nicolson integrators. We show these too can can derive from a FET interpretation, similarly offering potential extensions to higher-order-in-time particle pushers. The FET formulation is used also to consider how the stochastic drift terms can be incorporated into the pushers. Stochastic gyrokinetic expansions are also discussed.

        Different options for the numerical implementation of these schemes are considered.

        Due to the efficacy of FET in the development of SP timesteppers for both the fluid and kinetic component, we hope this approach will prove effective in the future for developing SP timesteppers for the full hybrid model. We hope this will give us the opportunity to incorporate previously inaccessible kinetic effects into the highly effective, modern, finite-element MHD models.
    \end{abstract}
    
    
    \newpage
    \tableofcontents
    
    
    \newpage
    \pagenumbering{arabic}
    %\linenumbers\renewcommand\thelinenumber{\color{black!50}\arabic{linenumber}}
            \input{0 - introduction/main.tex}
        \part{Research}
            \input{1 - low-noise PiC models/main.tex}
            \input{2 - kinetic component/main.tex}
            \input{3 - fluid component/main.tex}
            \input{4 - numerical implementation/main.tex}
        \part{Project Overview}
            \input{5 - research plan/main.tex}
            \input{6 - summary/main.tex}
    
    
    %\section{}
    \newpage
    \pagenumbering{gobble}
        \printbibliography


    \newpage
    \pagenumbering{roman}
    \appendix
        \part{Appendices}
            \input{8 - Hilbert complexes/main.tex}
            \input{9 - weak conservation proofs/main.tex}
\end{document}

            \documentclass[12pt, a4paper]{report}

\input{template/main.tex}

\title{\BA{Title in Progress...}}
\author{Boris Andrews}
\affil{Mathematical Institute, University of Oxford}
\date{\today}


\begin{document}
    \pagenumbering{gobble}
    \maketitle
    
    
    \begin{abstract}
        Magnetic confinement reactors---in particular tokamaks---offer one of the most promising options for achieving practical nuclear fusion, with the potential to provide virtually limitless, clean energy. The theoretical and numerical modeling of tokamak plasmas is simultaneously an essential component of effective reactor design, and a great research barrier. Tokamak operational conditions exhibit comparatively low Knudsen numbers. Kinetic effects, including kinetic waves and instabilities, Landau damping, bump-on-tail instabilities and more, are therefore highly influential in tokamak plasma dynamics. Purely fluid models are inherently incapable of capturing these effects, whereas the high dimensionality in purely kinetic models render them practically intractable for most relevant purposes.

        We consider a $\delta\!f$ decomposition model, with a macroscopic fluid background and microscopic kinetic correction, both fully coupled to each other. A similar manner of discretization is proposed to that used in the recent \texttt{STRUPHY} code \cite{Holderied_Possanner_Wang_2021, Holderied_2022, Li_et_al_2023} with a finite-element model for the background and a pseudo-particle/PiC model for the correction.

        The fluid background satisfies the full, non-linear, resistive, compressible, Hall MHD equations. \cite{Laakmann_Hu_Farrell_2022} introduces finite-element(-in-space) implicit timesteppers for the incompressible analogue to this system with structure-preserving (SP) properties in the ideal case, alongside parameter-robust preconditioners. We show that these timesteppers can derive from a finite-element-in-time (FET) (and finite-element-in-space) interpretation. The benefits of this reformulation are discussed, including the derivation of timesteppers that are higher order in time, and the quantifiable dissipative SP properties in the non-ideal, resistive case.
        
        We discuss possible options for extending this FET approach to timesteppers for the compressible case.

        The kinetic corrections satisfy linearized Boltzmann equations. Using a Lénard--Bernstein collision operator, these take Fokker--Planck-like forms \cite{Fokker_1914, Planck_1917} wherein pseudo-particles in the numerical model obey the neoclassical transport equations, with particle-independent Brownian drift terms. This offers a rigorous methodology for incorporating collisions into the particle transport model, without coupling the equations of motions for each particle.
        
        Works by Chen, Chacón et al. \cite{Chen_Chacón_Barnes_2011, Chacón_Chen_Barnes_2013, Chen_Chacón_2014, Chen_Chacón_2015} have developed structure-preserving particle pushers for neoclassical transport in the Vlasov equations, derived from Crank--Nicolson integrators. We show these too can can derive from a FET interpretation, similarly offering potential extensions to higher-order-in-time particle pushers. The FET formulation is used also to consider how the stochastic drift terms can be incorporated into the pushers. Stochastic gyrokinetic expansions are also discussed.

        Different options for the numerical implementation of these schemes are considered.

        Due to the efficacy of FET in the development of SP timesteppers for both the fluid and kinetic component, we hope this approach will prove effective in the future for developing SP timesteppers for the full hybrid model. We hope this will give us the opportunity to incorporate previously inaccessible kinetic effects into the highly effective, modern, finite-element MHD models.
    \end{abstract}
    
    
    \newpage
    \tableofcontents
    
    
    \newpage
    \pagenumbering{arabic}
    %\linenumbers\renewcommand\thelinenumber{\color{black!50}\arabic{linenumber}}
            \input{0 - introduction/main.tex}
        \part{Research}
            \input{1 - low-noise PiC models/main.tex}
            \input{2 - kinetic component/main.tex}
            \input{3 - fluid component/main.tex}
            \input{4 - numerical implementation/main.tex}
        \part{Project Overview}
            \input{5 - research plan/main.tex}
            \input{6 - summary/main.tex}
    
    
    %\section{}
    \newpage
    \pagenumbering{gobble}
        \printbibliography


    \newpage
    \pagenumbering{roman}
    \appendix
        \part{Appendices}
            \input{8 - Hilbert complexes/main.tex}
            \input{9 - weak conservation proofs/main.tex}
\end{document}

    
    
    %\section{}
    \newpage
    \pagenumbering{gobble}
        \printbibliography


    \newpage
    \pagenumbering{roman}
    \appendix
        \part{Appendices}
            \documentclass[12pt, a4paper]{report}

\input{template/main.tex}

\title{\BA{Title in Progress...}}
\author{Boris Andrews}
\affil{Mathematical Institute, University of Oxford}
\date{\today}


\begin{document}
    \pagenumbering{gobble}
    \maketitle
    
    
    \begin{abstract}
        Magnetic confinement reactors---in particular tokamaks---offer one of the most promising options for achieving practical nuclear fusion, with the potential to provide virtually limitless, clean energy. The theoretical and numerical modeling of tokamak plasmas is simultaneously an essential component of effective reactor design, and a great research barrier. Tokamak operational conditions exhibit comparatively low Knudsen numbers. Kinetic effects, including kinetic waves and instabilities, Landau damping, bump-on-tail instabilities and more, are therefore highly influential in tokamak plasma dynamics. Purely fluid models are inherently incapable of capturing these effects, whereas the high dimensionality in purely kinetic models render them practically intractable for most relevant purposes.

        We consider a $\delta\!f$ decomposition model, with a macroscopic fluid background and microscopic kinetic correction, both fully coupled to each other. A similar manner of discretization is proposed to that used in the recent \texttt{STRUPHY} code \cite{Holderied_Possanner_Wang_2021, Holderied_2022, Li_et_al_2023} with a finite-element model for the background and a pseudo-particle/PiC model for the correction.

        The fluid background satisfies the full, non-linear, resistive, compressible, Hall MHD equations. \cite{Laakmann_Hu_Farrell_2022} introduces finite-element(-in-space) implicit timesteppers for the incompressible analogue to this system with structure-preserving (SP) properties in the ideal case, alongside parameter-robust preconditioners. We show that these timesteppers can derive from a finite-element-in-time (FET) (and finite-element-in-space) interpretation. The benefits of this reformulation are discussed, including the derivation of timesteppers that are higher order in time, and the quantifiable dissipative SP properties in the non-ideal, resistive case.
        
        We discuss possible options for extending this FET approach to timesteppers for the compressible case.

        The kinetic corrections satisfy linearized Boltzmann equations. Using a Lénard--Bernstein collision operator, these take Fokker--Planck-like forms \cite{Fokker_1914, Planck_1917} wherein pseudo-particles in the numerical model obey the neoclassical transport equations, with particle-independent Brownian drift terms. This offers a rigorous methodology for incorporating collisions into the particle transport model, without coupling the equations of motions for each particle.
        
        Works by Chen, Chacón et al. \cite{Chen_Chacón_Barnes_2011, Chacón_Chen_Barnes_2013, Chen_Chacón_2014, Chen_Chacón_2015} have developed structure-preserving particle pushers for neoclassical transport in the Vlasov equations, derived from Crank--Nicolson integrators. We show these too can can derive from a FET interpretation, similarly offering potential extensions to higher-order-in-time particle pushers. The FET formulation is used also to consider how the stochastic drift terms can be incorporated into the pushers. Stochastic gyrokinetic expansions are also discussed.

        Different options for the numerical implementation of these schemes are considered.

        Due to the efficacy of FET in the development of SP timesteppers for both the fluid and kinetic component, we hope this approach will prove effective in the future for developing SP timesteppers for the full hybrid model. We hope this will give us the opportunity to incorporate previously inaccessible kinetic effects into the highly effective, modern, finite-element MHD models.
    \end{abstract}
    
    
    \newpage
    \tableofcontents
    
    
    \newpage
    \pagenumbering{arabic}
    %\linenumbers\renewcommand\thelinenumber{\color{black!50}\arabic{linenumber}}
            \input{0 - introduction/main.tex}
        \part{Research}
            \input{1 - low-noise PiC models/main.tex}
            \input{2 - kinetic component/main.tex}
            \input{3 - fluid component/main.tex}
            \input{4 - numerical implementation/main.tex}
        \part{Project Overview}
            \input{5 - research plan/main.tex}
            \input{6 - summary/main.tex}
    
    
    %\section{}
    \newpage
    \pagenumbering{gobble}
        \printbibliography


    \newpage
    \pagenumbering{roman}
    \appendix
        \part{Appendices}
            \input{8 - Hilbert complexes/main.tex}
            \input{9 - weak conservation proofs/main.tex}
\end{document}

            \documentclass[12pt, a4paper]{report}

\input{template/main.tex}

\title{\BA{Title in Progress...}}
\author{Boris Andrews}
\affil{Mathematical Institute, University of Oxford}
\date{\today}


\begin{document}
    \pagenumbering{gobble}
    \maketitle
    
    
    \begin{abstract}
        Magnetic confinement reactors---in particular tokamaks---offer one of the most promising options for achieving practical nuclear fusion, with the potential to provide virtually limitless, clean energy. The theoretical and numerical modeling of tokamak plasmas is simultaneously an essential component of effective reactor design, and a great research barrier. Tokamak operational conditions exhibit comparatively low Knudsen numbers. Kinetic effects, including kinetic waves and instabilities, Landau damping, bump-on-tail instabilities and more, are therefore highly influential in tokamak plasma dynamics. Purely fluid models are inherently incapable of capturing these effects, whereas the high dimensionality in purely kinetic models render them practically intractable for most relevant purposes.

        We consider a $\delta\!f$ decomposition model, with a macroscopic fluid background and microscopic kinetic correction, both fully coupled to each other. A similar manner of discretization is proposed to that used in the recent \texttt{STRUPHY} code \cite{Holderied_Possanner_Wang_2021, Holderied_2022, Li_et_al_2023} with a finite-element model for the background and a pseudo-particle/PiC model for the correction.

        The fluid background satisfies the full, non-linear, resistive, compressible, Hall MHD equations. \cite{Laakmann_Hu_Farrell_2022} introduces finite-element(-in-space) implicit timesteppers for the incompressible analogue to this system with structure-preserving (SP) properties in the ideal case, alongside parameter-robust preconditioners. We show that these timesteppers can derive from a finite-element-in-time (FET) (and finite-element-in-space) interpretation. The benefits of this reformulation are discussed, including the derivation of timesteppers that are higher order in time, and the quantifiable dissipative SP properties in the non-ideal, resistive case.
        
        We discuss possible options for extending this FET approach to timesteppers for the compressible case.

        The kinetic corrections satisfy linearized Boltzmann equations. Using a Lénard--Bernstein collision operator, these take Fokker--Planck-like forms \cite{Fokker_1914, Planck_1917} wherein pseudo-particles in the numerical model obey the neoclassical transport equations, with particle-independent Brownian drift terms. This offers a rigorous methodology for incorporating collisions into the particle transport model, without coupling the equations of motions for each particle.
        
        Works by Chen, Chacón et al. \cite{Chen_Chacón_Barnes_2011, Chacón_Chen_Barnes_2013, Chen_Chacón_2014, Chen_Chacón_2015} have developed structure-preserving particle pushers for neoclassical transport in the Vlasov equations, derived from Crank--Nicolson integrators. We show these too can can derive from a FET interpretation, similarly offering potential extensions to higher-order-in-time particle pushers. The FET formulation is used also to consider how the stochastic drift terms can be incorporated into the pushers. Stochastic gyrokinetic expansions are also discussed.

        Different options for the numerical implementation of these schemes are considered.

        Due to the efficacy of FET in the development of SP timesteppers for both the fluid and kinetic component, we hope this approach will prove effective in the future for developing SP timesteppers for the full hybrid model. We hope this will give us the opportunity to incorporate previously inaccessible kinetic effects into the highly effective, modern, finite-element MHD models.
    \end{abstract}
    
    
    \newpage
    \tableofcontents
    
    
    \newpage
    \pagenumbering{arabic}
    %\linenumbers\renewcommand\thelinenumber{\color{black!50}\arabic{linenumber}}
            \input{0 - introduction/main.tex}
        \part{Research}
            \input{1 - low-noise PiC models/main.tex}
            \input{2 - kinetic component/main.tex}
            \input{3 - fluid component/main.tex}
            \input{4 - numerical implementation/main.tex}
        \part{Project Overview}
            \input{5 - research plan/main.tex}
            \input{6 - summary/main.tex}
    
    
    %\section{}
    \newpage
    \pagenumbering{gobble}
        \printbibliography


    \newpage
    \pagenumbering{roman}
    \appendix
        \part{Appendices}
            \input{8 - Hilbert complexes/main.tex}
            \input{9 - weak conservation proofs/main.tex}
\end{document}

\end{document}

            \documentclass[12pt, a4paper]{report}

\documentclass[12pt, a4paper]{report}

\input{template/main.tex}

\title{\BA{Title in Progress...}}
\author{Boris Andrews}
\affil{Mathematical Institute, University of Oxford}
\date{\today}


\begin{document}
    \pagenumbering{gobble}
    \maketitle
    
    
    \begin{abstract}
        Magnetic confinement reactors---in particular tokamaks---offer one of the most promising options for achieving practical nuclear fusion, with the potential to provide virtually limitless, clean energy. The theoretical and numerical modeling of tokamak plasmas is simultaneously an essential component of effective reactor design, and a great research barrier. Tokamak operational conditions exhibit comparatively low Knudsen numbers. Kinetic effects, including kinetic waves and instabilities, Landau damping, bump-on-tail instabilities and more, are therefore highly influential in tokamak plasma dynamics. Purely fluid models are inherently incapable of capturing these effects, whereas the high dimensionality in purely kinetic models render them practically intractable for most relevant purposes.

        We consider a $\delta\!f$ decomposition model, with a macroscopic fluid background and microscopic kinetic correction, both fully coupled to each other. A similar manner of discretization is proposed to that used in the recent \texttt{STRUPHY} code \cite{Holderied_Possanner_Wang_2021, Holderied_2022, Li_et_al_2023} with a finite-element model for the background and a pseudo-particle/PiC model for the correction.

        The fluid background satisfies the full, non-linear, resistive, compressible, Hall MHD equations. \cite{Laakmann_Hu_Farrell_2022} introduces finite-element(-in-space) implicit timesteppers for the incompressible analogue to this system with structure-preserving (SP) properties in the ideal case, alongside parameter-robust preconditioners. We show that these timesteppers can derive from a finite-element-in-time (FET) (and finite-element-in-space) interpretation. The benefits of this reformulation are discussed, including the derivation of timesteppers that are higher order in time, and the quantifiable dissipative SP properties in the non-ideal, resistive case.
        
        We discuss possible options for extending this FET approach to timesteppers for the compressible case.

        The kinetic corrections satisfy linearized Boltzmann equations. Using a Lénard--Bernstein collision operator, these take Fokker--Planck-like forms \cite{Fokker_1914, Planck_1917} wherein pseudo-particles in the numerical model obey the neoclassical transport equations, with particle-independent Brownian drift terms. This offers a rigorous methodology for incorporating collisions into the particle transport model, without coupling the equations of motions for each particle.
        
        Works by Chen, Chacón et al. \cite{Chen_Chacón_Barnes_2011, Chacón_Chen_Barnes_2013, Chen_Chacón_2014, Chen_Chacón_2015} have developed structure-preserving particle pushers for neoclassical transport in the Vlasov equations, derived from Crank--Nicolson integrators. We show these too can can derive from a FET interpretation, similarly offering potential extensions to higher-order-in-time particle pushers. The FET formulation is used also to consider how the stochastic drift terms can be incorporated into the pushers. Stochastic gyrokinetic expansions are also discussed.

        Different options for the numerical implementation of these schemes are considered.

        Due to the efficacy of FET in the development of SP timesteppers for both the fluid and kinetic component, we hope this approach will prove effective in the future for developing SP timesteppers for the full hybrid model. We hope this will give us the opportunity to incorporate previously inaccessible kinetic effects into the highly effective, modern, finite-element MHD models.
    \end{abstract}
    
    
    \newpage
    \tableofcontents
    
    
    \newpage
    \pagenumbering{arabic}
    %\linenumbers\renewcommand\thelinenumber{\color{black!50}\arabic{linenumber}}
            \input{0 - introduction/main.tex}
        \part{Research}
            \input{1 - low-noise PiC models/main.tex}
            \input{2 - kinetic component/main.tex}
            \input{3 - fluid component/main.tex}
            \input{4 - numerical implementation/main.tex}
        \part{Project Overview}
            \input{5 - research plan/main.tex}
            \input{6 - summary/main.tex}
    
    
    %\section{}
    \newpage
    \pagenumbering{gobble}
        \printbibliography


    \newpage
    \pagenumbering{roman}
    \appendix
        \part{Appendices}
            \input{8 - Hilbert complexes/main.tex}
            \input{9 - weak conservation proofs/main.tex}
\end{document}


\title{\BA{Title in Progress...}}
\author{Boris Andrews}
\affil{Mathematical Institute, University of Oxford}
\date{\today}


\begin{document}
    \pagenumbering{gobble}
    \maketitle
    
    
    \begin{abstract}
        Magnetic confinement reactors---in particular tokamaks---offer one of the most promising options for achieving practical nuclear fusion, with the potential to provide virtually limitless, clean energy. The theoretical and numerical modeling of tokamak plasmas is simultaneously an essential component of effective reactor design, and a great research barrier. Tokamak operational conditions exhibit comparatively low Knudsen numbers. Kinetic effects, including kinetic waves and instabilities, Landau damping, bump-on-tail instabilities and more, are therefore highly influential in tokamak plasma dynamics. Purely fluid models are inherently incapable of capturing these effects, whereas the high dimensionality in purely kinetic models render them practically intractable for most relevant purposes.

        We consider a $\delta\!f$ decomposition model, with a macroscopic fluid background and microscopic kinetic correction, both fully coupled to each other. A similar manner of discretization is proposed to that used in the recent \texttt{STRUPHY} code \cite{Holderied_Possanner_Wang_2021, Holderied_2022, Li_et_al_2023} with a finite-element model for the background and a pseudo-particle/PiC model for the correction.

        The fluid background satisfies the full, non-linear, resistive, compressible, Hall MHD equations. \cite{Laakmann_Hu_Farrell_2022} introduces finite-element(-in-space) implicit timesteppers for the incompressible analogue to this system with structure-preserving (SP) properties in the ideal case, alongside parameter-robust preconditioners. We show that these timesteppers can derive from a finite-element-in-time (FET) (and finite-element-in-space) interpretation. The benefits of this reformulation are discussed, including the derivation of timesteppers that are higher order in time, and the quantifiable dissipative SP properties in the non-ideal, resistive case.
        
        We discuss possible options for extending this FET approach to timesteppers for the compressible case.

        The kinetic corrections satisfy linearized Boltzmann equations. Using a Lénard--Bernstein collision operator, these take Fokker--Planck-like forms \cite{Fokker_1914, Planck_1917} wherein pseudo-particles in the numerical model obey the neoclassical transport equations, with particle-independent Brownian drift terms. This offers a rigorous methodology for incorporating collisions into the particle transport model, without coupling the equations of motions for each particle.
        
        Works by Chen, Chacón et al. \cite{Chen_Chacón_Barnes_2011, Chacón_Chen_Barnes_2013, Chen_Chacón_2014, Chen_Chacón_2015} have developed structure-preserving particle pushers for neoclassical transport in the Vlasov equations, derived from Crank--Nicolson integrators. We show these too can can derive from a FET interpretation, similarly offering potential extensions to higher-order-in-time particle pushers. The FET formulation is used also to consider how the stochastic drift terms can be incorporated into the pushers. Stochastic gyrokinetic expansions are also discussed.

        Different options for the numerical implementation of these schemes are considered.

        Due to the efficacy of FET in the development of SP timesteppers for both the fluid and kinetic component, we hope this approach will prove effective in the future for developing SP timesteppers for the full hybrid model. We hope this will give us the opportunity to incorporate previously inaccessible kinetic effects into the highly effective, modern, finite-element MHD models.
    \end{abstract}
    
    
    \newpage
    \tableofcontents
    
    
    \newpage
    \pagenumbering{arabic}
    %\linenumbers\renewcommand\thelinenumber{\color{black!50}\arabic{linenumber}}
            \documentclass[12pt, a4paper]{report}

\input{template/main.tex}

\title{\BA{Title in Progress...}}
\author{Boris Andrews}
\affil{Mathematical Institute, University of Oxford}
\date{\today}


\begin{document}
    \pagenumbering{gobble}
    \maketitle
    
    
    \begin{abstract}
        Magnetic confinement reactors---in particular tokamaks---offer one of the most promising options for achieving practical nuclear fusion, with the potential to provide virtually limitless, clean energy. The theoretical and numerical modeling of tokamak plasmas is simultaneously an essential component of effective reactor design, and a great research barrier. Tokamak operational conditions exhibit comparatively low Knudsen numbers. Kinetic effects, including kinetic waves and instabilities, Landau damping, bump-on-tail instabilities and more, are therefore highly influential in tokamak plasma dynamics. Purely fluid models are inherently incapable of capturing these effects, whereas the high dimensionality in purely kinetic models render them practically intractable for most relevant purposes.

        We consider a $\delta\!f$ decomposition model, with a macroscopic fluid background and microscopic kinetic correction, both fully coupled to each other. A similar manner of discretization is proposed to that used in the recent \texttt{STRUPHY} code \cite{Holderied_Possanner_Wang_2021, Holderied_2022, Li_et_al_2023} with a finite-element model for the background and a pseudo-particle/PiC model for the correction.

        The fluid background satisfies the full, non-linear, resistive, compressible, Hall MHD equations. \cite{Laakmann_Hu_Farrell_2022} introduces finite-element(-in-space) implicit timesteppers for the incompressible analogue to this system with structure-preserving (SP) properties in the ideal case, alongside parameter-robust preconditioners. We show that these timesteppers can derive from a finite-element-in-time (FET) (and finite-element-in-space) interpretation. The benefits of this reformulation are discussed, including the derivation of timesteppers that are higher order in time, and the quantifiable dissipative SP properties in the non-ideal, resistive case.
        
        We discuss possible options for extending this FET approach to timesteppers for the compressible case.

        The kinetic corrections satisfy linearized Boltzmann equations. Using a Lénard--Bernstein collision operator, these take Fokker--Planck-like forms \cite{Fokker_1914, Planck_1917} wherein pseudo-particles in the numerical model obey the neoclassical transport equations, with particle-independent Brownian drift terms. This offers a rigorous methodology for incorporating collisions into the particle transport model, without coupling the equations of motions for each particle.
        
        Works by Chen, Chacón et al. \cite{Chen_Chacón_Barnes_2011, Chacón_Chen_Barnes_2013, Chen_Chacón_2014, Chen_Chacón_2015} have developed structure-preserving particle pushers for neoclassical transport in the Vlasov equations, derived from Crank--Nicolson integrators. We show these too can can derive from a FET interpretation, similarly offering potential extensions to higher-order-in-time particle pushers. The FET formulation is used also to consider how the stochastic drift terms can be incorporated into the pushers. Stochastic gyrokinetic expansions are also discussed.

        Different options for the numerical implementation of these schemes are considered.

        Due to the efficacy of FET in the development of SP timesteppers for both the fluid and kinetic component, we hope this approach will prove effective in the future for developing SP timesteppers for the full hybrid model. We hope this will give us the opportunity to incorporate previously inaccessible kinetic effects into the highly effective, modern, finite-element MHD models.
    \end{abstract}
    
    
    \newpage
    \tableofcontents
    
    
    \newpage
    \pagenumbering{arabic}
    %\linenumbers\renewcommand\thelinenumber{\color{black!50}\arabic{linenumber}}
            \input{0 - introduction/main.tex}
        \part{Research}
            \input{1 - low-noise PiC models/main.tex}
            \input{2 - kinetic component/main.tex}
            \input{3 - fluid component/main.tex}
            \input{4 - numerical implementation/main.tex}
        \part{Project Overview}
            \input{5 - research plan/main.tex}
            \input{6 - summary/main.tex}
    
    
    %\section{}
    \newpage
    \pagenumbering{gobble}
        \printbibliography


    \newpage
    \pagenumbering{roman}
    \appendix
        \part{Appendices}
            \input{8 - Hilbert complexes/main.tex}
            \input{9 - weak conservation proofs/main.tex}
\end{document}

        \part{Research}
            \documentclass[12pt, a4paper]{report}

\input{template/main.tex}

\title{\BA{Title in Progress...}}
\author{Boris Andrews}
\affil{Mathematical Institute, University of Oxford}
\date{\today}


\begin{document}
    \pagenumbering{gobble}
    \maketitle
    
    
    \begin{abstract}
        Magnetic confinement reactors---in particular tokamaks---offer one of the most promising options for achieving practical nuclear fusion, with the potential to provide virtually limitless, clean energy. The theoretical and numerical modeling of tokamak plasmas is simultaneously an essential component of effective reactor design, and a great research barrier. Tokamak operational conditions exhibit comparatively low Knudsen numbers. Kinetic effects, including kinetic waves and instabilities, Landau damping, bump-on-tail instabilities and more, are therefore highly influential in tokamak plasma dynamics. Purely fluid models are inherently incapable of capturing these effects, whereas the high dimensionality in purely kinetic models render them practically intractable for most relevant purposes.

        We consider a $\delta\!f$ decomposition model, with a macroscopic fluid background and microscopic kinetic correction, both fully coupled to each other. A similar manner of discretization is proposed to that used in the recent \texttt{STRUPHY} code \cite{Holderied_Possanner_Wang_2021, Holderied_2022, Li_et_al_2023} with a finite-element model for the background and a pseudo-particle/PiC model for the correction.

        The fluid background satisfies the full, non-linear, resistive, compressible, Hall MHD equations. \cite{Laakmann_Hu_Farrell_2022} introduces finite-element(-in-space) implicit timesteppers for the incompressible analogue to this system with structure-preserving (SP) properties in the ideal case, alongside parameter-robust preconditioners. We show that these timesteppers can derive from a finite-element-in-time (FET) (and finite-element-in-space) interpretation. The benefits of this reformulation are discussed, including the derivation of timesteppers that are higher order in time, and the quantifiable dissipative SP properties in the non-ideal, resistive case.
        
        We discuss possible options for extending this FET approach to timesteppers for the compressible case.

        The kinetic corrections satisfy linearized Boltzmann equations. Using a Lénard--Bernstein collision operator, these take Fokker--Planck-like forms \cite{Fokker_1914, Planck_1917} wherein pseudo-particles in the numerical model obey the neoclassical transport equations, with particle-independent Brownian drift terms. This offers a rigorous methodology for incorporating collisions into the particle transport model, without coupling the equations of motions for each particle.
        
        Works by Chen, Chacón et al. \cite{Chen_Chacón_Barnes_2011, Chacón_Chen_Barnes_2013, Chen_Chacón_2014, Chen_Chacón_2015} have developed structure-preserving particle pushers for neoclassical transport in the Vlasov equations, derived from Crank--Nicolson integrators. We show these too can can derive from a FET interpretation, similarly offering potential extensions to higher-order-in-time particle pushers. The FET formulation is used also to consider how the stochastic drift terms can be incorporated into the pushers. Stochastic gyrokinetic expansions are also discussed.

        Different options for the numerical implementation of these schemes are considered.

        Due to the efficacy of FET in the development of SP timesteppers for both the fluid and kinetic component, we hope this approach will prove effective in the future for developing SP timesteppers for the full hybrid model. We hope this will give us the opportunity to incorporate previously inaccessible kinetic effects into the highly effective, modern, finite-element MHD models.
    \end{abstract}
    
    
    \newpage
    \tableofcontents
    
    
    \newpage
    \pagenumbering{arabic}
    %\linenumbers\renewcommand\thelinenumber{\color{black!50}\arabic{linenumber}}
            \input{0 - introduction/main.tex}
        \part{Research}
            \input{1 - low-noise PiC models/main.tex}
            \input{2 - kinetic component/main.tex}
            \input{3 - fluid component/main.tex}
            \input{4 - numerical implementation/main.tex}
        \part{Project Overview}
            \input{5 - research plan/main.tex}
            \input{6 - summary/main.tex}
    
    
    %\section{}
    \newpage
    \pagenumbering{gobble}
        \printbibliography


    \newpage
    \pagenumbering{roman}
    \appendix
        \part{Appendices}
            \input{8 - Hilbert complexes/main.tex}
            \input{9 - weak conservation proofs/main.tex}
\end{document}

            \documentclass[12pt, a4paper]{report}

\input{template/main.tex}

\title{\BA{Title in Progress...}}
\author{Boris Andrews}
\affil{Mathematical Institute, University of Oxford}
\date{\today}


\begin{document}
    \pagenumbering{gobble}
    \maketitle
    
    
    \begin{abstract}
        Magnetic confinement reactors---in particular tokamaks---offer one of the most promising options for achieving practical nuclear fusion, with the potential to provide virtually limitless, clean energy. The theoretical and numerical modeling of tokamak plasmas is simultaneously an essential component of effective reactor design, and a great research barrier. Tokamak operational conditions exhibit comparatively low Knudsen numbers. Kinetic effects, including kinetic waves and instabilities, Landau damping, bump-on-tail instabilities and more, are therefore highly influential in tokamak plasma dynamics. Purely fluid models are inherently incapable of capturing these effects, whereas the high dimensionality in purely kinetic models render them practically intractable for most relevant purposes.

        We consider a $\delta\!f$ decomposition model, with a macroscopic fluid background and microscopic kinetic correction, both fully coupled to each other. A similar manner of discretization is proposed to that used in the recent \texttt{STRUPHY} code \cite{Holderied_Possanner_Wang_2021, Holderied_2022, Li_et_al_2023} with a finite-element model for the background and a pseudo-particle/PiC model for the correction.

        The fluid background satisfies the full, non-linear, resistive, compressible, Hall MHD equations. \cite{Laakmann_Hu_Farrell_2022} introduces finite-element(-in-space) implicit timesteppers for the incompressible analogue to this system with structure-preserving (SP) properties in the ideal case, alongside parameter-robust preconditioners. We show that these timesteppers can derive from a finite-element-in-time (FET) (and finite-element-in-space) interpretation. The benefits of this reformulation are discussed, including the derivation of timesteppers that are higher order in time, and the quantifiable dissipative SP properties in the non-ideal, resistive case.
        
        We discuss possible options for extending this FET approach to timesteppers for the compressible case.

        The kinetic corrections satisfy linearized Boltzmann equations. Using a Lénard--Bernstein collision operator, these take Fokker--Planck-like forms \cite{Fokker_1914, Planck_1917} wherein pseudo-particles in the numerical model obey the neoclassical transport equations, with particle-independent Brownian drift terms. This offers a rigorous methodology for incorporating collisions into the particle transport model, without coupling the equations of motions for each particle.
        
        Works by Chen, Chacón et al. \cite{Chen_Chacón_Barnes_2011, Chacón_Chen_Barnes_2013, Chen_Chacón_2014, Chen_Chacón_2015} have developed structure-preserving particle pushers for neoclassical transport in the Vlasov equations, derived from Crank--Nicolson integrators. We show these too can can derive from a FET interpretation, similarly offering potential extensions to higher-order-in-time particle pushers. The FET formulation is used also to consider how the stochastic drift terms can be incorporated into the pushers. Stochastic gyrokinetic expansions are also discussed.

        Different options for the numerical implementation of these schemes are considered.

        Due to the efficacy of FET in the development of SP timesteppers for both the fluid and kinetic component, we hope this approach will prove effective in the future for developing SP timesteppers for the full hybrid model. We hope this will give us the opportunity to incorporate previously inaccessible kinetic effects into the highly effective, modern, finite-element MHD models.
    \end{abstract}
    
    
    \newpage
    \tableofcontents
    
    
    \newpage
    \pagenumbering{arabic}
    %\linenumbers\renewcommand\thelinenumber{\color{black!50}\arabic{linenumber}}
            \input{0 - introduction/main.tex}
        \part{Research}
            \input{1 - low-noise PiC models/main.tex}
            \input{2 - kinetic component/main.tex}
            \input{3 - fluid component/main.tex}
            \input{4 - numerical implementation/main.tex}
        \part{Project Overview}
            \input{5 - research plan/main.tex}
            \input{6 - summary/main.tex}
    
    
    %\section{}
    \newpage
    \pagenumbering{gobble}
        \printbibliography


    \newpage
    \pagenumbering{roman}
    \appendix
        \part{Appendices}
            \input{8 - Hilbert complexes/main.tex}
            \input{9 - weak conservation proofs/main.tex}
\end{document}

            \documentclass[12pt, a4paper]{report}

\input{template/main.tex}

\title{\BA{Title in Progress...}}
\author{Boris Andrews}
\affil{Mathematical Institute, University of Oxford}
\date{\today}


\begin{document}
    \pagenumbering{gobble}
    \maketitle
    
    
    \begin{abstract}
        Magnetic confinement reactors---in particular tokamaks---offer one of the most promising options for achieving practical nuclear fusion, with the potential to provide virtually limitless, clean energy. The theoretical and numerical modeling of tokamak plasmas is simultaneously an essential component of effective reactor design, and a great research barrier. Tokamak operational conditions exhibit comparatively low Knudsen numbers. Kinetic effects, including kinetic waves and instabilities, Landau damping, bump-on-tail instabilities and more, are therefore highly influential in tokamak plasma dynamics. Purely fluid models are inherently incapable of capturing these effects, whereas the high dimensionality in purely kinetic models render them practically intractable for most relevant purposes.

        We consider a $\delta\!f$ decomposition model, with a macroscopic fluid background and microscopic kinetic correction, both fully coupled to each other. A similar manner of discretization is proposed to that used in the recent \texttt{STRUPHY} code \cite{Holderied_Possanner_Wang_2021, Holderied_2022, Li_et_al_2023} with a finite-element model for the background and a pseudo-particle/PiC model for the correction.

        The fluid background satisfies the full, non-linear, resistive, compressible, Hall MHD equations. \cite{Laakmann_Hu_Farrell_2022} introduces finite-element(-in-space) implicit timesteppers for the incompressible analogue to this system with structure-preserving (SP) properties in the ideal case, alongside parameter-robust preconditioners. We show that these timesteppers can derive from a finite-element-in-time (FET) (and finite-element-in-space) interpretation. The benefits of this reformulation are discussed, including the derivation of timesteppers that are higher order in time, and the quantifiable dissipative SP properties in the non-ideal, resistive case.
        
        We discuss possible options for extending this FET approach to timesteppers for the compressible case.

        The kinetic corrections satisfy linearized Boltzmann equations. Using a Lénard--Bernstein collision operator, these take Fokker--Planck-like forms \cite{Fokker_1914, Planck_1917} wherein pseudo-particles in the numerical model obey the neoclassical transport equations, with particle-independent Brownian drift terms. This offers a rigorous methodology for incorporating collisions into the particle transport model, without coupling the equations of motions for each particle.
        
        Works by Chen, Chacón et al. \cite{Chen_Chacón_Barnes_2011, Chacón_Chen_Barnes_2013, Chen_Chacón_2014, Chen_Chacón_2015} have developed structure-preserving particle pushers for neoclassical transport in the Vlasov equations, derived from Crank--Nicolson integrators. We show these too can can derive from a FET interpretation, similarly offering potential extensions to higher-order-in-time particle pushers. The FET formulation is used also to consider how the stochastic drift terms can be incorporated into the pushers. Stochastic gyrokinetic expansions are also discussed.

        Different options for the numerical implementation of these schemes are considered.

        Due to the efficacy of FET in the development of SP timesteppers for both the fluid and kinetic component, we hope this approach will prove effective in the future for developing SP timesteppers for the full hybrid model. We hope this will give us the opportunity to incorporate previously inaccessible kinetic effects into the highly effective, modern, finite-element MHD models.
    \end{abstract}
    
    
    \newpage
    \tableofcontents
    
    
    \newpage
    \pagenumbering{arabic}
    %\linenumbers\renewcommand\thelinenumber{\color{black!50}\arabic{linenumber}}
            \input{0 - introduction/main.tex}
        \part{Research}
            \input{1 - low-noise PiC models/main.tex}
            \input{2 - kinetic component/main.tex}
            \input{3 - fluid component/main.tex}
            \input{4 - numerical implementation/main.tex}
        \part{Project Overview}
            \input{5 - research plan/main.tex}
            \input{6 - summary/main.tex}
    
    
    %\section{}
    \newpage
    \pagenumbering{gobble}
        \printbibliography


    \newpage
    \pagenumbering{roman}
    \appendix
        \part{Appendices}
            \input{8 - Hilbert complexes/main.tex}
            \input{9 - weak conservation proofs/main.tex}
\end{document}

            \documentclass[12pt, a4paper]{report}

\input{template/main.tex}

\title{\BA{Title in Progress...}}
\author{Boris Andrews}
\affil{Mathematical Institute, University of Oxford}
\date{\today}


\begin{document}
    \pagenumbering{gobble}
    \maketitle
    
    
    \begin{abstract}
        Magnetic confinement reactors---in particular tokamaks---offer one of the most promising options for achieving practical nuclear fusion, with the potential to provide virtually limitless, clean energy. The theoretical and numerical modeling of tokamak plasmas is simultaneously an essential component of effective reactor design, and a great research barrier. Tokamak operational conditions exhibit comparatively low Knudsen numbers. Kinetic effects, including kinetic waves and instabilities, Landau damping, bump-on-tail instabilities and more, are therefore highly influential in tokamak plasma dynamics. Purely fluid models are inherently incapable of capturing these effects, whereas the high dimensionality in purely kinetic models render them practically intractable for most relevant purposes.

        We consider a $\delta\!f$ decomposition model, with a macroscopic fluid background and microscopic kinetic correction, both fully coupled to each other. A similar manner of discretization is proposed to that used in the recent \texttt{STRUPHY} code \cite{Holderied_Possanner_Wang_2021, Holderied_2022, Li_et_al_2023} with a finite-element model for the background and a pseudo-particle/PiC model for the correction.

        The fluid background satisfies the full, non-linear, resistive, compressible, Hall MHD equations. \cite{Laakmann_Hu_Farrell_2022} introduces finite-element(-in-space) implicit timesteppers for the incompressible analogue to this system with structure-preserving (SP) properties in the ideal case, alongside parameter-robust preconditioners. We show that these timesteppers can derive from a finite-element-in-time (FET) (and finite-element-in-space) interpretation. The benefits of this reformulation are discussed, including the derivation of timesteppers that are higher order in time, and the quantifiable dissipative SP properties in the non-ideal, resistive case.
        
        We discuss possible options for extending this FET approach to timesteppers for the compressible case.

        The kinetic corrections satisfy linearized Boltzmann equations. Using a Lénard--Bernstein collision operator, these take Fokker--Planck-like forms \cite{Fokker_1914, Planck_1917} wherein pseudo-particles in the numerical model obey the neoclassical transport equations, with particle-independent Brownian drift terms. This offers a rigorous methodology for incorporating collisions into the particle transport model, without coupling the equations of motions for each particle.
        
        Works by Chen, Chacón et al. \cite{Chen_Chacón_Barnes_2011, Chacón_Chen_Barnes_2013, Chen_Chacón_2014, Chen_Chacón_2015} have developed structure-preserving particle pushers for neoclassical transport in the Vlasov equations, derived from Crank--Nicolson integrators. We show these too can can derive from a FET interpretation, similarly offering potential extensions to higher-order-in-time particle pushers. The FET formulation is used also to consider how the stochastic drift terms can be incorporated into the pushers. Stochastic gyrokinetic expansions are also discussed.

        Different options for the numerical implementation of these schemes are considered.

        Due to the efficacy of FET in the development of SP timesteppers for both the fluid and kinetic component, we hope this approach will prove effective in the future for developing SP timesteppers for the full hybrid model. We hope this will give us the opportunity to incorporate previously inaccessible kinetic effects into the highly effective, modern, finite-element MHD models.
    \end{abstract}
    
    
    \newpage
    \tableofcontents
    
    
    \newpage
    \pagenumbering{arabic}
    %\linenumbers\renewcommand\thelinenumber{\color{black!50}\arabic{linenumber}}
            \input{0 - introduction/main.tex}
        \part{Research}
            \input{1 - low-noise PiC models/main.tex}
            \input{2 - kinetic component/main.tex}
            \input{3 - fluid component/main.tex}
            \input{4 - numerical implementation/main.tex}
        \part{Project Overview}
            \input{5 - research plan/main.tex}
            \input{6 - summary/main.tex}
    
    
    %\section{}
    \newpage
    \pagenumbering{gobble}
        \printbibliography


    \newpage
    \pagenumbering{roman}
    \appendix
        \part{Appendices}
            \input{8 - Hilbert complexes/main.tex}
            \input{9 - weak conservation proofs/main.tex}
\end{document}

        \part{Project Overview}
            \documentclass[12pt, a4paper]{report}

\input{template/main.tex}

\title{\BA{Title in Progress...}}
\author{Boris Andrews}
\affil{Mathematical Institute, University of Oxford}
\date{\today}


\begin{document}
    \pagenumbering{gobble}
    \maketitle
    
    
    \begin{abstract}
        Magnetic confinement reactors---in particular tokamaks---offer one of the most promising options for achieving practical nuclear fusion, with the potential to provide virtually limitless, clean energy. The theoretical and numerical modeling of tokamak plasmas is simultaneously an essential component of effective reactor design, and a great research barrier. Tokamak operational conditions exhibit comparatively low Knudsen numbers. Kinetic effects, including kinetic waves and instabilities, Landau damping, bump-on-tail instabilities and more, are therefore highly influential in tokamak plasma dynamics. Purely fluid models are inherently incapable of capturing these effects, whereas the high dimensionality in purely kinetic models render them practically intractable for most relevant purposes.

        We consider a $\delta\!f$ decomposition model, with a macroscopic fluid background and microscopic kinetic correction, both fully coupled to each other. A similar manner of discretization is proposed to that used in the recent \texttt{STRUPHY} code \cite{Holderied_Possanner_Wang_2021, Holderied_2022, Li_et_al_2023} with a finite-element model for the background and a pseudo-particle/PiC model for the correction.

        The fluid background satisfies the full, non-linear, resistive, compressible, Hall MHD equations. \cite{Laakmann_Hu_Farrell_2022} introduces finite-element(-in-space) implicit timesteppers for the incompressible analogue to this system with structure-preserving (SP) properties in the ideal case, alongside parameter-robust preconditioners. We show that these timesteppers can derive from a finite-element-in-time (FET) (and finite-element-in-space) interpretation. The benefits of this reformulation are discussed, including the derivation of timesteppers that are higher order in time, and the quantifiable dissipative SP properties in the non-ideal, resistive case.
        
        We discuss possible options for extending this FET approach to timesteppers for the compressible case.

        The kinetic corrections satisfy linearized Boltzmann equations. Using a Lénard--Bernstein collision operator, these take Fokker--Planck-like forms \cite{Fokker_1914, Planck_1917} wherein pseudo-particles in the numerical model obey the neoclassical transport equations, with particle-independent Brownian drift terms. This offers a rigorous methodology for incorporating collisions into the particle transport model, without coupling the equations of motions for each particle.
        
        Works by Chen, Chacón et al. \cite{Chen_Chacón_Barnes_2011, Chacón_Chen_Barnes_2013, Chen_Chacón_2014, Chen_Chacón_2015} have developed structure-preserving particle pushers for neoclassical transport in the Vlasov equations, derived from Crank--Nicolson integrators. We show these too can can derive from a FET interpretation, similarly offering potential extensions to higher-order-in-time particle pushers. The FET formulation is used also to consider how the stochastic drift terms can be incorporated into the pushers. Stochastic gyrokinetic expansions are also discussed.

        Different options for the numerical implementation of these schemes are considered.

        Due to the efficacy of FET in the development of SP timesteppers for both the fluid and kinetic component, we hope this approach will prove effective in the future for developing SP timesteppers for the full hybrid model. We hope this will give us the opportunity to incorporate previously inaccessible kinetic effects into the highly effective, modern, finite-element MHD models.
    \end{abstract}
    
    
    \newpage
    \tableofcontents
    
    
    \newpage
    \pagenumbering{arabic}
    %\linenumbers\renewcommand\thelinenumber{\color{black!50}\arabic{linenumber}}
            \input{0 - introduction/main.tex}
        \part{Research}
            \input{1 - low-noise PiC models/main.tex}
            \input{2 - kinetic component/main.tex}
            \input{3 - fluid component/main.tex}
            \input{4 - numerical implementation/main.tex}
        \part{Project Overview}
            \input{5 - research plan/main.tex}
            \input{6 - summary/main.tex}
    
    
    %\section{}
    \newpage
    \pagenumbering{gobble}
        \printbibliography


    \newpage
    \pagenumbering{roman}
    \appendix
        \part{Appendices}
            \input{8 - Hilbert complexes/main.tex}
            \input{9 - weak conservation proofs/main.tex}
\end{document}

            \documentclass[12pt, a4paper]{report}

\input{template/main.tex}

\title{\BA{Title in Progress...}}
\author{Boris Andrews}
\affil{Mathematical Institute, University of Oxford}
\date{\today}


\begin{document}
    \pagenumbering{gobble}
    \maketitle
    
    
    \begin{abstract}
        Magnetic confinement reactors---in particular tokamaks---offer one of the most promising options for achieving practical nuclear fusion, with the potential to provide virtually limitless, clean energy. The theoretical and numerical modeling of tokamak plasmas is simultaneously an essential component of effective reactor design, and a great research barrier. Tokamak operational conditions exhibit comparatively low Knudsen numbers. Kinetic effects, including kinetic waves and instabilities, Landau damping, bump-on-tail instabilities and more, are therefore highly influential in tokamak plasma dynamics. Purely fluid models are inherently incapable of capturing these effects, whereas the high dimensionality in purely kinetic models render them practically intractable for most relevant purposes.

        We consider a $\delta\!f$ decomposition model, with a macroscopic fluid background and microscopic kinetic correction, both fully coupled to each other. A similar manner of discretization is proposed to that used in the recent \texttt{STRUPHY} code \cite{Holderied_Possanner_Wang_2021, Holderied_2022, Li_et_al_2023} with a finite-element model for the background and a pseudo-particle/PiC model for the correction.

        The fluid background satisfies the full, non-linear, resistive, compressible, Hall MHD equations. \cite{Laakmann_Hu_Farrell_2022} introduces finite-element(-in-space) implicit timesteppers for the incompressible analogue to this system with structure-preserving (SP) properties in the ideal case, alongside parameter-robust preconditioners. We show that these timesteppers can derive from a finite-element-in-time (FET) (and finite-element-in-space) interpretation. The benefits of this reformulation are discussed, including the derivation of timesteppers that are higher order in time, and the quantifiable dissipative SP properties in the non-ideal, resistive case.
        
        We discuss possible options for extending this FET approach to timesteppers for the compressible case.

        The kinetic corrections satisfy linearized Boltzmann equations. Using a Lénard--Bernstein collision operator, these take Fokker--Planck-like forms \cite{Fokker_1914, Planck_1917} wherein pseudo-particles in the numerical model obey the neoclassical transport equations, with particle-independent Brownian drift terms. This offers a rigorous methodology for incorporating collisions into the particle transport model, without coupling the equations of motions for each particle.
        
        Works by Chen, Chacón et al. \cite{Chen_Chacón_Barnes_2011, Chacón_Chen_Barnes_2013, Chen_Chacón_2014, Chen_Chacón_2015} have developed structure-preserving particle pushers for neoclassical transport in the Vlasov equations, derived from Crank--Nicolson integrators. We show these too can can derive from a FET interpretation, similarly offering potential extensions to higher-order-in-time particle pushers. The FET formulation is used also to consider how the stochastic drift terms can be incorporated into the pushers. Stochastic gyrokinetic expansions are also discussed.

        Different options for the numerical implementation of these schemes are considered.

        Due to the efficacy of FET in the development of SP timesteppers for both the fluid and kinetic component, we hope this approach will prove effective in the future for developing SP timesteppers for the full hybrid model. We hope this will give us the opportunity to incorporate previously inaccessible kinetic effects into the highly effective, modern, finite-element MHD models.
    \end{abstract}
    
    
    \newpage
    \tableofcontents
    
    
    \newpage
    \pagenumbering{arabic}
    %\linenumbers\renewcommand\thelinenumber{\color{black!50}\arabic{linenumber}}
            \input{0 - introduction/main.tex}
        \part{Research}
            \input{1 - low-noise PiC models/main.tex}
            \input{2 - kinetic component/main.tex}
            \input{3 - fluid component/main.tex}
            \input{4 - numerical implementation/main.tex}
        \part{Project Overview}
            \input{5 - research plan/main.tex}
            \input{6 - summary/main.tex}
    
    
    %\section{}
    \newpage
    \pagenumbering{gobble}
        \printbibliography


    \newpage
    \pagenumbering{roman}
    \appendix
        \part{Appendices}
            \input{8 - Hilbert complexes/main.tex}
            \input{9 - weak conservation proofs/main.tex}
\end{document}

    
    
    %\section{}
    \newpage
    \pagenumbering{gobble}
        \printbibliography


    \newpage
    \pagenumbering{roman}
    \appendix
        \part{Appendices}
            \documentclass[12pt, a4paper]{report}

\input{template/main.tex}

\title{\BA{Title in Progress...}}
\author{Boris Andrews}
\affil{Mathematical Institute, University of Oxford}
\date{\today}


\begin{document}
    \pagenumbering{gobble}
    \maketitle
    
    
    \begin{abstract}
        Magnetic confinement reactors---in particular tokamaks---offer one of the most promising options for achieving practical nuclear fusion, with the potential to provide virtually limitless, clean energy. The theoretical and numerical modeling of tokamak plasmas is simultaneously an essential component of effective reactor design, and a great research barrier. Tokamak operational conditions exhibit comparatively low Knudsen numbers. Kinetic effects, including kinetic waves and instabilities, Landau damping, bump-on-tail instabilities and more, are therefore highly influential in tokamak plasma dynamics. Purely fluid models are inherently incapable of capturing these effects, whereas the high dimensionality in purely kinetic models render them practically intractable for most relevant purposes.

        We consider a $\delta\!f$ decomposition model, with a macroscopic fluid background and microscopic kinetic correction, both fully coupled to each other. A similar manner of discretization is proposed to that used in the recent \texttt{STRUPHY} code \cite{Holderied_Possanner_Wang_2021, Holderied_2022, Li_et_al_2023} with a finite-element model for the background and a pseudo-particle/PiC model for the correction.

        The fluid background satisfies the full, non-linear, resistive, compressible, Hall MHD equations. \cite{Laakmann_Hu_Farrell_2022} introduces finite-element(-in-space) implicit timesteppers for the incompressible analogue to this system with structure-preserving (SP) properties in the ideal case, alongside parameter-robust preconditioners. We show that these timesteppers can derive from a finite-element-in-time (FET) (and finite-element-in-space) interpretation. The benefits of this reformulation are discussed, including the derivation of timesteppers that are higher order in time, and the quantifiable dissipative SP properties in the non-ideal, resistive case.
        
        We discuss possible options for extending this FET approach to timesteppers for the compressible case.

        The kinetic corrections satisfy linearized Boltzmann equations. Using a Lénard--Bernstein collision operator, these take Fokker--Planck-like forms \cite{Fokker_1914, Planck_1917} wherein pseudo-particles in the numerical model obey the neoclassical transport equations, with particle-independent Brownian drift terms. This offers a rigorous methodology for incorporating collisions into the particle transport model, without coupling the equations of motions for each particle.
        
        Works by Chen, Chacón et al. \cite{Chen_Chacón_Barnes_2011, Chacón_Chen_Barnes_2013, Chen_Chacón_2014, Chen_Chacón_2015} have developed structure-preserving particle pushers for neoclassical transport in the Vlasov equations, derived from Crank--Nicolson integrators. We show these too can can derive from a FET interpretation, similarly offering potential extensions to higher-order-in-time particle pushers. The FET formulation is used also to consider how the stochastic drift terms can be incorporated into the pushers. Stochastic gyrokinetic expansions are also discussed.

        Different options for the numerical implementation of these schemes are considered.

        Due to the efficacy of FET in the development of SP timesteppers for both the fluid and kinetic component, we hope this approach will prove effective in the future for developing SP timesteppers for the full hybrid model. We hope this will give us the opportunity to incorporate previously inaccessible kinetic effects into the highly effective, modern, finite-element MHD models.
    \end{abstract}
    
    
    \newpage
    \tableofcontents
    
    
    \newpage
    \pagenumbering{arabic}
    %\linenumbers\renewcommand\thelinenumber{\color{black!50}\arabic{linenumber}}
            \input{0 - introduction/main.tex}
        \part{Research}
            \input{1 - low-noise PiC models/main.tex}
            \input{2 - kinetic component/main.tex}
            \input{3 - fluid component/main.tex}
            \input{4 - numerical implementation/main.tex}
        \part{Project Overview}
            \input{5 - research plan/main.tex}
            \input{6 - summary/main.tex}
    
    
    %\section{}
    \newpage
    \pagenumbering{gobble}
        \printbibliography


    \newpage
    \pagenumbering{roman}
    \appendix
        \part{Appendices}
            \input{8 - Hilbert complexes/main.tex}
            \input{9 - weak conservation proofs/main.tex}
\end{document}

            \documentclass[12pt, a4paper]{report}

\input{template/main.tex}

\title{\BA{Title in Progress...}}
\author{Boris Andrews}
\affil{Mathematical Institute, University of Oxford}
\date{\today}


\begin{document}
    \pagenumbering{gobble}
    \maketitle
    
    
    \begin{abstract}
        Magnetic confinement reactors---in particular tokamaks---offer one of the most promising options for achieving practical nuclear fusion, with the potential to provide virtually limitless, clean energy. The theoretical and numerical modeling of tokamak plasmas is simultaneously an essential component of effective reactor design, and a great research barrier. Tokamak operational conditions exhibit comparatively low Knudsen numbers. Kinetic effects, including kinetic waves and instabilities, Landau damping, bump-on-tail instabilities and more, are therefore highly influential in tokamak plasma dynamics. Purely fluid models are inherently incapable of capturing these effects, whereas the high dimensionality in purely kinetic models render them practically intractable for most relevant purposes.

        We consider a $\delta\!f$ decomposition model, with a macroscopic fluid background and microscopic kinetic correction, both fully coupled to each other. A similar manner of discretization is proposed to that used in the recent \texttt{STRUPHY} code \cite{Holderied_Possanner_Wang_2021, Holderied_2022, Li_et_al_2023} with a finite-element model for the background and a pseudo-particle/PiC model for the correction.

        The fluid background satisfies the full, non-linear, resistive, compressible, Hall MHD equations. \cite{Laakmann_Hu_Farrell_2022} introduces finite-element(-in-space) implicit timesteppers for the incompressible analogue to this system with structure-preserving (SP) properties in the ideal case, alongside parameter-robust preconditioners. We show that these timesteppers can derive from a finite-element-in-time (FET) (and finite-element-in-space) interpretation. The benefits of this reformulation are discussed, including the derivation of timesteppers that are higher order in time, and the quantifiable dissipative SP properties in the non-ideal, resistive case.
        
        We discuss possible options for extending this FET approach to timesteppers for the compressible case.

        The kinetic corrections satisfy linearized Boltzmann equations. Using a Lénard--Bernstein collision operator, these take Fokker--Planck-like forms \cite{Fokker_1914, Planck_1917} wherein pseudo-particles in the numerical model obey the neoclassical transport equations, with particle-independent Brownian drift terms. This offers a rigorous methodology for incorporating collisions into the particle transport model, without coupling the equations of motions for each particle.
        
        Works by Chen, Chacón et al. \cite{Chen_Chacón_Barnes_2011, Chacón_Chen_Barnes_2013, Chen_Chacón_2014, Chen_Chacón_2015} have developed structure-preserving particle pushers for neoclassical transport in the Vlasov equations, derived from Crank--Nicolson integrators. We show these too can can derive from a FET interpretation, similarly offering potential extensions to higher-order-in-time particle pushers. The FET formulation is used also to consider how the stochastic drift terms can be incorporated into the pushers. Stochastic gyrokinetic expansions are also discussed.

        Different options for the numerical implementation of these schemes are considered.

        Due to the efficacy of FET in the development of SP timesteppers for both the fluid and kinetic component, we hope this approach will prove effective in the future for developing SP timesteppers for the full hybrid model. We hope this will give us the opportunity to incorporate previously inaccessible kinetic effects into the highly effective, modern, finite-element MHD models.
    \end{abstract}
    
    
    \newpage
    \tableofcontents
    
    
    \newpage
    \pagenumbering{arabic}
    %\linenumbers\renewcommand\thelinenumber{\color{black!50}\arabic{linenumber}}
            \input{0 - introduction/main.tex}
        \part{Research}
            \input{1 - low-noise PiC models/main.tex}
            \input{2 - kinetic component/main.tex}
            \input{3 - fluid component/main.tex}
            \input{4 - numerical implementation/main.tex}
        \part{Project Overview}
            \input{5 - research plan/main.tex}
            \input{6 - summary/main.tex}
    
    
    %\section{}
    \newpage
    \pagenumbering{gobble}
        \printbibliography


    \newpage
    \pagenumbering{roman}
    \appendix
        \part{Appendices}
            \input{8 - Hilbert complexes/main.tex}
            \input{9 - weak conservation proofs/main.tex}
\end{document}

\end{document}

    
    
    %\section{}
    \newpage
    \pagenumbering{gobble}
        \printbibliography


    \newpage
    \pagenumbering{roman}
    \appendix
        \part{Appendices}
            \documentclass[12pt, a4paper]{report}

\documentclass[12pt, a4paper]{report}

\input{template/main.tex}

\title{\BA{Title in Progress...}}
\author{Boris Andrews}
\affil{Mathematical Institute, University of Oxford}
\date{\today}


\begin{document}
    \pagenumbering{gobble}
    \maketitle
    
    
    \begin{abstract}
        Magnetic confinement reactors---in particular tokamaks---offer one of the most promising options for achieving practical nuclear fusion, with the potential to provide virtually limitless, clean energy. The theoretical and numerical modeling of tokamak plasmas is simultaneously an essential component of effective reactor design, and a great research barrier. Tokamak operational conditions exhibit comparatively low Knudsen numbers. Kinetic effects, including kinetic waves and instabilities, Landau damping, bump-on-tail instabilities and more, are therefore highly influential in tokamak plasma dynamics. Purely fluid models are inherently incapable of capturing these effects, whereas the high dimensionality in purely kinetic models render them practically intractable for most relevant purposes.

        We consider a $\delta\!f$ decomposition model, with a macroscopic fluid background and microscopic kinetic correction, both fully coupled to each other. A similar manner of discretization is proposed to that used in the recent \texttt{STRUPHY} code \cite{Holderied_Possanner_Wang_2021, Holderied_2022, Li_et_al_2023} with a finite-element model for the background and a pseudo-particle/PiC model for the correction.

        The fluid background satisfies the full, non-linear, resistive, compressible, Hall MHD equations. \cite{Laakmann_Hu_Farrell_2022} introduces finite-element(-in-space) implicit timesteppers for the incompressible analogue to this system with structure-preserving (SP) properties in the ideal case, alongside parameter-robust preconditioners. We show that these timesteppers can derive from a finite-element-in-time (FET) (and finite-element-in-space) interpretation. The benefits of this reformulation are discussed, including the derivation of timesteppers that are higher order in time, and the quantifiable dissipative SP properties in the non-ideal, resistive case.
        
        We discuss possible options for extending this FET approach to timesteppers for the compressible case.

        The kinetic corrections satisfy linearized Boltzmann equations. Using a Lénard--Bernstein collision operator, these take Fokker--Planck-like forms \cite{Fokker_1914, Planck_1917} wherein pseudo-particles in the numerical model obey the neoclassical transport equations, with particle-independent Brownian drift terms. This offers a rigorous methodology for incorporating collisions into the particle transport model, without coupling the equations of motions for each particle.
        
        Works by Chen, Chacón et al. \cite{Chen_Chacón_Barnes_2011, Chacón_Chen_Barnes_2013, Chen_Chacón_2014, Chen_Chacón_2015} have developed structure-preserving particle pushers for neoclassical transport in the Vlasov equations, derived from Crank--Nicolson integrators. We show these too can can derive from a FET interpretation, similarly offering potential extensions to higher-order-in-time particle pushers. The FET formulation is used also to consider how the stochastic drift terms can be incorporated into the pushers. Stochastic gyrokinetic expansions are also discussed.

        Different options for the numerical implementation of these schemes are considered.

        Due to the efficacy of FET in the development of SP timesteppers for both the fluid and kinetic component, we hope this approach will prove effective in the future for developing SP timesteppers for the full hybrid model. We hope this will give us the opportunity to incorporate previously inaccessible kinetic effects into the highly effective, modern, finite-element MHD models.
    \end{abstract}
    
    
    \newpage
    \tableofcontents
    
    
    \newpage
    \pagenumbering{arabic}
    %\linenumbers\renewcommand\thelinenumber{\color{black!50}\arabic{linenumber}}
            \input{0 - introduction/main.tex}
        \part{Research}
            \input{1 - low-noise PiC models/main.tex}
            \input{2 - kinetic component/main.tex}
            \input{3 - fluid component/main.tex}
            \input{4 - numerical implementation/main.tex}
        \part{Project Overview}
            \input{5 - research plan/main.tex}
            \input{6 - summary/main.tex}
    
    
    %\section{}
    \newpage
    \pagenumbering{gobble}
        \printbibliography


    \newpage
    \pagenumbering{roman}
    \appendix
        \part{Appendices}
            \input{8 - Hilbert complexes/main.tex}
            \input{9 - weak conservation proofs/main.tex}
\end{document}


\title{\BA{Title in Progress...}}
\author{Boris Andrews}
\affil{Mathematical Institute, University of Oxford}
\date{\today}


\begin{document}
    \pagenumbering{gobble}
    \maketitle
    
    
    \begin{abstract}
        Magnetic confinement reactors---in particular tokamaks---offer one of the most promising options for achieving practical nuclear fusion, with the potential to provide virtually limitless, clean energy. The theoretical and numerical modeling of tokamak plasmas is simultaneously an essential component of effective reactor design, and a great research barrier. Tokamak operational conditions exhibit comparatively low Knudsen numbers. Kinetic effects, including kinetic waves and instabilities, Landau damping, bump-on-tail instabilities and more, are therefore highly influential in tokamak plasma dynamics. Purely fluid models are inherently incapable of capturing these effects, whereas the high dimensionality in purely kinetic models render them practically intractable for most relevant purposes.

        We consider a $\delta\!f$ decomposition model, with a macroscopic fluid background and microscopic kinetic correction, both fully coupled to each other. A similar manner of discretization is proposed to that used in the recent \texttt{STRUPHY} code \cite{Holderied_Possanner_Wang_2021, Holderied_2022, Li_et_al_2023} with a finite-element model for the background and a pseudo-particle/PiC model for the correction.

        The fluid background satisfies the full, non-linear, resistive, compressible, Hall MHD equations. \cite{Laakmann_Hu_Farrell_2022} introduces finite-element(-in-space) implicit timesteppers for the incompressible analogue to this system with structure-preserving (SP) properties in the ideal case, alongside parameter-robust preconditioners. We show that these timesteppers can derive from a finite-element-in-time (FET) (and finite-element-in-space) interpretation. The benefits of this reformulation are discussed, including the derivation of timesteppers that are higher order in time, and the quantifiable dissipative SP properties in the non-ideal, resistive case.
        
        We discuss possible options for extending this FET approach to timesteppers for the compressible case.

        The kinetic corrections satisfy linearized Boltzmann equations. Using a Lénard--Bernstein collision operator, these take Fokker--Planck-like forms \cite{Fokker_1914, Planck_1917} wherein pseudo-particles in the numerical model obey the neoclassical transport equations, with particle-independent Brownian drift terms. This offers a rigorous methodology for incorporating collisions into the particle transport model, without coupling the equations of motions for each particle.
        
        Works by Chen, Chacón et al. \cite{Chen_Chacón_Barnes_2011, Chacón_Chen_Barnes_2013, Chen_Chacón_2014, Chen_Chacón_2015} have developed structure-preserving particle pushers for neoclassical transport in the Vlasov equations, derived from Crank--Nicolson integrators. We show these too can can derive from a FET interpretation, similarly offering potential extensions to higher-order-in-time particle pushers. The FET formulation is used also to consider how the stochastic drift terms can be incorporated into the pushers. Stochastic gyrokinetic expansions are also discussed.

        Different options for the numerical implementation of these schemes are considered.

        Due to the efficacy of FET in the development of SP timesteppers for both the fluid and kinetic component, we hope this approach will prove effective in the future for developing SP timesteppers for the full hybrid model. We hope this will give us the opportunity to incorporate previously inaccessible kinetic effects into the highly effective, modern, finite-element MHD models.
    \end{abstract}
    
    
    \newpage
    \tableofcontents
    
    
    \newpage
    \pagenumbering{arabic}
    %\linenumbers\renewcommand\thelinenumber{\color{black!50}\arabic{linenumber}}
            \documentclass[12pt, a4paper]{report}

\input{template/main.tex}

\title{\BA{Title in Progress...}}
\author{Boris Andrews}
\affil{Mathematical Institute, University of Oxford}
\date{\today}


\begin{document}
    \pagenumbering{gobble}
    \maketitle
    
    
    \begin{abstract}
        Magnetic confinement reactors---in particular tokamaks---offer one of the most promising options for achieving practical nuclear fusion, with the potential to provide virtually limitless, clean energy. The theoretical and numerical modeling of tokamak plasmas is simultaneously an essential component of effective reactor design, and a great research barrier. Tokamak operational conditions exhibit comparatively low Knudsen numbers. Kinetic effects, including kinetic waves and instabilities, Landau damping, bump-on-tail instabilities and more, are therefore highly influential in tokamak plasma dynamics. Purely fluid models are inherently incapable of capturing these effects, whereas the high dimensionality in purely kinetic models render them practically intractable for most relevant purposes.

        We consider a $\delta\!f$ decomposition model, with a macroscopic fluid background and microscopic kinetic correction, both fully coupled to each other. A similar manner of discretization is proposed to that used in the recent \texttt{STRUPHY} code \cite{Holderied_Possanner_Wang_2021, Holderied_2022, Li_et_al_2023} with a finite-element model for the background and a pseudo-particle/PiC model for the correction.

        The fluid background satisfies the full, non-linear, resistive, compressible, Hall MHD equations. \cite{Laakmann_Hu_Farrell_2022} introduces finite-element(-in-space) implicit timesteppers for the incompressible analogue to this system with structure-preserving (SP) properties in the ideal case, alongside parameter-robust preconditioners. We show that these timesteppers can derive from a finite-element-in-time (FET) (and finite-element-in-space) interpretation. The benefits of this reformulation are discussed, including the derivation of timesteppers that are higher order in time, and the quantifiable dissipative SP properties in the non-ideal, resistive case.
        
        We discuss possible options for extending this FET approach to timesteppers for the compressible case.

        The kinetic corrections satisfy linearized Boltzmann equations. Using a Lénard--Bernstein collision operator, these take Fokker--Planck-like forms \cite{Fokker_1914, Planck_1917} wherein pseudo-particles in the numerical model obey the neoclassical transport equations, with particle-independent Brownian drift terms. This offers a rigorous methodology for incorporating collisions into the particle transport model, without coupling the equations of motions for each particle.
        
        Works by Chen, Chacón et al. \cite{Chen_Chacón_Barnes_2011, Chacón_Chen_Barnes_2013, Chen_Chacón_2014, Chen_Chacón_2015} have developed structure-preserving particle pushers for neoclassical transport in the Vlasov equations, derived from Crank--Nicolson integrators. We show these too can can derive from a FET interpretation, similarly offering potential extensions to higher-order-in-time particle pushers. The FET formulation is used also to consider how the stochastic drift terms can be incorporated into the pushers. Stochastic gyrokinetic expansions are also discussed.

        Different options for the numerical implementation of these schemes are considered.

        Due to the efficacy of FET in the development of SP timesteppers for both the fluid and kinetic component, we hope this approach will prove effective in the future for developing SP timesteppers for the full hybrid model. We hope this will give us the opportunity to incorporate previously inaccessible kinetic effects into the highly effective, modern, finite-element MHD models.
    \end{abstract}
    
    
    \newpage
    \tableofcontents
    
    
    \newpage
    \pagenumbering{arabic}
    %\linenumbers\renewcommand\thelinenumber{\color{black!50}\arabic{linenumber}}
            \input{0 - introduction/main.tex}
        \part{Research}
            \input{1 - low-noise PiC models/main.tex}
            \input{2 - kinetic component/main.tex}
            \input{3 - fluid component/main.tex}
            \input{4 - numerical implementation/main.tex}
        \part{Project Overview}
            \input{5 - research plan/main.tex}
            \input{6 - summary/main.tex}
    
    
    %\section{}
    \newpage
    \pagenumbering{gobble}
        \printbibliography


    \newpage
    \pagenumbering{roman}
    \appendix
        \part{Appendices}
            \input{8 - Hilbert complexes/main.tex}
            \input{9 - weak conservation proofs/main.tex}
\end{document}

        \part{Research}
            \documentclass[12pt, a4paper]{report}

\input{template/main.tex}

\title{\BA{Title in Progress...}}
\author{Boris Andrews}
\affil{Mathematical Institute, University of Oxford}
\date{\today}


\begin{document}
    \pagenumbering{gobble}
    \maketitle
    
    
    \begin{abstract}
        Magnetic confinement reactors---in particular tokamaks---offer one of the most promising options for achieving practical nuclear fusion, with the potential to provide virtually limitless, clean energy. The theoretical and numerical modeling of tokamak plasmas is simultaneously an essential component of effective reactor design, and a great research barrier. Tokamak operational conditions exhibit comparatively low Knudsen numbers. Kinetic effects, including kinetic waves and instabilities, Landau damping, bump-on-tail instabilities and more, are therefore highly influential in tokamak plasma dynamics. Purely fluid models are inherently incapable of capturing these effects, whereas the high dimensionality in purely kinetic models render them practically intractable for most relevant purposes.

        We consider a $\delta\!f$ decomposition model, with a macroscopic fluid background and microscopic kinetic correction, both fully coupled to each other. A similar manner of discretization is proposed to that used in the recent \texttt{STRUPHY} code \cite{Holderied_Possanner_Wang_2021, Holderied_2022, Li_et_al_2023} with a finite-element model for the background and a pseudo-particle/PiC model for the correction.

        The fluid background satisfies the full, non-linear, resistive, compressible, Hall MHD equations. \cite{Laakmann_Hu_Farrell_2022} introduces finite-element(-in-space) implicit timesteppers for the incompressible analogue to this system with structure-preserving (SP) properties in the ideal case, alongside parameter-robust preconditioners. We show that these timesteppers can derive from a finite-element-in-time (FET) (and finite-element-in-space) interpretation. The benefits of this reformulation are discussed, including the derivation of timesteppers that are higher order in time, and the quantifiable dissipative SP properties in the non-ideal, resistive case.
        
        We discuss possible options for extending this FET approach to timesteppers for the compressible case.

        The kinetic corrections satisfy linearized Boltzmann equations. Using a Lénard--Bernstein collision operator, these take Fokker--Planck-like forms \cite{Fokker_1914, Planck_1917} wherein pseudo-particles in the numerical model obey the neoclassical transport equations, with particle-independent Brownian drift terms. This offers a rigorous methodology for incorporating collisions into the particle transport model, without coupling the equations of motions for each particle.
        
        Works by Chen, Chacón et al. \cite{Chen_Chacón_Barnes_2011, Chacón_Chen_Barnes_2013, Chen_Chacón_2014, Chen_Chacón_2015} have developed structure-preserving particle pushers for neoclassical transport in the Vlasov equations, derived from Crank--Nicolson integrators. We show these too can can derive from a FET interpretation, similarly offering potential extensions to higher-order-in-time particle pushers. The FET formulation is used also to consider how the stochastic drift terms can be incorporated into the pushers. Stochastic gyrokinetic expansions are also discussed.

        Different options for the numerical implementation of these schemes are considered.

        Due to the efficacy of FET in the development of SP timesteppers for both the fluid and kinetic component, we hope this approach will prove effective in the future for developing SP timesteppers for the full hybrid model. We hope this will give us the opportunity to incorporate previously inaccessible kinetic effects into the highly effective, modern, finite-element MHD models.
    \end{abstract}
    
    
    \newpage
    \tableofcontents
    
    
    \newpage
    \pagenumbering{arabic}
    %\linenumbers\renewcommand\thelinenumber{\color{black!50}\arabic{linenumber}}
            \input{0 - introduction/main.tex}
        \part{Research}
            \input{1 - low-noise PiC models/main.tex}
            \input{2 - kinetic component/main.tex}
            \input{3 - fluid component/main.tex}
            \input{4 - numerical implementation/main.tex}
        \part{Project Overview}
            \input{5 - research plan/main.tex}
            \input{6 - summary/main.tex}
    
    
    %\section{}
    \newpage
    \pagenumbering{gobble}
        \printbibliography


    \newpage
    \pagenumbering{roman}
    \appendix
        \part{Appendices}
            \input{8 - Hilbert complexes/main.tex}
            \input{9 - weak conservation proofs/main.tex}
\end{document}

            \documentclass[12pt, a4paper]{report}

\input{template/main.tex}

\title{\BA{Title in Progress...}}
\author{Boris Andrews}
\affil{Mathematical Institute, University of Oxford}
\date{\today}


\begin{document}
    \pagenumbering{gobble}
    \maketitle
    
    
    \begin{abstract}
        Magnetic confinement reactors---in particular tokamaks---offer one of the most promising options for achieving practical nuclear fusion, with the potential to provide virtually limitless, clean energy. The theoretical and numerical modeling of tokamak plasmas is simultaneously an essential component of effective reactor design, and a great research barrier. Tokamak operational conditions exhibit comparatively low Knudsen numbers. Kinetic effects, including kinetic waves and instabilities, Landau damping, bump-on-tail instabilities and more, are therefore highly influential in tokamak plasma dynamics. Purely fluid models are inherently incapable of capturing these effects, whereas the high dimensionality in purely kinetic models render them practically intractable for most relevant purposes.

        We consider a $\delta\!f$ decomposition model, with a macroscopic fluid background and microscopic kinetic correction, both fully coupled to each other. A similar manner of discretization is proposed to that used in the recent \texttt{STRUPHY} code \cite{Holderied_Possanner_Wang_2021, Holderied_2022, Li_et_al_2023} with a finite-element model for the background and a pseudo-particle/PiC model for the correction.

        The fluid background satisfies the full, non-linear, resistive, compressible, Hall MHD equations. \cite{Laakmann_Hu_Farrell_2022} introduces finite-element(-in-space) implicit timesteppers for the incompressible analogue to this system with structure-preserving (SP) properties in the ideal case, alongside parameter-robust preconditioners. We show that these timesteppers can derive from a finite-element-in-time (FET) (and finite-element-in-space) interpretation. The benefits of this reformulation are discussed, including the derivation of timesteppers that are higher order in time, and the quantifiable dissipative SP properties in the non-ideal, resistive case.
        
        We discuss possible options for extending this FET approach to timesteppers for the compressible case.

        The kinetic corrections satisfy linearized Boltzmann equations. Using a Lénard--Bernstein collision operator, these take Fokker--Planck-like forms \cite{Fokker_1914, Planck_1917} wherein pseudo-particles in the numerical model obey the neoclassical transport equations, with particle-independent Brownian drift terms. This offers a rigorous methodology for incorporating collisions into the particle transport model, without coupling the equations of motions for each particle.
        
        Works by Chen, Chacón et al. \cite{Chen_Chacón_Barnes_2011, Chacón_Chen_Barnes_2013, Chen_Chacón_2014, Chen_Chacón_2015} have developed structure-preserving particle pushers for neoclassical transport in the Vlasov equations, derived from Crank--Nicolson integrators. We show these too can can derive from a FET interpretation, similarly offering potential extensions to higher-order-in-time particle pushers. The FET formulation is used also to consider how the stochastic drift terms can be incorporated into the pushers. Stochastic gyrokinetic expansions are also discussed.

        Different options for the numerical implementation of these schemes are considered.

        Due to the efficacy of FET in the development of SP timesteppers for both the fluid and kinetic component, we hope this approach will prove effective in the future for developing SP timesteppers for the full hybrid model. We hope this will give us the opportunity to incorporate previously inaccessible kinetic effects into the highly effective, modern, finite-element MHD models.
    \end{abstract}
    
    
    \newpage
    \tableofcontents
    
    
    \newpage
    \pagenumbering{arabic}
    %\linenumbers\renewcommand\thelinenumber{\color{black!50}\arabic{linenumber}}
            \input{0 - introduction/main.tex}
        \part{Research}
            \input{1 - low-noise PiC models/main.tex}
            \input{2 - kinetic component/main.tex}
            \input{3 - fluid component/main.tex}
            \input{4 - numerical implementation/main.tex}
        \part{Project Overview}
            \input{5 - research plan/main.tex}
            \input{6 - summary/main.tex}
    
    
    %\section{}
    \newpage
    \pagenumbering{gobble}
        \printbibliography


    \newpage
    \pagenumbering{roman}
    \appendix
        \part{Appendices}
            \input{8 - Hilbert complexes/main.tex}
            \input{9 - weak conservation proofs/main.tex}
\end{document}

            \documentclass[12pt, a4paper]{report}

\input{template/main.tex}

\title{\BA{Title in Progress...}}
\author{Boris Andrews}
\affil{Mathematical Institute, University of Oxford}
\date{\today}


\begin{document}
    \pagenumbering{gobble}
    \maketitle
    
    
    \begin{abstract}
        Magnetic confinement reactors---in particular tokamaks---offer one of the most promising options for achieving practical nuclear fusion, with the potential to provide virtually limitless, clean energy. The theoretical and numerical modeling of tokamak plasmas is simultaneously an essential component of effective reactor design, and a great research barrier. Tokamak operational conditions exhibit comparatively low Knudsen numbers. Kinetic effects, including kinetic waves and instabilities, Landau damping, bump-on-tail instabilities and more, are therefore highly influential in tokamak plasma dynamics. Purely fluid models are inherently incapable of capturing these effects, whereas the high dimensionality in purely kinetic models render them practically intractable for most relevant purposes.

        We consider a $\delta\!f$ decomposition model, with a macroscopic fluid background and microscopic kinetic correction, both fully coupled to each other. A similar manner of discretization is proposed to that used in the recent \texttt{STRUPHY} code \cite{Holderied_Possanner_Wang_2021, Holderied_2022, Li_et_al_2023} with a finite-element model for the background and a pseudo-particle/PiC model for the correction.

        The fluid background satisfies the full, non-linear, resistive, compressible, Hall MHD equations. \cite{Laakmann_Hu_Farrell_2022} introduces finite-element(-in-space) implicit timesteppers for the incompressible analogue to this system with structure-preserving (SP) properties in the ideal case, alongside parameter-robust preconditioners. We show that these timesteppers can derive from a finite-element-in-time (FET) (and finite-element-in-space) interpretation. The benefits of this reformulation are discussed, including the derivation of timesteppers that are higher order in time, and the quantifiable dissipative SP properties in the non-ideal, resistive case.
        
        We discuss possible options for extending this FET approach to timesteppers for the compressible case.

        The kinetic corrections satisfy linearized Boltzmann equations. Using a Lénard--Bernstein collision operator, these take Fokker--Planck-like forms \cite{Fokker_1914, Planck_1917} wherein pseudo-particles in the numerical model obey the neoclassical transport equations, with particle-independent Brownian drift terms. This offers a rigorous methodology for incorporating collisions into the particle transport model, without coupling the equations of motions for each particle.
        
        Works by Chen, Chacón et al. \cite{Chen_Chacón_Barnes_2011, Chacón_Chen_Barnes_2013, Chen_Chacón_2014, Chen_Chacón_2015} have developed structure-preserving particle pushers for neoclassical transport in the Vlasov equations, derived from Crank--Nicolson integrators. We show these too can can derive from a FET interpretation, similarly offering potential extensions to higher-order-in-time particle pushers. The FET formulation is used also to consider how the stochastic drift terms can be incorporated into the pushers. Stochastic gyrokinetic expansions are also discussed.

        Different options for the numerical implementation of these schemes are considered.

        Due to the efficacy of FET in the development of SP timesteppers for both the fluid and kinetic component, we hope this approach will prove effective in the future for developing SP timesteppers for the full hybrid model. We hope this will give us the opportunity to incorporate previously inaccessible kinetic effects into the highly effective, modern, finite-element MHD models.
    \end{abstract}
    
    
    \newpage
    \tableofcontents
    
    
    \newpage
    \pagenumbering{arabic}
    %\linenumbers\renewcommand\thelinenumber{\color{black!50}\arabic{linenumber}}
            \input{0 - introduction/main.tex}
        \part{Research}
            \input{1 - low-noise PiC models/main.tex}
            \input{2 - kinetic component/main.tex}
            \input{3 - fluid component/main.tex}
            \input{4 - numerical implementation/main.tex}
        \part{Project Overview}
            \input{5 - research plan/main.tex}
            \input{6 - summary/main.tex}
    
    
    %\section{}
    \newpage
    \pagenumbering{gobble}
        \printbibliography


    \newpage
    \pagenumbering{roman}
    \appendix
        \part{Appendices}
            \input{8 - Hilbert complexes/main.tex}
            \input{9 - weak conservation proofs/main.tex}
\end{document}

            \documentclass[12pt, a4paper]{report}

\input{template/main.tex}

\title{\BA{Title in Progress...}}
\author{Boris Andrews}
\affil{Mathematical Institute, University of Oxford}
\date{\today}


\begin{document}
    \pagenumbering{gobble}
    \maketitle
    
    
    \begin{abstract}
        Magnetic confinement reactors---in particular tokamaks---offer one of the most promising options for achieving practical nuclear fusion, with the potential to provide virtually limitless, clean energy. The theoretical and numerical modeling of tokamak plasmas is simultaneously an essential component of effective reactor design, and a great research barrier. Tokamak operational conditions exhibit comparatively low Knudsen numbers. Kinetic effects, including kinetic waves and instabilities, Landau damping, bump-on-tail instabilities and more, are therefore highly influential in tokamak plasma dynamics. Purely fluid models are inherently incapable of capturing these effects, whereas the high dimensionality in purely kinetic models render them practically intractable for most relevant purposes.

        We consider a $\delta\!f$ decomposition model, with a macroscopic fluid background and microscopic kinetic correction, both fully coupled to each other. A similar manner of discretization is proposed to that used in the recent \texttt{STRUPHY} code \cite{Holderied_Possanner_Wang_2021, Holderied_2022, Li_et_al_2023} with a finite-element model for the background and a pseudo-particle/PiC model for the correction.

        The fluid background satisfies the full, non-linear, resistive, compressible, Hall MHD equations. \cite{Laakmann_Hu_Farrell_2022} introduces finite-element(-in-space) implicit timesteppers for the incompressible analogue to this system with structure-preserving (SP) properties in the ideal case, alongside parameter-robust preconditioners. We show that these timesteppers can derive from a finite-element-in-time (FET) (and finite-element-in-space) interpretation. The benefits of this reformulation are discussed, including the derivation of timesteppers that are higher order in time, and the quantifiable dissipative SP properties in the non-ideal, resistive case.
        
        We discuss possible options for extending this FET approach to timesteppers for the compressible case.

        The kinetic corrections satisfy linearized Boltzmann equations. Using a Lénard--Bernstein collision operator, these take Fokker--Planck-like forms \cite{Fokker_1914, Planck_1917} wherein pseudo-particles in the numerical model obey the neoclassical transport equations, with particle-independent Brownian drift terms. This offers a rigorous methodology for incorporating collisions into the particle transport model, without coupling the equations of motions for each particle.
        
        Works by Chen, Chacón et al. \cite{Chen_Chacón_Barnes_2011, Chacón_Chen_Barnes_2013, Chen_Chacón_2014, Chen_Chacón_2015} have developed structure-preserving particle pushers for neoclassical transport in the Vlasov equations, derived from Crank--Nicolson integrators. We show these too can can derive from a FET interpretation, similarly offering potential extensions to higher-order-in-time particle pushers. The FET formulation is used also to consider how the stochastic drift terms can be incorporated into the pushers. Stochastic gyrokinetic expansions are also discussed.

        Different options for the numerical implementation of these schemes are considered.

        Due to the efficacy of FET in the development of SP timesteppers for both the fluid and kinetic component, we hope this approach will prove effective in the future for developing SP timesteppers for the full hybrid model. We hope this will give us the opportunity to incorporate previously inaccessible kinetic effects into the highly effective, modern, finite-element MHD models.
    \end{abstract}
    
    
    \newpage
    \tableofcontents
    
    
    \newpage
    \pagenumbering{arabic}
    %\linenumbers\renewcommand\thelinenumber{\color{black!50}\arabic{linenumber}}
            \input{0 - introduction/main.tex}
        \part{Research}
            \input{1 - low-noise PiC models/main.tex}
            \input{2 - kinetic component/main.tex}
            \input{3 - fluid component/main.tex}
            \input{4 - numerical implementation/main.tex}
        \part{Project Overview}
            \input{5 - research plan/main.tex}
            \input{6 - summary/main.tex}
    
    
    %\section{}
    \newpage
    \pagenumbering{gobble}
        \printbibliography


    \newpage
    \pagenumbering{roman}
    \appendix
        \part{Appendices}
            \input{8 - Hilbert complexes/main.tex}
            \input{9 - weak conservation proofs/main.tex}
\end{document}

        \part{Project Overview}
            \documentclass[12pt, a4paper]{report}

\input{template/main.tex}

\title{\BA{Title in Progress...}}
\author{Boris Andrews}
\affil{Mathematical Institute, University of Oxford}
\date{\today}


\begin{document}
    \pagenumbering{gobble}
    \maketitle
    
    
    \begin{abstract}
        Magnetic confinement reactors---in particular tokamaks---offer one of the most promising options for achieving practical nuclear fusion, with the potential to provide virtually limitless, clean energy. The theoretical and numerical modeling of tokamak plasmas is simultaneously an essential component of effective reactor design, and a great research barrier. Tokamak operational conditions exhibit comparatively low Knudsen numbers. Kinetic effects, including kinetic waves and instabilities, Landau damping, bump-on-tail instabilities and more, are therefore highly influential in tokamak plasma dynamics. Purely fluid models are inherently incapable of capturing these effects, whereas the high dimensionality in purely kinetic models render them practically intractable for most relevant purposes.

        We consider a $\delta\!f$ decomposition model, with a macroscopic fluid background and microscopic kinetic correction, both fully coupled to each other. A similar manner of discretization is proposed to that used in the recent \texttt{STRUPHY} code \cite{Holderied_Possanner_Wang_2021, Holderied_2022, Li_et_al_2023} with a finite-element model for the background and a pseudo-particle/PiC model for the correction.

        The fluid background satisfies the full, non-linear, resistive, compressible, Hall MHD equations. \cite{Laakmann_Hu_Farrell_2022} introduces finite-element(-in-space) implicit timesteppers for the incompressible analogue to this system with structure-preserving (SP) properties in the ideal case, alongside parameter-robust preconditioners. We show that these timesteppers can derive from a finite-element-in-time (FET) (and finite-element-in-space) interpretation. The benefits of this reformulation are discussed, including the derivation of timesteppers that are higher order in time, and the quantifiable dissipative SP properties in the non-ideal, resistive case.
        
        We discuss possible options for extending this FET approach to timesteppers for the compressible case.

        The kinetic corrections satisfy linearized Boltzmann equations. Using a Lénard--Bernstein collision operator, these take Fokker--Planck-like forms \cite{Fokker_1914, Planck_1917} wherein pseudo-particles in the numerical model obey the neoclassical transport equations, with particle-independent Brownian drift terms. This offers a rigorous methodology for incorporating collisions into the particle transport model, without coupling the equations of motions for each particle.
        
        Works by Chen, Chacón et al. \cite{Chen_Chacón_Barnes_2011, Chacón_Chen_Barnes_2013, Chen_Chacón_2014, Chen_Chacón_2015} have developed structure-preserving particle pushers for neoclassical transport in the Vlasov equations, derived from Crank--Nicolson integrators. We show these too can can derive from a FET interpretation, similarly offering potential extensions to higher-order-in-time particle pushers. The FET formulation is used also to consider how the stochastic drift terms can be incorporated into the pushers. Stochastic gyrokinetic expansions are also discussed.

        Different options for the numerical implementation of these schemes are considered.

        Due to the efficacy of FET in the development of SP timesteppers for both the fluid and kinetic component, we hope this approach will prove effective in the future for developing SP timesteppers for the full hybrid model. We hope this will give us the opportunity to incorporate previously inaccessible kinetic effects into the highly effective, modern, finite-element MHD models.
    \end{abstract}
    
    
    \newpage
    \tableofcontents
    
    
    \newpage
    \pagenumbering{arabic}
    %\linenumbers\renewcommand\thelinenumber{\color{black!50}\arabic{linenumber}}
            \input{0 - introduction/main.tex}
        \part{Research}
            \input{1 - low-noise PiC models/main.tex}
            \input{2 - kinetic component/main.tex}
            \input{3 - fluid component/main.tex}
            \input{4 - numerical implementation/main.tex}
        \part{Project Overview}
            \input{5 - research plan/main.tex}
            \input{6 - summary/main.tex}
    
    
    %\section{}
    \newpage
    \pagenumbering{gobble}
        \printbibliography


    \newpage
    \pagenumbering{roman}
    \appendix
        \part{Appendices}
            \input{8 - Hilbert complexes/main.tex}
            \input{9 - weak conservation proofs/main.tex}
\end{document}

            \documentclass[12pt, a4paper]{report}

\input{template/main.tex}

\title{\BA{Title in Progress...}}
\author{Boris Andrews}
\affil{Mathematical Institute, University of Oxford}
\date{\today}


\begin{document}
    \pagenumbering{gobble}
    \maketitle
    
    
    \begin{abstract}
        Magnetic confinement reactors---in particular tokamaks---offer one of the most promising options for achieving practical nuclear fusion, with the potential to provide virtually limitless, clean energy. The theoretical and numerical modeling of tokamak plasmas is simultaneously an essential component of effective reactor design, and a great research barrier. Tokamak operational conditions exhibit comparatively low Knudsen numbers. Kinetic effects, including kinetic waves and instabilities, Landau damping, bump-on-tail instabilities and more, are therefore highly influential in tokamak plasma dynamics. Purely fluid models are inherently incapable of capturing these effects, whereas the high dimensionality in purely kinetic models render them practically intractable for most relevant purposes.

        We consider a $\delta\!f$ decomposition model, with a macroscopic fluid background and microscopic kinetic correction, both fully coupled to each other. A similar manner of discretization is proposed to that used in the recent \texttt{STRUPHY} code \cite{Holderied_Possanner_Wang_2021, Holderied_2022, Li_et_al_2023} with a finite-element model for the background and a pseudo-particle/PiC model for the correction.

        The fluid background satisfies the full, non-linear, resistive, compressible, Hall MHD equations. \cite{Laakmann_Hu_Farrell_2022} introduces finite-element(-in-space) implicit timesteppers for the incompressible analogue to this system with structure-preserving (SP) properties in the ideal case, alongside parameter-robust preconditioners. We show that these timesteppers can derive from a finite-element-in-time (FET) (and finite-element-in-space) interpretation. The benefits of this reformulation are discussed, including the derivation of timesteppers that are higher order in time, and the quantifiable dissipative SP properties in the non-ideal, resistive case.
        
        We discuss possible options for extending this FET approach to timesteppers for the compressible case.

        The kinetic corrections satisfy linearized Boltzmann equations. Using a Lénard--Bernstein collision operator, these take Fokker--Planck-like forms \cite{Fokker_1914, Planck_1917} wherein pseudo-particles in the numerical model obey the neoclassical transport equations, with particle-independent Brownian drift terms. This offers a rigorous methodology for incorporating collisions into the particle transport model, without coupling the equations of motions for each particle.
        
        Works by Chen, Chacón et al. \cite{Chen_Chacón_Barnes_2011, Chacón_Chen_Barnes_2013, Chen_Chacón_2014, Chen_Chacón_2015} have developed structure-preserving particle pushers for neoclassical transport in the Vlasov equations, derived from Crank--Nicolson integrators. We show these too can can derive from a FET interpretation, similarly offering potential extensions to higher-order-in-time particle pushers. The FET formulation is used also to consider how the stochastic drift terms can be incorporated into the pushers. Stochastic gyrokinetic expansions are also discussed.

        Different options for the numerical implementation of these schemes are considered.

        Due to the efficacy of FET in the development of SP timesteppers for both the fluid and kinetic component, we hope this approach will prove effective in the future for developing SP timesteppers for the full hybrid model. We hope this will give us the opportunity to incorporate previously inaccessible kinetic effects into the highly effective, modern, finite-element MHD models.
    \end{abstract}
    
    
    \newpage
    \tableofcontents
    
    
    \newpage
    \pagenumbering{arabic}
    %\linenumbers\renewcommand\thelinenumber{\color{black!50}\arabic{linenumber}}
            \input{0 - introduction/main.tex}
        \part{Research}
            \input{1 - low-noise PiC models/main.tex}
            \input{2 - kinetic component/main.tex}
            \input{3 - fluid component/main.tex}
            \input{4 - numerical implementation/main.tex}
        \part{Project Overview}
            \input{5 - research plan/main.tex}
            \input{6 - summary/main.tex}
    
    
    %\section{}
    \newpage
    \pagenumbering{gobble}
        \printbibliography


    \newpage
    \pagenumbering{roman}
    \appendix
        \part{Appendices}
            \input{8 - Hilbert complexes/main.tex}
            \input{9 - weak conservation proofs/main.tex}
\end{document}

    
    
    %\section{}
    \newpage
    \pagenumbering{gobble}
        \printbibliography


    \newpage
    \pagenumbering{roman}
    \appendix
        \part{Appendices}
            \documentclass[12pt, a4paper]{report}

\input{template/main.tex}

\title{\BA{Title in Progress...}}
\author{Boris Andrews}
\affil{Mathematical Institute, University of Oxford}
\date{\today}


\begin{document}
    \pagenumbering{gobble}
    \maketitle
    
    
    \begin{abstract}
        Magnetic confinement reactors---in particular tokamaks---offer one of the most promising options for achieving practical nuclear fusion, with the potential to provide virtually limitless, clean energy. The theoretical and numerical modeling of tokamak plasmas is simultaneously an essential component of effective reactor design, and a great research barrier. Tokamak operational conditions exhibit comparatively low Knudsen numbers. Kinetic effects, including kinetic waves and instabilities, Landau damping, bump-on-tail instabilities and more, are therefore highly influential in tokamak plasma dynamics. Purely fluid models are inherently incapable of capturing these effects, whereas the high dimensionality in purely kinetic models render them practically intractable for most relevant purposes.

        We consider a $\delta\!f$ decomposition model, with a macroscopic fluid background and microscopic kinetic correction, both fully coupled to each other. A similar manner of discretization is proposed to that used in the recent \texttt{STRUPHY} code \cite{Holderied_Possanner_Wang_2021, Holderied_2022, Li_et_al_2023} with a finite-element model for the background and a pseudo-particle/PiC model for the correction.

        The fluid background satisfies the full, non-linear, resistive, compressible, Hall MHD equations. \cite{Laakmann_Hu_Farrell_2022} introduces finite-element(-in-space) implicit timesteppers for the incompressible analogue to this system with structure-preserving (SP) properties in the ideal case, alongside parameter-robust preconditioners. We show that these timesteppers can derive from a finite-element-in-time (FET) (and finite-element-in-space) interpretation. The benefits of this reformulation are discussed, including the derivation of timesteppers that are higher order in time, and the quantifiable dissipative SP properties in the non-ideal, resistive case.
        
        We discuss possible options for extending this FET approach to timesteppers for the compressible case.

        The kinetic corrections satisfy linearized Boltzmann equations. Using a Lénard--Bernstein collision operator, these take Fokker--Planck-like forms \cite{Fokker_1914, Planck_1917} wherein pseudo-particles in the numerical model obey the neoclassical transport equations, with particle-independent Brownian drift terms. This offers a rigorous methodology for incorporating collisions into the particle transport model, without coupling the equations of motions for each particle.
        
        Works by Chen, Chacón et al. \cite{Chen_Chacón_Barnes_2011, Chacón_Chen_Barnes_2013, Chen_Chacón_2014, Chen_Chacón_2015} have developed structure-preserving particle pushers for neoclassical transport in the Vlasov equations, derived from Crank--Nicolson integrators. We show these too can can derive from a FET interpretation, similarly offering potential extensions to higher-order-in-time particle pushers. The FET formulation is used also to consider how the stochastic drift terms can be incorporated into the pushers. Stochastic gyrokinetic expansions are also discussed.

        Different options for the numerical implementation of these schemes are considered.

        Due to the efficacy of FET in the development of SP timesteppers for both the fluid and kinetic component, we hope this approach will prove effective in the future for developing SP timesteppers for the full hybrid model. We hope this will give us the opportunity to incorporate previously inaccessible kinetic effects into the highly effective, modern, finite-element MHD models.
    \end{abstract}
    
    
    \newpage
    \tableofcontents
    
    
    \newpage
    \pagenumbering{arabic}
    %\linenumbers\renewcommand\thelinenumber{\color{black!50}\arabic{linenumber}}
            \input{0 - introduction/main.tex}
        \part{Research}
            \input{1 - low-noise PiC models/main.tex}
            \input{2 - kinetic component/main.tex}
            \input{3 - fluid component/main.tex}
            \input{4 - numerical implementation/main.tex}
        \part{Project Overview}
            \input{5 - research plan/main.tex}
            \input{6 - summary/main.tex}
    
    
    %\section{}
    \newpage
    \pagenumbering{gobble}
        \printbibliography


    \newpage
    \pagenumbering{roman}
    \appendix
        \part{Appendices}
            \input{8 - Hilbert complexes/main.tex}
            \input{9 - weak conservation proofs/main.tex}
\end{document}

            \documentclass[12pt, a4paper]{report}

\input{template/main.tex}

\title{\BA{Title in Progress...}}
\author{Boris Andrews}
\affil{Mathematical Institute, University of Oxford}
\date{\today}


\begin{document}
    \pagenumbering{gobble}
    \maketitle
    
    
    \begin{abstract}
        Magnetic confinement reactors---in particular tokamaks---offer one of the most promising options for achieving practical nuclear fusion, with the potential to provide virtually limitless, clean energy. The theoretical and numerical modeling of tokamak plasmas is simultaneously an essential component of effective reactor design, and a great research barrier. Tokamak operational conditions exhibit comparatively low Knudsen numbers. Kinetic effects, including kinetic waves and instabilities, Landau damping, bump-on-tail instabilities and more, are therefore highly influential in tokamak plasma dynamics. Purely fluid models are inherently incapable of capturing these effects, whereas the high dimensionality in purely kinetic models render them practically intractable for most relevant purposes.

        We consider a $\delta\!f$ decomposition model, with a macroscopic fluid background and microscopic kinetic correction, both fully coupled to each other. A similar manner of discretization is proposed to that used in the recent \texttt{STRUPHY} code \cite{Holderied_Possanner_Wang_2021, Holderied_2022, Li_et_al_2023} with a finite-element model for the background and a pseudo-particle/PiC model for the correction.

        The fluid background satisfies the full, non-linear, resistive, compressible, Hall MHD equations. \cite{Laakmann_Hu_Farrell_2022} introduces finite-element(-in-space) implicit timesteppers for the incompressible analogue to this system with structure-preserving (SP) properties in the ideal case, alongside parameter-robust preconditioners. We show that these timesteppers can derive from a finite-element-in-time (FET) (and finite-element-in-space) interpretation. The benefits of this reformulation are discussed, including the derivation of timesteppers that are higher order in time, and the quantifiable dissipative SP properties in the non-ideal, resistive case.
        
        We discuss possible options for extending this FET approach to timesteppers for the compressible case.

        The kinetic corrections satisfy linearized Boltzmann equations. Using a Lénard--Bernstein collision operator, these take Fokker--Planck-like forms \cite{Fokker_1914, Planck_1917} wherein pseudo-particles in the numerical model obey the neoclassical transport equations, with particle-independent Brownian drift terms. This offers a rigorous methodology for incorporating collisions into the particle transport model, without coupling the equations of motions for each particle.
        
        Works by Chen, Chacón et al. \cite{Chen_Chacón_Barnes_2011, Chacón_Chen_Barnes_2013, Chen_Chacón_2014, Chen_Chacón_2015} have developed structure-preserving particle pushers for neoclassical transport in the Vlasov equations, derived from Crank--Nicolson integrators. We show these too can can derive from a FET interpretation, similarly offering potential extensions to higher-order-in-time particle pushers. The FET formulation is used also to consider how the stochastic drift terms can be incorporated into the pushers. Stochastic gyrokinetic expansions are also discussed.

        Different options for the numerical implementation of these schemes are considered.

        Due to the efficacy of FET in the development of SP timesteppers for both the fluid and kinetic component, we hope this approach will prove effective in the future for developing SP timesteppers for the full hybrid model. We hope this will give us the opportunity to incorporate previously inaccessible kinetic effects into the highly effective, modern, finite-element MHD models.
    \end{abstract}
    
    
    \newpage
    \tableofcontents
    
    
    \newpage
    \pagenumbering{arabic}
    %\linenumbers\renewcommand\thelinenumber{\color{black!50}\arabic{linenumber}}
            \input{0 - introduction/main.tex}
        \part{Research}
            \input{1 - low-noise PiC models/main.tex}
            \input{2 - kinetic component/main.tex}
            \input{3 - fluid component/main.tex}
            \input{4 - numerical implementation/main.tex}
        \part{Project Overview}
            \input{5 - research plan/main.tex}
            \input{6 - summary/main.tex}
    
    
    %\section{}
    \newpage
    \pagenumbering{gobble}
        \printbibliography


    \newpage
    \pagenumbering{roman}
    \appendix
        \part{Appendices}
            \input{8 - Hilbert complexes/main.tex}
            \input{9 - weak conservation proofs/main.tex}
\end{document}

\end{document}

            \documentclass[12pt, a4paper]{report}

\documentclass[12pt, a4paper]{report}

\input{template/main.tex}

\title{\BA{Title in Progress...}}
\author{Boris Andrews}
\affil{Mathematical Institute, University of Oxford}
\date{\today}


\begin{document}
    \pagenumbering{gobble}
    \maketitle
    
    
    \begin{abstract}
        Magnetic confinement reactors---in particular tokamaks---offer one of the most promising options for achieving practical nuclear fusion, with the potential to provide virtually limitless, clean energy. The theoretical and numerical modeling of tokamak plasmas is simultaneously an essential component of effective reactor design, and a great research barrier. Tokamak operational conditions exhibit comparatively low Knudsen numbers. Kinetic effects, including kinetic waves and instabilities, Landau damping, bump-on-tail instabilities and more, are therefore highly influential in tokamak plasma dynamics. Purely fluid models are inherently incapable of capturing these effects, whereas the high dimensionality in purely kinetic models render them practically intractable for most relevant purposes.

        We consider a $\delta\!f$ decomposition model, with a macroscopic fluid background and microscopic kinetic correction, both fully coupled to each other. A similar manner of discretization is proposed to that used in the recent \texttt{STRUPHY} code \cite{Holderied_Possanner_Wang_2021, Holderied_2022, Li_et_al_2023} with a finite-element model for the background and a pseudo-particle/PiC model for the correction.

        The fluid background satisfies the full, non-linear, resistive, compressible, Hall MHD equations. \cite{Laakmann_Hu_Farrell_2022} introduces finite-element(-in-space) implicit timesteppers for the incompressible analogue to this system with structure-preserving (SP) properties in the ideal case, alongside parameter-robust preconditioners. We show that these timesteppers can derive from a finite-element-in-time (FET) (and finite-element-in-space) interpretation. The benefits of this reformulation are discussed, including the derivation of timesteppers that are higher order in time, and the quantifiable dissipative SP properties in the non-ideal, resistive case.
        
        We discuss possible options for extending this FET approach to timesteppers for the compressible case.

        The kinetic corrections satisfy linearized Boltzmann equations. Using a Lénard--Bernstein collision operator, these take Fokker--Planck-like forms \cite{Fokker_1914, Planck_1917} wherein pseudo-particles in the numerical model obey the neoclassical transport equations, with particle-independent Brownian drift terms. This offers a rigorous methodology for incorporating collisions into the particle transport model, without coupling the equations of motions for each particle.
        
        Works by Chen, Chacón et al. \cite{Chen_Chacón_Barnes_2011, Chacón_Chen_Barnes_2013, Chen_Chacón_2014, Chen_Chacón_2015} have developed structure-preserving particle pushers for neoclassical transport in the Vlasov equations, derived from Crank--Nicolson integrators. We show these too can can derive from a FET interpretation, similarly offering potential extensions to higher-order-in-time particle pushers. The FET formulation is used also to consider how the stochastic drift terms can be incorporated into the pushers. Stochastic gyrokinetic expansions are also discussed.

        Different options for the numerical implementation of these schemes are considered.

        Due to the efficacy of FET in the development of SP timesteppers for both the fluid and kinetic component, we hope this approach will prove effective in the future for developing SP timesteppers for the full hybrid model. We hope this will give us the opportunity to incorporate previously inaccessible kinetic effects into the highly effective, modern, finite-element MHD models.
    \end{abstract}
    
    
    \newpage
    \tableofcontents
    
    
    \newpage
    \pagenumbering{arabic}
    %\linenumbers\renewcommand\thelinenumber{\color{black!50}\arabic{linenumber}}
            \input{0 - introduction/main.tex}
        \part{Research}
            \input{1 - low-noise PiC models/main.tex}
            \input{2 - kinetic component/main.tex}
            \input{3 - fluid component/main.tex}
            \input{4 - numerical implementation/main.tex}
        \part{Project Overview}
            \input{5 - research plan/main.tex}
            \input{6 - summary/main.tex}
    
    
    %\section{}
    \newpage
    \pagenumbering{gobble}
        \printbibliography


    \newpage
    \pagenumbering{roman}
    \appendix
        \part{Appendices}
            \input{8 - Hilbert complexes/main.tex}
            \input{9 - weak conservation proofs/main.tex}
\end{document}


\title{\BA{Title in Progress...}}
\author{Boris Andrews}
\affil{Mathematical Institute, University of Oxford}
\date{\today}


\begin{document}
    \pagenumbering{gobble}
    \maketitle
    
    
    \begin{abstract}
        Magnetic confinement reactors---in particular tokamaks---offer one of the most promising options for achieving practical nuclear fusion, with the potential to provide virtually limitless, clean energy. The theoretical and numerical modeling of tokamak plasmas is simultaneously an essential component of effective reactor design, and a great research barrier. Tokamak operational conditions exhibit comparatively low Knudsen numbers. Kinetic effects, including kinetic waves and instabilities, Landau damping, bump-on-tail instabilities and more, are therefore highly influential in tokamak plasma dynamics. Purely fluid models are inherently incapable of capturing these effects, whereas the high dimensionality in purely kinetic models render them practically intractable for most relevant purposes.

        We consider a $\delta\!f$ decomposition model, with a macroscopic fluid background and microscopic kinetic correction, both fully coupled to each other. A similar manner of discretization is proposed to that used in the recent \texttt{STRUPHY} code \cite{Holderied_Possanner_Wang_2021, Holderied_2022, Li_et_al_2023} with a finite-element model for the background and a pseudo-particle/PiC model for the correction.

        The fluid background satisfies the full, non-linear, resistive, compressible, Hall MHD equations. \cite{Laakmann_Hu_Farrell_2022} introduces finite-element(-in-space) implicit timesteppers for the incompressible analogue to this system with structure-preserving (SP) properties in the ideal case, alongside parameter-robust preconditioners. We show that these timesteppers can derive from a finite-element-in-time (FET) (and finite-element-in-space) interpretation. The benefits of this reformulation are discussed, including the derivation of timesteppers that are higher order in time, and the quantifiable dissipative SP properties in the non-ideal, resistive case.
        
        We discuss possible options for extending this FET approach to timesteppers for the compressible case.

        The kinetic corrections satisfy linearized Boltzmann equations. Using a Lénard--Bernstein collision operator, these take Fokker--Planck-like forms \cite{Fokker_1914, Planck_1917} wherein pseudo-particles in the numerical model obey the neoclassical transport equations, with particle-independent Brownian drift terms. This offers a rigorous methodology for incorporating collisions into the particle transport model, without coupling the equations of motions for each particle.
        
        Works by Chen, Chacón et al. \cite{Chen_Chacón_Barnes_2011, Chacón_Chen_Barnes_2013, Chen_Chacón_2014, Chen_Chacón_2015} have developed structure-preserving particle pushers for neoclassical transport in the Vlasov equations, derived from Crank--Nicolson integrators. We show these too can can derive from a FET interpretation, similarly offering potential extensions to higher-order-in-time particle pushers. The FET formulation is used also to consider how the stochastic drift terms can be incorporated into the pushers. Stochastic gyrokinetic expansions are also discussed.

        Different options for the numerical implementation of these schemes are considered.

        Due to the efficacy of FET in the development of SP timesteppers for both the fluid and kinetic component, we hope this approach will prove effective in the future for developing SP timesteppers for the full hybrid model. We hope this will give us the opportunity to incorporate previously inaccessible kinetic effects into the highly effective, modern, finite-element MHD models.
    \end{abstract}
    
    
    \newpage
    \tableofcontents
    
    
    \newpage
    \pagenumbering{arabic}
    %\linenumbers\renewcommand\thelinenumber{\color{black!50}\arabic{linenumber}}
            \documentclass[12pt, a4paper]{report}

\input{template/main.tex}

\title{\BA{Title in Progress...}}
\author{Boris Andrews}
\affil{Mathematical Institute, University of Oxford}
\date{\today}


\begin{document}
    \pagenumbering{gobble}
    \maketitle
    
    
    \begin{abstract}
        Magnetic confinement reactors---in particular tokamaks---offer one of the most promising options for achieving practical nuclear fusion, with the potential to provide virtually limitless, clean energy. The theoretical and numerical modeling of tokamak plasmas is simultaneously an essential component of effective reactor design, and a great research barrier. Tokamak operational conditions exhibit comparatively low Knudsen numbers. Kinetic effects, including kinetic waves and instabilities, Landau damping, bump-on-tail instabilities and more, are therefore highly influential in tokamak plasma dynamics. Purely fluid models are inherently incapable of capturing these effects, whereas the high dimensionality in purely kinetic models render them practically intractable for most relevant purposes.

        We consider a $\delta\!f$ decomposition model, with a macroscopic fluid background and microscopic kinetic correction, both fully coupled to each other. A similar manner of discretization is proposed to that used in the recent \texttt{STRUPHY} code \cite{Holderied_Possanner_Wang_2021, Holderied_2022, Li_et_al_2023} with a finite-element model for the background and a pseudo-particle/PiC model for the correction.

        The fluid background satisfies the full, non-linear, resistive, compressible, Hall MHD equations. \cite{Laakmann_Hu_Farrell_2022} introduces finite-element(-in-space) implicit timesteppers for the incompressible analogue to this system with structure-preserving (SP) properties in the ideal case, alongside parameter-robust preconditioners. We show that these timesteppers can derive from a finite-element-in-time (FET) (and finite-element-in-space) interpretation. The benefits of this reformulation are discussed, including the derivation of timesteppers that are higher order in time, and the quantifiable dissipative SP properties in the non-ideal, resistive case.
        
        We discuss possible options for extending this FET approach to timesteppers for the compressible case.

        The kinetic corrections satisfy linearized Boltzmann equations. Using a Lénard--Bernstein collision operator, these take Fokker--Planck-like forms \cite{Fokker_1914, Planck_1917} wherein pseudo-particles in the numerical model obey the neoclassical transport equations, with particle-independent Brownian drift terms. This offers a rigorous methodology for incorporating collisions into the particle transport model, without coupling the equations of motions for each particle.
        
        Works by Chen, Chacón et al. \cite{Chen_Chacón_Barnes_2011, Chacón_Chen_Barnes_2013, Chen_Chacón_2014, Chen_Chacón_2015} have developed structure-preserving particle pushers for neoclassical transport in the Vlasov equations, derived from Crank--Nicolson integrators. We show these too can can derive from a FET interpretation, similarly offering potential extensions to higher-order-in-time particle pushers. The FET formulation is used also to consider how the stochastic drift terms can be incorporated into the pushers. Stochastic gyrokinetic expansions are also discussed.

        Different options for the numerical implementation of these schemes are considered.

        Due to the efficacy of FET in the development of SP timesteppers for both the fluid and kinetic component, we hope this approach will prove effective in the future for developing SP timesteppers for the full hybrid model. We hope this will give us the opportunity to incorporate previously inaccessible kinetic effects into the highly effective, modern, finite-element MHD models.
    \end{abstract}
    
    
    \newpage
    \tableofcontents
    
    
    \newpage
    \pagenumbering{arabic}
    %\linenumbers\renewcommand\thelinenumber{\color{black!50}\arabic{linenumber}}
            \input{0 - introduction/main.tex}
        \part{Research}
            \input{1 - low-noise PiC models/main.tex}
            \input{2 - kinetic component/main.tex}
            \input{3 - fluid component/main.tex}
            \input{4 - numerical implementation/main.tex}
        \part{Project Overview}
            \input{5 - research plan/main.tex}
            \input{6 - summary/main.tex}
    
    
    %\section{}
    \newpage
    \pagenumbering{gobble}
        \printbibliography


    \newpage
    \pagenumbering{roman}
    \appendix
        \part{Appendices}
            \input{8 - Hilbert complexes/main.tex}
            \input{9 - weak conservation proofs/main.tex}
\end{document}

        \part{Research}
            \documentclass[12pt, a4paper]{report}

\input{template/main.tex}

\title{\BA{Title in Progress...}}
\author{Boris Andrews}
\affil{Mathematical Institute, University of Oxford}
\date{\today}


\begin{document}
    \pagenumbering{gobble}
    \maketitle
    
    
    \begin{abstract}
        Magnetic confinement reactors---in particular tokamaks---offer one of the most promising options for achieving practical nuclear fusion, with the potential to provide virtually limitless, clean energy. The theoretical and numerical modeling of tokamak plasmas is simultaneously an essential component of effective reactor design, and a great research barrier. Tokamak operational conditions exhibit comparatively low Knudsen numbers. Kinetic effects, including kinetic waves and instabilities, Landau damping, bump-on-tail instabilities and more, are therefore highly influential in tokamak plasma dynamics. Purely fluid models are inherently incapable of capturing these effects, whereas the high dimensionality in purely kinetic models render them practically intractable for most relevant purposes.

        We consider a $\delta\!f$ decomposition model, with a macroscopic fluid background and microscopic kinetic correction, both fully coupled to each other. A similar manner of discretization is proposed to that used in the recent \texttt{STRUPHY} code \cite{Holderied_Possanner_Wang_2021, Holderied_2022, Li_et_al_2023} with a finite-element model for the background and a pseudo-particle/PiC model for the correction.

        The fluid background satisfies the full, non-linear, resistive, compressible, Hall MHD equations. \cite{Laakmann_Hu_Farrell_2022} introduces finite-element(-in-space) implicit timesteppers for the incompressible analogue to this system with structure-preserving (SP) properties in the ideal case, alongside parameter-robust preconditioners. We show that these timesteppers can derive from a finite-element-in-time (FET) (and finite-element-in-space) interpretation. The benefits of this reformulation are discussed, including the derivation of timesteppers that are higher order in time, and the quantifiable dissipative SP properties in the non-ideal, resistive case.
        
        We discuss possible options for extending this FET approach to timesteppers for the compressible case.

        The kinetic corrections satisfy linearized Boltzmann equations. Using a Lénard--Bernstein collision operator, these take Fokker--Planck-like forms \cite{Fokker_1914, Planck_1917} wherein pseudo-particles in the numerical model obey the neoclassical transport equations, with particle-independent Brownian drift terms. This offers a rigorous methodology for incorporating collisions into the particle transport model, without coupling the equations of motions for each particle.
        
        Works by Chen, Chacón et al. \cite{Chen_Chacón_Barnes_2011, Chacón_Chen_Barnes_2013, Chen_Chacón_2014, Chen_Chacón_2015} have developed structure-preserving particle pushers for neoclassical transport in the Vlasov equations, derived from Crank--Nicolson integrators. We show these too can can derive from a FET interpretation, similarly offering potential extensions to higher-order-in-time particle pushers. The FET formulation is used also to consider how the stochastic drift terms can be incorporated into the pushers. Stochastic gyrokinetic expansions are also discussed.

        Different options for the numerical implementation of these schemes are considered.

        Due to the efficacy of FET in the development of SP timesteppers for both the fluid and kinetic component, we hope this approach will prove effective in the future for developing SP timesteppers for the full hybrid model. We hope this will give us the opportunity to incorporate previously inaccessible kinetic effects into the highly effective, modern, finite-element MHD models.
    \end{abstract}
    
    
    \newpage
    \tableofcontents
    
    
    \newpage
    \pagenumbering{arabic}
    %\linenumbers\renewcommand\thelinenumber{\color{black!50}\arabic{linenumber}}
            \input{0 - introduction/main.tex}
        \part{Research}
            \input{1 - low-noise PiC models/main.tex}
            \input{2 - kinetic component/main.tex}
            \input{3 - fluid component/main.tex}
            \input{4 - numerical implementation/main.tex}
        \part{Project Overview}
            \input{5 - research plan/main.tex}
            \input{6 - summary/main.tex}
    
    
    %\section{}
    \newpage
    \pagenumbering{gobble}
        \printbibliography


    \newpage
    \pagenumbering{roman}
    \appendix
        \part{Appendices}
            \input{8 - Hilbert complexes/main.tex}
            \input{9 - weak conservation proofs/main.tex}
\end{document}

            \documentclass[12pt, a4paper]{report}

\input{template/main.tex}

\title{\BA{Title in Progress...}}
\author{Boris Andrews}
\affil{Mathematical Institute, University of Oxford}
\date{\today}


\begin{document}
    \pagenumbering{gobble}
    \maketitle
    
    
    \begin{abstract}
        Magnetic confinement reactors---in particular tokamaks---offer one of the most promising options for achieving practical nuclear fusion, with the potential to provide virtually limitless, clean energy. The theoretical and numerical modeling of tokamak plasmas is simultaneously an essential component of effective reactor design, and a great research barrier. Tokamak operational conditions exhibit comparatively low Knudsen numbers. Kinetic effects, including kinetic waves and instabilities, Landau damping, bump-on-tail instabilities and more, are therefore highly influential in tokamak plasma dynamics. Purely fluid models are inherently incapable of capturing these effects, whereas the high dimensionality in purely kinetic models render them practically intractable for most relevant purposes.

        We consider a $\delta\!f$ decomposition model, with a macroscopic fluid background and microscopic kinetic correction, both fully coupled to each other. A similar manner of discretization is proposed to that used in the recent \texttt{STRUPHY} code \cite{Holderied_Possanner_Wang_2021, Holderied_2022, Li_et_al_2023} with a finite-element model for the background and a pseudo-particle/PiC model for the correction.

        The fluid background satisfies the full, non-linear, resistive, compressible, Hall MHD equations. \cite{Laakmann_Hu_Farrell_2022} introduces finite-element(-in-space) implicit timesteppers for the incompressible analogue to this system with structure-preserving (SP) properties in the ideal case, alongside parameter-robust preconditioners. We show that these timesteppers can derive from a finite-element-in-time (FET) (and finite-element-in-space) interpretation. The benefits of this reformulation are discussed, including the derivation of timesteppers that are higher order in time, and the quantifiable dissipative SP properties in the non-ideal, resistive case.
        
        We discuss possible options for extending this FET approach to timesteppers for the compressible case.

        The kinetic corrections satisfy linearized Boltzmann equations. Using a Lénard--Bernstein collision operator, these take Fokker--Planck-like forms \cite{Fokker_1914, Planck_1917} wherein pseudo-particles in the numerical model obey the neoclassical transport equations, with particle-independent Brownian drift terms. This offers a rigorous methodology for incorporating collisions into the particle transport model, without coupling the equations of motions for each particle.
        
        Works by Chen, Chacón et al. \cite{Chen_Chacón_Barnes_2011, Chacón_Chen_Barnes_2013, Chen_Chacón_2014, Chen_Chacón_2015} have developed structure-preserving particle pushers for neoclassical transport in the Vlasov equations, derived from Crank--Nicolson integrators. We show these too can can derive from a FET interpretation, similarly offering potential extensions to higher-order-in-time particle pushers. The FET formulation is used also to consider how the stochastic drift terms can be incorporated into the pushers. Stochastic gyrokinetic expansions are also discussed.

        Different options for the numerical implementation of these schemes are considered.

        Due to the efficacy of FET in the development of SP timesteppers for both the fluid and kinetic component, we hope this approach will prove effective in the future for developing SP timesteppers for the full hybrid model. We hope this will give us the opportunity to incorporate previously inaccessible kinetic effects into the highly effective, modern, finite-element MHD models.
    \end{abstract}
    
    
    \newpage
    \tableofcontents
    
    
    \newpage
    \pagenumbering{arabic}
    %\linenumbers\renewcommand\thelinenumber{\color{black!50}\arabic{linenumber}}
            \input{0 - introduction/main.tex}
        \part{Research}
            \input{1 - low-noise PiC models/main.tex}
            \input{2 - kinetic component/main.tex}
            \input{3 - fluid component/main.tex}
            \input{4 - numerical implementation/main.tex}
        \part{Project Overview}
            \input{5 - research plan/main.tex}
            \input{6 - summary/main.tex}
    
    
    %\section{}
    \newpage
    \pagenumbering{gobble}
        \printbibliography


    \newpage
    \pagenumbering{roman}
    \appendix
        \part{Appendices}
            \input{8 - Hilbert complexes/main.tex}
            \input{9 - weak conservation proofs/main.tex}
\end{document}

            \documentclass[12pt, a4paper]{report}

\input{template/main.tex}

\title{\BA{Title in Progress...}}
\author{Boris Andrews}
\affil{Mathematical Institute, University of Oxford}
\date{\today}


\begin{document}
    \pagenumbering{gobble}
    \maketitle
    
    
    \begin{abstract}
        Magnetic confinement reactors---in particular tokamaks---offer one of the most promising options for achieving practical nuclear fusion, with the potential to provide virtually limitless, clean energy. The theoretical and numerical modeling of tokamak plasmas is simultaneously an essential component of effective reactor design, and a great research barrier. Tokamak operational conditions exhibit comparatively low Knudsen numbers. Kinetic effects, including kinetic waves and instabilities, Landau damping, bump-on-tail instabilities and more, are therefore highly influential in tokamak plasma dynamics. Purely fluid models are inherently incapable of capturing these effects, whereas the high dimensionality in purely kinetic models render them practically intractable for most relevant purposes.

        We consider a $\delta\!f$ decomposition model, with a macroscopic fluid background and microscopic kinetic correction, both fully coupled to each other. A similar manner of discretization is proposed to that used in the recent \texttt{STRUPHY} code \cite{Holderied_Possanner_Wang_2021, Holderied_2022, Li_et_al_2023} with a finite-element model for the background and a pseudo-particle/PiC model for the correction.

        The fluid background satisfies the full, non-linear, resistive, compressible, Hall MHD equations. \cite{Laakmann_Hu_Farrell_2022} introduces finite-element(-in-space) implicit timesteppers for the incompressible analogue to this system with structure-preserving (SP) properties in the ideal case, alongside parameter-robust preconditioners. We show that these timesteppers can derive from a finite-element-in-time (FET) (and finite-element-in-space) interpretation. The benefits of this reformulation are discussed, including the derivation of timesteppers that are higher order in time, and the quantifiable dissipative SP properties in the non-ideal, resistive case.
        
        We discuss possible options for extending this FET approach to timesteppers for the compressible case.

        The kinetic corrections satisfy linearized Boltzmann equations. Using a Lénard--Bernstein collision operator, these take Fokker--Planck-like forms \cite{Fokker_1914, Planck_1917} wherein pseudo-particles in the numerical model obey the neoclassical transport equations, with particle-independent Brownian drift terms. This offers a rigorous methodology for incorporating collisions into the particle transport model, without coupling the equations of motions for each particle.
        
        Works by Chen, Chacón et al. \cite{Chen_Chacón_Barnes_2011, Chacón_Chen_Barnes_2013, Chen_Chacón_2014, Chen_Chacón_2015} have developed structure-preserving particle pushers for neoclassical transport in the Vlasov equations, derived from Crank--Nicolson integrators. We show these too can can derive from a FET interpretation, similarly offering potential extensions to higher-order-in-time particle pushers. The FET formulation is used also to consider how the stochastic drift terms can be incorporated into the pushers. Stochastic gyrokinetic expansions are also discussed.

        Different options for the numerical implementation of these schemes are considered.

        Due to the efficacy of FET in the development of SP timesteppers for both the fluid and kinetic component, we hope this approach will prove effective in the future for developing SP timesteppers for the full hybrid model. We hope this will give us the opportunity to incorporate previously inaccessible kinetic effects into the highly effective, modern, finite-element MHD models.
    \end{abstract}
    
    
    \newpage
    \tableofcontents
    
    
    \newpage
    \pagenumbering{arabic}
    %\linenumbers\renewcommand\thelinenumber{\color{black!50}\arabic{linenumber}}
            \input{0 - introduction/main.tex}
        \part{Research}
            \input{1 - low-noise PiC models/main.tex}
            \input{2 - kinetic component/main.tex}
            \input{3 - fluid component/main.tex}
            \input{4 - numerical implementation/main.tex}
        \part{Project Overview}
            \input{5 - research plan/main.tex}
            \input{6 - summary/main.tex}
    
    
    %\section{}
    \newpage
    \pagenumbering{gobble}
        \printbibliography


    \newpage
    \pagenumbering{roman}
    \appendix
        \part{Appendices}
            \input{8 - Hilbert complexes/main.tex}
            \input{9 - weak conservation proofs/main.tex}
\end{document}

            \documentclass[12pt, a4paper]{report}

\input{template/main.tex}

\title{\BA{Title in Progress...}}
\author{Boris Andrews}
\affil{Mathematical Institute, University of Oxford}
\date{\today}


\begin{document}
    \pagenumbering{gobble}
    \maketitle
    
    
    \begin{abstract}
        Magnetic confinement reactors---in particular tokamaks---offer one of the most promising options for achieving practical nuclear fusion, with the potential to provide virtually limitless, clean energy. The theoretical and numerical modeling of tokamak plasmas is simultaneously an essential component of effective reactor design, and a great research barrier. Tokamak operational conditions exhibit comparatively low Knudsen numbers. Kinetic effects, including kinetic waves and instabilities, Landau damping, bump-on-tail instabilities and more, are therefore highly influential in tokamak plasma dynamics. Purely fluid models are inherently incapable of capturing these effects, whereas the high dimensionality in purely kinetic models render them practically intractable for most relevant purposes.

        We consider a $\delta\!f$ decomposition model, with a macroscopic fluid background and microscopic kinetic correction, both fully coupled to each other. A similar manner of discretization is proposed to that used in the recent \texttt{STRUPHY} code \cite{Holderied_Possanner_Wang_2021, Holderied_2022, Li_et_al_2023} with a finite-element model for the background and a pseudo-particle/PiC model for the correction.

        The fluid background satisfies the full, non-linear, resistive, compressible, Hall MHD equations. \cite{Laakmann_Hu_Farrell_2022} introduces finite-element(-in-space) implicit timesteppers for the incompressible analogue to this system with structure-preserving (SP) properties in the ideal case, alongside parameter-robust preconditioners. We show that these timesteppers can derive from a finite-element-in-time (FET) (and finite-element-in-space) interpretation. The benefits of this reformulation are discussed, including the derivation of timesteppers that are higher order in time, and the quantifiable dissipative SP properties in the non-ideal, resistive case.
        
        We discuss possible options for extending this FET approach to timesteppers for the compressible case.

        The kinetic corrections satisfy linearized Boltzmann equations. Using a Lénard--Bernstein collision operator, these take Fokker--Planck-like forms \cite{Fokker_1914, Planck_1917} wherein pseudo-particles in the numerical model obey the neoclassical transport equations, with particle-independent Brownian drift terms. This offers a rigorous methodology for incorporating collisions into the particle transport model, without coupling the equations of motions for each particle.
        
        Works by Chen, Chacón et al. \cite{Chen_Chacón_Barnes_2011, Chacón_Chen_Barnes_2013, Chen_Chacón_2014, Chen_Chacón_2015} have developed structure-preserving particle pushers for neoclassical transport in the Vlasov equations, derived from Crank--Nicolson integrators. We show these too can can derive from a FET interpretation, similarly offering potential extensions to higher-order-in-time particle pushers. The FET formulation is used also to consider how the stochastic drift terms can be incorporated into the pushers. Stochastic gyrokinetic expansions are also discussed.

        Different options for the numerical implementation of these schemes are considered.

        Due to the efficacy of FET in the development of SP timesteppers for both the fluid and kinetic component, we hope this approach will prove effective in the future for developing SP timesteppers for the full hybrid model. We hope this will give us the opportunity to incorporate previously inaccessible kinetic effects into the highly effective, modern, finite-element MHD models.
    \end{abstract}
    
    
    \newpage
    \tableofcontents
    
    
    \newpage
    \pagenumbering{arabic}
    %\linenumbers\renewcommand\thelinenumber{\color{black!50}\arabic{linenumber}}
            \input{0 - introduction/main.tex}
        \part{Research}
            \input{1 - low-noise PiC models/main.tex}
            \input{2 - kinetic component/main.tex}
            \input{3 - fluid component/main.tex}
            \input{4 - numerical implementation/main.tex}
        \part{Project Overview}
            \input{5 - research plan/main.tex}
            \input{6 - summary/main.tex}
    
    
    %\section{}
    \newpage
    \pagenumbering{gobble}
        \printbibliography


    \newpage
    \pagenumbering{roman}
    \appendix
        \part{Appendices}
            \input{8 - Hilbert complexes/main.tex}
            \input{9 - weak conservation proofs/main.tex}
\end{document}

        \part{Project Overview}
            \documentclass[12pt, a4paper]{report}

\input{template/main.tex}

\title{\BA{Title in Progress...}}
\author{Boris Andrews}
\affil{Mathematical Institute, University of Oxford}
\date{\today}


\begin{document}
    \pagenumbering{gobble}
    \maketitle
    
    
    \begin{abstract}
        Magnetic confinement reactors---in particular tokamaks---offer one of the most promising options for achieving practical nuclear fusion, with the potential to provide virtually limitless, clean energy. The theoretical and numerical modeling of tokamak plasmas is simultaneously an essential component of effective reactor design, and a great research barrier. Tokamak operational conditions exhibit comparatively low Knudsen numbers. Kinetic effects, including kinetic waves and instabilities, Landau damping, bump-on-tail instabilities and more, are therefore highly influential in tokamak plasma dynamics. Purely fluid models are inherently incapable of capturing these effects, whereas the high dimensionality in purely kinetic models render them practically intractable for most relevant purposes.

        We consider a $\delta\!f$ decomposition model, with a macroscopic fluid background and microscopic kinetic correction, both fully coupled to each other. A similar manner of discretization is proposed to that used in the recent \texttt{STRUPHY} code \cite{Holderied_Possanner_Wang_2021, Holderied_2022, Li_et_al_2023} with a finite-element model for the background and a pseudo-particle/PiC model for the correction.

        The fluid background satisfies the full, non-linear, resistive, compressible, Hall MHD equations. \cite{Laakmann_Hu_Farrell_2022} introduces finite-element(-in-space) implicit timesteppers for the incompressible analogue to this system with structure-preserving (SP) properties in the ideal case, alongside parameter-robust preconditioners. We show that these timesteppers can derive from a finite-element-in-time (FET) (and finite-element-in-space) interpretation. The benefits of this reformulation are discussed, including the derivation of timesteppers that are higher order in time, and the quantifiable dissipative SP properties in the non-ideal, resistive case.
        
        We discuss possible options for extending this FET approach to timesteppers for the compressible case.

        The kinetic corrections satisfy linearized Boltzmann equations. Using a Lénard--Bernstein collision operator, these take Fokker--Planck-like forms \cite{Fokker_1914, Planck_1917} wherein pseudo-particles in the numerical model obey the neoclassical transport equations, with particle-independent Brownian drift terms. This offers a rigorous methodology for incorporating collisions into the particle transport model, without coupling the equations of motions for each particle.
        
        Works by Chen, Chacón et al. \cite{Chen_Chacón_Barnes_2011, Chacón_Chen_Barnes_2013, Chen_Chacón_2014, Chen_Chacón_2015} have developed structure-preserving particle pushers for neoclassical transport in the Vlasov equations, derived from Crank--Nicolson integrators. We show these too can can derive from a FET interpretation, similarly offering potential extensions to higher-order-in-time particle pushers. The FET formulation is used also to consider how the stochastic drift terms can be incorporated into the pushers. Stochastic gyrokinetic expansions are also discussed.

        Different options for the numerical implementation of these schemes are considered.

        Due to the efficacy of FET in the development of SP timesteppers for both the fluid and kinetic component, we hope this approach will prove effective in the future for developing SP timesteppers for the full hybrid model. We hope this will give us the opportunity to incorporate previously inaccessible kinetic effects into the highly effective, modern, finite-element MHD models.
    \end{abstract}
    
    
    \newpage
    \tableofcontents
    
    
    \newpage
    \pagenumbering{arabic}
    %\linenumbers\renewcommand\thelinenumber{\color{black!50}\arabic{linenumber}}
            \input{0 - introduction/main.tex}
        \part{Research}
            \input{1 - low-noise PiC models/main.tex}
            \input{2 - kinetic component/main.tex}
            \input{3 - fluid component/main.tex}
            \input{4 - numerical implementation/main.tex}
        \part{Project Overview}
            \input{5 - research plan/main.tex}
            \input{6 - summary/main.tex}
    
    
    %\section{}
    \newpage
    \pagenumbering{gobble}
        \printbibliography


    \newpage
    \pagenumbering{roman}
    \appendix
        \part{Appendices}
            \input{8 - Hilbert complexes/main.tex}
            \input{9 - weak conservation proofs/main.tex}
\end{document}

            \documentclass[12pt, a4paper]{report}

\input{template/main.tex}

\title{\BA{Title in Progress...}}
\author{Boris Andrews}
\affil{Mathematical Institute, University of Oxford}
\date{\today}


\begin{document}
    \pagenumbering{gobble}
    \maketitle
    
    
    \begin{abstract}
        Magnetic confinement reactors---in particular tokamaks---offer one of the most promising options for achieving practical nuclear fusion, with the potential to provide virtually limitless, clean energy. The theoretical and numerical modeling of tokamak plasmas is simultaneously an essential component of effective reactor design, and a great research barrier. Tokamak operational conditions exhibit comparatively low Knudsen numbers. Kinetic effects, including kinetic waves and instabilities, Landau damping, bump-on-tail instabilities and more, are therefore highly influential in tokamak plasma dynamics. Purely fluid models are inherently incapable of capturing these effects, whereas the high dimensionality in purely kinetic models render them practically intractable for most relevant purposes.

        We consider a $\delta\!f$ decomposition model, with a macroscopic fluid background and microscopic kinetic correction, both fully coupled to each other. A similar manner of discretization is proposed to that used in the recent \texttt{STRUPHY} code \cite{Holderied_Possanner_Wang_2021, Holderied_2022, Li_et_al_2023} with a finite-element model for the background and a pseudo-particle/PiC model for the correction.

        The fluid background satisfies the full, non-linear, resistive, compressible, Hall MHD equations. \cite{Laakmann_Hu_Farrell_2022} introduces finite-element(-in-space) implicit timesteppers for the incompressible analogue to this system with structure-preserving (SP) properties in the ideal case, alongside parameter-robust preconditioners. We show that these timesteppers can derive from a finite-element-in-time (FET) (and finite-element-in-space) interpretation. The benefits of this reformulation are discussed, including the derivation of timesteppers that are higher order in time, and the quantifiable dissipative SP properties in the non-ideal, resistive case.
        
        We discuss possible options for extending this FET approach to timesteppers for the compressible case.

        The kinetic corrections satisfy linearized Boltzmann equations. Using a Lénard--Bernstein collision operator, these take Fokker--Planck-like forms \cite{Fokker_1914, Planck_1917} wherein pseudo-particles in the numerical model obey the neoclassical transport equations, with particle-independent Brownian drift terms. This offers a rigorous methodology for incorporating collisions into the particle transport model, without coupling the equations of motions for each particle.
        
        Works by Chen, Chacón et al. \cite{Chen_Chacón_Barnes_2011, Chacón_Chen_Barnes_2013, Chen_Chacón_2014, Chen_Chacón_2015} have developed structure-preserving particle pushers for neoclassical transport in the Vlasov equations, derived from Crank--Nicolson integrators. We show these too can can derive from a FET interpretation, similarly offering potential extensions to higher-order-in-time particle pushers. The FET formulation is used also to consider how the stochastic drift terms can be incorporated into the pushers. Stochastic gyrokinetic expansions are also discussed.

        Different options for the numerical implementation of these schemes are considered.

        Due to the efficacy of FET in the development of SP timesteppers for both the fluid and kinetic component, we hope this approach will prove effective in the future for developing SP timesteppers for the full hybrid model. We hope this will give us the opportunity to incorporate previously inaccessible kinetic effects into the highly effective, modern, finite-element MHD models.
    \end{abstract}
    
    
    \newpage
    \tableofcontents
    
    
    \newpage
    \pagenumbering{arabic}
    %\linenumbers\renewcommand\thelinenumber{\color{black!50}\arabic{linenumber}}
            \input{0 - introduction/main.tex}
        \part{Research}
            \input{1 - low-noise PiC models/main.tex}
            \input{2 - kinetic component/main.tex}
            \input{3 - fluid component/main.tex}
            \input{4 - numerical implementation/main.tex}
        \part{Project Overview}
            \input{5 - research plan/main.tex}
            \input{6 - summary/main.tex}
    
    
    %\section{}
    \newpage
    \pagenumbering{gobble}
        \printbibliography


    \newpage
    \pagenumbering{roman}
    \appendix
        \part{Appendices}
            \input{8 - Hilbert complexes/main.tex}
            \input{9 - weak conservation proofs/main.tex}
\end{document}

    
    
    %\section{}
    \newpage
    \pagenumbering{gobble}
        \printbibliography


    \newpage
    \pagenumbering{roman}
    \appendix
        \part{Appendices}
            \documentclass[12pt, a4paper]{report}

\input{template/main.tex}

\title{\BA{Title in Progress...}}
\author{Boris Andrews}
\affil{Mathematical Institute, University of Oxford}
\date{\today}


\begin{document}
    \pagenumbering{gobble}
    \maketitle
    
    
    \begin{abstract}
        Magnetic confinement reactors---in particular tokamaks---offer one of the most promising options for achieving practical nuclear fusion, with the potential to provide virtually limitless, clean energy. The theoretical and numerical modeling of tokamak plasmas is simultaneously an essential component of effective reactor design, and a great research barrier. Tokamak operational conditions exhibit comparatively low Knudsen numbers. Kinetic effects, including kinetic waves and instabilities, Landau damping, bump-on-tail instabilities and more, are therefore highly influential in tokamak plasma dynamics. Purely fluid models are inherently incapable of capturing these effects, whereas the high dimensionality in purely kinetic models render them practically intractable for most relevant purposes.

        We consider a $\delta\!f$ decomposition model, with a macroscopic fluid background and microscopic kinetic correction, both fully coupled to each other. A similar manner of discretization is proposed to that used in the recent \texttt{STRUPHY} code \cite{Holderied_Possanner_Wang_2021, Holderied_2022, Li_et_al_2023} with a finite-element model for the background and a pseudo-particle/PiC model for the correction.

        The fluid background satisfies the full, non-linear, resistive, compressible, Hall MHD equations. \cite{Laakmann_Hu_Farrell_2022} introduces finite-element(-in-space) implicit timesteppers for the incompressible analogue to this system with structure-preserving (SP) properties in the ideal case, alongside parameter-robust preconditioners. We show that these timesteppers can derive from a finite-element-in-time (FET) (and finite-element-in-space) interpretation. The benefits of this reformulation are discussed, including the derivation of timesteppers that are higher order in time, and the quantifiable dissipative SP properties in the non-ideal, resistive case.
        
        We discuss possible options for extending this FET approach to timesteppers for the compressible case.

        The kinetic corrections satisfy linearized Boltzmann equations. Using a Lénard--Bernstein collision operator, these take Fokker--Planck-like forms \cite{Fokker_1914, Planck_1917} wherein pseudo-particles in the numerical model obey the neoclassical transport equations, with particle-independent Brownian drift terms. This offers a rigorous methodology for incorporating collisions into the particle transport model, without coupling the equations of motions for each particle.
        
        Works by Chen, Chacón et al. \cite{Chen_Chacón_Barnes_2011, Chacón_Chen_Barnes_2013, Chen_Chacón_2014, Chen_Chacón_2015} have developed structure-preserving particle pushers for neoclassical transport in the Vlasov equations, derived from Crank--Nicolson integrators. We show these too can can derive from a FET interpretation, similarly offering potential extensions to higher-order-in-time particle pushers. The FET formulation is used also to consider how the stochastic drift terms can be incorporated into the pushers. Stochastic gyrokinetic expansions are also discussed.

        Different options for the numerical implementation of these schemes are considered.

        Due to the efficacy of FET in the development of SP timesteppers for both the fluid and kinetic component, we hope this approach will prove effective in the future for developing SP timesteppers for the full hybrid model. We hope this will give us the opportunity to incorporate previously inaccessible kinetic effects into the highly effective, modern, finite-element MHD models.
    \end{abstract}
    
    
    \newpage
    \tableofcontents
    
    
    \newpage
    \pagenumbering{arabic}
    %\linenumbers\renewcommand\thelinenumber{\color{black!50}\arabic{linenumber}}
            \input{0 - introduction/main.tex}
        \part{Research}
            \input{1 - low-noise PiC models/main.tex}
            \input{2 - kinetic component/main.tex}
            \input{3 - fluid component/main.tex}
            \input{4 - numerical implementation/main.tex}
        \part{Project Overview}
            \input{5 - research plan/main.tex}
            \input{6 - summary/main.tex}
    
    
    %\section{}
    \newpage
    \pagenumbering{gobble}
        \printbibliography


    \newpage
    \pagenumbering{roman}
    \appendix
        \part{Appendices}
            \input{8 - Hilbert complexes/main.tex}
            \input{9 - weak conservation proofs/main.tex}
\end{document}

            \documentclass[12pt, a4paper]{report}

\input{template/main.tex}

\title{\BA{Title in Progress...}}
\author{Boris Andrews}
\affil{Mathematical Institute, University of Oxford}
\date{\today}


\begin{document}
    \pagenumbering{gobble}
    \maketitle
    
    
    \begin{abstract}
        Magnetic confinement reactors---in particular tokamaks---offer one of the most promising options for achieving practical nuclear fusion, with the potential to provide virtually limitless, clean energy. The theoretical and numerical modeling of tokamak plasmas is simultaneously an essential component of effective reactor design, and a great research barrier. Tokamak operational conditions exhibit comparatively low Knudsen numbers. Kinetic effects, including kinetic waves and instabilities, Landau damping, bump-on-tail instabilities and more, are therefore highly influential in tokamak plasma dynamics. Purely fluid models are inherently incapable of capturing these effects, whereas the high dimensionality in purely kinetic models render them practically intractable for most relevant purposes.

        We consider a $\delta\!f$ decomposition model, with a macroscopic fluid background and microscopic kinetic correction, both fully coupled to each other. A similar manner of discretization is proposed to that used in the recent \texttt{STRUPHY} code \cite{Holderied_Possanner_Wang_2021, Holderied_2022, Li_et_al_2023} with a finite-element model for the background and a pseudo-particle/PiC model for the correction.

        The fluid background satisfies the full, non-linear, resistive, compressible, Hall MHD equations. \cite{Laakmann_Hu_Farrell_2022} introduces finite-element(-in-space) implicit timesteppers for the incompressible analogue to this system with structure-preserving (SP) properties in the ideal case, alongside parameter-robust preconditioners. We show that these timesteppers can derive from a finite-element-in-time (FET) (and finite-element-in-space) interpretation. The benefits of this reformulation are discussed, including the derivation of timesteppers that are higher order in time, and the quantifiable dissipative SP properties in the non-ideal, resistive case.
        
        We discuss possible options for extending this FET approach to timesteppers for the compressible case.

        The kinetic corrections satisfy linearized Boltzmann equations. Using a Lénard--Bernstein collision operator, these take Fokker--Planck-like forms \cite{Fokker_1914, Planck_1917} wherein pseudo-particles in the numerical model obey the neoclassical transport equations, with particle-independent Brownian drift terms. This offers a rigorous methodology for incorporating collisions into the particle transport model, without coupling the equations of motions for each particle.
        
        Works by Chen, Chacón et al. \cite{Chen_Chacón_Barnes_2011, Chacón_Chen_Barnes_2013, Chen_Chacón_2014, Chen_Chacón_2015} have developed structure-preserving particle pushers for neoclassical transport in the Vlasov equations, derived from Crank--Nicolson integrators. We show these too can can derive from a FET interpretation, similarly offering potential extensions to higher-order-in-time particle pushers. The FET formulation is used also to consider how the stochastic drift terms can be incorporated into the pushers. Stochastic gyrokinetic expansions are also discussed.

        Different options for the numerical implementation of these schemes are considered.

        Due to the efficacy of FET in the development of SP timesteppers for both the fluid and kinetic component, we hope this approach will prove effective in the future for developing SP timesteppers for the full hybrid model. We hope this will give us the opportunity to incorporate previously inaccessible kinetic effects into the highly effective, modern, finite-element MHD models.
    \end{abstract}
    
    
    \newpage
    \tableofcontents
    
    
    \newpage
    \pagenumbering{arabic}
    %\linenumbers\renewcommand\thelinenumber{\color{black!50}\arabic{linenumber}}
            \input{0 - introduction/main.tex}
        \part{Research}
            \input{1 - low-noise PiC models/main.tex}
            \input{2 - kinetic component/main.tex}
            \input{3 - fluid component/main.tex}
            \input{4 - numerical implementation/main.tex}
        \part{Project Overview}
            \input{5 - research plan/main.tex}
            \input{6 - summary/main.tex}
    
    
    %\section{}
    \newpage
    \pagenumbering{gobble}
        \printbibliography


    \newpage
    \pagenumbering{roman}
    \appendix
        \part{Appendices}
            \input{8 - Hilbert complexes/main.tex}
            \input{9 - weak conservation proofs/main.tex}
\end{document}

\end{document}

\end{document}

    \documentclass[12pt, a4paper]{report}

\documentclass[12pt, a4paper]{report}

\documentclass[12pt, a4paper]{report}

\input{template/main.tex}

\title{\BA{Title in Progress...}}
\author{Boris Andrews}
\affil{Mathematical Institute, University of Oxford}
\date{\today}


\begin{document}
    \pagenumbering{gobble}
    \maketitle
    
    
    \begin{abstract}
        Magnetic confinement reactors---in particular tokamaks---offer one of the most promising options for achieving practical nuclear fusion, with the potential to provide virtually limitless, clean energy. The theoretical and numerical modeling of tokamak plasmas is simultaneously an essential component of effective reactor design, and a great research barrier. Tokamak operational conditions exhibit comparatively low Knudsen numbers. Kinetic effects, including kinetic waves and instabilities, Landau damping, bump-on-tail instabilities and more, are therefore highly influential in tokamak plasma dynamics. Purely fluid models are inherently incapable of capturing these effects, whereas the high dimensionality in purely kinetic models render them practically intractable for most relevant purposes.

        We consider a $\delta\!f$ decomposition model, with a macroscopic fluid background and microscopic kinetic correction, both fully coupled to each other. A similar manner of discretization is proposed to that used in the recent \texttt{STRUPHY} code \cite{Holderied_Possanner_Wang_2021, Holderied_2022, Li_et_al_2023} with a finite-element model for the background and a pseudo-particle/PiC model for the correction.

        The fluid background satisfies the full, non-linear, resistive, compressible, Hall MHD equations. \cite{Laakmann_Hu_Farrell_2022} introduces finite-element(-in-space) implicit timesteppers for the incompressible analogue to this system with structure-preserving (SP) properties in the ideal case, alongside parameter-robust preconditioners. We show that these timesteppers can derive from a finite-element-in-time (FET) (and finite-element-in-space) interpretation. The benefits of this reformulation are discussed, including the derivation of timesteppers that are higher order in time, and the quantifiable dissipative SP properties in the non-ideal, resistive case.
        
        We discuss possible options for extending this FET approach to timesteppers for the compressible case.

        The kinetic corrections satisfy linearized Boltzmann equations. Using a Lénard--Bernstein collision operator, these take Fokker--Planck-like forms \cite{Fokker_1914, Planck_1917} wherein pseudo-particles in the numerical model obey the neoclassical transport equations, with particle-independent Brownian drift terms. This offers a rigorous methodology for incorporating collisions into the particle transport model, without coupling the equations of motions for each particle.
        
        Works by Chen, Chacón et al. \cite{Chen_Chacón_Barnes_2011, Chacón_Chen_Barnes_2013, Chen_Chacón_2014, Chen_Chacón_2015} have developed structure-preserving particle pushers for neoclassical transport in the Vlasov equations, derived from Crank--Nicolson integrators. We show these too can can derive from a FET interpretation, similarly offering potential extensions to higher-order-in-time particle pushers. The FET formulation is used also to consider how the stochastic drift terms can be incorporated into the pushers. Stochastic gyrokinetic expansions are also discussed.

        Different options for the numerical implementation of these schemes are considered.

        Due to the efficacy of FET in the development of SP timesteppers for both the fluid and kinetic component, we hope this approach will prove effective in the future for developing SP timesteppers for the full hybrid model. We hope this will give us the opportunity to incorporate previously inaccessible kinetic effects into the highly effective, modern, finite-element MHD models.
    \end{abstract}
    
    
    \newpage
    \tableofcontents
    
    
    \newpage
    \pagenumbering{arabic}
    %\linenumbers\renewcommand\thelinenumber{\color{black!50}\arabic{linenumber}}
            \input{0 - introduction/main.tex}
        \part{Research}
            \input{1 - low-noise PiC models/main.tex}
            \input{2 - kinetic component/main.tex}
            \input{3 - fluid component/main.tex}
            \input{4 - numerical implementation/main.tex}
        \part{Project Overview}
            \input{5 - research plan/main.tex}
            \input{6 - summary/main.tex}
    
    
    %\section{}
    \newpage
    \pagenumbering{gobble}
        \printbibliography


    \newpage
    \pagenumbering{roman}
    \appendix
        \part{Appendices}
            \input{8 - Hilbert complexes/main.tex}
            \input{9 - weak conservation proofs/main.tex}
\end{document}


\title{\BA{Title in Progress...}}
\author{Boris Andrews}
\affil{Mathematical Institute, University of Oxford}
\date{\today}


\begin{document}
    \pagenumbering{gobble}
    \maketitle
    
    
    \begin{abstract}
        Magnetic confinement reactors---in particular tokamaks---offer one of the most promising options for achieving practical nuclear fusion, with the potential to provide virtually limitless, clean energy. The theoretical and numerical modeling of tokamak plasmas is simultaneously an essential component of effective reactor design, and a great research barrier. Tokamak operational conditions exhibit comparatively low Knudsen numbers. Kinetic effects, including kinetic waves and instabilities, Landau damping, bump-on-tail instabilities and more, are therefore highly influential in tokamak plasma dynamics. Purely fluid models are inherently incapable of capturing these effects, whereas the high dimensionality in purely kinetic models render them practically intractable for most relevant purposes.

        We consider a $\delta\!f$ decomposition model, with a macroscopic fluid background and microscopic kinetic correction, both fully coupled to each other. A similar manner of discretization is proposed to that used in the recent \texttt{STRUPHY} code \cite{Holderied_Possanner_Wang_2021, Holderied_2022, Li_et_al_2023} with a finite-element model for the background and a pseudo-particle/PiC model for the correction.

        The fluid background satisfies the full, non-linear, resistive, compressible, Hall MHD equations. \cite{Laakmann_Hu_Farrell_2022} introduces finite-element(-in-space) implicit timesteppers for the incompressible analogue to this system with structure-preserving (SP) properties in the ideal case, alongside parameter-robust preconditioners. We show that these timesteppers can derive from a finite-element-in-time (FET) (and finite-element-in-space) interpretation. The benefits of this reformulation are discussed, including the derivation of timesteppers that are higher order in time, and the quantifiable dissipative SP properties in the non-ideal, resistive case.
        
        We discuss possible options for extending this FET approach to timesteppers for the compressible case.

        The kinetic corrections satisfy linearized Boltzmann equations. Using a Lénard--Bernstein collision operator, these take Fokker--Planck-like forms \cite{Fokker_1914, Planck_1917} wherein pseudo-particles in the numerical model obey the neoclassical transport equations, with particle-independent Brownian drift terms. This offers a rigorous methodology for incorporating collisions into the particle transport model, without coupling the equations of motions for each particle.
        
        Works by Chen, Chacón et al. \cite{Chen_Chacón_Barnes_2011, Chacón_Chen_Barnes_2013, Chen_Chacón_2014, Chen_Chacón_2015} have developed structure-preserving particle pushers for neoclassical transport in the Vlasov equations, derived from Crank--Nicolson integrators. We show these too can can derive from a FET interpretation, similarly offering potential extensions to higher-order-in-time particle pushers. The FET formulation is used also to consider how the stochastic drift terms can be incorporated into the pushers. Stochastic gyrokinetic expansions are also discussed.

        Different options for the numerical implementation of these schemes are considered.

        Due to the efficacy of FET in the development of SP timesteppers for both the fluid and kinetic component, we hope this approach will prove effective in the future for developing SP timesteppers for the full hybrid model. We hope this will give us the opportunity to incorporate previously inaccessible kinetic effects into the highly effective, modern, finite-element MHD models.
    \end{abstract}
    
    
    \newpage
    \tableofcontents
    
    
    \newpage
    \pagenumbering{arabic}
    %\linenumbers\renewcommand\thelinenumber{\color{black!50}\arabic{linenumber}}
            \documentclass[12pt, a4paper]{report}

\input{template/main.tex}

\title{\BA{Title in Progress...}}
\author{Boris Andrews}
\affil{Mathematical Institute, University of Oxford}
\date{\today}


\begin{document}
    \pagenumbering{gobble}
    \maketitle
    
    
    \begin{abstract}
        Magnetic confinement reactors---in particular tokamaks---offer one of the most promising options for achieving practical nuclear fusion, with the potential to provide virtually limitless, clean energy. The theoretical and numerical modeling of tokamak plasmas is simultaneously an essential component of effective reactor design, and a great research barrier. Tokamak operational conditions exhibit comparatively low Knudsen numbers. Kinetic effects, including kinetic waves and instabilities, Landau damping, bump-on-tail instabilities and more, are therefore highly influential in tokamak plasma dynamics. Purely fluid models are inherently incapable of capturing these effects, whereas the high dimensionality in purely kinetic models render them practically intractable for most relevant purposes.

        We consider a $\delta\!f$ decomposition model, with a macroscopic fluid background and microscopic kinetic correction, both fully coupled to each other. A similar manner of discretization is proposed to that used in the recent \texttt{STRUPHY} code \cite{Holderied_Possanner_Wang_2021, Holderied_2022, Li_et_al_2023} with a finite-element model for the background and a pseudo-particle/PiC model for the correction.

        The fluid background satisfies the full, non-linear, resistive, compressible, Hall MHD equations. \cite{Laakmann_Hu_Farrell_2022} introduces finite-element(-in-space) implicit timesteppers for the incompressible analogue to this system with structure-preserving (SP) properties in the ideal case, alongside parameter-robust preconditioners. We show that these timesteppers can derive from a finite-element-in-time (FET) (and finite-element-in-space) interpretation. The benefits of this reformulation are discussed, including the derivation of timesteppers that are higher order in time, and the quantifiable dissipative SP properties in the non-ideal, resistive case.
        
        We discuss possible options for extending this FET approach to timesteppers for the compressible case.

        The kinetic corrections satisfy linearized Boltzmann equations. Using a Lénard--Bernstein collision operator, these take Fokker--Planck-like forms \cite{Fokker_1914, Planck_1917} wherein pseudo-particles in the numerical model obey the neoclassical transport equations, with particle-independent Brownian drift terms. This offers a rigorous methodology for incorporating collisions into the particle transport model, without coupling the equations of motions for each particle.
        
        Works by Chen, Chacón et al. \cite{Chen_Chacón_Barnes_2011, Chacón_Chen_Barnes_2013, Chen_Chacón_2014, Chen_Chacón_2015} have developed structure-preserving particle pushers for neoclassical transport in the Vlasov equations, derived from Crank--Nicolson integrators. We show these too can can derive from a FET interpretation, similarly offering potential extensions to higher-order-in-time particle pushers. The FET formulation is used also to consider how the stochastic drift terms can be incorporated into the pushers. Stochastic gyrokinetic expansions are also discussed.

        Different options for the numerical implementation of these schemes are considered.

        Due to the efficacy of FET in the development of SP timesteppers for both the fluid and kinetic component, we hope this approach will prove effective in the future for developing SP timesteppers for the full hybrid model. We hope this will give us the opportunity to incorporate previously inaccessible kinetic effects into the highly effective, modern, finite-element MHD models.
    \end{abstract}
    
    
    \newpage
    \tableofcontents
    
    
    \newpage
    \pagenumbering{arabic}
    %\linenumbers\renewcommand\thelinenumber{\color{black!50}\arabic{linenumber}}
            \input{0 - introduction/main.tex}
        \part{Research}
            \input{1 - low-noise PiC models/main.tex}
            \input{2 - kinetic component/main.tex}
            \input{3 - fluid component/main.tex}
            \input{4 - numerical implementation/main.tex}
        \part{Project Overview}
            \input{5 - research plan/main.tex}
            \input{6 - summary/main.tex}
    
    
    %\section{}
    \newpage
    \pagenumbering{gobble}
        \printbibliography


    \newpage
    \pagenumbering{roman}
    \appendix
        \part{Appendices}
            \input{8 - Hilbert complexes/main.tex}
            \input{9 - weak conservation proofs/main.tex}
\end{document}

        \part{Research}
            \documentclass[12pt, a4paper]{report}

\input{template/main.tex}

\title{\BA{Title in Progress...}}
\author{Boris Andrews}
\affil{Mathematical Institute, University of Oxford}
\date{\today}


\begin{document}
    \pagenumbering{gobble}
    \maketitle
    
    
    \begin{abstract}
        Magnetic confinement reactors---in particular tokamaks---offer one of the most promising options for achieving practical nuclear fusion, with the potential to provide virtually limitless, clean energy. The theoretical and numerical modeling of tokamak plasmas is simultaneously an essential component of effective reactor design, and a great research barrier. Tokamak operational conditions exhibit comparatively low Knudsen numbers. Kinetic effects, including kinetic waves and instabilities, Landau damping, bump-on-tail instabilities and more, are therefore highly influential in tokamak plasma dynamics. Purely fluid models are inherently incapable of capturing these effects, whereas the high dimensionality in purely kinetic models render them practically intractable for most relevant purposes.

        We consider a $\delta\!f$ decomposition model, with a macroscopic fluid background and microscopic kinetic correction, both fully coupled to each other. A similar manner of discretization is proposed to that used in the recent \texttt{STRUPHY} code \cite{Holderied_Possanner_Wang_2021, Holderied_2022, Li_et_al_2023} with a finite-element model for the background and a pseudo-particle/PiC model for the correction.

        The fluid background satisfies the full, non-linear, resistive, compressible, Hall MHD equations. \cite{Laakmann_Hu_Farrell_2022} introduces finite-element(-in-space) implicit timesteppers for the incompressible analogue to this system with structure-preserving (SP) properties in the ideal case, alongside parameter-robust preconditioners. We show that these timesteppers can derive from a finite-element-in-time (FET) (and finite-element-in-space) interpretation. The benefits of this reformulation are discussed, including the derivation of timesteppers that are higher order in time, and the quantifiable dissipative SP properties in the non-ideal, resistive case.
        
        We discuss possible options for extending this FET approach to timesteppers for the compressible case.

        The kinetic corrections satisfy linearized Boltzmann equations. Using a Lénard--Bernstein collision operator, these take Fokker--Planck-like forms \cite{Fokker_1914, Planck_1917} wherein pseudo-particles in the numerical model obey the neoclassical transport equations, with particle-independent Brownian drift terms. This offers a rigorous methodology for incorporating collisions into the particle transport model, without coupling the equations of motions for each particle.
        
        Works by Chen, Chacón et al. \cite{Chen_Chacón_Barnes_2011, Chacón_Chen_Barnes_2013, Chen_Chacón_2014, Chen_Chacón_2015} have developed structure-preserving particle pushers for neoclassical transport in the Vlasov equations, derived from Crank--Nicolson integrators. We show these too can can derive from a FET interpretation, similarly offering potential extensions to higher-order-in-time particle pushers. The FET formulation is used also to consider how the stochastic drift terms can be incorporated into the pushers. Stochastic gyrokinetic expansions are also discussed.

        Different options for the numerical implementation of these schemes are considered.

        Due to the efficacy of FET in the development of SP timesteppers for both the fluid and kinetic component, we hope this approach will prove effective in the future for developing SP timesteppers for the full hybrid model. We hope this will give us the opportunity to incorporate previously inaccessible kinetic effects into the highly effective, modern, finite-element MHD models.
    \end{abstract}
    
    
    \newpage
    \tableofcontents
    
    
    \newpage
    \pagenumbering{arabic}
    %\linenumbers\renewcommand\thelinenumber{\color{black!50}\arabic{linenumber}}
            \input{0 - introduction/main.tex}
        \part{Research}
            \input{1 - low-noise PiC models/main.tex}
            \input{2 - kinetic component/main.tex}
            \input{3 - fluid component/main.tex}
            \input{4 - numerical implementation/main.tex}
        \part{Project Overview}
            \input{5 - research plan/main.tex}
            \input{6 - summary/main.tex}
    
    
    %\section{}
    \newpage
    \pagenumbering{gobble}
        \printbibliography


    \newpage
    \pagenumbering{roman}
    \appendix
        \part{Appendices}
            \input{8 - Hilbert complexes/main.tex}
            \input{9 - weak conservation proofs/main.tex}
\end{document}

            \documentclass[12pt, a4paper]{report}

\input{template/main.tex}

\title{\BA{Title in Progress...}}
\author{Boris Andrews}
\affil{Mathematical Institute, University of Oxford}
\date{\today}


\begin{document}
    \pagenumbering{gobble}
    \maketitle
    
    
    \begin{abstract}
        Magnetic confinement reactors---in particular tokamaks---offer one of the most promising options for achieving practical nuclear fusion, with the potential to provide virtually limitless, clean energy. The theoretical and numerical modeling of tokamak plasmas is simultaneously an essential component of effective reactor design, and a great research barrier. Tokamak operational conditions exhibit comparatively low Knudsen numbers. Kinetic effects, including kinetic waves and instabilities, Landau damping, bump-on-tail instabilities and more, are therefore highly influential in tokamak plasma dynamics. Purely fluid models are inherently incapable of capturing these effects, whereas the high dimensionality in purely kinetic models render them practically intractable for most relevant purposes.

        We consider a $\delta\!f$ decomposition model, with a macroscopic fluid background and microscopic kinetic correction, both fully coupled to each other. A similar manner of discretization is proposed to that used in the recent \texttt{STRUPHY} code \cite{Holderied_Possanner_Wang_2021, Holderied_2022, Li_et_al_2023} with a finite-element model for the background and a pseudo-particle/PiC model for the correction.

        The fluid background satisfies the full, non-linear, resistive, compressible, Hall MHD equations. \cite{Laakmann_Hu_Farrell_2022} introduces finite-element(-in-space) implicit timesteppers for the incompressible analogue to this system with structure-preserving (SP) properties in the ideal case, alongside parameter-robust preconditioners. We show that these timesteppers can derive from a finite-element-in-time (FET) (and finite-element-in-space) interpretation. The benefits of this reformulation are discussed, including the derivation of timesteppers that are higher order in time, and the quantifiable dissipative SP properties in the non-ideal, resistive case.
        
        We discuss possible options for extending this FET approach to timesteppers for the compressible case.

        The kinetic corrections satisfy linearized Boltzmann equations. Using a Lénard--Bernstein collision operator, these take Fokker--Planck-like forms \cite{Fokker_1914, Planck_1917} wherein pseudo-particles in the numerical model obey the neoclassical transport equations, with particle-independent Brownian drift terms. This offers a rigorous methodology for incorporating collisions into the particle transport model, without coupling the equations of motions for each particle.
        
        Works by Chen, Chacón et al. \cite{Chen_Chacón_Barnes_2011, Chacón_Chen_Barnes_2013, Chen_Chacón_2014, Chen_Chacón_2015} have developed structure-preserving particle pushers for neoclassical transport in the Vlasov equations, derived from Crank--Nicolson integrators. We show these too can can derive from a FET interpretation, similarly offering potential extensions to higher-order-in-time particle pushers. The FET formulation is used also to consider how the stochastic drift terms can be incorporated into the pushers. Stochastic gyrokinetic expansions are also discussed.

        Different options for the numerical implementation of these schemes are considered.

        Due to the efficacy of FET in the development of SP timesteppers for both the fluid and kinetic component, we hope this approach will prove effective in the future for developing SP timesteppers for the full hybrid model. We hope this will give us the opportunity to incorporate previously inaccessible kinetic effects into the highly effective, modern, finite-element MHD models.
    \end{abstract}
    
    
    \newpage
    \tableofcontents
    
    
    \newpage
    \pagenumbering{arabic}
    %\linenumbers\renewcommand\thelinenumber{\color{black!50}\arabic{linenumber}}
            \input{0 - introduction/main.tex}
        \part{Research}
            \input{1 - low-noise PiC models/main.tex}
            \input{2 - kinetic component/main.tex}
            \input{3 - fluid component/main.tex}
            \input{4 - numerical implementation/main.tex}
        \part{Project Overview}
            \input{5 - research plan/main.tex}
            \input{6 - summary/main.tex}
    
    
    %\section{}
    \newpage
    \pagenumbering{gobble}
        \printbibliography


    \newpage
    \pagenumbering{roman}
    \appendix
        \part{Appendices}
            \input{8 - Hilbert complexes/main.tex}
            \input{9 - weak conservation proofs/main.tex}
\end{document}

            \documentclass[12pt, a4paper]{report}

\input{template/main.tex}

\title{\BA{Title in Progress...}}
\author{Boris Andrews}
\affil{Mathematical Institute, University of Oxford}
\date{\today}


\begin{document}
    \pagenumbering{gobble}
    \maketitle
    
    
    \begin{abstract}
        Magnetic confinement reactors---in particular tokamaks---offer one of the most promising options for achieving practical nuclear fusion, with the potential to provide virtually limitless, clean energy. The theoretical and numerical modeling of tokamak plasmas is simultaneously an essential component of effective reactor design, and a great research barrier. Tokamak operational conditions exhibit comparatively low Knudsen numbers. Kinetic effects, including kinetic waves and instabilities, Landau damping, bump-on-tail instabilities and more, are therefore highly influential in tokamak plasma dynamics. Purely fluid models are inherently incapable of capturing these effects, whereas the high dimensionality in purely kinetic models render them practically intractable for most relevant purposes.

        We consider a $\delta\!f$ decomposition model, with a macroscopic fluid background and microscopic kinetic correction, both fully coupled to each other. A similar manner of discretization is proposed to that used in the recent \texttt{STRUPHY} code \cite{Holderied_Possanner_Wang_2021, Holderied_2022, Li_et_al_2023} with a finite-element model for the background and a pseudo-particle/PiC model for the correction.

        The fluid background satisfies the full, non-linear, resistive, compressible, Hall MHD equations. \cite{Laakmann_Hu_Farrell_2022} introduces finite-element(-in-space) implicit timesteppers for the incompressible analogue to this system with structure-preserving (SP) properties in the ideal case, alongside parameter-robust preconditioners. We show that these timesteppers can derive from a finite-element-in-time (FET) (and finite-element-in-space) interpretation. The benefits of this reformulation are discussed, including the derivation of timesteppers that are higher order in time, and the quantifiable dissipative SP properties in the non-ideal, resistive case.
        
        We discuss possible options for extending this FET approach to timesteppers for the compressible case.

        The kinetic corrections satisfy linearized Boltzmann equations. Using a Lénard--Bernstein collision operator, these take Fokker--Planck-like forms \cite{Fokker_1914, Planck_1917} wherein pseudo-particles in the numerical model obey the neoclassical transport equations, with particle-independent Brownian drift terms. This offers a rigorous methodology for incorporating collisions into the particle transport model, without coupling the equations of motions for each particle.
        
        Works by Chen, Chacón et al. \cite{Chen_Chacón_Barnes_2011, Chacón_Chen_Barnes_2013, Chen_Chacón_2014, Chen_Chacón_2015} have developed structure-preserving particle pushers for neoclassical transport in the Vlasov equations, derived from Crank--Nicolson integrators. We show these too can can derive from a FET interpretation, similarly offering potential extensions to higher-order-in-time particle pushers. The FET formulation is used also to consider how the stochastic drift terms can be incorporated into the pushers. Stochastic gyrokinetic expansions are also discussed.

        Different options for the numerical implementation of these schemes are considered.

        Due to the efficacy of FET in the development of SP timesteppers for both the fluid and kinetic component, we hope this approach will prove effective in the future for developing SP timesteppers for the full hybrid model. We hope this will give us the opportunity to incorporate previously inaccessible kinetic effects into the highly effective, modern, finite-element MHD models.
    \end{abstract}
    
    
    \newpage
    \tableofcontents
    
    
    \newpage
    \pagenumbering{arabic}
    %\linenumbers\renewcommand\thelinenumber{\color{black!50}\arabic{linenumber}}
            \input{0 - introduction/main.tex}
        \part{Research}
            \input{1 - low-noise PiC models/main.tex}
            \input{2 - kinetic component/main.tex}
            \input{3 - fluid component/main.tex}
            \input{4 - numerical implementation/main.tex}
        \part{Project Overview}
            \input{5 - research plan/main.tex}
            \input{6 - summary/main.tex}
    
    
    %\section{}
    \newpage
    \pagenumbering{gobble}
        \printbibliography


    \newpage
    \pagenumbering{roman}
    \appendix
        \part{Appendices}
            \input{8 - Hilbert complexes/main.tex}
            \input{9 - weak conservation proofs/main.tex}
\end{document}

            \documentclass[12pt, a4paper]{report}

\input{template/main.tex}

\title{\BA{Title in Progress...}}
\author{Boris Andrews}
\affil{Mathematical Institute, University of Oxford}
\date{\today}


\begin{document}
    \pagenumbering{gobble}
    \maketitle
    
    
    \begin{abstract}
        Magnetic confinement reactors---in particular tokamaks---offer one of the most promising options for achieving practical nuclear fusion, with the potential to provide virtually limitless, clean energy. The theoretical and numerical modeling of tokamak plasmas is simultaneously an essential component of effective reactor design, and a great research barrier. Tokamak operational conditions exhibit comparatively low Knudsen numbers. Kinetic effects, including kinetic waves and instabilities, Landau damping, bump-on-tail instabilities and more, are therefore highly influential in tokamak plasma dynamics. Purely fluid models are inherently incapable of capturing these effects, whereas the high dimensionality in purely kinetic models render them practically intractable for most relevant purposes.

        We consider a $\delta\!f$ decomposition model, with a macroscopic fluid background and microscopic kinetic correction, both fully coupled to each other. A similar manner of discretization is proposed to that used in the recent \texttt{STRUPHY} code \cite{Holderied_Possanner_Wang_2021, Holderied_2022, Li_et_al_2023} with a finite-element model for the background and a pseudo-particle/PiC model for the correction.

        The fluid background satisfies the full, non-linear, resistive, compressible, Hall MHD equations. \cite{Laakmann_Hu_Farrell_2022} introduces finite-element(-in-space) implicit timesteppers for the incompressible analogue to this system with structure-preserving (SP) properties in the ideal case, alongside parameter-robust preconditioners. We show that these timesteppers can derive from a finite-element-in-time (FET) (and finite-element-in-space) interpretation. The benefits of this reformulation are discussed, including the derivation of timesteppers that are higher order in time, and the quantifiable dissipative SP properties in the non-ideal, resistive case.
        
        We discuss possible options for extending this FET approach to timesteppers for the compressible case.

        The kinetic corrections satisfy linearized Boltzmann equations. Using a Lénard--Bernstein collision operator, these take Fokker--Planck-like forms \cite{Fokker_1914, Planck_1917} wherein pseudo-particles in the numerical model obey the neoclassical transport equations, with particle-independent Brownian drift terms. This offers a rigorous methodology for incorporating collisions into the particle transport model, without coupling the equations of motions for each particle.
        
        Works by Chen, Chacón et al. \cite{Chen_Chacón_Barnes_2011, Chacón_Chen_Barnes_2013, Chen_Chacón_2014, Chen_Chacón_2015} have developed structure-preserving particle pushers for neoclassical transport in the Vlasov equations, derived from Crank--Nicolson integrators. We show these too can can derive from a FET interpretation, similarly offering potential extensions to higher-order-in-time particle pushers. The FET formulation is used also to consider how the stochastic drift terms can be incorporated into the pushers. Stochastic gyrokinetic expansions are also discussed.

        Different options for the numerical implementation of these schemes are considered.

        Due to the efficacy of FET in the development of SP timesteppers for both the fluid and kinetic component, we hope this approach will prove effective in the future for developing SP timesteppers for the full hybrid model. We hope this will give us the opportunity to incorporate previously inaccessible kinetic effects into the highly effective, modern, finite-element MHD models.
    \end{abstract}
    
    
    \newpage
    \tableofcontents
    
    
    \newpage
    \pagenumbering{arabic}
    %\linenumbers\renewcommand\thelinenumber{\color{black!50}\arabic{linenumber}}
            \input{0 - introduction/main.tex}
        \part{Research}
            \input{1 - low-noise PiC models/main.tex}
            \input{2 - kinetic component/main.tex}
            \input{3 - fluid component/main.tex}
            \input{4 - numerical implementation/main.tex}
        \part{Project Overview}
            \input{5 - research plan/main.tex}
            \input{6 - summary/main.tex}
    
    
    %\section{}
    \newpage
    \pagenumbering{gobble}
        \printbibliography


    \newpage
    \pagenumbering{roman}
    \appendix
        \part{Appendices}
            \input{8 - Hilbert complexes/main.tex}
            \input{9 - weak conservation proofs/main.tex}
\end{document}

        \part{Project Overview}
            \documentclass[12pt, a4paper]{report}

\input{template/main.tex}

\title{\BA{Title in Progress...}}
\author{Boris Andrews}
\affil{Mathematical Institute, University of Oxford}
\date{\today}


\begin{document}
    \pagenumbering{gobble}
    \maketitle
    
    
    \begin{abstract}
        Magnetic confinement reactors---in particular tokamaks---offer one of the most promising options for achieving practical nuclear fusion, with the potential to provide virtually limitless, clean energy. The theoretical and numerical modeling of tokamak plasmas is simultaneously an essential component of effective reactor design, and a great research barrier. Tokamak operational conditions exhibit comparatively low Knudsen numbers. Kinetic effects, including kinetic waves and instabilities, Landau damping, bump-on-tail instabilities and more, are therefore highly influential in tokamak plasma dynamics. Purely fluid models are inherently incapable of capturing these effects, whereas the high dimensionality in purely kinetic models render them practically intractable for most relevant purposes.

        We consider a $\delta\!f$ decomposition model, with a macroscopic fluid background and microscopic kinetic correction, both fully coupled to each other. A similar manner of discretization is proposed to that used in the recent \texttt{STRUPHY} code \cite{Holderied_Possanner_Wang_2021, Holderied_2022, Li_et_al_2023} with a finite-element model for the background and a pseudo-particle/PiC model for the correction.

        The fluid background satisfies the full, non-linear, resistive, compressible, Hall MHD equations. \cite{Laakmann_Hu_Farrell_2022} introduces finite-element(-in-space) implicit timesteppers for the incompressible analogue to this system with structure-preserving (SP) properties in the ideal case, alongside parameter-robust preconditioners. We show that these timesteppers can derive from a finite-element-in-time (FET) (and finite-element-in-space) interpretation. The benefits of this reformulation are discussed, including the derivation of timesteppers that are higher order in time, and the quantifiable dissipative SP properties in the non-ideal, resistive case.
        
        We discuss possible options for extending this FET approach to timesteppers for the compressible case.

        The kinetic corrections satisfy linearized Boltzmann equations. Using a Lénard--Bernstein collision operator, these take Fokker--Planck-like forms \cite{Fokker_1914, Planck_1917} wherein pseudo-particles in the numerical model obey the neoclassical transport equations, with particle-independent Brownian drift terms. This offers a rigorous methodology for incorporating collisions into the particle transport model, without coupling the equations of motions for each particle.
        
        Works by Chen, Chacón et al. \cite{Chen_Chacón_Barnes_2011, Chacón_Chen_Barnes_2013, Chen_Chacón_2014, Chen_Chacón_2015} have developed structure-preserving particle pushers for neoclassical transport in the Vlasov equations, derived from Crank--Nicolson integrators. We show these too can can derive from a FET interpretation, similarly offering potential extensions to higher-order-in-time particle pushers. The FET formulation is used also to consider how the stochastic drift terms can be incorporated into the pushers. Stochastic gyrokinetic expansions are also discussed.

        Different options for the numerical implementation of these schemes are considered.

        Due to the efficacy of FET in the development of SP timesteppers for both the fluid and kinetic component, we hope this approach will prove effective in the future for developing SP timesteppers for the full hybrid model. We hope this will give us the opportunity to incorporate previously inaccessible kinetic effects into the highly effective, modern, finite-element MHD models.
    \end{abstract}
    
    
    \newpage
    \tableofcontents
    
    
    \newpage
    \pagenumbering{arabic}
    %\linenumbers\renewcommand\thelinenumber{\color{black!50}\arabic{linenumber}}
            \input{0 - introduction/main.tex}
        \part{Research}
            \input{1 - low-noise PiC models/main.tex}
            \input{2 - kinetic component/main.tex}
            \input{3 - fluid component/main.tex}
            \input{4 - numerical implementation/main.tex}
        \part{Project Overview}
            \input{5 - research plan/main.tex}
            \input{6 - summary/main.tex}
    
    
    %\section{}
    \newpage
    \pagenumbering{gobble}
        \printbibliography


    \newpage
    \pagenumbering{roman}
    \appendix
        \part{Appendices}
            \input{8 - Hilbert complexes/main.tex}
            \input{9 - weak conservation proofs/main.tex}
\end{document}

            \documentclass[12pt, a4paper]{report}

\input{template/main.tex}

\title{\BA{Title in Progress...}}
\author{Boris Andrews}
\affil{Mathematical Institute, University of Oxford}
\date{\today}


\begin{document}
    \pagenumbering{gobble}
    \maketitle
    
    
    \begin{abstract}
        Magnetic confinement reactors---in particular tokamaks---offer one of the most promising options for achieving practical nuclear fusion, with the potential to provide virtually limitless, clean energy. The theoretical and numerical modeling of tokamak plasmas is simultaneously an essential component of effective reactor design, and a great research barrier. Tokamak operational conditions exhibit comparatively low Knudsen numbers. Kinetic effects, including kinetic waves and instabilities, Landau damping, bump-on-tail instabilities and more, are therefore highly influential in tokamak plasma dynamics. Purely fluid models are inherently incapable of capturing these effects, whereas the high dimensionality in purely kinetic models render them practically intractable for most relevant purposes.

        We consider a $\delta\!f$ decomposition model, with a macroscopic fluid background and microscopic kinetic correction, both fully coupled to each other. A similar manner of discretization is proposed to that used in the recent \texttt{STRUPHY} code \cite{Holderied_Possanner_Wang_2021, Holderied_2022, Li_et_al_2023} with a finite-element model for the background and a pseudo-particle/PiC model for the correction.

        The fluid background satisfies the full, non-linear, resistive, compressible, Hall MHD equations. \cite{Laakmann_Hu_Farrell_2022} introduces finite-element(-in-space) implicit timesteppers for the incompressible analogue to this system with structure-preserving (SP) properties in the ideal case, alongside parameter-robust preconditioners. We show that these timesteppers can derive from a finite-element-in-time (FET) (and finite-element-in-space) interpretation. The benefits of this reformulation are discussed, including the derivation of timesteppers that are higher order in time, and the quantifiable dissipative SP properties in the non-ideal, resistive case.
        
        We discuss possible options for extending this FET approach to timesteppers for the compressible case.

        The kinetic corrections satisfy linearized Boltzmann equations. Using a Lénard--Bernstein collision operator, these take Fokker--Planck-like forms \cite{Fokker_1914, Planck_1917} wherein pseudo-particles in the numerical model obey the neoclassical transport equations, with particle-independent Brownian drift terms. This offers a rigorous methodology for incorporating collisions into the particle transport model, without coupling the equations of motions for each particle.
        
        Works by Chen, Chacón et al. \cite{Chen_Chacón_Barnes_2011, Chacón_Chen_Barnes_2013, Chen_Chacón_2014, Chen_Chacón_2015} have developed structure-preserving particle pushers for neoclassical transport in the Vlasov equations, derived from Crank--Nicolson integrators. We show these too can can derive from a FET interpretation, similarly offering potential extensions to higher-order-in-time particle pushers. The FET formulation is used also to consider how the stochastic drift terms can be incorporated into the pushers. Stochastic gyrokinetic expansions are also discussed.

        Different options for the numerical implementation of these schemes are considered.

        Due to the efficacy of FET in the development of SP timesteppers for both the fluid and kinetic component, we hope this approach will prove effective in the future for developing SP timesteppers for the full hybrid model. We hope this will give us the opportunity to incorporate previously inaccessible kinetic effects into the highly effective, modern, finite-element MHD models.
    \end{abstract}
    
    
    \newpage
    \tableofcontents
    
    
    \newpage
    \pagenumbering{arabic}
    %\linenumbers\renewcommand\thelinenumber{\color{black!50}\arabic{linenumber}}
            \input{0 - introduction/main.tex}
        \part{Research}
            \input{1 - low-noise PiC models/main.tex}
            \input{2 - kinetic component/main.tex}
            \input{3 - fluid component/main.tex}
            \input{4 - numerical implementation/main.tex}
        \part{Project Overview}
            \input{5 - research plan/main.tex}
            \input{6 - summary/main.tex}
    
    
    %\section{}
    \newpage
    \pagenumbering{gobble}
        \printbibliography


    \newpage
    \pagenumbering{roman}
    \appendix
        \part{Appendices}
            \input{8 - Hilbert complexes/main.tex}
            \input{9 - weak conservation proofs/main.tex}
\end{document}

    
    
    %\section{}
    \newpage
    \pagenumbering{gobble}
        \printbibliography


    \newpage
    \pagenumbering{roman}
    \appendix
        \part{Appendices}
            \documentclass[12pt, a4paper]{report}

\input{template/main.tex}

\title{\BA{Title in Progress...}}
\author{Boris Andrews}
\affil{Mathematical Institute, University of Oxford}
\date{\today}


\begin{document}
    \pagenumbering{gobble}
    \maketitle
    
    
    \begin{abstract}
        Magnetic confinement reactors---in particular tokamaks---offer one of the most promising options for achieving practical nuclear fusion, with the potential to provide virtually limitless, clean energy. The theoretical and numerical modeling of tokamak plasmas is simultaneously an essential component of effective reactor design, and a great research barrier. Tokamak operational conditions exhibit comparatively low Knudsen numbers. Kinetic effects, including kinetic waves and instabilities, Landau damping, bump-on-tail instabilities and more, are therefore highly influential in tokamak plasma dynamics. Purely fluid models are inherently incapable of capturing these effects, whereas the high dimensionality in purely kinetic models render them practically intractable for most relevant purposes.

        We consider a $\delta\!f$ decomposition model, with a macroscopic fluid background and microscopic kinetic correction, both fully coupled to each other. A similar manner of discretization is proposed to that used in the recent \texttt{STRUPHY} code \cite{Holderied_Possanner_Wang_2021, Holderied_2022, Li_et_al_2023} with a finite-element model for the background and a pseudo-particle/PiC model for the correction.

        The fluid background satisfies the full, non-linear, resistive, compressible, Hall MHD equations. \cite{Laakmann_Hu_Farrell_2022} introduces finite-element(-in-space) implicit timesteppers for the incompressible analogue to this system with structure-preserving (SP) properties in the ideal case, alongside parameter-robust preconditioners. We show that these timesteppers can derive from a finite-element-in-time (FET) (and finite-element-in-space) interpretation. The benefits of this reformulation are discussed, including the derivation of timesteppers that are higher order in time, and the quantifiable dissipative SP properties in the non-ideal, resistive case.
        
        We discuss possible options for extending this FET approach to timesteppers for the compressible case.

        The kinetic corrections satisfy linearized Boltzmann equations. Using a Lénard--Bernstein collision operator, these take Fokker--Planck-like forms \cite{Fokker_1914, Planck_1917} wherein pseudo-particles in the numerical model obey the neoclassical transport equations, with particle-independent Brownian drift terms. This offers a rigorous methodology for incorporating collisions into the particle transport model, without coupling the equations of motions for each particle.
        
        Works by Chen, Chacón et al. \cite{Chen_Chacón_Barnes_2011, Chacón_Chen_Barnes_2013, Chen_Chacón_2014, Chen_Chacón_2015} have developed structure-preserving particle pushers for neoclassical transport in the Vlasov equations, derived from Crank--Nicolson integrators. We show these too can can derive from a FET interpretation, similarly offering potential extensions to higher-order-in-time particle pushers. The FET formulation is used also to consider how the stochastic drift terms can be incorporated into the pushers. Stochastic gyrokinetic expansions are also discussed.

        Different options for the numerical implementation of these schemes are considered.

        Due to the efficacy of FET in the development of SP timesteppers for both the fluid and kinetic component, we hope this approach will prove effective in the future for developing SP timesteppers for the full hybrid model. We hope this will give us the opportunity to incorporate previously inaccessible kinetic effects into the highly effective, modern, finite-element MHD models.
    \end{abstract}
    
    
    \newpage
    \tableofcontents
    
    
    \newpage
    \pagenumbering{arabic}
    %\linenumbers\renewcommand\thelinenumber{\color{black!50}\arabic{linenumber}}
            \input{0 - introduction/main.tex}
        \part{Research}
            \input{1 - low-noise PiC models/main.tex}
            \input{2 - kinetic component/main.tex}
            \input{3 - fluid component/main.tex}
            \input{4 - numerical implementation/main.tex}
        \part{Project Overview}
            \input{5 - research plan/main.tex}
            \input{6 - summary/main.tex}
    
    
    %\section{}
    \newpage
    \pagenumbering{gobble}
        \printbibliography


    \newpage
    \pagenumbering{roman}
    \appendix
        \part{Appendices}
            \input{8 - Hilbert complexes/main.tex}
            \input{9 - weak conservation proofs/main.tex}
\end{document}

            \documentclass[12pt, a4paper]{report}

\input{template/main.tex}

\title{\BA{Title in Progress...}}
\author{Boris Andrews}
\affil{Mathematical Institute, University of Oxford}
\date{\today}


\begin{document}
    \pagenumbering{gobble}
    \maketitle
    
    
    \begin{abstract}
        Magnetic confinement reactors---in particular tokamaks---offer one of the most promising options for achieving practical nuclear fusion, with the potential to provide virtually limitless, clean energy. The theoretical and numerical modeling of tokamak plasmas is simultaneously an essential component of effective reactor design, and a great research barrier. Tokamak operational conditions exhibit comparatively low Knudsen numbers. Kinetic effects, including kinetic waves and instabilities, Landau damping, bump-on-tail instabilities and more, are therefore highly influential in tokamak plasma dynamics. Purely fluid models are inherently incapable of capturing these effects, whereas the high dimensionality in purely kinetic models render them practically intractable for most relevant purposes.

        We consider a $\delta\!f$ decomposition model, with a macroscopic fluid background and microscopic kinetic correction, both fully coupled to each other. A similar manner of discretization is proposed to that used in the recent \texttt{STRUPHY} code \cite{Holderied_Possanner_Wang_2021, Holderied_2022, Li_et_al_2023} with a finite-element model for the background and a pseudo-particle/PiC model for the correction.

        The fluid background satisfies the full, non-linear, resistive, compressible, Hall MHD equations. \cite{Laakmann_Hu_Farrell_2022} introduces finite-element(-in-space) implicit timesteppers for the incompressible analogue to this system with structure-preserving (SP) properties in the ideal case, alongside parameter-robust preconditioners. We show that these timesteppers can derive from a finite-element-in-time (FET) (and finite-element-in-space) interpretation. The benefits of this reformulation are discussed, including the derivation of timesteppers that are higher order in time, and the quantifiable dissipative SP properties in the non-ideal, resistive case.
        
        We discuss possible options for extending this FET approach to timesteppers for the compressible case.

        The kinetic corrections satisfy linearized Boltzmann equations. Using a Lénard--Bernstein collision operator, these take Fokker--Planck-like forms \cite{Fokker_1914, Planck_1917} wherein pseudo-particles in the numerical model obey the neoclassical transport equations, with particle-independent Brownian drift terms. This offers a rigorous methodology for incorporating collisions into the particle transport model, without coupling the equations of motions for each particle.
        
        Works by Chen, Chacón et al. \cite{Chen_Chacón_Barnes_2011, Chacón_Chen_Barnes_2013, Chen_Chacón_2014, Chen_Chacón_2015} have developed structure-preserving particle pushers for neoclassical transport in the Vlasov equations, derived from Crank--Nicolson integrators. We show these too can can derive from a FET interpretation, similarly offering potential extensions to higher-order-in-time particle pushers. The FET formulation is used also to consider how the stochastic drift terms can be incorporated into the pushers. Stochastic gyrokinetic expansions are also discussed.

        Different options for the numerical implementation of these schemes are considered.

        Due to the efficacy of FET in the development of SP timesteppers for both the fluid and kinetic component, we hope this approach will prove effective in the future for developing SP timesteppers for the full hybrid model. We hope this will give us the opportunity to incorporate previously inaccessible kinetic effects into the highly effective, modern, finite-element MHD models.
    \end{abstract}
    
    
    \newpage
    \tableofcontents
    
    
    \newpage
    \pagenumbering{arabic}
    %\linenumbers\renewcommand\thelinenumber{\color{black!50}\arabic{linenumber}}
            \input{0 - introduction/main.tex}
        \part{Research}
            \input{1 - low-noise PiC models/main.tex}
            \input{2 - kinetic component/main.tex}
            \input{3 - fluid component/main.tex}
            \input{4 - numerical implementation/main.tex}
        \part{Project Overview}
            \input{5 - research plan/main.tex}
            \input{6 - summary/main.tex}
    
    
    %\section{}
    \newpage
    \pagenumbering{gobble}
        \printbibliography


    \newpage
    \pagenumbering{roman}
    \appendix
        \part{Appendices}
            \input{8 - Hilbert complexes/main.tex}
            \input{9 - weak conservation proofs/main.tex}
\end{document}

\end{document}


\title{\BA{Title in Progress...}}
\author{Boris Andrews}
\affil{Mathematical Institute, University of Oxford}
\date{\today}


\begin{document}
    \pagenumbering{gobble}
    \maketitle
    
    
    \begin{abstract}
        Magnetic confinement reactors---in particular tokamaks---offer one of the most promising options for achieving practical nuclear fusion, with the potential to provide virtually limitless, clean energy. The theoretical and numerical modeling of tokamak plasmas is simultaneously an essential component of effective reactor design, and a great research barrier. Tokamak operational conditions exhibit comparatively low Knudsen numbers. Kinetic effects, including kinetic waves and instabilities, Landau damping, bump-on-tail instabilities and more, are therefore highly influential in tokamak plasma dynamics. Purely fluid models are inherently incapable of capturing these effects, whereas the high dimensionality in purely kinetic models render them practically intractable for most relevant purposes.

        We consider a $\delta\!f$ decomposition model, with a macroscopic fluid background and microscopic kinetic correction, both fully coupled to each other. A similar manner of discretization is proposed to that used in the recent \texttt{STRUPHY} code \cite{Holderied_Possanner_Wang_2021, Holderied_2022, Li_et_al_2023} with a finite-element model for the background and a pseudo-particle/PiC model for the correction.

        The fluid background satisfies the full, non-linear, resistive, compressible, Hall MHD equations. \cite{Laakmann_Hu_Farrell_2022} introduces finite-element(-in-space) implicit timesteppers for the incompressible analogue to this system with structure-preserving (SP) properties in the ideal case, alongside parameter-robust preconditioners. We show that these timesteppers can derive from a finite-element-in-time (FET) (and finite-element-in-space) interpretation. The benefits of this reformulation are discussed, including the derivation of timesteppers that are higher order in time, and the quantifiable dissipative SP properties in the non-ideal, resistive case.
        
        We discuss possible options for extending this FET approach to timesteppers for the compressible case.

        The kinetic corrections satisfy linearized Boltzmann equations. Using a Lénard--Bernstein collision operator, these take Fokker--Planck-like forms \cite{Fokker_1914, Planck_1917} wherein pseudo-particles in the numerical model obey the neoclassical transport equations, with particle-independent Brownian drift terms. This offers a rigorous methodology for incorporating collisions into the particle transport model, without coupling the equations of motions for each particle.
        
        Works by Chen, Chacón et al. \cite{Chen_Chacón_Barnes_2011, Chacón_Chen_Barnes_2013, Chen_Chacón_2014, Chen_Chacón_2015} have developed structure-preserving particle pushers for neoclassical transport in the Vlasov equations, derived from Crank--Nicolson integrators. We show these too can can derive from a FET interpretation, similarly offering potential extensions to higher-order-in-time particle pushers. The FET formulation is used also to consider how the stochastic drift terms can be incorporated into the pushers. Stochastic gyrokinetic expansions are also discussed.

        Different options for the numerical implementation of these schemes are considered.

        Due to the efficacy of FET in the development of SP timesteppers for both the fluid and kinetic component, we hope this approach will prove effective in the future for developing SP timesteppers for the full hybrid model. We hope this will give us the opportunity to incorporate previously inaccessible kinetic effects into the highly effective, modern, finite-element MHD models.
    \end{abstract}
    
    
    \newpage
    \tableofcontents
    
    
    \newpage
    \pagenumbering{arabic}
    %\linenumbers\renewcommand\thelinenumber{\color{black!50}\arabic{linenumber}}
            \documentclass[12pt, a4paper]{report}

\documentclass[12pt, a4paper]{report}

\input{template/main.tex}

\title{\BA{Title in Progress...}}
\author{Boris Andrews}
\affil{Mathematical Institute, University of Oxford}
\date{\today}


\begin{document}
    \pagenumbering{gobble}
    \maketitle
    
    
    \begin{abstract}
        Magnetic confinement reactors---in particular tokamaks---offer one of the most promising options for achieving practical nuclear fusion, with the potential to provide virtually limitless, clean energy. The theoretical and numerical modeling of tokamak plasmas is simultaneously an essential component of effective reactor design, and a great research barrier. Tokamak operational conditions exhibit comparatively low Knudsen numbers. Kinetic effects, including kinetic waves and instabilities, Landau damping, bump-on-tail instabilities and more, are therefore highly influential in tokamak plasma dynamics. Purely fluid models are inherently incapable of capturing these effects, whereas the high dimensionality in purely kinetic models render them practically intractable for most relevant purposes.

        We consider a $\delta\!f$ decomposition model, with a macroscopic fluid background and microscopic kinetic correction, both fully coupled to each other. A similar manner of discretization is proposed to that used in the recent \texttt{STRUPHY} code \cite{Holderied_Possanner_Wang_2021, Holderied_2022, Li_et_al_2023} with a finite-element model for the background and a pseudo-particle/PiC model for the correction.

        The fluid background satisfies the full, non-linear, resistive, compressible, Hall MHD equations. \cite{Laakmann_Hu_Farrell_2022} introduces finite-element(-in-space) implicit timesteppers for the incompressible analogue to this system with structure-preserving (SP) properties in the ideal case, alongside parameter-robust preconditioners. We show that these timesteppers can derive from a finite-element-in-time (FET) (and finite-element-in-space) interpretation. The benefits of this reformulation are discussed, including the derivation of timesteppers that are higher order in time, and the quantifiable dissipative SP properties in the non-ideal, resistive case.
        
        We discuss possible options for extending this FET approach to timesteppers for the compressible case.

        The kinetic corrections satisfy linearized Boltzmann equations. Using a Lénard--Bernstein collision operator, these take Fokker--Planck-like forms \cite{Fokker_1914, Planck_1917} wherein pseudo-particles in the numerical model obey the neoclassical transport equations, with particle-independent Brownian drift terms. This offers a rigorous methodology for incorporating collisions into the particle transport model, without coupling the equations of motions for each particle.
        
        Works by Chen, Chacón et al. \cite{Chen_Chacón_Barnes_2011, Chacón_Chen_Barnes_2013, Chen_Chacón_2014, Chen_Chacón_2015} have developed structure-preserving particle pushers for neoclassical transport in the Vlasov equations, derived from Crank--Nicolson integrators. We show these too can can derive from a FET interpretation, similarly offering potential extensions to higher-order-in-time particle pushers. The FET formulation is used also to consider how the stochastic drift terms can be incorporated into the pushers. Stochastic gyrokinetic expansions are also discussed.

        Different options for the numerical implementation of these schemes are considered.

        Due to the efficacy of FET in the development of SP timesteppers for both the fluid and kinetic component, we hope this approach will prove effective in the future for developing SP timesteppers for the full hybrid model. We hope this will give us the opportunity to incorporate previously inaccessible kinetic effects into the highly effective, modern, finite-element MHD models.
    \end{abstract}
    
    
    \newpage
    \tableofcontents
    
    
    \newpage
    \pagenumbering{arabic}
    %\linenumbers\renewcommand\thelinenumber{\color{black!50}\arabic{linenumber}}
            \input{0 - introduction/main.tex}
        \part{Research}
            \input{1 - low-noise PiC models/main.tex}
            \input{2 - kinetic component/main.tex}
            \input{3 - fluid component/main.tex}
            \input{4 - numerical implementation/main.tex}
        \part{Project Overview}
            \input{5 - research plan/main.tex}
            \input{6 - summary/main.tex}
    
    
    %\section{}
    \newpage
    \pagenumbering{gobble}
        \printbibliography


    \newpage
    \pagenumbering{roman}
    \appendix
        \part{Appendices}
            \input{8 - Hilbert complexes/main.tex}
            \input{9 - weak conservation proofs/main.tex}
\end{document}


\title{\BA{Title in Progress...}}
\author{Boris Andrews}
\affil{Mathematical Institute, University of Oxford}
\date{\today}


\begin{document}
    \pagenumbering{gobble}
    \maketitle
    
    
    \begin{abstract}
        Magnetic confinement reactors---in particular tokamaks---offer one of the most promising options for achieving practical nuclear fusion, with the potential to provide virtually limitless, clean energy. The theoretical and numerical modeling of tokamak plasmas is simultaneously an essential component of effective reactor design, and a great research barrier. Tokamak operational conditions exhibit comparatively low Knudsen numbers. Kinetic effects, including kinetic waves and instabilities, Landau damping, bump-on-tail instabilities and more, are therefore highly influential in tokamak plasma dynamics. Purely fluid models are inherently incapable of capturing these effects, whereas the high dimensionality in purely kinetic models render them practically intractable for most relevant purposes.

        We consider a $\delta\!f$ decomposition model, with a macroscopic fluid background and microscopic kinetic correction, both fully coupled to each other. A similar manner of discretization is proposed to that used in the recent \texttt{STRUPHY} code \cite{Holderied_Possanner_Wang_2021, Holderied_2022, Li_et_al_2023} with a finite-element model for the background and a pseudo-particle/PiC model for the correction.

        The fluid background satisfies the full, non-linear, resistive, compressible, Hall MHD equations. \cite{Laakmann_Hu_Farrell_2022} introduces finite-element(-in-space) implicit timesteppers for the incompressible analogue to this system with structure-preserving (SP) properties in the ideal case, alongside parameter-robust preconditioners. We show that these timesteppers can derive from a finite-element-in-time (FET) (and finite-element-in-space) interpretation. The benefits of this reformulation are discussed, including the derivation of timesteppers that are higher order in time, and the quantifiable dissipative SP properties in the non-ideal, resistive case.
        
        We discuss possible options for extending this FET approach to timesteppers for the compressible case.

        The kinetic corrections satisfy linearized Boltzmann equations. Using a Lénard--Bernstein collision operator, these take Fokker--Planck-like forms \cite{Fokker_1914, Planck_1917} wherein pseudo-particles in the numerical model obey the neoclassical transport equations, with particle-independent Brownian drift terms. This offers a rigorous methodology for incorporating collisions into the particle transport model, without coupling the equations of motions for each particle.
        
        Works by Chen, Chacón et al. \cite{Chen_Chacón_Barnes_2011, Chacón_Chen_Barnes_2013, Chen_Chacón_2014, Chen_Chacón_2015} have developed structure-preserving particle pushers for neoclassical transport in the Vlasov equations, derived from Crank--Nicolson integrators. We show these too can can derive from a FET interpretation, similarly offering potential extensions to higher-order-in-time particle pushers. The FET formulation is used also to consider how the stochastic drift terms can be incorporated into the pushers. Stochastic gyrokinetic expansions are also discussed.

        Different options for the numerical implementation of these schemes are considered.

        Due to the efficacy of FET in the development of SP timesteppers for both the fluid and kinetic component, we hope this approach will prove effective in the future for developing SP timesteppers for the full hybrid model. We hope this will give us the opportunity to incorporate previously inaccessible kinetic effects into the highly effective, modern, finite-element MHD models.
    \end{abstract}
    
    
    \newpage
    \tableofcontents
    
    
    \newpage
    \pagenumbering{arabic}
    %\linenumbers\renewcommand\thelinenumber{\color{black!50}\arabic{linenumber}}
            \documentclass[12pt, a4paper]{report}

\input{template/main.tex}

\title{\BA{Title in Progress...}}
\author{Boris Andrews}
\affil{Mathematical Institute, University of Oxford}
\date{\today}


\begin{document}
    \pagenumbering{gobble}
    \maketitle
    
    
    \begin{abstract}
        Magnetic confinement reactors---in particular tokamaks---offer one of the most promising options for achieving practical nuclear fusion, with the potential to provide virtually limitless, clean energy. The theoretical and numerical modeling of tokamak plasmas is simultaneously an essential component of effective reactor design, and a great research barrier. Tokamak operational conditions exhibit comparatively low Knudsen numbers. Kinetic effects, including kinetic waves and instabilities, Landau damping, bump-on-tail instabilities and more, are therefore highly influential in tokamak plasma dynamics. Purely fluid models are inherently incapable of capturing these effects, whereas the high dimensionality in purely kinetic models render them practically intractable for most relevant purposes.

        We consider a $\delta\!f$ decomposition model, with a macroscopic fluid background and microscopic kinetic correction, both fully coupled to each other. A similar manner of discretization is proposed to that used in the recent \texttt{STRUPHY} code \cite{Holderied_Possanner_Wang_2021, Holderied_2022, Li_et_al_2023} with a finite-element model for the background and a pseudo-particle/PiC model for the correction.

        The fluid background satisfies the full, non-linear, resistive, compressible, Hall MHD equations. \cite{Laakmann_Hu_Farrell_2022} introduces finite-element(-in-space) implicit timesteppers for the incompressible analogue to this system with structure-preserving (SP) properties in the ideal case, alongside parameter-robust preconditioners. We show that these timesteppers can derive from a finite-element-in-time (FET) (and finite-element-in-space) interpretation. The benefits of this reformulation are discussed, including the derivation of timesteppers that are higher order in time, and the quantifiable dissipative SP properties in the non-ideal, resistive case.
        
        We discuss possible options for extending this FET approach to timesteppers for the compressible case.

        The kinetic corrections satisfy linearized Boltzmann equations. Using a Lénard--Bernstein collision operator, these take Fokker--Planck-like forms \cite{Fokker_1914, Planck_1917} wherein pseudo-particles in the numerical model obey the neoclassical transport equations, with particle-independent Brownian drift terms. This offers a rigorous methodology for incorporating collisions into the particle transport model, without coupling the equations of motions for each particle.
        
        Works by Chen, Chacón et al. \cite{Chen_Chacón_Barnes_2011, Chacón_Chen_Barnes_2013, Chen_Chacón_2014, Chen_Chacón_2015} have developed structure-preserving particle pushers for neoclassical transport in the Vlasov equations, derived from Crank--Nicolson integrators. We show these too can can derive from a FET interpretation, similarly offering potential extensions to higher-order-in-time particle pushers. The FET formulation is used also to consider how the stochastic drift terms can be incorporated into the pushers. Stochastic gyrokinetic expansions are also discussed.

        Different options for the numerical implementation of these schemes are considered.

        Due to the efficacy of FET in the development of SP timesteppers for both the fluid and kinetic component, we hope this approach will prove effective in the future for developing SP timesteppers for the full hybrid model. We hope this will give us the opportunity to incorporate previously inaccessible kinetic effects into the highly effective, modern, finite-element MHD models.
    \end{abstract}
    
    
    \newpage
    \tableofcontents
    
    
    \newpage
    \pagenumbering{arabic}
    %\linenumbers\renewcommand\thelinenumber{\color{black!50}\arabic{linenumber}}
            \input{0 - introduction/main.tex}
        \part{Research}
            \input{1 - low-noise PiC models/main.tex}
            \input{2 - kinetic component/main.tex}
            \input{3 - fluid component/main.tex}
            \input{4 - numerical implementation/main.tex}
        \part{Project Overview}
            \input{5 - research plan/main.tex}
            \input{6 - summary/main.tex}
    
    
    %\section{}
    \newpage
    \pagenumbering{gobble}
        \printbibliography


    \newpage
    \pagenumbering{roman}
    \appendix
        \part{Appendices}
            \input{8 - Hilbert complexes/main.tex}
            \input{9 - weak conservation proofs/main.tex}
\end{document}

        \part{Research}
            \documentclass[12pt, a4paper]{report}

\input{template/main.tex}

\title{\BA{Title in Progress...}}
\author{Boris Andrews}
\affil{Mathematical Institute, University of Oxford}
\date{\today}


\begin{document}
    \pagenumbering{gobble}
    \maketitle
    
    
    \begin{abstract}
        Magnetic confinement reactors---in particular tokamaks---offer one of the most promising options for achieving practical nuclear fusion, with the potential to provide virtually limitless, clean energy. The theoretical and numerical modeling of tokamak plasmas is simultaneously an essential component of effective reactor design, and a great research barrier. Tokamak operational conditions exhibit comparatively low Knudsen numbers. Kinetic effects, including kinetic waves and instabilities, Landau damping, bump-on-tail instabilities and more, are therefore highly influential in tokamak plasma dynamics. Purely fluid models are inherently incapable of capturing these effects, whereas the high dimensionality in purely kinetic models render them practically intractable for most relevant purposes.

        We consider a $\delta\!f$ decomposition model, with a macroscopic fluid background and microscopic kinetic correction, both fully coupled to each other. A similar manner of discretization is proposed to that used in the recent \texttt{STRUPHY} code \cite{Holderied_Possanner_Wang_2021, Holderied_2022, Li_et_al_2023} with a finite-element model for the background and a pseudo-particle/PiC model for the correction.

        The fluid background satisfies the full, non-linear, resistive, compressible, Hall MHD equations. \cite{Laakmann_Hu_Farrell_2022} introduces finite-element(-in-space) implicit timesteppers for the incompressible analogue to this system with structure-preserving (SP) properties in the ideal case, alongside parameter-robust preconditioners. We show that these timesteppers can derive from a finite-element-in-time (FET) (and finite-element-in-space) interpretation. The benefits of this reformulation are discussed, including the derivation of timesteppers that are higher order in time, and the quantifiable dissipative SP properties in the non-ideal, resistive case.
        
        We discuss possible options for extending this FET approach to timesteppers for the compressible case.

        The kinetic corrections satisfy linearized Boltzmann equations. Using a Lénard--Bernstein collision operator, these take Fokker--Planck-like forms \cite{Fokker_1914, Planck_1917} wherein pseudo-particles in the numerical model obey the neoclassical transport equations, with particle-independent Brownian drift terms. This offers a rigorous methodology for incorporating collisions into the particle transport model, without coupling the equations of motions for each particle.
        
        Works by Chen, Chacón et al. \cite{Chen_Chacón_Barnes_2011, Chacón_Chen_Barnes_2013, Chen_Chacón_2014, Chen_Chacón_2015} have developed structure-preserving particle pushers for neoclassical transport in the Vlasov equations, derived from Crank--Nicolson integrators. We show these too can can derive from a FET interpretation, similarly offering potential extensions to higher-order-in-time particle pushers. The FET formulation is used also to consider how the stochastic drift terms can be incorporated into the pushers. Stochastic gyrokinetic expansions are also discussed.

        Different options for the numerical implementation of these schemes are considered.

        Due to the efficacy of FET in the development of SP timesteppers for both the fluid and kinetic component, we hope this approach will prove effective in the future for developing SP timesteppers for the full hybrid model. We hope this will give us the opportunity to incorporate previously inaccessible kinetic effects into the highly effective, modern, finite-element MHD models.
    \end{abstract}
    
    
    \newpage
    \tableofcontents
    
    
    \newpage
    \pagenumbering{arabic}
    %\linenumbers\renewcommand\thelinenumber{\color{black!50}\arabic{linenumber}}
            \input{0 - introduction/main.tex}
        \part{Research}
            \input{1 - low-noise PiC models/main.tex}
            \input{2 - kinetic component/main.tex}
            \input{3 - fluid component/main.tex}
            \input{4 - numerical implementation/main.tex}
        \part{Project Overview}
            \input{5 - research plan/main.tex}
            \input{6 - summary/main.tex}
    
    
    %\section{}
    \newpage
    \pagenumbering{gobble}
        \printbibliography


    \newpage
    \pagenumbering{roman}
    \appendix
        \part{Appendices}
            \input{8 - Hilbert complexes/main.tex}
            \input{9 - weak conservation proofs/main.tex}
\end{document}

            \documentclass[12pt, a4paper]{report}

\input{template/main.tex}

\title{\BA{Title in Progress...}}
\author{Boris Andrews}
\affil{Mathematical Institute, University of Oxford}
\date{\today}


\begin{document}
    \pagenumbering{gobble}
    \maketitle
    
    
    \begin{abstract}
        Magnetic confinement reactors---in particular tokamaks---offer one of the most promising options for achieving practical nuclear fusion, with the potential to provide virtually limitless, clean energy. The theoretical and numerical modeling of tokamak plasmas is simultaneously an essential component of effective reactor design, and a great research barrier. Tokamak operational conditions exhibit comparatively low Knudsen numbers. Kinetic effects, including kinetic waves and instabilities, Landau damping, bump-on-tail instabilities and more, are therefore highly influential in tokamak plasma dynamics. Purely fluid models are inherently incapable of capturing these effects, whereas the high dimensionality in purely kinetic models render them practically intractable for most relevant purposes.

        We consider a $\delta\!f$ decomposition model, with a macroscopic fluid background and microscopic kinetic correction, both fully coupled to each other. A similar manner of discretization is proposed to that used in the recent \texttt{STRUPHY} code \cite{Holderied_Possanner_Wang_2021, Holderied_2022, Li_et_al_2023} with a finite-element model for the background and a pseudo-particle/PiC model for the correction.

        The fluid background satisfies the full, non-linear, resistive, compressible, Hall MHD equations. \cite{Laakmann_Hu_Farrell_2022} introduces finite-element(-in-space) implicit timesteppers for the incompressible analogue to this system with structure-preserving (SP) properties in the ideal case, alongside parameter-robust preconditioners. We show that these timesteppers can derive from a finite-element-in-time (FET) (and finite-element-in-space) interpretation. The benefits of this reformulation are discussed, including the derivation of timesteppers that are higher order in time, and the quantifiable dissipative SP properties in the non-ideal, resistive case.
        
        We discuss possible options for extending this FET approach to timesteppers for the compressible case.

        The kinetic corrections satisfy linearized Boltzmann equations. Using a Lénard--Bernstein collision operator, these take Fokker--Planck-like forms \cite{Fokker_1914, Planck_1917} wherein pseudo-particles in the numerical model obey the neoclassical transport equations, with particle-independent Brownian drift terms. This offers a rigorous methodology for incorporating collisions into the particle transport model, without coupling the equations of motions for each particle.
        
        Works by Chen, Chacón et al. \cite{Chen_Chacón_Barnes_2011, Chacón_Chen_Barnes_2013, Chen_Chacón_2014, Chen_Chacón_2015} have developed structure-preserving particle pushers for neoclassical transport in the Vlasov equations, derived from Crank--Nicolson integrators. We show these too can can derive from a FET interpretation, similarly offering potential extensions to higher-order-in-time particle pushers. The FET formulation is used also to consider how the stochastic drift terms can be incorporated into the pushers. Stochastic gyrokinetic expansions are also discussed.

        Different options for the numerical implementation of these schemes are considered.

        Due to the efficacy of FET in the development of SP timesteppers for both the fluid and kinetic component, we hope this approach will prove effective in the future for developing SP timesteppers for the full hybrid model. We hope this will give us the opportunity to incorporate previously inaccessible kinetic effects into the highly effective, modern, finite-element MHD models.
    \end{abstract}
    
    
    \newpage
    \tableofcontents
    
    
    \newpage
    \pagenumbering{arabic}
    %\linenumbers\renewcommand\thelinenumber{\color{black!50}\arabic{linenumber}}
            \input{0 - introduction/main.tex}
        \part{Research}
            \input{1 - low-noise PiC models/main.tex}
            \input{2 - kinetic component/main.tex}
            \input{3 - fluid component/main.tex}
            \input{4 - numerical implementation/main.tex}
        \part{Project Overview}
            \input{5 - research plan/main.tex}
            \input{6 - summary/main.tex}
    
    
    %\section{}
    \newpage
    \pagenumbering{gobble}
        \printbibliography


    \newpage
    \pagenumbering{roman}
    \appendix
        \part{Appendices}
            \input{8 - Hilbert complexes/main.tex}
            \input{9 - weak conservation proofs/main.tex}
\end{document}

            \documentclass[12pt, a4paper]{report}

\input{template/main.tex}

\title{\BA{Title in Progress...}}
\author{Boris Andrews}
\affil{Mathematical Institute, University of Oxford}
\date{\today}


\begin{document}
    \pagenumbering{gobble}
    \maketitle
    
    
    \begin{abstract}
        Magnetic confinement reactors---in particular tokamaks---offer one of the most promising options for achieving practical nuclear fusion, with the potential to provide virtually limitless, clean energy. The theoretical and numerical modeling of tokamak plasmas is simultaneously an essential component of effective reactor design, and a great research barrier. Tokamak operational conditions exhibit comparatively low Knudsen numbers. Kinetic effects, including kinetic waves and instabilities, Landau damping, bump-on-tail instabilities and more, are therefore highly influential in tokamak plasma dynamics. Purely fluid models are inherently incapable of capturing these effects, whereas the high dimensionality in purely kinetic models render them practically intractable for most relevant purposes.

        We consider a $\delta\!f$ decomposition model, with a macroscopic fluid background and microscopic kinetic correction, both fully coupled to each other. A similar manner of discretization is proposed to that used in the recent \texttt{STRUPHY} code \cite{Holderied_Possanner_Wang_2021, Holderied_2022, Li_et_al_2023} with a finite-element model for the background and a pseudo-particle/PiC model for the correction.

        The fluid background satisfies the full, non-linear, resistive, compressible, Hall MHD equations. \cite{Laakmann_Hu_Farrell_2022} introduces finite-element(-in-space) implicit timesteppers for the incompressible analogue to this system with structure-preserving (SP) properties in the ideal case, alongside parameter-robust preconditioners. We show that these timesteppers can derive from a finite-element-in-time (FET) (and finite-element-in-space) interpretation. The benefits of this reformulation are discussed, including the derivation of timesteppers that are higher order in time, and the quantifiable dissipative SP properties in the non-ideal, resistive case.
        
        We discuss possible options for extending this FET approach to timesteppers for the compressible case.

        The kinetic corrections satisfy linearized Boltzmann equations. Using a Lénard--Bernstein collision operator, these take Fokker--Planck-like forms \cite{Fokker_1914, Planck_1917} wherein pseudo-particles in the numerical model obey the neoclassical transport equations, with particle-independent Brownian drift terms. This offers a rigorous methodology for incorporating collisions into the particle transport model, without coupling the equations of motions for each particle.
        
        Works by Chen, Chacón et al. \cite{Chen_Chacón_Barnes_2011, Chacón_Chen_Barnes_2013, Chen_Chacón_2014, Chen_Chacón_2015} have developed structure-preserving particle pushers for neoclassical transport in the Vlasov equations, derived from Crank--Nicolson integrators. We show these too can can derive from a FET interpretation, similarly offering potential extensions to higher-order-in-time particle pushers. The FET formulation is used also to consider how the stochastic drift terms can be incorporated into the pushers. Stochastic gyrokinetic expansions are also discussed.

        Different options for the numerical implementation of these schemes are considered.

        Due to the efficacy of FET in the development of SP timesteppers for both the fluid and kinetic component, we hope this approach will prove effective in the future for developing SP timesteppers for the full hybrid model. We hope this will give us the opportunity to incorporate previously inaccessible kinetic effects into the highly effective, modern, finite-element MHD models.
    \end{abstract}
    
    
    \newpage
    \tableofcontents
    
    
    \newpage
    \pagenumbering{arabic}
    %\linenumbers\renewcommand\thelinenumber{\color{black!50}\arabic{linenumber}}
            \input{0 - introduction/main.tex}
        \part{Research}
            \input{1 - low-noise PiC models/main.tex}
            \input{2 - kinetic component/main.tex}
            \input{3 - fluid component/main.tex}
            \input{4 - numerical implementation/main.tex}
        \part{Project Overview}
            \input{5 - research plan/main.tex}
            \input{6 - summary/main.tex}
    
    
    %\section{}
    \newpage
    \pagenumbering{gobble}
        \printbibliography


    \newpage
    \pagenumbering{roman}
    \appendix
        \part{Appendices}
            \input{8 - Hilbert complexes/main.tex}
            \input{9 - weak conservation proofs/main.tex}
\end{document}

            \documentclass[12pt, a4paper]{report}

\input{template/main.tex}

\title{\BA{Title in Progress...}}
\author{Boris Andrews}
\affil{Mathematical Institute, University of Oxford}
\date{\today}


\begin{document}
    \pagenumbering{gobble}
    \maketitle
    
    
    \begin{abstract}
        Magnetic confinement reactors---in particular tokamaks---offer one of the most promising options for achieving practical nuclear fusion, with the potential to provide virtually limitless, clean energy. The theoretical and numerical modeling of tokamak plasmas is simultaneously an essential component of effective reactor design, and a great research barrier. Tokamak operational conditions exhibit comparatively low Knudsen numbers. Kinetic effects, including kinetic waves and instabilities, Landau damping, bump-on-tail instabilities and more, are therefore highly influential in tokamak plasma dynamics. Purely fluid models are inherently incapable of capturing these effects, whereas the high dimensionality in purely kinetic models render them practically intractable for most relevant purposes.

        We consider a $\delta\!f$ decomposition model, with a macroscopic fluid background and microscopic kinetic correction, both fully coupled to each other. A similar manner of discretization is proposed to that used in the recent \texttt{STRUPHY} code \cite{Holderied_Possanner_Wang_2021, Holderied_2022, Li_et_al_2023} with a finite-element model for the background and a pseudo-particle/PiC model for the correction.

        The fluid background satisfies the full, non-linear, resistive, compressible, Hall MHD equations. \cite{Laakmann_Hu_Farrell_2022} introduces finite-element(-in-space) implicit timesteppers for the incompressible analogue to this system with structure-preserving (SP) properties in the ideal case, alongside parameter-robust preconditioners. We show that these timesteppers can derive from a finite-element-in-time (FET) (and finite-element-in-space) interpretation. The benefits of this reformulation are discussed, including the derivation of timesteppers that are higher order in time, and the quantifiable dissipative SP properties in the non-ideal, resistive case.
        
        We discuss possible options for extending this FET approach to timesteppers for the compressible case.

        The kinetic corrections satisfy linearized Boltzmann equations. Using a Lénard--Bernstein collision operator, these take Fokker--Planck-like forms \cite{Fokker_1914, Planck_1917} wherein pseudo-particles in the numerical model obey the neoclassical transport equations, with particle-independent Brownian drift terms. This offers a rigorous methodology for incorporating collisions into the particle transport model, without coupling the equations of motions for each particle.
        
        Works by Chen, Chacón et al. \cite{Chen_Chacón_Barnes_2011, Chacón_Chen_Barnes_2013, Chen_Chacón_2014, Chen_Chacón_2015} have developed structure-preserving particle pushers for neoclassical transport in the Vlasov equations, derived from Crank--Nicolson integrators. We show these too can can derive from a FET interpretation, similarly offering potential extensions to higher-order-in-time particle pushers. The FET formulation is used also to consider how the stochastic drift terms can be incorporated into the pushers. Stochastic gyrokinetic expansions are also discussed.

        Different options for the numerical implementation of these schemes are considered.

        Due to the efficacy of FET in the development of SP timesteppers for both the fluid and kinetic component, we hope this approach will prove effective in the future for developing SP timesteppers for the full hybrid model. We hope this will give us the opportunity to incorporate previously inaccessible kinetic effects into the highly effective, modern, finite-element MHD models.
    \end{abstract}
    
    
    \newpage
    \tableofcontents
    
    
    \newpage
    \pagenumbering{arabic}
    %\linenumbers\renewcommand\thelinenumber{\color{black!50}\arabic{linenumber}}
            \input{0 - introduction/main.tex}
        \part{Research}
            \input{1 - low-noise PiC models/main.tex}
            \input{2 - kinetic component/main.tex}
            \input{3 - fluid component/main.tex}
            \input{4 - numerical implementation/main.tex}
        \part{Project Overview}
            \input{5 - research plan/main.tex}
            \input{6 - summary/main.tex}
    
    
    %\section{}
    \newpage
    \pagenumbering{gobble}
        \printbibliography


    \newpage
    \pagenumbering{roman}
    \appendix
        \part{Appendices}
            \input{8 - Hilbert complexes/main.tex}
            \input{9 - weak conservation proofs/main.tex}
\end{document}

        \part{Project Overview}
            \documentclass[12pt, a4paper]{report}

\input{template/main.tex}

\title{\BA{Title in Progress...}}
\author{Boris Andrews}
\affil{Mathematical Institute, University of Oxford}
\date{\today}


\begin{document}
    \pagenumbering{gobble}
    \maketitle
    
    
    \begin{abstract}
        Magnetic confinement reactors---in particular tokamaks---offer one of the most promising options for achieving practical nuclear fusion, with the potential to provide virtually limitless, clean energy. The theoretical and numerical modeling of tokamak plasmas is simultaneously an essential component of effective reactor design, and a great research barrier. Tokamak operational conditions exhibit comparatively low Knudsen numbers. Kinetic effects, including kinetic waves and instabilities, Landau damping, bump-on-tail instabilities and more, are therefore highly influential in tokamak plasma dynamics. Purely fluid models are inherently incapable of capturing these effects, whereas the high dimensionality in purely kinetic models render them practically intractable for most relevant purposes.

        We consider a $\delta\!f$ decomposition model, with a macroscopic fluid background and microscopic kinetic correction, both fully coupled to each other. A similar manner of discretization is proposed to that used in the recent \texttt{STRUPHY} code \cite{Holderied_Possanner_Wang_2021, Holderied_2022, Li_et_al_2023} with a finite-element model for the background and a pseudo-particle/PiC model for the correction.

        The fluid background satisfies the full, non-linear, resistive, compressible, Hall MHD equations. \cite{Laakmann_Hu_Farrell_2022} introduces finite-element(-in-space) implicit timesteppers for the incompressible analogue to this system with structure-preserving (SP) properties in the ideal case, alongside parameter-robust preconditioners. We show that these timesteppers can derive from a finite-element-in-time (FET) (and finite-element-in-space) interpretation. The benefits of this reformulation are discussed, including the derivation of timesteppers that are higher order in time, and the quantifiable dissipative SP properties in the non-ideal, resistive case.
        
        We discuss possible options for extending this FET approach to timesteppers for the compressible case.

        The kinetic corrections satisfy linearized Boltzmann equations. Using a Lénard--Bernstein collision operator, these take Fokker--Planck-like forms \cite{Fokker_1914, Planck_1917} wherein pseudo-particles in the numerical model obey the neoclassical transport equations, with particle-independent Brownian drift terms. This offers a rigorous methodology for incorporating collisions into the particle transport model, without coupling the equations of motions for each particle.
        
        Works by Chen, Chacón et al. \cite{Chen_Chacón_Barnes_2011, Chacón_Chen_Barnes_2013, Chen_Chacón_2014, Chen_Chacón_2015} have developed structure-preserving particle pushers for neoclassical transport in the Vlasov equations, derived from Crank--Nicolson integrators. We show these too can can derive from a FET interpretation, similarly offering potential extensions to higher-order-in-time particle pushers. The FET formulation is used also to consider how the stochastic drift terms can be incorporated into the pushers. Stochastic gyrokinetic expansions are also discussed.

        Different options for the numerical implementation of these schemes are considered.

        Due to the efficacy of FET in the development of SP timesteppers for both the fluid and kinetic component, we hope this approach will prove effective in the future for developing SP timesteppers for the full hybrid model. We hope this will give us the opportunity to incorporate previously inaccessible kinetic effects into the highly effective, modern, finite-element MHD models.
    \end{abstract}
    
    
    \newpage
    \tableofcontents
    
    
    \newpage
    \pagenumbering{arabic}
    %\linenumbers\renewcommand\thelinenumber{\color{black!50}\arabic{linenumber}}
            \input{0 - introduction/main.tex}
        \part{Research}
            \input{1 - low-noise PiC models/main.tex}
            \input{2 - kinetic component/main.tex}
            \input{3 - fluid component/main.tex}
            \input{4 - numerical implementation/main.tex}
        \part{Project Overview}
            \input{5 - research plan/main.tex}
            \input{6 - summary/main.tex}
    
    
    %\section{}
    \newpage
    \pagenumbering{gobble}
        \printbibliography


    \newpage
    \pagenumbering{roman}
    \appendix
        \part{Appendices}
            \input{8 - Hilbert complexes/main.tex}
            \input{9 - weak conservation proofs/main.tex}
\end{document}

            \documentclass[12pt, a4paper]{report}

\input{template/main.tex}

\title{\BA{Title in Progress...}}
\author{Boris Andrews}
\affil{Mathematical Institute, University of Oxford}
\date{\today}


\begin{document}
    \pagenumbering{gobble}
    \maketitle
    
    
    \begin{abstract}
        Magnetic confinement reactors---in particular tokamaks---offer one of the most promising options for achieving practical nuclear fusion, with the potential to provide virtually limitless, clean energy. The theoretical and numerical modeling of tokamak plasmas is simultaneously an essential component of effective reactor design, and a great research barrier. Tokamak operational conditions exhibit comparatively low Knudsen numbers. Kinetic effects, including kinetic waves and instabilities, Landau damping, bump-on-tail instabilities and more, are therefore highly influential in tokamak plasma dynamics. Purely fluid models are inherently incapable of capturing these effects, whereas the high dimensionality in purely kinetic models render them practically intractable for most relevant purposes.

        We consider a $\delta\!f$ decomposition model, with a macroscopic fluid background and microscopic kinetic correction, both fully coupled to each other. A similar manner of discretization is proposed to that used in the recent \texttt{STRUPHY} code \cite{Holderied_Possanner_Wang_2021, Holderied_2022, Li_et_al_2023} with a finite-element model for the background and a pseudo-particle/PiC model for the correction.

        The fluid background satisfies the full, non-linear, resistive, compressible, Hall MHD equations. \cite{Laakmann_Hu_Farrell_2022} introduces finite-element(-in-space) implicit timesteppers for the incompressible analogue to this system with structure-preserving (SP) properties in the ideal case, alongside parameter-robust preconditioners. We show that these timesteppers can derive from a finite-element-in-time (FET) (and finite-element-in-space) interpretation. The benefits of this reformulation are discussed, including the derivation of timesteppers that are higher order in time, and the quantifiable dissipative SP properties in the non-ideal, resistive case.
        
        We discuss possible options for extending this FET approach to timesteppers for the compressible case.

        The kinetic corrections satisfy linearized Boltzmann equations. Using a Lénard--Bernstein collision operator, these take Fokker--Planck-like forms \cite{Fokker_1914, Planck_1917} wherein pseudo-particles in the numerical model obey the neoclassical transport equations, with particle-independent Brownian drift terms. This offers a rigorous methodology for incorporating collisions into the particle transport model, without coupling the equations of motions for each particle.
        
        Works by Chen, Chacón et al. \cite{Chen_Chacón_Barnes_2011, Chacón_Chen_Barnes_2013, Chen_Chacón_2014, Chen_Chacón_2015} have developed structure-preserving particle pushers for neoclassical transport in the Vlasov equations, derived from Crank--Nicolson integrators. We show these too can can derive from a FET interpretation, similarly offering potential extensions to higher-order-in-time particle pushers. The FET formulation is used also to consider how the stochastic drift terms can be incorporated into the pushers. Stochastic gyrokinetic expansions are also discussed.

        Different options for the numerical implementation of these schemes are considered.

        Due to the efficacy of FET in the development of SP timesteppers for both the fluid and kinetic component, we hope this approach will prove effective in the future for developing SP timesteppers for the full hybrid model. We hope this will give us the opportunity to incorporate previously inaccessible kinetic effects into the highly effective, modern, finite-element MHD models.
    \end{abstract}
    
    
    \newpage
    \tableofcontents
    
    
    \newpage
    \pagenumbering{arabic}
    %\linenumbers\renewcommand\thelinenumber{\color{black!50}\arabic{linenumber}}
            \input{0 - introduction/main.tex}
        \part{Research}
            \input{1 - low-noise PiC models/main.tex}
            \input{2 - kinetic component/main.tex}
            \input{3 - fluid component/main.tex}
            \input{4 - numerical implementation/main.tex}
        \part{Project Overview}
            \input{5 - research plan/main.tex}
            \input{6 - summary/main.tex}
    
    
    %\section{}
    \newpage
    \pagenumbering{gobble}
        \printbibliography


    \newpage
    \pagenumbering{roman}
    \appendix
        \part{Appendices}
            \input{8 - Hilbert complexes/main.tex}
            \input{9 - weak conservation proofs/main.tex}
\end{document}

    
    
    %\section{}
    \newpage
    \pagenumbering{gobble}
        \printbibliography


    \newpage
    \pagenumbering{roman}
    \appendix
        \part{Appendices}
            \documentclass[12pt, a4paper]{report}

\input{template/main.tex}

\title{\BA{Title in Progress...}}
\author{Boris Andrews}
\affil{Mathematical Institute, University of Oxford}
\date{\today}


\begin{document}
    \pagenumbering{gobble}
    \maketitle
    
    
    \begin{abstract}
        Magnetic confinement reactors---in particular tokamaks---offer one of the most promising options for achieving practical nuclear fusion, with the potential to provide virtually limitless, clean energy. The theoretical and numerical modeling of tokamak plasmas is simultaneously an essential component of effective reactor design, and a great research barrier. Tokamak operational conditions exhibit comparatively low Knudsen numbers. Kinetic effects, including kinetic waves and instabilities, Landau damping, bump-on-tail instabilities and more, are therefore highly influential in tokamak plasma dynamics. Purely fluid models are inherently incapable of capturing these effects, whereas the high dimensionality in purely kinetic models render them practically intractable for most relevant purposes.

        We consider a $\delta\!f$ decomposition model, with a macroscopic fluid background and microscopic kinetic correction, both fully coupled to each other. A similar manner of discretization is proposed to that used in the recent \texttt{STRUPHY} code \cite{Holderied_Possanner_Wang_2021, Holderied_2022, Li_et_al_2023} with a finite-element model for the background and a pseudo-particle/PiC model for the correction.

        The fluid background satisfies the full, non-linear, resistive, compressible, Hall MHD equations. \cite{Laakmann_Hu_Farrell_2022} introduces finite-element(-in-space) implicit timesteppers for the incompressible analogue to this system with structure-preserving (SP) properties in the ideal case, alongside parameter-robust preconditioners. We show that these timesteppers can derive from a finite-element-in-time (FET) (and finite-element-in-space) interpretation. The benefits of this reformulation are discussed, including the derivation of timesteppers that are higher order in time, and the quantifiable dissipative SP properties in the non-ideal, resistive case.
        
        We discuss possible options for extending this FET approach to timesteppers for the compressible case.

        The kinetic corrections satisfy linearized Boltzmann equations. Using a Lénard--Bernstein collision operator, these take Fokker--Planck-like forms \cite{Fokker_1914, Planck_1917} wherein pseudo-particles in the numerical model obey the neoclassical transport equations, with particle-independent Brownian drift terms. This offers a rigorous methodology for incorporating collisions into the particle transport model, without coupling the equations of motions for each particle.
        
        Works by Chen, Chacón et al. \cite{Chen_Chacón_Barnes_2011, Chacón_Chen_Barnes_2013, Chen_Chacón_2014, Chen_Chacón_2015} have developed structure-preserving particle pushers for neoclassical transport in the Vlasov equations, derived from Crank--Nicolson integrators. We show these too can can derive from a FET interpretation, similarly offering potential extensions to higher-order-in-time particle pushers. The FET formulation is used also to consider how the stochastic drift terms can be incorporated into the pushers. Stochastic gyrokinetic expansions are also discussed.

        Different options for the numerical implementation of these schemes are considered.

        Due to the efficacy of FET in the development of SP timesteppers for both the fluid and kinetic component, we hope this approach will prove effective in the future for developing SP timesteppers for the full hybrid model. We hope this will give us the opportunity to incorporate previously inaccessible kinetic effects into the highly effective, modern, finite-element MHD models.
    \end{abstract}
    
    
    \newpage
    \tableofcontents
    
    
    \newpage
    \pagenumbering{arabic}
    %\linenumbers\renewcommand\thelinenumber{\color{black!50}\arabic{linenumber}}
            \input{0 - introduction/main.tex}
        \part{Research}
            \input{1 - low-noise PiC models/main.tex}
            \input{2 - kinetic component/main.tex}
            \input{3 - fluid component/main.tex}
            \input{4 - numerical implementation/main.tex}
        \part{Project Overview}
            \input{5 - research plan/main.tex}
            \input{6 - summary/main.tex}
    
    
    %\section{}
    \newpage
    \pagenumbering{gobble}
        \printbibliography


    \newpage
    \pagenumbering{roman}
    \appendix
        \part{Appendices}
            \input{8 - Hilbert complexes/main.tex}
            \input{9 - weak conservation proofs/main.tex}
\end{document}

            \documentclass[12pt, a4paper]{report}

\input{template/main.tex}

\title{\BA{Title in Progress...}}
\author{Boris Andrews}
\affil{Mathematical Institute, University of Oxford}
\date{\today}


\begin{document}
    \pagenumbering{gobble}
    \maketitle
    
    
    \begin{abstract}
        Magnetic confinement reactors---in particular tokamaks---offer one of the most promising options for achieving practical nuclear fusion, with the potential to provide virtually limitless, clean energy. The theoretical and numerical modeling of tokamak plasmas is simultaneously an essential component of effective reactor design, and a great research barrier. Tokamak operational conditions exhibit comparatively low Knudsen numbers. Kinetic effects, including kinetic waves and instabilities, Landau damping, bump-on-tail instabilities and more, are therefore highly influential in tokamak plasma dynamics. Purely fluid models are inherently incapable of capturing these effects, whereas the high dimensionality in purely kinetic models render them practically intractable for most relevant purposes.

        We consider a $\delta\!f$ decomposition model, with a macroscopic fluid background and microscopic kinetic correction, both fully coupled to each other. A similar manner of discretization is proposed to that used in the recent \texttt{STRUPHY} code \cite{Holderied_Possanner_Wang_2021, Holderied_2022, Li_et_al_2023} with a finite-element model for the background and a pseudo-particle/PiC model for the correction.

        The fluid background satisfies the full, non-linear, resistive, compressible, Hall MHD equations. \cite{Laakmann_Hu_Farrell_2022} introduces finite-element(-in-space) implicit timesteppers for the incompressible analogue to this system with structure-preserving (SP) properties in the ideal case, alongside parameter-robust preconditioners. We show that these timesteppers can derive from a finite-element-in-time (FET) (and finite-element-in-space) interpretation. The benefits of this reformulation are discussed, including the derivation of timesteppers that are higher order in time, and the quantifiable dissipative SP properties in the non-ideal, resistive case.
        
        We discuss possible options for extending this FET approach to timesteppers for the compressible case.

        The kinetic corrections satisfy linearized Boltzmann equations. Using a Lénard--Bernstein collision operator, these take Fokker--Planck-like forms \cite{Fokker_1914, Planck_1917} wherein pseudo-particles in the numerical model obey the neoclassical transport equations, with particle-independent Brownian drift terms. This offers a rigorous methodology for incorporating collisions into the particle transport model, without coupling the equations of motions for each particle.
        
        Works by Chen, Chacón et al. \cite{Chen_Chacón_Barnes_2011, Chacón_Chen_Barnes_2013, Chen_Chacón_2014, Chen_Chacón_2015} have developed structure-preserving particle pushers for neoclassical transport in the Vlasov equations, derived from Crank--Nicolson integrators. We show these too can can derive from a FET interpretation, similarly offering potential extensions to higher-order-in-time particle pushers. The FET formulation is used also to consider how the stochastic drift terms can be incorporated into the pushers. Stochastic gyrokinetic expansions are also discussed.

        Different options for the numerical implementation of these schemes are considered.

        Due to the efficacy of FET in the development of SP timesteppers for both the fluid and kinetic component, we hope this approach will prove effective in the future for developing SP timesteppers for the full hybrid model. We hope this will give us the opportunity to incorporate previously inaccessible kinetic effects into the highly effective, modern, finite-element MHD models.
    \end{abstract}
    
    
    \newpage
    \tableofcontents
    
    
    \newpage
    \pagenumbering{arabic}
    %\linenumbers\renewcommand\thelinenumber{\color{black!50}\arabic{linenumber}}
            \input{0 - introduction/main.tex}
        \part{Research}
            \input{1 - low-noise PiC models/main.tex}
            \input{2 - kinetic component/main.tex}
            \input{3 - fluid component/main.tex}
            \input{4 - numerical implementation/main.tex}
        \part{Project Overview}
            \input{5 - research plan/main.tex}
            \input{6 - summary/main.tex}
    
    
    %\section{}
    \newpage
    \pagenumbering{gobble}
        \printbibliography


    \newpage
    \pagenumbering{roman}
    \appendix
        \part{Appendices}
            \input{8 - Hilbert complexes/main.tex}
            \input{9 - weak conservation proofs/main.tex}
\end{document}

\end{document}

        \part{Research}
            \documentclass[12pt, a4paper]{report}

\documentclass[12pt, a4paper]{report}

\input{template/main.tex}

\title{\BA{Title in Progress...}}
\author{Boris Andrews}
\affil{Mathematical Institute, University of Oxford}
\date{\today}


\begin{document}
    \pagenumbering{gobble}
    \maketitle
    
    
    \begin{abstract}
        Magnetic confinement reactors---in particular tokamaks---offer one of the most promising options for achieving practical nuclear fusion, with the potential to provide virtually limitless, clean energy. The theoretical and numerical modeling of tokamak plasmas is simultaneously an essential component of effective reactor design, and a great research barrier. Tokamak operational conditions exhibit comparatively low Knudsen numbers. Kinetic effects, including kinetic waves and instabilities, Landau damping, bump-on-tail instabilities and more, are therefore highly influential in tokamak plasma dynamics. Purely fluid models are inherently incapable of capturing these effects, whereas the high dimensionality in purely kinetic models render them practically intractable for most relevant purposes.

        We consider a $\delta\!f$ decomposition model, with a macroscopic fluid background and microscopic kinetic correction, both fully coupled to each other. A similar manner of discretization is proposed to that used in the recent \texttt{STRUPHY} code \cite{Holderied_Possanner_Wang_2021, Holderied_2022, Li_et_al_2023} with a finite-element model for the background and a pseudo-particle/PiC model for the correction.

        The fluid background satisfies the full, non-linear, resistive, compressible, Hall MHD equations. \cite{Laakmann_Hu_Farrell_2022} introduces finite-element(-in-space) implicit timesteppers for the incompressible analogue to this system with structure-preserving (SP) properties in the ideal case, alongside parameter-robust preconditioners. We show that these timesteppers can derive from a finite-element-in-time (FET) (and finite-element-in-space) interpretation. The benefits of this reformulation are discussed, including the derivation of timesteppers that are higher order in time, and the quantifiable dissipative SP properties in the non-ideal, resistive case.
        
        We discuss possible options for extending this FET approach to timesteppers for the compressible case.

        The kinetic corrections satisfy linearized Boltzmann equations. Using a Lénard--Bernstein collision operator, these take Fokker--Planck-like forms \cite{Fokker_1914, Planck_1917} wherein pseudo-particles in the numerical model obey the neoclassical transport equations, with particle-independent Brownian drift terms. This offers a rigorous methodology for incorporating collisions into the particle transport model, without coupling the equations of motions for each particle.
        
        Works by Chen, Chacón et al. \cite{Chen_Chacón_Barnes_2011, Chacón_Chen_Barnes_2013, Chen_Chacón_2014, Chen_Chacón_2015} have developed structure-preserving particle pushers for neoclassical transport in the Vlasov equations, derived from Crank--Nicolson integrators. We show these too can can derive from a FET interpretation, similarly offering potential extensions to higher-order-in-time particle pushers. The FET formulation is used also to consider how the stochastic drift terms can be incorporated into the pushers. Stochastic gyrokinetic expansions are also discussed.

        Different options for the numerical implementation of these schemes are considered.

        Due to the efficacy of FET in the development of SP timesteppers for both the fluid and kinetic component, we hope this approach will prove effective in the future for developing SP timesteppers for the full hybrid model. We hope this will give us the opportunity to incorporate previously inaccessible kinetic effects into the highly effective, modern, finite-element MHD models.
    \end{abstract}
    
    
    \newpage
    \tableofcontents
    
    
    \newpage
    \pagenumbering{arabic}
    %\linenumbers\renewcommand\thelinenumber{\color{black!50}\arabic{linenumber}}
            \input{0 - introduction/main.tex}
        \part{Research}
            \input{1 - low-noise PiC models/main.tex}
            \input{2 - kinetic component/main.tex}
            \input{3 - fluid component/main.tex}
            \input{4 - numerical implementation/main.tex}
        \part{Project Overview}
            \input{5 - research plan/main.tex}
            \input{6 - summary/main.tex}
    
    
    %\section{}
    \newpage
    \pagenumbering{gobble}
        \printbibliography


    \newpage
    \pagenumbering{roman}
    \appendix
        \part{Appendices}
            \input{8 - Hilbert complexes/main.tex}
            \input{9 - weak conservation proofs/main.tex}
\end{document}


\title{\BA{Title in Progress...}}
\author{Boris Andrews}
\affil{Mathematical Institute, University of Oxford}
\date{\today}


\begin{document}
    \pagenumbering{gobble}
    \maketitle
    
    
    \begin{abstract}
        Magnetic confinement reactors---in particular tokamaks---offer one of the most promising options for achieving practical nuclear fusion, with the potential to provide virtually limitless, clean energy. The theoretical and numerical modeling of tokamak plasmas is simultaneously an essential component of effective reactor design, and a great research barrier. Tokamak operational conditions exhibit comparatively low Knudsen numbers. Kinetic effects, including kinetic waves and instabilities, Landau damping, bump-on-tail instabilities and more, are therefore highly influential in tokamak plasma dynamics. Purely fluid models are inherently incapable of capturing these effects, whereas the high dimensionality in purely kinetic models render them practically intractable for most relevant purposes.

        We consider a $\delta\!f$ decomposition model, with a macroscopic fluid background and microscopic kinetic correction, both fully coupled to each other. A similar manner of discretization is proposed to that used in the recent \texttt{STRUPHY} code \cite{Holderied_Possanner_Wang_2021, Holderied_2022, Li_et_al_2023} with a finite-element model for the background and a pseudo-particle/PiC model for the correction.

        The fluid background satisfies the full, non-linear, resistive, compressible, Hall MHD equations. \cite{Laakmann_Hu_Farrell_2022} introduces finite-element(-in-space) implicit timesteppers for the incompressible analogue to this system with structure-preserving (SP) properties in the ideal case, alongside parameter-robust preconditioners. We show that these timesteppers can derive from a finite-element-in-time (FET) (and finite-element-in-space) interpretation. The benefits of this reformulation are discussed, including the derivation of timesteppers that are higher order in time, and the quantifiable dissipative SP properties in the non-ideal, resistive case.
        
        We discuss possible options for extending this FET approach to timesteppers for the compressible case.

        The kinetic corrections satisfy linearized Boltzmann equations. Using a Lénard--Bernstein collision operator, these take Fokker--Planck-like forms \cite{Fokker_1914, Planck_1917} wherein pseudo-particles in the numerical model obey the neoclassical transport equations, with particle-independent Brownian drift terms. This offers a rigorous methodology for incorporating collisions into the particle transport model, without coupling the equations of motions for each particle.
        
        Works by Chen, Chacón et al. \cite{Chen_Chacón_Barnes_2011, Chacón_Chen_Barnes_2013, Chen_Chacón_2014, Chen_Chacón_2015} have developed structure-preserving particle pushers for neoclassical transport in the Vlasov equations, derived from Crank--Nicolson integrators. We show these too can can derive from a FET interpretation, similarly offering potential extensions to higher-order-in-time particle pushers. The FET formulation is used also to consider how the stochastic drift terms can be incorporated into the pushers. Stochastic gyrokinetic expansions are also discussed.

        Different options for the numerical implementation of these schemes are considered.

        Due to the efficacy of FET in the development of SP timesteppers for both the fluid and kinetic component, we hope this approach will prove effective in the future for developing SP timesteppers for the full hybrid model. We hope this will give us the opportunity to incorporate previously inaccessible kinetic effects into the highly effective, modern, finite-element MHD models.
    \end{abstract}
    
    
    \newpage
    \tableofcontents
    
    
    \newpage
    \pagenumbering{arabic}
    %\linenumbers\renewcommand\thelinenumber{\color{black!50}\arabic{linenumber}}
            \documentclass[12pt, a4paper]{report}

\input{template/main.tex}

\title{\BA{Title in Progress...}}
\author{Boris Andrews}
\affil{Mathematical Institute, University of Oxford}
\date{\today}


\begin{document}
    \pagenumbering{gobble}
    \maketitle
    
    
    \begin{abstract}
        Magnetic confinement reactors---in particular tokamaks---offer one of the most promising options for achieving practical nuclear fusion, with the potential to provide virtually limitless, clean energy. The theoretical and numerical modeling of tokamak plasmas is simultaneously an essential component of effective reactor design, and a great research barrier. Tokamak operational conditions exhibit comparatively low Knudsen numbers. Kinetic effects, including kinetic waves and instabilities, Landau damping, bump-on-tail instabilities and more, are therefore highly influential in tokamak plasma dynamics. Purely fluid models are inherently incapable of capturing these effects, whereas the high dimensionality in purely kinetic models render them practically intractable for most relevant purposes.

        We consider a $\delta\!f$ decomposition model, with a macroscopic fluid background and microscopic kinetic correction, both fully coupled to each other. A similar manner of discretization is proposed to that used in the recent \texttt{STRUPHY} code \cite{Holderied_Possanner_Wang_2021, Holderied_2022, Li_et_al_2023} with a finite-element model for the background and a pseudo-particle/PiC model for the correction.

        The fluid background satisfies the full, non-linear, resistive, compressible, Hall MHD equations. \cite{Laakmann_Hu_Farrell_2022} introduces finite-element(-in-space) implicit timesteppers for the incompressible analogue to this system with structure-preserving (SP) properties in the ideal case, alongside parameter-robust preconditioners. We show that these timesteppers can derive from a finite-element-in-time (FET) (and finite-element-in-space) interpretation. The benefits of this reformulation are discussed, including the derivation of timesteppers that are higher order in time, and the quantifiable dissipative SP properties in the non-ideal, resistive case.
        
        We discuss possible options for extending this FET approach to timesteppers for the compressible case.

        The kinetic corrections satisfy linearized Boltzmann equations. Using a Lénard--Bernstein collision operator, these take Fokker--Planck-like forms \cite{Fokker_1914, Planck_1917} wherein pseudo-particles in the numerical model obey the neoclassical transport equations, with particle-independent Brownian drift terms. This offers a rigorous methodology for incorporating collisions into the particle transport model, without coupling the equations of motions for each particle.
        
        Works by Chen, Chacón et al. \cite{Chen_Chacón_Barnes_2011, Chacón_Chen_Barnes_2013, Chen_Chacón_2014, Chen_Chacón_2015} have developed structure-preserving particle pushers for neoclassical transport in the Vlasov equations, derived from Crank--Nicolson integrators. We show these too can can derive from a FET interpretation, similarly offering potential extensions to higher-order-in-time particle pushers. The FET formulation is used also to consider how the stochastic drift terms can be incorporated into the pushers. Stochastic gyrokinetic expansions are also discussed.

        Different options for the numerical implementation of these schemes are considered.

        Due to the efficacy of FET in the development of SP timesteppers for both the fluid and kinetic component, we hope this approach will prove effective in the future for developing SP timesteppers for the full hybrid model. We hope this will give us the opportunity to incorporate previously inaccessible kinetic effects into the highly effective, modern, finite-element MHD models.
    \end{abstract}
    
    
    \newpage
    \tableofcontents
    
    
    \newpage
    \pagenumbering{arabic}
    %\linenumbers\renewcommand\thelinenumber{\color{black!50}\arabic{linenumber}}
            \input{0 - introduction/main.tex}
        \part{Research}
            \input{1 - low-noise PiC models/main.tex}
            \input{2 - kinetic component/main.tex}
            \input{3 - fluid component/main.tex}
            \input{4 - numerical implementation/main.tex}
        \part{Project Overview}
            \input{5 - research plan/main.tex}
            \input{6 - summary/main.tex}
    
    
    %\section{}
    \newpage
    \pagenumbering{gobble}
        \printbibliography


    \newpage
    \pagenumbering{roman}
    \appendix
        \part{Appendices}
            \input{8 - Hilbert complexes/main.tex}
            \input{9 - weak conservation proofs/main.tex}
\end{document}

        \part{Research}
            \documentclass[12pt, a4paper]{report}

\input{template/main.tex}

\title{\BA{Title in Progress...}}
\author{Boris Andrews}
\affil{Mathematical Institute, University of Oxford}
\date{\today}


\begin{document}
    \pagenumbering{gobble}
    \maketitle
    
    
    \begin{abstract}
        Magnetic confinement reactors---in particular tokamaks---offer one of the most promising options for achieving practical nuclear fusion, with the potential to provide virtually limitless, clean energy. The theoretical and numerical modeling of tokamak plasmas is simultaneously an essential component of effective reactor design, and a great research barrier. Tokamak operational conditions exhibit comparatively low Knudsen numbers. Kinetic effects, including kinetic waves and instabilities, Landau damping, bump-on-tail instabilities and more, are therefore highly influential in tokamak plasma dynamics. Purely fluid models are inherently incapable of capturing these effects, whereas the high dimensionality in purely kinetic models render them practically intractable for most relevant purposes.

        We consider a $\delta\!f$ decomposition model, with a macroscopic fluid background and microscopic kinetic correction, both fully coupled to each other. A similar manner of discretization is proposed to that used in the recent \texttt{STRUPHY} code \cite{Holderied_Possanner_Wang_2021, Holderied_2022, Li_et_al_2023} with a finite-element model for the background and a pseudo-particle/PiC model for the correction.

        The fluid background satisfies the full, non-linear, resistive, compressible, Hall MHD equations. \cite{Laakmann_Hu_Farrell_2022} introduces finite-element(-in-space) implicit timesteppers for the incompressible analogue to this system with structure-preserving (SP) properties in the ideal case, alongside parameter-robust preconditioners. We show that these timesteppers can derive from a finite-element-in-time (FET) (and finite-element-in-space) interpretation. The benefits of this reformulation are discussed, including the derivation of timesteppers that are higher order in time, and the quantifiable dissipative SP properties in the non-ideal, resistive case.
        
        We discuss possible options for extending this FET approach to timesteppers for the compressible case.

        The kinetic corrections satisfy linearized Boltzmann equations. Using a Lénard--Bernstein collision operator, these take Fokker--Planck-like forms \cite{Fokker_1914, Planck_1917} wherein pseudo-particles in the numerical model obey the neoclassical transport equations, with particle-independent Brownian drift terms. This offers a rigorous methodology for incorporating collisions into the particle transport model, without coupling the equations of motions for each particle.
        
        Works by Chen, Chacón et al. \cite{Chen_Chacón_Barnes_2011, Chacón_Chen_Barnes_2013, Chen_Chacón_2014, Chen_Chacón_2015} have developed structure-preserving particle pushers for neoclassical transport in the Vlasov equations, derived from Crank--Nicolson integrators. We show these too can can derive from a FET interpretation, similarly offering potential extensions to higher-order-in-time particle pushers. The FET formulation is used also to consider how the stochastic drift terms can be incorporated into the pushers. Stochastic gyrokinetic expansions are also discussed.

        Different options for the numerical implementation of these schemes are considered.

        Due to the efficacy of FET in the development of SP timesteppers for both the fluid and kinetic component, we hope this approach will prove effective in the future for developing SP timesteppers for the full hybrid model. We hope this will give us the opportunity to incorporate previously inaccessible kinetic effects into the highly effective, modern, finite-element MHD models.
    \end{abstract}
    
    
    \newpage
    \tableofcontents
    
    
    \newpage
    \pagenumbering{arabic}
    %\linenumbers\renewcommand\thelinenumber{\color{black!50}\arabic{linenumber}}
            \input{0 - introduction/main.tex}
        \part{Research}
            \input{1 - low-noise PiC models/main.tex}
            \input{2 - kinetic component/main.tex}
            \input{3 - fluid component/main.tex}
            \input{4 - numerical implementation/main.tex}
        \part{Project Overview}
            \input{5 - research plan/main.tex}
            \input{6 - summary/main.tex}
    
    
    %\section{}
    \newpage
    \pagenumbering{gobble}
        \printbibliography


    \newpage
    \pagenumbering{roman}
    \appendix
        \part{Appendices}
            \input{8 - Hilbert complexes/main.tex}
            \input{9 - weak conservation proofs/main.tex}
\end{document}

            \documentclass[12pt, a4paper]{report}

\input{template/main.tex}

\title{\BA{Title in Progress...}}
\author{Boris Andrews}
\affil{Mathematical Institute, University of Oxford}
\date{\today}


\begin{document}
    \pagenumbering{gobble}
    \maketitle
    
    
    \begin{abstract}
        Magnetic confinement reactors---in particular tokamaks---offer one of the most promising options for achieving practical nuclear fusion, with the potential to provide virtually limitless, clean energy. The theoretical and numerical modeling of tokamak plasmas is simultaneously an essential component of effective reactor design, and a great research barrier. Tokamak operational conditions exhibit comparatively low Knudsen numbers. Kinetic effects, including kinetic waves and instabilities, Landau damping, bump-on-tail instabilities and more, are therefore highly influential in tokamak plasma dynamics. Purely fluid models are inherently incapable of capturing these effects, whereas the high dimensionality in purely kinetic models render them practically intractable for most relevant purposes.

        We consider a $\delta\!f$ decomposition model, with a macroscopic fluid background and microscopic kinetic correction, both fully coupled to each other. A similar manner of discretization is proposed to that used in the recent \texttt{STRUPHY} code \cite{Holderied_Possanner_Wang_2021, Holderied_2022, Li_et_al_2023} with a finite-element model for the background and a pseudo-particle/PiC model for the correction.

        The fluid background satisfies the full, non-linear, resistive, compressible, Hall MHD equations. \cite{Laakmann_Hu_Farrell_2022} introduces finite-element(-in-space) implicit timesteppers for the incompressible analogue to this system with structure-preserving (SP) properties in the ideal case, alongside parameter-robust preconditioners. We show that these timesteppers can derive from a finite-element-in-time (FET) (and finite-element-in-space) interpretation. The benefits of this reformulation are discussed, including the derivation of timesteppers that are higher order in time, and the quantifiable dissipative SP properties in the non-ideal, resistive case.
        
        We discuss possible options for extending this FET approach to timesteppers for the compressible case.

        The kinetic corrections satisfy linearized Boltzmann equations. Using a Lénard--Bernstein collision operator, these take Fokker--Planck-like forms \cite{Fokker_1914, Planck_1917} wherein pseudo-particles in the numerical model obey the neoclassical transport equations, with particle-independent Brownian drift terms. This offers a rigorous methodology for incorporating collisions into the particle transport model, without coupling the equations of motions for each particle.
        
        Works by Chen, Chacón et al. \cite{Chen_Chacón_Barnes_2011, Chacón_Chen_Barnes_2013, Chen_Chacón_2014, Chen_Chacón_2015} have developed structure-preserving particle pushers for neoclassical transport in the Vlasov equations, derived from Crank--Nicolson integrators. We show these too can can derive from a FET interpretation, similarly offering potential extensions to higher-order-in-time particle pushers. The FET formulation is used also to consider how the stochastic drift terms can be incorporated into the pushers. Stochastic gyrokinetic expansions are also discussed.

        Different options for the numerical implementation of these schemes are considered.

        Due to the efficacy of FET in the development of SP timesteppers for both the fluid and kinetic component, we hope this approach will prove effective in the future for developing SP timesteppers for the full hybrid model. We hope this will give us the opportunity to incorporate previously inaccessible kinetic effects into the highly effective, modern, finite-element MHD models.
    \end{abstract}
    
    
    \newpage
    \tableofcontents
    
    
    \newpage
    \pagenumbering{arabic}
    %\linenumbers\renewcommand\thelinenumber{\color{black!50}\arabic{linenumber}}
            \input{0 - introduction/main.tex}
        \part{Research}
            \input{1 - low-noise PiC models/main.tex}
            \input{2 - kinetic component/main.tex}
            \input{3 - fluid component/main.tex}
            \input{4 - numerical implementation/main.tex}
        \part{Project Overview}
            \input{5 - research plan/main.tex}
            \input{6 - summary/main.tex}
    
    
    %\section{}
    \newpage
    \pagenumbering{gobble}
        \printbibliography


    \newpage
    \pagenumbering{roman}
    \appendix
        \part{Appendices}
            \input{8 - Hilbert complexes/main.tex}
            \input{9 - weak conservation proofs/main.tex}
\end{document}

            \documentclass[12pt, a4paper]{report}

\input{template/main.tex}

\title{\BA{Title in Progress...}}
\author{Boris Andrews}
\affil{Mathematical Institute, University of Oxford}
\date{\today}


\begin{document}
    \pagenumbering{gobble}
    \maketitle
    
    
    \begin{abstract}
        Magnetic confinement reactors---in particular tokamaks---offer one of the most promising options for achieving practical nuclear fusion, with the potential to provide virtually limitless, clean energy. The theoretical and numerical modeling of tokamak plasmas is simultaneously an essential component of effective reactor design, and a great research barrier. Tokamak operational conditions exhibit comparatively low Knudsen numbers. Kinetic effects, including kinetic waves and instabilities, Landau damping, bump-on-tail instabilities and more, are therefore highly influential in tokamak plasma dynamics. Purely fluid models are inherently incapable of capturing these effects, whereas the high dimensionality in purely kinetic models render them practically intractable for most relevant purposes.

        We consider a $\delta\!f$ decomposition model, with a macroscopic fluid background and microscopic kinetic correction, both fully coupled to each other. A similar manner of discretization is proposed to that used in the recent \texttt{STRUPHY} code \cite{Holderied_Possanner_Wang_2021, Holderied_2022, Li_et_al_2023} with a finite-element model for the background and a pseudo-particle/PiC model for the correction.

        The fluid background satisfies the full, non-linear, resistive, compressible, Hall MHD equations. \cite{Laakmann_Hu_Farrell_2022} introduces finite-element(-in-space) implicit timesteppers for the incompressible analogue to this system with structure-preserving (SP) properties in the ideal case, alongside parameter-robust preconditioners. We show that these timesteppers can derive from a finite-element-in-time (FET) (and finite-element-in-space) interpretation. The benefits of this reformulation are discussed, including the derivation of timesteppers that are higher order in time, and the quantifiable dissipative SP properties in the non-ideal, resistive case.
        
        We discuss possible options for extending this FET approach to timesteppers for the compressible case.

        The kinetic corrections satisfy linearized Boltzmann equations. Using a Lénard--Bernstein collision operator, these take Fokker--Planck-like forms \cite{Fokker_1914, Planck_1917} wherein pseudo-particles in the numerical model obey the neoclassical transport equations, with particle-independent Brownian drift terms. This offers a rigorous methodology for incorporating collisions into the particle transport model, without coupling the equations of motions for each particle.
        
        Works by Chen, Chacón et al. \cite{Chen_Chacón_Barnes_2011, Chacón_Chen_Barnes_2013, Chen_Chacón_2014, Chen_Chacón_2015} have developed structure-preserving particle pushers for neoclassical transport in the Vlasov equations, derived from Crank--Nicolson integrators. We show these too can can derive from a FET interpretation, similarly offering potential extensions to higher-order-in-time particle pushers. The FET formulation is used also to consider how the stochastic drift terms can be incorporated into the pushers. Stochastic gyrokinetic expansions are also discussed.

        Different options for the numerical implementation of these schemes are considered.

        Due to the efficacy of FET in the development of SP timesteppers for both the fluid and kinetic component, we hope this approach will prove effective in the future for developing SP timesteppers for the full hybrid model. We hope this will give us the opportunity to incorporate previously inaccessible kinetic effects into the highly effective, modern, finite-element MHD models.
    \end{abstract}
    
    
    \newpage
    \tableofcontents
    
    
    \newpage
    \pagenumbering{arabic}
    %\linenumbers\renewcommand\thelinenumber{\color{black!50}\arabic{linenumber}}
            \input{0 - introduction/main.tex}
        \part{Research}
            \input{1 - low-noise PiC models/main.tex}
            \input{2 - kinetic component/main.tex}
            \input{3 - fluid component/main.tex}
            \input{4 - numerical implementation/main.tex}
        \part{Project Overview}
            \input{5 - research plan/main.tex}
            \input{6 - summary/main.tex}
    
    
    %\section{}
    \newpage
    \pagenumbering{gobble}
        \printbibliography


    \newpage
    \pagenumbering{roman}
    \appendix
        \part{Appendices}
            \input{8 - Hilbert complexes/main.tex}
            \input{9 - weak conservation proofs/main.tex}
\end{document}

            \documentclass[12pt, a4paper]{report}

\input{template/main.tex}

\title{\BA{Title in Progress...}}
\author{Boris Andrews}
\affil{Mathematical Institute, University of Oxford}
\date{\today}


\begin{document}
    \pagenumbering{gobble}
    \maketitle
    
    
    \begin{abstract}
        Magnetic confinement reactors---in particular tokamaks---offer one of the most promising options for achieving practical nuclear fusion, with the potential to provide virtually limitless, clean energy. The theoretical and numerical modeling of tokamak plasmas is simultaneously an essential component of effective reactor design, and a great research barrier. Tokamak operational conditions exhibit comparatively low Knudsen numbers. Kinetic effects, including kinetic waves and instabilities, Landau damping, bump-on-tail instabilities and more, are therefore highly influential in tokamak plasma dynamics. Purely fluid models are inherently incapable of capturing these effects, whereas the high dimensionality in purely kinetic models render them practically intractable for most relevant purposes.

        We consider a $\delta\!f$ decomposition model, with a macroscopic fluid background and microscopic kinetic correction, both fully coupled to each other. A similar manner of discretization is proposed to that used in the recent \texttt{STRUPHY} code \cite{Holderied_Possanner_Wang_2021, Holderied_2022, Li_et_al_2023} with a finite-element model for the background and a pseudo-particle/PiC model for the correction.

        The fluid background satisfies the full, non-linear, resistive, compressible, Hall MHD equations. \cite{Laakmann_Hu_Farrell_2022} introduces finite-element(-in-space) implicit timesteppers for the incompressible analogue to this system with structure-preserving (SP) properties in the ideal case, alongside parameter-robust preconditioners. We show that these timesteppers can derive from a finite-element-in-time (FET) (and finite-element-in-space) interpretation. The benefits of this reformulation are discussed, including the derivation of timesteppers that are higher order in time, and the quantifiable dissipative SP properties in the non-ideal, resistive case.
        
        We discuss possible options for extending this FET approach to timesteppers for the compressible case.

        The kinetic corrections satisfy linearized Boltzmann equations. Using a Lénard--Bernstein collision operator, these take Fokker--Planck-like forms \cite{Fokker_1914, Planck_1917} wherein pseudo-particles in the numerical model obey the neoclassical transport equations, with particle-independent Brownian drift terms. This offers a rigorous methodology for incorporating collisions into the particle transport model, without coupling the equations of motions for each particle.
        
        Works by Chen, Chacón et al. \cite{Chen_Chacón_Barnes_2011, Chacón_Chen_Barnes_2013, Chen_Chacón_2014, Chen_Chacón_2015} have developed structure-preserving particle pushers for neoclassical transport in the Vlasov equations, derived from Crank--Nicolson integrators. We show these too can can derive from a FET interpretation, similarly offering potential extensions to higher-order-in-time particle pushers. The FET formulation is used also to consider how the stochastic drift terms can be incorporated into the pushers. Stochastic gyrokinetic expansions are also discussed.

        Different options for the numerical implementation of these schemes are considered.

        Due to the efficacy of FET in the development of SP timesteppers for both the fluid and kinetic component, we hope this approach will prove effective in the future for developing SP timesteppers for the full hybrid model. We hope this will give us the opportunity to incorporate previously inaccessible kinetic effects into the highly effective, modern, finite-element MHD models.
    \end{abstract}
    
    
    \newpage
    \tableofcontents
    
    
    \newpage
    \pagenumbering{arabic}
    %\linenumbers\renewcommand\thelinenumber{\color{black!50}\arabic{linenumber}}
            \input{0 - introduction/main.tex}
        \part{Research}
            \input{1 - low-noise PiC models/main.tex}
            \input{2 - kinetic component/main.tex}
            \input{3 - fluid component/main.tex}
            \input{4 - numerical implementation/main.tex}
        \part{Project Overview}
            \input{5 - research plan/main.tex}
            \input{6 - summary/main.tex}
    
    
    %\section{}
    \newpage
    \pagenumbering{gobble}
        \printbibliography


    \newpage
    \pagenumbering{roman}
    \appendix
        \part{Appendices}
            \input{8 - Hilbert complexes/main.tex}
            \input{9 - weak conservation proofs/main.tex}
\end{document}

        \part{Project Overview}
            \documentclass[12pt, a4paper]{report}

\input{template/main.tex}

\title{\BA{Title in Progress...}}
\author{Boris Andrews}
\affil{Mathematical Institute, University of Oxford}
\date{\today}


\begin{document}
    \pagenumbering{gobble}
    \maketitle
    
    
    \begin{abstract}
        Magnetic confinement reactors---in particular tokamaks---offer one of the most promising options for achieving practical nuclear fusion, with the potential to provide virtually limitless, clean energy. The theoretical and numerical modeling of tokamak plasmas is simultaneously an essential component of effective reactor design, and a great research barrier. Tokamak operational conditions exhibit comparatively low Knudsen numbers. Kinetic effects, including kinetic waves and instabilities, Landau damping, bump-on-tail instabilities and more, are therefore highly influential in tokamak plasma dynamics. Purely fluid models are inherently incapable of capturing these effects, whereas the high dimensionality in purely kinetic models render them practically intractable for most relevant purposes.

        We consider a $\delta\!f$ decomposition model, with a macroscopic fluid background and microscopic kinetic correction, both fully coupled to each other. A similar manner of discretization is proposed to that used in the recent \texttt{STRUPHY} code \cite{Holderied_Possanner_Wang_2021, Holderied_2022, Li_et_al_2023} with a finite-element model for the background and a pseudo-particle/PiC model for the correction.

        The fluid background satisfies the full, non-linear, resistive, compressible, Hall MHD equations. \cite{Laakmann_Hu_Farrell_2022} introduces finite-element(-in-space) implicit timesteppers for the incompressible analogue to this system with structure-preserving (SP) properties in the ideal case, alongside parameter-robust preconditioners. We show that these timesteppers can derive from a finite-element-in-time (FET) (and finite-element-in-space) interpretation. The benefits of this reformulation are discussed, including the derivation of timesteppers that are higher order in time, and the quantifiable dissipative SP properties in the non-ideal, resistive case.
        
        We discuss possible options for extending this FET approach to timesteppers for the compressible case.

        The kinetic corrections satisfy linearized Boltzmann equations. Using a Lénard--Bernstein collision operator, these take Fokker--Planck-like forms \cite{Fokker_1914, Planck_1917} wherein pseudo-particles in the numerical model obey the neoclassical transport equations, with particle-independent Brownian drift terms. This offers a rigorous methodology for incorporating collisions into the particle transport model, without coupling the equations of motions for each particle.
        
        Works by Chen, Chacón et al. \cite{Chen_Chacón_Barnes_2011, Chacón_Chen_Barnes_2013, Chen_Chacón_2014, Chen_Chacón_2015} have developed structure-preserving particle pushers for neoclassical transport in the Vlasov equations, derived from Crank--Nicolson integrators. We show these too can can derive from a FET interpretation, similarly offering potential extensions to higher-order-in-time particle pushers. The FET formulation is used also to consider how the stochastic drift terms can be incorporated into the pushers. Stochastic gyrokinetic expansions are also discussed.

        Different options for the numerical implementation of these schemes are considered.

        Due to the efficacy of FET in the development of SP timesteppers for both the fluid and kinetic component, we hope this approach will prove effective in the future for developing SP timesteppers for the full hybrid model. We hope this will give us the opportunity to incorporate previously inaccessible kinetic effects into the highly effective, modern, finite-element MHD models.
    \end{abstract}
    
    
    \newpage
    \tableofcontents
    
    
    \newpage
    \pagenumbering{arabic}
    %\linenumbers\renewcommand\thelinenumber{\color{black!50}\arabic{linenumber}}
            \input{0 - introduction/main.tex}
        \part{Research}
            \input{1 - low-noise PiC models/main.tex}
            \input{2 - kinetic component/main.tex}
            \input{3 - fluid component/main.tex}
            \input{4 - numerical implementation/main.tex}
        \part{Project Overview}
            \input{5 - research plan/main.tex}
            \input{6 - summary/main.tex}
    
    
    %\section{}
    \newpage
    \pagenumbering{gobble}
        \printbibliography


    \newpage
    \pagenumbering{roman}
    \appendix
        \part{Appendices}
            \input{8 - Hilbert complexes/main.tex}
            \input{9 - weak conservation proofs/main.tex}
\end{document}

            \documentclass[12pt, a4paper]{report}

\input{template/main.tex}

\title{\BA{Title in Progress...}}
\author{Boris Andrews}
\affil{Mathematical Institute, University of Oxford}
\date{\today}


\begin{document}
    \pagenumbering{gobble}
    \maketitle
    
    
    \begin{abstract}
        Magnetic confinement reactors---in particular tokamaks---offer one of the most promising options for achieving practical nuclear fusion, with the potential to provide virtually limitless, clean energy. The theoretical and numerical modeling of tokamak plasmas is simultaneously an essential component of effective reactor design, and a great research barrier. Tokamak operational conditions exhibit comparatively low Knudsen numbers. Kinetic effects, including kinetic waves and instabilities, Landau damping, bump-on-tail instabilities and more, are therefore highly influential in tokamak plasma dynamics. Purely fluid models are inherently incapable of capturing these effects, whereas the high dimensionality in purely kinetic models render them practically intractable for most relevant purposes.

        We consider a $\delta\!f$ decomposition model, with a macroscopic fluid background and microscopic kinetic correction, both fully coupled to each other. A similar manner of discretization is proposed to that used in the recent \texttt{STRUPHY} code \cite{Holderied_Possanner_Wang_2021, Holderied_2022, Li_et_al_2023} with a finite-element model for the background and a pseudo-particle/PiC model for the correction.

        The fluid background satisfies the full, non-linear, resistive, compressible, Hall MHD equations. \cite{Laakmann_Hu_Farrell_2022} introduces finite-element(-in-space) implicit timesteppers for the incompressible analogue to this system with structure-preserving (SP) properties in the ideal case, alongside parameter-robust preconditioners. We show that these timesteppers can derive from a finite-element-in-time (FET) (and finite-element-in-space) interpretation. The benefits of this reformulation are discussed, including the derivation of timesteppers that are higher order in time, and the quantifiable dissipative SP properties in the non-ideal, resistive case.
        
        We discuss possible options for extending this FET approach to timesteppers for the compressible case.

        The kinetic corrections satisfy linearized Boltzmann equations. Using a Lénard--Bernstein collision operator, these take Fokker--Planck-like forms \cite{Fokker_1914, Planck_1917} wherein pseudo-particles in the numerical model obey the neoclassical transport equations, with particle-independent Brownian drift terms. This offers a rigorous methodology for incorporating collisions into the particle transport model, without coupling the equations of motions for each particle.
        
        Works by Chen, Chacón et al. \cite{Chen_Chacón_Barnes_2011, Chacón_Chen_Barnes_2013, Chen_Chacón_2014, Chen_Chacón_2015} have developed structure-preserving particle pushers for neoclassical transport in the Vlasov equations, derived from Crank--Nicolson integrators. We show these too can can derive from a FET interpretation, similarly offering potential extensions to higher-order-in-time particle pushers. The FET formulation is used also to consider how the stochastic drift terms can be incorporated into the pushers. Stochastic gyrokinetic expansions are also discussed.

        Different options for the numerical implementation of these schemes are considered.

        Due to the efficacy of FET in the development of SP timesteppers for both the fluid and kinetic component, we hope this approach will prove effective in the future for developing SP timesteppers for the full hybrid model. We hope this will give us the opportunity to incorporate previously inaccessible kinetic effects into the highly effective, modern, finite-element MHD models.
    \end{abstract}
    
    
    \newpage
    \tableofcontents
    
    
    \newpage
    \pagenumbering{arabic}
    %\linenumbers\renewcommand\thelinenumber{\color{black!50}\arabic{linenumber}}
            \input{0 - introduction/main.tex}
        \part{Research}
            \input{1 - low-noise PiC models/main.tex}
            \input{2 - kinetic component/main.tex}
            \input{3 - fluid component/main.tex}
            \input{4 - numerical implementation/main.tex}
        \part{Project Overview}
            \input{5 - research plan/main.tex}
            \input{6 - summary/main.tex}
    
    
    %\section{}
    \newpage
    \pagenumbering{gobble}
        \printbibliography


    \newpage
    \pagenumbering{roman}
    \appendix
        \part{Appendices}
            \input{8 - Hilbert complexes/main.tex}
            \input{9 - weak conservation proofs/main.tex}
\end{document}

    
    
    %\section{}
    \newpage
    \pagenumbering{gobble}
        \printbibliography


    \newpage
    \pagenumbering{roman}
    \appendix
        \part{Appendices}
            \documentclass[12pt, a4paper]{report}

\input{template/main.tex}

\title{\BA{Title in Progress...}}
\author{Boris Andrews}
\affil{Mathematical Institute, University of Oxford}
\date{\today}


\begin{document}
    \pagenumbering{gobble}
    \maketitle
    
    
    \begin{abstract}
        Magnetic confinement reactors---in particular tokamaks---offer one of the most promising options for achieving practical nuclear fusion, with the potential to provide virtually limitless, clean energy. The theoretical and numerical modeling of tokamak plasmas is simultaneously an essential component of effective reactor design, and a great research barrier. Tokamak operational conditions exhibit comparatively low Knudsen numbers. Kinetic effects, including kinetic waves and instabilities, Landau damping, bump-on-tail instabilities and more, are therefore highly influential in tokamak plasma dynamics. Purely fluid models are inherently incapable of capturing these effects, whereas the high dimensionality in purely kinetic models render them practically intractable for most relevant purposes.

        We consider a $\delta\!f$ decomposition model, with a macroscopic fluid background and microscopic kinetic correction, both fully coupled to each other. A similar manner of discretization is proposed to that used in the recent \texttt{STRUPHY} code \cite{Holderied_Possanner_Wang_2021, Holderied_2022, Li_et_al_2023} with a finite-element model for the background and a pseudo-particle/PiC model for the correction.

        The fluid background satisfies the full, non-linear, resistive, compressible, Hall MHD equations. \cite{Laakmann_Hu_Farrell_2022} introduces finite-element(-in-space) implicit timesteppers for the incompressible analogue to this system with structure-preserving (SP) properties in the ideal case, alongside parameter-robust preconditioners. We show that these timesteppers can derive from a finite-element-in-time (FET) (and finite-element-in-space) interpretation. The benefits of this reformulation are discussed, including the derivation of timesteppers that are higher order in time, and the quantifiable dissipative SP properties in the non-ideal, resistive case.
        
        We discuss possible options for extending this FET approach to timesteppers for the compressible case.

        The kinetic corrections satisfy linearized Boltzmann equations. Using a Lénard--Bernstein collision operator, these take Fokker--Planck-like forms \cite{Fokker_1914, Planck_1917} wherein pseudo-particles in the numerical model obey the neoclassical transport equations, with particle-independent Brownian drift terms. This offers a rigorous methodology for incorporating collisions into the particle transport model, without coupling the equations of motions for each particle.
        
        Works by Chen, Chacón et al. \cite{Chen_Chacón_Barnes_2011, Chacón_Chen_Barnes_2013, Chen_Chacón_2014, Chen_Chacón_2015} have developed structure-preserving particle pushers for neoclassical transport in the Vlasov equations, derived from Crank--Nicolson integrators. We show these too can can derive from a FET interpretation, similarly offering potential extensions to higher-order-in-time particle pushers. The FET formulation is used also to consider how the stochastic drift terms can be incorporated into the pushers. Stochastic gyrokinetic expansions are also discussed.

        Different options for the numerical implementation of these schemes are considered.

        Due to the efficacy of FET in the development of SP timesteppers for both the fluid and kinetic component, we hope this approach will prove effective in the future for developing SP timesteppers for the full hybrid model. We hope this will give us the opportunity to incorporate previously inaccessible kinetic effects into the highly effective, modern, finite-element MHD models.
    \end{abstract}
    
    
    \newpage
    \tableofcontents
    
    
    \newpage
    \pagenumbering{arabic}
    %\linenumbers\renewcommand\thelinenumber{\color{black!50}\arabic{linenumber}}
            \input{0 - introduction/main.tex}
        \part{Research}
            \input{1 - low-noise PiC models/main.tex}
            \input{2 - kinetic component/main.tex}
            \input{3 - fluid component/main.tex}
            \input{4 - numerical implementation/main.tex}
        \part{Project Overview}
            \input{5 - research plan/main.tex}
            \input{6 - summary/main.tex}
    
    
    %\section{}
    \newpage
    \pagenumbering{gobble}
        \printbibliography


    \newpage
    \pagenumbering{roman}
    \appendix
        \part{Appendices}
            \input{8 - Hilbert complexes/main.tex}
            \input{9 - weak conservation proofs/main.tex}
\end{document}

            \documentclass[12pt, a4paper]{report}

\input{template/main.tex}

\title{\BA{Title in Progress...}}
\author{Boris Andrews}
\affil{Mathematical Institute, University of Oxford}
\date{\today}


\begin{document}
    \pagenumbering{gobble}
    \maketitle
    
    
    \begin{abstract}
        Magnetic confinement reactors---in particular tokamaks---offer one of the most promising options for achieving practical nuclear fusion, with the potential to provide virtually limitless, clean energy. The theoretical and numerical modeling of tokamak plasmas is simultaneously an essential component of effective reactor design, and a great research barrier. Tokamak operational conditions exhibit comparatively low Knudsen numbers. Kinetic effects, including kinetic waves and instabilities, Landau damping, bump-on-tail instabilities and more, are therefore highly influential in tokamak plasma dynamics. Purely fluid models are inherently incapable of capturing these effects, whereas the high dimensionality in purely kinetic models render them practically intractable for most relevant purposes.

        We consider a $\delta\!f$ decomposition model, with a macroscopic fluid background and microscopic kinetic correction, both fully coupled to each other. A similar manner of discretization is proposed to that used in the recent \texttt{STRUPHY} code \cite{Holderied_Possanner_Wang_2021, Holderied_2022, Li_et_al_2023} with a finite-element model for the background and a pseudo-particle/PiC model for the correction.

        The fluid background satisfies the full, non-linear, resistive, compressible, Hall MHD equations. \cite{Laakmann_Hu_Farrell_2022} introduces finite-element(-in-space) implicit timesteppers for the incompressible analogue to this system with structure-preserving (SP) properties in the ideal case, alongside parameter-robust preconditioners. We show that these timesteppers can derive from a finite-element-in-time (FET) (and finite-element-in-space) interpretation. The benefits of this reformulation are discussed, including the derivation of timesteppers that are higher order in time, and the quantifiable dissipative SP properties in the non-ideal, resistive case.
        
        We discuss possible options for extending this FET approach to timesteppers for the compressible case.

        The kinetic corrections satisfy linearized Boltzmann equations. Using a Lénard--Bernstein collision operator, these take Fokker--Planck-like forms \cite{Fokker_1914, Planck_1917} wherein pseudo-particles in the numerical model obey the neoclassical transport equations, with particle-independent Brownian drift terms. This offers a rigorous methodology for incorporating collisions into the particle transport model, without coupling the equations of motions for each particle.
        
        Works by Chen, Chacón et al. \cite{Chen_Chacón_Barnes_2011, Chacón_Chen_Barnes_2013, Chen_Chacón_2014, Chen_Chacón_2015} have developed structure-preserving particle pushers for neoclassical transport in the Vlasov equations, derived from Crank--Nicolson integrators. We show these too can can derive from a FET interpretation, similarly offering potential extensions to higher-order-in-time particle pushers. The FET formulation is used also to consider how the stochastic drift terms can be incorporated into the pushers. Stochastic gyrokinetic expansions are also discussed.

        Different options for the numerical implementation of these schemes are considered.

        Due to the efficacy of FET in the development of SP timesteppers for both the fluid and kinetic component, we hope this approach will prove effective in the future for developing SP timesteppers for the full hybrid model. We hope this will give us the opportunity to incorporate previously inaccessible kinetic effects into the highly effective, modern, finite-element MHD models.
    \end{abstract}
    
    
    \newpage
    \tableofcontents
    
    
    \newpage
    \pagenumbering{arabic}
    %\linenumbers\renewcommand\thelinenumber{\color{black!50}\arabic{linenumber}}
            \input{0 - introduction/main.tex}
        \part{Research}
            \input{1 - low-noise PiC models/main.tex}
            \input{2 - kinetic component/main.tex}
            \input{3 - fluid component/main.tex}
            \input{4 - numerical implementation/main.tex}
        \part{Project Overview}
            \input{5 - research plan/main.tex}
            \input{6 - summary/main.tex}
    
    
    %\section{}
    \newpage
    \pagenumbering{gobble}
        \printbibliography


    \newpage
    \pagenumbering{roman}
    \appendix
        \part{Appendices}
            \input{8 - Hilbert complexes/main.tex}
            \input{9 - weak conservation proofs/main.tex}
\end{document}

\end{document}

            \documentclass[12pt, a4paper]{report}

\documentclass[12pt, a4paper]{report}

\input{template/main.tex}

\title{\BA{Title in Progress...}}
\author{Boris Andrews}
\affil{Mathematical Institute, University of Oxford}
\date{\today}


\begin{document}
    \pagenumbering{gobble}
    \maketitle
    
    
    \begin{abstract}
        Magnetic confinement reactors---in particular tokamaks---offer one of the most promising options for achieving practical nuclear fusion, with the potential to provide virtually limitless, clean energy. The theoretical and numerical modeling of tokamak plasmas is simultaneously an essential component of effective reactor design, and a great research barrier. Tokamak operational conditions exhibit comparatively low Knudsen numbers. Kinetic effects, including kinetic waves and instabilities, Landau damping, bump-on-tail instabilities and more, are therefore highly influential in tokamak plasma dynamics. Purely fluid models are inherently incapable of capturing these effects, whereas the high dimensionality in purely kinetic models render them practically intractable for most relevant purposes.

        We consider a $\delta\!f$ decomposition model, with a macroscopic fluid background and microscopic kinetic correction, both fully coupled to each other. A similar manner of discretization is proposed to that used in the recent \texttt{STRUPHY} code \cite{Holderied_Possanner_Wang_2021, Holderied_2022, Li_et_al_2023} with a finite-element model for the background and a pseudo-particle/PiC model for the correction.

        The fluid background satisfies the full, non-linear, resistive, compressible, Hall MHD equations. \cite{Laakmann_Hu_Farrell_2022} introduces finite-element(-in-space) implicit timesteppers for the incompressible analogue to this system with structure-preserving (SP) properties in the ideal case, alongside parameter-robust preconditioners. We show that these timesteppers can derive from a finite-element-in-time (FET) (and finite-element-in-space) interpretation. The benefits of this reformulation are discussed, including the derivation of timesteppers that are higher order in time, and the quantifiable dissipative SP properties in the non-ideal, resistive case.
        
        We discuss possible options for extending this FET approach to timesteppers for the compressible case.

        The kinetic corrections satisfy linearized Boltzmann equations. Using a Lénard--Bernstein collision operator, these take Fokker--Planck-like forms \cite{Fokker_1914, Planck_1917} wherein pseudo-particles in the numerical model obey the neoclassical transport equations, with particle-independent Brownian drift terms. This offers a rigorous methodology for incorporating collisions into the particle transport model, without coupling the equations of motions for each particle.
        
        Works by Chen, Chacón et al. \cite{Chen_Chacón_Barnes_2011, Chacón_Chen_Barnes_2013, Chen_Chacón_2014, Chen_Chacón_2015} have developed structure-preserving particle pushers for neoclassical transport in the Vlasov equations, derived from Crank--Nicolson integrators. We show these too can can derive from a FET interpretation, similarly offering potential extensions to higher-order-in-time particle pushers. The FET formulation is used also to consider how the stochastic drift terms can be incorporated into the pushers. Stochastic gyrokinetic expansions are also discussed.

        Different options for the numerical implementation of these schemes are considered.

        Due to the efficacy of FET in the development of SP timesteppers for both the fluid and kinetic component, we hope this approach will prove effective in the future for developing SP timesteppers for the full hybrid model. We hope this will give us the opportunity to incorporate previously inaccessible kinetic effects into the highly effective, modern, finite-element MHD models.
    \end{abstract}
    
    
    \newpage
    \tableofcontents
    
    
    \newpage
    \pagenumbering{arabic}
    %\linenumbers\renewcommand\thelinenumber{\color{black!50}\arabic{linenumber}}
            \input{0 - introduction/main.tex}
        \part{Research}
            \input{1 - low-noise PiC models/main.tex}
            \input{2 - kinetic component/main.tex}
            \input{3 - fluid component/main.tex}
            \input{4 - numerical implementation/main.tex}
        \part{Project Overview}
            \input{5 - research plan/main.tex}
            \input{6 - summary/main.tex}
    
    
    %\section{}
    \newpage
    \pagenumbering{gobble}
        \printbibliography


    \newpage
    \pagenumbering{roman}
    \appendix
        \part{Appendices}
            \input{8 - Hilbert complexes/main.tex}
            \input{9 - weak conservation proofs/main.tex}
\end{document}


\title{\BA{Title in Progress...}}
\author{Boris Andrews}
\affil{Mathematical Institute, University of Oxford}
\date{\today}


\begin{document}
    \pagenumbering{gobble}
    \maketitle
    
    
    \begin{abstract}
        Magnetic confinement reactors---in particular tokamaks---offer one of the most promising options for achieving practical nuclear fusion, with the potential to provide virtually limitless, clean energy. The theoretical and numerical modeling of tokamak plasmas is simultaneously an essential component of effective reactor design, and a great research barrier. Tokamak operational conditions exhibit comparatively low Knudsen numbers. Kinetic effects, including kinetic waves and instabilities, Landau damping, bump-on-tail instabilities and more, are therefore highly influential in tokamak plasma dynamics. Purely fluid models are inherently incapable of capturing these effects, whereas the high dimensionality in purely kinetic models render them practically intractable for most relevant purposes.

        We consider a $\delta\!f$ decomposition model, with a macroscopic fluid background and microscopic kinetic correction, both fully coupled to each other. A similar manner of discretization is proposed to that used in the recent \texttt{STRUPHY} code \cite{Holderied_Possanner_Wang_2021, Holderied_2022, Li_et_al_2023} with a finite-element model for the background and a pseudo-particle/PiC model for the correction.

        The fluid background satisfies the full, non-linear, resistive, compressible, Hall MHD equations. \cite{Laakmann_Hu_Farrell_2022} introduces finite-element(-in-space) implicit timesteppers for the incompressible analogue to this system with structure-preserving (SP) properties in the ideal case, alongside parameter-robust preconditioners. We show that these timesteppers can derive from a finite-element-in-time (FET) (and finite-element-in-space) interpretation. The benefits of this reformulation are discussed, including the derivation of timesteppers that are higher order in time, and the quantifiable dissipative SP properties in the non-ideal, resistive case.
        
        We discuss possible options for extending this FET approach to timesteppers for the compressible case.

        The kinetic corrections satisfy linearized Boltzmann equations. Using a Lénard--Bernstein collision operator, these take Fokker--Planck-like forms \cite{Fokker_1914, Planck_1917} wherein pseudo-particles in the numerical model obey the neoclassical transport equations, with particle-independent Brownian drift terms. This offers a rigorous methodology for incorporating collisions into the particle transport model, without coupling the equations of motions for each particle.
        
        Works by Chen, Chacón et al. \cite{Chen_Chacón_Barnes_2011, Chacón_Chen_Barnes_2013, Chen_Chacón_2014, Chen_Chacón_2015} have developed structure-preserving particle pushers for neoclassical transport in the Vlasov equations, derived from Crank--Nicolson integrators. We show these too can can derive from a FET interpretation, similarly offering potential extensions to higher-order-in-time particle pushers. The FET formulation is used also to consider how the stochastic drift terms can be incorporated into the pushers. Stochastic gyrokinetic expansions are also discussed.

        Different options for the numerical implementation of these schemes are considered.

        Due to the efficacy of FET in the development of SP timesteppers for both the fluid and kinetic component, we hope this approach will prove effective in the future for developing SP timesteppers for the full hybrid model. We hope this will give us the opportunity to incorporate previously inaccessible kinetic effects into the highly effective, modern, finite-element MHD models.
    \end{abstract}
    
    
    \newpage
    \tableofcontents
    
    
    \newpage
    \pagenumbering{arabic}
    %\linenumbers\renewcommand\thelinenumber{\color{black!50}\arabic{linenumber}}
            \documentclass[12pt, a4paper]{report}

\input{template/main.tex}

\title{\BA{Title in Progress...}}
\author{Boris Andrews}
\affil{Mathematical Institute, University of Oxford}
\date{\today}


\begin{document}
    \pagenumbering{gobble}
    \maketitle
    
    
    \begin{abstract}
        Magnetic confinement reactors---in particular tokamaks---offer one of the most promising options for achieving practical nuclear fusion, with the potential to provide virtually limitless, clean energy. The theoretical and numerical modeling of tokamak plasmas is simultaneously an essential component of effective reactor design, and a great research barrier. Tokamak operational conditions exhibit comparatively low Knudsen numbers. Kinetic effects, including kinetic waves and instabilities, Landau damping, bump-on-tail instabilities and more, are therefore highly influential in tokamak plasma dynamics. Purely fluid models are inherently incapable of capturing these effects, whereas the high dimensionality in purely kinetic models render them practically intractable for most relevant purposes.

        We consider a $\delta\!f$ decomposition model, with a macroscopic fluid background and microscopic kinetic correction, both fully coupled to each other. A similar manner of discretization is proposed to that used in the recent \texttt{STRUPHY} code \cite{Holderied_Possanner_Wang_2021, Holderied_2022, Li_et_al_2023} with a finite-element model for the background and a pseudo-particle/PiC model for the correction.

        The fluid background satisfies the full, non-linear, resistive, compressible, Hall MHD equations. \cite{Laakmann_Hu_Farrell_2022} introduces finite-element(-in-space) implicit timesteppers for the incompressible analogue to this system with structure-preserving (SP) properties in the ideal case, alongside parameter-robust preconditioners. We show that these timesteppers can derive from a finite-element-in-time (FET) (and finite-element-in-space) interpretation. The benefits of this reformulation are discussed, including the derivation of timesteppers that are higher order in time, and the quantifiable dissipative SP properties in the non-ideal, resistive case.
        
        We discuss possible options for extending this FET approach to timesteppers for the compressible case.

        The kinetic corrections satisfy linearized Boltzmann equations. Using a Lénard--Bernstein collision operator, these take Fokker--Planck-like forms \cite{Fokker_1914, Planck_1917} wherein pseudo-particles in the numerical model obey the neoclassical transport equations, with particle-independent Brownian drift terms. This offers a rigorous methodology for incorporating collisions into the particle transport model, without coupling the equations of motions for each particle.
        
        Works by Chen, Chacón et al. \cite{Chen_Chacón_Barnes_2011, Chacón_Chen_Barnes_2013, Chen_Chacón_2014, Chen_Chacón_2015} have developed structure-preserving particle pushers for neoclassical transport in the Vlasov equations, derived from Crank--Nicolson integrators. We show these too can can derive from a FET interpretation, similarly offering potential extensions to higher-order-in-time particle pushers. The FET formulation is used also to consider how the stochastic drift terms can be incorporated into the pushers. Stochastic gyrokinetic expansions are also discussed.

        Different options for the numerical implementation of these schemes are considered.

        Due to the efficacy of FET in the development of SP timesteppers for both the fluid and kinetic component, we hope this approach will prove effective in the future for developing SP timesteppers for the full hybrid model. We hope this will give us the opportunity to incorporate previously inaccessible kinetic effects into the highly effective, modern, finite-element MHD models.
    \end{abstract}
    
    
    \newpage
    \tableofcontents
    
    
    \newpage
    \pagenumbering{arabic}
    %\linenumbers\renewcommand\thelinenumber{\color{black!50}\arabic{linenumber}}
            \input{0 - introduction/main.tex}
        \part{Research}
            \input{1 - low-noise PiC models/main.tex}
            \input{2 - kinetic component/main.tex}
            \input{3 - fluid component/main.tex}
            \input{4 - numerical implementation/main.tex}
        \part{Project Overview}
            \input{5 - research plan/main.tex}
            \input{6 - summary/main.tex}
    
    
    %\section{}
    \newpage
    \pagenumbering{gobble}
        \printbibliography


    \newpage
    \pagenumbering{roman}
    \appendix
        \part{Appendices}
            \input{8 - Hilbert complexes/main.tex}
            \input{9 - weak conservation proofs/main.tex}
\end{document}

        \part{Research}
            \documentclass[12pt, a4paper]{report}

\input{template/main.tex}

\title{\BA{Title in Progress...}}
\author{Boris Andrews}
\affil{Mathematical Institute, University of Oxford}
\date{\today}


\begin{document}
    \pagenumbering{gobble}
    \maketitle
    
    
    \begin{abstract}
        Magnetic confinement reactors---in particular tokamaks---offer one of the most promising options for achieving practical nuclear fusion, with the potential to provide virtually limitless, clean energy. The theoretical and numerical modeling of tokamak plasmas is simultaneously an essential component of effective reactor design, and a great research barrier. Tokamak operational conditions exhibit comparatively low Knudsen numbers. Kinetic effects, including kinetic waves and instabilities, Landau damping, bump-on-tail instabilities and more, are therefore highly influential in tokamak plasma dynamics. Purely fluid models are inherently incapable of capturing these effects, whereas the high dimensionality in purely kinetic models render them practically intractable for most relevant purposes.

        We consider a $\delta\!f$ decomposition model, with a macroscopic fluid background and microscopic kinetic correction, both fully coupled to each other. A similar manner of discretization is proposed to that used in the recent \texttt{STRUPHY} code \cite{Holderied_Possanner_Wang_2021, Holderied_2022, Li_et_al_2023} with a finite-element model for the background and a pseudo-particle/PiC model for the correction.

        The fluid background satisfies the full, non-linear, resistive, compressible, Hall MHD equations. \cite{Laakmann_Hu_Farrell_2022} introduces finite-element(-in-space) implicit timesteppers for the incompressible analogue to this system with structure-preserving (SP) properties in the ideal case, alongside parameter-robust preconditioners. We show that these timesteppers can derive from a finite-element-in-time (FET) (and finite-element-in-space) interpretation. The benefits of this reformulation are discussed, including the derivation of timesteppers that are higher order in time, and the quantifiable dissipative SP properties in the non-ideal, resistive case.
        
        We discuss possible options for extending this FET approach to timesteppers for the compressible case.

        The kinetic corrections satisfy linearized Boltzmann equations. Using a Lénard--Bernstein collision operator, these take Fokker--Planck-like forms \cite{Fokker_1914, Planck_1917} wherein pseudo-particles in the numerical model obey the neoclassical transport equations, with particle-independent Brownian drift terms. This offers a rigorous methodology for incorporating collisions into the particle transport model, without coupling the equations of motions for each particle.
        
        Works by Chen, Chacón et al. \cite{Chen_Chacón_Barnes_2011, Chacón_Chen_Barnes_2013, Chen_Chacón_2014, Chen_Chacón_2015} have developed structure-preserving particle pushers for neoclassical transport in the Vlasov equations, derived from Crank--Nicolson integrators. We show these too can can derive from a FET interpretation, similarly offering potential extensions to higher-order-in-time particle pushers. The FET formulation is used also to consider how the stochastic drift terms can be incorporated into the pushers. Stochastic gyrokinetic expansions are also discussed.

        Different options for the numerical implementation of these schemes are considered.

        Due to the efficacy of FET in the development of SP timesteppers for both the fluid and kinetic component, we hope this approach will prove effective in the future for developing SP timesteppers for the full hybrid model. We hope this will give us the opportunity to incorporate previously inaccessible kinetic effects into the highly effective, modern, finite-element MHD models.
    \end{abstract}
    
    
    \newpage
    \tableofcontents
    
    
    \newpage
    \pagenumbering{arabic}
    %\linenumbers\renewcommand\thelinenumber{\color{black!50}\arabic{linenumber}}
            \input{0 - introduction/main.tex}
        \part{Research}
            \input{1 - low-noise PiC models/main.tex}
            \input{2 - kinetic component/main.tex}
            \input{3 - fluid component/main.tex}
            \input{4 - numerical implementation/main.tex}
        \part{Project Overview}
            \input{5 - research plan/main.tex}
            \input{6 - summary/main.tex}
    
    
    %\section{}
    \newpage
    \pagenumbering{gobble}
        \printbibliography


    \newpage
    \pagenumbering{roman}
    \appendix
        \part{Appendices}
            \input{8 - Hilbert complexes/main.tex}
            \input{9 - weak conservation proofs/main.tex}
\end{document}

            \documentclass[12pt, a4paper]{report}

\input{template/main.tex}

\title{\BA{Title in Progress...}}
\author{Boris Andrews}
\affil{Mathematical Institute, University of Oxford}
\date{\today}


\begin{document}
    \pagenumbering{gobble}
    \maketitle
    
    
    \begin{abstract}
        Magnetic confinement reactors---in particular tokamaks---offer one of the most promising options for achieving practical nuclear fusion, with the potential to provide virtually limitless, clean energy. The theoretical and numerical modeling of tokamak plasmas is simultaneously an essential component of effective reactor design, and a great research barrier. Tokamak operational conditions exhibit comparatively low Knudsen numbers. Kinetic effects, including kinetic waves and instabilities, Landau damping, bump-on-tail instabilities and more, are therefore highly influential in tokamak plasma dynamics. Purely fluid models are inherently incapable of capturing these effects, whereas the high dimensionality in purely kinetic models render them practically intractable for most relevant purposes.

        We consider a $\delta\!f$ decomposition model, with a macroscopic fluid background and microscopic kinetic correction, both fully coupled to each other. A similar manner of discretization is proposed to that used in the recent \texttt{STRUPHY} code \cite{Holderied_Possanner_Wang_2021, Holderied_2022, Li_et_al_2023} with a finite-element model for the background and a pseudo-particle/PiC model for the correction.

        The fluid background satisfies the full, non-linear, resistive, compressible, Hall MHD equations. \cite{Laakmann_Hu_Farrell_2022} introduces finite-element(-in-space) implicit timesteppers for the incompressible analogue to this system with structure-preserving (SP) properties in the ideal case, alongside parameter-robust preconditioners. We show that these timesteppers can derive from a finite-element-in-time (FET) (and finite-element-in-space) interpretation. The benefits of this reformulation are discussed, including the derivation of timesteppers that are higher order in time, and the quantifiable dissipative SP properties in the non-ideal, resistive case.
        
        We discuss possible options for extending this FET approach to timesteppers for the compressible case.

        The kinetic corrections satisfy linearized Boltzmann equations. Using a Lénard--Bernstein collision operator, these take Fokker--Planck-like forms \cite{Fokker_1914, Planck_1917} wherein pseudo-particles in the numerical model obey the neoclassical transport equations, with particle-independent Brownian drift terms. This offers a rigorous methodology for incorporating collisions into the particle transport model, without coupling the equations of motions for each particle.
        
        Works by Chen, Chacón et al. \cite{Chen_Chacón_Barnes_2011, Chacón_Chen_Barnes_2013, Chen_Chacón_2014, Chen_Chacón_2015} have developed structure-preserving particle pushers for neoclassical transport in the Vlasov equations, derived from Crank--Nicolson integrators. We show these too can can derive from a FET interpretation, similarly offering potential extensions to higher-order-in-time particle pushers. The FET formulation is used also to consider how the stochastic drift terms can be incorporated into the pushers. Stochastic gyrokinetic expansions are also discussed.

        Different options for the numerical implementation of these schemes are considered.

        Due to the efficacy of FET in the development of SP timesteppers for both the fluid and kinetic component, we hope this approach will prove effective in the future for developing SP timesteppers for the full hybrid model. We hope this will give us the opportunity to incorporate previously inaccessible kinetic effects into the highly effective, modern, finite-element MHD models.
    \end{abstract}
    
    
    \newpage
    \tableofcontents
    
    
    \newpage
    \pagenumbering{arabic}
    %\linenumbers\renewcommand\thelinenumber{\color{black!50}\arabic{linenumber}}
            \input{0 - introduction/main.tex}
        \part{Research}
            \input{1 - low-noise PiC models/main.tex}
            \input{2 - kinetic component/main.tex}
            \input{3 - fluid component/main.tex}
            \input{4 - numerical implementation/main.tex}
        \part{Project Overview}
            \input{5 - research plan/main.tex}
            \input{6 - summary/main.tex}
    
    
    %\section{}
    \newpage
    \pagenumbering{gobble}
        \printbibliography


    \newpage
    \pagenumbering{roman}
    \appendix
        \part{Appendices}
            \input{8 - Hilbert complexes/main.tex}
            \input{9 - weak conservation proofs/main.tex}
\end{document}

            \documentclass[12pt, a4paper]{report}

\input{template/main.tex}

\title{\BA{Title in Progress...}}
\author{Boris Andrews}
\affil{Mathematical Institute, University of Oxford}
\date{\today}


\begin{document}
    \pagenumbering{gobble}
    \maketitle
    
    
    \begin{abstract}
        Magnetic confinement reactors---in particular tokamaks---offer one of the most promising options for achieving practical nuclear fusion, with the potential to provide virtually limitless, clean energy. The theoretical and numerical modeling of tokamak plasmas is simultaneously an essential component of effective reactor design, and a great research barrier. Tokamak operational conditions exhibit comparatively low Knudsen numbers. Kinetic effects, including kinetic waves and instabilities, Landau damping, bump-on-tail instabilities and more, are therefore highly influential in tokamak plasma dynamics. Purely fluid models are inherently incapable of capturing these effects, whereas the high dimensionality in purely kinetic models render them practically intractable for most relevant purposes.

        We consider a $\delta\!f$ decomposition model, with a macroscopic fluid background and microscopic kinetic correction, both fully coupled to each other. A similar manner of discretization is proposed to that used in the recent \texttt{STRUPHY} code \cite{Holderied_Possanner_Wang_2021, Holderied_2022, Li_et_al_2023} with a finite-element model for the background and a pseudo-particle/PiC model for the correction.

        The fluid background satisfies the full, non-linear, resistive, compressible, Hall MHD equations. \cite{Laakmann_Hu_Farrell_2022} introduces finite-element(-in-space) implicit timesteppers for the incompressible analogue to this system with structure-preserving (SP) properties in the ideal case, alongside parameter-robust preconditioners. We show that these timesteppers can derive from a finite-element-in-time (FET) (and finite-element-in-space) interpretation. The benefits of this reformulation are discussed, including the derivation of timesteppers that are higher order in time, and the quantifiable dissipative SP properties in the non-ideal, resistive case.
        
        We discuss possible options for extending this FET approach to timesteppers for the compressible case.

        The kinetic corrections satisfy linearized Boltzmann equations. Using a Lénard--Bernstein collision operator, these take Fokker--Planck-like forms \cite{Fokker_1914, Planck_1917} wherein pseudo-particles in the numerical model obey the neoclassical transport equations, with particle-independent Brownian drift terms. This offers a rigorous methodology for incorporating collisions into the particle transport model, without coupling the equations of motions for each particle.
        
        Works by Chen, Chacón et al. \cite{Chen_Chacón_Barnes_2011, Chacón_Chen_Barnes_2013, Chen_Chacón_2014, Chen_Chacón_2015} have developed structure-preserving particle pushers for neoclassical transport in the Vlasov equations, derived from Crank--Nicolson integrators. We show these too can can derive from a FET interpretation, similarly offering potential extensions to higher-order-in-time particle pushers. The FET formulation is used also to consider how the stochastic drift terms can be incorporated into the pushers. Stochastic gyrokinetic expansions are also discussed.

        Different options for the numerical implementation of these schemes are considered.

        Due to the efficacy of FET in the development of SP timesteppers for both the fluid and kinetic component, we hope this approach will prove effective in the future for developing SP timesteppers for the full hybrid model. We hope this will give us the opportunity to incorporate previously inaccessible kinetic effects into the highly effective, modern, finite-element MHD models.
    \end{abstract}
    
    
    \newpage
    \tableofcontents
    
    
    \newpage
    \pagenumbering{arabic}
    %\linenumbers\renewcommand\thelinenumber{\color{black!50}\arabic{linenumber}}
            \input{0 - introduction/main.tex}
        \part{Research}
            \input{1 - low-noise PiC models/main.tex}
            \input{2 - kinetic component/main.tex}
            \input{3 - fluid component/main.tex}
            \input{4 - numerical implementation/main.tex}
        \part{Project Overview}
            \input{5 - research plan/main.tex}
            \input{6 - summary/main.tex}
    
    
    %\section{}
    \newpage
    \pagenumbering{gobble}
        \printbibliography


    \newpage
    \pagenumbering{roman}
    \appendix
        \part{Appendices}
            \input{8 - Hilbert complexes/main.tex}
            \input{9 - weak conservation proofs/main.tex}
\end{document}

            \documentclass[12pt, a4paper]{report}

\input{template/main.tex}

\title{\BA{Title in Progress...}}
\author{Boris Andrews}
\affil{Mathematical Institute, University of Oxford}
\date{\today}


\begin{document}
    \pagenumbering{gobble}
    \maketitle
    
    
    \begin{abstract}
        Magnetic confinement reactors---in particular tokamaks---offer one of the most promising options for achieving practical nuclear fusion, with the potential to provide virtually limitless, clean energy. The theoretical and numerical modeling of tokamak plasmas is simultaneously an essential component of effective reactor design, and a great research barrier. Tokamak operational conditions exhibit comparatively low Knudsen numbers. Kinetic effects, including kinetic waves and instabilities, Landau damping, bump-on-tail instabilities and more, are therefore highly influential in tokamak plasma dynamics. Purely fluid models are inherently incapable of capturing these effects, whereas the high dimensionality in purely kinetic models render them practically intractable for most relevant purposes.

        We consider a $\delta\!f$ decomposition model, with a macroscopic fluid background and microscopic kinetic correction, both fully coupled to each other. A similar manner of discretization is proposed to that used in the recent \texttt{STRUPHY} code \cite{Holderied_Possanner_Wang_2021, Holderied_2022, Li_et_al_2023} with a finite-element model for the background and a pseudo-particle/PiC model for the correction.

        The fluid background satisfies the full, non-linear, resistive, compressible, Hall MHD equations. \cite{Laakmann_Hu_Farrell_2022} introduces finite-element(-in-space) implicit timesteppers for the incompressible analogue to this system with structure-preserving (SP) properties in the ideal case, alongside parameter-robust preconditioners. We show that these timesteppers can derive from a finite-element-in-time (FET) (and finite-element-in-space) interpretation. The benefits of this reformulation are discussed, including the derivation of timesteppers that are higher order in time, and the quantifiable dissipative SP properties in the non-ideal, resistive case.
        
        We discuss possible options for extending this FET approach to timesteppers for the compressible case.

        The kinetic corrections satisfy linearized Boltzmann equations. Using a Lénard--Bernstein collision operator, these take Fokker--Planck-like forms \cite{Fokker_1914, Planck_1917} wherein pseudo-particles in the numerical model obey the neoclassical transport equations, with particle-independent Brownian drift terms. This offers a rigorous methodology for incorporating collisions into the particle transport model, without coupling the equations of motions for each particle.
        
        Works by Chen, Chacón et al. \cite{Chen_Chacón_Barnes_2011, Chacón_Chen_Barnes_2013, Chen_Chacón_2014, Chen_Chacón_2015} have developed structure-preserving particle pushers for neoclassical transport in the Vlasov equations, derived from Crank--Nicolson integrators. We show these too can can derive from a FET interpretation, similarly offering potential extensions to higher-order-in-time particle pushers. The FET formulation is used also to consider how the stochastic drift terms can be incorporated into the pushers. Stochastic gyrokinetic expansions are also discussed.

        Different options for the numerical implementation of these schemes are considered.

        Due to the efficacy of FET in the development of SP timesteppers for both the fluid and kinetic component, we hope this approach will prove effective in the future for developing SP timesteppers for the full hybrid model. We hope this will give us the opportunity to incorporate previously inaccessible kinetic effects into the highly effective, modern, finite-element MHD models.
    \end{abstract}
    
    
    \newpage
    \tableofcontents
    
    
    \newpage
    \pagenumbering{arabic}
    %\linenumbers\renewcommand\thelinenumber{\color{black!50}\arabic{linenumber}}
            \input{0 - introduction/main.tex}
        \part{Research}
            \input{1 - low-noise PiC models/main.tex}
            \input{2 - kinetic component/main.tex}
            \input{3 - fluid component/main.tex}
            \input{4 - numerical implementation/main.tex}
        \part{Project Overview}
            \input{5 - research plan/main.tex}
            \input{6 - summary/main.tex}
    
    
    %\section{}
    \newpage
    \pagenumbering{gobble}
        \printbibliography


    \newpage
    \pagenumbering{roman}
    \appendix
        \part{Appendices}
            \input{8 - Hilbert complexes/main.tex}
            \input{9 - weak conservation proofs/main.tex}
\end{document}

        \part{Project Overview}
            \documentclass[12pt, a4paper]{report}

\input{template/main.tex}

\title{\BA{Title in Progress...}}
\author{Boris Andrews}
\affil{Mathematical Institute, University of Oxford}
\date{\today}


\begin{document}
    \pagenumbering{gobble}
    \maketitle
    
    
    \begin{abstract}
        Magnetic confinement reactors---in particular tokamaks---offer one of the most promising options for achieving practical nuclear fusion, with the potential to provide virtually limitless, clean energy. The theoretical and numerical modeling of tokamak plasmas is simultaneously an essential component of effective reactor design, and a great research barrier. Tokamak operational conditions exhibit comparatively low Knudsen numbers. Kinetic effects, including kinetic waves and instabilities, Landau damping, bump-on-tail instabilities and more, are therefore highly influential in tokamak plasma dynamics. Purely fluid models are inherently incapable of capturing these effects, whereas the high dimensionality in purely kinetic models render them practically intractable for most relevant purposes.

        We consider a $\delta\!f$ decomposition model, with a macroscopic fluid background and microscopic kinetic correction, both fully coupled to each other. A similar manner of discretization is proposed to that used in the recent \texttt{STRUPHY} code \cite{Holderied_Possanner_Wang_2021, Holderied_2022, Li_et_al_2023} with a finite-element model for the background and a pseudo-particle/PiC model for the correction.

        The fluid background satisfies the full, non-linear, resistive, compressible, Hall MHD equations. \cite{Laakmann_Hu_Farrell_2022} introduces finite-element(-in-space) implicit timesteppers for the incompressible analogue to this system with structure-preserving (SP) properties in the ideal case, alongside parameter-robust preconditioners. We show that these timesteppers can derive from a finite-element-in-time (FET) (and finite-element-in-space) interpretation. The benefits of this reformulation are discussed, including the derivation of timesteppers that are higher order in time, and the quantifiable dissipative SP properties in the non-ideal, resistive case.
        
        We discuss possible options for extending this FET approach to timesteppers for the compressible case.

        The kinetic corrections satisfy linearized Boltzmann equations. Using a Lénard--Bernstein collision operator, these take Fokker--Planck-like forms \cite{Fokker_1914, Planck_1917} wherein pseudo-particles in the numerical model obey the neoclassical transport equations, with particle-independent Brownian drift terms. This offers a rigorous methodology for incorporating collisions into the particle transport model, without coupling the equations of motions for each particle.
        
        Works by Chen, Chacón et al. \cite{Chen_Chacón_Barnes_2011, Chacón_Chen_Barnes_2013, Chen_Chacón_2014, Chen_Chacón_2015} have developed structure-preserving particle pushers for neoclassical transport in the Vlasov equations, derived from Crank--Nicolson integrators. We show these too can can derive from a FET interpretation, similarly offering potential extensions to higher-order-in-time particle pushers. The FET formulation is used also to consider how the stochastic drift terms can be incorporated into the pushers. Stochastic gyrokinetic expansions are also discussed.

        Different options for the numerical implementation of these schemes are considered.

        Due to the efficacy of FET in the development of SP timesteppers for both the fluid and kinetic component, we hope this approach will prove effective in the future for developing SP timesteppers for the full hybrid model. We hope this will give us the opportunity to incorporate previously inaccessible kinetic effects into the highly effective, modern, finite-element MHD models.
    \end{abstract}
    
    
    \newpage
    \tableofcontents
    
    
    \newpage
    \pagenumbering{arabic}
    %\linenumbers\renewcommand\thelinenumber{\color{black!50}\arabic{linenumber}}
            \input{0 - introduction/main.tex}
        \part{Research}
            \input{1 - low-noise PiC models/main.tex}
            \input{2 - kinetic component/main.tex}
            \input{3 - fluid component/main.tex}
            \input{4 - numerical implementation/main.tex}
        \part{Project Overview}
            \input{5 - research plan/main.tex}
            \input{6 - summary/main.tex}
    
    
    %\section{}
    \newpage
    \pagenumbering{gobble}
        \printbibliography


    \newpage
    \pagenumbering{roman}
    \appendix
        \part{Appendices}
            \input{8 - Hilbert complexes/main.tex}
            \input{9 - weak conservation proofs/main.tex}
\end{document}

            \documentclass[12pt, a4paper]{report}

\input{template/main.tex}

\title{\BA{Title in Progress...}}
\author{Boris Andrews}
\affil{Mathematical Institute, University of Oxford}
\date{\today}


\begin{document}
    \pagenumbering{gobble}
    \maketitle
    
    
    \begin{abstract}
        Magnetic confinement reactors---in particular tokamaks---offer one of the most promising options for achieving practical nuclear fusion, with the potential to provide virtually limitless, clean energy. The theoretical and numerical modeling of tokamak plasmas is simultaneously an essential component of effective reactor design, and a great research barrier. Tokamak operational conditions exhibit comparatively low Knudsen numbers. Kinetic effects, including kinetic waves and instabilities, Landau damping, bump-on-tail instabilities and more, are therefore highly influential in tokamak plasma dynamics. Purely fluid models are inherently incapable of capturing these effects, whereas the high dimensionality in purely kinetic models render them practically intractable for most relevant purposes.

        We consider a $\delta\!f$ decomposition model, with a macroscopic fluid background and microscopic kinetic correction, both fully coupled to each other. A similar manner of discretization is proposed to that used in the recent \texttt{STRUPHY} code \cite{Holderied_Possanner_Wang_2021, Holderied_2022, Li_et_al_2023} with a finite-element model for the background and a pseudo-particle/PiC model for the correction.

        The fluid background satisfies the full, non-linear, resistive, compressible, Hall MHD equations. \cite{Laakmann_Hu_Farrell_2022} introduces finite-element(-in-space) implicit timesteppers for the incompressible analogue to this system with structure-preserving (SP) properties in the ideal case, alongside parameter-robust preconditioners. We show that these timesteppers can derive from a finite-element-in-time (FET) (and finite-element-in-space) interpretation. The benefits of this reformulation are discussed, including the derivation of timesteppers that are higher order in time, and the quantifiable dissipative SP properties in the non-ideal, resistive case.
        
        We discuss possible options for extending this FET approach to timesteppers for the compressible case.

        The kinetic corrections satisfy linearized Boltzmann equations. Using a Lénard--Bernstein collision operator, these take Fokker--Planck-like forms \cite{Fokker_1914, Planck_1917} wherein pseudo-particles in the numerical model obey the neoclassical transport equations, with particle-independent Brownian drift terms. This offers a rigorous methodology for incorporating collisions into the particle transport model, without coupling the equations of motions for each particle.
        
        Works by Chen, Chacón et al. \cite{Chen_Chacón_Barnes_2011, Chacón_Chen_Barnes_2013, Chen_Chacón_2014, Chen_Chacón_2015} have developed structure-preserving particle pushers for neoclassical transport in the Vlasov equations, derived from Crank--Nicolson integrators. We show these too can can derive from a FET interpretation, similarly offering potential extensions to higher-order-in-time particle pushers. The FET formulation is used also to consider how the stochastic drift terms can be incorporated into the pushers. Stochastic gyrokinetic expansions are also discussed.

        Different options for the numerical implementation of these schemes are considered.

        Due to the efficacy of FET in the development of SP timesteppers for both the fluid and kinetic component, we hope this approach will prove effective in the future for developing SP timesteppers for the full hybrid model. We hope this will give us the opportunity to incorporate previously inaccessible kinetic effects into the highly effective, modern, finite-element MHD models.
    \end{abstract}
    
    
    \newpage
    \tableofcontents
    
    
    \newpage
    \pagenumbering{arabic}
    %\linenumbers\renewcommand\thelinenumber{\color{black!50}\arabic{linenumber}}
            \input{0 - introduction/main.tex}
        \part{Research}
            \input{1 - low-noise PiC models/main.tex}
            \input{2 - kinetic component/main.tex}
            \input{3 - fluid component/main.tex}
            \input{4 - numerical implementation/main.tex}
        \part{Project Overview}
            \input{5 - research plan/main.tex}
            \input{6 - summary/main.tex}
    
    
    %\section{}
    \newpage
    \pagenumbering{gobble}
        \printbibliography


    \newpage
    \pagenumbering{roman}
    \appendix
        \part{Appendices}
            \input{8 - Hilbert complexes/main.tex}
            \input{9 - weak conservation proofs/main.tex}
\end{document}

    
    
    %\section{}
    \newpage
    \pagenumbering{gobble}
        \printbibliography


    \newpage
    \pagenumbering{roman}
    \appendix
        \part{Appendices}
            \documentclass[12pt, a4paper]{report}

\input{template/main.tex}

\title{\BA{Title in Progress...}}
\author{Boris Andrews}
\affil{Mathematical Institute, University of Oxford}
\date{\today}


\begin{document}
    \pagenumbering{gobble}
    \maketitle
    
    
    \begin{abstract}
        Magnetic confinement reactors---in particular tokamaks---offer one of the most promising options for achieving practical nuclear fusion, with the potential to provide virtually limitless, clean energy. The theoretical and numerical modeling of tokamak plasmas is simultaneously an essential component of effective reactor design, and a great research barrier. Tokamak operational conditions exhibit comparatively low Knudsen numbers. Kinetic effects, including kinetic waves and instabilities, Landau damping, bump-on-tail instabilities and more, are therefore highly influential in tokamak plasma dynamics. Purely fluid models are inherently incapable of capturing these effects, whereas the high dimensionality in purely kinetic models render them practically intractable for most relevant purposes.

        We consider a $\delta\!f$ decomposition model, with a macroscopic fluid background and microscopic kinetic correction, both fully coupled to each other. A similar manner of discretization is proposed to that used in the recent \texttt{STRUPHY} code \cite{Holderied_Possanner_Wang_2021, Holderied_2022, Li_et_al_2023} with a finite-element model for the background and a pseudo-particle/PiC model for the correction.

        The fluid background satisfies the full, non-linear, resistive, compressible, Hall MHD equations. \cite{Laakmann_Hu_Farrell_2022} introduces finite-element(-in-space) implicit timesteppers for the incompressible analogue to this system with structure-preserving (SP) properties in the ideal case, alongside parameter-robust preconditioners. We show that these timesteppers can derive from a finite-element-in-time (FET) (and finite-element-in-space) interpretation. The benefits of this reformulation are discussed, including the derivation of timesteppers that are higher order in time, and the quantifiable dissipative SP properties in the non-ideal, resistive case.
        
        We discuss possible options for extending this FET approach to timesteppers for the compressible case.

        The kinetic corrections satisfy linearized Boltzmann equations. Using a Lénard--Bernstein collision operator, these take Fokker--Planck-like forms \cite{Fokker_1914, Planck_1917} wherein pseudo-particles in the numerical model obey the neoclassical transport equations, with particle-independent Brownian drift terms. This offers a rigorous methodology for incorporating collisions into the particle transport model, without coupling the equations of motions for each particle.
        
        Works by Chen, Chacón et al. \cite{Chen_Chacón_Barnes_2011, Chacón_Chen_Barnes_2013, Chen_Chacón_2014, Chen_Chacón_2015} have developed structure-preserving particle pushers for neoclassical transport in the Vlasov equations, derived from Crank--Nicolson integrators. We show these too can can derive from a FET interpretation, similarly offering potential extensions to higher-order-in-time particle pushers. The FET formulation is used also to consider how the stochastic drift terms can be incorporated into the pushers. Stochastic gyrokinetic expansions are also discussed.

        Different options for the numerical implementation of these schemes are considered.

        Due to the efficacy of FET in the development of SP timesteppers for both the fluid and kinetic component, we hope this approach will prove effective in the future for developing SP timesteppers for the full hybrid model. We hope this will give us the opportunity to incorporate previously inaccessible kinetic effects into the highly effective, modern, finite-element MHD models.
    \end{abstract}
    
    
    \newpage
    \tableofcontents
    
    
    \newpage
    \pagenumbering{arabic}
    %\linenumbers\renewcommand\thelinenumber{\color{black!50}\arabic{linenumber}}
            \input{0 - introduction/main.tex}
        \part{Research}
            \input{1 - low-noise PiC models/main.tex}
            \input{2 - kinetic component/main.tex}
            \input{3 - fluid component/main.tex}
            \input{4 - numerical implementation/main.tex}
        \part{Project Overview}
            \input{5 - research plan/main.tex}
            \input{6 - summary/main.tex}
    
    
    %\section{}
    \newpage
    \pagenumbering{gobble}
        \printbibliography


    \newpage
    \pagenumbering{roman}
    \appendix
        \part{Appendices}
            \input{8 - Hilbert complexes/main.tex}
            \input{9 - weak conservation proofs/main.tex}
\end{document}

            \documentclass[12pt, a4paper]{report}

\input{template/main.tex}

\title{\BA{Title in Progress...}}
\author{Boris Andrews}
\affil{Mathematical Institute, University of Oxford}
\date{\today}


\begin{document}
    \pagenumbering{gobble}
    \maketitle
    
    
    \begin{abstract}
        Magnetic confinement reactors---in particular tokamaks---offer one of the most promising options for achieving practical nuclear fusion, with the potential to provide virtually limitless, clean energy. The theoretical and numerical modeling of tokamak plasmas is simultaneously an essential component of effective reactor design, and a great research barrier. Tokamak operational conditions exhibit comparatively low Knudsen numbers. Kinetic effects, including kinetic waves and instabilities, Landau damping, bump-on-tail instabilities and more, are therefore highly influential in tokamak plasma dynamics. Purely fluid models are inherently incapable of capturing these effects, whereas the high dimensionality in purely kinetic models render them practically intractable for most relevant purposes.

        We consider a $\delta\!f$ decomposition model, with a macroscopic fluid background and microscopic kinetic correction, both fully coupled to each other. A similar manner of discretization is proposed to that used in the recent \texttt{STRUPHY} code \cite{Holderied_Possanner_Wang_2021, Holderied_2022, Li_et_al_2023} with a finite-element model for the background and a pseudo-particle/PiC model for the correction.

        The fluid background satisfies the full, non-linear, resistive, compressible, Hall MHD equations. \cite{Laakmann_Hu_Farrell_2022} introduces finite-element(-in-space) implicit timesteppers for the incompressible analogue to this system with structure-preserving (SP) properties in the ideal case, alongside parameter-robust preconditioners. We show that these timesteppers can derive from a finite-element-in-time (FET) (and finite-element-in-space) interpretation. The benefits of this reformulation are discussed, including the derivation of timesteppers that are higher order in time, and the quantifiable dissipative SP properties in the non-ideal, resistive case.
        
        We discuss possible options for extending this FET approach to timesteppers for the compressible case.

        The kinetic corrections satisfy linearized Boltzmann equations. Using a Lénard--Bernstein collision operator, these take Fokker--Planck-like forms \cite{Fokker_1914, Planck_1917} wherein pseudo-particles in the numerical model obey the neoclassical transport equations, with particle-independent Brownian drift terms. This offers a rigorous methodology for incorporating collisions into the particle transport model, without coupling the equations of motions for each particle.
        
        Works by Chen, Chacón et al. \cite{Chen_Chacón_Barnes_2011, Chacón_Chen_Barnes_2013, Chen_Chacón_2014, Chen_Chacón_2015} have developed structure-preserving particle pushers for neoclassical transport in the Vlasov equations, derived from Crank--Nicolson integrators. We show these too can can derive from a FET interpretation, similarly offering potential extensions to higher-order-in-time particle pushers. The FET formulation is used also to consider how the stochastic drift terms can be incorporated into the pushers. Stochastic gyrokinetic expansions are also discussed.

        Different options for the numerical implementation of these schemes are considered.

        Due to the efficacy of FET in the development of SP timesteppers for both the fluid and kinetic component, we hope this approach will prove effective in the future for developing SP timesteppers for the full hybrid model. We hope this will give us the opportunity to incorporate previously inaccessible kinetic effects into the highly effective, modern, finite-element MHD models.
    \end{abstract}
    
    
    \newpage
    \tableofcontents
    
    
    \newpage
    \pagenumbering{arabic}
    %\linenumbers\renewcommand\thelinenumber{\color{black!50}\arabic{linenumber}}
            \input{0 - introduction/main.tex}
        \part{Research}
            \input{1 - low-noise PiC models/main.tex}
            \input{2 - kinetic component/main.tex}
            \input{3 - fluid component/main.tex}
            \input{4 - numerical implementation/main.tex}
        \part{Project Overview}
            \input{5 - research plan/main.tex}
            \input{6 - summary/main.tex}
    
    
    %\section{}
    \newpage
    \pagenumbering{gobble}
        \printbibliography


    \newpage
    \pagenumbering{roman}
    \appendix
        \part{Appendices}
            \input{8 - Hilbert complexes/main.tex}
            \input{9 - weak conservation proofs/main.tex}
\end{document}

\end{document}

            \documentclass[12pt, a4paper]{report}

\documentclass[12pt, a4paper]{report}

\input{template/main.tex}

\title{\BA{Title in Progress...}}
\author{Boris Andrews}
\affil{Mathematical Institute, University of Oxford}
\date{\today}


\begin{document}
    \pagenumbering{gobble}
    \maketitle
    
    
    \begin{abstract}
        Magnetic confinement reactors---in particular tokamaks---offer one of the most promising options for achieving practical nuclear fusion, with the potential to provide virtually limitless, clean energy. The theoretical and numerical modeling of tokamak plasmas is simultaneously an essential component of effective reactor design, and a great research barrier. Tokamak operational conditions exhibit comparatively low Knudsen numbers. Kinetic effects, including kinetic waves and instabilities, Landau damping, bump-on-tail instabilities and more, are therefore highly influential in tokamak plasma dynamics. Purely fluid models are inherently incapable of capturing these effects, whereas the high dimensionality in purely kinetic models render them practically intractable for most relevant purposes.

        We consider a $\delta\!f$ decomposition model, with a macroscopic fluid background and microscopic kinetic correction, both fully coupled to each other. A similar manner of discretization is proposed to that used in the recent \texttt{STRUPHY} code \cite{Holderied_Possanner_Wang_2021, Holderied_2022, Li_et_al_2023} with a finite-element model for the background and a pseudo-particle/PiC model for the correction.

        The fluid background satisfies the full, non-linear, resistive, compressible, Hall MHD equations. \cite{Laakmann_Hu_Farrell_2022} introduces finite-element(-in-space) implicit timesteppers for the incompressible analogue to this system with structure-preserving (SP) properties in the ideal case, alongside parameter-robust preconditioners. We show that these timesteppers can derive from a finite-element-in-time (FET) (and finite-element-in-space) interpretation. The benefits of this reformulation are discussed, including the derivation of timesteppers that are higher order in time, and the quantifiable dissipative SP properties in the non-ideal, resistive case.
        
        We discuss possible options for extending this FET approach to timesteppers for the compressible case.

        The kinetic corrections satisfy linearized Boltzmann equations. Using a Lénard--Bernstein collision operator, these take Fokker--Planck-like forms \cite{Fokker_1914, Planck_1917} wherein pseudo-particles in the numerical model obey the neoclassical transport equations, with particle-independent Brownian drift terms. This offers a rigorous methodology for incorporating collisions into the particle transport model, without coupling the equations of motions for each particle.
        
        Works by Chen, Chacón et al. \cite{Chen_Chacón_Barnes_2011, Chacón_Chen_Barnes_2013, Chen_Chacón_2014, Chen_Chacón_2015} have developed structure-preserving particle pushers for neoclassical transport in the Vlasov equations, derived from Crank--Nicolson integrators. We show these too can can derive from a FET interpretation, similarly offering potential extensions to higher-order-in-time particle pushers. The FET formulation is used also to consider how the stochastic drift terms can be incorporated into the pushers. Stochastic gyrokinetic expansions are also discussed.

        Different options for the numerical implementation of these schemes are considered.

        Due to the efficacy of FET in the development of SP timesteppers for both the fluid and kinetic component, we hope this approach will prove effective in the future for developing SP timesteppers for the full hybrid model. We hope this will give us the opportunity to incorporate previously inaccessible kinetic effects into the highly effective, modern, finite-element MHD models.
    \end{abstract}
    
    
    \newpage
    \tableofcontents
    
    
    \newpage
    \pagenumbering{arabic}
    %\linenumbers\renewcommand\thelinenumber{\color{black!50}\arabic{linenumber}}
            \input{0 - introduction/main.tex}
        \part{Research}
            \input{1 - low-noise PiC models/main.tex}
            \input{2 - kinetic component/main.tex}
            \input{3 - fluid component/main.tex}
            \input{4 - numerical implementation/main.tex}
        \part{Project Overview}
            \input{5 - research plan/main.tex}
            \input{6 - summary/main.tex}
    
    
    %\section{}
    \newpage
    \pagenumbering{gobble}
        \printbibliography


    \newpage
    \pagenumbering{roman}
    \appendix
        \part{Appendices}
            \input{8 - Hilbert complexes/main.tex}
            \input{9 - weak conservation proofs/main.tex}
\end{document}


\title{\BA{Title in Progress...}}
\author{Boris Andrews}
\affil{Mathematical Institute, University of Oxford}
\date{\today}


\begin{document}
    \pagenumbering{gobble}
    \maketitle
    
    
    \begin{abstract}
        Magnetic confinement reactors---in particular tokamaks---offer one of the most promising options for achieving practical nuclear fusion, with the potential to provide virtually limitless, clean energy. The theoretical and numerical modeling of tokamak plasmas is simultaneously an essential component of effective reactor design, and a great research barrier. Tokamak operational conditions exhibit comparatively low Knudsen numbers. Kinetic effects, including kinetic waves and instabilities, Landau damping, bump-on-tail instabilities and more, are therefore highly influential in tokamak plasma dynamics. Purely fluid models are inherently incapable of capturing these effects, whereas the high dimensionality in purely kinetic models render them practically intractable for most relevant purposes.

        We consider a $\delta\!f$ decomposition model, with a macroscopic fluid background and microscopic kinetic correction, both fully coupled to each other. A similar manner of discretization is proposed to that used in the recent \texttt{STRUPHY} code \cite{Holderied_Possanner_Wang_2021, Holderied_2022, Li_et_al_2023} with a finite-element model for the background and a pseudo-particle/PiC model for the correction.

        The fluid background satisfies the full, non-linear, resistive, compressible, Hall MHD equations. \cite{Laakmann_Hu_Farrell_2022} introduces finite-element(-in-space) implicit timesteppers for the incompressible analogue to this system with structure-preserving (SP) properties in the ideal case, alongside parameter-robust preconditioners. We show that these timesteppers can derive from a finite-element-in-time (FET) (and finite-element-in-space) interpretation. The benefits of this reformulation are discussed, including the derivation of timesteppers that are higher order in time, and the quantifiable dissipative SP properties in the non-ideal, resistive case.
        
        We discuss possible options for extending this FET approach to timesteppers for the compressible case.

        The kinetic corrections satisfy linearized Boltzmann equations. Using a Lénard--Bernstein collision operator, these take Fokker--Planck-like forms \cite{Fokker_1914, Planck_1917} wherein pseudo-particles in the numerical model obey the neoclassical transport equations, with particle-independent Brownian drift terms. This offers a rigorous methodology for incorporating collisions into the particle transport model, without coupling the equations of motions for each particle.
        
        Works by Chen, Chacón et al. \cite{Chen_Chacón_Barnes_2011, Chacón_Chen_Barnes_2013, Chen_Chacón_2014, Chen_Chacón_2015} have developed structure-preserving particle pushers for neoclassical transport in the Vlasov equations, derived from Crank--Nicolson integrators. We show these too can can derive from a FET interpretation, similarly offering potential extensions to higher-order-in-time particle pushers. The FET formulation is used also to consider how the stochastic drift terms can be incorporated into the pushers. Stochastic gyrokinetic expansions are also discussed.

        Different options for the numerical implementation of these schemes are considered.

        Due to the efficacy of FET in the development of SP timesteppers for both the fluid and kinetic component, we hope this approach will prove effective in the future for developing SP timesteppers for the full hybrid model. We hope this will give us the opportunity to incorporate previously inaccessible kinetic effects into the highly effective, modern, finite-element MHD models.
    \end{abstract}
    
    
    \newpage
    \tableofcontents
    
    
    \newpage
    \pagenumbering{arabic}
    %\linenumbers\renewcommand\thelinenumber{\color{black!50}\arabic{linenumber}}
            \documentclass[12pt, a4paper]{report}

\input{template/main.tex}

\title{\BA{Title in Progress...}}
\author{Boris Andrews}
\affil{Mathematical Institute, University of Oxford}
\date{\today}


\begin{document}
    \pagenumbering{gobble}
    \maketitle
    
    
    \begin{abstract}
        Magnetic confinement reactors---in particular tokamaks---offer one of the most promising options for achieving practical nuclear fusion, with the potential to provide virtually limitless, clean energy. The theoretical and numerical modeling of tokamak plasmas is simultaneously an essential component of effective reactor design, and a great research barrier. Tokamak operational conditions exhibit comparatively low Knudsen numbers. Kinetic effects, including kinetic waves and instabilities, Landau damping, bump-on-tail instabilities and more, are therefore highly influential in tokamak plasma dynamics. Purely fluid models are inherently incapable of capturing these effects, whereas the high dimensionality in purely kinetic models render them practically intractable for most relevant purposes.

        We consider a $\delta\!f$ decomposition model, with a macroscopic fluid background and microscopic kinetic correction, both fully coupled to each other. A similar manner of discretization is proposed to that used in the recent \texttt{STRUPHY} code \cite{Holderied_Possanner_Wang_2021, Holderied_2022, Li_et_al_2023} with a finite-element model for the background and a pseudo-particle/PiC model for the correction.

        The fluid background satisfies the full, non-linear, resistive, compressible, Hall MHD equations. \cite{Laakmann_Hu_Farrell_2022} introduces finite-element(-in-space) implicit timesteppers for the incompressible analogue to this system with structure-preserving (SP) properties in the ideal case, alongside parameter-robust preconditioners. We show that these timesteppers can derive from a finite-element-in-time (FET) (and finite-element-in-space) interpretation. The benefits of this reformulation are discussed, including the derivation of timesteppers that are higher order in time, and the quantifiable dissipative SP properties in the non-ideal, resistive case.
        
        We discuss possible options for extending this FET approach to timesteppers for the compressible case.

        The kinetic corrections satisfy linearized Boltzmann equations. Using a Lénard--Bernstein collision operator, these take Fokker--Planck-like forms \cite{Fokker_1914, Planck_1917} wherein pseudo-particles in the numerical model obey the neoclassical transport equations, with particle-independent Brownian drift terms. This offers a rigorous methodology for incorporating collisions into the particle transport model, without coupling the equations of motions for each particle.
        
        Works by Chen, Chacón et al. \cite{Chen_Chacón_Barnes_2011, Chacón_Chen_Barnes_2013, Chen_Chacón_2014, Chen_Chacón_2015} have developed structure-preserving particle pushers for neoclassical transport in the Vlasov equations, derived from Crank--Nicolson integrators. We show these too can can derive from a FET interpretation, similarly offering potential extensions to higher-order-in-time particle pushers. The FET formulation is used also to consider how the stochastic drift terms can be incorporated into the pushers. Stochastic gyrokinetic expansions are also discussed.

        Different options for the numerical implementation of these schemes are considered.

        Due to the efficacy of FET in the development of SP timesteppers for both the fluid and kinetic component, we hope this approach will prove effective in the future for developing SP timesteppers for the full hybrid model. We hope this will give us the opportunity to incorporate previously inaccessible kinetic effects into the highly effective, modern, finite-element MHD models.
    \end{abstract}
    
    
    \newpage
    \tableofcontents
    
    
    \newpage
    \pagenumbering{arabic}
    %\linenumbers\renewcommand\thelinenumber{\color{black!50}\arabic{linenumber}}
            \input{0 - introduction/main.tex}
        \part{Research}
            \input{1 - low-noise PiC models/main.tex}
            \input{2 - kinetic component/main.tex}
            \input{3 - fluid component/main.tex}
            \input{4 - numerical implementation/main.tex}
        \part{Project Overview}
            \input{5 - research plan/main.tex}
            \input{6 - summary/main.tex}
    
    
    %\section{}
    \newpage
    \pagenumbering{gobble}
        \printbibliography


    \newpage
    \pagenumbering{roman}
    \appendix
        \part{Appendices}
            \input{8 - Hilbert complexes/main.tex}
            \input{9 - weak conservation proofs/main.tex}
\end{document}

        \part{Research}
            \documentclass[12pt, a4paper]{report}

\input{template/main.tex}

\title{\BA{Title in Progress...}}
\author{Boris Andrews}
\affil{Mathematical Institute, University of Oxford}
\date{\today}


\begin{document}
    \pagenumbering{gobble}
    \maketitle
    
    
    \begin{abstract}
        Magnetic confinement reactors---in particular tokamaks---offer one of the most promising options for achieving practical nuclear fusion, with the potential to provide virtually limitless, clean energy. The theoretical and numerical modeling of tokamak plasmas is simultaneously an essential component of effective reactor design, and a great research barrier. Tokamak operational conditions exhibit comparatively low Knudsen numbers. Kinetic effects, including kinetic waves and instabilities, Landau damping, bump-on-tail instabilities and more, are therefore highly influential in tokamak plasma dynamics. Purely fluid models are inherently incapable of capturing these effects, whereas the high dimensionality in purely kinetic models render them practically intractable for most relevant purposes.

        We consider a $\delta\!f$ decomposition model, with a macroscopic fluid background and microscopic kinetic correction, both fully coupled to each other. A similar manner of discretization is proposed to that used in the recent \texttt{STRUPHY} code \cite{Holderied_Possanner_Wang_2021, Holderied_2022, Li_et_al_2023} with a finite-element model for the background and a pseudo-particle/PiC model for the correction.

        The fluid background satisfies the full, non-linear, resistive, compressible, Hall MHD equations. \cite{Laakmann_Hu_Farrell_2022} introduces finite-element(-in-space) implicit timesteppers for the incompressible analogue to this system with structure-preserving (SP) properties in the ideal case, alongside parameter-robust preconditioners. We show that these timesteppers can derive from a finite-element-in-time (FET) (and finite-element-in-space) interpretation. The benefits of this reformulation are discussed, including the derivation of timesteppers that are higher order in time, and the quantifiable dissipative SP properties in the non-ideal, resistive case.
        
        We discuss possible options for extending this FET approach to timesteppers for the compressible case.

        The kinetic corrections satisfy linearized Boltzmann equations. Using a Lénard--Bernstein collision operator, these take Fokker--Planck-like forms \cite{Fokker_1914, Planck_1917} wherein pseudo-particles in the numerical model obey the neoclassical transport equations, with particle-independent Brownian drift terms. This offers a rigorous methodology for incorporating collisions into the particle transport model, without coupling the equations of motions for each particle.
        
        Works by Chen, Chacón et al. \cite{Chen_Chacón_Barnes_2011, Chacón_Chen_Barnes_2013, Chen_Chacón_2014, Chen_Chacón_2015} have developed structure-preserving particle pushers for neoclassical transport in the Vlasov equations, derived from Crank--Nicolson integrators. We show these too can can derive from a FET interpretation, similarly offering potential extensions to higher-order-in-time particle pushers. The FET formulation is used also to consider how the stochastic drift terms can be incorporated into the pushers. Stochastic gyrokinetic expansions are also discussed.

        Different options for the numerical implementation of these schemes are considered.

        Due to the efficacy of FET in the development of SP timesteppers for both the fluid and kinetic component, we hope this approach will prove effective in the future for developing SP timesteppers for the full hybrid model. We hope this will give us the opportunity to incorporate previously inaccessible kinetic effects into the highly effective, modern, finite-element MHD models.
    \end{abstract}
    
    
    \newpage
    \tableofcontents
    
    
    \newpage
    \pagenumbering{arabic}
    %\linenumbers\renewcommand\thelinenumber{\color{black!50}\arabic{linenumber}}
            \input{0 - introduction/main.tex}
        \part{Research}
            \input{1 - low-noise PiC models/main.tex}
            \input{2 - kinetic component/main.tex}
            \input{3 - fluid component/main.tex}
            \input{4 - numerical implementation/main.tex}
        \part{Project Overview}
            \input{5 - research plan/main.tex}
            \input{6 - summary/main.tex}
    
    
    %\section{}
    \newpage
    \pagenumbering{gobble}
        \printbibliography


    \newpage
    \pagenumbering{roman}
    \appendix
        \part{Appendices}
            \input{8 - Hilbert complexes/main.tex}
            \input{9 - weak conservation proofs/main.tex}
\end{document}

            \documentclass[12pt, a4paper]{report}

\input{template/main.tex}

\title{\BA{Title in Progress...}}
\author{Boris Andrews}
\affil{Mathematical Institute, University of Oxford}
\date{\today}


\begin{document}
    \pagenumbering{gobble}
    \maketitle
    
    
    \begin{abstract}
        Magnetic confinement reactors---in particular tokamaks---offer one of the most promising options for achieving practical nuclear fusion, with the potential to provide virtually limitless, clean energy. The theoretical and numerical modeling of tokamak plasmas is simultaneously an essential component of effective reactor design, and a great research barrier. Tokamak operational conditions exhibit comparatively low Knudsen numbers. Kinetic effects, including kinetic waves and instabilities, Landau damping, bump-on-tail instabilities and more, are therefore highly influential in tokamak plasma dynamics. Purely fluid models are inherently incapable of capturing these effects, whereas the high dimensionality in purely kinetic models render them practically intractable for most relevant purposes.

        We consider a $\delta\!f$ decomposition model, with a macroscopic fluid background and microscopic kinetic correction, both fully coupled to each other. A similar manner of discretization is proposed to that used in the recent \texttt{STRUPHY} code \cite{Holderied_Possanner_Wang_2021, Holderied_2022, Li_et_al_2023} with a finite-element model for the background and a pseudo-particle/PiC model for the correction.

        The fluid background satisfies the full, non-linear, resistive, compressible, Hall MHD equations. \cite{Laakmann_Hu_Farrell_2022} introduces finite-element(-in-space) implicit timesteppers for the incompressible analogue to this system with structure-preserving (SP) properties in the ideal case, alongside parameter-robust preconditioners. We show that these timesteppers can derive from a finite-element-in-time (FET) (and finite-element-in-space) interpretation. The benefits of this reformulation are discussed, including the derivation of timesteppers that are higher order in time, and the quantifiable dissipative SP properties in the non-ideal, resistive case.
        
        We discuss possible options for extending this FET approach to timesteppers for the compressible case.

        The kinetic corrections satisfy linearized Boltzmann equations. Using a Lénard--Bernstein collision operator, these take Fokker--Planck-like forms \cite{Fokker_1914, Planck_1917} wherein pseudo-particles in the numerical model obey the neoclassical transport equations, with particle-independent Brownian drift terms. This offers a rigorous methodology for incorporating collisions into the particle transport model, without coupling the equations of motions for each particle.
        
        Works by Chen, Chacón et al. \cite{Chen_Chacón_Barnes_2011, Chacón_Chen_Barnes_2013, Chen_Chacón_2014, Chen_Chacón_2015} have developed structure-preserving particle pushers for neoclassical transport in the Vlasov equations, derived from Crank--Nicolson integrators. We show these too can can derive from a FET interpretation, similarly offering potential extensions to higher-order-in-time particle pushers. The FET formulation is used also to consider how the stochastic drift terms can be incorporated into the pushers. Stochastic gyrokinetic expansions are also discussed.

        Different options for the numerical implementation of these schemes are considered.

        Due to the efficacy of FET in the development of SP timesteppers for both the fluid and kinetic component, we hope this approach will prove effective in the future for developing SP timesteppers for the full hybrid model. We hope this will give us the opportunity to incorporate previously inaccessible kinetic effects into the highly effective, modern, finite-element MHD models.
    \end{abstract}
    
    
    \newpage
    \tableofcontents
    
    
    \newpage
    \pagenumbering{arabic}
    %\linenumbers\renewcommand\thelinenumber{\color{black!50}\arabic{linenumber}}
            \input{0 - introduction/main.tex}
        \part{Research}
            \input{1 - low-noise PiC models/main.tex}
            \input{2 - kinetic component/main.tex}
            \input{3 - fluid component/main.tex}
            \input{4 - numerical implementation/main.tex}
        \part{Project Overview}
            \input{5 - research plan/main.tex}
            \input{6 - summary/main.tex}
    
    
    %\section{}
    \newpage
    \pagenumbering{gobble}
        \printbibliography


    \newpage
    \pagenumbering{roman}
    \appendix
        \part{Appendices}
            \input{8 - Hilbert complexes/main.tex}
            \input{9 - weak conservation proofs/main.tex}
\end{document}

            \documentclass[12pt, a4paper]{report}

\input{template/main.tex}

\title{\BA{Title in Progress...}}
\author{Boris Andrews}
\affil{Mathematical Institute, University of Oxford}
\date{\today}


\begin{document}
    \pagenumbering{gobble}
    \maketitle
    
    
    \begin{abstract}
        Magnetic confinement reactors---in particular tokamaks---offer one of the most promising options for achieving practical nuclear fusion, with the potential to provide virtually limitless, clean energy. The theoretical and numerical modeling of tokamak plasmas is simultaneously an essential component of effective reactor design, and a great research barrier. Tokamak operational conditions exhibit comparatively low Knudsen numbers. Kinetic effects, including kinetic waves and instabilities, Landau damping, bump-on-tail instabilities and more, are therefore highly influential in tokamak plasma dynamics. Purely fluid models are inherently incapable of capturing these effects, whereas the high dimensionality in purely kinetic models render them practically intractable for most relevant purposes.

        We consider a $\delta\!f$ decomposition model, with a macroscopic fluid background and microscopic kinetic correction, both fully coupled to each other. A similar manner of discretization is proposed to that used in the recent \texttt{STRUPHY} code \cite{Holderied_Possanner_Wang_2021, Holderied_2022, Li_et_al_2023} with a finite-element model for the background and a pseudo-particle/PiC model for the correction.

        The fluid background satisfies the full, non-linear, resistive, compressible, Hall MHD equations. \cite{Laakmann_Hu_Farrell_2022} introduces finite-element(-in-space) implicit timesteppers for the incompressible analogue to this system with structure-preserving (SP) properties in the ideal case, alongside parameter-robust preconditioners. We show that these timesteppers can derive from a finite-element-in-time (FET) (and finite-element-in-space) interpretation. The benefits of this reformulation are discussed, including the derivation of timesteppers that are higher order in time, and the quantifiable dissipative SP properties in the non-ideal, resistive case.
        
        We discuss possible options for extending this FET approach to timesteppers for the compressible case.

        The kinetic corrections satisfy linearized Boltzmann equations. Using a Lénard--Bernstein collision operator, these take Fokker--Planck-like forms \cite{Fokker_1914, Planck_1917} wherein pseudo-particles in the numerical model obey the neoclassical transport equations, with particle-independent Brownian drift terms. This offers a rigorous methodology for incorporating collisions into the particle transport model, without coupling the equations of motions for each particle.
        
        Works by Chen, Chacón et al. \cite{Chen_Chacón_Barnes_2011, Chacón_Chen_Barnes_2013, Chen_Chacón_2014, Chen_Chacón_2015} have developed structure-preserving particle pushers for neoclassical transport in the Vlasov equations, derived from Crank--Nicolson integrators. We show these too can can derive from a FET interpretation, similarly offering potential extensions to higher-order-in-time particle pushers. The FET formulation is used also to consider how the stochastic drift terms can be incorporated into the pushers. Stochastic gyrokinetic expansions are also discussed.

        Different options for the numerical implementation of these schemes are considered.

        Due to the efficacy of FET in the development of SP timesteppers for both the fluid and kinetic component, we hope this approach will prove effective in the future for developing SP timesteppers for the full hybrid model. We hope this will give us the opportunity to incorporate previously inaccessible kinetic effects into the highly effective, modern, finite-element MHD models.
    \end{abstract}
    
    
    \newpage
    \tableofcontents
    
    
    \newpage
    \pagenumbering{arabic}
    %\linenumbers\renewcommand\thelinenumber{\color{black!50}\arabic{linenumber}}
            \input{0 - introduction/main.tex}
        \part{Research}
            \input{1 - low-noise PiC models/main.tex}
            \input{2 - kinetic component/main.tex}
            \input{3 - fluid component/main.tex}
            \input{4 - numerical implementation/main.tex}
        \part{Project Overview}
            \input{5 - research plan/main.tex}
            \input{6 - summary/main.tex}
    
    
    %\section{}
    \newpage
    \pagenumbering{gobble}
        \printbibliography


    \newpage
    \pagenumbering{roman}
    \appendix
        \part{Appendices}
            \input{8 - Hilbert complexes/main.tex}
            \input{9 - weak conservation proofs/main.tex}
\end{document}

            \documentclass[12pt, a4paper]{report}

\input{template/main.tex}

\title{\BA{Title in Progress...}}
\author{Boris Andrews}
\affil{Mathematical Institute, University of Oxford}
\date{\today}


\begin{document}
    \pagenumbering{gobble}
    \maketitle
    
    
    \begin{abstract}
        Magnetic confinement reactors---in particular tokamaks---offer one of the most promising options for achieving practical nuclear fusion, with the potential to provide virtually limitless, clean energy. The theoretical and numerical modeling of tokamak plasmas is simultaneously an essential component of effective reactor design, and a great research barrier. Tokamak operational conditions exhibit comparatively low Knudsen numbers. Kinetic effects, including kinetic waves and instabilities, Landau damping, bump-on-tail instabilities and more, are therefore highly influential in tokamak plasma dynamics. Purely fluid models are inherently incapable of capturing these effects, whereas the high dimensionality in purely kinetic models render them practically intractable for most relevant purposes.

        We consider a $\delta\!f$ decomposition model, with a macroscopic fluid background and microscopic kinetic correction, both fully coupled to each other. A similar manner of discretization is proposed to that used in the recent \texttt{STRUPHY} code \cite{Holderied_Possanner_Wang_2021, Holderied_2022, Li_et_al_2023} with a finite-element model for the background and a pseudo-particle/PiC model for the correction.

        The fluid background satisfies the full, non-linear, resistive, compressible, Hall MHD equations. \cite{Laakmann_Hu_Farrell_2022} introduces finite-element(-in-space) implicit timesteppers for the incompressible analogue to this system with structure-preserving (SP) properties in the ideal case, alongside parameter-robust preconditioners. We show that these timesteppers can derive from a finite-element-in-time (FET) (and finite-element-in-space) interpretation. The benefits of this reformulation are discussed, including the derivation of timesteppers that are higher order in time, and the quantifiable dissipative SP properties in the non-ideal, resistive case.
        
        We discuss possible options for extending this FET approach to timesteppers for the compressible case.

        The kinetic corrections satisfy linearized Boltzmann equations. Using a Lénard--Bernstein collision operator, these take Fokker--Planck-like forms \cite{Fokker_1914, Planck_1917} wherein pseudo-particles in the numerical model obey the neoclassical transport equations, with particle-independent Brownian drift terms. This offers a rigorous methodology for incorporating collisions into the particle transport model, without coupling the equations of motions for each particle.
        
        Works by Chen, Chacón et al. \cite{Chen_Chacón_Barnes_2011, Chacón_Chen_Barnes_2013, Chen_Chacón_2014, Chen_Chacón_2015} have developed structure-preserving particle pushers for neoclassical transport in the Vlasov equations, derived from Crank--Nicolson integrators. We show these too can can derive from a FET interpretation, similarly offering potential extensions to higher-order-in-time particle pushers. The FET formulation is used also to consider how the stochastic drift terms can be incorporated into the pushers. Stochastic gyrokinetic expansions are also discussed.

        Different options for the numerical implementation of these schemes are considered.

        Due to the efficacy of FET in the development of SP timesteppers for both the fluid and kinetic component, we hope this approach will prove effective in the future for developing SP timesteppers for the full hybrid model. We hope this will give us the opportunity to incorporate previously inaccessible kinetic effects into the highly effective, modern, finite-element MHD models.
    \end{abstract}
    
    
    \newpage
    \tableofcontents
    
    
    \newpage
    \pagenumbering{arabic}
    %\linenumbers\renewcommand\thelinenumber{\color{black!50}\arabic{linenumber}}
            \input{0 - introduction/main.tex}
        \part{Research}
            \input{1 - low-noise PiC models/main.tex}
            \input{2 - kinetic component/main.tex}
            \input{3 - fluid component/main.tex}
            \input{4 - numerical implementation/main.tex}
        \part{Project Overview}
            \input{5 - research plan/main.tex}
            \input{6 - summary/main.tex}
    
    
    %\section{}
    \newpage
    \pagenumbering{gobble}
        \printbibliography


    \newpage
    \pagenumbering{roman}
    \appendix
        \part{Appendices}
            \input{8 - Hilbert complexes/main.tex}
            \input{9 - weak conservation proofs/main.tex}
\end{document}

        \part{Project Overview}
            \documentclass[12pt, a4paper]{report}

\input{template/main.tex}

\title{\BA{Title in Progress...}}
\author{Boris Andrews}
\affil{Mathematical Institute, University of Oxford}
\date{\today}


\begin{document}
    \pagenumbering{gobble}
    \maketitle
    
    
    \begin{abstract}
        Magnetic confinement reactors---in particular tokamaks---offer one of the most promising options for achieving practical nuclear fusion, with the potential to provide virtually limitless, clean energy. The theoretical and numerical modeling of tokamak plasmas is simultaneously an essential component of effective reactor design, and a great research barrier. Tokamak operational conditions exhibit comparatively low Knudsen numbers. Kinetic effects, including kinetic waves and instabilities, Landau damping, bump-on-tail instabilities and more, are therefore highly influential in tokamak plasma dynamics. Purely fluid models are inherently incapable of capturing these effects, whereas the high dimensionality in purely kinetic models render them practically intractable for most relevant purposes.

        We consider a $\delta\!f$ decomposition model, with a macroscopic fluid background and microscopic kinetic correction, both fully coupled to each other. A similar manner of discretization is proposed to that used in the recent \texttt{STRUPHY} code \cite{Holderied_Possanner_Wang_2021, Holderied_2022, Li_et_al_2023} with a finite-element model for the background and a pseudo-particle/PiC model for the correction.

        The fluid background satisfies the full, non-linear, resistive, compressible, Hall MHD equations. \cite{Laakmann_Hu_Farrell_2022} introduces finite-element(-in-space) implicit timesteppers for the incompressible analogue to this system with structure-preserving (SP) properties in the ideal case, alongside parameter-robust preconditioners. We show that these timesteppers can derive from a finite-element-in-time (FET) (and finite-element-in-space) interpretation. The benefits of this reformulation are discussed, including the derivation of timesteppers that are higher order in time, and the quantifiable dissipative SP properties in the non-ideal, resistive case.
        
        We discuss possible options for extending this FET approach to timesteppers for the compressible case.

        The kinetic corrections satisfy linearized Boltzmann equations. Using a Lénard--Bernstein collision operator, these take Fokker--Planck-like forms \cite{Fokker_1914, Planck_1917} wherein pseudo-particles in the numerical model obey the neoclassical transport equations, with particle-independent Brownian drift terms. This offers a rigorous methodology for incorporating collisions into the particle transport model, without coupling the equations of motions for each particle.
        
        Works by Chen, Chacón et al. \cite{Chen_Chacón_Barnes_2011, Chacón_Chen_Barnes_2013, Chen_Chacón_2014, Chen_Chacón_2015} have developed structure-preserving particle pushers for neoclassical transport in the Vlasov equations, derived from Crank--Nicolson integrators. We show these too can can derive from a FET interpretation, similarly offering potential extensions to higher-order-in-time particle pushers. The FET formulation is used also to consider how the stochastic drift terms can be incorporated into the pushers. Stochastic gyrokinetic expansions are also discussed.

        Different options for the numerical implementation of these schemes are considered.

        Due to the efficacy of FET in the development of SP timesteppers for both the fluid and kinetic component, we hope this approach will prove effective in the future for developing SP timesteppers for the full hybrid model. We hope this will give us the opportunity to incorporate previously inaccessible kinetic effects into the highly effective, modern, finite-element MHD models.
    \end{abstract}
    
    
    \newpage
    \tableofcontents
    
    
    \newpage
    \pagenumbering{arabic}
    %\linenumbers\renewcommand\thelinenumber{\color{black!50}\arabic{linenumber}}
            \input{0 - introduction/main.tex}
        \part{Research}
            \input{1 - low-noise PiC models/main.tex}
            \input{2 - kinetic component/main.tex}
            \input{3 - fluid component/main.tex}
            \input{4 - numerical implementation/main.tex}
        \part{Project Overview}
            \input{5 - research plan/main.tex}
            \input{6 - summary/main.tex}
    
    
    %\section{}
    \newpage
    \pagenumbering{gobble}
        \printbibliography


    \newpage
    \pagenumbering{roman}
    \appendix
        \part{Appendices}
            \input{8 - Hilbert complexes/main.tex}
            \input{9 - weak conservation proofs/main.tex}
\end{document}

            \documentclass[12pt, a4paper]{report}

\input{template/main.tex}

\title{\BA{Title in Progress...}}
\author{Boris Andrews}
\affil{Mathematical Institute, University of Oxford}
\date{\today}


\begin{document}
    \pagenumbering{gobble}
    \maketitle
    
    
    \begin{abstract}
        Magnetic confinement reactors---in particular tokamaks---offer one of the most promising options for achieving practical nuclear fusion, with the potential to provide virtually limitless, clean energy. The theoretical and numerical modeling of tokamak plasmas is simultaneously an essential component of effective reactor design, and a great research barrier. Tokamak operational conditions exhibit comparatively low Knudsen numbers. Kinetic effects, including kinetic waves and instabilities, Landau damping, bump-on-tail instabilities and more, are therefore highly influential in tokamak plasma dynamics. Purely fluid models are inherently incapable of capturing these effects, whereas the high dimensionality in purely kinetic models render them practically intractable for most relevant purposes.

        We consider a $\delta\!f$ decomposition model, with a macroscopic fluid background and microscopic kinetic correction, both fully coupled to each other. A similar manner of discretization is proposed to that used in the recent \texttt{STRUPHY} code \cite{Holderied_Possanner_Wang_2021, Holderied_2022, Li_et_al_2023} with a finite-element model for the background and a pseudo-particle/PiC model for the correction.

        The fluid background satisfies the full, non-linear, resistive, compressible, Hall MHD equations. \cite{Laakmann_Hu_Farrell_2022} introduces finite-element(-in-space) implicit timesteppers for the incompressible analogue to this system with structure-preserving (SP) properties in the ideal case, alongside parameter-robust preconditioners. We show that these timesteppers can derive from a finite-element-in-time (FET) (and finite-element-in-space) interpretation. The benefits of this reformulation are discussed, including the derivation of timesteppers that are higher order in time, and the quantifiable dissipative SP properties in the non-ideal, resistive case.
        
        We discuss possible options for extending this FET approach to timesteppers for the compressible case.

        The kinetic corrections satisfy linearized Boltzmann equations. Using a Lénard--Bernstein collision operator, these take Fokker--Planck-like forms \cite{Fokker_1914, Planck_1917} wherein pseudo-particles in the numerical model obey the neoclassical transport equations, with particle-independent Brownian drift terms. This offers a rigorous methodology for incorporating collisions into the particle transport model, without coupling the equations of motions for each particle.
        
        Works by Chen, Chacón et al. \cite{Chen_Chacón_Barnes_2011, Chacón_Chen_Barnes_2013, Chen_Chacón_2014, Chen_Chacón_2015} have developed structure-preserving particle pushers for neoclassical transport in the Vlasov equations, derived from Crank--Nicolson integrators. We show these too can can derive from a FET interpretation, similarly offering potential extensions to higher-order-in-time particle pushers. The FET formulation is used also to consider how the stochastic drift terms can be incorporated into the pushers. Stochastic gyrokinetic expansions are also discussed.

        Different options for the numerical implementation of these schemes are considered.

        Due to the efficacy of FET in the development of SP timesteppers for both the fluid and kinetic component, we hope this approach will prove effective in the future for developing SP timesteppers for the full hybrid model. We hope this will give us the opportunity to incorporate previously inaccessible kinetic effects into the highly effective, modern, finite-element MHD models.
    \end{abstract}
    
    
    \newpage
    \tableofcontents
    
    
    \newpage
    \pagenumbering{arabic}
    %\linenumbers\renewcommand\thelinenumber{\color{black!50}\arabic{linenumber}}
            \input{0 - introduction/main.tex}
        \part{Research}
            \input{1 - low-noise PiC models/main.tex}
            \input{2 - kinetic component/main.tex}
            \input{3 - fluid component/main.tex}
            \input{4 - numerical implementation/main.tex}
        \part{Project Overview}
            \input{5 - research plan/main.tex}
            \input{6 - summary/main.tex}
    
    
    %\section{}
    \newpage
    \pagenumbering{gobble}
        \printbibliography


    \newpage
    \pagenumbering{roman}
    \appendix
        \part{Appendices}
            \input{8 - Hilbert complexes/main.tex}
            \input{9 - weak conservation proofs/main.tex}
\end{document}

    
    
    %\section{}
    \newpage
    \pagenumbering{gobble}
        \printbibliography


    \newpage
    \pagenumbering{roman}
    \appendix
        \part{Appendices}
            \documentclass[12pt, a4paper]{report}

\input{template/main.tex}

\title{\BA{Title in Progress...}}
\author{Boris Andrews}
\affil{Mathematical Institute, University of Oxford}
\date{\today}


\begin{document}
    \pagenumbering{gobble}
    \maketitle
    
    
    \begin{abstract}
        Magnetic confinement reactors---in particular tokamaks---offer one of the most promising options for achieving practical nuclear fusion, with the potential to provide virtually limitless, clean energy. The theoretical and numerical modeling of tokamak plasmas is simultaneously an essential component of effective reactor design, and a great research barrier. Tokamak operational conditions exhibit comparatively low Knudsen numbers. Kinetic effects, including kinetic waves and instabilities, Landau damping, bump-on-tail instabilities and more, are therefore highly influential in tokamak plasma dynamics. Purely fluid models are inherently incapable of capturing these effects, whereas the high dimensionality in purely kinetic models render them practically intractable for most relevant purposes.

        We consider a $\delta\!f$ decomposition model, with a macroscopic fluid background and microscopic kinetic correction, both fully coupled to each other. A similar manner of discretization is proposed to that used in the recent \texttt{STRUPHY} code \cite{Holderied_Possanner_Wang_2021, Holderied_2022, Li_et_al_2023} with a finite-element model for the background and a pseudo-particle/PiC model for the correction.

        The fluid background satisfies the full, non-linear, resistive, compressible, Hall MHD equations. \cite{Laakmann_Hu_Farrell_2022} introduces finite-element(-in-space) implicit timesteppers for the incompressible analogue to this system with structure-preserving (SP) properties in the ideal case, alongside parameter-robust preconditioners. We show that these timesteppers can derive from a finite-element-in-time (FET) (and finite-element-in-space) interpretation. The benefits of this reformulation are discussed, including the derivation of timesteppers that are higher order in time, and the quantifiable dissipative SP properties in the non-ideal, resistive case.
        
        We discuss possible options for extending this FET approach to timesteppers for the compressible case.

        The kinetic corrections satisfy linearized Boltzmann equations. Using a Lénard--Bernstein collision operator, these take Fokker--Planck-like forms \cite{Fokker_1914, Planck_1917} wherein pseudo-particles in the numerical model obey the neoclassical transport equations, with particle-independent Brownian drift terms. This offers a rigorous methodology for incorporating collisions into the particle transport model, without coupling the equations of motions for each particle.
        
        Works by Chen, Chacón et al. \cite{Chen_Chacón_Barnes_2011, Chacón_Chen_Barnes_2013, Chen_Chacón_2014, Chen_Chacón_2015} have developed structure-preserving particle pushers for neoclassical transport in the Vlasov equations, derived from Crank--Nicolson integrators. We show these too can can derive from a FET interpretation, similarly offering potential extensions to higher-order-in-time particle pushers. The FET formulation is used also to consider how the stochastic drift terms can be incorporated into the pushers. Stochastic gyrokinetic expansions are also discussed.

        Different options for the numerical implementation of these schemes are considered.

        Due to the efficacy of FET in the development of SP timesteppers for both the fluid and kinetic component, we hope this approach will prove effective in the future for developing SP timesteppers for the full hybrid model. We hope this will give us the opportunity to incorporate previously inaccessible kinetic effects into the highly effective, modern, finite-element MHD models.
    \end{abstract}
    
    
    \newpage
    \tableofcontents
    
    
    \newpage
    \pagenumbering{arabic}
    %\linenumbers\renewcommand\thelinenumber{\color{black!50}\arabic{linenumber}}
            \input{0 - introduction/main.tex}
        \part{Research}
            \input{1 - low-noise PiC models/main.tex}
            \input{2 - kinetic component/main.tex}
            \input{3 - fluid component/main.tex}
            \input{4 - numerical implementation/main.tex}
        \part{Project Overview}
            \input{5 - research plan/main.tex}
            \input{6 - summary/main.tex}
    
    
    %\section{}
    \newpage
    \pagenumbering{gobble}
        \printbibliography


    \newpage
    \pagenumbering{roman}
    \appendix
        \part{Appendices}
            \input{8 - Hilbert complexes/main.tex}
            \input{9 - weak conservation proofs/main.tex}
\end{document}

            \documentclass[12pt, a4paper]{report}

\input{template/main.tex}

\title{\BA{Title in Progress...}}
\author{Boris Andrews}
\affil{Mathematical Institute, University of Oxford}
\date{\today}


\begin{document}
    \pagenumbering{gobble}
    \maketitle
    
    
    \begin{abstract}
        Magnetic confinement reactors---in particular tokamaks---offer one of the most promising options for achieving practical nuclear fusion, with the potential to provide virtually limitless, clean energy. The theoretical and numerical modeling of tokamak plasmas is simultaneously an essential component of effective reactor design, and a great research barrier. Tokamak operational conditions exhibit comparatively low Knudsen numbers. Kinetic effects, including kinetic waves and instabilities, Landau damping, bump-on-tail instabilities and more, are therefore highly influential in tokamak plasma dynamics. Purely fluid models are inherently incapable of capturing these effects, whereas the high dimensionality in purely kinetic models render them practically intractable for most relevant purposes.

        We consider a $\delta\!f$ decomposition model, with a macroscopic fluid background and microscopic kinetic correction, both fully coupled to each other. A similar manner of discretization is proposed to that used in the recent \texttt{STRUPHY} code \cite{Holderied_Possanner_Wang_2021, Holderied_2022, Li_et_al_2023} with a finite-element model for the background and a pseudo-particle/PiC model for the correction.

        The fluid background satisfies the full, non-linear, resistive, compressible, Hall MHD equations. \cite{Laakmann_Hu_Farrell_2022} introduces finite-element(-in-space) implicit timesteppers for the incompressible analogue to this system with structure-preserving (SP) properties in the ideal case, alongside parameter-robust preconditioners. We show that these timesteppers can derive from a finite-element-in-time (FET) (and finite-element-in-space) interpretation. The benefits of this reformulation are discussed, including the derivation of timesteppers that are higher order in time, and the quantifiable dissipative SP properties in the non-ideal, resistive case.
        
        We discuss possible options for extending this FET approach to timesteppers for the compressible case.

        The kinetic corrections satisfy linearized Boltzmann equations. Using a Lénard--Bernstein collision operator, these take Fokker--Planck-like forms \cite{Fokker_1914, Planck_1917} wherein pseudo-particles in the numerical model obey the neoclassical transport equations, with particle-independent Brownian drift terms. This offers a rigorous methodology for incorporating collisions into the particle transport model, without coupling the equations of motions for each particle.
        
        Works by Chen, Chacón et al. \cite{Chen_Chacón_Barnes_2011, Chacón_Chen_Barnes_2013, Chen_Chacón_2014, Chen_Chacón_2015} have developed structure-preserving particle pushers for neoclassical transport in the Vlasov equations, derived from Crank--Nicolson integrators. We show these too can can derive from a FET interpretation, similarly offering potential extensions to higher-order-in-time particle pushers. The FET formulation is used also to consider how the stochastic drift terms can be incorporated into the pushers. Stochastic gyrokinetic expansions are also discussed.

        Different options for the numerical implementation of these schemes are considered.

        Due to the efficacy of FET in the development of SP timesteppers for both the fluid and kinetic component, we hope this approach will prove effective in the future for developing SP timesteppers for the full hybrid model. We hope this will give us the opportunity to incorporate previously inaccessible kinetic effects into the highly effective, modern, finite-element MHD models.
    \end{abstract}
    
    
    \newpage
    \tableofcontents
    
    
    \newpage
    \pagenumbering{arabic}
    %\linenumbers\renewcommand\thelinenumber{\color{black!50}\arabic{linenumber}}
            \input{0 - introduction/main.tex}
        \part{Research}
            \input{1 - low-noise PiC models/main.tex}
            \input{2 - kinetic component/main.tex}
            \input{3 - fluid component/main.tex}
            \input{4 - numerical implementation/main.tex}
        \part{Project Overview}
            \input{5 - research plan/main.tex}
            \input{6 - summary/main.tex}
    
    
    %\section{}
    \newpage
    \pagenumbering{gobble}
        \printbibliography


    \newpage
    \pagenumbering{roman}
    \appendix
        \part{Appendices}
            \input{8 - Hilbert complexes/main.tex}
            \input{9 - weak conservation proofs/main.tex}
\end{document}

\end{document}

            \documentclass[12pt, a4paper]{report}

\documentclass[12pt, a4paper]{report}

\input{template/main.tex}

\title{\BA{Title in Progress...}}
\author{Boris Andrews}
\affil{Mathematical Institute, University of Oxford}
\date{\today}


\begin{document}
    \pagenumbering{gobble}
    \maketitle
    
    
    \begin{abstract}
        Magnetic confinement reactors---in particular tokamaks---offer one of the most promising options for achieving practical nuclear fusion, with the potential to provide virtually limitless, clean energy. The theoretical and numerical modeling of tokamak plasmas is simultaneously an essential component of effective reactor design, and a great research barrier. Tokamak operational conditions exhibit comparatively low Knudsen numbers. Kinetic effects, including kinetic waves and instabilities, Landau damping, bump-on-tail instabilities and more, are therefore highly influential in tokamak plasma dynamics. Purely fluid models are inherently incapable of capturing these effects, whereas the high dimensionality in purely kinetic models render them practically intractable for most relevant purposes.

        We consider a $\delta\!f$ decomposition model, with a macroscopic fluid background and microscopic kinetic correction, both fully coupled to each other. A similar manner of discretization is proposed to that used in the recent \texttt{STRUPHY} code \cite{Holderied_Possanner_Wang_2021, Holderied_2022, Li_et_al_2023} with a finite-element model for the background and a pseudo-particle/PiC model for the correction.

        The fluid background satisfies the full, non-linear, resistive, compressible, Hall MHD equations. \cite{Laakmann_Hu_Farrell_2022} introduces finite-element(-in-space) implicit timesteppers for the incompressible analogue to this system with structure-preserving (SP) properties in the ideal case, alongside parameter-robust preconditioners. We show that these timesteppers can derive from a finite-element-in-time (FET) (and finite-element-in-space) interpretation. The benefits of this reformulation are discussed, including the derivation of timesteppers that are higher order in time, and the quantifiable dissipative SP properties in the non-ideal, resistive case.
        
        We discuss possible options for extending this FET approach to timesteppers for the compressible case.

        The kinetic corrections satisfy linearized Boltzmann equations. Using a Lénard--Bernstein collision operator, these take Fokker--Planck-like forms \cite{Fokker_1914, Planck_1917} wherein pseudo-particles in the numerical model obey the neoclassical transport equations, with particle-independent Brownian drift terms. This offers a rigorous methodology for incorporating collisions into the particle transport model, without coupling the equations of motions for each particle.
        
        Works by Chen, Chacón et al. \cite{Chen_Chacón_Barnes_2011, Chacón_Chen_Barnes_2013, Chen_Chacón_2014, Chen_Chacón_2015} have developed structure-preserving particle pushers for neoclassical transport in the Vlasov equations, derived from Crank--Nicolson integrators. We show these too can can derive from a FET interpretation, similarly offering potential extensions to higher-order-in-time particle pushers. The FET formulation is used also to consider how the stochastic drift terms can be incorporated into the pushers. Stochastic gyrokinetic expansions are also discussed.

        Different options for the numerical implementation of these schemes are considered.

        Due to the efficacy of FET in the development of SP timesteppers for both the fluid and kinetic component, we hope this approach will prove effective in the future for developing SP timesteppers for the full hybrid model. We hope this will give us the opportunity to incorporate previously inaccessible kinetic effects into the highly effective, modern, finite-element MHD models.
    \end{abstract}
    
    
    \newpage
    \tableofcontents
    
    
    \newpage
    \pagenumbering{arabic}
    %\linenumbers\renewcommand\thelinenumber{\color{black!50}\arabic{linenumber}}
            \input{0 - introduction/main.tex}
        \part{Research}
            \input{1 - low-noise PiC models/main.tex}
            \input{2 - kinetic component/main.tex}
            \input{3 - fluid component/main.tex}
            \input{4 - numerical implementation/main.tex}
        \part{Project Overview}
            \input{5 - research plan/main.tex}
            \input{6 - summary/main.tex}
    
    
    %\section{}
    \newpage
    \pagenumbering{gobble}
        \printbibliography


    \newpage
    \pagenumbering{roman}
    \appendix
        \part{Appendices}
            \input{8 - Hilbert complexes/main.tex}
            \input{9 - weak conservation proofs/main.tex}
\end{document}


\title{\BA{Title in Progress...}}
\author{Boris Andrews}
\affil{Mathematical Institute, University of Oxford}
\date{\today}


\begin{document}
    \pagenumbering{gobble}
    \maketitle
    
    
    \begin{abstract}
        Magnetic confinement reactors---in particular tokamaks---offer one of the most promising options for achieving practical nuclear fusion, with the potential to provide virtually limitless, clean energy. The theoretical and numerical modeling of tokamak plasmas is simultaneously an essential component of effective reactor design, and a great research barrier. Tokamak operational conditions exhibit comparatively low Knudsen numbers. Kinetic effects, including kinetic waves and instabilities, Landau damping, bump-on-tail instabilities and more, are therefore highly influential in tokamak plasma dynamics. Purely fluid models are inherently incapable of capturing these effects, whereas the high dimensionality in purely kinetic models render them practically intractable for most relevant purposes.

        We consider a $\delta\!f$ decomposition model, with a macroscopic fluid background and microscopic kinetic correction, both fully coupled to each other. A similar manner of discretization is proposed to that used in the recent \texttt{STRUPHY} code \cite{Holderied_Possanner_Wang_2021, Holderied_2022, Li_et_al_2023} with a finite-element model for the background and a pseudo-particle/PiC model for the correction.

        The fluid background satisfies the full, non-linear, resistive, compressible, Hall MHD equations. \cite{Laakmann_Hu_Farrell_2022} introduces finite-element(-in-space) implicit timesteppers for the incompressible analogue to this system with structure-preserving (SP) properties in the ideal case, alongside parameter-robust preconditioners. We show that these timesteppers can derive from a finite-element-in-time (FET) (and finite-element-in-space) interpretation. The benefits of this reformulation are discussed, including the derivation of timesteppers that are higher order in time, and the quantifiable dissipative SP properties in the non-ideal, resistive case.
        
        We discuss possible options for extending this FET approach to timesteppers for the compressible case.

        The kinetic corrections satisfy linearized Boltzmann equations. Using a Lénard--Bernstein collision operator, these take Fokker--Planck-like forms \cite{Fokker_1914, Planck_1917} wherein pseudo-particles in the numerical model obey the neoclassical transport equations, with particle-independent Brownian drift terms. This offers a rigorous methodology for incorporating collisions into the particle transport model, without coupling the equations of motions for each particle.
        
        Works by Chen, Chacón et al. \cite{Chen_Chacón_Barnes_2011, Chacón_Chen_Barnes_2013, Chen_Chacón_2014, Chen_Chacón_2015} have developed structure-preserving particle pushers for neoclassical transport in the Vlasov equations, derived from Crank--Nicolson integrators. We show these too can can derive from a FET interpretation, similarly offering potential extensions to higher-order-in-time particle pushers. The FET formulation is used also to consider how the stochastic drift terms can be incorporated into the pushers. Stochastic gyrokinetic expansions are also discussed.

        Different options for the numerical implementation of these schemes are considered.

        Due to the efficacy of FET in the development of SP timesteppers for both the fluid and kinetic component, we hope this approach will prove effective in the future for developing SP timesteppers for the full hybrid model. We hope this will give us the opportunity to incorporate previously inaccessible kinetic effects into the highly effective, modern, finite-element MHD models.
    \end{abstract}
    
    
    \newpage
    \tableofcontents
    
    
    \newpage
    \pagenumbering{arabic}
    %\linenumbers\renewcommand\thelinenumber{\color{black!50}\arabic{linenumber}}
            \documentclass[12pt, a4paper]{report}

\input{template/main.tex}

\title{\BA{Title in Progress...}}
\author{Boris Andrews}
\affil{Mathematical Institute, University of Oxford}
\date{\today}


\begin{document}
    \pagenumbering{gobble}
    \maketitle
    
    
    \begin{abstract}
        Magnetic confinement reactors---in particular tokamaks---offer one of the most promising options for achieving practical nuclear fusion, with the potential to provide virtually limitless, clean energy. The theoretical and numerical modeling of tokamak plasmas is simultaneously an essential component of effective reactor design, and a great research barrier. Tokamak operational conditions exhibit comparatively low Knudsen numbers. Kinetic effects, including kinetic waves and instabilities, Landau damping, bump-on-tail instabilities and more, are therefore highly influential in tokamak plasma dynamics. Purely fluid models are inherently incapable of capturing these effects, whereas the high dimensionality in purely kinetic models render them practically intractable for most relevant purposes.

        We consider a $\delta\!f$ decomposition model, with a macroscopic fluid background and microscopic kinetic correction, both fully coupled to each other. A similar manner of discretization is proposed to that used in the recent \texttt{STRUPHY} code \cite{Holderied_Possanner_Wang_2021, Holderied_2022, Li_et_al_2023} with a finite-element model for the background and a pseudo-particle/PiC model for the correction.

        The fluid background satisfies the full, non-linear, resistive, compressible, Hall MHD equations. \cite{Laakmann_Hu_Farrell_2022} introduces finite-element(-in-space) implicit timesteppers for the incompressible analogue to this system with structure-preserving (SP) properties in the ideal case, alongside parameter-robust preconditioners. We show that these timesteppers can derive from a finite-element-in-time (FET) (and finite-element-in-space) interpretation. The benefits of this reformulation are discussed, including the derivation of timesteppers that are higher order in time, and the quantifiable dissipative SP properties in the non-ideal, resistive case.
        
        We discuss possible options for extending this FET approach to timesteppers for the compressible case.

        The kinetic corrections satisfy linearized Boltzmann equations. Using a Lénard--Bernstein collision operator, these take Fokker--Planck-like forms \cite{Fokker_1914, Planck_1917} wherein pseudo-particles in the numerical model obey the neoclassical transport equations, with particle-independent Brownian drift terms. This offers a rigorous methodology for incorporating collisions into the particle transport model, without coupling the equations of motions for each particle.
        
        Works by Chen, Chacón et al. \cite{Chen_Chacón_Barnes_2011, Chacón_Chen_Barnes_2013, Chen_Chacón_2014, Chen_Chacón_2015} have developed structure-preserving particle pushers for neoclassical transport in the Vlasov equations, derived from Crank--Nicolson integrators. We show these too can can derive from a FET interpretation, similarly offering potential extensions to higher-order-in-time particle pushers. The FET formulation is used also to consider how the stochastic drift terms can be incorporated into the pushers. Stochastic gyrokinetic expansions are also discussed.

        Different options for the numerical implementation of these schemes are considered.

        Due to the efficacy of FET in the development of SP timesteppers for both the fluid and kinetic component, we hope this approach will prove effective in the future for developing SP timesteppers for the full hybrid model. We hope this will give us the opportunity to incorporate previously inaccessible kinetic effects into the highly effective, modern, finite-element MHD models.
    \end{abstract}
    
    
    \newpage
    \tableofcontents
    
    
    \newpage
    \pagenumbering{arabic}
    %\linenumbers\renewcommand\thelinenumber{\color{black!50}\arabic{linenumber}}
            \input{0 - introduction/main.tex}
        \part{Research}
            \input{1 - low-noise PiC models/main.tex}
            \input{2 - kinetic component/main.tex}
            \input{3 - fluid component/main.tex}
            \input{4 - numerical implementation/main.tex}
        \part{Project Overview}
            \input{5 - research plan/main.tex}
            \input{6 - summary/main.tex}
    
    
    %\section{}
    \newpage
    \pagenumbering{gobble}
        \printbibliography


    \newpage
    \pagenumbering{roman}
    \appendix
        \part{Appendices}
            \input{8 - Hilbert complexes/main.tex}
            \input{9 - weak conservation proofs/main.tex}
\end{document}

        \part{Research}
            \documentclass[12pt, a4paper]{report}

\input{template/main.tex}

\title{\BA{Title in Progress...}}
\author{Boris Andrews}
\affil{Mathematical Institute, University of Oxford}
\date{\today}


\begin{document}
    \pagenumbering{gobble}
    \maketitle
    
    
    \begin{abstract}
        Magnetic confinement reactors---in particular tokamaks---offer one of the most promising options for achieving practical nuclear fusion, with the potential to provide virtually limitless, clean energy. The theoretical and numerical modeling of tokamak plasmas is simultaneously an essential component of effective reactor design, and a great research barrier. Tokamak operational conditions exhibit comparatively low Knudsen numbers. Kinetic effects, including kinetic waves and instabilities, Landau damping, bump-on-tail instabilities and more, are therefore highly influential in tokamak plasma dynamics. Purely fluid models are inherently incapable of capturing these effects, whereas the high dimensionality in purely kinetic models render them practically intractable for most relevant purposes.

        We consider a $\delta\!f$ decomposition model, with a macroscopic fluid background and microscopic kinetic correction, both fully coupled to each other. A similar manner of discretization is proposed to that used in the recent \texttt{STRUPHY} code \cite{Holderied_Possanner_Wang_2021, Holderied_2022, Li_et_al_2023} with a finite-element model for the background and a pseudo-particle/PiC model for the correction.

        The fluid background satisfies the full, non-linear, resistive, compressible, Hall MHD equations. \cite{Laakmann_Hu_Farrell_2022} introduces finite-element(-in-space) implicit timesteppers for the incompressible analogue to this system with structure-preserving (SP) properties in the ideal case, alongside parameter-robust preconditioners. We show that these timesteppers can derive from a finite-element-in-time (FET) (and finite-element-in-space) interpretation. The benefits of this reformulation are discussed, including the derivation of timesteppers that are higher order in time, and the quantifiable dissipative SP properties in the non-ideal, resistive case.
        
        We discuss possible options for extending this FET approach to timesteppers for the compressible case.

        The kinetic corrections satisfy linearized Boltzmann equations. Using a Lénard--Bernstein collision operator, these take Fokker--Planck-like forms \cite{Fokker_1914, Planck_1917} wherein pseudo-particles in the numerical model obey the neoclassical transport equations, with particle-independent Brownian drift terms. This offers a rigorous methodology for incorporating collisions into the particle transport model, without coupling the equations of motions for each particle.
        
        Works by Chen, Chacón et al. \cite{Chen_Chacón_Barnes_2011, Chacón_Chen_Barnes_2013, Chen_Chacón_2014, Chen_Chacón_2015} have developed structure-preserving particle pushers for neoclassical transport in the Vlasov equations, derived from Crank--Nicolson integrators. We show these too can can derive from a FET interpretation, similarly offering potential extensions to higher-order-in-time particle pushers. The FET formulation is used also to consider how the stochastic drift terms can be incorporated into the pushers. Stochastic gyrokinetic expansions are also discussed.

        Different options for the numerical implementation of these schemes are considered.

        Due to the efficacy of FET in the development of SP timesteppers for both the fluid and kinetic component, we hope this approach will prove effective in the future for developing SP timesteppers for the full hybrid model. We hope this will give us the opportunity to incorporate previously inaccessible kinetic effects into the highly effective, modern, finite-element MHD models.
    \end{abstract}
    
    
    \newpage
    \tableofcontents
    
    
    \newpage
    \pagenumbering{arabic}
    %\linenumbers\renewcommand\thelinenumber{\color{black!50}\arabic{linenumber}}
            \input{0 - introduction/main.tex}
        \part{Research}
            \input{1 - low-noise PiC models/main.tex}
            \input{2 - kinetic component/main.tex}
            \input{3 - fluid component/main.tex}
            \input{4 - numerical implementation/main.tex}
        \part{Project Overview}
            \input{5 - research plan/main.tex}
            \input{6 - summary/main.tex}
    
    
    %\section{}
    \newpage
    \pagenumbering{gobble}
        \printbibliography


    \newpage
    \pagenumbering{roman}
    \appendix
        \part{Appendices}
            \input{8 - Hilbert complexes/main.tex}
            \input{9 - weak conservation proofs/main.tex}
\end{document}

            \documentclass[12pt, a4paper]{report}

\input{template/main.tex}

\title{\BA{Title in Progress...}}
\author{Boris Andrews}
\affil{Mathematical Institute, University of Oxford}
\date{\today}


\begin{document}
    \pagenumbering{gobble}
    \maketitle
    
    
    \begin{abstract}
        Magnetic confinement reactors---in particular tokamaks---offer one of the most promising options for achieving practical nuclear fusion, with the potential to provide virtually limitless, clean energy. The theoretical and numerical modeling of tokamak plasmas is simultaneously an essential component of effective reactor design, and a great research barrier. Tokamak operational conditions exhibit comparatively low Knudsen numbers. Kinetic effects, including kinetic waves and instabilities, Landau damping, bump-on-tail instabilities and more, are therefore highly influential in tokamak plasma dynamics. Purely fluid models are inherently incapable of capturing these effects, whereas the high dimensionality in purely kinetic models render them practically intractable for most relevant purposes.

        We consider a $\delta\!f$ decomposition model, with a macroscopic fluid background and microscopic kinetic correction, both fully coupled to each other. A similar manner of discretization is proposed to that used in the recent \texttt{STRUPHY} code \cite{Holderied_Possanner_Wang_2021, Holderied_2022, Li_et_al_2023} with a finite-element model for the background and a pseudo-particle/PiC model for the correction.

        The fluid background satisfies the full, non-linear, resistive, compressible, Hall MHD equations. \cite{Laakmann_Hu_Farrell_2022} introduces finite-element(-in-space) implicit timesteppers for the incompressible analogue to this system with structure-preserving (SP) properties in the ideal case, alongside parameter-robust preconditioners. We show that these timesteppers can derive from a finite-element-in-time (FET) (and finite-element-in-space) interpretation. The benefits of this reformulation are discussed, including the derivation of timesteppers that are higher order in time, and the quantifiable dissipative SP properties in the non-ideal, resistive case.
        
        We discuss possible options for extending this FET approach to timesteppers for the compressible case.

        The kinetic corrections satisfy linearized Boltzmann equations. Using a Lénard--Bernstein collision operator, these take Fokker--Planck-like forms \cite{Fokker_1914, Planck_1917} wherein pseudo-particles in the numerical model obey the neoclassical transport equations, with particle-independent Brownian drift terms. This offers a rigorous methodology for incorporating collisions into the particle transport model, without coupling the equations of motions for each particle.
        
        Works by Chen, Chacón et al. \cite{Chen_Chacón_Barnes_2011, Chacón_Chen_Barnes_2013, Chen_Chacón_2014, Chen_Chacón_2015} have developed structure-preserving particle pushers for neoclassical transport in the Vlasov equations, derived from Crank--Nicolson integrators. We show these too can can derive from a FET interpretation, similarly offering potential extensions to higher-order-in-time particle pushers. The FET formulation is used also to consider how the stochastic drift terms can be incorporated into the pushers. Stochastic gyrokinetic expansions are also discussed.

        Different options for the numerical implementation of these schemes are considered.

        Due to the efficacy of FET in the development of SP timesteppers for both the fluid and kinetic component, we hope this approach will prove effective in the future for developing SP timesteppers for the full hybrid model. We hope this will give us the opportunity to incorporate previously inaccessible kinetic effects into the highly effective, modern, finite-element MHD models.
    \end{abstract}
    
    
    \newpage
    \tableofcontents
    
    
    \newpage
    \pagenumbering{arabic}
    %\linenumbers\renewcommand\thelinenumber{\color{black!50}\arabic{linenumber}}
            \input{0 - introduction/main.tex}
        \part{Research}
            \input{1 - low-noise PiC models/main.tex}
            \input{2 - kinetic component/main.tex}
            \input{3 - fluid component/main.tex}
            \input{4 - numerical implementation/main.tex}
        \part{Project Overview}
            \input{5 - research plan/main.tex}
            \input{6 - summary/main.tex}
    
    
    %\section{}
    \newpage
    \pagenumbering{gobble}
        \printbibliography


    \newpage
    \pagenumbering{roman}
    \appendix
        \part{Appendices}
            \input{8 - Hilbert complexes/main.tex}
            \input{9 - weak conservation proofs/main.tex}
\end{document}

            \documentclass[12pt, a4paper]{report}

\input{template/main.tex}

\title{\BA{Title in Progress...}}
\author{Boris Andrews}
\affil{Mathematical Institute, University of Oxford}
\date{\today}


\begin{document}
    \pagenumbering{gobble}
    \maketitle
    
    
    \begin{abstract}
        Magnetic confinement reactors---in particular tokamaks---offer one of the most promising options for achieving practical nuclear fusion, with the potential to provide virtually limitless, clean energy. The theoretical and numerical modeling of tokamak plasmas is simultaneously an essential component of effective reactor design, and a great research barrier. Tokamak operational conditions exhibit comparatively low Knudsen numbers. Kinetic effects, including kinetic waves and instabilities, Landau damping, bump-on-tail instabilities and more, are therefore highly influential in tokamak plasma dynamics. Purely fluid models are inherently incapable of capturing these effects, whereas the high dimensionality in purely kinetic models render them practically intractable for most relevant purposes.

        We consider a $\delta\!f$ decomposition model, with a macroscopic fluid background and microscopic kinetic correction, both fully coupled to each other. A similar manner of discretization is proposed to that used in the recent \texttt{STRUPHY} code \cite{Holderied_Possanner_Wang_2021, Holderied_2022, Li_et_al_2023} with a finite-element model for the background and a pseudo-particle/PiC model for the correction.

        The fluid background satisfies the full, non-linear, resistive, compressible, Hall MHD equations. \cite{Laakmann_Hu_Farrell_2022} introduces finite-element(-in-space) implicit timesteppers for the incompressible analogue to this system with structure-preserving (SP) properties in the ideal case, alongside parameter-robust preconditioners. We show that these timesteppers can derive from a finite-element-in-time (FET) (and finite-element-in-space) interpretation. The benefits of this reformulation are discussed, including the derivation of timesteppers that are higher order in time, and the quantifiable dissipative SP properties in the non-ideal, resistive case.
        
        We discuss possible options for extending this FET approach to timesteppers for the compressible case.

        The kinetic corrections satisfy linearized Boltzmann equations. Using a Lénard--Bernstein collision operator, these take Fokker--Planck-like forms \cite{Fokker_1914, Planck_1917} wherein pseudo-particles in the numerical model obey the neoclassical transport equations, with particle-independent Brownian drift terms. This offers a rigorous methodology for incorporating collisions into the particle transport model, without coupling the equations of motions for each particle.
        
        Works by Chen, Chacón et al. \cite{Chen_Chacón_Barnes_2011, Chacón_Chen_Barnes_2013, Chen_Chacón_2014, Chen_Chacón_2015} have developed structure-preserving particle pushers for neoclassical transport in the Vlasov equations, derived from Crank--Nicolson integrators. We show these too can can derive from a FET interpretation, similarly offering potential extensions to higher-order-in-time particle pushers. The FET formulation is used also to consider how the stochastic drift terms can be incorporated into the pushers. Stochastic gyrokinetic expansions are also discussed.

        Different options for the numerical implementation of these schemes are considered.

        Due to the efficacy of FET in the development of SP timesteppers for both the fluid and kinetic component, we hope this approach will prove effective in the future for developing SP timesteppers for the full hybrid model. We hope this will give us the opportunity to incorporate previously inaccessible kinetic effects into the highly effective, modern, finite-element MHD models.
    \end{abstract}
    
    
    \newpage
    \tableofcontents
    
    
    \newpage
    \pagenumbering{arabic}
    %\linenumbers\renewcommand\thelinenumber{\color{black!50}\arabic{linenumber}}
            \input{0 - introduction/main.tex}
        \part{Research}
            \input{1 - low-noise PiC models/main.tex}
            \input{2 - kinetic component/main.tex}
            \input{3 - fluid component/main.tex}
            \input{4 - numerical implementation/main.tex}
        \part{Project Overview}
            \input{5 - research plan/main.tex}
            \input{6 - summary/main.tex}
    
    
    %\section{}
    \newpage
    \pagenumbering{gobble}
        \printbibliography


    \newpage
    \pagenumbering{roman}
    \appendix
        \part{Appendices}
            \input{8 - Hilbert complexes/main.tex}
            \input{9 - weak conservation proofs/main.tex}
\end{document}

            \documentclass[12pt, a4paper]{report}

\input{template/main.tex}

\title{\BA{Title in Progress...}}
\author{Boris Andrews}
\affil{Mathematical Institute, University of Oxford}
\date{\today}


\begin{document}
    \pagenumbering{gobble}
    \maketitle
    
    
    \begin{abstract}
        Magnetic confinement reactors---in particular tokamaks---offer one of the most promising options for achieving practical nuclear fusion, with the potential to provide virtually limitless, clean energy. The theoretical and numerical modeling of tokamak plasmas is simultaneously an essential component of effective reactor design, and a great research barrier. Tokamak operational conditions exhibit comparatively low Knudsen numbers. Kinetic effects, including kinetic waves and instabilities, Landau damping, bump-on-tail instabilities and more, are therefore highly influential in tokamak plasma dynamics. Purely fluid models are inherently incapable of capturing these effects, whereas the high dimensionality in purely kinetic models render them practically intractable for most relevant purposes.

        We consider a $\delta\!f$ decomposition model, with a macroscopic fluid background and microscopic kinetic correction, both fully coupled to each other. A similar manner of discretization is proposed to that used in the recent \texttt{STRUPHY} code \cite{Holderied_Possanner_Wang_2021, Holderied_2022, Li_et_al_2023} with a finite-element model for the background and a pseudo-particle/PiC model for the correction.

        The fluid background satisfies the full, non-linear, resistive, compressible, Hall MHD equations. \cite{Laakmann_Hu_Farrell_2022} introduces finite-element(-in-space) implicit timesteppers for the incompressible analogue to this system with structure-preserving (SP) properties in the ideal case, alongside parameter-robust preconditioners. We show that these timesteppers can derive from a finite-element-in-time (FET) (and finite-element-in-space) interpretation. The benefits of this reformulation are discussed, including the derivation of timesteppers that are higher order in time, and the quantifiable dissipative SP properties in the non-ideal, resistive case.
        
        We discuss possible options for extending this FET approach to timesteppers for the compressible case.

        The kinetic corrections satisfy linearized Boltzmann equations. Using a Lénard--Bernstein collision operator, these take Fokker--Planck-like forms \cite{Fokker_1914, Planck_1917} wherein pseudo-particles in the numerical model obey the neoclassical transport equations, with particle-independent Brownian drift terms. This offers a rigorous methodology for incorporating collisions into the particle transport model, without coupling the equations of motions for each particle.
        
        Works by Chen, Chacón et al. \cite{Chen_Chacón_Barnes_2011, Chacón_Chen_Barnes_2013, Chen_Chacón_2014, Chen_Chacón_2015} have developed structure-preserving particle pushers for neoclassical transport in the Vlasov equations, derived from Crank--Nicolson integrators. We show these too can can derive from a FET interpretation, similarly offering potential extensions to higher-order-in-time particle pushers. The FET formulation is used also to consider how the stochastic drift terms can be incorporated into the pushers. Stochastic gyrokinetic expansions are also discussed.

        Different options for the numerical implementation of these schemes are considered.

        Due to the efficacy of FET in the development of SP timesteppers for both the fluid and kinetic component, we hope this approach will prove effective in the future for developing SP timesteppers for the full hybrid model. We hope this will give us the opportunity to incorporate previously inaccessible kinetic effects into the highly effective, modern, finite-element MHD models.
    \end{abstract}
    
    
    \newpage
    \tableofcontents
    
    
    \newpage
    \pagenumbering{arabic}
    %\linenumbers\renewcommand\thelinenumber{\color{black!50}\arabic{linenumber}}
            \input{0 - introduction/main.tex}
        \part{Research}
            \input{1 - low-noise PiC models/main.tex}
            \input{2 - kinetic component/main.tex}
            \input{3 - fluid component/main.tex}
            \input{4 - numerical implementation/main.tex}
        \part{Project Overview}
            \input{5 - research plan/main.tex}
            \input{6 - summary/main.tex}
    
    
    %\section{}
    \newpage
    \pagenumbering{gobble}
        \printbibliography


    \newpage
    \pagenumbering{roman}
    \appendix
        \part{Appendices}
            \input{8 - Hilbert complexes/main.tex}
            \input{9 - weak conservation proofs/main.tex}
\end{document}

        \part{Project Overview}
            \documentclass[12pt, a4paper]{report}

\input{template/main.tex}

\title{\BA{Title in Progress...}}
\author{Boris Andrews}
\affil{Mathematical Institute, University of Oxford}
\date{\today}


\begin{document}
    \pagenumbering{gobble}
    \maketitle
    
    
    \begin{abstract}
        Magnetic confinement reactors---in particular tokamaks---offer one of the most promising options for achieving practical nuclear fusion, with the potential to provide virtually limitless, clean energy. The theoretical and numerical modeling of tokamak plasmas is simultaneously an essential component of effective reactor design, and a great research barrier. Tokamak operational conditions exhibit comparatively low Knudsen numbers. Kinetic effects, including kinetic waves and instabilities, Landau damping, bump-on-tail instabilities and more, are therefore highly influential in tokamak plasma dynamics. Purely fluid models are inherently incapable of capturing these effects, whereas the high dimensionality in purely kinetic models render them practically intractable for most relevant purposes.

        We consider a $\delta\!f$ decomposition model, with a macroscopic fluid background and microscopic kinetic correction, both fully coupled to each other. A similar manner of discretization is proposed to that used in the recent \texttt{STRUPHY} code \cite{Holderied_Possanner_Wang_2021, Holderied_2022, Li_et_al_2023} with a finite-element model for the background and a pseudo-particle/PiC model for the correction.

        The fluid background satisfies the full, non-linear, resistive, compressible, Hall MHD equations. \cite{Laakmann_Hu_Farrell_2022} introduces finite-element(-in-space) implicit timesteppers for the incompressible analogue to this system with structure-preserving (SP) properties in the ideal case, alongside parameter-robust preconditioners. We show that these timesteppers can derive from a finite-element-in-time (FET) (and finite-element-in-space) interpretation. The benefits of this reformulation are discussed, including the derivation of timesteppers that are higher order in time, and the quantifiable dissipative SP properties in the non-ideal, resistive case.
        
        We discuss possible options for extending this FET approach to timesteppers for the compressible case.

        The kinetic corrections satisfy linearized Boltzmann equations. Using a Lénard--Bernstein collision operator, these take Fokker--Planck-like forms \cite{Fokker_1914, Planck_1917} wherein pseudo-particles in the numerical model obey the neoclassical transport equations, with particle-independent Brownian drift terms. This offers a rigorous methodology for incorporating collisions into the particle transport model, without coupling the equations of motions for each particle.
        
        Works by Chen, Chacón et al. \cite{Chen_Chacón_Barnes_2011, Chacón_Chen_Barnes_2013, Chen_Chacón_2014, Chen_Chacón_2015} have developed structure-preserving particle pushers for neoclassical transport in the Vlasov equations, derived from Crank--Nicolson integrators. We show these too can can derive from a FET interpretation, similarly offering potential extensions to higher-order-in-time particle pushers. The FET formulation is used also to consider how the stochastic drift terms can be incorporated into the pushers. Stochastic gyrokinetic expansions are also discussed.

        Different options for the numerical implementation of these schemes are considered.

        Due to the efficacy of FET in the development of SP timesteppers for both the fluid and kinetic component, we hope this approach will prove effective in the future for developing SP timesteppers for the full hybrid model. We hope this will give us the opportunity to incorporate previously inaccessible kinetic effects into the highly effective, modern, finite-element MHD models.
    \end{abstract}
    
    
    \newpage
    \tableofcontents
    
    
    \newpage
    \pagenumbering{arabic}
    %\linenumbers\renewcommand\thelinenumber{\color{black!50}\arabic{linenumber}}
            \input{0 - introduction/main.tex}
        \part{Research}
            \input{1 - low-noise PiC models/main.tex}
            \input{2 - kinetic component/main.tex}
            \input{3 - fluid component/main.tex}
            \input{4 - numerical implementation/main.tex}
        \part{Project Overview}
            \input{5 - research plan/main.tex}
            \input{6 - summary/main.tex}
    
    
    %\section{}
    \newpage
    \pagenumbering{gobble}
        \printbibliography


    \newpage
    \pagenumbering{roman}
    \appendix
        \part{Appendices}
            \input{8 - Hilbert complexes/main.tex}
            \input{9 - weak conservation proofs/main.tex}
\end{document}

            \documentclass[12pt, a4paper]{report}

\input{template/main.tex}

\title{\BA{Title in Progress...}}
\author{Boris Andrews}
\affil{Mathematical Institute, University of Oxford}
\date{\today}


\begin{document}
    \pagenumbering{gobble}
    \maketitle
    
    
    \begin{abstract}
        Magnetic confinement reactors---in particular tokamaks---offer one of the most promising options for achieving practical nuclear fusion, with the potential to provide virtually limitless, clean energy. The theoretical and numerical modeling of tokamak plasmas is simultaneously an essential component of effective reactor design, and a great research barrier. Tokamak operational conditions exhibit comparatively low Knudsen numbers. Kinetic effects, including kinetic waves and instabilities, Landau damping, bump-on-tail instabilities and more, are therefore highly influential in tokamak plasma dynamics. Purely fluid models are inherently incapable of capturing these effects, whereas the high dimensionality in purely kinetic models render them practically intractable for most relevant purposes.

        We consider a $\delta\!f$ decomposition model, with a macroscopic fluid background and microscopic kinetic correction, both fully coupled to each other. A similar manner of discretization is proposed to that used in the recent \texttt{STRUPHY} code \cite{Holderied_Possanner_Wang_2021, Holderied_2022, Li_et_al_2023} with a finite-element model for the background and a pseudo-particle/PiC model for the correction.

        The fluid background satisfies the full, non-linear, resistive, compressible, Hall MHD equations. \cite{Laakmann_Hu_Farrell_2022} introduces finite-element(-in-space) implicit timesteppers for the incompressible analogue to this system with structure-preserving (SP) properties in the ideal case, alongside parameter-robust preconditioners. We show that these timesteppers can derive from a finite-element-in-time (FET) (and finite-element-in-space) interpretation. The benefits of this reformulation are discussed, including the derivation of timesteppers that are higher order in time, and the quantifiable dissipative SP properties in the non-ideal, resistive case.
        
        We discuss possible options for extending this FET approach to timesteppers for the compressible case.

        The kinetic corrections satisfy linearized Boltzmann equations. Using a Lénard--Bernstein collision operator, these take Fokker--Planck-like forms \cite{Fokker_1914, Planck_1917} wherein pseudo-particles in the numerical model obey the neoclassical transport equations, with particle-independent Brownian drift terms. This offers a rigorous methodology for incorporating collisions into the particle transport model, without coupling the equations of motions for each particle.
        
        Works by Chen, Chacón et al. \cite{Chen_Chacón_Barnes_2011, Chacón_Chen_Barnes_2013, Chen_Chacón_2014, Chen_Chacón_2015} have developed structure-preserving particle pushers for neoclassical transport in the Vlasov equations, derived from Crank--Nicolson integrators. We show these too can can derive from a FET interpretation, similarly offering potential extensions to higher-order-in-time particle pushers. The FET formulation is used also to consider how the stochastic drift terms can be incorporated into the pushers. Stochastic gyrokinetic expansions are also discussed.

        Different options for the numerical implementation of these schemes are considered.

        Due to the efficacy of FET in the development of SP timesteppers for both the fluid and kinetic component, we hope this approach will prove effective in the future for developing SP timesteppers for the full hybrid model. We hope this will give us the opportunity to incorporate previously inaccessible kinetic effects into the highly effective, modern, finite-element MHD models.
    \end{abstract}
    
    
    \newpage
    \tableofcontents
    
    
    \newpage
    \pagenumbering{arabic}
    %\linenumbers\renewcommand\thelinenumber{\color{black!50}\arabic{linenumber}}
            \input{0 - introduction/main.tex}
        \part{Research}
            \input{1 - low-noise PiC models/main.tex}
            \input{2 - kinetic component/main.tex}
            \input{3 - fluid component/main.tex}
            \input{4 - numerical implementation/main.tex}
        \part{Project Overview}
            \input{5 - research plan/main.tex}
            \input{6 - summary/main.tex}
    
    
    %\section{}
    \newpage
    \pagenumbering{gobble}
        \printbibliography


    \newpage
    \pagenumbering{roman}
    \appendix
        \part{Appendices}
            \input{8 - Hilbert complexes/main.tex}
            \input{9 - weak conservation proofs/main.tex}
\end{document}

    
    
    %\section{}
    \newpage
    \pagenumbering{gobble}
        \printbibliography


    \newpage
    \pagenumbering{roman}
    \appendix
        \part{Appendices}
            \documentclass[12pt, a4paper]{report}

\input{template/main.tex}

\title{\BA{Title in Progress...}}
\author{Boris Andrews}
\affil{Mathematical Institute, University of Oxford}
\date{\today}


\begin{document}
    \pagenumbering{gobble}
    \maketitle
    
    
    \begin{abstract}
        Magnetic confinement reactors---in particular tokamaks---offer one of the most promising options for achieving practical nuclear fusion, with the potential to provide virtually limitless, clean energy. The theoretical and numerical modeling of tokamak plasmas is simultaneously an essential component of effective reactor design, and a great research barrier. Tokamak operational conditions exhibit comparatively low Knudsen numbers. Kinetic effects, including kinetic waves and instabilities, Landau damping, bump-on-tail instabilities and more, are therefore highly influential in tokamak plasma dynamics. Purely fluid models are inherently incapable of capturing these effects, whereas the high dimensionality in purely kinetic models render them practically intractable for most relevant purposes.

        We consider a $\delta\!f$ decomposition model, with a macroscopic fluid background and microscopic kinetic correction, both fully coupled to each other. A similar manner of discretization is proposed to that used in the recent \texttt{STRUPHY} code \cite{Holderied_Possanner_Wang_2021, Holderied_2022, Li_et_al_2023} with a finite-element model for the background and a pseudo-particle/PiC model for the correction.

        The fluid background satisfies the full, non-linear, resistive, compressible, Hall MHD equations. \cite{Laakmann_Hu_Farrell_2022} introduces finite-element(-in-space) implicit timesteppers for the incompressible analogue to this system with structure-preserving (SP) properties in the ideal case, alongside parameter-robust preconditioners. We show that these timesteppers can derive from a finite-element-in-time (FET) (and finite-element-in-space) interpretation. The benefits of this reformulation are discussed, including the derivation of timesteppers that are higher order in time, and the quantifiable dissipative SP properties in the non-ideal, resistive case.
        
        We discuss possible options for extending this FET approach to timesteppers for the compressible case.

        The kinetic corrections satisfy linearized Boltzmann equations. Using a Lénard--Bernstein collision operator, these take Fokker--Planck-like forms \cite{Fokker_1914, Planck_1917} wherein pseudo-particles in the numerical model obey the neoclassical transport equations, with particle-independent Brownian drift terms. This offers a rigorous methodology for incorporating collisions into the particle transport model, without coupling the equations of motions for each particle.
        
        Works by Chen, Chacón et al. \cite{Chen_Chacón_Barnes_2011, Chacón_Chen_Barnes_2013, Chen_Chacón_2014, Chen_Chacón_2015} have developed structure-preserving particle pushers for neoclassical transport in the Vlasov equations, derived from Crank--Nicolson integrators. We show these too can can derive from a FET interpretation, similarly offering potential extensions to higher-order-in-time particle pushers. The FET formulation is used also to consider how the stochastic drift terms can be incorporated into the pushers. Stochastic gyrokinetic expansions are also discussed.

        Different options for the numerical implementation of these schemes are considered.

        Due to the efficacy of FET in the development of SP timesteppers for both the fluid and kinetic component, we hope this approach will prove effective in the future for developing SP timesteppers for the full hybrid model. We hope this will give us the opportunity to incorporate previously inaccessible kinetic effects into the highly effective, modern, finite-element MHD models.
    \end{abstract}
    
    
    \newpage
    \tableofcontents
    
    
    \newpage
    \pagenumbering{arabic}
    %\linenumbers\renewcommand\thelinenumber{\color{black!50}\arabic{linenumber}}
            \input{0 - introduction/main.tex}
        \part{Research}
            \input{1 - low-noise PiC models/main.tex}
            \input{2 - kinetic component/main.tex}
            \input{3 - fluid component/main.tex}
            \input{4 - numerical implementation/main.tex}
        \part{Project Overview}
            \input{5 - research plan/main.tex}
            \input{6 - summary/main.tex}
    
    
    %\section{}
    \newpage
    \pagenumbering{gobble}
        \printbibliography


    \newpage
    \pagenumbering{roman}
    \appendix
        \part{Appendices}
            \input{8 - Hilbert complexes/main.tex}
            \input{9 - weak conservation proofs/main.tex}
\end{document}

            \documentclass[12pt, a4paper]{report}

\input{template/main.tex}

\title{\BA{Title in Progress...}}
\author{Boris Andrews}
\affil{Mathematical Institute, University of Oxford}
\date{\today}


\begin{document}
    \pagenumbering{gobble}
    \maketitle
    
    
    \begin{abstract}
        Magnetic confinement reactors---in particular tokamaks---offer one of the most promising options for achieving practical nuclear fusion, with the potential to provide virtually limitless, clean energy. The theoretical and numerical modeling of tokamak plasmas is simultaneously an essential component of effective reactor design, and a great research barrier. Tokamak operational conditions exhibit comparatively low Knudsen numbers. Kinetic effects, including kinetic waves and instabilities, Landau damping, bump-on-tail instabilities and more, are therefore highly influential in tokamak plasma dynamics. Purely fluid models are inherently incapable of capturing these effects, whereas the high dimensionality in purely kinetic models render them practically intractable for most relevant purposes.

        We consider a $\delta\!f$ decomposition model, with a macroscopic fluid background and microscopic kinetic correction, both fully coupled to each other. A similar manner of discretization is proposed to that used in the recent \texttt{STRUPHY} code \cite{Holderied_Possanner_Wang_2021, Holderied_2022, Li_et_al_2023} with a finite-element model for the background and a pseudo-particle/PiC model for the correction.

        The fluid background satisfies the full, non-linear, resistive, compressible, Hall MHD equations. \cite{Laakmann_Hu_Farrell_2022} introduces finite-element(-in-space) implicit timesteppers for the incompressible analogue to this system with structure-preserving (SP) properties in the ideal case, alongside parameter-robust preconditioners. We show that these timesteppers can derive from a finite-element-in-time (FET) (and finite-element-in-space) interpretation. The benefits of this reformulation are discussed, including the derivation of timesteppers that are higher order in time, and the quantifiable dissipative SP properties in the non-ideal, resistive case.
        
        We discuss possible options for extending this FET approach to timesteppers for the compressible case.

        The kinetic corrections satisfy linearized Boltzmann equations. Using a Lénard--Bernstein collision operator, these take Fokker--Planck-like forms \cite{Fokker_1914, Planck_1917} wherein pseudo-particles in the numerical model obey the neoclassical transport equations, with particle-independent Brownian drift terms. This offers a rigorous methodology for incorporating collisions into the particle transport model, without coupling the equations of motions for each particle.
        
        Works by Chen, Chacón et al. \cite{Chen_Chacón_Barnes_2011, Chacón_Chen_Barnes_2013, Chen_Chacón_2014, Chen_Chacón_2015} have developed structure-preserving particle pushers for neoclassical transport in the Vlasov equations, derived from Crank--Nicolson integrators. We show these too can can derive from a FET interpretation, similarly offering potential extensions to higher-order-in-time particle pushers. The FET formulation is used also to consider how the stochastic drift terms can be incorporated into the pushers. Stochastic gyrokinetic expansions are also discussed.

        Different options for the numerical implementation of these schemes are considered.

        Due to the efficacy of FET in the development of SP timesteppers for both the fluid and kinetic component, we hope this approach will prove effective in the future for developing SP timesteppers for the full hybrid model. We hope this will give us the opportunity to incorporate previously inaccessible kinetic effects into the highly effective, modern, finite-element MHD models.
    \end{abstract}
    
    
    \newpage
    \tableofcontents
    
    
    \newpage
    \pagenumbering{arabic}
    %\linenumbers\renewcommand\thelinenumber{\color{black!50}\arabic{linenumber}}
            \input{0 - introduction/main.tex}
        \part{Research}
            \input{1 - low-noise PiC models/main.tex}
            \input{2 - kinetic component/main.tex}
            \input{3 - fluid component/main.tex}
            \input{4 - numerical implementation/main.tex}
        \part{Project Overview}
            \input{5 - research plan/main.tex}
            \input{6 - summary/main.tex}
    
    
    %\section{}
    \newpage
    \pagenumbering{gobble}
        \printbibliography


    \newpage
    \pagenumbering{roman}
    \appendix
        \part{Appendices}
            \input{8 - Hilbert complexes/main.tex}
            \input{9 - weak conservation proofs/main.tex}
\end{document}

\end{document}

        \part{Project Overview}
            \documentclass[12pt, a4paper]{report}

\documentclass[12pt, a4paper]{report}

\input{template/main.tex}

\title{\BA{Title in Progress...}}
\author{Boris Andrews}
\affil{Mathematical Institute, University of Oxford}
\date{\today}


\begin{document}
    \pagenumbering{gobble}
    \maketitle
    
    
    \begin{abstract}
        Magnetic confinement reactors---in particular tokamaks---offer one of the most promising options for achieving practical nuclear fusion, with the potential to provide virtually limitless, clean energy. The theoretical and numerical modeling of tokamak plasmas is simultaneously an essential component of effective reactor design, and a great research barrier. Tokamak operational conditions exhibit comparatively low Knudsen numbers. Kinetic effects, including kinetic waves and instabilities, Landau damping, bump-on-tail instabilities and more, are therefore highly influential in tokamak plasma dynamics. Purely fluid models are inherently incapable of capturing these effects, whereas the high dimensionality in purely kinetic models render them practically intractable for most relevant purposes.

        We consider a $\delta\!f$ decomposition model, with a macroscopic fluid background and microscopic kinetic correction, both fully coupled to each other. A similar manner of discretization is proposed to that used in the recent \texttt{STRUPHY} code \cite{Holderied_Possanner_Wang_2021, Holderied_2022, Li_et_al_2023} with a finite-element model for the background and a pseudo-particle/PiC model for the correction.

        The fluid background satisfies the full, non-linear, resistive, compressible, Hall MHD equations. \cite{Laakmann_Hu_Farrell_2022} introduces finite-element(-in-space) implicit timesteppers for the incompressible analogue to this system with structure-preserving (SP) properties in the ideal case, alongside parameter-robust preconditioners. We show that these timesteppers can derive from a finite-element-in-time (FET) (and finite-element-in-space) interpretation. The benefits of this reformulation are discussed, including the derivation of timesteppers that are higher order in time, and the quantifiable dissipative SP properties in the non-ideal, resistive case.
        
        We discuss possible options for extending this FET approach to timesteppers for the compressible case.

        The kinetic corrections satisfy linearized Boltzmann equations. Using a Lénard--Bernstein collision operator, these take Fokker--Planck-like forms \cite{Fokker_1914, Planck_1917} wherein pseudo-particles in the numerical model obey the neoclassical transport equations, with particle-independent Brownian drift terms. This offers a rigorous methodology for incorporating collisions into the particle transport model, without coupling the equations of motions for each particle.
        
        Works by Chen, Chacón et al. \cite{Chen_Chacón_Barnes_2011, Chacón_Chen_Barnes_2013, Chen_Chacón_2014, Chen_Chacón_2015} have developed structure-preserving particle pushers for neoclassical transport in the Vlasov equations, derived from Crank--Nicolson integrators. We show these too can can derive from a FET interpretation, similarly offering potential extensions to higher-order-in-time particle pushers. The FET formulation is used also to consider how the stochastic drift terms can be incorporated into the pushers. Stochastic gyrokinetic expansions are also discussed.

        Different options for the numerical implementation of these schemes are considered.

        Due to the efficacy of FET in the development of SP timesteppers for both the fluid and kinetic component, we hope this approach will prove effective in the future for developing SP timesteppers for the full hybrid model. We hope this will give us the opportunity to incorporate previously inaccessible kinetic effects into the highly effective, modern, finite-element MHD models.
    \end{abstract}
    
    
    \newpage
    \tableofcontents
    
    
    \newpage
    \pagenumbering{arabic}
    %\linenumbers\renewcommand\thelinenumber{\color{black!50}\arabic{linenumber}}
            \input{0 - introduction/main.tex}
        \part{Research}
            \input{1 - low-noise PiC models/main.tex}
            \input{2 - kinetic component/main.tex}
            \input{3 - fluid component/main.tex}
            \input{4 - numerical implementation/main.tex}
        \part{Project Overview}
            \input{5 - research plan/main.tex}
            \input{6 - summary/main.tex}
    
    
    %\section{}
    \newpage
    \pagenumbering{gobble}
        \printbibliography


    \newpage
    \pagenumbering{roman}
    \appendix
        \part{Appendices}
            \input{8 - Hilbert complexes/main.tex}
            \input{9 - weak conservation proofs/main.tex}
\end{document}


\title{\BA{Title in Progress...}}
\author{Boris Andrews}
\affil{Mathematical Institute, University of Oxford}
\date{\today}


\begin{document}
    \pagenumbering{gobble}
    \maketitle
    
    
    \begin{abstract}
        Magnetic confinement reactors---in particular tokamaks---offer one of the most promising options for achieving practical nuclear fusion, with the potential to provide virtually limitless, clean energy. The theoretical and numerical modeling of tokamak plasmas is simultaneously an essential component of effective reactor design, and a great research barrier. Tokamak operational conditions exhibit comparatively low Knudsen numbers. Kinetic effects, including kinetic waves and instabilities, Landau damping, bump-on-tail instabilities and more, are therefore highly influential in tokamak plasma dynamics. Purely fluid models are inherently incapable of capturing these effects, whereas the high dimensionality in purely kinetic models render them practically intractable for most relevant purposes.

        We consider a $\delta\!f$ decomposition model, with a macroscopic fluid background and microscopic kinetic correction, both fully coupled to each other. A similar manner of discretization is proposed to that used in the recent \texttt{STRUPHY} code \cite{Holderied_Possanner_Wang_2021, Holderied_2022, Li_et_al_2023} with a finite-element model for the background and a pseudo-particle/PiC model for the correction.

        The fluid background satisfies the full, non-linear, resistive, compressible, Hall MHD equations. \cite{Laakmann_Hu_Farrell_2022} introduces finite-element(-in-space) implicit timesteppers for the incompressible analogue to this system with structure-preserving (SP) properties in the ideal case, alongside parameter-robust preconditioners. We show that these timesteppers can derive from a finite-element-in-time (FET) (and finite-element-in-space) interpretation. The benefits of this reformulation are discussed, including the derivation of timesteppers that are higher order in time, and the quantifiable dissipative SP properties in the non-ideal, resistive case.
        
        We discuss possible options for extending this FET approach to timesteppers for the compressible case.

        The kinetic corrections satisfy linearized Boltzmann equations. Using a Lénard--Bernstein collision operator, these take Fokker--Planck-like forms \cite{Fokker_1914, Planck_1917} wherein pseudo-particles in the numerical model obey the neoclassical transport equations, with particle-independent Brownian drift terms. This offers a rigorous methodology for incorporating collisions into the particle transport model, without coupling the equations of motions for each particle.
        
        Works by Chen, Chacón et al. \cite{Chen_Chacón_Barnes_2011, Chacón_Chen_Barnes_2013, Chen_Chacón_2014, Chen_Chacón_2015} have developed structure-preserving particle pushers for neoclassical transport in the Vlasov equations, derived from Crank--Nicolson integrators. We show these too can can derive from a FET interpretation, similarly offering potential extensions to higher-order-in-time particle pushers. The FET formulation is used also to consider how the stochastic drift terms can be incorporated into the pushers. Stochastic gyrokinetic expansions are also discussed.

        Different options for the numerical implementation of these schemes are considered.

        Due to the efficacy of FET in the development of SP timesteppers for both the fluid and kinetic component, we hope this approach will prove effective in the future for developing SP timesteppers for the full hybrid model. We hope this will give us the opportunity to incorporate previously inaccessible kinetic effects into the highly effective, modern, finite-element MHD models.
    \end{abstract}
    
    
    \newpage
    \tableofcontents
    
    
    \newpage
    \pagenumbering{arabic}
    %\linenumbers\renewcommand\thelinenumber{\color{black!50}\arabic{linenumber}}
            \documentclass[12pt, a4paper]{report}

\input{template/main.tex}

\title{\BA{Title in Progress...}}
\author{Boris Andrews}
\affil{Mathematical Institute, University of Oxford}
\date{\today}


\begin{document}
    \pagenumbering{gobble}
    \maketitle
    
    
    \begin{abstract}
        Magnetic confinement reactors---in particular tokamaks---offer one of the most promising options for achieving practical nuclear fusion, with the potential to provide virtually limitless, clean energy. The theoretical and numerical modeling of tokamak plasmas is simultaneously an essential component of effective reactor design, and a great research barrier. Tokamak operational conditions exhibit comparatively low Knudsen numbers. Kinetic effects, including kinetic waves and instabilities, Landau damping, bump-on-tail instabilities and more, are therefore highly influential in tokamak plasma dynamics. Purely fluid models are inherently incapable of capturing these effects, whereas the high dimensionality in purely kinetic models render them practically intractable for most relevant purposes.

        We consider a $\delta\!f$ decomposition model, with a macroscopic fluid background and microscopic kinetic correction, both fully coupled to each other. A similar manner of discretization is proposed to that used in the recent \texttt{STRUPHY} code \cite{Holderied_Possanner_Wang_2021, Holderied_2022, Li_et_al_2023} with a finite-element model for the background and a pseudo-particle/PiC model for the correction.

        The fluid background satisfies the full, non-linear, resistive, compressible, Hall MHD equations. \cite{Laakmann_Hu_Farrell_2022} introduces finite-element(-in-space) implicit timesteppers for the incompressible analogue to this system with structure-preserving (SP) properties in the ideal case, alongside parameter-robust preconditioners. We show that these timesteppers can derive from a finite-element-in-time (FET) (and finite-element-in-space) interpretation. The benefits of this reformulation are discussed, including the derivation of timesteppers that are higher order in time, and the quantifiable dissipative SP properties in the non-ideal, resistive case.
        
        We discuss possible options for extending this FET approach to timesteppers for the compressible case.

        The kinetic corrections satisfy linearized Boltzmann equations. Using a Lénard--Bernstein collision operator, these take Fokker--Planck-like forms \cite{Fokker_1914, Planck_1917} wherein pseudo-particles in the numerical model obey the neoclassical transport equations, with particle-independent Brownian drift terms. This offers a rigorous methodology for incorporating collisions into the particle transport model, without coupling the equations of motions for each particle.
        
        Works by Chen, Chacón et al. \cite{Chen_Chacón_Barnes_2011, Chacón_Chen_Barnes_2013, Chen_Chacón_2014, Chen_Chacón_2015} have developed structure-preserving particle pushers for neoclassical transport in the Vlasov equations, derived from Crank--Nicolson integrators. We show these too can can derive from a FET interpretation, similarly offering potential extensions to higher-order-in-time particle pushers. The FET formulation is used also to consider how the stochastic drift terms can be incorporated into the pushers. Stochastic gyrokinetic expansions are also discussed.

        Different options for the numerical implementation of these schemes are considered.

        Due to the efficacy of FET in the development of SP timesteppers for both the fluid and kinetic component, we hope this approach will prove effective in the future for developing SP timesteppers for the full hybrid model. We hope this will give us the opportunity to incorporate previously inaccessible kinetic effects into the highly effective, modern, finite-element MHD models.
    \end{abstract}
    
    
    \newpage
    \tableofcontents
    
    
    \newpage
    \pagenumbering{arabic}
    %\linenumbers\renewcommand\thelinenumber{\color{black!50}\arabic{linenumber}}
            \input{0 - introduction/main.tex}
        \part{Research}
            \input{1 - low-noise PiC models/main.tex}
            \input{2 - kinetic component/main.tex}
            \input{3 - fluid component/main.tex}
            \input{4 - numerical implementation/main.tex}
        \part{Project Overview}
            \input{5 - research plan/main.tex}
            \input{6 - summary/main.tex}
    
    
    %\section{}
    \newpage
    \pagenumbering{gobble}
        \printbibliography


    \newpage
    \pagenumbering{roman}
    \appendix
        \part{Appendices}
            \input{8 - Hilbert complexes/main.tex}
            \input{9 - weak conservation proofs/main.tex}
\end{document}

        \part{Research}
            \documentclass[12pt, a4paper]{report}

\input{template/main.tex}

\title{\BA{Title in Progress...}}
\author{Boris Andrews}
\affil{Mathematical Institute, University of Oxford}
\date{\today}


\begin{document}
    \pagenumbering{gobble}
    \maketitle
    
    
    \begin{abstract}
        Magnetic confinement reactors---in particular tokamaks---offer one of the most promising options for achieving practical nuclear fusion, with the potential to provide virtually limitless, clean energy. The theoretical and numerical modeling of tokamak plasmas is simultaneously an essential component of effective reactor design, and a great research barrier. Tokamak operational conditions exhibit comparatively low Knudsen numbers. Kinetic effects, including kinetic waves and instabilities, Landau damping, bump-on-tail instabilities and more, are therefore highly influential in tokamak plasma dynamics. Purely fluid models are inherently incapable of capturing these effects, whereas the high dimensionality in purely kinetic models render them practically intractable for most relevant purposes.

        We consider a $\delta\!f$ decomposition model, with a macroscopic fluid background and microscopic kinetic correction, both fully coupled to each other. A similar manner of discretization is proposed to that used in the recent \texttt{STRUPHY} code \cite{Holderied_Possanner_Wang_2021, Holderied_2022, Li_et_al_2023} with a finite-element model for the background and a pseudo-particle/PiC model for the correction.

        The fluid background satisfies the full, non-linear, resistive, compressible, Hall MHD equations. \cite{Laakmann_Hu_Farrell_2022} introduces finite-element(-in-space) implicit timesteppers for the incompressible analogue to this system with structure-preserving (SP) properties in the ideal case, alongside parameter-robust preconditioners. We show that these timesteppers can derive from a finite-element-in-time (FET) (and finite-element-in-space) interpretation. The benefits of this reformulation are discussed, including the derivation of timesteppers that are higher order in time, and the quantifiable dissipative SP properties in the non-ideal, resistive case.
        
        We discuss possible options for extending this FET approach to timesteppers for the compressible case.

        The kinetic corrections satisfy linearized Boltzmann equations. Using a Lénard--Bernstein collision operator, these take Fokker--Planck-like forms \cite{Fokker_1914, Planck_1917} wherein pseudo-particles in the numerical model obey the neoclassical transport equations, with particle-independent Brownian drift terms. This offers a rigorous methodology for incorporating collisions into the particle transport model, without coupling the equations of motions for each particle.
        
        Works by Chen, Chacón et al. \cite{Chen_Chacón_Barnes_2011, Chacón_Chen_Barnes_2013, Chen_Chacón_2014, Chen_Chacón_2015} have developed structure-preserving particle pushers for neoclassical transport in the Vlasov equations, derived from Crank--Nicolson integrators. We show these too can can derive from a FET interpretation, similarly offering potential extensions to higher-order-in-time particle pushers. The FET formulation is used also to consider how the stochastic drift terms can be incorporated into the pushers. Stochastic gyrokinetic expansions are also discussed.

        Different options for the numerical implementation of these schemes are considered.

        Due to the efficacy of FET in the development of SP timesteppers for both the fluid and kinetic component, we hope this approach will prove effective in the future for developing SP timesteppers for the full hybrid model. We hope this will give us the opportunity to incorporate previously inaccessible kinetic effects into the highly effective, modern, finite-element MHD models.
    \end{abstract}
    
    
    \newpage
    \tableofcontents
    
    
    \newpage
    \pagenumbering{arabic}
    %\linenumbers\renewcommand\thelinenumber{\color{black!50}\arabic{linenumber}}
            \input{0 - introduction/main.tex}
        \part{Research}
            \input{1 - low-noise PiC models/main.tex}
            \input{2 - kinetic component/main.tex}
            \input{3 - fluid component/main.tex}
            \input{4 - numerical implementation/main.tex}
        \part{Project Overview}
            \input{5 - research plan/main.tex}
            \input{6 - summary/main.tex}
    
    
    %\section{}
    \newpage
    \pagenumbering{gobble}
        \printbibliography


    \newpage
    \pagenumbering{roman}
    \appendix
        \part{Appendices}
            \input{8 - Hilbert complexes/main.tex}
            \input{9 - weak conservation proofs/main.tex}
\end{document}

            \documentclass[12pt, a4paper]{report}

\input{template/main.tex}

\title{\BA{Title in Progress...}}
\author{Boris Andrews}
\affil{Mathematical Institute, University of Oxford}
\date{\today}


\begin{document}
    \pagenumbering{gobble}
    \maketitle
    
    
    \begin{abstract}
        Magnetic confinement reactors---in particular tokamaks---offer one of the most promising options for achieving practical nuclear fusion, with the potential to provide virtually limitless, clean energy. The theoretical and numerical modeling of tokamak plasmas is simultaneously an essential component of effective reactor design, and a great research barrier. Tokamak operational conditions exhibit comparatively low Knudsen numbers. Kinetic effects, including kinetic waves and instabilities, Landau damping, bump-on-tail instabilities and more, are therefore highly influential in tokamak plasma dynamics. Purely fluid models are inherently incapable of capturing these effects, whereas the high dimensionality in purely kinetic models render them practically intractable for most relevant purposes.

        We consider a $\delta\!f$ decomposition model, with a macroscopic fluid background and microscopic kinetic correction, both fully coupled to each other. A similar manner of discretization is proposed to that used in the recent \texttt{STRUPHY} code \cite{Holderied_Possanner_Wang_2021, Holderied_2022, Li_et_al_2023} with a finite-element model for the background and a pseudo-particle/PiC model for the correction.

        The fluid background satisfies the full, non-linear, resistive, compressible, Hall MHD equations. \cite{Laakmann_Hu_Farrell_2022} introduces finite-element(-in-space) implicit timesteppers for the incompressible analogue to this system with structure-preserving (SP) properties in the ideal case, alongside parameter-robust preconditioners. We show that these timesteppers can derive from a finite-element-in-time (FET) (and finite-element-in-space) interpretation. The benefits of this reformulation are discussed, including the derivation of timesteppers that are higher order in time, and the quantifiable dissipative SP properties in the non-ideal, resistive case.
        
        We discuss possible options for extending this FET approach to timesteppers for the compressible case.

        The kinetic corrections satisfy linearized Boltzmann equations. Using a Lénard--Bernstein collision operator, these take Fokker--Planck-like forms \cite{Fokker_1914, Planck_1917} wherein pseudo-particles in the numerical model obey the neoclassical transport equations, with particle-independent Brownian drift terms. This offers a rigorous methodology for incorporating collisions into the particle transport model, without coupling the equations of motions for each particle.
        
        Works by Chen, Chacón et al. \cite{Chen_Chacón_Barnes_2011, Chacón_Chen_Barnes_2013, Chen_Chacón_2014, Chen_Chacón_2015} have developed structure-preserving particle pushers for neoclassical transport in the Vlasov equations, derived from Crank--Nicolson integrators. We show these too can can derive from a FET interpretation, similarly offering potential extensions to higher-order-in-time particle pushers. The FET formulation is used also to consider how the stochastic drift terms can be incorporated into the pushers. Stochastic gyrokinetic expansions are also discussed.

        Different options for the numerical implementation of these schemes are considered.

        Due to the efficacy of FET in the development of SP timesteppers for both the fluid and kinetic component, we hope this approach will prove effective in the future for developing SP timesteppers for the full hybrid model. We hope this will give us the opportunity to incorporate previously inaccessible kinetic effects into the highly effective, modern, finite-element MHD models.
    \end{abstract}
    
    
    \newpage
    \tableofcontents
    
    
    \newpage
    \pagenumbering{arabic}
    %\linenumbers\renewcommand\thelinenumber{\color{black!50}\arabic{linenumber}}
            \input{0 - introduction/main.tex}
        \part{Research}
            \input{1 - low-noise PiC models/main.tex}
            \input{2 - kinetic component/main.tex}
            \input{3 - fluid component/main.tex}
            \input{4 - numerical implementation/main.tex}
        \part{Project Overview}
            \input{5 - research plan/main.tex}
            \input{6 - summary/main.tex}
    
    
    %\section{}
    \newpage
    \pagenumbering{gobble}
        \printbibliography


    \newpage
    \pagenumbering{roman}
    \appendix
        \part{Appendices}
            \input{8 - Hilbert complexes/main.tex}
            \input{9 - weak conservation proofs/main.tex}
\end{document}

            \documentclass[12pt, a4paper]{report}

\input{template/main.tex}

\title{\BA{Title in Progress...}}
\author{Boris Andrews}
\affil{Mathematical Institute, University of Oxford}
\date{\today}


\begin{document}
    \pagenumbering{gobble}
    \maketitle
    
    
    \begin{abstract}
        Magnetic confinement reactors---in particular tokamaks---offer one of the most promising options for achieving practical nuclear fusion, with the potential to provide virtually limitless, clean energy. The theoretical and numerical modeling of tokamak plasmas is simultaneously an essential component of effective reactor design, and a great research barrier. Tokamak operational conditions exhibit comparatively low Knudsen numbers. Kinetic effects, including kinetic waves and instabilities, Landau damping, bump-on-tail instabilities and more, are therefore highly influential in tokamak plasma dynamics. Purely fluid models are inherently incapable of capturing these effects, whereas the high dimensionality in purely kinetic models render them practically intractable for most relevant purposes.

        We consider a $\delta\!f$ decomposition model, with a macroscopic fluid background and microscopic kinetic correction, both fully coupled to each other. A similar manner of discretization is proposed to that used in the recent \texttt{STRUPHY} code \cite{Holderied_Possanner_Wang_2021, Holderied_2022, Li_et_al_2023} with a finite-element model for the background and a pseudo-particle/PiC model for the correction.

        The fluid background satisfies the full, non-linear, resistive, compressible, Hall MHD equations. \cite{Laakmann_Hu_Farrell_2022} introduces finite-element(-in-space) implicit timesteppers for the incompressible analogue to this system with structure-preserving (SP) properties in the ideal case, alongside parameter-robust preconditioners. We show that these timesteppers can derive from a finite-element-in-time (FET) (and finite-element-in-space) interpretation. The benefits of this reformulation are discussed, including the derivation of timesteppers that are higher order in time, and the quantifiable dissipative SP properties in the non-ideal, resistive case.
        
        We discuss possible options for extending this FET approach to timesteppers for the compressible case.

        The kinetic corrections satisfy linearized Boltzmann equations. Using a Lénard--Bernstein collision operator, these take Fokker--Planck-like forms \cite{Fokker_1914, Planck_1917} wherein pseudo-particles in the numerical model obey the neoclassical transport equations, with particle-independent Brownian drift terms. This offers a rigorous methodology for incorporating collisions into the particle transport model, without coupling the equations of motions for each particle.
        
        Works by Chen, Chacón et al. \cite{Chen_Chacón_Barnes_2011, Chacón_Chen_Barnes_2013, Chen_Chacón_2014, Chen_Chacón_2015} have developed structure-preserving particle pushers for neoclassical transport in the Vlasov equations, derived from Crank--Nicolson integrators. We show these too can can derive from a FET interpretation, similarly offering potential extensions to higher-order-in-time particle pushers. The FET formulation is used also to consider how the stochastic drift terms can be incorporated into the pushers. Stochastic gyrokinetic expansions are also discussed.

        Different options for the numerical implementation of these schemes are considered.

        Due to the efficacy of FET in the development of SP timesteppers for both the fluid and kinetic component, we hope this approach will prove effective in the future for developing SP timesteppers for the full hybrid model. We hope this will give us the opportunity to incorporate previously inaccessible kinetic effects into the highly effective, modern, finite-element MHD models.
    \end{abstract}
    
    
    \newpage
    \tableofcontents
    
    
    \newpage
    \pagenumbering{arabic}
    %\linenumbers\renewcommand\thelinenumber{\color{black!50}\arabic{linenumber}}
            \input{0 - introduction/main.tex}
        \part{Research}
            \input{1 - low-noise PiC models/main.tex}
            \input{2 - kinetic component/main.tex}
            \input{3 - fluid component/main.tex}
            \input{4 - numerical implementation/main.tex}
        \part{Project Overview}
            \input{5 - research plan/main.tex}
            \input{6 - summary/main.tex}
    
    
    %\section{}
    \newpage
    \pagenumbering{gobble}
        \printbibliography


    \newpage
    \pagenumbering{roman}
    \appendix
        \part{Appendices}
            \input{8 - Hilbert complexes/main.tex}
            \input{9 - weak conservation proofs/main.tex}
\end{document}

            \documentclass[12pt, a4paper]{report}

\input{template/main.tex}

\title{\BA{Title in Progress...}}
\author{Boris Andrews}
\affil{Mathematical Institute, University of Oxford}
\date{\today}


\begin{document}
    \pagenumbering{gobble}
    \maketitle
    
    
    \begin{abstract}
        Magnetic confinement reactors---in particular tokamaks---offer one of the most promising options for achieving practical nuclear fusion, with the potential to provide virtually limitless, clean energy. The theoretical and numerical modeling of tokamak plasmas is simultaneously an essential component of effective reactor design, and a great research barrier. Tokamak operational conditions exhibit comparatively low Knudsen numbers. Kinetic effects, including kinetic waves and instabilities, Landau damping, bump-on-tail instabilities and more, are therefore highly influential in tokamak plasma dynamics. Purely fluid models are inherently incapable of capturing these effects, whereas the high dimensionality in purely kinetic models render them practically intractable for most relevant purposes.

        We consider a $\delta\!f$ decomposition model, with a macroscopic fluid background and microscopic kinetic correction, both fully coupled to each other. A similar manner of discretization is proposed to that used in the recent \texttt{STRUPHY} code \cite{Holderied_Possanner_Wang_2021, Holderied_2022, Li_et_al_2023} with a finite-element model for the background and a pseudo-particle/PiC model for the correction.

        The fluid background satisfies the full, non-linear, resistive, compressible, Hall MHD equations. \cite{Laakmann_Hu_Farrell_2022} introduces finite-element(-in-space) implicit timesteppers for the incompressible analogue to this system with structure-preserving (SP) properties in the ideal case, alongside parameter-robust preconditioners. We show that these timesteppers can derive from a finite-element-in-time (FET) (and finite-element-in-space) interpretation. The benefits of this reformulation are discussed, including the derivation of timesteppers that are higher order in time, and the quantifiable dissipative SP properties in the non-ideal, resistive case.
        
        We discuss possible options for extending this FET approach to timesteppers for the compressible case.

        The kinetic corrections satisfy linearized Boltzmann equations. Using a Lénard--Bernstein collision operator, these take Fokker--Planck-like forms \cite{Fokker_1914, Planck_1917} wherein pseudo-particles in the numerical model obey the neoclassical transport equations, with particle-independent Brownian drift terms. This offers a rigorous methodology for incorporating collisions into the particle transport model, without coupling the equations of motions for each particle.
        
        Works by Chen, Chacón et al. \cite{Chen_Chacón_Barnes_2011, Chacón_Chen_Barnes_2013, Chen_Chacón_2014, Chen_Chacón_2015} have developed structure-preserving particle pushers for neoclassical transport in the Vlasov equations, derived from Crank--Nicolson integrators. We show these too can can derive from a FET interpretation, similarly offering potential extensions to higher-order-in-time particle pushers. The FET formulation is used also to consider how the stochastic drift terms can be incorporated into the pushers. Stochastic gyrokinetic expansions are also discussed.

        Different options for the numerical implementation of these schemes are considered.

        Due to the efficacy of FET in the development of SP timesteppers for both the fluid and kinetic component, we hope this approach will prove effective in the future for developing SP timesteppers for the full hybrid model. We hope this will give us the opportunity to incorporate previously inaccessible kinetic effects into the highly effective, modern, finite-element MHD models.
    \end{abstract}
    
    
    \newpage
    \tableofcontents
    
    
    \newpage
    \pagenumbering{arabic}
    %\linenumbers\renewcommand\thelinenumber{\color{black!50}\arabic{linenumber}}
            \input{0 - introduction/main.tex}
        \part{Research}
            \input{1 - low-noise PiC models/main.tex}
            \input{2 - kinetic component/main.tex}
            \input{3 - fluid component/main.tex}
            \input{4 - numerical implementation/main.tex}
        \part{Project Overview}
            \input{5 - research plan/main.tex}
            \input{6 - summary/main.tex}
    
    
    %\section{}
    \newpage
    \pagenumbering{gobble}
        \printbibliography


    \newpage
    \pagenumbering{roman}
    \appendix
        \part{Appendices}
            \input{8 - Hilbert complexes/main.tex}
            \input{9 - weak conservation proofs/main.tex}
\end{document}

        \part{Project Overview}
            \documentclass[12pt, a4paper]{report}

\input{template/main.tex}

\title{\BA{Title in Progress...}}
\author{Boris Andrews}
\affil{Mathematical Institute, University of Oxford}
\date{\today}


\begin{document}
    \pagenumbering{gobble}
    \maketitle
    
    
    \begin{abstract}
        Magnetic confinement reactors---in particular tokamaks---offer one of the most promising options for achieving practical nuclear fusion, with the potential to provide virtually limitless, clean energy. The theoretical and numerical modeling of tokamak plasmas is simultaneously an essential component of effective reactor design, and a great research barrier. Tokamak operational conditions exhibit comparatively low Knudsen numbers. Kinetic effects, including kinetic waves and instabilities, Landau damping, bump-on-tail instabilities and more, are therefore highly influential in tokamak plasma dynamics. Purely fluid models are inherently incapable of capturing these effects, whereas the high dimensionality in purely kinetic models render them practically intractable for most relevant purposes.

        We consider a $\delta\!f$ decomposition model, with a macroscopic fluid background and microscopic kinetic correction, both fully coupled to each other. A similar manner of discretization is proposed to that used in the recent \texttt{STRUPHY} code \cite{Holderied_Possanner_Wang_2021, Holderied_2022, Li_et_al_2023} with a finite-element model for the background and a pseudo-particle/PiC model for the correction.

        The fluid background satisfies the full, non-linear, resistive, compressible, Hall MHD equations. \cite{Laakmann_Hu_Farrell_2022} introduces finite-element(-in-space) implicit timesteppers for the incompressible analogue to this system with structure-preserving (SP) properties in the ideal case, alongside parameter-robust preconditioners. We show that these timesteppers can derive from a finite-element-in-time (FET) (and finite-element-in-space) interpretation. The benefits of this reformulation are discussed, including the derivation of timesteppers that are higher order in time, and the quantifiable dissipative SP properties in the non-ideal, resistive case.
        
        We discuss possible options for extending this FET approach to timesteppers for the compressible case.

        The kinetic corrections satisfy linearized Boltzmann equations. Using a Lénard--Bernstein collision operator, these take Fokker--Planck-like forms \cite{Fokker_1914, Planck_1917} wherein pseudo-particles in the numerical model obey the neoclassical transport equations, with particle-independent Brownian drift terms. This offers a rigorous methodology for incorporating collisions into the particle transport model, without coupling the equations of motions for each particle.
        
        Works by Chen, Chacón et al. \cite{Chen_Chacón_Barnes_2011, Chacón_Chen_Barnes_2013, Chen_Chacón_2014, Chen_Chacón_2015} have developed structure-preserving particle pushers for neoclassical transport in the Vlasov equations, derived from Crank--Nicolson integrators. We show these too can can derive from a FET interpretation, similarly offering potential extensions to higher-order-in-time particle pushers. The FET formulation is used also to consider how the stochastic drift terms can be incorporated into the pushers. Stochastic gyrokinetic expansions are also discussed.

        Different options for the numerical implementation of these schemes are considered.

        Due to the efficacy of FET in the development of SP timesteppers for both the fluid and kinetic component, we hope this approach will prove effective in the future for developing SP timesteppers for the full hybrid model. We hope this will give us the opportunity to incorporate previously inaccessible kinetic effects into the highly effective, modern, finite-element MHD models.
    \end{abstract}
    
    
    \newpage
    \tableofcontents
    
    
    \newpage
    \pagenumbering{arabic}
    %\linenumbers\renewcommand\thelinenumber{\color{black!50}\arabic{linenumber}}
            \input{0 - introduction/main.tex}
        \part{Research}
            \input{1 - low-noise PiC models/main.tex}
            \input{2 - kinetic component/main.tex}
            \input{3 - fluid component/main.tex}
            \input{4 - numerical implementation/main.tex}
        \part{Project Overview}
            \input{5 - research plan/main.tex}
            \input{6 - summary/main.tex}
    
    
    %\section{}
    \newpage
    \pagenumbering{gobble}
        \printbibliography


    \newpage
    \pagenumbering{roman}
    \appendix
        \part{Appendices}
            \input{8 - Hilbert complexes/main.tex}
            \input{9 - weak conservation proofs/main.tex}
\end{document}

            \documentclass[12pt, a4paper]{report}

\input{template/main.tex}

\title{\BA{Title in Progress...}}
\author{Boris Andrews}
\affil{Mathematical Institute, University of Oxford}
\date{\today}


\begin{document}
    \pagenumbering{gobble}
    \maketitle
    
    
    \begin{abstract}
        Magnetic confinement reactors---in particular tokamaks---offer one of the most promising options for achieving practical nuclear fusion, with the potential to provide virtually limitless, clean energy. The theoretical and numerical modeling of tokamak plasmas is simultaneously an essential component of effective reactor design, and a great research barrier. Tokamak operational conditions exhibit comparatively low Knudsen numbers. Kinetic effects, including kinetic waves and instabilities, Landau damping, bump-on-tail instabilities and more, are therefore highly influential in tokamak plasma dynamics. Purely fluid models are inherently incapable of capturing these effects, whereas the high dimensionality in purely kinetic models render them practically intractable for most relevant purposes.

        We consider a $\delta\!f$ decomposition model, with a macroscopic fluid background and microscopic kinetic correction, both fully coupled to each other. A similar manner of discretization is proposed to that used in the recent \texttt{STRUPHY} code \cite{Holderied_Possanner_Wang_2021, Holderied_2022, Li_et_al_2023} with a finite-element model for the background and a pseudo-particle/PiC model for the correction.

        The fluid background satisfies the full, non-linear, resistive, compressible, Hall MHD equations. \cite{Laakmann_Hu_Farrell_2022} introduces finite-element(-in-space) implicit timesteppers for the incompressible analogue to this system with structure-preserving (SP) properties in the ideal case, alongside parameter-robust preconditioners. We show that these timesteppers can derive from a finite-element-in-time (FET) (and finite-element-in-space) interpretation. The benefits of this reformulation are discussed, including the derivation of timesteppers that are higher order in time, and the quantifiable dissipative SP properties in the non-ideal, resistive case.
        
        We discuss possible options for extending this FET approach to timesteppers for the compressible case.

        The kinetic corrections satisfy linearized Boltzmann equations. Using a Lénard--Bernstein collision operator, these take Fokker--Planck-like forms \cite{Fokker_1914, Planck_1917} wherein pseudo-particles in the numerical model obey the neoclassical transport equations, with particle-independent Brownian drift terms. This offers a rigorous methodology for incorporating collisions into the particle transport model, without coupling the equations of motions for each particle.
        
        Works by Chen, Chacón et al. \cite{Chen_Chacón_Barnes_2011, Chacón_Chen_Barnes_2013, Chen_Chacón_2014, Chen_Chacón_2015} have developed structure-preserving particle pushers for neoclassical transport in the Vlasov equations, derived from Crank--Nicolson integrators. We show these too can can derive from a FET interpretation, similarly offering potential extensions to higher-order-in-time particle pushers. The FET formulation is used also to consider how the stochastic drift terms can be incorporated into the pushers. Stochastic gyrokinetic expansions are also discussed.

        Different options for the numerical implementation of these schemes are considered.

        Due to the efficacy of FET in the development of SP timesteppers for both the fluid and kinetic component, we hope this approach will prove effective in the future for developing SP timesteppers for the full hybrid model. We hope this will give us the opportunity to incorporate previously inaccessible kinetic effects into the highly effective, modern, finite-element MHD models.
    \end{abstract}
    
    
    \newpage
    \tableofcontents
    
    
    \newpage
    \pagenumbering{arabic}
    %\linenumbers\renewcommand\thelinenumber{\color{black!50}\arabic{linenumber}}
            \input{0 - introduction/main.tex}
        \part{Research}
            \input{1 - low-noise PiC models/main.tex}
            \input{2 - kinetic component/main.tex}
            \input{3 - fluid component/main.tex}
            \input{4 - numerical implementation/main.tex}
        \part{Project Overview}
            \input{5 - research plan/main.tex}
            \input{6 - summary/main.tex}
    
    
    %\section{}
    \newpage
    \pagenumbering{gobble}
        \printbibliography


    \newpage
    \pagenumbering{roman}
    \appendix
        \part{Appendices}
            \input{8 - Hilbert complexes/main.tex}
            \input{9 - weak conservation proofs/main.tex}
\end{document}

    
    
    %\section{}
    \newpage
    \pagenumbering{gobble}
        \printbibliography


    \newpage
    \pagenumbering{roman}
    \appendix
        \part{Appendices}
            \documentclass[12pt, a4paper]{report}

\input{template/main.tex}

\title{\BA{Title in Progress...}}
\author{Boris Andrews}
\affil{Mathematical Institute, University of Oxford}
\date{\today}


\begin{document}
    \pagenumbering{gobble}
    \maketitle
    
    
    \begin{abstract}
        Magnetic confinement reactors---in particular tokamaks---offer one of the most promising options for achieving practical nuclear fusion, with the potential to provide virtually limitless, clean energy. The theoretical and numerical modeling of tokamak plasmas is simultaneously an essential component of effective reactor design, and a great research barrier. Tokamak operational conditions exhibit comparatively low Knudsen numbers. Kinetic effects, including kinetic waves and instabilities, Landau damping, bump-on-tail instabilities and more, are therefore highly influential in tokamak plasma dynamics. Purely fluid models are inherently incapable of capturing these effects, whereas the high dimensionality in purely kinetic models render them practically intractable for most relevant purposes.

        We consider a $\delta\!f$ decomposition model, with a macroscopic fluid background and microscopic kinetic correction, both fully coupled to each other. A similar manner of discretization is proposed to that used in the recent \texttt{STRUPHY} code \cite{Holderied_Possanner_Wang_2021, Holderied_2022, Li_et_al_2023} with a finite-element model for the background and a pseudo-particle/PiC model for the correction.

        The fluid background satisfies the full, non-linear, resistive, compressible, Hall MHD equations. \cite{Laakmann_Hu_Farrell_2022} introduces finite-element(-in-space) implicit timesteppers for the incompressible analogue to this system with structure-preserving (SP) properties in the ideal case, alongside parameter-robust preconditioners. We show that these timesteppers can derive from a finite-element-in-time (FET) (and finite-element-in-space) interpretation. The benefits of this reformulation are discussed, including the derivation of timesteppers that are higher order in time, and the quantifiable dissipative SP properties in the non-ideal, resistive case.
        
        We discuss possible options for extending this FET approach to timesteppers for the compressible case.

        The kinetic corrections satisfy linearized Boltzmann equations. Using a Lénard--Bernstein collision operator, these take Fokker--Planck-like forms \cite{Fokker_1914, Planck_1917} wherein pseudo-particles in the numerical model obey the neoclassical transport equations, with particle-independent Brownian drift terms. This offers a rigorous methodology for incorporating collisions into the particle transport model, without coupling the equations of motions for each particle.
        
        Works by Chen, Chacón et al. \cite{Chen_Chacón_Barnes_2011, Chacón_Chen_Barnes_2013, Chen_Chacón_2014, Chen_Chacón_2015} have developed structure-preserving particle pushers for neoclassical transport in the Vlasov equations, derived from Crank--Nicolson integrators. We show these too can can derive from a FET interpretation, similarly offering potential extensions to higher-order-in-time particle pushers. The FET formulation is used also to consider how the stochastic drift terms can be incorporated into the pushers. Stochastic gyrokinetic expansions are also discussed.

        Different options for the numerical implementation of these schemes are considered.

        Due to the efficacy of FET in the development of SP timesteppers for both the fluid and kinetic component, we hope this approach will prove effective in the future for developing SP timesteppers for the full hybrid model. We hope this will give us the opportunity to incorporate previously inaccessible kinetic effects into the highly effective, modern, finite-element MHD models.
    \end{abstract}
    
    
    \newpage
    \tableofcontents
    
    
    \newpage
    \pagenumbering{arabic}
    %\linenumbers\renewcommand\thelinenumber{\color{black!50}\arabic{linenumber}}
            \input{0 - introduction/main.tex}
        \part{Research}
            \input{1 - low-noise PiC models/main.tex}
            \input{2 - kinetic component/main.tex}
            \input{3 - fluid component/main.tex}
            \input{4 - numerical implementation/main.tex}
        \part{Project Overview}
            \input{5 - research plan/main.tex}
            \input{6 - summary/main.tex}
    
    
    %\section{}
    \newpage
    \pagenumbering{gobble}
        \printbibliography


    \newpage
    \pagenumbering{roman}
    \appendix
        \part{Appendices}
            \input{8 - Hilbert complexes/main.tex}
            \input{9 - weak conservation proofs/main.tex}
\end{document}

            \documentclass[12pt, a4paper]{report}

\input{template/main.tex}

\title{\BA{Title in Progress...}}
\author{Boris Andrews}
\affil{Mathematical Institute, University of Oxford}
\date{\today}


\begin{document}
    \pagenumbering{gobble}
    \maketitle
    
    
    \begin{abstract}
        Magnetic confinement reactors---in particular tokamaks---offer one of the most promising options for achieving practical nuclear fusion, with the potential to provide virtually limitless, clean energy. The theoretical and numerical modeling of tokamak plasmas is simultaneously an essential component of effective reactor design, and a great research barrier. Tokamak operational conditions exhibit comparatively low Knudsen numbers. Kinetic effects, including kinetic waves and instabilities, Landau damping, bump-on-tail instabilities and more, are therefore highly influential in tokamak plasma dynamics. Purely fluid models are inherently incapable of capturing these effects, whereas the high dimensionality in purely kinetic models render them practically intractable for most relevant purposes.

        We consider a $\delta\!f$ decomposition model, with a macroscopic fluid background and microscopic kinetic correction, both fully coupled to each other. A similar manner of discretization is proposed to that used in the recent \texttt{STRUPHY} code \cite{Holderied_Possanner_Wang_2021, Holderied_2022, Li_et_al_2023} with a finite-element model for the background and a pseudo-particle/PiC model for the correction.

        The fluid background satisfies the full, non-linear, resistive, compressible, Hall MHD equations. \cite{Laakmann_Hu_Farrell_2022} introduces finite-element(-in-space) implicit timesteppers for the incompressible analogue to this system with structure-preserving (SP) properties in the ideal case, alongside parameter-robust preconditioners. We show that these timesteppers can derive from a finite-element-in-time (FET) (and finite-element-in-space) interpretation. The benefits of this reformulation are discussed, including the derivation of timesteppers that are higher order in time, and the quantifiable dissipative SP properties in the non-ideal, resistive case.
        
        We discuss possible options for extending this FET approach to timesteppers for the compressible case.

        The kinetic corrections satisfy linearized Boltzmann equations. Using a Lénard--Bernstein collision operator, these take Fokker--Planck-like forms \cite{Fokker_1914, Planck_1917} wherein pseudo-particles in the numerical model obey the neoclassical transport equations, with particle-independent Brownian drift terms. This offers a rigorous methodology for incorporating collisions into the particle transport model, without coupling the equations of motions for each particle.
        
        Works by Chen, Chacón et al. \cite{Chen_Chacón_Barnes_2011, Chacón_Chen_Barnes_2013, Chen_Chacón_2014, Chen_Chacón_2015} have developed structure-preserving particle pushers for neoclassical transport in the Vlasov equations, derived from Crank--Nicolson integrators. We show these too can can derive from a FET interpretation, similarly offering potential extensions to higher-order-in-time particle pushers. The FET formulation is used also to consider how the stochastic drift terms can be incorporated into the pushers. Stochastic gyrokinetic expansions are also discussed.

        Different options for the numerical implementation of these schemes are considered.

        Due to the efficacy of FET in the development of SP timesteppers for both the fluid and kinetic component, we hope this approach will prove effective in the future for developing SP timesteppers for the full hybrid model. We hope this will give us the opportunity to incorporate previously inaccessible kinetic effects into the highly effective, modern, finite-element MHD models.
    \end{abstract}
    
    
    \newpage
    \tableofcontents
    
    
    \newpage
    \pagenumbering{arabic}
    %\linenumbers\renewcommand\thelinenumber{\color{black!50}\arabic{linenumber}}
            \input{0 - introduction/main.tex}
        \part{Research}
            \input{1 - low-noise PiC models/main.tex}
            \input{2 - kinetic component/main.tex}
            \input{3 - fluid component/main.tex}
            \input{4 - numerical implementation/main.tex}
        \part{Project Overview}
            \input{5 - research plan/main.tex}
            \input{6 - summary/main.tex}
    
    
    %\section{}
    \newpage
    \pagenumbering{gobble}
        \printbibliography


    \newpage
    \pagenumbering{roman}
    \appendix
        \part{Appendices}
            \input{8 - Hilbert complexes/main.tex}
            \input{9 - weak conservation proofs/main.tex}
\end{document}

\end{document}

            \documentclass[12pt, a4paper]{report}

\documentclass[12pt, a4paper]{report}

\input{template/main.tex}

\title{\BA{Title in Progress...}}
\author{Boris Andrews}
\affil{Mathematical Institute, University of Oxford}
\date{\today}


\begin{document}
    \pagenumbering{gobble}
    \maketitle
    
    
    \begin{abstract}
        Magnetic confinement reactors---in particular tokamaks---offer one of the most promising options for achieving practical nuclear fusion, with the potential to provide virtually limitless, clean energy. The theoretical and numerical modeling of tokamak plasmas is simultaneously an essential component of effective reactor design, and a great research barrier. Tokamak operational conditions exhibit comparatively low Knudsen numbers. Kinetic effects, including kinetic waves and instabilities, Landau damping, bump-on-tail instabilities and more, are therefore highly influential in tokamak plasma dynamics. Purely fluid models are inherently incapable of capturing these effects, whereas the high dimensionality in purely kinetic models render them practically intractable for most relevant purposes.

        We consider a $\delta\!f$ decomposition model, with a macroscopic fluid background and microscopic kinetic correction, both fully coupled to each other. A similar manner of discretization is proposed to that used in the recent \texttt{STRUPHY} code \cite{Holderied_Possanner_Wang_2021, Holderied_2022, Li_et_al_2023} with a finite-element model for the background and a pseudo-particle/PiC model for the correction.

        The fluid background satisfies the full, non-linear, resistive, compressible, Hall MHD equations. \cite{Laakmann_Hu_Farrell_2022} introduces finite-element(-in-space) implicit timesteppers for the incompressible analogue to this system with structure-preserving (SP) properties in the ideal case, alongside parameter-robust preconditioners. We show that these timesteppers can derive from a finite-element-in-time (FET) (and finite-element-in-space) interpretation. The benefits of this reformulation are discussed, including the derivation of timesteppers that are higher order in time, and the quantifiable dissipative SP properties in the non-ideal, resistive case.
        
        We discuss possible options for extending this FET approach to timesteppers for the compressible case.

        The kinetic corrections satisfy linearized Boltzmann equations. Using a Lénard--Bernstein collision operator, these take Fokker--Planck-like forms \cite{Fokker_1914, Planck_1917} wherein pseudo-particles in the numerical model obey the neoclassical transport equations, with particle-independent Brownian drift terms. This offers a rigorous methodology for incorporating collisions into the particle transport model, without coupling the equations of motions for each particle.
        
        Works by Chen, Chacón et al. \cite{Chen_Chacón_Barnes_2011, Chacón_Chen_Barnes_2013, Chen_Chacón_2014, Chen_Chacón_2015} have developed structure-preserving particle pushers for neoclassical transport in the Vlasov equations, derived from Crank--Nicolson integrators. We show these too can can derive from a FET interpretation, similarly offering potential extensions to higher-order-in-time particle pushers. The FET formulation is used also to consider how the stochastic drift terms can be incorporated into the pushers. Stochastic gyrokinetic expansions are also discussed.

        Different options for the numerical implementation of these schemes are considered.

        Due to the efficacy of FET in the development of SP timesteppers for both the fluid and kinetic component, we hope this approach will prove effective in the future for developing SP timesteppers for the full hybrid model. We hope this will give us the opportunity to incorporate previously inaccessible kinetic effects into the highly effective, modern, finite-element MHD models.
    \end{abstract}
    
    
    \newpage
    \tableofcontents
    
    
    \newpage
    \pagenumbering{arabic}
    %\linenumbers\renewcommand\thelinenumber{\color{black!50}\arabic{linenumber}}
            \input{0 - introduction/main.tex}
        \part{Research}
            \input{1 - low-noise PiC models/main.tex}
            \input{2 - kinetic component/main.tex}
            \input{3 - fluid component/main.tex}
            \input{4 - numerical implementation/main.tex}
        \part{Project Overview}
            \input{5 - research plan/main.tex}
            \input{6 - summary/main.tex}
    
    
    %\section{}
    \newpage
    \pagenumbering{gobble}
        \printbibliography


    \newpage
    \pagenumbering{roman}
    \appendix
        \part{Appendices}
            \input{8 - Hilbert complexes/main.tex}
            \input{9 - weak conservation proofs/main.tex}
\end{document}


\title{\BA{Title in Progress...}}
\author{Boris Andrews}
\affil{Mathematical Institute, University of Oxford}
\date{\today}


\begin{document}
    \pagenumbering{gobble}
    \maketitle
    
    
    \begin{abstract}
        Magnetic confinement reactors---in particular tokamaks---offer one of the most promising options for achieving practical nuclear fusion, with the potential to provide virtually limitless, clean energy. The theoretical and numerical modeling of tokamak plasmas is simultaneously an essential component of effective reactor design, and a great research barrier. Tokamak operational conditions exhibit comparatively low Knudsen numbers. Kinetic effects, including kinetic waves and instabilities, Landau damping, bump-on-tail instabilities and more, are therefore highly influential in tokamak plasma dynamics. Purely fluid models are inherently incapable of capturing these effects, whereas the high dimensionality in purely kinetic models render them practically intractable for most relevant purposes.

        We consider a $\delta\!f$ decomposition model, with a macroscopic fluid background and microscopic kinetic correction, both fully coupled to each other. A similar manner of discretization is proposed to that used in the recent \texttt{STRUPHY} code \cite{Holderied_Possanner_Wang_2021, Holderied_2022, Li_et_al_2023} with a finite-element model for the background and a pseudo-particle/PiC model for the correction.

        The fluid background satisfies the full, non-linear, resistive, compressible, Hall MHD equations. \cite{Laakmann_Hu_Farrell_2022} introduces finite-element(-in-space) implicit timesteppers for the incompressible analogue to this system with structure-preserving (SP) properties in the ideal case, alongside parameter-robust preconditioners. We show that these timesteppers can derive from a finite-element-in-time (FET) (and finite-element-in-space) interpretation. The benefits of this reformulation are discussed, including the derivation of timesteppers that are higher order in time, and the quantifiable dissipative SP properties in the non-ideal, resistive case.
        
        We discuss possible options for extending this FET approach to timesteppers for the compressible case.

        The kinetic corrections satisfy linearized Boltzmann equations. Using a Lénard--Bernstein collision operator, these take Fokker--Planck-like forms \cite{Fokker_1914, Planck_1917} wherein pseudo-particles in the numerical model obey the neoclassical transport equations, with particle-independent Brownian drift terms. This offers a rigorous methodology for incorporating collisions into the particle transport model, without coupling the equations of motions for each particle.
        
        Works by Chen, Chacón et al. \cite{Chen_Chacón_Barnes_2011, Chacón_Chen_Barnes_2013, Chen_Chacón_2014, Chen_Chacón_2015} have developed structure-preserving particle pushers for neoclassical transport in the Vlasov equations, derived from Crank--Nicolson integrators. We show these too can can derive from a FET interpretation, similarly offering potential extensions to higher-order-in-time particle pushers. The FET formulation is used also to consider how the stochastic drift terms can be incorporated into the pushers. Stochastic gyrokinetic expansions are also discussed.

        Different options for the numerical implementation of these schemes are considered.

        Due to the efficacy of FET in the development of SP timesteppers for both the fluid and kinetic component, we hope this approach will prove effective in the future for developing SP timesteppers for the full hybrid model. We hope this will give us the opportunity to incorporate previously inaccessible kinetic effects into the highly effective, modern, finite-element MHD models.
    \end{abstract}
    
    
    \newpage
    \tableofcontents
    
    
    \newpage
    \pagenumbering{arabic}
    %\linenumbers\renewcommand\thelinenumber{\color{black!50}\arabic{linenumber}}
            \documentclass[12pt, a4paper]{report}

\input{template/main.tex}

\title{\BA{Title in Progress...}}
\author{Boris Andrews}
\affil{Mathematical Institute, University of Oxford}
\date{\today}


\begin{document}
    \pagenumbering{gobble}
    \maketitle
    
    
    \begin{abstract}
        Magnetic confinement reactors---in particular tokamaks---offer one of the most promising options for achieving practical nuclear fusion, with the potential to provide virtually limitless, clean energy. The theoretical and numerical modeling of tokamak plasmas is simultaneously an essential component of effective reactor design, and a great research barrier. Tokamak operational conditions exhibit comparatively low Knudsen numbers. Kinetic effects, including kinetic waves and instabilities, Landau damping, bump-on-tail instabilities and more, are therefore highly influential in tokamak plasma dynamics. Purely fluid models are inherently incapable of capturing these effects, whereas the high dimensionality in purely kinetic models render them practically intractable for most relevant purposes.

        We consider a $\delta\!f$ decomposition model, with a macroscopic fluid background and microscopic kinetic correction, both fully coupled to each other. A similar manner of discretization is proposed to that used in the recent \texttt{STRUPHY} code \cite{Holderied_Possanner_Wang_2021, Holderied_2022, Li_et_al_2023} with a finite-element model for the background and a pseudo-particle/PiC model for the correction.

        The fluid background satisfies the full, non-linear, resistive, compressible, Hall MHD equations. \cite{Laakmann_Hu_Farrell_2022} introduces finite-element(-in-space) implicit timesteppers for the incompressible analogue to this system with structure-preserving (SP) properties in the ideal case, alongside parameter-robust preconditioners. We show that these timesteppers can derive from a finite-element-in-time (FET) (and finite-element-in-space) interpretation. The benefits of this reformulation are discussed, including the derivation of timesteppers that are higher order in time, and the quantifiable dissipative SP properties in the non-ideal, resistive case.
        
        We discuss possible options for extending this FET approach to timesteppers for the compressible case.

        The kinetic corrections satisfy linearized Boltzmann equations. Using a Lénard--Bernstein collision operator, these take Fokker--Planck-like forms \cite{Fokker_1914, Planck_1917} wherein pseudo-particles in the numerical model obey the neoclassical transport equations, with particle-independent Brownian drift terms. This offers a rigorous methodology for incorporating collisions into the particle transport model, without coupling the equations of motions for each particle.
        
        Works by Chen, Chacón et al. \cite{Chen_Chacón_Barnes_2011, Chacón_Chen_Barnes_2013, Chen_Chacón_2014, Chen_Chacón_2015} have developed structure-preserving particle pushers for neoclassical transport in the Vlasov equations, derived from Crank--Nicolson integrators. We show these too can can derive from a FET interpretation, similarly offering potential extensions to higher-order-in-time particle pushers. The FET formulation is used also to consider how the stochastic drift terms can be incorporated into the pushers. Stochastic gyrokinetic expansions are also discussed.

        Different options for the numerical implementation of these schemes are considered.

        Due to the efficacy of FET in the development of SP timesteppers for both the fluid and kinetic component, we hope this approach will prove effective in the future for developing SP timesteppers for the full hybrid model. We hope this will give us the opportunity to incorporate previously inaccessible kinetic effects into the highly effective, modern, finite-element MHD models.
    \end{abstract}
    
    
    \newpage
    \tableofcontents
    
    
    \newpage
    \pagenumbering{arabic}
    %\linenumbers\renewcommand\thelinenumber{\color{black!50}\arabic{linenumber}}
            \input{0 - introduction/main.tex}
        \part{Research}
            \input{1 - low-noise PiC models/main.tex}
            \input{2 - kinetic component/main.tex}
            \input{3 - fluid component/main.tex}
            \input{4 - numerical implementation/main.tex}
        \part{Project Overview}
            \input{5 - research plan/main.tex}
            \input{6 - summary/main.tex}
    
    
    %\section{}
    \newpage
    \pagenumbering{gobble}
        \printbibliography


    \newpage
    \pagenumbering{roman}
    \appendix
        \part{Appendices}
            \input{8 - Hilbert complexes/main.tex}
            \input{9 - weak conservation proofs/main.tex}
\end{document}

        \part{Research}
            \documentclass[12pt, a4paper]{report}

\input{template/main.tex}

\title{\BA{Title in Progress...}}
\author{Boris Andrews}
\affil{Mathematical Institute, University of Oxford}
\date{\today}


\begin{document}
    \pagenumbering{gobble}
    \maketitle
    
    
    \begin{abstract}
        Magnetic confinement reactors---in particular tokamaks---offer one of the most promising options for achieving practical nuclear fusion, with the potential to provide virtually limitless, clean energy. The theoretical and numerical modeling of tokamak plasmas is simultaneously an essential component of effective reactor design, and a great research barrier. Tokamak operational conditions exhibit comparatively low Knudsen numbers. Kinetic effects, including kinetic waves and instabilities, Landau damping, bump-on-tail instabilities and more, are therefore highly influential in tokamak plasma dynamics. Purely fluid models are inherently incapable of capturing these effects, whereas the high dimensionality in purely kinetic models render them practically intractable for most relevant purposes.

        We consider a $\delta\!f$ decomposition model, with a macroscopic fluid background and microscopic kinetic correction, both fully coupled to each other. A similar manner of discretization is proposed to that used in the recent \texttt{STRUPHY} code \cite{Holderied_Possanner_Wang_2021, Holderied_2022, Li_et_al_2023} with a finite-element model for the background and a pseudo-particle/PiC model for the correction.

        The fluid background satisfies the full, non-linear, resistive, compressible, Hall MHD equations. \cite{Laakmann_Hu_Farrell_2022} introduces finite-element(-in-space) implicit timesteppers for the incompressible analogue to this system with structure-preserving (SP) properties in the ideal case, alongside parameter-robust preconditioners. We show that these timesteppers can derive from a finite-element-in-time (FET) (and finite-element-in-space) interpretation. The benefits of this reformulation are discussed, including the derivation of timesteppers that are higher order in time, and the quantifiable dissipative SP properties in the non-ideal, resistive case.
        
        We discuss possible options for extending this FET approach to timesteppers for the compressible case.

        The kinetic corrections satisfy linearized Boltzmann equations. Using a Lénard--Bernstein collision operator, these take Fokker--Planck-like forms \cite{Fokker_1914, Planck_1917} wherein pseudo-particles in the numerical model obey the neoclassical transport equations, with particle-independent Brownian drift terms. This offers a rigorous methodology for incorporating collisions into the particle transport model, without coupling the equations of motions for each particle.
        
        Works by Chen, Chacón et al. \cite{Chen_Chacón_Barnes_2011, Chacón_Chen_Barnes_2013, Chen_Chacón_2014, Chen_Chacón_2015} have developed structure-preserving particle pushers for neoclassical transport in the Vlasov equations, derived from Crank--Nicolson integrators. We show these too can can derive from a FET interpretation, similarly offering potential extensions to higher-order-in-time particle pushers. The FET formulation is used also to consider how the stochastic drift terms can be incorporated into the pushers. Stochastic gyrokinetic expansions are also discussed.

        Different options for the numerical implementation of these schemes are considered.

        Due to the efficacy of FET in the development of SP timesteppers for both the fluid and kinetic component, we hope this approach will prove effective in the future for developing SP timesteppers for the full hybrid model. We hope this will give us the opportunity to incorporate previously inaccessible kinetic effects into the highly effective, modern, finite-element MHD models.
    \end{abstract}
    
    
    \newpage
    \tableofcontents
    
    
    \newpage
    \pagenumbering{arabic}
    %\linenumbers\renewcommand\thelinenumber{\color{black!50}\arabic{linenumber}}
            \input{0 - introduction/main.tex}
        \part{Research}
            \input{1 - low-noise PiC models/main.tex}
            \input{2 - kinetic component/main.tex}
            \input{3 - fluid component/main.tex}
            \input{4 - numerical implementation/main.tex}
        \part{Project Overview}
            \input{5 - research plan/main.tex}
            \input{6 - summary/main.tex}
    
    
    %\section{}
    \newpage
    \pagenumbering{gobble}
        \printbibliography


    \newpage
    \pagenumbering{roman}
    \appendix
        \part{Appendices}
            \input{8 - Hilbert complexes/main.tex}
            \input{9 - weak conservation proofs/main.tex}
\end{document}

            \documentclass[12pt, a4paper]{report}

\input{template/main.tex}

\title{\BA{Title in Progress...}}
\author{Boris Andrews}
\affil{Mathematical Institute, University of Oxford}
\date{\today}


\begin{document}
    \pagenumbering{gobble}
    \maketitle
    
    
    \begin{abstract}
        Magnetic confinement reactors---in particular tokamaks---offer one of the most promising options for achieving practical nuclear fusion, with the potential to provide virtually limitless, clean energy. The theoretical and numerical modeling of tokamak plasmas is simultaneously an essential component of effective reactor design, and a great research barrier. Tokamak operational conditions exhibit comparatively low Knudsen numbers. Kinetic effects, including kinetic waves and instabilities, Landau damping, bump-on-tail instabilities and more, are therefore highly influential in tokamak plasma dynamics. Purely fluid models are inherently incapable of capturing these effects, whereas the high dimensionality in purely kinetic models render them practically intractable for most relevant purposes.

        We consider a $\delta\!f$ decomposition model, with a macroscopic fluid background and microscopic kinetic correction, both fully coupled to each other. A similar manner of discretization is proposed to that used in the recent \texttt{STRUPHY} code \cite{Holderied_Possanner_Wang_2021, Holderied_2022, Li_et_al_2023} with a finite-element model for the background and a pseudo-particle/PiC model for the correction.

        The fluid background satisfies the full, non-linear, resistive, compressible, Hall MHD equations. \cite{Laakmann_Hu_Farrell_2022} introduces finite-element(-in-space) implicit timesteppers for the incompressible analogue to this system with structure-preserving (SP) properties in the ideal case, alongside parameter-robust preconditioners. We show that these timesteppers can derive from a finite-element-in-time (FET) (and finite-element-in-space) interpretation. The benefits of this reformulation are discussed, including the derivation of timesteppers that are higher order in time, and the quantifiable dissipative SP properties in the non-ideal, resistive case.
        
        We discuss possible options for extending this FET approach to timesteppers for the compressible case.

        The kinetic corrections satisfy linearized Boltzmann equations. Using a Lénard--Bernstein collision operator, these take Fokker--Planck-like forms \cite{Fokker_1914, Planck_1917} wherein pseudo-particles in the numerical model obey the neoclassical transport equations, with particle-independent Brownian drift terms. This offers a rigorous methodology for incorporating collisions into the particle transport model, without coupling the equations of motions for each particle.
        
        Works by Chen, Chacón et al. \cite{Chen_Chacón_Barnes_2011, Chacón_Chen_Barnes_2013, Chen_Chacón_2014, Chen_Chacón_2015} have developed structure-preserving particle pushers for neoclassical transport in the Vlasov equations, derived from Crank--Nicolson integrators. We show these too can can derive from a FET interpretation, similarly offering potential extensions to higher-order-in-time particle pushers. The FET formulation is used also to consider how the stochastic drift terms can be incorporated into the pushers. Stochastic gyrokinetic expansions are also discussed.

        Different options for the numerical implementation of these schemes are considered.

        Due to the efficacy of FET in the development of SP timesteppers for both the fluid and kinetic component, we hope this approach will prove effective in the future for developing SP timesteppers for the full hybrid model. We hope this will give us the opportunity to incorporate previously inaccessible kinetic effects into the highly effective, modern, finite-element MHD models.
    \end{abstract}
    
    
    \newpage
    \tableofcontents
    
    
    \newpage
    \pagenumbering{arabic}
    %\linenumbers\renewcommand\thelinenumber{\color{black!50}\arabic{linenumber}}
            \input{0 - introduction/main.tex}
        \part{Research}
            \input{1 - low-noise PiC models/main.tex}
            \input{2 - kinetic component/main.tex}
            \input{3 - fluid component/main.tex}
            \input{4 - numerical implementation/main.tex}
        \part{Project Overview}
            \input{5 - research plan/main.tex}
            \input{6 - summary/main.tex}
    
    
    %\section{}
    \newpage
    \pagenumbering{gobble}
        \printbibliography


    \newpage
    \pagenumbering{roman}
    \appendix
        \part{Appendices}
            \input{8 - Hilbert complexes/main.tex}
            \input{9 - weak conservation proofs/main.tex}
\end{document}

            \documentclass[12pt, a4paper]{report}

\input{template/main.tex}

\title{\BA{Title in Progress...}}
\author{Boris Andrews}
\affil{Mathematical Institute, University of Oxford}
\date{\today}


\begin{document}
    \pagenumbering{gobble}
    \maketitle
    
    
    \begin{abstract}
        Magnetic confinement reactors---in particular tokamaks---offer one of the most promising options for achieving practical nuclear fusion, with the potential to provide virtually limitless, clean energy. The theoretical and numerical modeling of tokamak plasmas is simultaneously an essential component of effective reactor design, and a great research barrier. Tokamak operational conditions exhibit comparatively low Knudsen numbers. Kinetic effects, including kinetic waves and instabilities, Landau damping, bump-on-tail instabilities and more, are therefore highly influential in tokamak plasma dynamics. Purely fluid models are inherently incapable of capturing these effects, whereas the high dimensionality in purely kinetic models render them practically intractable for most relevant purposes.

        We consider a $\delta\!f$ decomposition model, with a macroscopic fluid background and microscopic kinetic correction, both fully coupled to each other. A similar manner of discretization is proposed to that used in the recent \texttt{STRUPHY} code \cite{Holderied_Possanner_Wang_2021, Holderied_2022, Li_et_al_2023} with a finite-element model for the background and a pseudo-particle/PiC model for the correction.

        The fluid background satisfies the full, non-linear, resistive, compressible, Hall MHD equations. \cite{Laakmann_Hu_Farrell_2022} introduces finite-element(-in-space) implicit timesteppers for the incompressible analogue to this system with structure-preserving (SP) properties in the ideal case, alongside parameter-robust preconditioners. We show that these timesteppers can derive from a finite-element-in-time (FET) (and finite-element-in-space) interpretation. The benefits of this reformulation are discussed, including the derivation of timesteppers that are higher order in time, and the quantifiable dissipative SP properties in the non-ideal, resistive case.
        
        We discuss possible options for extending this FET approach to timesteppers for the compressible case.

        The kinetic corrections satisfy linearized Boltzmann equations. Using a Lénard--Bernstein collision operator, these take Fokker--Planck-like forms \cite{Fokker_1914, Planck_1917} wherein pseudo-particles in the numerical model obey the neoclassical transport equations, with particle-independent Brownian drift terms. This offers a rigorous methodology for incorporating collisions into the particle transport model, without coupling the equations of motions for each particle.
        
        Works by Chen, Chacón et al. \cite{Chen_Chacón_Barnes_2011, Chacón_Chen_Barnes_2013, Chen_Chacón_2014, Chen_Chacón_2015} have developed structure-preserving particle pushers for neoclassical transport in the Vlasov equations, derived from Crank--Nicolson integrators. We show these too can can derive from a FET interpretation, similarly offering potential extensions to higher-order-in-time particle pushers. The FET formulation is used also to consider how the stochastic drift terms can be incorporated into the pushers. Stochastic gyrokinetic expansions are also discussed.

        Different options for the numerical implementation of these schemes are considered.

        Due to the efficacy of FET in the development of SP timesteppers for both the fluid and kinetic component, we hope this approach will prove effective in the future for developing SP timesteppers for the full hybrid model. We hope this will give us the opportunity to incorporate previously inaccessible kinetic effects into the highly effective, modern, finite-element MHD models.
    \end{abstract}
    
    
    \newpage
    \tableofcontents
    
    
    \newpage
    \pagenumbering{arabic}
    %\linenumbers\renewcommand\thelinenumber{\color{black!50}\arabic{linenumber}}
            \input{0 - introduction/main.tex}
        \part{Research}
            \input{1 - low-noise PiC models/main.tex}
            \input{2 - kinetic component/main.tex}
            \input{3 - fluid component/main.tex}
            \input{4 - numerical implementation/main.tex}
        \part{Project Overview}
            \input{5 - research plan/main.tex}
            \input{6 - summary/main.tex}
    
    
    %\section{}
    \newpage
    \pagenumbering{gobble}
        \printbibliography


    \newpage
    \pagenumbering{roman}
    \appendix
        \part{Appendices}
            \input{8 - Hilbert complexes/main.tex}
            \input{9 - weak conservation proofs/main.tex}
\end{document}

            \documentclass[12pt, a4paper]{report}

\input{template/main.tex}

\title{\BA{Title in Progress...}}
\author{Boris Andrews}
\affil{Mathematical Institute, University of Oxford}
\date{\today}


\begin{document}
    \pagenumbering{gobble}
    \maketitle
    
    
    \begin{abstract}
        Magnetic confinement reactors---in particular tokamaks---offer one of the most promising options for achieving practical nuclear fusion, with the potential to provide virtually limitless, clean energy. The theoretical and numerical modeling of tokamak plasmas is simultaneously an essential component of effective reactor design, and a great research barrier. Tokamak operational conditions exhibit comparatively low Knudsen numbers. Kinetic effects, including kinetic waves and instabilities, Landau damping, bump-on-tail instabilities and more, are therefore highly influential in tokamak plasma dynamics. Purely fluid models are inherently incapable of capturing these effects, whereas the high dimensionality in purely kinetic models render them practically intractable for most relevant purposes.

        We consider a $\delta\!f$ decomposition model, with a macroscopic fluid background and microscopic kinetic correction, both fully coupled to each other. A similar manner of discretization is proposed to that used in the recent \texttt{STRUPHY} code \cite{Holderied_Possanner_Wang_2021, Holderied_2022, Li_et_al_2023} with a finite-element model for the background and a pseudo-particle/PiC model for the correction.

        The fluid background satisfies the full, non-linear, resistive, compressible, Hall MHD equations. \cite{Laakmann_Hu_Farrell_2022} introduces finite-element(-in-space) implicit timesteppers for the incompressible analogue to this system with structure-preserving (SP) properties in the ideal case, alongside parameter-robust preconditioners. We show that these timesteppers can derive from a finite-element-in-time (FET) (and finite-element-in-space) interpretation. The benefits of this reformulation are discussed, including the derivation of timesteppers that are higher order in time, and the quantifiable dissipative SP properties in the non-ideal, resistive case.
        
        We discuss possible options for extending this FET approach to timesteppers for the compressible case.

        The kinetic corrections satisfy linearized Boltzmann equations. Using a Lénard--Bernstein collision operator, these take Fokker--Planck-like forms \cite{Fokker_1914, Planck_1917} wherein pseudo-particles in the numerical model obey the neoclassical transport equations, with particle-independent Brownian drift terms. This offers a rigorous methodology for incorporating collisions into the particle transport model, without coupling the equations of motions for each particle.
        
        Works by Chen, Chacón et al. \cite{Chen_Chacón_Barnes_2011, Chacón_Chen_Barnes_2013, Chen_Chacón_2014, Chen_Chacón_2015} have developed structure-preserving particle pushers for neoclassical transport in the Vlasov equations, derived from Crank--Nicolson integrators. We show these too can can derive from a FET interpretation, similarly offering potential extensions to higher-order-in-time particle pushers. The FET formulation is used also to consider how the stochastic drift terms can be incorporated into the pushers. Stochastic gyrokinetic expansions are also discussed.

        Different options for the numerical implementation of these schemes are considered.

        Due to the efficacy of FET in the development of SP timesteppers for both the fluid and kinetic component, we hope this approach will prove effective in the future for developing SP timesteppers for the full hybrid model. We hope this will give us the opportunity to incorporate previously inaccessible kinetic effects into the highly effective, modern, finite-element MHD models.
    \end{abstract}
    
    
    \newpage
    \tableofcontents
    
    
    \newpage
    \pagenumbering{arabic}
    %\linenumbers\renewcommand\thelinenumber{\color{black!50}\arabic{linenumber}}
            \input{0 - introduction/main.tex}
        \part{Research}
            \input{1 - low-noise PiC models/main.tex}
            \input{2 - kinetic component/main.tex}
            \input{3 - fluid component/main.tex}
            \input{4 - numerical implementation/main.tex}
        \part{Project Overview}
            \input{5 - research plan/main.tex}
            \input{6 - summary/main.tex}
    
    
    %\section{}
    \newpage
    \pagenumbering{gobble}
        \printbibliography


    \newpage
    \pagenumbering{roman}
    \appendix
        \part{Appendices}
            \input{8 - Hilbert complexes/main.tex}
            \input{9 - weak conservation proofs/main.tex}
\end{document}

        \part{Project Overview}
            \documentclass[12pt, a4paper]{report}

\input{template/main.tex}

\title{\BA{Title in Progress...}}
\author{Boris Andrews}
\affil{Mathematical Institute, University of Oxford}
\date{\today}


\begin{document}
    \pagenumbering{gobble}
    \maketitle
    
    
    \begin{abstract}
        Magnetic confinement reactors---in particular tokamaks---offer one of the most promising options for achieving practical nuclear fusion, with the potential to provide virtually limitless, clean energy. The theoretical and numerical modeling of tokamak plasmas is simultaneously an essential component of effective reactor design, and a great research barrier. Tokamak operational conditions exhibit comparatively low Knudsen numbers. Kinetic effects, including kinetic waves and instabilities, Landau damping, bump-on-tail instabilities and more, are therefore highly influential in tokamak plasma dynamics. Purely fluid models are inherently incapable of capturing these effects, whereas the high dimensionality in purely kinetic models render them practically intractable for most relevant purposes.

        We consider a $\delta\!f$ decomposition model, with a macroscopic fluid background and microscopic kinetic correction, both fully coupled to each other. A similar manner of discretization is proposed to that used in the recent \texttt{STRUPHY} code \cite{Holderied_Possanner_Wang_2021, Holderied_2022, Li_et_al_2023} with a finite-element model for the background and a pseudo-particle/PiC model for the correction.

        The fluid background satisfies the full, non-linear, resistive, compressible, Hall MHD equations. \cite{Laakmann_Hu_Farrell_2022} introduces finite-element(-in-space) implicit timesteppers for the incompressible analogue to this system with structure-preserving (SP) properties in the ideal case, alongside parameter-robust preconditioners. We show that these timesteppers can derive from a finite-element-in-time (FET) (and finite-element-in-space) interpretation. The benefits of this reformulation are discussed, including the derivation of timesteppers that are higher order in time, and the quantifiable dissipative SP properties in the non-ideal, resistive case.
        
        We discuss possible options for extending this FET approach to timesteppers for the compressible case.

        The kinetic corrections satisfy linearized Boltzmann equations. Using a Lénard--Bernstein collision operator, these take Fokker--Planck-like forms \cite{Fokker_1914, Planck_1917} wherein pseudo-particles in the numerical model obey the neoclassical transport equations, with particle-independent Brownian drift terms. This offers a rigorous methodology for incorporating collisions into the particle transport model, without coupling the equations of motions for each particle.
        
        Works by Chen, Chacón et al. \cite{Chen_Chacón_Barnes_2011, Chacón_Chen_Barnes_2013, Chen_Chacón_2014, Chen_Chacón_2015} have developed structure-preserving particle pushers for neoclassical transport in the Vlasov equations, derived from Crank--Nicolson integrators. We show these too can can derive from a FET interpretation, similarly offering potential extensions to higher-order-in-time particle pushers. The FET formulation is used also to consider how the stochastic drift terms can be incorporated into the pushers. Stochastic gyrokinetic expansions are also discussed.

        Different options for the numerical implementation of these schemes are considered.

        Due to the efficacy of FET in the development of SP timesteppers for both the fluid and kinetic component, we hope this approach will prove effective in the future for developing SP timesteppers for the full hybrid model. We hope this will give us the opportunity to incorporate previously inaccessible kinetic effects into the highly effective, modern, finite-element MHD models.
    \end{abstract}
    
    
    \newpage
    \tableofcontents
    
    
    \newpage
    \pagenumbering{arabic}
    %\linenumbers\renewcommand\thelinenumber{\color{black!50}\arabic{linenumber}}
            \input{0 - introduction/main.tex}
        \part{Research}
            \input{1 - low-noise PiC models/main.tex}
            \input{2 - kinetic component/main.tex}
            \input{3 - fluid component/main.tex}
            \input{4 - numerical implementation/main.tex}
        \part{Project Overview}
            \input{5 - research plan/main.tex}
            \input{6 - summary/main.tex}
    
    
    %\section{}
    \newpage
    \pagenumbering{gobble}
        \printbibliography


    \newpage
    \pagenumbering{roman}
    \appendix
        \part{Appendices}
            \input{8 - Hilbert complexes/main.tex}
            \input{9 - weak conservation proofs/main.tex}
\end{document}

            \documentclass[12pt, a4paper]{report}

\input{template/main.tex}

\title{\BA{Title in Progress...}}
\author{Boris Andrews}
\affil{Mathematical Institute, University of Oxford}
\date{\today}


\begin{document}
    \pagenumbering{gobble}
    \maketitle
    
    
    \begin{abstract}
        Magnetic confinement reactors---in particular tokamaks---offer one of the most promising options for achieving practical nuclear fusion, with the potential to provide virtually limitless, clean energy. The theoretical and numerical modeling of tokamak plasmas is simultaneously an essential component of effective reactor design, and a great research barrier. Tokamak operational conditions exhibit comparatively low Knudsen numbers. Kinetic effects, including kinetic waves and instabilities, Landau damping, bump-on-tail instabilities and more, are therefore highly influential in tokamak plasma dynamics. Purely fluid models are inherently incapable of capturing these effects, whereas the high dimensionality in purely kinetic models render them practically intractable for most relevant purposes.

        We consider a $\delta\!f$ decomposition model, with a macroscopic fluid background and microscopic kinetic correction, both fully coupled to each other. A similar manner of discretization is proposed to that used in the recent \texttt{STRUPHY} code \cite{Holderied_Possanner_Wang_2021, Holderied_2022, Li_et_al_2023} with a finite-element model for the background and a pseudo-particle/PiC model for the correction.

        The fluid background satisfies the full, non-linear, resistive, compressible, Hall MHD equations. \cite{Laakmann_Hu_Farrell_2022} introduces finite-element(-in-space) implicit timesteppers for the incompressible analogue to this system with structure-preserving (SP) properties in the ideal case, alongside parameter-robust preconditioners. We show that these timesteppers can derive from a finite-element-in-time (FET) (and finite-element-in-space) interpretation. The benefits of this reformulation are discussed, including the derivation of timesteppers that are higher order in time, and the quantifiable dissipative SP properties in the non-ideal, resistive case.
        
        We discuss possible options for extending this FET approach to timesteppers for the compressible case.

        The kinetic corrections satisfy linearized Boltzmann equations. Using a Lénard--Bernstein collision operator, these take Fokker--Planck-like forms \cite{Fokker_1914, Planck_1917} wherein pseudo-particles in the numerical model obey the neoclassical transport equations, with particle-independent Brownian drift terms. This offers a rigorous methodology for incorporating collisions into the particle transport model, without coupling the equations of motions for each particle.
        
        Works by Chen, Chacón et al. \cite{Chen_Chacón_Barnes_2011, Chacón_Chen_Barnes_2013, Chen_Chacón_2014, Chen_Chacón_2015} have developed structure-preserving particle pushers for neoclassical transport in the Vlasov equations, derived from Crank--Nicolson integrators. We show these too can can derive from a FET interpretation, similarly offering potential extensions to higher-order-in-time particle pushers. The FET formulation is used also to consider how the stochastic drift terms can be incorporated into the pushers. Stochastic gyrokinetic expansions are also discussed.

        Different options for the numerical implementation of these schemes are considered.

        Due to the efficacy of FET in the development of SP timesteppers for both the fluid and kinetic component, we hope this approach will prove effective in the future for developing SP timesteppers for the full hybrid model. We hope this will give us the opportunity to incorporate previously inaccessible kinetic effects into the highly effective, modern, finite-element MHD models.
    \end{abstract}
    
    
    \newpage
    \tableofcontents
    
    
    \newpage
    \pagenumbering{arabic}
    %\linenumbers\renewcommand\thelinenumber{\color{black!50}\arabic{linenumber}}
            \input{0 - introduction/main.tex}
        \part{Research}
            \input{1 - low-noise PiC models/main.tex}
            \input{2 - kinetic component/main.tex}
            \input{3 - fluid component/main.tex}
            \input{4 - numerical implementation/main.tex}
        \part{Project Overview}
            \input{5 - research plan/main.tex}
            \input{6 - summary/main.tex}
    
    
    %\section{}
    \newpage
    \pagenumbering{gobble}
        \printbibliography


    \newpage
    \pagenumbering{roman}
    \appendix
        \part{Appendices}
            \input{8 - Hilbert complexes/main.tex}
            \input{9 - weak conservation proofs/main.tex}
\end{document}

    
    
    %\section{}
    \newpage
    \pagenumbering{gobble}
        \printbibliography


    \newpage
    \pagenumbering{roman}
    \appendix
        \part{Appendices}
            \documentclass[12pt, a4paper]{report}

\input{template/main.tex}

\title{\BA{Title in Progress...}}
\author{Boris Andrews}
\affil{Mathematical Institute, University of Oxford}
\date{\today}


\begin{document}
    \pagenumbering{gobble}
    \maketitle
    
    
    \begin{abstract}
        Magnetic confinement reactors---in particular tokamaks---offer one of the most promising options for achieving practical nuclear fusion, with the potential to provide virtually limitless, clean energy. The theoretical and numerical modeling of tokamak plasmas is simultaneously an essential component of effective reactor design, and a great research barrier. Tokamak operational conditions exhibit comparatively low Knudsen numbers. Kinetic effects, including kinetic waves and instabilities, Landau damping, bump-on-tail instabilities and more, are therefore highly influential in tokamak plasma dynamics. Purely fluid models are inherently incapable of capturing these effects, whereas the high dimensionality in purely kinetic models render them practically intractable for most relevant purposes.

        We consider a $\delta\!f$ decomposition model, with a macroscopic fluid background and microscopic kinetic correction, both fully coupled to each other. A similar manner of discretization is proposed to that used in the recent \texttt{STRUPHY} code \cite{Holderied_Possanner_Wang_2021, Holderied_2022, Li_et_al_2023} with a finite-element model for the background and a pseudo-particle/PiC model for the correction.

        The fluid background satisfies the full, non-linear, resistive, compressible, Hall MHD equations. \cite{Laakmann_Hu_Farrell_2022} introduces finite-element(-in-space) implicit timesteppers for the incompressible analogue to this system with structure-preserving (SP) properties in the ideal case, alongside parameter-robust preconditioners. We show that these timesteppers can derive from a finite-element-in-time (FET) (and finite-element-in-space) interpretation. The benefits of this reformulation are discussed, including the derivation of timesteppers that are higher order in time, and the quantifiable dissipative SP properties in the non-ideal, resistive case.
        
        We discuss possible options for extending this FET approach to timesteppers for the compressible case.

        The kinetic corrections satisfy linearized Boltzmann equations. Using a Lénard--Bernstein collision operator, these take Fokker--Planck-like forms \cite{Fokker_1914, Planck_1917} wherein pseudo-particles in the numerical model obey the neoclassical transport equations, with particle-independent Brownian drift terms. This offers a rigorous methodology for incorporating collisions into the particle transport model, without coupling the equations of motions for each particle.
        
        Works by Chen, Chacón et al. \cite{Chen_Chacón_Barnes_2011, Chacón_Chen_Barnes_2013, Chen_Chacón_2014, Chen_Chacón_2015} have developed structure-preserving particle pushers for neoclassical transport in the Vlasov equations, derived from Crank--Nicolson integrators. We show these too can can derive from a FET interpretation, similarly offering potential extensions to higher-order-in-time particle pushers. The FET formulation is used also to consider how the stochastic drift terms can be incorporated into the pushers. Stochastic gyrokinetic expansions are also discussed.

        Different options for the numerical implementation of these schemes are considered.

        Due to the efficacy of FET in the development of SP timesteppers for both the fluid and kinetic component, we hope this approach will prove effective in the future for developing SP timesteppers for the full hybrid model. We hope this will give us the opportunity to incorporate previously inaccessible kinetic effects into the highly effective, modern, finite-element MHD models.
    \end{abstract}
    
    
    \newpage
    \tableofcontents
    
    
    \newpage
    \pagenumbering{arabic}
    %\linenumbers\renewcommand\thelinenumber{\color{black!50}\arabic{linenumber}}
            \input{0 - introduction/main.tex}
        \part{Research}
            \input{1 - low-noise PiC models/main.tex}
            \input{2 - kinetic component/main.tex}
            \input{3 - fluid component/main.tex}
            \input{4 - numerical implementation/main.tex}
        \part{Project Overview}
            \input{5 - research plan/main.tex}
            \input{6 - summary/main.tex}
    
    
    %\section{}
    \newpage
    \pagenumbering{gobble}
        \printbibliography


    \newpage
    \pagenumbering{roman}
    \appendix
        \part{Appendices}
            \input{8 - Hilbert complexes/main.tex}
            \input{9 - weak conservation proofs/main.tex}
\end{document}

            \documentclass[12pt, a4paper]{report}

\input{template/main.tex}

\title{\BA{Title in Progress...}}
\author{Boris Andrews}
\affil{Mathematical Institute, University of Oxford}
\date{\today}


\begin{document}
    \pagenumbering{gobble}
    \maketitle
    
    
    \begin{abstract}
        Magnetic confinement reactors---in particular tokamaks---offer one of the most promising options for achieving practical nuclear fusion, with the potential to provide virtually limitless, clean energy. The theoretical and numerical modeling of tokamak plasmas is simultaneously an essential component of effective reactor design, and a great research barrier. Tokamak operational conditions exhibit comparatively low Knudsen numbers. Kinetic effects, including kinetic waves and instabilities, Landau damping, bump-on-tail instabilities and more, are therefore highly influential in tokamak plasma dynamics. Purely fluid models are inherently incapable of capturing these effects, whereas the high dimensionality in purely kinetic models render them practically intractable for most relevant purposes.

        We consider a $\delta\!f$ decomposition model, with a macroscopic fluid background and microscopic kinetic correction, both fully coupled to each other. A similar manner of discretization is proposed to that used in the recent \texttt{STRUPHY} code \cite{Holderied_Possanner_Wang_2021, Holderied_2022, Li_et_al_2023} with a finite-element model for the background and a pseudo-particle/PiC model for the correction.

        The fluid background satisfies the full, non-linear, resistive, compressible, Hall MHD equations. \cite{Laakmann_Hu_Farrell_2022} introduces finite-element(-in-space) implicit timesteppers for the incompressible analogue to this system with structure-preserving (SP) properties in the ideal case, alongside parameter-robust preconditioners. We show that these timesteppers can derive from a finite-element-in-time (FET) (and finite-element-in-space) interpretation. The benefits of this reformulation are discussed, including the derivation of timesteppers that are higher order in time, and the quantifiable dissipative SP properties in the non-ideal, resistive case.
        
        We discuss possible options for extending this FET approach to timesteppers for the compressible case.

        The kinetic corrections satisfy linearized Boltzmann equations. Using a Lénard--Bernstein collision operator, these take Fokker--Planck-like forms \cite{Fokker_1914, Planck_1917} wherein pseudo-particles in the numerical model obey the neoclassical transport equations, with particle-independent Brownian drift terms. This offers a rigorous methodology for incorporating collisions into the particle transport model, without coupling the equations of motions for each particle.
        
        Works by Chen, Chacón et al. \cite{Chen_Chacón_Barnes_2011, Chacón_Chen_Barnes_2013, Chen_Chacón_2014, Chen_Chacón_2015} have developed structure-preserving particle pushers for neoclassical transport in the Vlasov equations, derived from Crank--Nicolson integrators. We show these too can can derive from a FET interpretation, similarly offering potential extensions to higher-order-in-time particle pushers. The FET formulation is used also to consider how the stochastic drift terms can be incorporated into the pushers. Stochastic gyrokinetic expansions are also discussed.

        Different options for the numerical implementation of these schemes are considered.

        Due to the efficacy of FET in the development of SP timesteppers for both the fluid and kinetic component, we hope this approach will prove effective in the future for developing SP timesteppers for the full hybrid model. We hope this will give us the opportunity to incorporate previously inaccessible kinetic effects into the highly effective, modern, finite-element MHD models.
    \end{abstract}
    
    
    \newpage
    \tableofcontents
    
    
    \newpage
    \pagenumbering{arabic}
    %\linenumbers\renewcommand\thelinenumber{\color{black!50}\arabic{linenumber}}
            \input{0 - introduction/main.tex}
        \part{Research}
            \input{1 - low-noise PiC models/main.tex}
            \input{2 - kinetic component/main.tex}
            \input{3 - fluid component/main.tex}
            \input{4 - numerical implementation/main.tex}
        \part{Project Overview}
            \input{5 - research plan/main.tex}
            \input{6 - summary/main.tex}
    
    
    %\section{}
    \newpage
    \pagenumbering{gobble}
        \printbibliography


    \newpage
    \pagenumbering{roman}
    \appendix
        \part{Appendices}
            \input{8 - Hilbert complexes/main.tex}
            \input{9 - weak conservation proofs/main.tex}
\end{document}

\end{document}

    
    
    %\section{}
    \newpage
    \pagenumbering{gobble}
        \printbibliography


    \newpage
    \pagenumbering{roman}
    \appendix
        \part{Appendices}
            \documentclass[12pt, a4paper]{report}

\documentclass[12pt, a4paper]{report}

\input{template/main.tex}

\title{\BA{Title in Progress...}}
\author{Boris Andrews}
\affil{Mathematical Institute, University of Oxford}
\date{\today}


\begin{document}
    \pagenumbering{gobble}
    \maketitle
    
    
    \begin{abstract}
        Magnetic confinement reactors---in particular tokamaks---offer one of the most promising options for achieving practical nuclear fusion, with the potential to provide virtually limitless, clean energy. The theoretical and numerical modeling of tokamak plasmas is simultaneously an essential component of effective reactor design, and a great research barrier. Tokamak operational conditions exhibit comparatively low Knudsen numbers. Kinetic effects, including kinetic waves and instabilities, Landau damping, bump-on-tail instabilities and more, are therefore highly influential in tokamak plasma dynamics. Purely fluid models are inherently incapable of capturing these effects, whereas the high dimensionality in purely kinetic models render them practically intractable for most relevant purposes.

        We consider a $\delta\!f$ decomposition model, with a macroscopic fluid background and microscopic kinetic correction, both fully coupled to each other. A similar manner of discretization is proposed to that used in the recent \texttt{STRUPHY} code \cite{Holderied_Possanner_Wang_2021, Holderied_2022, Li_et_al_2023} with a finite-element model for the background and a pseudo-particle/PiC model for the correction.

        The fluid background satisfies the full, non-linear, resistive, compressible, Hall MHD equations. \cite{Laakmann_Hu_Farrell_2022} introduces finite-element(-in-space) implicit timesteppers for the incompressible analogue to this system with structure-preserving (SP) properties in the ideal case, alongside parameter-robust preconditioners. We show that these timesteppers can derive from a finite-element-in-time (FET) (and finite-element-in-space) interpretation. The benefits of this reformulation are discussed, including the derivation of timesteppers that are higher order in time, and the quantifiable dissipative SP properties in the non-ideal, resistive case.
        
        We discuss possible options for extending this FET approach to timesteppers for the compressible case.

        The kinetic corrections satisfy linearized Boltzmann equations. Using a Lénard--Bernstein collision operator, these take Fokker--Planck-like forms \cite{Fokker_1914, Planck_1917} wherein pseudo-particles in the numerical model obey the neoclassical transport equations, with particle-independent Brownian drift terms. This offers a rigorous methodology for incorporating collisions into the particle transport model, without coupling the equations of motions for each particle.
        
        Works by Chen, Chacón et al. \cite{Chen_Chacón_Barnes_2011, Chacón_Chen_Barnes_2013, Chen_Chacón_2014, Chen_Chacón_2015} have developed structure-preserving particle pushers for neoclassical transport in the Vlasov equations, derived from Crank--Nicolson integrators. We show these too can can derive from a FET interpretation, similarly offering potential extensions to higher-order-in-time particle pushers. The FET formulation is used also to consider how the stochastic drift terms can be incorporated into the pushers. Stochastic gyrokinetic expansions are also discussed.

        Different options for the numerical implementation of these schemes are considered.

        Due to the efficacy of FET in the development of SP timesteppers for both the fluid and kinetic component, we hope this approach will prove effective in the future for developing SP timesteppers for the full hybrid model. We hope this will give us the opportunity to incorporate previously inaccessible kinetic effects into the highly effective, modern, finite-element MHD models.
    \end{abstract}
    
    
    \newpage
    \tableofcontents
    
    
    \newpage
    \pagenumbering{arabic}
    %\linenumbers\renewcommand\thelinenumber{\color{black!50}\arabic{linenumber}}
            \input{0 - introduction/main.tex}
        \part{Research}
            \input{1 - low-noise PiC models/main.tex}
            \input{2 - kinetic component/main.tex}
            \input{3 - fluid component/main.tex}
            \input{4 - numerical implementation/main.tex}
        \part{Project Overview}
            \input{5 - research plan/main.tex}
            \input{6 - summary/main.tex}
    
    
    %\section{}
    \newpage
    \pagenumbering{gobble}
        \printbibliography


    \newpage
    \pagenumbering{roman}
    \appendix
        \part{Appendices}
            \input{8 - Hilbert complexes/main.tex}
            \input{9 - weak conservation proofs/main.tex}
\end{document}


\title{\BA{Title in Progress...}}
\author{Boris Andrews}
\affil{Mathematical Institute, University of Oxford}
\date{\today}


\begin{document}
    \pagenumbering{gobble}
    \maketitle
    
    
    \begin{abstract}
        Magnetic confinement reactors---in particular tokamaks---offer one of the most promising options for achieving practical nuclear fusion, with the potential to provide virtually limitless, clean energy. The theoretical and numerical modeling of tokamak plasmas is simultaneously an essential component of effective reactor design, and a great research barrier. Tokamak operational conditions exhibit comparatively low Knudsen numbers. Kinetic effects, including kinetic waves and instabilities, Landau damping, bump-on-tail instabilities and more, are therefore highly influential in tokamak plasma dynamics. Purely fluid models are inherently incapable of capturing these effects, whereas the high dimensionality in purely kinetic models render them practically intractable for most relevant purposes.

        We consider a $\delta\!f$ decomposition model, with a macroscopic fluid background and microscopic kinetic correction, both fully coupled to each other. A similar manner of discretization is proposed to that used in the recent \texttt{STRUPHY} code \cite{Holderied_Possanner_Wang_2021, Holderied_2022, Li_et_al_2023} with a finite-element model for the background and a pseudo-particle/PiC model for the correction.

        The fluid background satisfies the full, non-linear, resistive, compressible, Hall MHD equations. \cite{Laakmann_Hu_Farrell_2022} introduces finite-element(-in-space) implicit timesteppers for the incompressible analogue to this system with structure-preserving (SP) properties in the ideal case, alongside parameter-robust preconditioners. We show that these timesteppers can derive from a finite-element-in-time (FET) (and finite-element-in-space) interpretation. The benefits of this reformulation are discussed, including the derivation of timesteppers that are higher order in time, and the quantifiable dissipative SP properties in the non-ideal, resistive case.
        
        We discuss possible options for extending this FET approach to timesteppers for the compressible case.

        The kinetic corrections satisfy linearized Boltzmann equations. Using a Lénard--Bernstein collision operator, these take Fokker--Planck-like forms \cite{Fokker_1914, Planck_1917} wherein pseudo-particles in the numerical model obey the neoclassical transport equations, with particle-independent Brownian drift terms. This offers a rigorous methodology for incorporating collisions into the particle transport model, without coupling the equations of motions for each particle.
        
        Works by Chen, Chacón et al. \cite{Chen_Chacón_Barnes_2011, Chacón_Chen_Barnes_2013, Chen_Chacón_2014, Chen_Chacón_2015} have developed structure-preserving particle pushers for neoclassical transport in the Vlasov equations, derived from Crank--Nicolson integrators. We show these too can can derive from a FET interpretation, similarly offering potential extensions to higher-order-in-time particle pushers. The FET formulation is used also to consider how the stochastic drift terms can be incorporated into the pushers. Stochastic gyrokinetic expansions are also discussed.

        Different options for the numerical implementation of these schemes are considered.

        Due to the efficacy of FET in the development of SP timesteppers for both the fluid and kinetic component, we hope this approach will prove effective in the future for developing SP timesteppers for the full hybrid model. We hope this will give us the opportunity to incorporate previously inaccessible kinetic effects into the highly effective, modern, finite-element MHD models.
    \end{abstract}
    
    
    \newpage
    \tableofcontents
    
    
    \newpage
    \pagenumbering{arabic}
    %\linenumbers\renewcommand\thelinenumber{\color{black!50}\arabic{linenumber}}
            \documentclass[12pt, a4paper]{report}

\input{template/main.tex}

\title{\BA{Title in Progress...}}
\author{Boris Andrews}
\affil{Mathematical Institute, University of Oxford}
\date{\today}


\begin{document}
    \pagenumbering{gobble}
    \maketitle
    
    
    \begin{abstract}
        Magnetic confinement reactors---in particular tokamaks---offer one of the most promising options for achieving practical nuclear fusion, with the potential to provide virtually limitless, clean energy. The theoretical and numerical modeling of tokamak plasmas is simultaneously an essential component of effective reactor design, and a great research barrier. Tokamak operational conditions exhibit comparatively low Knudsen numbers. Kinetic effects, including kinetic waves and instabilities, Landau damping, bump-on-tail instabilities and more, are therefore highly influential in tokamak plasma dynamics. Purely fluid models are inherently incapable of capturing these effects, whereas the high dimensionality in purely kinetic models render them practically intractable for most relevant purposes.

        We consider a $\delta\!f$ decomposition model, with a macroscopic fluid background and microscopic kinetic correction, both fully coupled to each other. A similar manner of discretization is proposed to that used in the recent \texttt{STRUPHY} code \cite{Holderied_Possanner_Wang_2021, Holderied_2022, Li_et_al_2023} with a finite-element model for the background and a pseudo-particle/PiC model for the correction.

        The fluid background satisfies the full, non-linear, resistive, compressible, Hall MHD equations. \cite{Laakmann_Hu_Farrell_2022} introduces finite-element(-in-space) implicit timesteppers for the incompressible analogue to this system with structure-preserving (SP) properties in the ideal case, alongside parameter-robust preconditioners. We show that these timesteppers can derive from a finite-element-in-time (FET) (and finite-element-in-space) interpretation. The benefits of this reformulation are discussed, including the derivation of timesteppers that are higher order in time, and the quantifiable dissipative SP properties in the non-ideal, resistive case.
        
        We discuss possible options for extending this FET approach to timesteppers for the compressible case.

        The kinetic corrections satisfy linearized Boltzmann equations. Using a Lénard--Bernstein collision operator, these take Fokker--Planck-like forms \cite{Fokker_1914, Planck_1917} wherein pseudo-particles in the numerical model obey the neoclassical transport equations, with particle-independent Brownian drift terms. This offers a rigorous methodology for incorporating collisions into the particle transport model, without coupling the equations of motions for each particle.
        
        Works by Chen, Chacón et al. \cite{Chen_Chacón_Barnes_2011, Chacón_Chen_Barnes_2013, Chen_Chacón_2014, Chen_Chacón_2015} have developed structure-preserving particle pushers for neoclassical transport in the Vlasov equations, derived from Crank--Nicolson integrators. We show these too can can derive from a FET interpretation, similarly offering potential extensions to higher-order-in-time particle pushers. The FET formulation is used also to consider how the stochastic drift terms can be incorporated into the pushers. Stochastic gyrokinetic expansions are also discussed.

        Different options for the numerical implementation of these schemes are considered.

        Due to the efficacy of FET in the development of SP timesteppers for both the fluid and kinetic component, we hope this approach will prove effective in the future for developing SP timesteppers for the full hybrid model. We hope this will give us the opportunity to incorporate previously inaccessible kinetic effects into the highly effective, modern, finite-element MHD models.
    \end{abstract}
    
    
    \newpage
    \tableofcontents
    
    
    \newpage
    \pagenumbering{arabic}
    %\linenumbers\renewcommand\thelinenumber{\color{black!50}\arabic{linenumber}}
            \input{0 - introduction/main.tex}
        \part{Research}
            \input{1 - low-noise PiC models/main.tex}
            \input{2 - kinetic component/main.tex}
            \input{3 - fluid component/main.tex}
            \input{4 - numerical implementation/main.tex}
        \part{Project Overview}
            \input{5 - research plan/main.tex}
            \input{6 - summary/main.tex}
    
    
    %\section{}
    \newpage
    \pagenumbering{gobble}
        \printbibliography


    \newpage
    \pagenumbering{roman}
    \appendix
        \part{Appendices}
            \input{8 - Hilbert complexes/main.tex}
            \input{9 - weak conservation proofs/main.tex}
\end{document}

        \part{Research}
            \documentclass[12pt, a4paper]{report}

\input{template/main.tex}

\title{\BA{Title in Progress...}}
\author{Boris Andrews}
\affil{Mathematical Institute, University of Oxford}
\date{\today}


\begin{document}
    \pagenumbering{gobble}
    \maketitle
    
    
    \begin{abstract}
        Magnetic confinement reactors---in particular tokamaks---offer one of the most promising options for achieving practical nuclear fusion, with the potential to provide virtually limitless, clean energy. The theoretical and numerical modeling of tokamak plasmas is simultaneously an essential component of effective reactor design, and a great research barrier. Tokamak operational conditions exhibit comparatively low Knudsen numbers. Kinetic effects, including kinetic waves and instabilities, Landau damping, bump-on-tail instabilities and more, are therefore highly influential in tokamak plasma dynamics. Purely fluid models are inherently incapable of capturing these effects, whereas the high dimensionality in purely kinetic models render them practically intractable for most relevant purposes.

        We consider a $\delta\!f$ decomposition model, with a macroscopic fluid background and microscopic kinetic correction, both fully coupled to each other. A similar manner of discretization is proposed to that used in the recent \texttt{STRUPHY} code \cite{Holderied_Possanner_Wang_2021, Holderied_2022, Li_et_al_2023} with a finite-element model for the background and a pseudo-particle/PiC model for the correction.

        The fluid background satisfies the full, non-linear, resistive, compressible, Hall MHD equations. \cite{Laakmann_Hu_Farrell_2022} introduces finite-element(-in-space) implicit timesteppers for the incompressible analogue to this system with structure-preserving (SP) properties in the ideal case, alongside parameter-robust preconditioners. We show that these timesteppers can derive from a finite-element-in-time (FET) (and finite-element-in-space) interpretation. The benefits of this reformulation are discussed, including the derivation of timesteppers that are higher order in time, and the quantifiable dissipative SP properties in the non-ideal, resistive case.
        
        We discuss possible options for extending this FET approach to timesteppers for the compressible case.

        The kinetic corrections satisfy linearized Boltzmann equations. Using a Lénard--Bernstein collision operator, these take Fokker--Planck-like forms \cite{Fokker_1914, Planck_1917} wherein pseudo-particles in the numerical model obey the neoclassical transport equations, with particle-independent Brownian drift terms. This offers a rigorous methodology for incorporating collisions into the particle transport model, without coupling the equations of motions for each particle.
        
        Works by Chen, Chacón et al. \cite{Chen_Chacón_Barnes_2011, Chacón_Chen_Barnes_2013, Chen_Chacón_2014, Chen_Chacón_2015} have developed structure-preserving particle pushers for neoclassical transport in the Vlasov equations, derived from Crank--Nicolson integrators. We show these too can can derive from a FET interpretation, similarly offering potential extensions to higher-order-in-time particle pushers. The FET formulation is used also to consider how the stochastic drift terms can be incorporated into the pushers. Stochastic gyrokinetic expansions are also discussed.

        Different options for the numerical implementation of these schemes are considered.

        Due to the efficacy of FET in the development of SP timesteppers for both the fluid and kinetic component, we hope this approach will prove effective in the future for developing SP timesteppers for the full hybrid model. We hope this will give us the opportunity to incorporate previously inaccessible kinetic effects into the highly effective, modern, finite-element MHD models.
    \end{abstract}
    
    
    \newpage
    \tableofcontents
    
    
    \newpage
    \pagenumbering{arabic}
    %\linenumbers\renewcommand\thelinenumber{\color{black!50}\arabic{linenumber}}
            \input{0 - introduction/main.tex}
        \part{Research}
            \input{1 - low-noise PiC models/main.tex}
            \input{2 - kinetic component/main.tex}
            \input{3 - fluid component/main.tex}
            \input{4 - numerical implementation/main.tex}
        \part{Project Overview}
            \input{5 - research plan/main.tex}
            \input{6 - summary/main.tex}
    
    
    %\section{}
    \newpage
    \pagenumbering{gobble}
        \printbibliography


    \newpage
    \pagenumbering{roman}
    \appendix
        \part{Appendices}
            \input{8 - Hilbert complexes/main.tex}
            \input{9 - weak conservation proofs/main.tex}
\end{document}

            \documentclass[12pt, a4paper]{report}

\input{template/main.tex}

\title{\BA{Title in Progress...}}
\author{Boris Andrews}
\affil{Mathematical Institute, University of Oxford}
\date{\today}


\begin{document}
    \pagenumbering{gobble}
    \maketitle
    
    
    \begin{abstract}
        Magnetic confinement reactors---in particular tokamaks---offer one of the most promising options for achieving practical nuclear fusion, with the potential to provide virtually limitless, clean energy. The theoretical and numerical modeling of tokamak plasmas is simultaneously an essential component of effective reactor design, and a great research barrier. Tokamak operational conditions exhibit comparatively low Knudsen numbers. Kinetic effects, including kinetic waves and instabilities, Landau damping, bump-on-tail instabilities and more, are therefore highly influential in tokamak plasma dynamics. Purely fluid models are inherently incapable of capturing these effects, whereas the high dimensionality in purely kinetic models render them practically intractable for most relevant purposes.

        We consider a $\delta\!f$ decomposition model, with a macroscopic fluid background and microscopic kinetic correction, both fully coupled to each other. A similar manner of discretization is proposed to that used in the recent \texttt{STRUPHY} code \cite{Holderied_Possanner_Wang_2021, Holderied_2022, Li_et_al_2023} with a finite-element model for the background and a pseudo-particle/PiC model for the correction.

        The fluid background satisfies the full, non-linear, resistive, compressible, Hall MHD equations. \cite{Laakmann_Hu_Farrell_2022} introduces finite-element(-in-space) implicit timesteppers for the incompressible analogue to this system with structure-preserving (SP) properties in the ideal case, alongside parameter-robust preconditioners. We show that these timesteppers can derive from a finite-element-in-time (FET) (and finite-element-in-space) interpretation. The benefits of this reformulation are discussed, including the derivation of timesteppers that are higher order in time, and the quantifiable dissipative SP properties in the non-ideal, resistive case.
        
        We discuss possible options for extending this FET approach to timesteppers for the compressible case.

        The kinetic corrections satisfy linearized Boltzmann equations. Using a Lénard--Bernstein collision operator, these take Fokker--Planck-like forms \cite{Fokker_1914, Planck_1917} wherein pseudo-particles in the numerical model obey the neoclassical transport equations, with particle-independent Brownian drift terms. This offers a rigorous methodology for incorporating collisions into the particle transport model, without coupling the equations of motions for each particle.
        
        Works by Chen, Chacón et al. \cite{Chen_Chacón_Barnes_2011, Chacón_Chen_Barnes_2013, Chen_Chacón_2014, Chen_Chacón_2015} have developed structure-preserving particle pushers for neoclassical transport in the Vlasov equations, derived from Crank--Nicolson integrators. We show these too can can derive from a FET interpretation, similarly offering potential extensions to higher-order-in-time particle pushers. The FET formulation is used also to consider how the stochastic drift terms can be incorporated into the pushers. Stochastic gyrokinetic expansions are also discussed.

        Different options for the numerical implementation of these schemes are considered.

        Due to the efficacy of FET in the development of SP timesteppers for both the fluid and kinetic component, we hope this approach will prove effective in the future for developing SP timesteppers for the full hybrid model. We hope this will give us the opportunity to incorporate previously inaccessible kinetic effects into the highly effective, modern, finite-element MHD models.
    \end{abstract}
    
    
    \newpage
    \tableofcontents
    
    
    \newpage
    \pagenumbering{arabic}
    %\linenumbers\renewcommand\thelinenumber{\color{black!50}\arabic{linenumber}}
            \input{0 - introduction/main.tex}
        \part{Research}
            \input{1 - low-noise PiC models/main.tex}
            \input{2 - kinetic component/main.tex}
            \input{3 - fluid component/main.tex}
            \input{4 - numerical implementation/main.tex}
        \part{Project Overview}
            \input{5 - research plan/main.tex}
            \input{6 - summary/main.tex}
    
    
    %\section{}
    \newpage
    \pagenumbering{gobble}
        \printbibliography


    \newpage
    \pagenumbering{roman}
    \appendix
        \part{Appendices}
            \input{8 - Hilbert complexes/main.tex}
            \input{9 - weak conservation proofs/main.tex}
\end{document}

            \documentclass[12pt, a4paper]{report}

\input{template/main.tex}

\title{\BA{Title in Progress...}}
\author{Boris Andrews}
\affil{Mathematical Institute, University of Oxford}
\date{\today}


\begin{document}
    \pagenumbering{gobble}
    \maketitle
    
    
    \begin{abstract}
        Magnetic confinement reactors---in particular tokamaks---offer one of the most promising options for achieving practical nuclear fusion, with the potential to provide virtually limitless, clean energy. The theoretical and numerical modeling of tokamak plasmas is simultaneously an essential component of effective reactor design, and a great research barrier. Tokamak operational conditions exhibit comparatively low Knudsen numbers. Kinetic effects, including kinetic waves and instabilities, Landau damping, bump-on-tail instabilities and more, are therefore highly influential in tokamak plasma dynamics. Purely fluid models are inherently incapable of capturing these effects, whereas the high dimensionality in purely kinetic models render them practically intractable for most relevant purposes.

        We consider a $\delta\!f$ decomposition model, with a macroscopic fluid background and microscopic kinetic correction, both fully coupled to each other. A similar manner of discretization is proposed to that used in the recent \texttt{STRUPHY} code \cite{Holderied_Possanner_Wang_2021, Holderied_2022, Li_et_al_2023} with a finite-element model for the background and a pseudo-particle/PiC model for the correction.

        The fluid background satisfies the full, non-linear, resistive, compressible, Hall MHD equations. \cite{Laakmann_Hu_Farrell_2022} introduces finite-element(-in-space) implicit timesteppers for the incompressible analogue to this system with structure-preserving (SP) properties in the ideal case, alongside parameter-robust preconditioners. We show that these timesteppers can derive from a finite-element-in-time (FET) (and finite-element-in-space) interpretation. The benefits of this reformulation are discussed, including the derivation of timesteppers that are higher order in time, and the quantifiable dissipative SP properties in the non-ideal, resistive case.
        
        We discuss possible options for extending this FET approach to timesteppers for the compressible case.

        The kinetic corrections satisfy linearized Boltzmann equations. Using a Lénard--Bernstein collision operator, these take Fokker--Planck-like forms \cite{Fokker_1914, Planck_1917} wherein pseudo-particles in the numerical model obey the neoclassical transport equations, with particle-independent Brownian drift terms. This offers a rigorous methodology for incorporating collisions into the particle transport model, without coupling the equations of motions for each particle.
        
        Works by Chen, Chacón et al. \cite{Chen_Chacón_Barnes_2011, Chacón_Chen_Barnes_2013, Chen_Chacón_2014, Chen_Chacón_2015} have developed structure-preserving particle pushers for neoclassical transport in the Vlasov equations, derived from Crank--Nicolson integrators. We show these too can can derive from a FET interpretation, similarly offering potential extensions to higher-order-in-time particle pushers. The FET formulation is used also to consider how the stochastic drift terms can be incorporated into the pushers. Stochastic gyrokinetic expansions are also discussed.

        Different options for the numerical implementation of these schemes are considered.

        Due to the efficacy of FET in the development of SP timesteppers for both the fluid and kinetic component, we hope this approach will prove effective in the future for developing SP timesteppers for the full hybrid model. We hope this will give us the opportunity to incorporate previously inaccessible kinetic effects into the highly effective, modern, finite-element MHD models.
    \end{abstract}
    
    
    \newpage
    \tableofcontents
    
    
    \newpage
    \pagenumbering{arabic}
    %\linenumbers\renewcommand\thelinenumber{\color{black!50}\arabic{linenumber}}
            \input{0 - introduction/main.tex}
        \part{Research}
            \input{1 - low-noise PiC models/main.tex}
            \input{2 - kinetic component/main.tex}
            \input{3 - fluid component/main.tex}
            \input{4 - numerical implementation/main.tex}
        \part{Project Overview}
            \input{5 - research plan/main.tex}
            \input{6 - summary/main.tex}
    
    
    %\section{}
    \newpage
    \pagenumbering{gobble}
        \printbibliography


    \newpage
    \pagenumbering{roman}
    \appendix
        \part{Appendices}
            \input{8 - Hilbert complexes/main.tex}
            \input{9 - weak conservation proofs/main.tex}
\end{document}

            \documentclass[12pt, a4paper]{report}

\input{template/main.tex}

\title{\BA{Title in Progress...}}
\author{Boris Andrews}
\affil{Mathematical Institute, University of Oxford}
\date{\today}


\begin{document}
    \pagenumbering{gobble}
    \maketitle
    
    
    \begin{abstract}
        Magnetic confinement reactors---in particular tokamaks---offer one of the most promising options for achieving practical nuclear fusion, with the potential to provide virtually limitless, clean energy. The theoretical and numerical modeling of tokamak plasmas is simultaneously an essential component of effective reactor design, and a great research barrier. Tokamak operational conditions exhibit comparatively low Knudsen numbers. Kinetic effects, including kinetic waves and instabilities, Landau damping, bump-on-tail instabilities and more, are therefore highly influential in tokamak plasma dynamics. Purely fluid models are inherently incapable of capturing these effects, whereas the high dimensionality in purely kinetic models render them practically intractable for most relevant purposes.

        We consider a $\delta\!f$ decomposition model, with a macroscopic fluid background and microscopic kinetic correction, both fully coupled to each other. A similar manner of discretization is proposed to that used in the recent \texttt{STRUPHY} code \cite{Holderied_Possanner_Wang_2021, Holderied_2022, Li_et_al_2023} with a finite-element model for the background and a pseudo-particle/PiC model for the correction.

        The fluid background satisfies the full, non-linear, resistive, compressible, Hall MHD equations. \cite{Laakmann_Hu_Farrell_2022} introduces finite-element(-in-space) implicit timesteppers for the incompressible analogue to this system with structure-preserving (SP) properties in the ideal case, alongside parameter-robust preconditioners. We show that these timesteppers can derive from a finite-element-in-time (FET) (and finite-element-in-space) interpretation. The benefits of this reformulation are discussed, including the derivation of timesteppers that are higher order in time, and the quantifiable dissipative SP properties in the non-ideal, resistive case.
        
        We discuss possible options for extending this FET approach to timesteppers for the compressible case.

        The kinetic corrections satisfy linearized Boltzmann equations. Using a Lénard--Bernstein collision operator, these take Fokker--Planck-like forms \cite{Fokker_1914, Planck_1917} wherein pseudo-particles in the numerical model obey the neoclassical transport equations, with particle-independent Brownian drift terms. This offers a rigorous methodology for incorporating collisions into the particle transport model, without coupling the equations of motions for each particle.
        
        Works by Chen, Chacón et al. \cite{Chen_Chacón_Barnes_2011, Chacón_Chen_Barnes_2013, Chen_Chacón_2014, Chen_Chacón_2015} have developed structure-preserving particle pushers for neoclassical transport in the Vlasov equations, derived from Crank--Nicolson integrators. We show these too can can derive from a FET interpretation, similarly offering potential extensions to higher-order-in-time particle pushers. The FET formulation is used also to consider how the stochastic drift terms can be incorporated into the pushers. Stochastic gyrokinetic expansions are also discussed.

        Different options for the numerical implementation of these schemes are considered.

        Due to the efficacy of FET in the development of SP timesteppers for both the fluid and kinetic component, we hope this approach will prove effective in the future for developing SP timesteppers for the full hybrid model. We hope this will give us the opportunity to incorporate previously inaccessible kinetic effects into the highly effective, modern, finite-element MHD models.
    \end{abstract}
    
    
    \newpage
    \tableofcontents
    
    
    \newpage
    \pagenumbering{arabic}
    %\linenumbers\renewcommand\thelinenumber{\color{black!50}\arabic{linenumber}}
            \input{0 - introduction/main.tex}
        \part{Research}
            \input{1 - low-noise PiC models/main.tex}
            \input{2 - kinetic component/main.tex}
            \input{3 - fluid component/main.tex}
            \input{4 - numerical implementation/main.tex}
        \part{Project Overview}
            \input{5 - research plan/main.tex}
            \input{6 - summary/main.tex}
    
    
    %\section{}
    \newpage
    \pagenumbering{gobble}
        \printbibliography


    \newpage
    \pagenumbering{roman}
    \appendix
        \part{Appendices}
            \input{8 - Hilbert complexes/main.tex}
            \input{9 - weak conservation proofs/main.tex}
\end{document}

        \part{Project Overview}
            \documentclass[12pt, a4paper]{report}

\input{template/main.tex}

\title{\BA{Title in Progress...}}
\author{Boris Andrews}
\affil{Mathematical Institute, University of Oxford}
\date{\today}


\begin{document}
    \pagenumbering{gobble}
    \maketitle
    
    
    \begin{abstract}
        Magnetic confinement reactors---in particular tokamaks---offer one of the most promising options for achieving practical nuclear fusion, with the potential to provide virtually limitless, clean energy. The theoretical and numerical modeling of tokamak plasmas is simultaneously an essential component of effective reactor design, and a great research barrier. Tokamak operational conditions exhibit comparatively low Knudsen numbers. Kinetic effects, including kinetic waves and instabilities, Landau damping, bump-on-tail instabilities and more, are therefore highly influential in tokamak plasma dynamics. Purely fluid models are inherently incapable of capturing these effects, whereas the high dimensionality in purely kinetic models render them practically intractable for most relevant purposes.

        We consider a $\delta\!f$ decomposition model, with a macroscopic fluid background and microscopic kinetic correction, both fully coupled to each other. A similar manner of discretization is proposed to that used in the recent \texttt{STRUPHY} code \cite{Holderied_Possanner_Wang_2021, Holderied_2022, Li_et_al_2023} with a finite-element model for the background and a pseudo-particle/PiC model for the correction.

        The fluid background satisfies the full, non-linear, resistive, compressible, Hall MHD equations. \cite{Laakmann_Hu_Farrell_2022} introduces finite-element(-in-space) implicit timesteppers for the incompressible analogue to this system with structure-preserving (SP) properties in the ideal case, alongside parameter-robust preconditioners. We show that these timesteppers can derive from a finite-element-in-time (FET) (and finite-element-in-space) interpretation. The benefits of this reformulation are discussed, including the derivation of timesteppers that are higher order in time, and the quantifiable dissipative SP properties in the non-ideal, resistive case.
        
        We discuss possible options for extending this FET approach to timesteppers for the compressible case.

        The kinetic corrections satisfy linearized Boltzmann equations. Using a Lénard--Bernstein collision operator, these take Fokker--Planck-like forms \cite{Fokker_1914, Planck_1917} wherein pseudo-particles in the numerical model obey the neoclassical transport equations, with particle-independent Brownian drift terms. This offers a rigorous methodology for incorporating collisions into the particle transport model, without coupling the equations of motions for each particle.
        
        Works by Chen, Chacón et al. \cite{Chen_Chacón_Barnes_2011, Chacón_Chen_Barnes_2013, Chen_Chacón_2014, Chen_Chacón_2015} have developed structure-preserving particle pushers for neoclassical transport in the Vlasov equations, derived from Crank--Nicolson integrators. We show these too can can derive from a FET interpretation, similarly offering potential extensions to higher-order-in-time particle pushers. The FET formulation is used also to consider how the stochastic drift terms can be incorporated into the pushers. Stochastic gyrokinetic expansions are also discussed.

        Different options for the numerical implementation of these schemes are considered.

        Due to the efficacy of FET in the development of SP timesteppers for both the fluid and kinetic component, we hope this approach will prove effective in the future for developing SP timesteppers for the full hybrid model. We hope this will give us the opportunity to incorporate previously inaccessible kinetic effects into the highly effective, modern, finite-element MHD models.
    \end{abstract}
    
    
    \newpage
    \tableofcontents
    
    
    \newpage
    \pagenumbering{arabic}
    %\linenumbers\renewcommand\thelinenumber{\color{black!50}\arabic{linenumber}}
            \input{0 - introduction/main.tex}
        \part{Research}
            \input{1 - low-noise PiC models/main.tex}
            \input{2 - kinetic component/main.tex}
            \input{3 - fluid component/main.tex}
            \input{4 - numerical implementation/main.tex}
        \part{Project Overview}
            \input{5 - research plan/main.tex}
            \input{6 - summary/main.tex}
    
    
    %\section{}
    \newpage
    \pagenumbering{gobble}
        \printbibliography


    \newpage
    \pagenumbering{roman}
    \appendix
        \part{Appendices}
            \input{8 - Hilbert complexes/main.tex}
            \input{9 - weak conservation proofs/main.tex}
\end{document}

            \documentclass[12pt, a4paper]{report}

\input{template/main.tex}

\title{\BA{Title in Progress...}}
\author{Boris Andrews}
\affil{Mathematical Institute, University of Oxford}
\date{\today}


\begin{document}
    \pagenumbering{gobble}
    \maketitle
    
    
    \begin{abstract}
        Magnetic confinement reactors---in particular tokamaks---offer one of the most promising options for achieving practical nuclear fusion, with the potential to provide virtually limitless, clean energy. The theoretical and numerical modeling of tokamak plasmas is simultaneously an essential component of effective reactor design, and a great research barrier. Tokamak operational conditions exhibit comparatively low Knudsen numbers. Kinetic effects, including kinetic waves and instabilities, Landau damping, bump-on-tail instabilities and more, are therefore highly influential in tokamak plasma dynamics. Purely fluid models are inherently incapable of capturing these effects, whereas the high dimensionality in purely kinetic models render them practically intractable for most relevant purposes.

        We consider a $\delta\!f$ decomposition model, with a macroscopic fluid background and microscopic kinetic correction, both fully coupled to each other. A similar manner of discretization is proposed to that used in the recent \texttt{STRUPHY} code \cite{Holderied_Possanner_Wang_2021, Holderied_2022, Li_et_al_2023} with a finite-element model for the background and a pseudo-particle/PiC model for the correction.

        The fluid background satisfies the full, non-linear, resistive, compressible, Hall MHD equations. \cite{Laakmann_Hu_Farrell_2022} introduces finite-element(-in-space) implicit timesteppers for the incompressible analogue to this system with structure-preserving (SP) properties in the ideal case, alongside parameter-robust preconditioners. We show that these timesteppers can derive from a finite-element-in-time (FET) (and finite-element-in-space) interpretation. The benefits of this reformulation are discussed, including the derivation of timesteppers that are higher order in time, and the quantifiable dissipative SP properties in the non-ideal, resistive case.
        
        We discuss possible options for extending this FET approach to timesteppers for the compressible case.

        The kinetic corrections satisfy linearized Boltzmann equations. Using a Lénard--Bernstein collision operator, these take Fokker--Planck-like forms \cite{Fokker_1914, Planck_1917} wherein pseudo-particles in the numerical model obey the neoclassical transport equations, with particle-independent Brownian drift terms. This offers a rigorous methodology for incorporating collisions into the particle transport model, without coupling the equations of motions for each particle.
        
        Works by Chen, Chacón et al. \cite{Chen_Chacón_Barnes_2011, Chacón_Chen_Barnes_2013, Chen_Chacón_2014, Chen_Chacón_2015} have developed structure-preserving particle pushers for neoclassical transport in the Vlasov equations, derived from Crank--Nicolson integrators. We show these too can can derive from a FET interpretation, similarly offering potential extensions to higher-order-in-time particle pushers. The FET formulation is used also to consider how the stochastic drift terms can be incorporated into the pushers. Stochastic gyrokinetic expansions are also discussed.

        Different options for the numerical implementation of these schemes are considered.

        Due to the efficacy of FET in the development of SP timesteppers for both the fluid and kinetic component, we hope this approach will prove effective in the future for developing SP timesteppers for the full hybrid model. We hope this will give us the opportunity to incorporate previously inaccessible kinetic effects into the highly effective, modern, finite-element MHD models.
    \end{abstract}
    
    
    \newpage
    \tableofcontents
    
    
    \newpage
    \pagenumbering{arabic}
    %\linenumbers\renewcommand\thelinenumber{\color{black!50}\arabic{linenumber}}
            \input{0 - introduction/main.tex}
        \part{Research}
            \input{1 - low-noise PiC models/main.tex}
            \input{2 - kinetic component/main.tex}
            \input{3 - fluid component/main.tex}
            \input{4 - numerical implementation/main.tex}
        \part{Project Overview}
            \input{5 - research plan/main.tex}
            \input{6 - summary/main.tex}
    
    
    %\section{}
    \newpage
    \pagenumbering{gobble}
        \printbibliography


    \newpage
    \pagenumbering{roman}
    \appendix
        \part{Appendices}
            \input{8 - Hilbert complexes/main.tex}
            \input{9 - weak conservation proofs/main.tex}
\end{document}

    
    
    %\section{}
    \newpage
    \pagenumbering{gobble}
        \printbibliography


    \newpage
    \pagenumbering{roman}
    \appendix
        \part{Appendices}
            \documentclass[12pt, a4paper]{report}

\input{template/main.tex}

\title{\BA{Title in Progress...}}
\author{Boris Andrews}
\affil{Mathematical Institute, University of Oxford}
\date{\today}


\begin{document}
    \pagenumbering{gobble}
    \maketitle
    
    
    \begin{abstract}
        Magnetic confinement reactors---in particular tokamaks---offer one of the most promising options for achieving practical nuclear fusion, with the potential to provide virtually limitless, clean energy. The theoretical and numerical modeling of tokamak plasmas is simultaneously an essential component of effective reactor design, and a great research barrier. Tokamak operational conditions exhibit comparatively low Knudsen numbers. Kinetic effects, including kinetic waves and instabilities, Landau damping, bump-on-tail instabilities and more, are therefore highly influential in tokamak plasma dynamics. Purely fluid models are inherently incapable of capturing these effects, whereas the high dimensionality in purely kinetic models render them practically intractable for most relevant purposes.

        We consider a $\delta\!f$ decomposition model, with a macroscopic fluid background and microscopic kinetic correction, both fully coupled to each other. A similar manner of discretization is proposed to that used in the recent \texttt{STRUPHY} code \cite{Holderied_Possanner_Wang_2021, Holderied_2022, Li_et_al_2023} with a finite-element model for the background and a pseudo-particle/PiC model for the correction.

        The fluid background satisfies the full, non-linear, resistive, compressible, Hall MHD equations. \cite{Laakmann_Hu_Farrell_2022} introduces finite-element(-in-space) implicit timesteppers for the incompressible analogue to this system with structure-preserving (SP) properties in the ideal case, alongside parameter-robust preconditioners. We show that these timesteppers can derive from a finite-element-in-time (FET) (and finite-element-in-space) interpretation. The benefits of this reformulation are discussed, including the derivation of timesteppers that are higher order in time, and the quantifiable dissipative SP properties in the non-ideal, resistive case.
        
        We discuss possible options for extending this FET approach to timesteppers for the compressible case.

        The kinetic corrections satisfy linearized Boltzmann equations. Using a Lénard--Bernstein collision operator, these take Fokker--Planck-like forms \cite{Fokker_1914, Planck_1917} wherein pseudo-particles in the numerical model obey the neoclassical transport equations, with particle-independent Brownian drift terms. This offers a rigorous methodology for incorporating collisions into the particle transport model, without coupling the equations of motions for each particle.
        
        Works by Chen, Chacón et al. \cite{Chen_Chacón_Barnes_2011, Chacón_Chen_Barnes_2013, Chen_Chacón_2014, Chen_Chacón_2015} have developed structure-preserving particle pushers for neoclassical transport in the Vlasov equations, derived from Crank--Nicolson integrators. We show these too can can derive from a FET interpretation, similarly offering potential extensions to higher-order-in-time particle pushers. The FET formulation is used also to consider how the stochastic drift terms can be incorporated into the pushers. Stochastic gyrokinetic expansions are also discussed.

        Different options for the numerical implementation of these schemes are considered.

        Due to the efficacy of FET in the development of SP timesteppers for both the fluid and kinetic component, we hope this approach will prove effective in the future for developing SP timesteppers for the full hybrid model. We hope this will give us the opportunity to incorporate previously inaccessible kinetic effects into the highly effective, modern, finite-element MHD models.
    \end{abstract}
    
    
    \newpage
    \tableofcontents
    
    
    \newpage
    \pagenumbering{arabic}
    %\linenumbers\renewcommand\thelinenumber{\color{black!50}\arabic{linenumber}}
            \input{0 - introduction/main.tex}
        \part{Research}
            \input{1 - low-noise PiC models/main.tex}
            \input{2 - kinetic component/main.tex}
            \input{3 - fluid component/main.tex}
            \input{4 - numerical implementation/main.tex}
        \part{Project Overview}
            \input{5 - research plan/main.tex}
            \input{6 - summary/main.tex}
    
    
    %\section{}
    \newpage
    \pagenumbering{gobble}
        \printbibliography


    \newpage
    \pagenumbering{roman}
    \appendix
        \part{Appendices}
            \input{8 - Hilbert complexes/main.tex}
            \input{9 - weak conservation proofs/main.tex}
\end{document}

            \documentclass[12pt, a4paper]{report}

\input{template/main.tex}

\title{\BA{Title in Progress...}}
\author{Boris Andrews}
\affil{Mathematical Institute, University of Oxford}
\date{\today}


\begin{document}
    \pagenumbering{gobble}
    \maketitle
    
    
    \begin{abstract}
        Magnetic confinement reactors---in particular tokamaks---offer one of the most promising options for achieving practical nuclear fusion, with the potential to provide virtually limitless, clean energy. The theoretical and numerical modeling of tokamak plasmas is simultaneously an essential component of effective reactor design, and a great research barrier. Tokamak operational conditions exhibit comparatively low Knudsen numbers. Kinetic effects, including kinetic waves and instabilities, Landau damping, bump-on-tail instabilities and more, are therefore highly influential in tokamak plasma dynamics. Purely fluid models are inherently incapable of capturing these effects, whereas the high dimensionality in purely kinetic models render them practically intractable for most relevant purposes.

        We consider a $\delta\!f$ decomposition model, with a macroscopic fluid background and microscopic kinetic correction, both fully coupled to each other. A similar manner of discretization is proposed to that used in the recent \texttt{STRUPHY} code \cite{Holderied_Possanner_Wang_2021, Holderied_2022, Li_et_al_2023} with a finite-element model for the background and a pseudo-particle/PiC model for the correction.

        The fluid background satisfies the full, non-linear, resistive, compressible, Hall MHD equations. \cite{Laakmann_Hu_Farrell_2022} introduces finite-element(-in-space) implicit timesteppers for the incompressible analogue to this system with structure-preserving (SP) properties in the ideal case, alongside parameter-robust preconditioners. We show that these timesteppers can derive from a finite-element-in-time (FET) (and finite-element-in-space) interpretation. The benefits of this reformulation are discussed, including the derivation of timesteppers that are higher order in time, and the quantifiable dissipative SP properties in the non-ideal, resistive case.
        
        We discuss possible options for extending this FET approach to timesteppers for the compressible case.

        The kinetic corrections satisfy linearized Boltzmann equations. Using a Lénard--Bernstein collision operator, these take Fokker--Planck-like forms \cite{Fokker_1914, Planck_1917} wherein pseudo-particles in the numerical model obey the neoclassical transport equations, with particle-independent Brownian drift terms. This offers a rigorous methodology for incorporating collisions into the particle transport model, without coupling the equations of motions for each particle.
        
        Works by Chen, Chacón et al. \cite{Chen_Chacón_Barnes_2011, Chacón_Chen_Barnes_2013, Chen_Chacón_2014, Chen_Chacón_2015} have developed structure-preserving particle pushers for neoclassical transport in the Vlasov equations, derived from Crank--Nicolson integrators. We show these too can can derive from a FET interpretation, similarly offering potential extensions to higher-order-in-time particle pushers. The FET formulation is used also to consider how the stochastic drift terms can be incorporated into the pushers. Stochastic gyrokinetic expansions are also discussed.

        Different options for the numerical implementation of these schemes are considered.

        Due to the efficacy of FET in the development of SP timesteppers for both the fluid and kinetic component, we hope this approach will prove effective in the future for developing SP timesteppers for the full hybrid model. We hope this will give us the opportunity to incorporate previously inaccessible kinetic effects into the highly effective, modern, finite-element MHD models.
    \end{abstract}
    
    
    \newpage
    \tableofcontents
    
    
    \newpage
    \pagenumbering{arabic}
    %\linenumbers\renewcommand\thelinenumber{\color{black!50}\arabic{linenumber}}
            \input{0 - introduction/main.tex}
        \part{Research}
            \input{1 - low-noise PiC models/main.tex}
            \input{2 - kinetic component/main.tex}
            \input{3 - fluid component/main.tex}
            \input{4 - numerical implementation/main.tex}
        \part{Project Overview}
            \input{5 - research plan/main.tex}
            \input{6 - summary/main.tex}
    
    
    %\section{}
    \newpage
    \pagenumbering{gobble}
        \printbibliography


    \newpage
    \pagenumbering{roman}
    \appendix
        \part{Appendices}
            \input{8 - Hilbert complexes/main.tex}
            \input{9 - weak conservation proofs/main.tex}
\end{document}

\end{document}

            \documentclass[12pt, a4paper]{report}

\documentclass[12pt, a4paper]{report}

\input{template/main.tex}

\title{\BA{Title in Progress...}}
\author{Boris Andrews}
\affil{Mathematical Institute, University of Oxford}
\date{\today}


\begin{document}
    \pagenumbering{gobble}
    \maketitle
    
    
    \begin{abstract}
        Magnetic confinement reactors---in particular tokamaks---offer one of the most promising options for achieving practical nuclear fusion, with the potential to provide virtually limitless, clean energy. The theoretical and numerical modeling of tokamak plasmas is simultaneously an essential component of effective reactor design, and a great research barrier. Tokamak operational conditions exhibit comparatively low Knudsen numbers. Kinetic effects, including kinetic waves and instabilities, Landau damping, bump-on-tail instabilities and more, are therefore highly influential in tokamak plasma dynamics. Purely fluid models are inherently incapable of capturing these effects, whereas the high dimensionality in purely kinetic models render them practically intractable for most relevant purposes.

        We consider a $\delta\!f$ decomposition model, with a macroscopic fluid background and microscopic kinetic correction, both fully coupled to each other. A similar manner of discretization is proposed to that used in the recent \texttt{STRUPHY} code \cite{Holderied_Possanner_Wang_2021, Holderied_2022, Li_et_al_2023} with a finite-element model for the background and a pseudo-particle/PiC model for the correction.

        The fluid background satisfies the full, non-linear, resistive, compressible, Hall MHD equations. \cite{Laakmann_Hu_Farrell_2022} introduces finite-element(-in-space) implicit timesteppers for the incompressible analogue to this system with structure-preserving (SP) properties in the ideal case, alongside parameter-robust preconditioners. We show that these timesteppers can derive from a finite-element-in-time (FET) (and finite-element-in-space) interpretation. The benefits of this reformulation are discussed, including the derivation of timesteppers that are higher order in time, and the quantifiable dissipative SP properties in the non-ideal, resistive case.
        
        We discuss possible options for extending this FET approach to timesteppers for the compressible case.

        The kinetic corrections satisfy linearized Boltzmann equations. Using a Lénard--Bernstein collision operator, these take Fokker--Planck-like forms \cite{Fokker_1914, Planck_1917} wherein pseudo-particles in the numerical model obey the neoclassical transport equations, with particle-independent Brownian drift terms. This offers a rigorous methodology for incorporating collisions into the particle transport model, without coupling the equations of motions for each particle.
        
        Works by Chen, Chacón et al. \cite{Chen_Chacón_Barnes_2011, Chacón_Chen_Barnes_2013, Chen_Chacón_2014, Chen_Chacón_2015} have developed structure-preserving particle pushers for neoclassical transport in the Vlasov equations, derived from Crank--Nicolson integrators. We show these too can can derive from a FET interpretation, similarly offering potential extensions to higher-order-in-time particle pushers. The FET formulation is used also to consider how the stochastic drift terms can be incorporated into the pushers. Stochastic gyrokinetic expansions are also discussed.

        Different options for the numerical implementation of these schemes are considered.

        Due to the efficacy of FET in the development of SP timesteppers for both the fluid and kinetic component, we hope this approach will prove effective in the future for developing SP timesteppers for the full hybrid model. We hope this will give us the opportunity to incorporate previously inaccessible kinetic effects into the highly effective, modern, finite-element MHD models.
    \end{abstract}
    
    
    \newpage
    \tableofcontents
    
    
    \newpage
    \pagenumbering{arabic}
    %\linenumbers\renewcommand\thelinenumber{\color{black!50}\arabic{linenumber}}
            \input{0 - introduction/main.tex}
        \part{Research}
            \input{1 - low-noise PiC models/main.tex}
            \input{2 - kinetic component/main.tex}
            \input{3 - fluid component/main.tex}
            \input{4 - numerical implementation/main.tex}
        \part{Project Overview}
            \input{5 - research plan/main.tex}
            \input{6 - summary/main.tex}
    
    
    %\section{}
    \newpage
    \pagenumbering{gobble}
        \printbibliography


    \newpage
    \pagenumbering{roman}
    \appendix
        \part{Appendices}
            \input{8 - Hilbert complexes/main.tex}
            \input{9 - weak conservation proofs/main.tex}
\end{document}


\title{\BA{Title in Progress...}}
\author{Boris Andrews}
\affil{Mathematical Institute, University of Oxford}
\date{\today}


\begin{document}
    \pagenumbering{gobble}
    \maketitle
    
    
    \begin{abstract}
        Magnetic confinement reactors---in particular tokamaks---offer one of the most promising options for achieving practical nuclear fusion, with the potential to provide virtually limitless, clean energy. The theoretical and numerical modeling of tokamak plasmas is simultaneously an essential component of effective reactor design, and a great research barrier. Tokamak operational conditions exhibit comparatively low Knudsen numbers. Kinetic effects, including kinetic waves and instabilities, Landau damping, bump-on-tail instabilities and more, are therefore highly influential in tokamak plasma dynamics. Purely fluid models are inherently incapable of capturing these effects, whereas the high dimensionality in purely kinetic models render them practically intractable for most relevant purposes.

        We consider a $\delta\!f$ decomposition model, with a macroscopic fluid background and microscopic kinetic correction, both fully coupled to each other. A similar manner of discretization is proposed to that used in the recent \texttt{STRUPHY} code \cite{Holderied_Possanner_Wang_2021, Holderied_2022, Li_et_al_2023} with a finite-element model for the background and a pseudo-particle/PiC model for the correction.

        The fluid background satisfies the full, non-linear, resistive, compressible, Hall MHD equations. \cite{Laakmann_Hu_Farrell_2022} introduces finite-element(-in-space) implicit timesteppers for the incompressible analogue to this system with structure-preserving (SP) properties in the ideal case, alongside parameter-robust preconditioners. We show that these timesteppers can derive from a finite-element-in-time (FET) (and finite-element-in-space) interpretation. The benefits of this reformulation are discussed, including the derivation of timesteppers that are higher order in time, and the quantifiable dissipative SP properties in the non-ideal, resistive case.
        
        We discuss possible options for extending this FET approach to timesteppers for the compressible case.

        The kinetic corrections satisfy linearized Boltzmann equations. Using a Lénard--Bernstein collision operator, these take Fokker--Planck-like forms \cite{Fokker_1914, Planck_1917} wherein pseudo-particles in the numerical model obey the neoclassical transport equations, with particle-independent Brownian drift terms. This offers a rigorous methodology for incorporating collisions into the particle transport model, without coupling the equations of motions for each particle.
        
        Works by Chen, Chacón et al. \cite{Chen_Chacón_Barnes_2011, Chacón_Chen_Barnes_2013, Chen_Chacón_2014, Chen_Chacón_2015} have developed structure-preserving particle pushers for neoclassical transport in the Vlasov equations, derived from Crank--Nicolson integrators. We show these too can can derive from a FET interpretation, similarly offering potential extensions to higher-order-in-time particle pushers. The FET formulation is used also to consider how the stochastic drift terms can be incorporated into the pushers. Stochastic gyrokinetic expansions are also discussed.

        Different options for the numerical implementation of these schemes are considered.

        Due to the efficacy of FET in the development of SP timesteppers for both the fluid and kinetic component, we hope this approach will prove effective in the future for developing SP timesteppers for the full hybrid model. We hope this will give us the opportunity to incorporate previously inaccessible kinetic effects into the highly effective, modern, finite-element MHD models.
    \end{abstract}
    
    
    \newpage
    \tableofcontents
    
    
    \newpage
    \pagenumbering{arabic}
    %\linenumbers\renewcommand\thelinenumber{\color{black!50}\arabic{linenumber}}
            \documentclass[12pt, a4paper]{report}

\input{template/main.tex}

\title{\BA{Title in Progress...}}
\author{Boris Andrews}
\affil{Mathematical Institute, University of Oxford}
\date{\today}


\begin{document}
    \pagenumbering{gobble}
    \maketitle
    
    
    \begin{abstract}
        Magnetic confinement reactors---in particular tokamaks---offer one of the most promising options for achieving practical nuclear fusion, with the potential to provide virtually limitless, clean energy. The theoretical and numerical modeling of tokamak plasmas is simultaneously an essential component of effective reactor design, and a great research barrier. Tokamak operational conditions exhibit comparatively low Knudsen numbers. Kinetic effects, including kinetic waves and instabilities, Landau damping, bump-on-tail instabilities and more, are therefore highly influential in tokamak plasma dynamics. Purely fluid models are inherently incapable of capturing these effects, whereas the high dimensionality in purely kinetic models render them practically intractable for most relevant purposes.

        We consider a $\delta\!f$ decomposition model, with a macroscopic fluid background and microscopic kinetic correction, both fully coupled to each other. A similar manner of discretization is proposed to that used in the recent \texttt{STRUPHY} code \cite{Holderied_Possanner_Wang_2021, Holderied_2022, Li_et_al_2023} with a finite-element model for the background and a pseudo-particle/PiC model for the correction.

        The fluid background satisfies the full, non-linear, resistive, compressible, Hall MHD equations. \cite{Laakmann_Hu_Farrell_2022} introduces finite-element(-in-space) implicit timesteppers for the incompressible analogue to this system with structure-preserving (SP) properties in the ideal case, alongside parameter-robust preconditioners. We show that these timesteppers can derive from a finite-element-in-time (FET) (and finite-element-in-space) interpretation. The benefits of this reformulation are discussed, including the derivation of timesteppers that are higher order in time, and the quantifiable dissipative SP properties in the non-ideal, resistive case.
        
        We discuss possible options for extending this FET approach to timesteppers for the compressible case.

        The kinetic corrections satisfy linearized Boltzmann equations. Using a Lénard--Bernstein collision operator, these take Fokker--Planck-like forms \cite{Fokker_1914, Planck_1917} wherein pseudo-particles in the numerical model obey the neoclassical transport equations, with particle-independent Brownian drift terms. This offers a rigorous methodology for incorporating collisions into the particle transport model, without coupling the equations of motions for each particle.
        
        Works by Chen, Chacón et al. \cite{Chen_Chacón_Barnes_2011, Chacón_Chen_Barnes_2013, Chen_Chacón_2014, Chen_Chacón_2015} have developed structure-preserving particle pushers for neoclassical transport in the Vlasov equations, derived from Crank--Nicolson integrators. We show these too can can derive from a FET interpretation, similarly offering potential extensions to higher-order-in-time particle pushers. The FET formulation is used also to consider how the stochastic drift terms can be incorporated into the pushers. Stochastic gyrokinetic expansions are also discussed.

        Different options for the numerical implementation of these schemes are considered.

        Due to the efficacy of FET in the development of SP timesteppers for both the fluid and kinetic component, we hope this approach will prove effective in the future for developing SP timesteppers for the full hybrid model. We hope this will give us the opportunity to incorporate previously inaccessible kinetic effects into the highly effective, modern, finite-element MHD models.
    \end{abstract}
    
    
    \newpage
    \tableofcontents
    
    
    \newpage
    \pagenumbering{arabic}
    %\linenumbers\renewcommand\thelinenumber{\color{black!50}\arabic{linenumber}}
            \input{0 - introduction/main.tex}
        \part{Research}
            \input{1 - low-noise PiC models/main.tex}
            \input{2 - kinetic component/main.tex}
            \input{3 - fluid component/main.tex}
            \input{4 - numerical implementation/main.tex}
        \part{Project Overview}
            \input{5 - research plan/main.tex}
            \input{6 - summary/main.tex}
    
    
    %\section{}
    \newpage
    \pagenumbering{gobble}
        \printbibliography


    \newpage
    \pagenumbering{roman}
    \appendix
        \part{Appendices}
            \input{8 - Hilbert complexes/main.tex}
            \input{9 - weak conservation proofs/main.tex}
\end{document}

        \part{Research}
            \documentclass[12pt, a4paper]{report}

\input{template/main.tex}

\title{\BA{Title in Progress...}}
\author{Boris Andrews}
\affil{Mathematical Institute, University of Oxford}
\date{\today}


\begin{document}
    \pagenumbering{gobble}
    \maketitle
    
    
    \begin{abstract}
        Magnetic confinement reactors---in particular tokamaks---offer one of the most promising options for achieving practical nuclear fusion, with the potential to provide virtually limitless, clean energy. The theoretical and numerical modeling of tokamak plasmas is simultaneously an essential component of effective reactor design, and a great research barrier. Tokamak operational conditions exhibit comparatively low Knudsen numbers. Kinetic effects, including kinetic waves and instabilities, Landau damping, bump-on-tail instabilities and more, are therefore highly influential in tokamak plasma dynamics. Purely fluid models are inherently incapable of capturing these effects, whereas the high dimensionality in purely kinetic models render them practically intractable for most relevant purposes.

        We consider a $\delta\!f$ decomposition model, with a macroscopic fluid background and microscopic kinetic correction, both fully coupled to each other. A similar manner of discretization is proposed to that used in the recent \texttt{STRUPHY} code \cite{Holderied_Possanner_Wang_2021, Holderied_2022, Li_et_al_2023} with a finite-element model for the background and a pseudo-particle/PiC model for the correction.

        The fluid background satisfies the full, non-linear, resistive, compressible, Hall MHD equations. \cite{Laakmann_Hu_Farrell_2022} introduces finite-element(-in-space) implicit timesteppers for the incompressible analogue to this system with structure-preserving (SP) properties in the ideal case, alongside parameter-robust preconditioners. We show that these timesteppers can derive from a finite-element-in-time (FET) (and finite-element-in-space) interpretation. The benefits of this reformulation are discussed, including the derivation of timesteppers that are higher order in time, and the quantifiable dissipative SP properties in the non-ideal, resistive case.
        
        We discuss possible options for extending this FET approach to timesteppers for the compressible case.

        The kinetic corrections satisfy linearized Boltzmann equations. Using a Lénard--Bernstein collision operator, these take Fokker--Planck-like forms \cite{Fokker_1914, Planck_1917} wherein pseudo-particles in the numerical model obey the neoclassical transport equations, with particle-independent Brownian drift terms. This offers a rigorous methodology for incorporating collisions into the particle transport model, without coupling the equations of motions for each particle.
        
        Works by Chen, Chacón et al. \cite{Chen_Chacón_Barnes_2011, Chacón_Chen_Barnes_2013, Chen_Chacón_2014, Chen_Chacón_2015} have developed structure-preserving particle pushers for neoclassical transport in the Vlasov equations, derived from Crank--Nicolson integrators. We show these too can can derive from a FET interpretation, similarly offering potential extensions to higher-order-in-time particle pushers. The FET formulation is used also to consider how the stochastic drift terms can be incorporated into the pushers. Stochastic gyrokinetic expansions are also discussed.

        Different options for the numerical implementation of these schemes are considered.

        Due to the efficacy of FET in the development of SP timesteppers for both the fluid and kinetic component, we hope this approach will prove effective in the future for developing SP timesteppers for the full hybrid model. We hope this will give us the opportunity to incorporate previously inaccessible kinetic effects into the highly effective, modern, finite-element MHD models.
    \end{abstract}
    
    
    \newpage
    \tableofcontents
    
    
    \newpage
    \pagenumbering{arabic}
    %\linenumbers\renewcommand\thelinenumber{\color{black!50}\arabic{linenumber}}
            \input{0 - introduction/main.tex}
        \part{Research}
            \input{1 - low-noise PiC models/main.tex}
            \input{2 - kinetic component/main.tex}
            \input{3 - fluid component/main.tex}
            \input{4 - numerical implementation/main.tex}
        \part{Project Overview}
            \input{5 - research plan/main.tex}
            \input{6 - summary/main.tex}
    
    
    %\section{}
    \newpage
    \pagenumbering{gobble}
        \printbibliography


    \newpage
    \pagenumbering{roman}
    \appendix
        \part{Appendices}
            \input{8 - Hilbert complexes/main.tex}
            \input{9 - weak conservation proofs/main.tex}
\end{document}

            \documentclass[12pt, a4paper]{report}

\input{template/main.tex}

\title{\BA{Title in Progress...}}
\author{Boris Andrews}
\affil{Mathematical Institute, University of Oxford}
\date{\today}


\begin{document}
    \pagenumbering{gobble}
    \maketitle
    
    
    \begin{abstract}
        Magnetic confinement reactors---in particular tokamaks---offer one of the most promising options for achieving practical nuclear fusion, with the potential to provide virtually limitless, clean energy. The theoretical and numerical modeling of tokamak plasmas is simultaneously an essential component of effective reactor design, and a great research barrier. Tokamak operational conditions exhibit comparatively low Knudsen numbers. Kinetic effects, including kinetic waves and instabilities, Landau damping, bump-on-tail instabilities and more, are therefore highly influential in tokamak plasma dynamics. Purely fluid models are inherently incapable of capturing these effects, whereas the high dimensionality in purely kinetic models render them practically intractable for most relevant purposes.

        We consider a $\delta\!f$ decomposition model, with a macroscopic fluid background and microscopic kinetic correction, both fully coupled to each other. A similar manner of discretization is proposed to that used in the recent \texttt{STRUPHY} code \cite{Holderied_Possanner_Wang_2021, Holderied_2022, Li_et_al_2023} with a finite-element model for the background and a pseudo-particle/PiC model for the correction.

        The fluid background satisfies the full, non-linear, resistive, compressible, Hall MHD equations. \cite{Laakmann_Hu_Farrell_2022} introduces finite-element(-in-space) implicit timesteppers for the incompressible analogue to this system with structure-preserving (SP) properties in the ideal case, alongside parameter-robust preconditioners. We show that these timesteppers can derive from a finite-element-in-time (FET) (and finite-element-in-space) interpretation. The benefits of this reformulation are discussed, including the derivation of timesteppers that are higher order in time, and the quantifiable dissipative SP properties in the non-ideal, resistive case.
        
        We discuss possible options for extending this FET approach to timesteppers for the compressible case.

        The kinetic corrections satisfy linearized Boltzmann equations. Using a Lénard--Bernstein collision operator, these take Fokker--Planck-like forms \cite{Fokker_1914, Planck_1917} wherein pseudo-particles in the numerical model obey the neoclassical transport equations, with particle-independent Brownian drift terms. This offers a rigorous methodology for incorporating collisions into the particle transport model, without coupling the equations of motions for each particle.
        
        Works by Chen, Chacón et al. \cite{Chen_Chacón_Barnes_2011, Chacón_Chen_Barnes_2013, Chen_Chacón_2014, Chen_Chacón_2015} have developed structure-preserving particle pushers for neoclassical transport in the Vlasov equations, derived from Crank--Nicolson integrators. We show these too can can derive from a FET interpretation, similarly offering potential extensions to higher-order-in-time particle pushers. The FET formulation is used also to consider how the stochastic drift terms can be incorporated into the pushers. Stochastic gyrokinetic expansions are also discussed.

        Different options for the numerical implementation of these schemes are considered.

        Due to the efficacy of FET in the development of SP timesteppers for both the fluid and kinetic component, we hope this approach will prove effective in the future for developing SP timesteppers for the full hybrid model. We hope this will give us the opportunity to incorporate previously inaccessible kinetic effects into the highly effective, modern, finite-element MHD models.
    \end{abstract}
    
    
    \newpage
    \tableofcontents
    
    
    \newpage
    \pagenumbering{arabic}
    %\linenumbers\renewcommand\thelinenumber{\color{black!50}\arabic{linenumber}}
            \input{0 - introduction/main.tex}
        \part{Research}
            \input{1 - low-noise PiC models/main.tex}
            \input{2 - kinetic component/main.tex}
            \input{3 - fluid component/main.tex}
            \input{4 - numerical implementation/main.tex}
        \part{Project Overview}
            \input{5 - research plan/main.tex}
            \input{6 - summary/main.tex}
    
    
    %\section{}
    \newpage
    \pagenumbering{gobble}
        \printbibliography


    \newpage
    \pagenumbering{roman}
    \appendix
        \part{Appendices}
            \input{8 - Hilbert complexes/main.tex}
            \input{9 - weak conservation proofs/main.tex}
\end{document}

            \documentclass[12pt, a4paper]{report}

\input{template/main.tex}

\title{\BA{Title in Progress...}}
\author{Boris Andrews}
\affil{Mathematical Institute, University of Oxford}
\date{\today}


\begin{document}
    \pagenumbering{gobble}
    \maketitle
    
    
    \begin{abstract}
        Magnetic confinement reactors---in particular tokamaks---offer one of the most promising options for achieving practical nuclear fusion, with the potential to provide virtually limitless, clean energy. The theoretical and numerical modeling of tokamak plasmas is simultaneously an essential component of effective reactor design, and a great research barrier. Tokamak operational conditions exhibit comparatively low Knudsen numbers. Kinetic effects, including kinetic waves and instabilities, Landau damping, bump-on-tail instabilities and more, are therefore highly influential in tokamak plasma dynamics. Purely fluid models are inherently incapable of capturing these effects, whereas the high dimensionality in purely kinetic models render them practically intractable for most relevant purposes.

        We consider a $\delta\!f$ decomposition model, with a macroscopic fluid background and microscopic kinetic correction, both fully coupled to each other. A similar manner of discretization is proposed to that used in the recent \texttt{STRUPHY} code \cite{Holderied_Possanner_Wang_2021, Holderied_2022, Li_et_al_2023} with a finite-element model for the background and a pseudo-particle/PiC model for the correction.

        The fluid background satisfies the full, non-linear, resistive, compressible, Hall MHD equations. \cite{Laakmann_Hu_Farrell_2022} introduces finite-element(-in-space) implicit timesteppers for the incompressible analogue to this system with structure-preserving (SP) properties in the ideal case, alongside parameter-robust preconditioners. We show that these timesteppers can derive from a finite-element-in-time (FET) (and finite-element-in-space) interpretation. The benefits of this reformulation are discussed, including the derivation of timesteppers that are higher order in time, and the quantifiable dissipative SP properties in the non-ideal, resistive case.
        
        We discuss possible options for extending this FET approach to timesteppers for the compressible case.

        The kinetic corrections satisfy linearized Boltzmann equations. Using a Lénard--Bernstein collision operator, these take Fokker--Planck-like forms \cite{Fokker_1914, Planck_1917} wherein pseudo-particles in the numerical model obey the neoclassical transport equations, with particle-independent Brownian drift terms. This offers a rigorous methodology for incorporating collisions into the particle transport model, without coupling the equations of motions for each particle.
        
        Works by Chen, Chacón et al. \cite{Chen_Chacón_Barnes_2011, Chacón_Chen_Barnes_2013, Chen_Chacón_2014, Chen_Chacón_2015} have developed structure-preserving particle pushers for neoclassical transport in the Vlasov equations, derived from Crank--Nicolson integrators. We show these too can can derive from a FET interpretation, similarly offering potential extensions to higher-order-in-time particle pushers. The FET formulation is used also to consider how the stochastic drift terms can be incorporated into the pushers. Stochastic gyrokinetic expansions are also discussed.

        Different options for the numerical implementation of these schemes are considered.

        Due to the efficacy of FET in the development of SP timesteppers for both the fluid and kinetic component, we hope this approach will prove effective in the future for developing SP timesteppers for the full hybrid model. We hope this will give us the opportunity to incorporate previously inaccessible kinetic effects into the highly effective, modern, finite-element MHD models.
    \end{abstract}
    
    
    \newpage
    \tableofcontents
    
    
    \newpage
    \pagenumbering{arabic}
    %\linenumbers\renewcommand\thelinenumber{\color{black!50}\arabic{linenumber}}
            \input{0 - introduction/main.tex}
        \part{Research}
            \input{1 - low-noise PiC models/main.tex}
            \input{2 - kinetic component/main.tex}
            \input{3 - fluid component/main.tex}
            \input{4 - numerical implementation/main.tex}
        \part{Project Overview}
            \input{5 - research plan/main.tex}
            \input{6 - summary/main.tex}
    
    
    %\section{}
    \newpage
    \pagenumbering{gobble}
        \printbibliography


    \newpage
    \pagenumbering{roman}
    \appendix
        \part{Appendices}
            \input{8 - Hilbert complexes/main.tex}
            \input{9 - weak conservation proofs/main.tex}
\end{document}

            \documentclass[12pt, a4paper]{report}

\input{template/main.tex}

\title{\BA{Title in Progress...}}
\author{Boris Andrews}
\affil{Mathematical Institute, University of Oxford}
\date{\today}


\begin{document}
    \pagenumbering{gobble}
    \maketitle
    
    
    \begin{abstract}
        Magnetic confinement reactors---in particular tokamaks---offer one of the most promising options for achieving practical nuclear fusion, with the potential to provide virtually limitless, clean energy. The theoretical and numerical modeling of tokamak plasmas is simultaneously an essential component of effective reactor design, and a great research barrier. Tokamak operational conditions exhibit comparatively low Knudsen numbers. Kinetic effects, including kinetic waves and instabilities, Landau damping, bump-on-tail instabilities and more, are therefore highly influential in tokamak plasma dynamics. Purely fluid models are inherently incapable of capturing these effects, whereas the high dimensionality in purely kinetic models render them practically intractable for most relevant purposes.

        We consider a $\delta\!f$ decomposition model, with a macroscopic fluid background and microscopic kinetic correction, both fully coupled to each other. A similar manner of discretization is proposed to that used in the recent \texttt{STRUPHY} code \cite{Holderied_Possanner_Wang_2021, Holderied_2022, Li_et_al_2023} with a finite-element model for the background and a pseudo-particle/PiC model for the correction.

        The fluid background satisfies the full, non-linear, resistive, compressible, Hall MHD equations. \cite{Laakmann_Hu_Farrell_2022} introduces finite-element(-in-space) implicit timesteppers for the incompressible analogue to this system with structure-preserving (SP) properties in the ideal case, alongside parameter-robust preconditioners. We show that these timesteppers can derive from a finite-element-in-time (FET) (and finite-element-in-space) interpretation. The benefits of this reformulation are discussed, including the derivation of timesteppers that are higher order in time, and the quantifiable dissipative SP properties in the non-ideal, resistive case.
        
        We discuss possible options for extending this FET approach to timesteppers for the compressible case.

        The kinetic corrections satisfy linearized Boltzmann equations. Using a Lénard--Bernstein collision operator, these take Fokker--Planck-like forms \cite{Fokker_1914, Planck_1917} wherein pseudo-particles in the numerical model obey the neoclassical transport equations, with particle-independent Brownian drift terms. This offers a rigorous methodology for incorporating collisions into the particle transport model, without coupling the equations of motions for each particle.
        
        Works by Chen, Chacón et al. \cite{Chen_Chacón_Barnes_2011, Chacón_Chen_Barnes_2013, Chen_Chacón_2014, Chen_Chacón_2015} have developed structure-preserving particle pushers for neoclassical transport in the Vlasov equations, derived from Crank--Nicolson integrators. We show these too can can derive from a FET interpretation, similarly offering potential extensions to higher-order-in-time particle pushers. The FET formulation is used also to consider how the stochastic drift terms can be incorporated into the pushers. Stochastic gyrokinetic expansions are also discussed.

        Different options for the numerical implementation of these schemes are considered.

        Due to the efficacy of FET in the development of SP timesteppers for both the fluid and kinetic component, we hope this approach will prove effective in the future for developing SP timesteppers for the full hybrid model. We hope this will give us the opportunity to incorporate previously inaccessible kinetic effects into the highly effective, modern, finite-element MHD models.
    \end{abstract}
    
    
    \newpage
    \tableofcontents
    
    
    \newpage
    \pagenumbering{arabic}
    %\linenumbers\renewcommand\thelinenumber{\color{black!50}\arabic{linenumber}}
            \input{0 - introduction/main.tex}
        \part{Research}
            \input{1 - low-noise PiC models/main.tex}
            \input{2 - kinetic component/main.tex}
            \input{3 - fluid component/main.tex}
            \input{4 - numerical implementation/main.tex}
        \part{Project Overview}
            \input{5 - research plan/main.tex}
            \input{6 - summary/main.tex}
    
    
    %\section{}
    \newpage
    \pagenumbering{gobble}
        \printbibliography


    \newpage
    \pagenumbering{roman}
    \appendix
        \part{Appendices}
            \input{8 - Hilbert complexes/main.tex}
            \input{9 - weak conservation proofs/main.tex}
\end{document}

        \part{Project Overview}
            \documentclass[12pt, a4paper]{report}

\input{template/main.tex}

\title{\BA{Title in Progress...}}
\author{Boris Andrews}
\affil{Mathematical Institute, University of Oxford}
\date{\today}


\begin{document}
    \pagenumbering{gobble}
    \maketitle
    
    
    \begin{abstract}
        Magnetic confinement reactors---in particular tokamaks---offer one of the most promising options for achieving practical nuclear fusion, with the potential to provide virtually limitless, clean energy. The theoretical and numerical modeling of tokamak plasmas is simultaneously an essential component of effective reactor design, and a great research barrier. Tokamak operational conditions exhibit comparatively low Knudsen numbers. Kinetic effects, including kinetic waves and instabilities, Landau damping, bump-on-tail instabilities and more, are therefore highly influential in tokamak plasma dynamics. Purely fluid models are inherently incapable of capturing these effects, whereas the high dimensionality in purely kinetic models render them practically intractable for most relevant purposes.

        We consider a $\delta\!f$ decomposition model, with a macroscopic fluid background and microscopic kinetic correction, both fully coupled to each other. A similar manner of discretization is proposed to that used in the recent \texttt{STRUPHY} code \cite{Holderied_Possanner_Wang_2021, Holderied_2022, Li_et_al_2023} with a finite-element model for the background and a pseudo-particle/PiC model for the correction.

        The fluid background satisfies the full, non-linear, resistive, compressible, Hall MHD equations. \cite{Laakmann_Hu_Farrell_2022} introduces finite-element(-in-space) implicit timesteppers for the incompressible analogue to this system with structure-preserving (SP) properties in the ideal case, alongside parameter-robust preconditioners. We show that these timesteppers can derive from a finite-element-in-time (FET) (and finite-element-in-space) interpretation. The benefits of this reformulation are discussed, including the derivation of timesteppers that are higher order in time, and the quantifiable dissipative SP properties in the non-ideal, resistive case.
        
        We discuss possible options for extending this FET approach to timesteppers for the compressible case.

        The kinetic corrections satisfy linearized Boltzmann equations. Using a Lénard--Bernstein collision operator, these take Fokker--Planck-like forms \cite{Fokker_1914, Planck_1917} wherein pseudo-particles in the numerical model obey the neoclassical transport equations, with particle-independent Brownian drift terms. This offers a rigorous methodology for incorporating collisions into the particle transport model, without coupling the equations of motions for each particle.
        
        Works by Chen, Chacón et al. \cite{Chen_Chacón_Barnes_2011, Chacón_Chen_Barnes_2013, Chen_Chacón_2014, Chen_Chacón_2015} have developed structure-preserving particle pushers for neoclassical transport in the Vlasov equations, derived from Crank--Nicolson integrators. We show these too can can derive from a FET interpretation, similarly offering potential extensions to higher-order-in-time particle pushers. The FET formulation is used also to consider how the stochastic drift terms can be incorporated into the pushers. Stochastic gyrokinetic expansions are also discussed.

        Different options for the numerical implementation of these schemes are considered.

        Due to the efficacy of FET in the development of SP timesteppers for both the fluid and kinetic component, we hope this approach will prove effective in the future for developing SP timesteppers for the full hybrid model. We hope this will give us the opportunity to incorporate previously inaccessible kinetic effects into the highly effective, modern, finite-element MHD models.
    \end{abstract}
    
    
    \newpage
    \tableofcontents
    
    
    \newpage
    \pagenumbering{arabic}
    %\linenumbers\renewcommand\thelinenumber{\color{black!50}\arabic{linenumber}}
            \input{0 - introduction/main.tex}
        \part{Research}
            \input{1 - low-noise PiC models/main.tex}
            \input{2 - kinetic component/main.tex}
            \input{3 - fluid component/main.tex}
            \input{4 - numerical implementation/main.tex}
        \part{Project Overview}
            \input{5 - research plan/main.tex}
            \input{6 - summary/main.tex}
    
    
    %\section{}
    \newpage
    \pagenumbering{gobble}
        \printbibliography


    \newpage
    \pagenumbering{roman}
    \appendix
        \part{Appendices}
            \input{8 - Hilbert complexes/main.tex}
            \input{9 - weak conservation proofs/main.tex}
\end{document}

            \documentclass[12pt, a4paper]{report}

\input{template/main.tex}

\title{\BA{Title in Progress...}}
\author{Boris Andrews}
\affil{Mathematical Institute, University of Oxford}
\date{\today}


\begin{document}
    \pagenumbering{gobble}
    \maketitle
    
    
    \begin{abstract}
        Magnetic confinement reactors---in particular tokamaks---offer one of the most promising options for achieving practical nuclear fusion, with the potential to provide virtually limitless, clean energy. The theoretical and numerical modeling of tokamak plasmas is simultaneously an essential component of effective reactor design, and a great research barrier. Tokamak operational conditions exhibit comparatively low Knudsen numbers. Kinetic effects, including kinetic waves and instabilities, Landau damping, bump-on-tail instabilities and more, are therefore highly influential in tokamak plasma dynamics. Purely fluid models are inherently incapable of capturing these effects, whereas the high dimensionality in purely kinetic models render them practically intractable for most relevant purposes.

        We consider a $\delta\!f$ decomposition model, with a macroscopic fluid background and microscopic kinetic correction, both fully coupled to each other. A similar manner of discretization is proposed to that used in the recent \texttt{STRUPHY} code \cite{Holderied_Possanner_Wang_2021, Holderied_2022, Li_et_al_2023} with a finite-element model for the background and a pseudo-particle/PiC model for the correction.

        The fluid background satisfies the full, non-linear, resistive, compressible, Hall MHD equations. \cite{Laakmann_Hu_Farrell_2022} introduces finite-element(-in-space) implicit timesteppers for the incompressible analogue to this system with structure-preserving (SP) properties in the ideal case, alongside parameter-robust preconditioners. We show that these timesteppers can derive from a finite-element-in-time (FET) (and finite-element-in-space) interpretation. The benefits of this reformulation are discussed, including the derivation of timesteppers that are higher order in time, and the quantifiable dissipative SP properties in the non-ideal, resistive case.
        
        We discuss possible options for extending this FET approach to timesteppers for the compressible case.

        The kinetic corrections satisfy linearized Boltzmann equations. Using a Lénard--Bernstein collision operator, these take Fokker--Planck-like forms \cite{Fokker_1914, Planck_1917} wherein pseudo-particles in the numerical model obey the neoclassical transport equations, with particle-independent Brownian drift terms. This offers a rigorous methodology for incorporating collisions into the particle transport model, without coupling the equations of motions for each particle.
        
        Works by Chen, Chacón et al. \cite{Chen_Chacón_Barnes_2011, Chacón_Chen_Barnes_2013, Chen_Chacón_2014, Chen_Chacón_2015} have developed structure-preserving particle pushers for neoclassical transport in the Vlasov equations, derived from Crank--Nicolson integrators. We show these too can can derive from a FET interpretation, similarly offering potential extensions to higher-order-in-time particle pushers. The FET formulation is used also to consider how the stochastic drift terms can be incorporated into the pushers. Stochastic gyrokinetic expansions are also discussed.

        Different options for the numerical implementation of these schemes are considered.

        Due to the efficacy of FET in the development of SP timesteppers for both the fluid and kinetic component, we hope this approach will prove effective in the future for developing SP timesteppers for the full hybrid model. We hope this will give us the opportunity to incorporate previously inaccessible kinetic effects into the highly effective, modern, finite-element MHD models.
    \end{abstract}
    
    
    \newpage
    \tableofcontents
    
    
    \newpage
    \pagenumbering{arabic}
    %\linenumbers\renewcommand\thelinenumber{\color{black!50}\arabic{linenumber}}
            \input{0 - introduction/main.tex}
        \part{Research}
            \input{1 - low-noise PiC models/main.tex}
            \input{2 - kinetic component/main.tex}
            \input{3 - fluid component/main.tex}
            \input{4 - numerical implementation/main.tex}
        \part{Project Overview}
            \input{5 - research plan/main.tex}
            \input{6 - summary/main.tex}
    
    
    %\section{}
    \newpage
    \pagenumbering{gobble}
        \printbibliography


    \newpage
    \pagenumbering{roman}
    \appendix
        \part{Appendices}
            \input{8 - Hilbert complexes/main.tex}
            \input{9 - weak conservation proofs/main.tex}
\end{document}

    
    
    %\section{}
    \newpage
    \pagenumbering{gobble}
        \printbibliography


    \newpage
    \pagenumbering{roman}
    \appendix
        \part{Appendices}
            \documentclass[12pt, a4paper]{report}

\input{template/main.tex}

\title{\BA{Title in Progress...}}
\author{Boris Andrews}
\affil{Mathematical Institute, University of Oxford}
\date{\today}


\begin{document}
    \pagenumbering{gobble}
    \maketitle
    
    
    \begin{abstract}
        Magnetic confinement reactors---in particular tokamaks---offer one of the most promising options for achieving practical nuclear fusion, with the potential to provide virtually limitless, clean energy. The theoretical and numerical modeling of tokamak plasmas is simultaneously an essential component of effective reactor design, and a great research barrier. Tokamak operational conditions exhibit comparatively low Knudsen numbers. Kinetic effects, including kinetic waves and instabilities, Landau damping, bump-on-tail instabilities and more, are therefore highly influential in tokamak plasma dynamics. Purely fluid models are inherently incapable of capturing these effects, whereas the high dimensionality in purely kinetic models render them practically intractable for most relevant purposes.

        We consider a $\delta\!f$ decomposition model, with a macroscopic fluid background and microscopic kinetic correction, both fully coupled to each other. A similar manner of discretization is proposed to that used in the recent \texttt{STRUPHY} code \cite{Holderied_Possanner_Wang_2021, Holderied_2022, Li_et_al_2023} with a finite-element model for the background and a pseudo-particle/PiC model for the correction.

        The fluid background satisfies the full, non-linear, resistive, compressible, Hall MHD equations. \cite{Laakmann_Hu_Farrell_2022} introduces finite-element(-in-space) implicit timesteppers for the incompressible analogue to this system with structure-preserving (SP) properties in the ideal case, alongside parameter-robust preconditioners. We show that these timesteppers can derive from a finite-element-in-time (FET) (and finite-element-in-space) interpretation. The benefits of this reformulation are discussed, including the derivation of timesteppers that are higher order in time, and the quantifiable dissipative SP properties in the non-ideal, resistive case.
        
        We discuss possible options for extending this FET approach to timesteppers for the compressible case.

        The kinetic corrections satisfy linearized Boltzmann equations. Using a Lénard--Bernstein collision operator, these take Fokker--Planck-like forms \cite{Fokker_1914, Planck_1917} wherein pseudo-particles in the numerical model obey the neoclassical transport equations, with particle-independent Brownian drift terms. This offers a rigorous methodology for incorporating collisions into the particle transport model, without coupling the equations of motions for each particle.
        
        Works by Chen, Chacón et al. \cite{Chen_Chacón_Barnes_2011, Chacón_Chen_Barnes_2013, Chen_Chacón_2014, Chen_Chacón_2015} have developed structure-preserving particle pushers for neoclassical transport in the Vlasov equations, derived from Crank--Nicolson integrators. We show these too can can derive from a FET interpretation, similarly offering potential extensions to higher-order-in-time particle pushers. The FET formulation is used also to consider how the stochastic drift terms can be incorporated into the pushers. Stochastic gyrokinetic expansions are also discussed.

        Different options for the numerical implementation of these schemes are considered.

        Due to the efficacy of FET in the development of SP timesteppers for both the fluid and kinetic component, we hope this approach will prove effective in the future for developing SP timesteppers for the full hybrid model. We hope this will give us the opportunity to incorporate previously inaccessible kinetic effects into the highly effective, modern, finite-element MHD models.
    \end{abstract}
    
    
    \newpage
    \tableofcontents
    
    
    \newpage
    \pagenumbering{arabic}
    %\linenumbers\renewcommand\thelinenumber{\color{black!50}\arabic{linenumber}}
            \input{0 - introduction/main.tex}
        \part{Research}
            \input{1 - low-noise PiC models/main.tex}
            \input{2 - kinetic component/main.tex}
            \input{3 - fluid component/main.tex}
            \input{4 - numerical implementation/main.tex}
        \part{Project Overview}
            \input{5 - research plan/main.tex}
            \input{6 - summary/main.tex}
    
    
    %\section{}
    \newpage
    \pagenumbering{gobble}
        \printbibliography


    \newpage
    \pagenumbering{roman}
    \appendix
        \part{Appendices}
            \input{8 - Hilbert complexes/main.tex}
            \input{9 - weak conservation proofs/main.tex}
\end{document}

            \documentclass[12pt, a4paper]{report}

\input{template/main.tex}

\title{\BA{Title in Progress...}}
\author{Boris Andrews}
\affil{Mathematical Institute, University of Oxford}
\date{\today}


\begin{document}
    \pagenumbering{gobble}
    \maketitle
    
    
    \begin{abstract}
        Magnetic confinement reactors---in particular tokamaks---offer one of the most promising options for achieving practical nuclear fusion, with the potential to provide virtually limitless, clean energy. The theoretical and numerical modeling of tokamak plasmas is simultaneously an essential component of effective reactor design, and a great research barrier. Tokamak operational conditions exhibit comparatively low Knudsen numbers. Kinetic effects, including kinetic waves and instabilities, Landau damping, bump-on-tail instabilities and more, are therefore highly influential in tokamak plasma dynamics. Purely fluid models are inherently incapable of capturing these effects, whereas the high dimensionality in purely kinetic models render them practically intractable for most relevant purposes.

        We consider a $\delta\!f$ decomposition model, with a macroscopic fluid background and microscopic kinetic correction, both fully coupled to each other. A similar manner of discretization is proposed to that used in the recent \texttt{STRUPHY} code \cite{Holderied_Possanner_Wang_2021, Holderied_2022, Li_et_al_2023} with a finite-element model for the background and a pseudo-particle/PiC model for the correction.

        The fluid background satisfies the full, non-linear, resistive, compressible, Hall MHD equations. \cite{Laakmann_Hu_Farrell_2022} introduces finite-element(-in-space) implicit timesteppers for the incompressible analogue to this system with structure-preserving (SP) properties in the ideal case, alongside parameter-robust preconditioners. We show that these timesteppers can derive from a finite-element-in-time (FET) (and finite-element-in-space) interpretation. The benefits of this reformulation are discussed, including the derivation of timesteppers that are higher order in time, and the quantifiable dissipative SP properties in the non-ideal, resistive case.
        
        We discuss possible options for extending this FET approach to timesteppers for the compressible case.

        The kinetic corrections satisfy linearized Boltzmann equations. Using a Lénard--Bernstein collision operator, these take Fokker--Planck-like forms \cite{Fokker_1914, Planck_1917} wherein pseudo-particles in the numerical model obey the neoclassical transport equations, with particle-independent Brownian drift terms. This offers a rigorous methodology for incorporating collisions into the particle transport model, without coupling the equations of motions for each particle.
        
        Works by Chen, Chacón et al. \cite{Chen_Chacón_Barnes_2011, Chacón_Chen_Barnes_2013, Chen_Chacón_2014, Chen_Chacón_2015} have developed structure-preserving particle pushers for neoclassical transport in the Vlasov equations, derived from Crank--Nicolson integrators. We show these too can can derive from a FET interpretation, similarly offering potential extensions to higher-order-in-time particle pushers. The FET formulation is used also to consider how the stochastic drift terms can be incorporated into the pushers. Stochastic gyrokinetic expansions are also discussed.

        Different options for the numerical implementation of these schemes are considered.

        Due to the efficacy of FET in the development of SP timesteppers for both the fluid and kinetic component, we hope this approach will prove effective in the future for developing SP timesteppers for the full hybrid model. We hope this will give us the opportunity to incorporate previously inaccessible kinetic effects into the highly effective, modern, finite-element MHD models.
    \end{abstract}
    
    
    \newpage
    \tableofcontents
    
    
    \newpage
    \pagenumbering{arabic}
    %\linenumbers\renewcommand\thelinenumber{\color{black!50}\arabic{linenumber}}
            \input{0 - introduction/main.tex}
        \part{Research}
            \input{1 - low-noise PiC models/main.tex}
            \input{2 - kinetic component/main.tex}
            \input{3 - fluid component/main.tex}
            \input{4 - numerical implementation/main.tex}
        \part{Project Overview}
            \input{5 - research plan/main.tex}
            \input{6 - summary/main.tex}
    
    
    %\section{}
    \newpage
    \pagenumbering{gobble}
        \printbibliography


    \newpage
    \pagenumbering{roman}
    \appendix
        \part{Appendices}
            \input{8 - Hilbert complexes/main.tex}
            \input{9 - weak conservation proofs/main.tex}
\end{document}

\end{document}

\end{document}

    \documentclass[12pt, a4paper]{report}

\documentclass[12pt, a4paper]{report}

\documentclass[12pt, a4paper]{report}

\input{template/main.tex}

\title{\BA{Title in Progress...}}
\author{Boris Andrews}
\affil{Mathematical Institute, University of Oxford}
\date{\today}


\begin{document}
    \pagenumbering{gobble}
    \maketitle
    
    
    \begin{abstract}
        Magnetic confinement reactors---in particular tokamaks---offer one of the most promising options for achieving practical nuclear fusion, with the potential to provide virtually limitless, clean energy. The theoretical and numerical modeling of tokamak plasmas is simultaneously an essential component of effective reactor design, and a great research barrier. Tokamak operational conditions exhibit comparatively low Knudsen numbers. Kinetic effects, including kinetic waves and instabilities, Landau damping, bump-on-tail instabilities and more, are therefore highly influential in tokamak plasma dynamics. Purely fluid models are inherently incapable of capturing these effects, whereas the high dimensionality in purely kinetic models render them practically intractable for most relevant purposes.

        We consider a $\delta\!f$ decomposition model, with a macroscopic fluid background and microscopic kinetic correction, both fully coupled to each other. A similar manner of discretization is proposed to that used in the recent \texttt{STRUPHY} code \cite{Holderied_Possanner_Wang_2021, Holderied_2022, Li_et_al_2023} with a finite-element model for the background and a pseudo-particle/PiC model for the correction.

        The fluid background satisfies the full, non-linear, resistive, compressible, Hall MHD equations. \cite{Laakmann_Hu_Farrell_2022} introduces finite-element(-in-space) implicit timesteppers for the incompressible analogue to this system with structure-preserving (SP) properties in the ideal case, alongside parameter-robust preconditioners. We show that these timesteppers can derive from a finite-element-in-time (FET) (and finite-element-in-space) interpretation. The benefits of this reformulation are discussed, including the derivation of timesteppers that are higher order in time, and the quantifiable dissipative SP properties in the non-ideal, resistive case.
        
        We discuss possible options for extending this FET approach to timesteppers for the compressible case.

        The kinetic corrections satisfy linearized Boltzmann equations. Using a Lénard--Bernstein collision operator, these take Fokker--Planck-like forms \cite{Fokker_1914, Planck_1917} wherein pseudo-particles in the numerical model obey the neoclassical transport equations, with particle-independent Brownian drift terms. This offers a rigorous methodology for incorporating collisions into the particle transport model, without coupling the equations of motions for each particle.
        
        Works by Chen, Chacón et al. \cite{Chen_Chacón_Barnes_2011, Chacón_Chen_Barnes_2013, Chen_Chacón_2014, Chen_Chacón_2015} have developed structure-preserving particle pushers for neoclassical transport in the Vlasov equations, derived from Crank--Nicolson integrators. We show these too can can derive from a FET interpretation, similarly offering potential extensions to higher-order-in-time particle pushers. The FET formulation is used also to consider how the stochastic drift terms can be incorporated into the pushers. Stochastic gyrokinetic expansions are also discussed.

        Different options for the numerical implementation of these schemes are considered.

        Due to the efficacy of FET in the development of SP timesteppers for both the fluid and kinetic component, we hope this approach will prove effective in the future for developing SP timesteppers for the full hybrid model. We hope this will give us the opportunity to incorporate previously inaccessible kinetic effects into the highly effective, modern, finite-element MHD models.
    \end{abstract}
    
    
    \newpage
    \tableofcontents
    
    
    \newpage
    \pagenumbering{arabic}
    %\linenumbers\renewcommand\thelinenumber{\color{black!50}\arabic{linenumber}}
            \input{0 - introduction/main.tex}
        \part{Research}
            \input{1 - low-noise PiC models/main.tex}
            \input{2 - kinetic component/main.tex}
            \input{3 - fluid component/main.tex}
            \input{4 - numerical implementation/main.tex}
        \part{Project Overview}
            \input{5 - research plan/main.tex}
            \input{6 - summary/main.tex}
    
    
    %\section{}
    \newpage
    \pagenumbering{gobble}
        \printbibliography


    \newpage
    \pagenumbering{roman}
    \appendix
        \part{Appendices}
            \input{8 - Hilbert complexes/main.tex}
            \input{9 - weak conservation proofs/main.tex}
\end{document}


\title{\BA{Title in Progress...}}
\author{Boris Andrews}
\affil{Mathematical Institute, University of Oxford}
\date{\today}


\begin{document}
    \pagenumbering{gobble}
    \maketitle
    
    
    \begin{abstract}
        Magnetic confinement reactors---in particular tokamaks---offer one of the most promising options for achieving practical nuclear fusion, with the potential to provide virtually limitless, clean energy. The theoretical and numerical modeling of tokamak plasmas is simultaneously an essential component of effective reactor design, and a great research barrier. Tokamak operational conditions exhibit comparatively low Knudsen numbers. Kinetic effects, including kinetic waves and instabilities, Landau damping, bump-on-tail instabilities and more, are therefore highly influential in tokamak plasma dynamics. Purely fluid models are inherently incapable of capturing these effects, whereas the high dimensionality in purely kinetic models render them practically intractable for most relevant purposes.

        We consider a $\delta\!f$ decomposition model, with a macroscopic fluid background and microscopic kinetic correction, both fully coupled to each other. A similar manner of discretization is proposed to that used in the recent \texttt{STRUPHY} code \cite{Holderied_Possanner_Wang_2021, Holderied_2022, Li_et_al_2023} with a finite-element model for the background and a pseudo-particle/PiC model for the correction.

        The fluid background satisfies the full, non-linear, resistive, compressible, Hall MHD equations. \cite{Laakmann_Hu_Farrell_2022} introduces finite-element(-in-space) implicit timesteppers for the incompressible analogue to this system with structure-preserving (SP) properties in the ideal case, alongside parameter-robust preconditioners. We show that these timesteppers can derive from a finite-element-in-time (FET) (and finite-element-in-space) interpretation. The benefits of this reformulation are discussed, including the derivation of timesteppers that are higher order in time, and the quantifiable dissipative SP properties in the non-ideal, resistive case.
        
        We discuss possible options for extending this FET approach to timesteppers for the compressible case.

        The kinetic corrections satisfy linearized Boltzmann equations. Using a Lénard--Bernstein collision operator, these take Fokker--Planck-like forms \cite{Fokker_1914, Planck_1917} wherein pseudo-particles in the numerical model obey the neoclassical transport equations, with particle-independent Brownian drift terms. This offers a rigorous methodology for incorporating collisions into the particle transport model, without coupling the equations of motions for each particle.
        
        Works by Chen, Chacón et al. \cite{Chen_Chacón_Barnes_2011, Chacón_Chen_Barnes_2013, Chen_Chacón_2014, Chen_Chacón_2015} have developed structure-preserving particle pushers for neoclassical transport in the Vlasov equations, derived from Crank--Nicolson integrators. We show these too can can derive from a FET interpretation, similarly offering potential extensions to higher-order-in-time particle pushers. The FET formulation is used also to consider how the stochastic drift terms can be incorporated into the pushers. Stochastic gyrokinetic expansions are also discussed.

        Different options for the numerical implementation of these schemes are considered.

        Due to the efficacy of FET in the development of SP timesteppers for both the fluid and kinetic component, we hope this approach will prove effective in the future for developing SP timesteppers for the full hybrid model. We hope this will give us the opportunity to incorporate previously inaccessible kinetic effects into the highly effective, modern, finite-element MHD models.
    \end{abstract}
    
    
    \newpage
    \tableofcontents
    
    
    \newpage
    \pagenumbering{arabic}
    %\linenumbers\renewcommand\thelinenumber{\color{black!50}\arabic{linenumber}}
            \documentclass[12pt, a4paper]{report}

\input{template/main.tex}

\title{\BA{Title in Progress...}}
\author{Boris Andrews}
\affil{Mathematical Institute, University of Oxford}
\date{\today}


\begin{document}
    \pagenumbering{gobble}
    \maketitle
    
    
    \begin{abstract}
        Magnetic confinement reactors---in particular tokamaks---offer one of the most promising options for achieving practical nuclear fusion, with the potential to provide virtually limitless, clean energy. The theoretical and numerical modeling of tokamak plasmas is simultaneously an essential component of effective reactor design, and a great research barrier. Tokamak operational conditions exhibit comparatively low Knudsen numbers. Kinetic effects, including kinetic waves and instabilities, Landau damping, bump-on-tail instabilities and more, are therefore highly influential in tokamak plasma dynamics. Purely fluid models are inherently incapable of capturing these effects, whereas the high dimensionality in purely kinetic models render them practically intractable for most relevant purposes.

        We consider a $\delta\!f$ decomposition model, with a macroscopic fluid background and microscopic kinetic correction, both fully coupled to each other. A similar manner of discretization is proposed to that used in the recent \texttt{STRUPHY} code \cite{Holderied_Possanner_Wang_2021, Holderied_2022, Li_et_al_2023} with a finite-element model for the background and a pseudo-particle/PiC model for the correction.

        The fluid background satisfies the full, non-linear, resistive, compressible, Hall MHD equations. \cite{Laakmann_Hu_Farrell_2022} introduces finite-element(-in-space) implicit timesteppers for the incompressible analogue to this system with structure-preserving (SP) properties in the ideal case, alongside parameter-robust preconditioners. We show that these timesteppers can derive from a finite-element-in-time (FET) (and finite-element-in-space) interpretation. The benefits of this reformulation are discussed, including the derivation of timesteppers that are higher order in time, and the quantifiable dissipative SP properties in the non-ideal, resistive case.
        
        We discuss possible options for extending this FET approach to timesteppers for the compressible case.

        The kinetic corrections satisfy linearized Boltzmann equations. Using a Lénard--Bernstein collision operator, these take Fokker--Planck-like forms \cite{Fokker_1914, Planck_1917} wherein pseudo-particles in the numerical model obey the neoclassical transport equations, with particle-independent Brownian drift terms. This offers a rigorous methodology for incorporating collisions into the particle transport model, without coupling the equations of motions for each particle.
        
        Works by Chen, Chacón et al. \cite{Chen_Chacón_Barnes_2011, Chacón_Chen_Barnes_2013, Chen_Chacón_2014, Chen_Chacón_2015} have developed structure-preserving particle pushers for neoclassical transport in the Vlasov equations, derived from Crank--Nicolson integrators. We show these too can can derive from a FET interpretation, similarly offering potential extensions to higher-order-in-time particle pushers. The FET formulation is used also to consider how the stochastic drift terms can be incorporated into the pushers. Stochastic gyrokinetic expansions are also discussed.

        Different options for the numerical implementation of these schemes are considered.

        Due to the efficacy of FET in the development of SP timesteppers for both the fluid and kinetic component, we hope this approach will prove effective in the future for developing SP timesteppers for the full hybrid model. We hope this will give us the opportunity to incorporate previously inaccessible kinetic effects into the highly effective, modern, finite-element MHD models.
    \end{abstract}
    
    
    \newpage
    \tableofcontents
    
    
    \newpage
    \pagenumbering{arabic}
    %\linenumbers\renewcommand\thelinenumber{\color{black!50}\arabic{linenumber}}
            \input{0 - introduction/main.tex}
        \part{Research}
            \input{1 - low-noise PiC models/main.tex}
            \input{2 - kinetic component/main.tex}
            \input{3 - fluid component/main.tex}
            \input{4 - numerical implementation/main.tex}
        \part{Project Overview}
            \input{5 - research plan/main.tex}
            \input{6 - summary/main.tex}
    
    
    %\section{}
    \newpage
    \pagenumbering{gobble}
        \printbibliography


    \newpage
    \pagenumbering{roman}
    \appendix
        \part{Appendices}
            \input{8 - Hilbert complexes/main.tex}
            \input{9 - weak conservation proofs/main.tex}
\end{document}

        \part{Research}
            \documentclass[12pt, a4paper]{report}

\input{template/main.tex}

\title{\BA{Title in Progress...}}
\author{Boris Andrews}
\affil{Mathematical Institute, University of Oxford}
\date{\today}


\begin{document}
    \pagenumbering{gobble}
    \maketitle
    
    
    \begin{abstract}
        Magnetic confinement reactors---in particular tokamaks---offer one of the most promising options for achieving practical nuclear fusion, with the potential to provide virtually limitless, clean energy. The theoretical and numerical modeling of tokamak plasmas is simultaneously an essential component of effective reactor design, and a great research barrier. Tokamak operational conditions exhibit comparatively low Knudsen numbers. Kinetic effects, including kinetic waves and instabilities, Landau damping, bump-on-tail instabilities and more, are therefore highly influential in tokamak plasma dynamics. Purely fluid models are inherently incapable of capturing these effects, whereas the high dimensionality in purely kinetic models render them practically intractable for most relevant purposes.

        We consider a $\delta\!f$ decomposition model, with a macroscopic fluid background and microscopic kinetic correction, both fully coupled to each other. A similar manner of discretization is proposed to that used in the recent \texttt{STRUPHY} code \cite{Holderied_Possanner_Wang_2021, Holderied_2022, Li_et_al_2023} with a finite-element model for the background and a pseudo-particle/PiC model for the correction.

        The fluid background satisfies the full, non-linear, resistive, compressible, Hall MHD equations. \cite{Laakmann_Hu_Farrell_2022} introduces finite-element(-in-space) implicit timesteppers for the incompressible analogue to this system with structure-preserving (SP) properties in the ideal case, alongside parameter-robust preconditioners. We show that these timesteppers can derive from a finite-element-in-time (FET) (and finite-element-in-space) interpretation. The benefits of this reformulation are discussed, including the derivation of timesteppers that are higher order in time, and the quantifiable dissipative SP properties in the non-ideal, resistive case.
        
        We discuss possible options for extending this FET approach to timesteppers for the compressible case.

        The kinetic corrections satisfy linearized Boltzmann equations. Using a Lénard--Bernstein collision operator, these take Fokker--Planck-like forms \cite{Fokker_1914, Planck_1917} wherein pseudo-particles in the numerical model obey the neoclassical transport equations, with particle-independent Brownian drift terms. This offers a rigorous methodology for incorporating collisions into the particle transport model, without coupling the equations of motions for each particle.
        
        Works by Chen, Chacón et al. \cite{Chen_Chacón_Barnes_2011, Chacón_Chen_Barnes_2013, Chen_Chacón_2014, Chen_Chacón_2015} have developed structure-preserving particle pushers for neoclassical transport in the Vlasov equations, derived from Crank--Nicolson integrators. We show these too can can derive from a FET interpretation, similarly offering potential extensions to higher-order-in-time particle pushers. The FET formulation is used also to consider how the stochastic drift terms can be incorporated into the pushers. Stochastic gyrokinetic expansions are also discussed.

        Different options for the numerical implementation of these schemes are considered.

        Due to the efficacy of FET in the development of SP timesteppers for both the fluid and kinetic component, we hope this approach will prove effective in the future for developing SP timesteppers for the full hybrid model. We hope this will give us the opportunity to incorporate previously inaccessible kinetic effects into the highly effective, modern, finite-element MHD models.
    \end{abstract}
    
    
    \newpage
    \tableofcontents
    
    
    \newpage
    \pagenumbering{arabic}
    %\linenumbers\renewcommand\thelinenumber{\color{black!50}\arabic{linenumber}}
            \input{0 - introduction/main.tex}
        \part{Research}
            \input{1 - low-noise PiC models/main.tex}
            \input{2 - kinetic component/main.tex}
            \input{3 - fluid component/main.tex}
            \input{4 - numerical implementation/main.tex}
        \part{Project Overview}
            \input{5 - research plan/main.tex}
            \input{6 - summary/main.tex}
    
    
    %\section{}
    \newpage
    \pagenumbering{gobble}
        \printbibliography


    \newpage
    \pagenumbering{roman}
    \appendix
        \part{Appendices}
            \input{8 - Hilbert complexes/main.tex}
            \input{9 - weak conservation proofs/main.tex}
\end{document}

            \documentclass[12pt, a4paper]{report}

\input{template/main.tex}

\title{\BA{Title in Progress...}}
\author{Boris Andrews}
\affil{Mathematical Institute, University of Oxford}
\date{\today}


\begin{document}
    \pagenumbering{gobble}
    \maketitle
    
    
    \begin{abstract}
        Magnetic confinement reactors---in particular tokamaks---offer one of the most promising options for achieving practical nuclear fusion, with the potential to provide virtually limitless, clean energy. The theoretical and numerical modeling of tokamak plasmas is simultaneously an essential component of effective reactor design, and a great research barrier. Tokamak operational conditions exhibit comparatively low Knudsen numbers. Kinetic effects, including kinetic waves and instabilities, Landau damping, bump-on-tail instabilities and more, are therefore highly influential in tokamak plasma dynamics. Purely fluid models are inherently incapable of capturing these effects, whereas the high dimensionality in purely kinetic models render them practically intractable for most relevant purposes.

        We consider a $\delta\!f$ decomposition model, with a macroscopic fluid background and microscopic kinetic correction, both fully coupled to each other. A similar manner of discretization is proposed to that used in the recent \texttt{STRUPHY} code \cite{Holderied_Possanner_Wang_2021, Holderied_2022, Li_et_al_2023} with a finite-element model for the background and a pseudo-particle/PiC model for the correction.

        The fluid background satisfies the full, non-linear, resistive, compressible, Hall MHD equations. \cite{Laakmann_Hu_Farrell_2022} introduces finite-element(-in-space) implicit timesteppers for the incompressible analogue to this system with structure-preserving (SP) properties in the ideal case, alongside parameter-robust preconditioners. We show that these timesteppers can derive from a finite-element-in-time (FET) (and finite-element-in-space) interpretation. The benefits of this reformulation are discussed, including the derivation of timesteppers that are higher order in time, and the quantifiable dissipative SP properties in the non-ideal, resistive case.
        
        We discuss possible options for extending this FET approach to timesteppers for the compressible case.

        The kinetic corrections satisfy linearized Boltzmann equations. Using a Lénard--Bernstein collision operator, these take Fokker--Planck-like forms \cite{Fokker_1914, Planck_1917} wherein pseudo-particles in the numerical model obey the neoclassical transport equations, with particle-independent Brownian drift terms. This offers a rigorous methodology for incorporating collisions into the particle transport model, without coupling the equations of motions for each particle.
        
        Works by Chen, Chacón et al. \cite{Chen_Chacón_Barnes_2011, Chacón_Chen_Barnes_2013, Chen_Chacón_2014, Chen_Chacón_2015} have developed structure-preserving particle pushers for neoclassical transport in the Vlasov equations, derived from Crank--Nicolson integrators. We show these too can can derive from a FET interpretation, similarly offering potential extensions to higher-order-in-time particle pushers. The FET formulation is used also to consider how the stochastic drift terms can be incorporated into the pushers. Stochastic gyrokinetic expansions are also discussed.

        Different options for the numerical implementation of these schemes are considered.

        Due to the efficacy of FET in the development of SP timesteppers for both the fluid and kinetic component, we hope this approach will prove effective in the future for developing SP timesteppers for the full hybrid model. We hope this will give us the opportunity to incorporate previously inaccessible kinetic effects into the highly effective, modern, finite-element MHD models.
    \end{abstract}
    
    
    \newpage
    \tableofcontents
    
    
    \newpage
    \pagenumbering{arabic}
    %\linenumbers\renewcommand\thelinenumber{\color{black!50}\arabic{linenumber}}
            \input{0 - introduction/main.tex}
        \part{Research}
            \input{1 - low-noise PiC models/main.tex}
            \input{2 - kinetic component/main.tex}
            \input{3 - fluid component/main.tex}
            \input{4 - numerical implementation/main.tex}
        \part{Project Overview}
            \input{5 - research plan/main.tex}
            \input{6 - summary/main.tex}
    
    
    %\section{}
    \newpage
    \pagenumbering{gobble}
        \printbibliography


    \newpage
    \pagenumbering{roman}
    \appendix
        \part{Appendices}
            \input{8 - Hilbert complexes/main.tex}
            \input{9 - weak conservation proofs/main.tex}
\end{document}

            \documentclass[12pt, a4paper]{report}

\input{template/main.tex}

\title{\BA{Title in Progress...}}
\author{Boris Andrews}
\affil{Mathematical Institute, University of Oxford}
\date{\today}


\begin{document}
    \pagenumbering{gobble}
    \maketitle
    
    
    \begin{abstract}
        Magnetic confinement reactors---in particular tokamaks---offer one of the most promising options for achieving practical nuclear fusion, with the potential to provide virtually limitless, clean energy. The theoretical and numerical modeling of tokamak plasmas is simultaneously an essential component of effective reactor design, and a great research barrier. Tokamak operational conditions exhibit comparatively low Knudsen numbers. Kinetic effects, including kinetic waves and instabilities, Landau damping, bump-on-tail instabilities and more, are therefore highly influential in tokamak plasma dynamics. Purely fluid models are inherently incapable of capturing these effects, whereas the high dimensionality in purely kinetic models render them practically intractable for most relevant purposes.

        We consider a $\delta\!f$ decomposition model, with a macroscopic fluid background and microscopic kinetic correction, both fully coupled to each other. A similar manner of discretization is proposed to that used in the recent \texttt{STRUPHY} code \cite{Holderied_Possanner_Wang_2021, Holderied_2022, Li_et_al_2023} with a finite-element model for the background and a pseudo-particle/PiC model for the correction.

        The fluid background satisfies the full, non-linear, resistive, compressible, Hall MHD equations. \cite{Laakmann_Hu_Farrell_2022} introduces finite-element(-in-space) implicit timesteppers for the incompressible analogue to this system with structure-preserving (SP) properties in the ideal case, alongside parameter-robust preconditioners. We show that these timesteppers can derive from a finite-element-in-time (FET) (and finite-element-in-space) interpretation. The benefits of this reformulation are discussed, including the derivation of timesteppers that are higher order in time, and the quantifiable dissipative SP properties in the non-ideal, resistive case.
        
        We discuss possible options for extending this FET approach to timesteppers for the compressible case.

        The kinetic corrections satisfy linearized Boltzmann equations. Using a Lénard--Bernstein collision operator, these take Fokker--Planck-like forms \cite{Fokker_1914, Planck_1917} wherein pseudo-particles in the numerical model obey the neoclassical transport equations, with particle-independent Brownian drift terms. This offers a rigorous methodology for incorporating collisions into the particle transport model, without coupling the equations of motions for each particle.
        
        Works by Chen, Chacón et al. \cite{Chen_Chacón_Barnes_2011, Chacón_Chen_Barnes_2013, Chen_Chacón_2014, Chen_Chacón_2015} have developed structure-preserving particle pushers for neoclassical transport in the Vlasov equations, derived from Crank--Nicolson integrators. We show these too can can derive from a FET interpretation, similarly offering potential extensions to higher-order-in-time particle pushers. The FET formulation is used also to consider how the stochastic drift terms can be incorporated into the pushers. Stochastic gyrokinetic expansions are also discussed.

        Different options for the numerical implementation of these schemes are considered.

        Due to the efficacy of FET in the development of SP timesteppers for both the fluid and kinetic component, we hope this approach will prove effective in the future for developing SP timesteppers for the full hybrid model. We hope this will give us the opportunity to incorporate previously inaccessible kinetic effects into the highly effective, modern, finite-element MHD models.
    \end{abstract}
    
    
    \newpage
    \tableofcontents
    
    
    \newpage
    \pagenumbering{arabic}
    %\linenumbers\renewcommand\thelinenumber{\color{black!50}\arabic{linenumber}}
            \input{0 - introduction/main.tex}
        \part{Research}
            \input{1 - low-noise PiC models/main.tex}
            \input{2 - kinetic component/main.tex}
            \input{3 - fluid component/main.tex}
            \input{4 - numerical implementation/main.tex}
        \part{Project Overview}
            \input{5 - research plan/main.tex}
            \input{6 - summary/main.tex}
    
    
    %\section{}
    \newpage
    \pagenumbering{gobble}
        \printbibliography


    \newpage
    \pagenumbering{roman}
    \appendix
        \part{Appendices}
            \input{8 - Hilbert complexes/main.tex}
            \input{9 - weak conservation proofs/main.tex}
\end{document}

            \documentclass[12pt, a4paper]{report}

\input{template/main.tex}

\title{\BA{Title in Progress...}}
\author{Boris Andrews}
\affil{Mathematical Institute, University of Oxford}
\date{\today}


\begin{document}
    \pagenumbering{gobble}
    \maketitle
    
    
    \begin{abstract}
        Magnetic confinement reactors---in particular tokamaks---offer one of the most promising options for achieving practical nuclear fusion, with the potential to provide virtually limitless, clean energy. The theoretical and numerical modeling of tokamak plasmas is simultaneously an essential component of effective reactor design, and a great research barrier. Tokamak operational conditions exhibit comparatively low Knudsen numbers. Kinetic effects, including kinetic waves and instabilities, Landau damping, bump-on-tail instabilities and more, are therefore highly influential in tokamak plasma dynamics. Purely fluid models are inherently incapable of capturing these effects, whereas the high dimensionality in purely kinetic models render them practically intractable for most relevant purposes.

        We consider a $\delta\!f$ decomposition model, with a macroscopic fluid background and microscopic kinetic correction, both fully coupled to each other. A similar manner of discretization is proposed to that used in the recent \texttt{STRUPHY} code \cite{Holderied_Possanner_Wang_2021, Holderied_2022, Li_et_al_2023} with a finite-element model for the background and a pseudo-particle/PiC model for the correction.

        The fluid background satisfies the full, non-linear, resistive, compressible, Hall MHD equations. \cite{Laakmann_Hu_Farrell_2022} introduces finite-element(-in-space) implicit timesteppers for the incompressible analogue to this system with structure-preserving (SP) properties in the ideal case, alongside parameter-robust preconditioners. We show that these timesteppers can derive from a finite-element-in-time (FET) (and finite-element-in-space) interpretation. The benefits of this reformulation are discussed, including the derivation of timesteppers that are higher order in time, and the quantifiable dissipative SP properties in the non-ideal, resistive case.
        
        We discuss possible options for extending this FET approach to timesteppers for the compressible case.

        The kinetic corrections satisfy linearized Boltzmann equations. Using a Lénard--Bernstein collision operator, these take Fokker--Planck-like forms \cite{Fokker_1914, Planck_1917} wherein pseudo-particles in the numerical model obey the neoclassical transport equations, with particle-independent Brownian drift terms. This offers a rigorous methodology for incorporating collisions into the particle transport model, without coupling the equations of motions for each particle.
        
        Works by Chen, Chacón et al. \cite{Chen_Chacón_Barnes_2011, Chacón_Chen_Barnes_2013, Chen_Chacón_2014, Chen_Chacón_2015} have developed structure-preserving particle pushers for neoclassical transport in the Vlasov equations, derived from Crank--Nicolson integrators. We show these too can can derive from a FET interpretation, similarly offering potential extensions to higher-order-in-time particle pushers. The FET formulation is used also to consider how the stochastic drift terms can be incorporated into the pushers. Stochastic gyrokinetic expansions are also discussed.

        Different options for the numerical implementation of these schemes are considered.

        Due to the efficacy of FET in the development of SP timesteppers for both the fluid and kinetic component, we hope this approach will prove effective in the future for developing SP timesteppers for the full hybrid model. We hope this will give us the opportunity to incorporate previously inaccessible kinetic effects into the highly effective, modern, finite-element MHD models.
    \end{abstract}
    
    
    \newpage
    \tableofcontents
    
    
    \newpage
    \pagenumbering{arabic}
    %\linenumbers\renewcommand\thelinenumber{\color{black!50}\arabic{linenumber}}
            \input{0 - introduction/main.tex}
        \part{Research}
            \input{1 - low-noise PiC models/main.tex}
            \input{2 - kinetic component/main.tex}
            \input{3 - fluid component/main.tex}
            \input{4 - numerical implementation/main.tex}
        \part{Project Overview}
            \input{5 - research plan/main.tex}
            \input{6 - summary/main.tex}
    
    
    %\section{}
    \newpage
    \pagenumbering{gobble}
        \printbibliography


    \newpage
    \pagenumbering{roman}
    \appendix
        \part{Appendices}
            \input{8 - Hilbert complexes/main.tex}
            \input{9 - weak conservation proofs/main.tex}
\end{document}

        \part{Project Overview}
            \documentclass[12pt, a4paper]{report}

\input{template/main.tex}

\title{\BA{Title in Progress...}}
\author{Boris Andrews}
\affil{Mathematical Institute, University of Oxford}
\date{\today}


\begin{document}
    \pagenumbering{gobble}
    \maketitle
    
    
    \begin{abstract}
        Magnetic confinement reactors---in particular tokamaks---offer one of the most promising options for achieving practical nuclear fusion, with the potential to provide virtually limitless, clean energy. The theoretical and numerical modeling of tokamak plasmas is simultaneously an essential component of effective reactor design, and a great research barrier. Tokamak operational conditions exhibit comparatively low Knudsen numbers. Kinetic effects, including kinetic waves and instabilities, Landau damping, bump-on-tail instabilities and more, are therefore highly influential in tokamak plasma dynamics. Purely fluid models are inherently incapable of capturing these effects, whereas the high dimensionality in purely kinetic models render them practically intractable for most relevant purposes.

        We consider a $\delta\!f$ decomposition model, with a macroscopic fluid background and microscopic kinetic correction, both fully coupled to each other. A similar manner of discretization is proposed to that used in the recent \texttt{STRUPHY} code \cite{Holderied_Possanner_Wang_2021, Holderied_2022, Li_et_al_2023} with a finite-element model for the background and a pseudo-particle/PiC model for the correction.

        The fluid background satisfies the full, non-linear, resistive, compressible, Hall MHD equations. \cite{Laakmann_Hu_Farrell_2022} introduces finite-element(-in-space) implicit timesteppers for the incompressible analogue to this system with structure-preserving (SP) properties in the ideal case, alongside parameter-robust preconditioners. We show that these timesteppers can derive from a finite-element-in-time (FET) (and finite-element-in-space) interpretation. The benefits of this reformulation are discussed, including the derivation of timesteppers that are higher order in time, and the quantifiable dissipative SP properties in the non-ideal, resistive case.
        
        We discuss possible options for extending this FET approach to timesteppers for the compressible case.

        The kinetic corrections satisfy linearized Boltzmann equations. Using a Lénard--Bernstein collision operator, these take Fokker--Planck-like forms \cite{Fokker_1914, Planck_1917} wherein pseudo-particles in the numerical model obey the neoclassical transport equations, with particle-independent Brownian drift terms. This offers a rigorous methodology for incorporating collisions into the particle transport model, without coupling the equations of motions for each particle.
        
        Works by Chen, Chacón et al. \cite{Chen_Chacón_Barnes_2011, Chacón_Chen_Barnes_2013, Chen_Chacón_2014, Chen_Chacón_2015} have developed structure-preserving particle pushers for neoclassical transport in the Vlasov equations, derived from Crank--Nicolson integrators. We show these too can can derive from a FET interpretation, similarly offering potential extensions to higher-order-in-time particle pushers. The FET formulation is used also to consider how the stochastic drift terms can be incorporated into the pushers. Stochastic gyrokinetic expansions are also discussed.

        Different options for the numerical implementation of these schemes are considered.

        Due to the efficacy of FET in the development of SP timesteppers for both the fluid and kinetic component, we hope this approach will prove effective in the future for developing SP timesteppers for the full hybrid model. We hope this will give us the opportunity to incorporate previously inaccessible kinetic effects into the highly effective, modern, finite-element MHD models.
    \end{abstract}
    
    
    \newpage
    \tableofcontents
    
    
    \newpage
    \pagenumbering{arabic}
    %\linenumbers\renewcommand\thelinenumber{\color{black!50}\arabic{linenumber}}
            \input{0 - introduction/main.tex}
        \part{Research}
            \input{1 - low-noise PiC models/main.tex}
            \input{2 - kinetic component/main.tex}
            \input{3 - fluid component/main.tex}
            \input{4 - numerical implementation/main.tex}
        \part{Project Overview}
            \input{5 - research plan/main.tex}
            \input{6 - summary/main.tex}
    
    
    %\section{}
    \newpage
    \pagenumbering{gobble}
        \printbibliography


    \newpage
    \pagenumbering{roman}
    \appendix
        \part{Appendices}
            \input{8 - Hilbert complexes/main.tex}
            \input{9 - weak conservation proofs/main.tex}
\end{document}

            \documentclass[12pt, a4paper]{report}

\input{template/main.tex}

\title{\BA{Title in Progress...}}
\author{Boris Andrews}
\affil{Mathematical Institute, University of Oxford}
\date{\today}


\begin{document}
    \pagenumbering{gobble}
    \maketitle
    
    
    \begin{abstract}
        Magnetic confinement reactors---in particular tokamaks---offer one of the most promising options for achieving practical nuclear fusion, with the potential to provide virtually limitless, clean energy. The theoretical and numerical modeling of tokamak plasmas is simultaneously an essential component of effective reactor design, and a great research barrier. Tokamak operational conditions exhibit comparatively low Knudsen numbers. Kinetic effects, including kinetic waves and instabilities, Landau damping, bump-on-tail instabilities and more, are therefore highly influential in tokamak plasma dynamics. Purely fluid models are inherently incapable of capturing these effects, whereas the high dimensionality in purely kinetic models render them practically intractable for most relevant purposes.

        We consider a $\delta\!f$ decomposition model, with a macroscopic fluid background and microscopic kinetic correction, both fully coupled to each other. A similar manner of discretization is proposed to that used in the recent \texttt{STRUPHY} code \cite{Holderied_Possanner_Wang_2021, Holderied_2022, Li_et_al_2023} with a finite-element model for the background and a pseudo-particle/PiC model for the correction.

        The fluid background satisfies the full, non-linear, resistive, compressible, Hall MHD equations. \cite{Laakmann_Hu_Farrell_2022} introduces finite-element(-in-space) implicit timesteppers for the incompressible analogue to this system with structure-preserving (SP) properties in the ideal case, alongside parameter-robust preconditioners. We show that these timesteppers can derive from a finite-element-in-time (FET) (and finite-element-in-space) interpretation. The benefits of this reformulation are discussed, including the derivation of timesteppers that are higher order in time, and the quantifiable dissipative SP properties in the non-ideal, resistive case.
        
        We discuss possible options for extending this FET approach to timesteppers for the compressible case.

        The kinetic corrections satisfy linearized Boltzmann equations. Using a Lénard--Bernstein collision operator, these take Fokker--Planck-like forms \cite{Fokker_1914, Planck_1917} wherein pseudo-particles in the numerical model obey the neoclassical transport equations, with particle-independent Brownian drift terms. This offers a rigorous methodology for incorporating collisions into the particle transport model, without coupling the equations of motions for each particle.
        
        Works by Chen, Chacón et al. \cite{Chen_Chacón_Barnes_2011, Chacón_Chen_Barnes_2013, Chen_Chacón_2014, Chen_Chacón_2015} have developed structure-preserving particle pushers for neoclassical transport in the Vlasov equations, derived from Crank--Nicolson integrators. We show these too can can derive from a FET interpretation, similarly offering potential extensions to higher-order-in-time particle pushers. The FET formulation is used also to consider how the stochastic drift terms can be incorporated into the pushers. Stochastic gyrokinetic expansions are also discussed.

        Different options for the numerical implementation of these schemes are considered.

        Due to the efficacy of FET in the development of SP timesteppers for both the fluid and kinetic component, we hope this approach will prove effective in the future for developing SP timesteppers for the full hybrid model. We hope this will give us the opportunity to incorporate previously inaccessible kinetic effects into the highly effective, modern, finite-element MHD models.
    \end{abstract}
    
    
    \newpage
    \tableofcontents
    
    
    \newpage
    \pagenumbering{arabic}
    %\linenumbers\renewcommand\thelinenumber{\color{black!50}\arabic{linenumber}}
            \input{0 - introduction/main.tex}
        \part{Research}
            \input{1 - low-noise PiC models/main.tex}
            \input{2 - kinetic component/main.tex}
            \input{3 - fluid component/main.tex}
            \input{4 - numerical implementation/main.tex}
        \part{Project Overview}
            \input{5 - research plan/main.tex}
            \input{6 - summary/main.tex}
    
    
    %\section{}
    \newpage
    \pagenumbering{gobble}
        \printbibliography


    \newpage
    \pagenumbering{roman}
    \appendix
        \part{Appendices}
            \input{8 - Hilbert complexes/main.tex}
            \input{9 - weak conservation proofs/main.tex}
\end{document}

    
    
    %\section{}
    \newpage
    \pagenumbering{gobble}
        \printbibliography


    \newpage
    \pagenumbering{roman}
    \appendix
        \part{Appendices}
            \documentclass[12pt, a4paper]{report}

\input{template/main.tex}

\title{\BA{Title in Progress...}}
\author{Boris Andrews}
\affil{Mathematical Institute, University of Oxford}
\date{\today}


\begin{document}
    \pagenumbering{gobble}
    \maketitle
    
    
    \begin{abstract}
        Magnetic confinement reactors---in particular tokamaks---offer one of the most promising options for achieving practical nuclear fusion, with the potential to provide virtually limitless, clean energy. The theoretical and numerical modeling of tokamak plasmas is simultaneously an essential component of effective reactor design, and a great research barrier. Tokamak operational conditions exhibit comparatively low Knudsen numbers. Kinetic effects, including kinetic waves and instabilities, Landau damping, bump-on-tail instabilities and more, are therefore highly influential in tokamak plasma dynamics. Purely fluid models are inherently incapable of capturing these effects, whereas the high dimensionality in purely kinetic models render them practically intractable for most relevant purposes.

        We consider a $\delta\!f$ decomposition model, with a macroscopic fluid background and microscopic kinetic correction, both fully coupled to each other. A similar manner of discretization is proposed to that used in the recent \texttt{STRUPHY} code \cite{Holderied_Possanner_Wang_2021, Holderied_2022, Li_et_al_2023} with a finite-element model for the background and a pseudo-particle/PiC model for the correction.

        The fluid background satisfies the full, non-linear, resistive, compressible, Hall MHD equations. \cite{Laakmann_Hu_Farrell_2022} introduces finite-element(-in-space) implicit timesteppers for the incompressible analogue to this system with structure-preserving (SP) properties in the ideal case, alongside parameter-robust preconditioners. We show that these timesteppers can derive from a finite-element-in-time (FET) (and finite-element-in-space) interpretation. The benefits of this reformulation are discussed, including the derivation of timesteppers that are higher order in time, and the quantifiable dissipative SP properties in the non-ideal, resistive case.
        
        We discuss possible options for extending this FET approach to timesteppers for the compressible case.

        The kinetic corrections satisfy linearized Boltzmann equations. Using a Lénard--Bernstein collision operator, these take Fokker--Planck-like forms \cite{Fokker_1914, Planck_1917} wherein pseudo-particles in the numerical model obey the neoclassical transport equations, with particle-independent Brownian drift terms. This offers a rigorous methodology for incorporating collisions into the particle transport model, without coupling the equations of motions for each particle.
        
        Works by Chen, Chacón et al. \cite{Chen_Chacón_Barnes_2011, Chacón_Chen_Barnes_2013, Chen_Chacón_2014, Chen_Chacón_2015} have developed structure-preserving particle pushers for neoclassical transport in the Vlasov equations, derived from Crank--Nicolson integrators. We show these too can can derive from a FET interpretation, similarly offering potential extensions to higher-order-in-time particle pushers. The FET formulation is used also to consider how the stochastic drift terms can be incorporated into the pushers. Stochastic gyrokinetic expansions are also discussed.

        Different options for the numerical implementation of these schemes are considered.

        Due to the efficacy of FET in the development of SP timesteppers for both the fluid and kinetic component, we hope this approach will prove effective in the future for developing SP timesteppers for the full hybrid model. We hope this will give us the opportunity to incorporate previously inaccessible kinetic effects into the highly effective, modern, finite-element MHD models.
    \end{abstract}
    
    
    \newpage
    \tableofcontents
    
    
    \newpage
    \pagenumbering{arabic}
    %\linenumbers\renewcommand\thelinenumber{\color{black!50}\arabic{linenumber}}
            \input{0 - introduction/main.tex}
        \part{Research}
            \input{1 - low-noise PiC models/main.tex}
            \input{2 - kinetic component/main.tex}
            \input{3 - fluid component/main.tex}
            \input{4 - numerical implementation/main.tex}
        \part{Project Overview}
            \input{5 - research plan/main.tex}
            \input{6 - summary/main.tex}
    
    
    %\section{}
    \newpage
    \pagenumbering{gobble}
        \printbibliography


    \newpage
    \pagenumbering{roman}
    \appendix
        \part{Appendices}
            \input{8 - Hilbert complexes/main.tex}
            \input{9 - weak conservation proofs/main.tex}
\end{document}

            \documentclass[12pt, a4paper]{report}

\input{template/main.tex}

\title{\BA{Title in Progress...}}
\author{Boris Andrews}
\affil{Mathematical Institute, University of Oxford}
\date{\today}


\begin{document}
    \pagenumbering{gobble}
    \maketitle
    
    
    \begin{abstract}
        Magnetic confinement reactors---in particular tokamaks---offer one of the most promising options for achieving practical nuclear fusion, with the potential to provide virtually limitless, clean energy. The theoretical and numerical modeling of tokamak plasmas is simultaneously an essential component of effective reactor design, and a great research barrier. Tokamak operational conditions exhibit comparatively low Knudsen numbers. Kinetic effects, including kinetic waves and instabilities, Landau damping, bump-on-tail instabilities and more, are therefore highly influential in tokamak plasma dynamics. Purely fluid models are inherently incapable of capturing these effects, whereas the high dimensionality in purely kinetic models render them practically intractable for most relevant purposes.

        We consider a $\delta\!f$ decomposition model, with a macroscopic fluid background and microscopic kinetic correction, both fully coupled to each other. A similar manner of discretization is proposed to that used in the recent \texttt{STRUPHY} code \cite{Holderied_Possanner_Wang_2021, Holderied_2022, Li_et_al_2023} with a finite-element model for the background and a pseudo-particle/PiC model for the correction.

        The fluid background satisfies the full, non-linear, resistive, compressible, Hall MHD equations. \cite{Laakmann_Hu_Farrell_2022} introduces finite-element(-in-space) implicit timesteppers for the incompressible analogue to this system with structure-preserving (SP) properties in the ideal case, alongside parameter-robust preconditioners. We show that these timesteppers can derive from a finite-element-in-time (FET) (and finite-element-in-space) interpretation. The benefits of this reformulation are discussed, including the derivation of timesteppers that are higher order in time, and the quantifiable dissipative SP properties in the non-ideal, resistive case.
        
        We discuss possible options for extending this FET approach to timesteppers for the compressible case.

        The kinetic corrections satisfy linearized Boltzmann equations. Using a Lénard--Bernstein collision operator, these take Fokker--Planck-like forms \cite{Fokker_1914, Planck_1917} wherein pseudo-particles in the numerical model obey the neoclassical transport equations, with particle-independent Brownian drift terms. This offers a rigorous methodology for incorporating collisions into the particle transport model, without coupling the equations of motions for each particle.
        
        Works by Chen, Chacón et al. \cite{Chen_Chacón_Barnes_2011, Chacón_Chen_Barnes_2013, Chen_Chacón_2014, Chen_Chacón_2015} have developed structure-preserving particle pushers for neoclassical transport in the Vlasov equations, derived from Crank--Nicolson integrators. We show these too can can derive from a FET interpretation, similarly offering potential extensions to higher-order-in-time particle pushers. The FET formulation is used also to consider how the stochastic drift terms can be incorporated into the pushers. Stochastic gyrokinetic expansions are also discussed.

        Different options for the numerical implementation of these schemes are considered.

        Due to the efficacy of FET in the development of SP timesteppers for both the fluid and kinetic component, we hope this approach will prove effective in the future for developing SP timesteppers for the full hybrid model. We hope this will give us the opportunity to incorporate previously inaccessible kinetic effects into the highly effective, modern, finite-element MHD models.
    \end{abstract}
    
    
    \newpage
    \tableofcontents
    
    
    \newpage
    \pagenumbering{arabic}
    %\linenumbers\renewcommand\thelinenumber{\color{black!50}\arabic{linenumber}}
            \input{0 - introduction/main.tex}
        \part{Research}
            \input{1 - low-noise PiC models/main.tex}
            \input{2 - kinetic component/main.tex}
            \input{3 - fluid component/main.tex}
            \input{4 - numerical implementation/main.tex}
        \part{Project Overview}
            \input{5 - research plan/main.tex}
            \input{6 - summary/main.tex}
    
    
    %\section{}
    \newpage
    \pagenumbering{gobble}
        \printbibliography


    \newpage
    \pagenumbering{roman}
    \appendix
        \part{Appendices}
            \input{8 - Hilbert complexes/main.tex}
            \input{9 - weak conservation proofs/main.tex}
\end{document}

\end{document}


\title{\BA{Title in Progress...}}
\author{Boris Andrews}
\affil{Mathematical Institute, University of Oxford}
\date{\today}


\begin{document}
    \pagenumbering{gobble}
    \maketitle
    
    
    \begin{abstract}
        Magnetic confinement reactors---in particular tokamaks---offer one of the most promising options for achieving practical nuclear fusion, with the potential to provide virtually limitless, clean energy. The theoretical and numerical modeling of tokamak plasmas is simultaneously an essential component of effective reactor design, and a great research barrier. Tokamak operational conditions exhibit comparatively low Knudsen numbers. Kinetic effects, including kinetic waves and instabilities, Landau damping, bump-on-tail instabilities and more, are therefore highly influential in tokamak plasma dynamics. Purely fluid models are inherently incapable of capturing these effects, whereas the high dimensionality in purely kinetic models render them practically intractable for most relevant purposes.

        We consider a $\delta\!f$ decomposition model, with a macroscopic fluid background and microscopic kinetic correction, both fully coupled to each other. A similar manner of discretization is proposed to that used in the recent \texttt{STRUPHY} code \cite{Holderied_Possanner_Wang_2021, Holderied_2022, Li_et_al_2023} with a finite-element model for the background and a pseudo-particle/PiC model for the correction.

        The fluid background satisfies the full, non-linear, resistive, compressible, Hall MHD equations. \cite{Laakmann_Hu_Farrell_2022} introduces finite-element(-in-space) implicit timesteppers for the incompressible analogue to this system with structure-preserving (SP) properties in the ideal case, alongside parameter-robust preconditioners. We show that these timesteppers can derive from a finite-element-in-time (FET) (and finite-element-in-space) interpretation. The benefits of this reformulation are discussed, including the derivation of timesteppers that are higher order in time, and the quantifiable dissipative SP properties in the non-ideal, resistive case.
        
        We discuss possible options for extending this FET approach to timesteppers for the compressible case.

        The kinetic corrections satisfy linearized Boltzmann equations. Using a Lénard--Bernstein collision operator, these take Fokker--Planck-like forms \cite{Fokker_1914, Planck_1917} wherein pseudo-particles in the numerical model obey the neoclassical transport equations, with particle-independent Brownian drift terms. This offers a rigorous methodology for incorporating collisions into the particle transport model, without coupling the equations of motions for each particle.
        
        Works by Chen, Chacón et al. \cite{Chen_Chacón_Barnes_2011, Chacón_Chen_Barnes_2013, Chen_Chacón_2014, Chen_Chacón_2015} have developed structure-preserving particle pushers for neoclassical transport in the Vlasov equations, derived from Crank--Nicolson integrators. We show these too can can derive from a FET interpretation, similarly offering potential extensions to higher-order-in-time particle pushers. The FET formulation is used also to consider how the stochastic drift terms can be incorporated into the pushers. Stochastic gyrokinetic expansions are also discussed.

        Different options for the numerical implementation of these schemes are considered.

        Due to the efficacy of FET in the development of SP timesteppers for both the fluid and kinetic component, we hope this approach will prove effective in the future for developing SP timesteppers for the full hybrid model. We hope this will give us the opportunity to incorporate previously inaccessible kinetic effects into the highly effective, modern, finite-element MHD models.
    \end{abstract}
    
    
    \newpage
    \tableofcontents
    
    
    \newpage
    \pagenumbering{arabic}
    %\linenumbers\renewcommand\thelinenumber{\color{black!50}\arabic{linenumber}}
            \documentclass[12pt, a4paper]{report}

\documentclass[12pt, a4paper]{report}

\input{template/main.tex}

\title{\BA{Title in Progress...}}
\author{Boris Andrews}
\affil{Mathematical Institute, University of Oxford}
\date{\today}


\begin{document}
    \pagenumbering{gobble}
    \maketitle
    
    
    \begin{abstract}
        Magnetic confinement reactors---in particular tokamaks---offer one of the most promising options for achieving practical nuclear fusion, with the potential to provide virtually limitless, clean energy. The theoretical and numerical modeling of tokamak plasmas is simultaneously an essential component of effective reactor design, and a great research barrier. Tokamak operational conditions exhibit comparatively low Knudsen numbers. Kinetic effects, including kinetic waves and instabilities, Landau damping, bump-on-tail instabilities and more, are therefore highly influential in tokamak plasma dynamics. Purely fluid models are inherently incapable of capturing these effects, whereas the high dimensionality in purely kinetic models render them practically intractable for most relevant purposes.

        We consider a $\delta\!f$ decomposition model, with a macroscopic fluid background and microscopic kinetic correction, both fully coupled to each other. A similar manner of discretization is proposed to that used in the recent \texttt{STRUPHY} code \cite{Holderied_Possanner_Wang_2021, Holderied_2022, Li_et_al_2023} with a finite-element model for the background and a pseudo-particle/PiC model for the correction.

        The fluid background satisfies the full, non-linear, resistive, compressible, Hall MHD equations. \cite{Laakmann_Hu_Farrell_2022} introduces finite-element(-in-space) implicit timesteppers for the incompressible analogue to this system with structure-preserving (SP) properties in the ideal case, alongside parameter-robust preconditioners. We show that these timesteppers can derive from a finite-element-in-time (FET) (and finite-element-in-space) interpretation. The benefits of this reformulation are discussed, including the derivation of timesteppers that are higher order in time, and the quantifiable dissipative SP properties in the non-ideal, resistive case.
        
        We discuss possible options for extending this FET approach to timesteppers for the compressible case.

        The kinetic corrections satisfy linearized Boltzmann equations. Using a Lénard--Bernstein collision operator, these take Fokker--Planck-like forms \cite{Fokker_1914, Planck_1917} wherein pseudo-particles in the numerical model obey the neoclassical transport equations, with particle-independent Brownian drift terms. This offers a rigorous methodology for incorporating collisions into the particle transport model, without coupling the equations of motions for each particle.
        
        Works by Chen, Chacón et al. \cite{Chen_Chacón_Barnes_2011, Chacón_Chen_Barnes_2013, Chen_Chacón_2014, Chen_Chacón_2015} have developed structure-preserving particle pushers for neoclassical transport in the Vlasov equations, derived from Crank--Nicolson integrators. We show these too can can derive from a FET interpretation, similarly offering potential extensions to higher-order-in-time particle pushers. The FET formulation is used also to consider how the stochastic drift terms can be incorporated into the pushers. Stochastic gyrokinetic expansions are also discussed.

        Different options for the numerical implementation of these schemes are considered.

        Due to the efficacy of FET in the development of SP timesteppers for both the fluid and kinetic component, we hope this approach will prove effective in the future for developing SP timesteppers for the full hybrid model. We hope this will give us the opportunity to incorporate previously inaccessible kinetic effects into the highly effective, modern, finite-element MHD models.
    \end{abstract}
    
    
    \newpage
    \tableofcontents
    
    
    \newpage
    \pagenumbering{arabic}
    %\linenumbers\renewcommand\thelinenumber{\color{black!50}\arabic{linenumber}}
            \input{0 - introduction/main.tex}
        \part{Research}
            \input{1 - low-noise PiC models/main.tex}
            \input{2 - kinetic component/main.tex}
            \input{3 - fluid component/main.tex}
            \input{4 - numerical implementation/main.tex}
        \part{Project Overview}
            \input{5 - research plan/main.tex}
            \input{6 - summary/main.tex}
    
    
    %\section{}
    \newpage
    \pagenumbering{gobble}
        \printbibliography


    \newpage
    \pagenumbering{roman}
    \appendix
        \part{Appendices}
            \input{8 - Hilbert complexes/main.tex}
            \input{9 - weak conservation proofs/main.tex}
\end{document}


\title{\BA{Title in Progress...}}
\author{Boris Andrews}
\affil{Mathematical Institute, University of Oxford}
\date{\today}


\begin{document}
    \pagenumbering{gobble}
    \maketitle
    
    
    \begin{abstract}
        Magnetic confinement reactors---in particular tokamaks---offer one of the most promising options for achieving practical nuclear fusion, with the potential to provide virtually limitless, clean energy. The theoretical and numerical modeling of tokamak plasmas is simultaneously an essential component of effective reactor design, and a great research barrier. Tokamak operational conditions exhibit comparatively low Knudsen numbers. Kinetic effects, including kinetic waves and instabilities, Landau damping, bump-on-tail instabilities and more, are therefore highly influential in tokamak plasma dynamics. Purely fluid models are inherently incapable of capturing these effects, whereas the high dimensionality in purely kinetic models render them practically intractable for most relevant purposes.

        We consider a $\delta\!f$ decomposition model, with a macroscopic fluid background and microscopic kinetic correction, both fully coupled to each other. A similar manner of discretization is proposed to that used in the recent \texttt{STRUPHY} code \cite{Holderied_Possanner_Wang_2021, Holderied_2022, Li_et_al_2023} with a finite-element model for the background and a pseudo-particle/PiC model for the correction.

        The fluid background satisfies the full, non-linear, resistive, compressible, Hall MHD equations. \cite{Laakmann_Hu_Farrell_2022} introduces finite-element(-in-space) implicit timesteppers for the incompressible analogue to this system with structure-preserving (SP) properties in the ideal case, alongside parameter-robust preconditioners. We show that these timesteppers can derive from a finite-element-in-time (FET) (and finite-element-in-space) interpretation. The benefits of this reformulation are discussed, including the derivation of timesteppers that are higher order in time, and the quantifiable dissipative SP properties in the non-ideal, resistive case.
        
        We discuss possible options for extending this FET approach to timesteppers for the compressible case.

        The kinetic corrections satisfy linearized Boltzmann equations. Using a Lénard--Bernstein collision operator, these take Fokker--Planck-like forms \cite{Fokker_1914, Planck_1917} wherein pseudo-particles in the numerical model obey the neoclassical transport equations, with particle-independent Brownian drift terms. This offers a rigorous methodology for incorporating collisions into the particle transport model, without coupling the equations of motions for each particle.
        
        Works by Chen, Chacón et al. \cite{Chen_Chacón_Barnes_2011, Chacón_Chen_Barnes_2013, Chen_Chacón_2014, Chen_Chacón_2015} have developed structure-preserving particle pushers for neoclassical transport in the Vlasov equations, derived from Crank--Nicolson integrators. We show these too can can derive from a FET interpretation, similarly offering potential extensions to higher-order-in-time particle pushers. The FET formulation is used also to consider how the stochastic drift terms can be incorporated into the pushers. Stochastic gyrokinetic expansions are also discussed.

        Different options for the numerical implementation of these schemes are considered.

        Due to the efficacy of FET in the development of SP timesteppers for both the fluid and kinetic component, we hope this approach will prove effective in the future for developing SP timesteppers for the full hybrid model. We hope this will give us the opportunity to incorporate previously inaccessible kinetic effects into the highly effective, modern, finite-element MHD models.
    \end{abstract}
    
    
    \newpage
    \tableofcontents
    
    
    \newpage
    \pagenumbering{arabic}
    %\linenumbers\renewcommand\thelinenumber{\color{black!50}\arabic{linenumber}}
            \documentclass[12pt, a4paper]{report}

\input{template/main.tex}

\title{\BA{Title in Progress...}}
\author{Boris Andrews}
\affil{Mathematical Institute, University of Oxford}
\date{\today}


\begin{document}
    \pagenumbering{gobble}
    \maketitle
    
    
    \begin{abstract}
        Magnetic confinement reactors---in particular tokamaks---offer one of the most promising options for achieving practical nuclear fusion, with the potential to provide virtually limitless, clean energy. The theoretical and numerical modeling of tokamak plasmas is simultaneously an essential component of effective reactor design, and a great research barrier. Tokamak operational conditions exhibit comparatively low Knudsen numbers. Kinetic effects, including kinetic waves and instabilities, Landau damping, bump-on-tail instabilities and more, are therefore highly influential in tokamak plasma dynamics. Purely fluid models are inherently incapable of capturing these effects, whereas the high dimensionality in purely kinetic models render them practically intractable for most relevant purposes.

        We consider a $\delta\!f$ decomposition model, with a macroscopic fluid background and microscopic kinetic correction, both fully coupled to each other. A similar manner of discretization is proposed to that used in the recent \texttt{STRUPHY} code \cite{Holderied_Possanner_Wang_2021, Holderied_2022, Li_et_al_2023} with a finite-element model for the background and a pseudo-particle/PiC model for the correction.

        The fluid background satisfies the full, non-linear, resistive, compressible, Hall MHD equations. \cite{Laakmann_Hu_Farrell_2022} introduces finite-element(-in-space) implicit timesteppers for the incompressible analogue to this system with structure-preserving (SP) properties in the ideal case, alongside parameter-robust preconditioners. We show that these timesteppers can derive from a finite-element-in-time (FET) (and finite-element-in-space) interpretation. The benefits of this reformulation are discussed, including the derivation of timesteppers that are higher order in time, and the quantifiable dissipative SP properties in the non-ideal, resistive case.
        
        We discuss possible options for extending this FET approach to timesteppers for the compressible case.

        The kinetic corrections satisfy linearized Boltzmann equations. Using a Lénard--Bernstein collision operator, these take Fokker--Planck-like forms \cite{Fokker_1914, Planck_1917} wherein pseudo-particles in the numerical model obey the neoclassical transport equations, with particle-independent Brownian drift terms. This offers a rigorous methodology for incorporating collisions into the particle transport model, without coupling the equations of motions for each particle.
        
        Works by Chen, Chacón et al. \cite{Chen_Chacón_Barnes_2011, Chacón_Chen_Barnes_2013, Chen_Chacón_2014, Chen_Chacón_2015} have developed structure-preserving particle pushers for neoclassical transport in the Vlasov equations, derived from Crank--Nicolson integrators. We show these too can can derive from a FET interpretation, similarly offering potential extensions to higher-order-in-time particle pushers. The FET formulation is used also to consider how the stochastic drift terms can be incorporated into the pushers. Stochastic gyrokinetic expansions are also discussed.

        Different options for the numerical implementation of these schemes are considered.

        Due to the efficacy of FET in the development of SP timesteppers for both the fluid and kinetic component, we hope this approach will prove effective in the future for developing SP timesteppers for the full hybrid model. We hope this will give us the opportunity to incorporate previously inaccessible kinetic effects into the highly effective, modern, finite-element MHD models.
    \end{abstract}
    
    
    \newpage
    \tableofcontents
    
    
    \newpage
    \pagenumbering{arabic}
    %\linenumbers\renewcommand\thelinenumber{\color{black!50}\arabic{linenumber}}
            \input{0 - introduction/main.tex}
        \part{Research}
            \input{1 - low-noise PiC models/main.tex}
            \input{2 - kinetic component/main.tex}
            \input{3 - fluid component/main.tex}
            \input{4 - numerical implementation/main.tex}
        \part{Project Overview}
            \input{5 - research plan/main.tex}
            \input{6 - summary/main.tex}
    
    
    %\section{}
    \newpage
    \pagenumbering{gobble}
        \printbibliography


    \newpage
    \pagenumbering{roman}
    \appendix
        \part{Appendices}
            \input{8 - Hilbert complexes/main.tex}
            \input{9 - weak conservation proofs/main.tex}
\end{document}

        \part{Research}
            \documentclass[12pt, a4paper]{report}

\input{template/main.tex}

\title{\BA{Title in Progress...}}
\author{Boris Andrews}
\affil{Mathematical Institute, University of Oxford}
\date{\today}


\begin{document}
    \pagenumbering{gobble}
    \maketitle
    
    
    \begin{abstract}
        Magnetic confinement reactors---in particular tokamaks---offer one of the most promising options for achieving practical nuclear fusion, with the potential to provide virtually limitless, clean energy. The theoretical and numerical modeling of tokamak plasmas is simultaneously an essential component of effective reactor design, and a great research barrier. Tokamak operational conditions exhibit comparatively low Knudsen numbers. Kinetic effects, including kinetic waves and instabilities, Landau damping, bump-on-tail instabilities and more, are therefore highly influential in tokamak plasma dynamics. Purely fluid models are inherently incapable of capturing these effects, whereas the high dimensionality in purely kinetic models render them practically intractable for most relevant purposes.

        We consider a $\delta\!f$ decomposition model, with a macroscopic fluid background and microscopic kinetic correction, both fully coupled to each other. A similar manner of discretization is proposed to that used in the recent \texttt{STRUPHY} code \cite{Holderied_Possanner_Wang_2021, Holderied_2022, Li_et_al_2023} with a finite-element model for the background and a pseudo-particle/PiC model for the correction.

        The fluid background satisfies the full, non-linear, resistive, compressible, Hall MHD equations. \cite{Laakmann_Hu_Farrell_2022} introduces finite-element(-in-space) implicit timesteppers for the incompressible analogue to this system with structure-preserving (SP) properties in the ideal case, alongside parameter-robust preconditioners. We show that these timesteppers can derive from a finite-element-in-time (FET) (and finite-element-in-space) interpretation. The benefits of this reformulation are discussed, including the derivation of timesteppers that are higher order in time, and the quantifiable dissipative SP properties in the non-ideal, resistive case.
        
        We discuss possible options for extending this FET approach to timesteppers for the compressible case.

        The kinetic corrections satisfy linearized Boltzmann equations. Using a Lénard--Bernstein collision operator, these take Fokker--Planck-like forms \cite{Fokker_1914, Planck_1917} wherein pseudo-particles in the numerical model obey the neoclassical transport equations, with particle-independent Brownian drift terms. This offers a rigorous methodology for incorporating collisions into the particle transport model, without coupling the equations of motions for each particle.
        
        Works by Chen, Chacón et al. \cite{Chen_Chacón_Barnes_2011, Chacón_Chen_Barnes_2013, Chen_Chacón_2014, Chen_Chacón_2015} have developed structure-preserving particle pushers for neoclassical transport in the Vlasov equations, derived from Crank--Nicolson integrators. We show these too can can derive from a FET interpretation, similarly offering potential extensions to higher-order-in-time particle pushers. The FET formulation is used also to consider how the stochastic drift terms can be incorporated into the pushers. Stochastic gyrokinetic expansions are also discussed.

        Different options for the numerical implementation of these schemes are considered.

        Due to the efficacy of FET in the development of SP timesteppers for both the fluid and kinetic component, we hope this approach will prove effective in the future for developing SP timesteppers for the full hybrid model. We hope this will give us the opportunity to incorporate previously inaccessible kinetic effects into the highly effective, modern, finite-element MHD models.
    \end{abstract}
    
    
    \newpage
    \tableofcontents
    
    
    \newpage
    \pagenumbering{arabic}
    %\linenumbers\renewcommand\thelinenumber{\color{black!50}\arabic{linenumber}}
            \input{0 - introduction/main.tex}
        \part{Research}
            \input{1 - low-noise PiC models/main.tex}
            \input{2 - kinetic component/main.tex}
            \input{3 - fluid component/main.tex}
            \input{4 - numerical implementation/main.tex}
        \part{Project Overview}
            \input{5 - research plan/main.tex}
            \input{6 - summary/main.tex}
    
    
    %\section{}
    \newpage
    \pagenumbering{gobble}
        \printbibliography


    \newpage
    \pagenumbering{roman}
    \appendix
        \part{Appendices}
            \input{8 - Hilbert complexes/main.tex}
            \input{9 - weak conservation proofs/main.tex}
\end{document}

            \documentclass[12pt, a4paper]{report}

\input{template/main.tex}

\title{\BA{Title in Progress...}}
\author{Boris Andrews}
\affil{Mathematical Institute, University of Oxford}
\date{\today}


\begin{document}
    \pagenumbering{gobble}
    \maketitle
    
    
    \begin{abstract}
        Magnetic confinement reactors---in particular tokamaks---offer one of the most promising options for achieving practical nuclear fusion, with the potential to provide virtually limitless, clean energy. The theoretical and numerical modeling of tokamak plasmas is simultaneously an essential component of effective reactor design, and a great research barrier. Tokamak operational conditions exhibit comparatively low Knudsen numbers. Kinetic effects, including kinetic waves and instabilities, Landau damping, bump-on-tail instabilities and more, are therefore highly influential in tokamak plasma dynamics. Purely fluid models are inherently incapable of capturing these effects, whereas the high dimensionality in purely kinetic models render them practically intractable for most relevant purposes.

        We consider a $\delta\!f$ decomposition model, with a macroscopic fluid background and microscopic kinetic correction, both fully coupled to each other. A similar manner of discretization is proposed to that used in the recent \texttt{STRUPHY} code \cite{Holderied_Possanner_Wang_2021, Holderied_2022, Li_et_al_2023} with a finite-element model for the background and a pseudo-particle/PiC model for the correction.

        The fluid background satisfies the full, non-linear, resistive, compressible, Hall MHD equations. \cite{Laakmann_Hu_Farrell_2022} introduces finite-element(-in-space) implicit timesteppers for the incompressible analogue to this system with structure-preserving (SP) properties in the ideal case, alongside parameter-robust preconditioners. We show that these timesteppers can derive from a finite-element-in-time (FET) (and finite-element-in-space) interpretation. The benefits of this reformulation are discussed, including the derivation of timesteppers that are higher order in time, and the quantifiable dissipative SP properties in the non-ideal, resistive case.
        
        We discuss possible options for extending this FET approach to timesteppers for the compressible case.

        The kinetic corrections satisfy linearized Boltzmann equations. Using a Lénard--Bernstein collision operator, these take Fokker--Planck-like forms \cite{Fokker_1914, Planck_1917} wherein pseudo-particles in the numerical model obey the neoclassical transport equations, with particle-independent Brownian drift terms. This offers a rigorous methodology for incorporating collisions into the particle transport model, without coupling the equations of motions for each particle.
        
        Works by Chen, Chacón et al. \cite{Chen_Chacón_Barnes_2011, Chacón_Chen_Barnes_2013, Chen_Chacón_2014, Chen_Chacón_2015} have developed structure-preserving particle pushers for neoclassical transport in the Vlasov equations, derived from Crank--Nicolson integrators. We show these too can can derive from a FET interpretation, similarly offering potential extensions to higher-order-in-time particle pushers. The FET formulation is used also to consider how the stochastic drift terms can be incorporated into the pushers. Stochastic gyrokinetic expansions are also discussed.

        Different options for the numerical implementation of these schemes are considered.

        Due to the efficacy of FET in the development of SP timesteppers for both the fluid and kinetic component, we hope this approach will prove effective in the future for developing SP timesteppers for the full hybrid model. We hope this will give us the opportunity to incorporate previously inaccessible kinetic effects into the highly effective, modern, finite-element MHD models.
    \end{abstract}
    
    
    \newpage
    \tableofcontents
    
    
    \newpage
    \pagenumbering{arabic}
    %\linenumbers\renewcommand\thelinenumber{\color{black!50}\arabic{linenumber}}
            \input{0 - introduction/main.tex}
        \part{Research}
            \input{1 - low-noise PiC models/main.tex}
            \input{2 - kinetic component/main.tex}
            \input{3 - fluid component/main.tex}
            \input{4 - numerical implementation/main.tex}
        \part{Project Overview}
            \input{5 - research plan/main.tex}
            \input{6 - summary/main.tex}
    
    
    %\section{}
    \newpage
    \pagenumbering{gobble}
        \printbibliography


    \newpage
    \pagenumbering{roman}
    \appendix
        \part{Appendices}
            \input{8 - Hilbert complexes/main.tex}
            \input{9 - weak conservation proofs/main.tex}
\end{document}

            \documentclass[12pt, a4paper]{report}

\input{template/main.tex}

\title{\BA{Title in Progress...}}
\author{Boris Andrews}
\affil{Mathematical Institute, University of Oxford}
\date{\today}


\begin{document}
    \pagenumbering{gobble}
    \maketitle
    
    
    \begin{abstract}
        Magnetic confinement reactors---in particular tokamaks---offer one of the most promising options for achieving practical nuclear fusion, with the potential to provide virtually limitless, clean energy. The theoretical and numerical modeling of tokamak plasmas is simultaneously an essential component of effective reactor design, and a great research barrier. Tokamak operational conditions exhibit comparatively low Knudsen numbers. Kinetic effects, including kinetic waves and instabilities, Landau damping, bump-on-tail instabilities and more, are therefore highly influential in tokamak plasma dynamics. Purely fluid models are inherently incapable of capturing these effects, whereas the high dimensionality in purely kinetic models render them practically intractable for most relevant purposes.

        We consider a $\delta\!f$ decomposition model, with a macroscopic fluid background and microscopic kinetic correction, both fully coupled to each other. A similar manner of discretization is proposed to that used in the recent \texttt{STRUPHY} code \cite{Holderied_Possanner_Wang_2021, Holderied_2022, Li_et_al_2023} with a finite-element model for the background and a pseudo-particle/PiC model for the correction.

        The fluid background satisfies the full, non-linear, resistive, compressible, Hall MHD equations. \cite{Laakmann_Hu_Farrell_2022} introduces finite-element(-in-space) implicit timesteppers for the incompressible analogue to this system with structure-preserving (SP) properties in the ideal case, alongside parameter-robust preconditioners. We show that these timesteppers can derive from a finite-element-in-time (FET) (and finite-element-in-space) interpretation. The benefits of this reformulation are discussed, including the derivation of timesteppers that are higher order in time, and the quantifiable dissipative SP properties in the non-ideal, resistive case.
        
        We discuss possible options for extending this FET approach to timesteppers for the compressible case.

        The kinetic corrections satisfy linearized Boltzmann equations. Using a Lénard--Bernstein collision operator, these take Fokker--Planck-like forms \cite{Fokker_1914, Planck_1917} wherein pseudo-particles in the numerical model obey the neoclassical transport equations, with particle-independent Brownian drift terms. This offers a rigorous methodology for incorporating collisions into the particle transport model, without coupling the equations of motions for each particle.
        
        Works by Chen, Chacón et al. \cite{Chen_Chacón_Barnes_2011, Chacón_Chen_Barnes_2013, Chen_Chacón_2014, Chen_Chacón_2015} have developed structure-preserving particle pushers for neoclassical transport in the Vlasov equations, derived from Crank--Nicolson integrators. We show these too can can derive from a FET interpretation, similarly offering potential extensions to higher-order-in-time particle pushers. The FET formulation is used also to consider how the stochastic drift terms can be incorporated into the pushers. Stochastic gyrokinetic expansions are also discussed.

        Different options for the numerical implementation of these schemes are considered.

        Due to the efficacy of FET in the development of SP timesteppers for both the fluid and kinetic component, we hope this approach will prove effective in the future for developing SP timesteppers for the full hybrid model. We hope this will give us the opportunity to incorporate previously inaccessible kinetic effects into the highly effective, modern, finite-element MHD models.
    \end{abstract}
    
    
    \newpage
    \tableofcontents
    
    
    \newpage
    \pagenumbering{arabic}
    %\linenumbers\renewcommand\thelinenumber{\color{black!50}\arabic{linenumber}}
            \input{0 - introduction/main.tex}
        \part{Research}
            \input{1 - low-noise PiC models/main.tex}
            \input{2 - kinetic component/main.tex}
            \input{3 - fluid component/main.tex}
            \input{4 - numerical implementation/main.tex}
        \part{Project Overview}
            \input{5 - research plan/main.tex}
            \input{6 - summary/main.tex}
    
    
    %\section{}
    \newpage
    \pagenumbering{gobble}
        \printbibliography


    \newpage
    \pagenumbering{roman}
    \appendix
        \part{Appendices}
            \input{8 - Hilbert complexes/main.tex}
            \input{9 - weak conservation proofs/main.tex}
\end{document}

            \documentclass[12pt, a4paper]{report}

\input{template/main.tex}

\title{\BA{Title in Progress...}}
\author{Boris Andrews}
\affil{Mathematical Institute, University of Oxford}
\date{\today}


\begin{document}
    \pagenumbering{gobble}
    \maketitle
    
    
    \begin{abstract}
        Magnetic confinement reactors---in particular tokamaks---offer one of the most promising options for achieving practical nuclear fusion, with the potential to provide virtually limitless, clean energy. The theoretical and numerical modeling of tokamak plasmas is simultaneously an essential component of effective reactor design, and a great research barrier. Tokamak operational conditions exhibit comparatively low Knudsen numbers. Kinetic effects, including kinetic waves and instabilities, Landau damping, bump-on-tail instabilities and more, are therefore highly influential in tokamak plasma dynamics. Purely fluid models are inherently incapable of capturing these effects, whereas the high dimensionality in purely kinetic models render them practically intractable for most relevant purposes.

        We consider a $\delta\!f$ decomposition model, with a macroscopic fluid background and microscopic kinetic correction, both fully coupled to each other. A similar manner of discretization is proposed to that used in the recent \texttt{STRUPHY} code \cite{Holderied_Possanner_Wang_2021, Holderied_2022, Li_et_al_2023} with a finite-element model for the background and a pseudo-particle/PiC model for the correction.

        The fluid background satisfies the full, non-linear, resistive, compressible, Hall MHD equations. \cite{Laakmann_Hu_Farrell_2022} introduces finite-element(-in-space) implicit timesteppers for the incompressible analogue to this system with structure-preserving (SP) properties in the ideal case, alongside parameter-robust preconditioners. We show that these timesteppers can derive from a finite-element-in-time (FET) (and finite-element-in-space) interpretation. The benefits of this reformulation are discussed, including the derivation of timesteppers that are higher order in time, and the quantifiable dissipative SP properties in the non-ideal, resistive case.
        
        We discuss possible options for extending this FET approach to timesteppers for the compressible case.

        The kinetic corrections satisfy linearized Boltzmann equations. Using a Lénard--Bernstein collision operator, these take Fokker--Planck-like forms \cite{Fokker_1914, Planck_1917} wherein pseudo-particles in the numerical model obey the neoclassical transport equations, with particle-independent Brownian drift terms. This offers a rigorous methodology for incorporating collisions into the particle transport model, without coupling the equations of motions for each particle.
        
        Works by Chen, Chacón et al. \cite{Chen_Chacón_Barnes_2011, Chacón_Chen_Barnes_2013, Chen_Chacón_2014, Chen_Chacón_2015} have developed structure-preserving particle pushers for neoclassical transport in the Vlasov equations, derived from Crank--Nicolson integrators. We show these too can can derive from a FET interpretation, similarly offering potential extensions to higher-order-in-time particle pushers. The FET formulation is used also to consider how the stochastic drift terms can be incorporated into the pushers. Stochastic gyrokinetic expansions are also discussed.

        Different options for the numerical implementation of these schemes are considered.

        Due to the efficacy of FET in the development of SP timesteppers for both the fluid and kinetic component, we hope this approach will prove effective in the future for developing SP timesteppers for the full hybrid model. We hope this will give us the opportunity to incorporate previously inaccessible kinetic effects into the highly effective, modern, finite-element MHD models.
    \end{abstract}
    
    
    \newpage
    \tableofcontents
    
    
    \newpage
    \pagenumbering{arabic}
    %\linenumbers\renewcommand\thelinenumber{\color{black!50}\arabic{linenumber}}
            \input{0 - introduction/main.tex}
        \part{Research}
            \input{1 - low-noise PiC models/main.tex}
            \input{2 - kinetic component/main.tex}
            \input{3 - fluid component/main.tex}
            \input{4 - numerical implementation/main.tex}
        \part{Project Overview}
            \input{5 - research plan/main.tex}
            \input{6 - summary/main.tex}
    
    
    %\section{}
    \newpage
    \pagenumbering{gobble}
        \printbibliography


    \newpage
    \pagenumbering{roman}
    \appendix
        \part{Appendices}
            \input{8 - Hilbert complexes/main.tex}
            \input{9 - weak conservation proofs/main.tex}
\end{document}

        \part{Project Overview}
            \documentclass[12pt, a4paper]{report}

\input{template/main.tex}

\title{\BA{Title in Progress...}}
\author{Boris Andrews}
\affil{Mathematical Institute, University of Oxford}
\date{\today}


\begin{document}
    \pagenumbering{gobble}
    \maketitle
    
    
    \begin{abstract}
        Magnetic confinement reactors---in particular tokamaks---offer one of the most promising options for achieving practical nuclear fusion, with the potential to provide virtually limitless, clean energy. The theoretical and numerical modeling of tokamak plasmas is simultaneously an essential component of effective reactor design, and a great research barrier. Tokamak operational conditions exhibit comparatively low Knudsen numbers. Kinetic effects, including kinetic waves and instabilities, Landau damping, bump-on-tail instabilities and more, are therefore highly influential in tokamak plasma dynamics. Purely fluid models are inherently incapable of capturing these effects, whereas the high dimensionality in purely kinetic models render them practically intractable for most relevant purposes.

        We consider a $\delta\!f$ decomposition model, with a macroscopic fluid background and microscopic kinetic correction, both fully coupled to each other. A similar manner of discretization is proposed to that used in the recent \texttt{STRUPHY} code \cite{Holderied_Possanner_Wang_2021, Holderied_2022, Li_et_al_2023} with a finite-element model for the background and a pseudo-particle/PiC model for the correction.

        The fluid background satisfies the full, non-linear, resistive, compressible, Hall MHD equations. \cite{Laakmann_Hu_Farrell_2022} introduces finite-element(-in-space) implicit timesteppers for the incompressible analogue to this system with structure-preserving (SP) properties in the ideal case, alongside parameter-robust preconditioners. We show that these timesteppers can derive from a finite-element-in-time (FET) (and finite-element-in-space) interpretation. The benefits of this reformulation are discussed, including the derivation of timesteppers that are higher order in time, and the quantifiable dissipative SP properties in the non-ideal, resistive case.
        
        We discuss possible options for extending this FET approach to timesteppers for the compressible case.

        The kinetic corrections satisfy linearized Boltzmann equations. Using a Lénard--Bernstein collision operator, these take Fokker--Planck-like forms \cite{Fokker_1914, Planck_1917} wherein pseudo-particles in the numerical model obey the neoclassical transport equations, with particle-independent Brownian drift terms. This offers a rigorous methodology for incorporating collisions into the particle transport model, without coupling the equations of motions for each particle.
        
        Works by Chen, Chacón et al. \cite{Chen_Chacón_Barnes_2011, Chacón_Chen_Barnes_2013, Chen_Chacón_2014, Chen_Chacón_2015} have developed structure-preserving particle pushers for neoclassical transport in the Vlasov equations, derived from Crank--Nicolson integrators. We show these too can can derive from a FET interpretation, similarly offering potential extensions to higher-order-in-time particle pushers. The FET formulation is used also to consider how the stochastic drift terms can be incorporated into the pushers. Stochastic gyrokinetic expansions are also discussed.

        Different options for the numerical implementation of these schemes are considered.

        Due to the efficacy of FET in the development of SP timesteppers for both the fluid and kinetic component, we hope this approach will prove effective in the future for developing SP timesteppers for the full hybrid model. We hope this will give us the opportunity to incorporate previously inaccessible kinetic effects into the highly effective, modern, finite-element MHD models.
    \end{abstract}
    
    
    \newpage
    \tableofcontents
    
    
    \newpage
    \pagenumbering{arabic}
    %\linenumbers\renewcommand\thelinenumber{\color{black!50}\arabic{linenumber}}
            \input{0 - introduction/main.tex}
        \part{Research}
            \input{1 - low-noise PiC models/main.tex}
            \input{2 - kinetic component/main.tex}
            \input{3 - fluid component/main.tex}
            \input{4 - numerical implementation/main.tex}
        \part{Project Overview}
            \input{5 - research plan/main.tex}
            \input{6 - summary/main.tex}
    
    
    %\section{}
    \newpage
    \pagenumbering{gobble}
        \printbibliography


    \newpage
    \pagenumbering{roman}
    \appendix
        \part{Appendices}
            \input{8 - Hilbert complexes/main.tex}
            \input{9 - weak conservation proofs/main.tex}
\end{document}

            \documentclass[12pt, a4paper]{report}

\input{template/main.tex}

\title{\BA{Title in Progress...}}
\author{Boris Andrews}
\affil{Mathematical Institute, University of Oxford}
\date{\today}


\begin{document}
    \pagenumbering{gobble}
    \maketitle
    
    
    \begin{abstract}
        Magnetic confinement reactors---in particular tokamaks---offer one of the most promising options for achieving practical nuclear fusion, with the potential to provide virtually limitless, clean energy. The theoretical and numerical modeling of tokamak plasmas is simultaneously an essential component of effective reactor design, and a great research barrier. Tokamak operational conditions exhibit comparatively low Knudsen numbers. Kinetic effects, including kinetic waves and instabilities, Landau damping, bump-on-tail instabilities and more, are therefore highly influential in tokamak plasma dynamics. Purely fluid models are inherently incapable of capturing these effects, whereas the high dimensionality in purely kinetic models render them practically intractable for most relevant purposes.

        We consider a $\delta\!f$ decomposition model, with a macroscopic fluid background and microscopic kinetic correction, both fully coupled to each other. A similar manner of discretization is proposed to that used in the recent \texttt{STRUPHY} code \cite{Holderied_Possanner_Wang_2021, Holderied_2022, Li_et_al_2023} with a finite-element model for the background and a pseudo-particle/PiC model for the correction.

        The fluid background satisfies the full, non-linear, resistive, compressible, Hall MHD equations. \cite{Laakmann_Hu_Farrell_2022} introduces finite-element(-in-space) implicit timesteppers for the incompressible analogue to this system with structure-preserving (SP) properties in the ideal case, alongside parameter-robust preconditioners. We show that these timesteppers can derive from a finite-element-in-time (FET) (and finite-element-in-space) interpretation. The benefits of this reformulation are discussed, including the derivation of timesteppers that are higher order in time, and the quantifiable dissipative SP properties in the non-ideal, resistive case.
        
        We discuss possible options for extending this FET approach to timesteppers for the compressible case.

        The kinetic corrections satisfy linearized Boltzmann equations. Using a Lénard--Bernstein collision operator, these take Fokker--Planck-like forms \cite{Fokker_1914, Planck_1917} wherein pseudo-particles in the numerical model obey the neoclassical transport equations, with particle-independent Brownian drift terms. This offers a rigorous methodology for incorporating collisions into the particle transport model, without coupling the equations of motions for each particle.
        
        Works by Chen, Chacón et al. \cite{Chen_Chacón_Barnes_2011, Chacón_Chen_Barnes_2013, Chen_Chacón_2014, Chen_Chacón_2015} have developed structure-preserving particle pushers for neoclassical transport in the Vlasov equations, derived from Crank--Nicolson integrators. We show these too can can derive from a FET interpretation, similarly offering potential extensions to higher-order-in-time particle pushers. The FET formulation is used also to consider how the stochastic drift terms can be incorporated into the pushers. Stochastic gyrokinetic expansions are also discussed.

        Different options for the numerical implementation of these schemes are considered.

        Due to the efficacy of FET in the development of SP timesteppers for both the fluid and kinetic component, we hope this approach will prove effective in the future for developing SP timesteppers for the full hybrid model. We hope this will give us the opportunity to incorporate previously inaccessible kinetic effects into the highly effective, modern, finite-element MHD models.
    \end{abstract}
    
    
    \newpage
    \tableofcontents
    
    
    \newpage
    \pagenumbering{arabic}
    %\linenumbers\renewcommand\thelinenumber{\color{black!50}\arabic{linenumber}}
            \input{0 - introduction/main.tex}
        \part{Research}
            \input{1 - low-noise PiC models/main.tex}
            \input{2 - kinetic component/main.tex}
            \input{3 - fluid component/main.tex}
            \input{4 - numerical implementation/main.tex}
        \part{Project Overview}
            \input{5 - research plan/main.tex}
            \input{6 - summary/main.tex}
    
    
    %\section{}
    \newpage
    \pagenumbering{gobble}
        \printbibliography


    \newpage
    \pagenumbering{roman}
    \appendix
        \part{Appendices}
            \input{8 - Hilbert complexes/main.tex}
            \input{9 - weak conservation proofs/main.tex}
\end{document}

    
    
    %\section{}
    \newpage
    \pagenumbering{gobble}
        \printbibliography


    \newpage
    \pagenumbering{roman}
    \appendix
        \part{Appendices}
            \documentclass[12pt, a4paper]{report}

\input{template/main.tex}

\title{\BA{Title in Progress...}}
\author{Boris Andrews}
\affil{Mathematical Institute, University of Oxford}
\date{\today}


\begin{document}
    \pagenumbering{gobble}
    \maketitle
    
    
    \begin{abstract}
        Magnetic confinement reactors---in particular tokamaks---offer one of the most promising options for achieving practical nuclear fusion, with the potential to provide virtually limitless, clean energy. The theoretical and numerical modeling of tokamak plasmas is simultaneously an essential component of effective reactor design, and a great research barrier. Tokamak operational conditions exhibit comparatively low Knudsen numbers. Kinetic effects, including kinetic waves and instabilities, Landau damping, bump-on-tail instabilities and more, are therefore highly influential in tokamak plasma dynamics. Purely fluid models are inherently incapable of capturing these effects, whereas the high dimensionality in purely kinetic models render them practically intractable for most relevant purposes.

        We consider a $\delta\!f$ decomposition model, with a macroscopic fluid background and microscopic kinetic correction, both fully coupled to each other. A similar manner of discretization is proposed to that used in the recent \texttt{STRUPHY} code \cite{Holderied_Possanner_Wang_2021, Holderied_2022, Li_et_al_2023} with a finite-element model for the background and a pseudo-particle/PiC model for the correction.

        The fluid background satisfies the full, non-linear, resistive, compressible, Hall MHD equations. \cite{Laakmann_Hu_Farrell_2022} introduces finite-element(-in-space) implicit timesteppers for the incompressible analogue to this system with structure-preserving (SP) properties in the ideal case, alongside parameter-robust preconditioners. We show that these timesteppers can derive from a finite-element-in-time (FET) (and finite-element-in-space) interpretation. The benefits of this reformulation are discussed, including the derivation of timesteppers that are higher order in time, and the quantifiable dissipative SP properties in the non-ideal, resistive case.
        
        We discuss possible options for extending this FET approach to timesteppers for the compressible case.

        The kinetic corrections satisfy linearized Boltzmann equations. Using a Lénard--Bernstein collision operator, these take Fokker--Planck-like forms \cite{Fokker_1914, Planck_1917} wherein pseudo-particles in the numerical model obey the neoclassical transport equations, with particle-independent Brownian drift terms. This offers a rigorous methodology for incorporating collisions into the particle transport model, without coupling the equations of motions for each particle.
        
        Works by Chen, Chacón et al. \cite{Chen_Chacón_Barnes_2011, Chacón_Chen_Barnes_2013, Chen_Chacón_2014, Chen_Chacón_2015} have developed structure-preserving particle pushers for neoclassical transport in the Vlasov equations, derived from Crank--Nicolson integrators. We show these too can can derive from a FET interpretation, similarly offering potential extensions to higher-order-in-time particle pushers. The FET formulation is used also to consider how the stochastic drift terms can be incorporated into the pushers. Stochastic gyrokinetic expansions are also discussed.

        Different options for the numerical implementation of these schemes are considered.

        Due to the efficacy of FET in the development of SP timesteppers for both the fluid and kinetic component, we hope this approach will prove effective in the future for developing SP timesteppers for the full hybrid model. We hope this will give us the opportunity to incorporate previously inaccessible kinetic effects into the highly effective, modern, finite-element MHD models.
    \end{abstract}
    
    
    \newpage
    \tableofcontents
    
    
    \newpage
    \pagenumbering{arabic}
    %\linenumbers\renewcommand\thelinenumber{\color{black!50}\arabic{linenumber}}
            \input{0 - introduction/main.tex}
        \part{Research}
            \input{1 - low-noise PiC models/main.tex}
            \input{2 - kinetic component/main.tex}
            \input{3 - fluid component/main.tex}
            \input{4 - numerical implementation/main.tex}
        \part{Project Overview}
            \input{5 - research plan/main.tex}
            \input{6 - summary/main.tex}
    
    
    %\section{}
    \newpage
    \pagenumbering{gobble}
        \printbibliography


    \newpage
    \pagenumbering{roman}
    \appendix
        \part{Appendices}
            \input{8 - Hilbert complexes/main.tex}
            \input{9 - weak conservation proofs/main.tex}
\end{document}

            \documentclass[12pt, a4paper]{report}

\input{template/main.tex}

\title{\BA{Title in Progress...}}
\author{Boris Andrews}
\affil{Mathematical Institute, University of Oxford}
\date{\today}


\begin{document}
    \pagenumbering{gobble}
    \maketitle
    
    
    \begin{abstract}
        Magnetic confinement reactors---in particular tokamaks---offer one of the most promising options for achieving practical nuclear fusion, with the potential to provide virtually limitless, clean energy. The theoretical and numerical modeling of tokamak plasmas is simultaneously an essential component of effective reactor design, and a great research barrier. Tokamak operational conditions exhibit comparatively low Knudsen numbers. Kinetic effects, including kinetic waves and instabilities, Landau damping, bump-on-tail instabilities and more, are therefore highly influential in tokamak plasma dynamics. Purely fluid models are inherently incapable of capturing these effects, whereas the high dimensionality in purely kinetic models render them practically intractable for most relevant purposes.

        We consider a $\delta\!f$ decomposition model, with a macroscopic fluid background and microscopic kinetic correction, both fully coupled to each other. A similar manner of discretization is proposed to that used in the recent \texttt{STRUPHY} code \cite{Holderied_Possanner_Wang_2021, Holderied_2022, Li_et_al_2023} with a finite-element model for the background and a pseudo-particle/PiC model for the correction.

        The fluid background satisfies the full, non-linear, resistive, compressible, Hall MHD equations. \cite{Laakmann_Hu_Farrell_2022} introduces finite-element(-in-space) implicit timesteppers for the incompressible analogue to this system with structure-preserving (SP) properties in the ideal case, alongside parameter-robust preconditioners. We show that these timesteppers can derive from a finite-element-in-time (FET) (and finite-element-in-space) interpretation. The benefits of this reformulation are discussed, including the derivation of timesteppers that are higher order in time, and the quantifiable dissipative SP properties in the non-ideal, resistive case.
        
        We discuss possible options for extending this FET approach to timesteppers for the compressible case.

        The kinetic corrections satisfy linearized Boltzmann equations. Using a Lénard--Bernstein collision operator, these take Fokker--Planck-like forms \cite{Fokker_1914, Planck_1917} wherein pseudo-particles in the numerical model obey the neoclassical transport equations, with particle-independent Brownian drift terms. This offers a rigorous methodology for incorporating collisions into the particle transport model, without coupling the equations of motions for each particle.
        
        Works by Chen, Chacón et al. \cite{Chen_Chacón_Barnes_2011, Chacón_Chen_Barnes_2013, Chen_Chacón_2014, Chen_Chacón_2015} have developed structure-preserving particle pushers for neoclassical transport in the Vlasov equations, derived from Crank--Nicolson integrators. We show these too can can derive from a FET interpretation, similarly offering potential extensions to higher-order-in-time particle pushers. The FET formulation is used also to consider how the stochastic drift terms can be incorporated into the pushers. Stochastic gyrokinetic expansions are also discussed.

        Different options for the numerical implementation of these schemes are considered.

        Due to the efficacy of FET in the development of SP timesteppers for both the fluid and kinetic component, we hope this approach will prove effective in the future for developing SP timesteppers for the full hybrid model. We hope this will give us the opportunity to incorporate previously inaccessible kinetic effects into the highly effective, modern, finite-element MHD models.
    \end{abstract}
    
    
    \newpage
    \tableofcontents
    
    
    \newpage
    \pagenumbering{arabic}
    %\linenumbers\renewcommand\thelinenumber{\color{black!50}\arabic{linenumber}}
            \input{0 - introduction/main.tex}
        \part{Research}
            \input{1 - low-noise PiC models/main.tex}
            \input{2 - kinetic component/main.tex}
            \input{3 - fluid component/main.tex}
            \input{4 - numerical implementation/main.tex}
        \part{Project Overview}
            \input{5 - research plan/main.tex}
            \input{6 - summary/main.tex}
    
    
    %\section{}
    \newpage
    \pagenumbering{gobble}
        \printbibliography


    \newpage
    \pagenumbering{roman}
    \appendix
        \part{Appendices}
            \input{8 - Hilbert complexes/main.tex}
            \input{9 - weak conservation proofs/main.tex}
\end{document}

\end{document}

        \part{Research}
            \documentclass[12pt, a4paper]{report}

\documentclass[12pt, a4paper]{report}

\input{template/main.tex}

\title{\BA{Title in Progress...}}
\author{Boris Andrews}
\affil{Mathematical Institute, University of Oxford}
\date{\today}


\begin{document}
    \pagenumbering{gobble}
    \maketitle
    
    
    \begin{abstract}
        Magnetic confinement reactors---in particular tokamaks---offer one of the most promising options for achieving practical nuclear fusion, with the potential to provide virtually limitless, clean energy. The theoretical and numerical modeling of tokamak plasmas is simultaneously an essential component of effective reactor design, and a great research barrier. Tokamak operational conditions exhibit comparatively low Knudsen numbers. Kinetic effects, including kinetic waves and instabilities, Landau damping, bump-on-tail instabilities and more, are therefore highly influential in tokamak plasma dynamics. Purely fluid models are inherently incapable of capturing these effects, whereas the high dimensionality in purely kinetic models render them practically intractable for most relevant purposes.

        We consider a $\delta\!f$ decomposition model, with a macroscopic fluid background and microscopic kinetic correction, both fully coupled to each other. A similar manner of discretization is proposed to that used in the recent \texttt{STRUPHY} code \cite{Holderied_Possanner_Wang_2021, Holderied_2022, Li_et_al_2023} with a finite-element model for the background and a pseudo-particle/PiC model for the correction.

        The fluid background satisfies the full, non-linear, resistive, compressible, Hall MHD equations. \cite{Laakmann_Hu_Farrell_2022} introduces finite-element(-in-space) implicit timesteppers for the incompressible analogue to this system with structure-preserving (SP) properties in the ideal case, alongside parameter-robust preconditioners. We show that these timesteppers can derive from a finite-element-in-time (FET) (and finite-element-in-space) interpretation. The benefits of this reformulation are discussed, including the derivation of timesteppers that are higher order in time, and the quantifiable dissipative SP properties in the non-ideal, resistive case.
        
        We discuss possible options for extending this FET approach to timesteppers for the compressible case.

        The kinetic corrections satisfy linearized Boltzmann equations. Using a Lénard--Bernstein collision operator, these take Fokker--Planck-like forms \cite{Fokker_1914, Planck_1917} wherein pseudo-particles in the numerical model obey the neoclassical transport equations, with particle-independent Brownian drift terms. This offers a rigorous methodology for incorporating collisions into the particle transport model, without coupling the equations of motions for each particle.
        
        Works by Chen, Chacón et al. \cite{Chen_Chacón_Barnes_2011, Chacón_Chen_Barnes_2013, Chen_Chacón_2014, Chen_Chacón_2015} have developed structure-preserving particle pushers for neoclassical transport in the Vlasov equations, derived from Crank--Nicolson integrators. We show these too can can derive from a FET interpretation, similarly offering potential extensions to higher-order-in-time particle pushers. The FET formulation is used also to consider how the stochastic drift terms can be incorporated into the pushers. Stochastic gyrokinetic expansions are also discussed.

        Different options for the numerical implementation of these schemes are considered.

        Due to the efficacy of FET in the development of SP timesteppers for both the fluid and kinetic component, we hope this approach will prove effective in the future for developing SP timesteppers for the full hybrid model. We hope this will give us the opportunity to incorporate previously inaccessible kinetic effects into the highly effective, modern, finite-element MHD models.
    \end{abstract}
    
    
    \newpage
    \tableofcontents
    
    
    \newpage
    \pagenumbering{arabic}
    %\linenumbers\renewcommand\thelinenumber{\color{black!50}\arabic{linenumber}}
            \input{0 - introduction/main.tex}
        \part{Research}
            \input{1 - low-noise PiC models/main.tex}
            \input{2 - kinetic component/main.tex}
            \input{3 - fluid component/main.tex}
            \input{4 - numerical implementation/main.tex}
        \part{Project Overview}
            \input{5 - research plan/main.tex}
            \input{6 - summary/main.tex}
    
    
    %\section{}
    \newpage
    \pagenumbering{gobble}
        \printbibliography


    \newpage
    \pagenumbering{roman}
    \appendix
        \part{Appendices}
            \input{8 - Hilbert complexes/main.tex}
            \input{9 - weak conservation proofs/main.tex}
\end{document}


\title{\BA{Title in Progress...}}
\author{Boris Andrews}
\affil{Mathematical Institute, University of Oxford}
\date{\today}


\begin{document}
    \pagenumbering{gobble}
    \maketitle
    
    
    \begin{abstract}
        Magnetic confinement reactors---in particular tokamaks---offer one of the most promising options for achieving practical nuclear fusion, with the potential to provide virtually limitless, clean energy. The theoretical and numerical modeling of tokamak plasmas is simultaneously an essential component of effective reactor design, and a great research barrier. Tokamak operational conditions exhibit comparatively low Knudsen numbers. Kinetic effects, including kinetic waves and instabilities, Landau damping, bump-on-tail instabilities and more, are therefore highly influential in tokamak plasma dynamics. Purely fluid models are inherently incapable of capturing these effects, whereas the high dimensionality in purely kinetic models render them practically intractable for most relevant purposes.

        We consider a $\delta\!f$ decomposition model, with a macroscopic fluid background and microscopic kinetic correction, both fully coupled to each other. A similar manner of discretization is proposed to that used in the recent \texttt{STRUPHY} code \cite{Holderied_Possanner_Wang_2021, Holderied_2022, Li_et_al_2023} with a finite-element model for the background and a pseudo-particle/PiC model for the correction.

        The fluid background satisfies the full, non-linear, resistive, compressible, Hall MHD equations. \cite{Laakmann_Hu_Farrell_2022} introduces finite-element(-in-space) implicit timesteppers for the incompressible analogue to this system with structure-preserving (SP) properties in the ideal case, alongside parameter-robust preconditioners. We show that these timesteppers can derive from a finite-element-in-time (FET) (and finite-element-in-space) interpretation. The benefits of this reformulation are discussed, including the derivation of timesteppers that are higher order in time, and the quantifiable dissipative SP properties in the non-ideal, resistive case.
        
        We discuss possible options for extending this FET approach to timesteppers for the compressible case.

        The kinetic corrections satisfy linearized Boltzmann equations. Using a Lénard--Bernstein collision operator, these take Fokker--Planck-like forms \cite{Fokker_1914, Planck_1917} wherein pseudo-particles in the numerical model obey the neoclassical transport equations, with particle-independent Brownian drift terms. This offers a rigorous methodology for incorporating collisions into the particle transport model, without coupling the equations of motions for each particle.
        
        Works by Chen, Chacón et al. \cite{Chen_Chacón_Barnes_2011, Chacón_Chen_Barnes_2013, Chen_Chacón_2014, Chen_Chacón_2015} have developed structure-preserving particle pushers for neoclassical transport in the Vlasov equations, derived from Crank--Nicolson integrators. We show these too can can derive from a FET interpretation, similarly offering potential extensions to higher-order-in-time particle pushers. The FET formulation is used also to consider how the stochastic drift terms can be incorporated into the pushers. Stochastic gyrokinetic expansions are also discussed.

        Different options for the numerical implementation of these schemes are considered.

        Due to the efficacy of FET in the development of SP timesteppers for both the fluid and kinetic component, we hope this approach will prove effective in the future for developing SP timesteppers for the full hybrid model. We hope this will give us the opportunity to incorporate previously inaccessible kinetic effects into the highly effective, modern, finite-element MHD models.
    \end{abstract}
    
    
    \newpage
    \tableofcontents
    
    
    \newpage
    \pagenumbering{arabic}
    %\linenumbers\renewcommand\thelinenumber{\color{black!50}\arabic{linenumber}}
            \documentclass[12pt, a4paper]{report}

\input{template/main.tex}

\title{\BA{Title in Progress...}}
\author{Boris Andrews}
\affil{Mathematical Institute, University of Oxford}
\date{\today}


\begin{document}
    \pagenumbering{gobble}
    \maketitle
    
    
    \begin{abstract}
        Magnetic confinement reactors---in particular tokamaks---offer one of the most promising options for achieving practical nuclear fusion, with the potential to provide virtually limitless, clean energy. The theoretical and numerical modeling of tokamak plasmas is simultaneously an essential component of effective reactor design, and a great research barrier. Tokamak operational conditions exhibit comparatively low Knudsen numbers. Kinetic effects, including kinetic waves and instabilities, Landau damping, bump-on-tail instabilities and more, are therefore highly influential in tokamak plasma dynamics. Purely fluid models are inherently incapable of capturing these effects, whereas the high dimensionality in purely kinetic models render them practically intractable for most relevant purposes.

        We consider a $\delta\!f$ decomposition model, with a macroscopic fluid background and microscopic kinetic correction, both fully coupled to each other. A similar manner of discretization is proposed to that used in the recent \texttt{STRUPHY} code \cite{Holderied_Possanner_Wang_2021, Holderied_2022, Li_et_al_2023} with a finite-element model for the background and a pseudo-particle/PiC model for the correction.

        The fluid background satisfies the full, non-linear, resistive, compressible, Hall MHD equations. \cite{Laakmann_Hu_Farrell_2022} introduces finite-element(-in-space) implicit timesteppers for the incompressible analogue to this system with structure-preserving (SP) properties in the ideal case, alongside parameter-robust preconditioners. We show that these timesteppers can derive from a finite-element-in-time (FET) (and finite-element-in-space) interpretation. The benefits of this reformulation are discussed, including the derivation of timesteppers that are higher order in time, and the quantifiable dissipative SP properties in the non-ideal, resistive case.
        
        We discuss possible options for extending this FET approach to timesteppers for the compressible case.

        The kinetic corrections satisfy linearized Boltzmann equations. Using a Lénard--Bernstein collision operator, these take Fokker--Planck-like forms \cite{Fokker_1914, Planck_1917} wherein pseudo-particles in the numerical model obey the neoclassical transport equations, with particle-independent Brownian drift terms. This offers a rigorous methodology for incorporating collisions into the particle transport model, without coupling the equations of motions for each particle.
        
        Works by Chen, Chacón et al. \cite{Chen_Chacón_Barnes_2011, Chacón_Chen_Barnes_2013, Chen_Chacón_2014, Chen_Chacón_2015} have developed structure-preserving particle pushers for neoclassical transport in the Vlasov equations, derived from Crank--Nicolson integrators. We show these too can can derive from a FET interpretation, similarly offering potential extensions to higher-order-in-time particle pushers. The FET formulation is used also to consider how the stochastic drift terms can be incorporated into the pushers. Stochastic gyrokinetic expansions are also discussed.

        Different options for the numerical implementation of these schemes are considered.

        Due to the efficacy of FET in the development of SP timesteppers for both the fluid and kinetic component, we hope this approach will prove effective in the future for developing SP timesteppers for the full hybrid model. We hope this will give us the opportunity to incorporate previously inaccessible kinetic effects into the highly effective, modern, finite-element MHD models.
    \end{abstract}
    
    
    \newpage
    \tableofcontents
    
    
    \newpage
    \pagenumbering{arabic}
    %\linenumbers\renewcommand\thelinenumber{\color{black!50}\arabic{linenumber}}
            \input{0 - introduction/main.tex}
        \part{Research}
            \input{1 - low-noise PiC models/main.tex}
            \input{2 - kinetic component/main.tex}
            \input{3 - fluid component/main.tex}
            \input{4 - numerical implementation/main.tex}
        \part{Project Overview}
            \input{5 - research plan/main.tex}
            \input{6 - summary/main.tex}
    
    
    %\section{}
    \newpage
    \pagenumbering{gobble}
        \printbibliography


    \newpage
    \pagenumbering{roman}
    \appendix
        \part{Appendices}
            \input{8 - Hilbert complexes/main.tex}
            \input{9 - weak conservation proofs/main.tex}
\end{document}

        \part{Research}
            \documentclass[12pt, a4paper]{report}

\input{template/main.tex}

\title{\BA{Title in Progress...}}
\author{Boris Andrews}
\affil{Mathematical Institute, University of Oxford}
\date{\today}


\begin{document}
    \pagenumbering{gobble}
    \maketitle
    
    
    \begin{abstract}
        Magnetic confinement reactors---in particular tokamaks---offer one of the most promising options for achieving practical nuclear fusion, with the potential to provide virtually limitless, clean energy. The theoretical and numerical modeling of tokamak plasmas is simultaneously an essential component of effective reactor design, and a great research barrier. Tokamak operational conditions exhibit comparatively low Knudsen numbers. Kinetic effects, including kinetic waves and instabilities, Landau damping, bump-on-tail instabilities and more, are therefore highly influential in tokamak plasma dynamics. Purely fluid models are inherently incapable of capturing these effects, whereas the high dimensionality in purely kinetic models render them practically intractable for most relevant purposes.

        We consider a $\delta\!f$ decomposition model, with a macroscopic fluid background and microscopic kinetic correction, both fully coupled to each other. A similar manner of discretization is proposed to that used in the recent \texttt{STRUPHY} code \cite{Holderied_Possanner_Wang_2021, Holderied_2022, Li_et_al_2023} with a finite-element model for the background and a pseudo-particle/PiC model for the correction.

        The fluid background satisfies the full, non-linear, resistive, compressible, Hall MHD equations. \cite{Laakmann_Hu_Farrell_2022} introduces finite-element(-in-space) implicit timesteppers for the incompressible analogue to this system with structure-preserving (SP) properties in the ideal case, alongside parameter-robust preconditioners. We show that these timesteppers can derive from a finite-element-in-time (FET) (and finite-element-in-space) interpretation. The benefits of this reformulation are discussed, including the derivation of timesteppers that are higher order in time, and the quantifiable dissipative SP properties in the non-ideal, resistive case.
        
        We discuss possible options for extending this FET approach to timesteppers for the compressible case.

        The kinetic corrections satisfy linearized Boltzmann equations. Using a Lénard--Bernstein collision operator, these take Fokker--Planck-like forms \cite{Fokker_1914, Planck_1917} wherein pseudo-particles in the numerical model obey the neoclassical transport equations, with particle-independent Brownian drift terms. This offers a rigorous methodology for incorporating collisions into the particle transport model, without coupling the equations of motions for each particle.
        
        Works by Chen, Chacón et al. \cite{Chen_Chacón_Barnes_2011, Chacón_Chen_Barnes_2013, Chen_Chacón_2014, Chen_Chacón_2015} have developed structure-preserving particle pushers for neoclassical transport in the Vlasov equations, derived from Crank--Nicolson integrators. We show these too can can derive from a FET interpretation, similarly offering potential extensions to higher-order-in-time particle pushers. The FET formulation is used also to consider how the stochastic drift terms can be incorporated into the pushers. Stochastic gyrokinetic expansions are also discussed.

        Different options for the numerical implementation of these schemes are considered.

        Due to the efficacy of FET in the development of SP timesteppers for both the fluid and kinetic component, we hope this approach will prove effective in the future for developing SP timesteppers for the full hybrid model. We hope this will give us the opportunity to incorporate previously inaccessible kinetic effects into the highly effective, modern, finite-element MHD models.
    \end{abstract}
    
    
    \newpage
    \tableofcontents
    
    
    \newpage
    \pagenumbering{arabic}
    %\linenumbers\renewcommand\thelinenumber{\color{black!50}\arabic{linenumber}}
            \input{0 - introduction/main.tex}
        \part{Research}
            \input{1 - low-noise PiC models/main.tex}
            \input{2 - kinetic component/main.tex}
            \input{3 - fluid component/main.tex}
            \input{4 - numerical implementation/main.tex}
        \part{Project Overview}
            \input{5 - research plan/main.tex}
            \input{6 - summary/main.tex}
    
    
    %\section{}
    \newpage
    \pagenumbering{gobble}
        \printbibliography


    \newpage
    \pagenumbering{roman}
    \appendix
        \part{Appendices}
            \input{8 - Hilbert complexes/main.tex}
            \input{9 - weak conservation proofs/main.tex}
\end{document}

            \documentclass[12pt, a4paper]{report}

\input{template/main.tex}

\title{\BA{Title in Progress...}}
\author{Boris Andrews}
\affil{Mathematical Institute, University of Oxford}
\date{\today}


\begin{document}
    \pagenumbering{gobble}
    \maketitle
    
    
    \begin{abstract}
        Magnetic confinement reactors---in particular tokamaks---offer one of the most promising options for achieving practical nuclear fusion, with the potential to provide virtually limitless, clean energy. The theoretical and numerical modeling of tokamak plasmas is simultaneously an essential component of effective reactor design, and a great research barrier. Tokamak operational conditions exhibit comparatively low Knudsen numbers. Kinetic effects, including kinetic waves and instabilities, Landau damping, bump-on-tail instabilities and more, are therefore highly influential in tokamak plasma dynamics. Purely fluid models are inherently incapable of capturing these effects, whereas the high dimensionality in purely kinetic models render them practically intractable for most relevant purposes.

        We consider a $\delta\!f$ decomposition model, with a macroscopic fluid background and microscopic kinetic correction, both fully coupled to each other. A similar manner of discretization is proposed to that used in the recent \texttt{STRUPHY} code \cite{Holderied_Possanner_Wang_2021, Holderied_2022, Li_et_al_2023} with a finite-element model for the background and a pseudo-particle/PiC model for the correction.

        The fluid background satisfies the full, non-linear, resistive, compressible, Hall MHD equations. \cite{Laakmann_Hu_Farrell_2022} introduces finite-element(-in-space) implicit timesteppers for the incompressible analogue to this system with structure-preserving (SP) properties in the ideal case, alongside parameter-robust preconditioners. We show that these timesteppers can derive from a finite-element-in-time (FET) (and finite-element-in-space) interpretation. The benefits of this reformulation are discussed, including the derivation of timesteppers that are higher order in time, and the quantifiable dissipative SP properties in the non-ideal, resistive case.
        
        We discuss possible options for extending this FET approach to timesteppers for the compressible case.

        The kinetic corrections satisfy linearized Boltzmann equations. Using a Lénard--Bernstein collision operator, these take Fokker--Planck-like forms \cite{Fokker_1914, Planck_1917} wherein pseudo-particles in the numerical model obey the neoclassical transport equations, with particle-independent Brownian drift terms. This offers a rigorous methodology for incorporating collisions into the particle transport model, without coupling the equations of motions for each particle.
        
        Works by Chen, Chacón et al. \cite{Chen_Chacón_Barnes_2011, Chacón_Chen_Barnes_2013, Chen_Chacón_2014, Chen_Chacón_2015} have developed structure-preserving particle pushers for neoclassical transport in the Vlasov equations, derived from Crank--Nicolson integrators. We show these too can can derive from a FET interpretation, similarly offering potential extensions to higher-order-in-time particle pushers. The FET formulation is used also to consider how the stochastic drift terms can be incorporated into the pushers. Stochastic gyrokinetic expansions are also discussed.

        Different options for the numerical implementation of these schemes are considered.

        Due to the efficacy of FET in the development of SP timesteppers for both the fluid and kinetic component, we hope this approach will prove effective in the future for developing SP timesteppers for the full hybrid model. We hope this will give us the opportunity to incorporate previously inaccessible kinetic effects into the highly effective, modern, finite-element MHD models.
    \end{abstract}
    
    
    \newpage
    \tableofcontents
    
    
    \newpage
    \pagenumbering{arabic}
    %\linenumbers\renewcommand\thelinenumber{\color{black!50}\arabic{linenumber}}
            \input{0 - introduction/main.tex}
        \part{Research}
            \input{1 - low-noise PiC models/main.tex}
            \input{2 - kinetic component/main.tex}
            \input{3 - fluid component/main.tex}
            \input{4 - numerical implementation/main.tex}
        \part{Project Overview}
            \input{5 - research plan/main.tex}
            \input{6 - summary/main.tex}
    
    
    %\section{}
    \newpage
    \pagenumbering{gobble}
        \printbibliography


    \newpage
    \pagenumbering{roman}
    \appendix
        \part{Appendices}
            \input{8 - Hilbert complexes/main.tex}
            \input{9 - weak conservation proofs/main.tex}
\end{document}

            \documentclass[12pt, a4paper]{report}

\input{template/main.tex}

\title{\BA{Title in Progress...}}
\author{Boris Andrews}
\affil{Mathematical Institute, University of Oxford}
\date{\today}


\begin{document}
    \pagenumbering{gobble}
    \maketitle
    
    
    \begin{abstract}
        Magnetic confinement reactors---in particular tokamaks---offer one of the most promising options for achieving practical nuclear fusion, with the potential to provide virtually limitless, clean energy. The theoretical and numerical modeling of tokamak plasmas is simultaneously an essential component of effective reactor design, and a great research barrier. Tokamak operational conditions exhibit comparatively low Knudsen numbers. Kinetic effects, including kinetic waves and instabilities, Landau damping, bump-on-tail instabilities and more, are therefore highly influential in tokamak plasma dynamics. Purely fluid models are inherently incapable of capturing these effects, whereas the high dimensionality in purely kinetic models render them practically intractable for most relevant purposes.

        We consider a $\delta\!f$ decomposition model, with a macroscopic fluid background and microscopic kinetic correction, both fully coupled to each other. A similar manner of discretization is proposed to that used in the recent \texttt{STRUPHY} code \cite{Holderied_Possanner_Wang_2021, Holderied_2022, Li_et_al_2023} with a finite-element model for the background and a pseudo-particle/PiC model for the correction.

        The fluid background satisfies the full, non-linear, resistive, compressible, Hall MHD equations. \cite{Laakmann_Hu_Farrell_2022} introduces finite-element(-in-space) implicit timesteppers for the incompressible analogue to this system with structure-preserving (SP) properties in the ideal case, alongside parameter-robust preconditioners. We show that these timesteppers can derive from a finite-element-in-time (FET) (and finite-element-in-space) interpretation. The benefits of this reformulation are discussed, including the derivation of timesteppers that are higher order in time, and the quantifiable dissipative SP properties in the non-ideal, resistive case.
        
        We discuss possible options for extending this FET approach to timesteppers for the compressible case.

        The kinetic corrections satisfy linearized Boltzmann equations. Using a Lénard--Bernstein collision operator, these take Fokker--Planck-like forms \cite{Fokker_1914, Planck_1917} wherein pseudo-particles in the numerical model obey the neoclassical transport equations, with particle-independent Brownian drift terms. This offers a rigorous methodology for incorporating collisions into the particle transport model, without coupling the equations of motions for each particle.
        
        Works by Chen, Chacón et al. \cite{Chen_Chacón_Barnes_2011, Chacón_Chen_Barnes_2013, Chen_Chacón_2014, Chen_Chacón_2015} have developed structure-preserving particle pushers for neoclassical transport in the Vlasov equations, derived from Crank--Nicolson integrators. We show these too can can derive from a FET interpretation, similarly offering potential extensions to higher-order-in-time particle pushers. The FET formulation is used also to consider how the stochastic drift terms can be incorporated into the pushers. Stochastic gyrokinetic expansions are also discussed.

        Different options for the numerical implementation of these schemes are considered.

        Due to the efficacy of FET in the development of SP timesteppers for both the fluid and kinetic component, we hope this approach will prove effective in the future for developing SP timesteppers for the full hybrid model. We hope this will give us the opportunity to incorporate previously inaccessible kinetic effects into the highly effective, modern, finite-element MHD models.
    \end{abstract}
    
    
    \newpage
    \tableofcontents
    
    
    \newpage
    \pagenumbering{arabic}
    %\linenumbers\renewcommand\thelinenumber{\color{black!50}\arabic{linenumber}}
            \input{0 - introduction/main.tex}
        \part{Research}
            \input{1 - low-noise PiC models/main.tex}
            \input{2 - kinetic component/main.tex}
            \input{3 - fluid component/main.tex}
            \input{4 - numerical implementation/main.tex}
        \part{Project Overview}
            \input{5 - research plan/main.tex}
            \input{6 - summary/main.tex}
    
    
    %\section{}
    \newpage
    \pagenumbering{gobble}
        \printbibliography


    \newpage
    \pagenumbering{roman}
    \appendix
        \part{Appendices}
            \input{8 - Hilbert complexes/main.tex}
            \input{9 - weak conservation proofs/main.tex}
\end{document}

            \documentclass[12pt, a4paper]{report}

\input{template/main.tex}

\title{\BA{Title in Progress...}}
\author{Boris Andrews}
\affil{Mathematical Institute, University of Oxford}
\date{\today}


\begin{document}
    \pagenumbering{gobble}
    \maketitle
    
    
    \begin{abstract}
        Magnetic confinement reactors---in particular tokamaks---offer one of the most promising options for achieving practical nuclear fusion, with the potential to provide virtually limitless, clean energy. The theoretical and numerical modeling of tokamak plasmas is simultaneously an essential component of effective reactor design, and a great research barrier. Tokamak operational conditions exhibit comparatively low Knudsen numbers. Kinetic effects, including kinetic waves and instabilities, Landau damping, bump-on-tail instabilities and more, are therefore highly influential in tokamak plasma dynamics. Purely fluid models are inherently incapable of capturing these effects, whereas the high dimensionality in purely kinetic models render them practically intractable for most relevant purposes.

        We consider a $\delta\!f$ decomposition model, with a macroscopic fluid background and microscopic kinetic correction, both fully coupled to each other. A similar manner of discretization is proposed to that used in the recent \texttt{STRUPHY} code \cite{Holderied_Possanner_Wang_2021, Holderied_2022, Li_et_al_2023} with a finite-element model for the background and a pseudo-particle/PiC model for the correction.

        The fluid background satisfies the full, non-linear, resistive, compressible, Hall MHD equations. \cite{Laakmann_Hu_Farrell_2022} introduces finite-element(-in-space) implicit timesteppers for the incompressible analogue to this system with structure-preserving (SP) properties in the ideal case, alongside parameter-robust preconditioners. We show that these timesteppers can derive from a finite-element-in-time (FET) (and finite-element-in-space) interpretation. The benefits of this reformulation are discussed, including the derivation of timesteppers that are higher order in time, and the quantifiable dissipative SP properties in the non-ideal, resistive case.
        
        We discuss possible options for extending this FET approach to timesteppers for the compressible case.

        The kinetic corrections satisfy linearized Boltzmann equations. Using a Lénard--Bernstein collision operator, these take Fokker--Planck-like forms \cite{Fokker_1914, Planck_1917} wherein pseudo-particles in the numerical model obey the neoclassical transport equations, with particle-independent Brownian drift terms. This offers a rigorous methodology for incorporating collisions into the particle transport model, without coupling the equations of motions for each particle.
        
        Works by Chen, Chacón et al. \cite{Chen_Chacón_Barnes_2011, Chacón_Chen_Barnes_2013, Chen_Chacón_2014, Chen_Chacón_2015} have developed structure-preserving particle pushers for neoclassical transport in the Vlasov equations, derived from Crank--Nicolson integrators. We show these too can can derive from a FET interpretation, similarly offering potential extensions to higher-order-in-time particle pushers. The FET formulation is used also to consider how the stochastic drift terms can be incorporated into the pushers. Stochastic gyrokinetic expansions are also discussed.

        Different options for the numerical implementation of these schemes are considered.

        Due to the efficacy of FET in the development of SP timesteppers for both the fluid and kinetic component, we hope this approach will prove effective in the future for developing SP timesteppers for the full hybrid model. We hope this will give us the opportunity to incorporate previously inaccessible kinetic effects into the highly effective, modern, finite-element MHD models.
    \end{abstract}
    
    
    \newpage
    \tableofcontents
    
    
    \newpage
    \pagenumbering{arabic}
    %\linenumbers\renewcommand\thelinenumber{\color{black!50}\arabic{linenumber}}
            \input{0 - introduction/main.tex}
        \part{Research}
            \input{1 - low-noise PiC models/main.tex}
            \input{2 - kinetic component/main.tex}
            \input{3 - fluid component/main.tex}
            \input{4 - numerical implementation/main.tex}
        \part{Project Overview}
            \input{5 - research plan/main.tex}
            \input{6 - summary/main.tex}
    
    
    %\section{}
    \newpage
    \pagenumbering{gobble}
        \printbibliography


    \newpage
    \pagenumbering{roman}
    \appendix
        \part{Appendices}
            \input{8 - Hilbert complexes/main.tex}
            \input{9 - weak conservation proofs/main.tex}
\end{document}

        \part{Project Overview}
            \documentclass[12pt, a4paper]{report}

\input{template/main.tex}

\title{\BA{Title in Progress...}}
\author{Boris Andrews}
\affil{Mathematical Institute, University of Oxford}
\date{\today}


\begin{document}
    \pagenumbering{gobble}
    \maketitle
    
    
    \begin{abstract}
        Magnetic confinement reactors---in particular tokamaks---offer one of the most promising options for achieving practical nuclear fusion, with the potential to provide virtually limitless, clean energy. The theoretical and numerical modeling of tokamak plasmas is simultaneously an essential component of effective reactor design, and a great research barrier. Tokamak operational conditions exhibit comparatively low Knudsen numbers. Kinetic effects, including kinetic waves and instabilities, Landau damping, bump-on-tail instabilities and more, are therefore highly influential in tokamak plasma dynamics. Purely fluid models are inherently incapable of capturing these effects, whereas the high dimensionality in purely kinetic models render them practically intractable for most relevant purposes.

        We consider a $\delta\!f$ decomposition model, with a macroscopic fluid background and microscopic kinetic correction, both fully coupled to each other. A similar manner of discretization is proposed to that used in the recent \texttt{STRUPHY} code \cite{Holderied_Possanner_Wang_2021, Holderied_2022, Li_et_al_2023} with a finite-element model for the background and a pseudo-particle/PiC model for the correction.

        The fluid background satisfies the full, non-linear, resistive, compressible, Hall MHD equations. \cite{Laakmann_Hu_Farrell_2022} introduces finite-element(-in-space) implicit timesteppers for the incompressible analogue to this system with structure-preserving (SP) properties in the ideal case, alongside parameter-robust preconditioners. We show that these timesteppers can derive from a finite-element-in-time (FET) (and finite-element-in-space) interpretation. The benefits of this reformulation are discussed, including the derivation of timesteppers that are higher order in time, and the quantifiable dissipative SP properties in the non-ideal, resistive case.
        
        We discuss possible options for extending this FET approach to timesteppers for the compressible case.

        The kinetic corrections satisfy linearized Boltzmann equations. Using a Lénard--Bernstein collision operator, these take Fokker--Planck-like forms \cite{Fokker_1914, Planck_1917} wherein pseudo-particles in the numerical model obey the neoclassical transport equations, with particle-independent Brownian drift terms. This offers a rigorous methodology for incorporating collisions into the particle transport model, without coupling the equations of motions for each particle.
        
        Works by Chen, Chacón et al. \cite{Chen_Chacón_Barnes_2011, Chacón_Chen_Barnes_2013, Chen_Chacón_2014, Chen_Chacón_2015} have developed structure-preserving particle pushers for neoclassical transport in the Vlasov equations, derived from Crank--Nicolson integrators. We show these too can can derive from a FET interpretation, similarly offering potential extensions to higher-order-in-time particle pushers. The FET formulation is used also to consider how the stochastic drift terms can be incorporated into the pushers. Stochastic gyrokinetic expansions are also discussed.

        Different options for the numerical implementation of these schemes are considered.

        Due to the efficacy of FET in the development of SP timesteppers for both the fluid and kinetic component, we hope this approach will prove effective in the future for developing SP timesteppers for the full hybrid model. We hope this will give us the opportunity to incorporate previously inaccessible kinetic effects into the highly effective, modern, finite-element MHD models.
    \end{abstract}
    
    
    \newpage
    \tableofcontents
    
    
    \newpage
    \pagenumbering{arabic}
    %\linenumbers\renewcommand\thelinenumber{\color{black!50}\arabic{linenumber}}
            \input{0 - introduction/main.tex}
        \part{Research}
            \input{1 - low-noise PiC models/main.tex}
            \input{2 - kinetic component/main.tex}
            \input{3 - fluid component/main.tex}
            \input{4 - numerical implementation/main.tex}
        \part{Project Overview}
            \input{5 - research plan/main.tex}
            \input{6 - summary/main.tex}
    
    
    %\section{}
    \newpage
    \pagenumbering{gobble}
        \printbibliography


    \newpage
    \pagenumbering{roman}
    \appendix
        \part{Appendices}
            \input{8 - Hilbert complexes/main.tex}
            \input{9 - weak conservation proofs/main.tex}
\end{document}

            \documentclass[12pt, a4paper]{report}

\input{template/main.tex}

\title{\BA{Title in Progress...}}
\author{Boris Andrews}
\affil{Mathematical Institute, University of Oxford}
\date{\today}


\begin{document}
    \pagenumbering{gobble}
    \maketitle
    
    
    \begin{abstract}
        Magnetic confinement reactors---in particular tokamaks---offer one of the most promising options for achieving practical nuclear fusion, with the potential to provide virtually limitless, clean energy. The theoretical and numerical modeling of tokamak plasmas is simultaneously an essential component of effective reactor design, and a great research barrier. Tokamak operational conditions exhibit comparatively low Knudsen numbers. Kinetic effects, including kinetic waves and instabilities, Landau damping, bump-on-tail instabilities and more, are therefore highly influential in tokamak plasma dynamics. Purely fluid models are inherently incapable of capturing these effects, whereas the high dimensionality in purely kinetic models render them practically intractable for most relevant purposes.

        We consider a $\delta\!f$ decomposition model, with a macroscopic fluid background and microscopic kinetic correction, both fully coupled to each other. A similar manner of discretization is proposed to that used in the recent \texttt{STRUPHY} code \cite{Holderied_Possanner_Wang_2021, Holderied_2022, Li_et_al_2023} with a finite-element model for the background and a pseudo-particle/PiC model for the correction.

        The fluid background satisfies the full, non-linear, resistive, compressible, Hall MHD equations. \cite{Laakmann_Hu_Farrell_2022} introduces finite-element(-in-space) implicit timesteppers for the incompressible analogue to this system with structure-preserving (SP) properties in the ideal case, alongside parameter-robust preconditioners. We show that these timesteppers can derive from a finite-element-in-time (FET) (and finite-element-in-space) interpretation. The benefits of this reformulation are discussed, including the derivation of timesteppers that are higher order in time, and the quantifiable dissipative SP properties in the non-ideal, resistive case.
        
        We discuss possible options for extending this FET approach to timesteppers for the compressible case.

        The kinetic corrections satisfy linearized Boltzmann equations. Using a Lénard--Bernstein collision operator, these take Fokker--Planck-like forms \cite{Fokker_1914, Planck_1917} wherein pseudo-particles in the numerical model obey the neoclassical transport equations, with particle-independent Brownian drift terms. This offers a rigorous methodology for incorporating collisions into the particle transport model, without coupling the equations of motions for each particle.
        
        Works by Chen, Chacón et al. \cite{Chen_Chacón_Barnes_2011, Chacón_Chen_Barnes_2013, Chen_Chacón_2014, Chen_Chacón_2015} have developed structure-preserving particle pushers for neoclassical transport in the Vlasov equations, derived from Crank--Nicolson integrators. We show these too can can derive from a FET interpretation, similarly offering potential extensions to higher-order-in-time particle pushers. The FET formulation is used also to consider how the stochastic drift terms can be incorporated into the pushers. Stochastic gyrokinetic expansions are also discussed.

        Different options for the numerical implementation of these schemes are considered.

        Due to the efficacy of FET in the development of SP timesteppers for both the fluid and kinetic component, we hope this approach will prove effective in the future for developing SP timesteppers for the full hybrid model. We hope this will give us the opportunity to incorporate previously inaccessible kinetic effects into the highly effective, modern, finite-element MHD models.
    \end{abstract}
    
    
    \newpage
    \tableofcontents
    
    
    \newpage
    \pagenumbering{arabic}
    %\linenumbers\renewcommand\thelinenumber{\color{black!50}\arabic{linenumber}}
            \input{0 - introduction/main.tex}
        \part{Research}
            \input{1 - low-noise PiC models/main.tex}
            \input{2 - kinetic component/main.tex}
            \input{3 - fluid component/main.tex}
            \input{4 - numerical implementation/main.tex}
        \part{Project Overview}
            \input{5 - research plan/main.tex}
            \input{6 - summary/main.tex}
    
    
    %\section{}
    \newpage
    \pagenumbering{gobble}
        \printbibliography


    \newpage
    \pagenumbering{roman}
    \appendix
        \part{Appendices}
            \input{8 - Hilbert complexes/main.tex}
            \input{9 - weak conservation proofs/main.tex}
\end{document}

    
    
    %\section{}
    \newpage
    \pagenumbering{gobble}
        \printbibliography


    \newpage
    \pagenumbering{roman}
    \appendix
        \part{Appendices}
            \documentclass[12pt, a4paper]{report}

\input{template/main.tex}

\title{\BA{Title in Progress...}}
\author{Boris Andrews}
\affil{Mathematical Institute, University of Oxford}
\date{\today}


\begin{document}
    \pagenumbering{gobble}
    \maketitle
    
    
    \begin{abstract}
        Magnetic confinement reactors---in particular tokamaks---offer one of the most promising options for achieving practical nuclear fusion, with the potential to provide virtually limitless, clean energy. The theoretical and numerical modeling of tokamak plasmas is simultaneously an essential component of effective reactor design, and a great research barrier. Tokamak operational conditions exhibit comparatively low Knudsen numbers. Kinetic effects, including kinetic waves and instabilities, Landau damping, bump-on-tail instabilities and more, are therefore highly influential in tokamak plasma dynamics. Purely fluid models are inherently incapable of capturing these effects, whereas the high dimensionality in purely kinetic models render them practically intractable for most relevant purposes.

        We consider a $\delta\!f$ decomposition model, with a macroscopic fluid background and microscopic kinetic correction, both fully coupled to each other. A similar manner of discretization is proposed to that used in the recent \texttt{STRUPHY} code \cite{Holderied_Possanner_Wang_2021, Holderied_2022, Li_et_al_2023} with a finite-element model for the background and a pseudo-particle/PiC model for the correction.

        The fluid background satisfies the full, non-linear, resistive, compressible, Hall MHD equations. \cite{Laakmann_Hu_Farrell_2022} introduces finite-element(-in-space) implicit timesteppers for the incompressible analogue to this system with structure-preserving (SP) properties in the ideal case, alongside parameter-robust preconditioners. We show that these timesteppers can derive from a finite-element-in-time (FET) (and finite-element-in-space) interpretation. The benefits of this reformulation are discussed, including the derivation of timesteppers that are higher order in time, and the quantifiable dissipative SP properties in the non-ideal, resistive case.
        
        We discuss possible options for extending this FET approach to timesteppers for the compressible case.

        The kinetic corrections satisfy linearized Boltzmann equations. Using a Lénard--Bernstein collision operator, these take Fokker--Planck-like forms \cite{Fokker_1914, Planck_1917} wherein pseudo-particles in the numerical model obey the neoclassical transport equations, with particle-independent Brownian drift terms. This offers a rigorous methodology for incorporating collisions into the particle transport model, without coupling the equations of motions for each particle.
        
        Works by Chen, Chacón et al. \cite{Chen_Chacón_Barnes_2011, Chacón_Chen_Barnes_2013, Chen_Chacón_2014, Chen_Chacón_2015} have developed structure-preserving particle pushers for neoclassical transport in the Vlasov equations, derived from Crank--Nicolson integrators. We show these too can can derive from a FET interpretation, similarly offering potential extensions to higher-order-in-time particle pushers. The FET formulation is used also to consider how the stochastic drift terms can be incorporated into the pushers. Stochastic gyrokinetic expansions are also discussed.

        Different options for the numerical implementation of these schemes are considered.

        Due to the efficacy of FET in the development of SP timesteppers for both the fluid and kinetic component, we hope this approach will prove effective in the future for developing SP timesteppers for the full hybrid model. We hope this will give us the opportunity to incorporate previously inaccessible kinetic effects into the highly effective, modern, finite-element MHD models.
    \end{abstract}
    
    
    \newpage
    \tableofcontents
    
    
    \newpage
    \pagenumbering{arabic}
    %\linenumbers\renewcommand\thelinenumber{\color{black!50}\arabic{linenumber}}
            \input{0 - introduction/main.tex}
        \part{Research}
            \input{1 - low-noise PiC models/main.tex}
            \input{2 - kinetic component/main.tex}
            \input{3 - fluid component/main.tex}
            \input{4 - numerical implementation/main.tex}
        \part{Project Overview}
            \input{5 - research plan/main.tex}
            \input{6 - summary/main.tex}
    
    
    %\section{}
    \newpage
    \pagenumbering{gobble}
        \printbibliography


    \newpage
    \pagenumbering{roman}
    \appendix
        \part{Appendices}
            \input{8 - Hilbert complexes/main.tex}
            \input{9 - weak conservation proofs/main.tex}
\end{document}

            \documentclass[12pt, a4paper]{report}

\input{template/main.tex}

\title{\BA{Title in Progress...}}
\author{Boris Andrews}
\affil{Mathematical Institute, University of Oxford}
\date{\today}


\begin{document}
    \pagenumbering{gobble}
    \maketitle
    
    
    \begin{abstract}
        Magnetic confinement reactors---in particular tokamaks---offer one of the most promising options for achieving practical nuclear fusion, with the potential to provide virtually limitless, clean energy. The theoretical and numerical modeling of tokamak plasmas is simultaneously an essential component of effective reactor design, and a great research barrier. Tokamak operational conditions exhibit comparatively low Knudsen numbers. Kinetic effects, including kinetic waves and instabilities, Landau damping, bump-on-tail instabilities and more, are therefore highly influential in tokamak plasma dynamics. Purely fluid models are inherently incapable of capturing these effects, whereas the high dimensionality in purely kinetic models render them practically intractable for most relevant purposes.

        We consider a $\delta\!f$ decomposition model, with a macroscopic fluid background and microscopic kinetic correction, both fully coupled to each other. A similar manner of discretization is proposed to that used in the recent \texttt{STRUPHY} code \cite{Holderied_Possanner_Wang_2021, Holderied_2022, Li_et_al_2023} with a finite-element model for the background and a pseudo-particle/PiC model for the correction.

        The fluid background satisfies the full, non-linear, resistive, compressible, Hall MHD equations. \cite{Laakmann_Hu_Farrell_2022} introduces finite-element(-in-space) implicit timesteppers for the incompressible analogue to this system with structure-preserving (SP) properties in the ideal case, alongside parameter-robust preconditioners. We show that these timesteppers can derive from a finite-element-in-time (FET) (and finite-element-in-space) interpretation. The benefits of this reformulation are discussed, including the derivation of timesteppers that are higher order in time, and the quantifiable dissipative SP properties in the non-ideal, resistive case.
        
        We discuss possible options for extending this FET approach to timesteppers for the compressible case.

        The kinetic corrections satisfy linearized Boltzmann equations. Using a Lénard--Bernstein collision operator, these take Fokker--Planck-like forms \cite{Fokker_1914, Planck_1917} wherein pseudo-particles in the numerical model obey the neoclassical transport equations, with particle-independent Brownian drift terms. This offers a rigorous methodology for incorporating collisions into the particle transport model, without coupling the equations of motions for each particle.
        
        Works by Chen, Chacón et al. \cite{Chen_Chacón_Barnes_2011, Chacón_Chen_Barnes_2013, Chen_Chacón_2014, Chen_Chacón_2015} have developed structure-preserving particle pushers for neoclassical transport in the Vlasov equations, derived from Crank--Nicolson integrators. We show these too can can derive from a FET interpretation, similarly offering potential extensions to higher-order-in-time particle pushers. The FET formulation is used also to consider how the stochastic drift terms can be incorporated into the pushers. Stochastic gyrokinetic expansions are also discussed.

        Different options for the numerical implementation of these schemes are considered.

        Due to the efficacy of FET in the development of SP timesteppers for both the fluid and kinetic component, we hope this approach will prove effective in the future for developing SP timesteppers for the full hybrid model. We hope this will give us the opportunity to incorporate previously inaccessible kinetic effects into the highly effective, modern, finite-element MHD models.
    \end{abstract}
    
    
    \newpage
    \tableofcontents
    
    
    \newpage
    \pagenumbering{arabic}
    %\linenumbers\renewcommand\thelinenumber{\color{black!50}\arabic{linenumber}}
            \input{0 - introduction/main.tex}
        \part{Research}
            \input{1 - low-noise PiC models/main.tex}
            \input{2 - kinetic component/main.tex}
            \input{3 - fluid component/main.tex}
            \input{4 - numerical implementation/main.tex}
        \part{Project Overview}
            \input{5 - research plan/main.tex}
            \input{6 - summary/main.tex}
    
    
    %\section{}
    \newpage
    \pagenumbering{gobble}
        \printbibliography


    \newpage
    \pagenumbering{roman}
    \appendix
        \part{Appendices}
            \input{8 - Hilbert complexes/main.tex}
            \input{9 - weak conservation proofs/main.tex}
\end{document}

\end{document}

            \documentclass[12pt, a4paper]{report}

\documentclass[12pt, a4paper]{report}

\input{template/main.tex}

\title{\BA{Title in Progress...}}
\author{Boris Andrews}
\affil{Mathematical Institute, University of Oxford}
\date{\today}


\begin{document}
    \pagenumbering{gobble}
    \maketitle
    
    
    \begin{abstract}
        Magnetic confinement reactors---in particular tokamaks---offer one of the most promising options for achieving practical nuclear fusion, with the potential to provide virtually limitless, clean energy. The theoretical and numerical modeling of tokamak plasmas is simultaneously an essential component of effective reactor design, and a great research barrier. Tokamak operational conditions exhibit comparatively low Knudsen numbers. Kinetic effects, including kinetic waves and instabilities, Landau damping, bump-on-tail instabilities and more, are therefore highly influential in tokamak plasma dynamics. Purely fluid models are inherently incapable of capturing these effects, whereas the high dimensionality in purely kinetic models render them practically intractable for most relevant purposes.

        We consider a $\delta\!f$ decomposition model, with a macroscopic fluid background and microscopic kinetic correction, both fully coupled to each other. A similar manner of discretization is proposed to that used in the recent \texttt{STRUPHY} code \cite{Holderied_Possanner_Wang_2021, Holderied_2022, Li_et_al_2023} with a finite-element model for the background and a pseudo-particle/PiC model for the correction.

        The fluid background satisfies the full, non-linear, resistive, compressible, Hall MHD equations. \cite{Laakmann_Hu_Farrell_2022} introduces finite-element(-in-space) implicit timesteppers for the incompressible analogue to this system with structure-preserving (SP) properties in the ideal case, alongside parameter-robust preconditioners. We show that these timesteppers can derive from a finite-element-in-time (FET) (and finite-element-in-space) interpretation. The benefits of this reformulation are discussed, including the derivation of timesteppers that are higher order in time, and the quantifiable dissipative SP properties in the non-ideal, resistive case.
        
        We discuss possible options for extending this FET approach to timesteppers for the compressible case.

        The kinetic corrections satisfy linearized Boltzmann equations. Using a Lénard--Bernstein collision operator, these take Fokker--Planck-like forms \cite{Fokker_1914, Planck_1917} wherein pseudo-particles in the numerical model obey the neoclassical transport equations, with particle-independent Brownian drift terms. This offers a rigorous methodology for incorporating collisions into the particle transport model, without coupling the equations of motions for each particle.
        
        Works by Chen, Chacón et al. \cite{Chen_Chacón_Barnes_2011, Chacón_Chen_Barnes_2013, Chen_Chacón_2014, Chen_Chacón_2015} have developed structure-preserving particle pushers for neoclassical transport in the Vlasov equations, derived from Crank--Nicolson integrators. We show these too can can derive from a FET interpretation, similarly offering potential extensions to higher-order-in-time particle pushers. The FET formulation is used also to consider how the stochastic drift terms can be incorporated into the pushers. Stochastic gyrokinetic expansions are also discussed.

        Different options for the numerical implementation of these schemes are considered.

        Due to the efficacy of FET in the development of SP timesteppers for both the fluid and kinetic component, we hope this approach will prove effective in the future for developing SP timesteppers for the full hybrid model. We hope this will give us the opportunity to incorporate previously inaccessible kinetic effects into the highly effective, modern, finite-element MHD models.
    \end{abstract}
    
    
    \newpage
    \tableofcontents
    
    
    \newpage
    \pagenumbering{arabic}
    %\linenumbers\renewcommand\thelinenumber{\color{black!50}\arabic{linenumber}}
            \input{0 - introduction/main.tex}
        \part{Research}
            \input{1 - low-noise PiC models/main.tex}
            \input{2 - kinetic component/main.tex}
            \input{3 - fluid component/main.tex}
            \input{4 - numerical implementation/main.tex}
        \part{Project Overview}
            \input{5 - research plan/main.tex}
            \input{6 - summary/main.tex}
    
    
    %\section{}
    \newpage
    \pagenumbering{gobble}
        \printbibliography


    \newpage
    \pagenumbering{roman}
    \appendix
        \part{Appendices}
            \input{8 - Hilbert complexes/main.tex}
            \input{9 - weak conservation proofs/main.tex}
\end{document}


\title{\BA{Title in Progress...}}
\author{Boris Andrews}
\affil{Mathematical Institute, University of Oxford}
\date{\today}


\begin{document}
    \pagenumbering{gobble}
    \maketitle
    
    
    \begin{abstract}
        Magnetic confinement reactors---in particular tokamaks---offer one of the most promising options for achieving practical nuclear fusion, with the potential to provide virtually limitless, clean energy. The theoretical and numerical modeling of tokamak plasmas is simultaneously an essential component of effective reactor design, and a great research barrier. Tokamak operational conditions exhibit comparatively low Knudsen numbers. Kinetic effects, including kinetic waves and instabilities, Landau damping, bump-on-tail instabilities and more, are therefore highly influential in tokamak plasma dynamics. Purely fluid models are inherently incapable of capturing these effects, whereas the high dimensionality in purely kinetic models render them practically intractable for most relevant purposes.

        We consider a $\delta\!f$ decomposition model, with a macroscopic fluid background and microscopic kinetic correction, both fully coupled to each other. A similar manner of discretization is proposed to that used in the recent \texttt{STRUPHY} code \cite{Holderied_Possanner_Wang_2021, Holderied_2022, Li_et_al_2023} with a finite-element model for the background and a pseudo-particle/PiC model for the correction.

        The fluid background satisfies the full, non-linear, resistive, compressible, Hall MHD equations. \cite{Laakmann_Hu_Farrell_2022} introduces finite-element(-in-space) implicit timesteppers for the incompressible analogue to this system with structure-preserving (SP) properties in the ideal case, alongside parameter-robust preconditioners. We show that these timesteppers can derive from a finite-element-in-time (FET) (and finite-element-in-space) interpretation. The benefits of this reformulation are discussed, including the derivation of timesteppers that are higher order in time, and the quantifiable dissipative SP properties in the non-ideal, resistive case.
        
        We discuss possible options for extending this FET approach to timesteppers for the compressible case.

        The kinetic corrections satisfy linearized Boltzmann equations. Using a Lénard--Bernstein collision operator, these take Fokker--Planck-like forms \cite{Fokker_1914, Planck_1917} wherein pseudo-particles in the numerical model obey the neoclassical transport equations, with particle-independent Brownian drift terms. This offers a rigorous methodology for incorporating collisions into the particle transport model, without coupling the equations of motions for each particle.
        
        Works by Chen, Chacón et al. \cite{Chen_Chacón_Barnes_2011, Chacón_Chen_Barnes_2013, Chen_Chacón_2014, Chen_Chacón_2015} have developed structure-preserving particle pushers for neoclassical transport in the Vlasov equations, derived from Crank--Nicolson integrators. We show these too can can derive from a FET interpretation, similarly offering potential extensions to higher-order-in-time particle pushers. The FET formulation is used also to consider how the stochastic drift terms can be incorporated into the pushers. Stochastic gyrokinetic expansions are also discussed.

        Different options for the numerical implementation of these schemes are considered.

        Due to the efficacy of FET in the development of SP timesteppers for both the fluid and kinetic component, we hope this approach will prove effective in the future for developing SP timesteppers for the full hybrid model. We hope this will give us the opportunity to incorporate previously inaccessible kinetic effects into the highly effective, modern, finite-element MHD models.
    \end{abstract}
    
    
    \newpage
    \tableofcontents
    
    
    \newpage
    \pagenumbering{arabic}
    %\linenumbers\renewcommand\thelinenumber{\color{black!50}\arabic{linenumber}}
            \documentclass[12pt, a4paper]{report}

\input{template/main.tex}

\title{\BA{Title in Progress...}}
\author{Boris Andrews}
\affil{Mathematical Institute, University of Oxford}
\date{\today}


\begin{document}
    \pagenumbering{gobble}
    \maketitle
    
    
    \begin{abstract}
        Magnetic confinement reactors---in particular tokamaks---offer one of the most promising options for achieving practical nuclear fusion, with the potential to provide virtually limitless, clean energy. The theoretical and numerical modeling of tokamak plasmas is simultaneously an essential component of effective reactor design, and a great research barrier. Tokamak operational conditions exhibit comparatively low Knudsen numbers. Kinetic effects, including kinetic waves and instabilities, Landau damping, bump-on-tail instabilities and more, are therefore highly influential in tokamak plasma dynamics. Purely fluid models are inherently incapable of capturing these effects, whereas the high dimensionality in purely kinetic models render them practically intractable for most relevant purposes.

        We consider a $\delta\!f$ decomposition model, with a macroscopic fluid background and microscopic kinetic correction, both fully coupled to each other. A similar manner of discretization is proposed to that used in the recent \texttt{STRUPHY} code \cite{Holderied_Possanner_Wang_2021, Holderied_2022, Li_et_al_2023} with a finite-element model for the background and a pseudo-particle/PiC model for the correction.

        The fluid background satisfies the full, non-linear, resistive, compressible, Hall MHD equations. \cite{Laakmann_Hu_Farrell_2022} introduces finite-element(-in-space) implicit timesteppers for the incompressible analogue to this system with structure-preserving (SP) properties in the ideal case, alongside parameter-robust preconditioners. We show that these timesteppers can derive from a finite-element-in-time (FET) (and finite-element-in-space) interpretation. The benefits of this reformulation are discussed, including the derivation of timesteppers that are higher order in time, and the quantifiable dissipative SP properties in the non-ideal, resistive case.
        
        We discuss possible options for extending this FET approach to timesteppers for the compressible case.

        The kinetic corrections satisfy linearized Boltzmann equations. Using a Lénard--Bernstein collision operator, these take Fokker--Planck-like forms \cite{Fokker_1914, Planck_1917} wherein pseudo-particles in the numerical model obey the neoclassical transport equations, with particle-independent Brownian drift terms. This offers a rigorous methodology for incorporating collisions into the particle transport model, without coupling the equations of motions for each particle.
        
        Works by Chen, Chacón et al. \cite{Chen_Chacón_Barnes_2011, Chacón_Chen_Barnes_2013, Chen_Chacón_2014, Chen_Chacón_2015} have developed structure-preserving particle pushers for neoclassical transport in the Vlasov equations, derived from Crank--Nicolson integrators. We show these too can can derive from a FET interpretation, similarly offering potential extensions to higher-order-in-time particle pushers. The FET formulation is used also to consider how the stochastic drift terms can be incorporated into the pushers. Stochastic gyrokinetic expansions are also discussed.

        Different options for the numerical implementation of these schemes are considered.

        Due to the efficacy of FET in the development of SP timesteppers for both the fluid and kinetic component, we hope this approach will prove effective in the future for developing SP timesteppers for the full hybrid model. We hope this will give us the opportunity to incorporate previously inaccessible kinetic effects into the highly effective, modern, finite-element MHD models.
    \end{abstract}
    
    
    \newpage
    \tableofcontents
    
    
    \newpage
    \pagenumbering{arabic}
    %\linenumbers\renewcommand\thelinenumber{\color{black!50}\arabic{linenumber}}
            \input{0 - introduction/main.tex}
        \part{Research}
            \input{1 - low-noise PiC models/main.tex}
            \input{2 - kinetic component/main.tex}
            \input{3 - fluid component/main.tex}
            \input{4 - numerical implementation/main.tex}
        \part{Project Overview}
            \input{5 - research plan/main.tex}
            \input{6 - summary/main.tex}
    
    
    %\section{}
    \newpage
    \pagenumbering{gobble}
        \printbibliography


    \newpage
    \pagenumbering{roman}
    \appendix
        \part{Appendices}
            \input{8 - Hilbert complexes/main.tex}
            \input{9 - weak conservation proofs/main.tex}
\end{document}

        \part{Research}
            \documentclass[12pt, a4paper]{report}

\input{template/main.tex}

\title{\BA{Title in Progress...}}
\author{Boris Andrews}
\affil{Mathematical Institute, University of Oxford}
\date{\today}


\begin{document}
    \pagenumbering{gobble}
    \maketitle
    
    
    \begin{abstract}
        Magnetic confinement reactors---in particular tokamaks---offer one of the most promising options for achieving practical nuclear fusion, with the potential to provide virtually limitless, clean energy. The theoretical and numerical modeling of tokamak plasmas is simultaneously an essential component of effective reactor design, and a great research barrier. Tokamak operational conditions exhibit comparatively low Knudsen numbers. Kinetic effects, including kinetic waves and instabilities, Landau damping, bump-on-tail instabilities and more, are therefore highly influential in tokamak plasma dynamics. Purely fluid models are inherently incapable of capturing these effects, whereas the high dimensionality in purely kinetic models render them practically intractable for most relevant purposes.

        We consider a $\delta\!f$ decomposition model, with a macroscopic fluid background and microscopic kinetic correction, both fully coupled to each other. A similar manner of discretization is proposed to that used in the recent \texttt{STRUPHY} code \cite{Holderied_Possanner_Wang_2021, Holderied_2022, Li_et_al_2023} with a finite-element model for the background and a pseudo-particle/PiC model for the correction.

        The fluid background satisfies the full, non-linear, resistive, compressible, Hall MHD equations. \cite{Laakmann_Hu_Farrell_2022} introduces finite-element(-in-space) implicit timesteppers for the incompressible analogue to this system with structure-preserving (SP) properties in the ideal case, alongside parameter-robust preconditioners. We show that these timesteppers can derive from a finite-element-in-time (FET) (and finite-element-in-space) interpretation. The benefits of this reformulation are discussed, including the derivation of timesteppers that are higher order in time, and the quantifiable dissipative SP properties in the non-ideal, resistive case.
        
        We discuss possible options for extending this FET approach to timesteppers for the compressible case.

        The kinetic corrections satisfy linearized Boltzmann equations. Using a Lénard--Bernstein collision operator, these take Fokker--Planck-like forms \cite{Fokker_1914, Planck_1917} wherein pseudo-particles in the numerical model obey the neoclassical transport equations, with particle-independent Brownian drift terms. This offers a rigorous methodology for incorporating collisions into the particle transport model, without coupling the equations of motions for each particle.
        
        Works by Chen, Chacón et al. \cite{Chen_Chacón_Barnes_2011, Chacón_Chen_Barnes_2013, Chen_Chacón_2014, Chen_Chacón_2015} have developed structure-preserving particle pushers for neoclassical transport in the Vlasov equations, derived from Crank--Nicolson integrators. We show these too can can derive from a FET interpretation, similarly offering potential extensions to higher-order-in-time particle pushers. The FET formulation is used also to consider how the stochastic drift terms can be incorporated into the pushers. Stochastic gyrokinetic expansions are also discussed.

        Different options for the numerical implementation of these schemes are considered.

        Due to the efficacy of FET in the development of SP timesteppers for both the fluid and kinetic component, we hope this approach will prove effective in the future for developing SP timesteppers for the full hybrid model. We hope this will give us the opportunity to incorporate previously inaccessible kinetic effects into the highly effective, modern, finite-element MHD models.
    \end{abstract}
    
    
    \newpage
    \tableofcontents
    
    
    \newpage
    \pagenumbering{arabic}
    %\linenumbers\renewcommand\thelinenumber{\color{black!50}\arabic{linenumber}}
            \input{0 - introduction/main.tex}
        \part{Research}
            \input{1 - low-noise PiC models/main.tex}
            \input{2 - kinetic component/main.tex}
            \input{3 - fluid component/main.tex}
            \input{4 - numerical implementation/main.tex}
        \part{Project Overview}
            \input{5 - research plan/main.tex}
            \input{6 - summary/main.tex}
    
    
    %\section{}
    \newpage
    \pagenumbering{gobble}
        \printbibliography


    \newpage
    \pagenumbering{roman}
    \appendix
        \part{Appendices}
            \input{8 - Hilbert complexes/main.tex}
            \input{9 - weak conservation proofs/main.tex}
\end{document}

            \documentclass[12pt, a4paper]{report}

\input{template/main.tex}

\title{\BA{Title in Progress...}}
\author{Boris Andrews}
\affil{Mathematical Institute, University of Oxford}
\date{\today}


\begin{document}
    \pagenumbering{gobble}
    \maketitle
    
    
    \begin{abstract}
        Magnetic confinement reactors---in particular tokamaks---offer one of the most promising options for achieving practical nuclear fusion, with the potential to provide virtually limitless, clean energy. The theoretical and numerical modeling of tokamak plasmas is simultaneously an essential component of effective reactor design, and a great research barrier. Tokamak operational conditions exhibit comparatively low Knudsen numbers. Kinetic effects, including kinetic waves and instabilities, Landau damping, bump-on-tail instabilities and more, are therefore highly influential in tokamak plasma dynamics. Purely fluid models are inherently incapable of capturing these effects, whereas the high dimensionality in purely kinetic models render them practically intractable for most relevant purposes.

        We consider a $\delta\!f$ decomposition model, with a macroscopic fluid background and microscopic kinetic correction, both fully coupled to each other. A similar manner of discretization is proposed to that used in the recent \texttt{STRUPHY} code \cite{Holderied_Possanner_Wang_2021, Holderied_2022, Li_et_al_2023} with a finite-element model for the background and a pseudo-particle/PiC model for the correction.

        The fluid background satisfies the full, non-linear, resistive, compressible, Hall MHD equations. \cite{Laakmann_Hu_Farrell_2022} introduces finite-element(-in-space) implicit timesteppers for the incompressible analogue to this system with structure-preserving (SP) properties in the ideal case, alongside parameter-robust preconditioners. We show that these timesteppers can derive from a finite-element-in-time (FET) (and finite-element-in-space) interpretation. The benefits of this reformulation are discussed, including the derivation of timesteppers that are higher order in time, and the quantifiable dissipative SP properties in the non-ideal, resistive case.
        
        We discuss possible options for extending this FET approach to timesteppers for the compressible case.

        The kinetic corrections satisfy linearized Boltzmann equations. Using a Lénard--Bernstein collision operator, these take Fokker--Planck-like forms \cite{Fokker_1914, Planck_1917} wherein pseudo-particles in the numerical model obey the neoclassical transport equations, with particle-independent Brownian drift terms. This offers a rigorous methodology for incorporating collisions into the particle transport model, without coupling the equations of motions for each particle.
        
        Works by Chen, Chacón et al. \cite{Chen_Chacón_Barnes_2011, Chacón_Chen_Barnes_2013, Chen_Chacón_2014, Chen_Chacón_2015} have developed structure-preserving particle pushers for neoclassical transport in the Vlasov equations, derived from Crank--Nicolson integrators. We show these too can can derive from a FET interpretation, similarly offering potential extensions to higher-order-in-time particle pushers. The FET formulation is used also to consider how the stochastic drift terms can be incorporated into the pushers. Stochastic gyrokinetic expansions are also discussed.

        Different options for the numerical implementation of these schemes are considered.

        Due to the efficacy of FET in the development of SP timesteppers for both the fluid and kinetic component, we hope this approach will prove effective in the future for developing SP timesteppers for the full hybrid model. We hope this will give us the opportunity to incorporate previously inaccessible kinetic effects into the highly effective, modern, finite-element MHD models.
    \end{abstract}
    
    
    \newpage
    \tableofcontents
    
    
    \newpage
    \pagenumbering{arabic}
    %\linenumbers\renewcommand\thelinenumber{\color{black!50}\arabic{linenumber}}
            \input{0 - introduction/main.tex}
        \part{Research}
            \input{1 - low-noise PiC models/main.tex}
            \input{2 - kinetic component/main.tex}
            \input{3 - fluid component/main.tex}
            \input{4 - numerical implementation/main.tex}
        \part{Project Overview}
            \input{5 - research plan/main.tex}
            \input{6 - summary/main.tex}
    
    
    %\section{}
    \newpage
    \pagenumbering{gobble}
        \printbibliography


    \newpage
    \pagenumbering{roman}
    \appendix
        \part{Appendices}
            \input{8 - Hilbert complexes/main.tex}
            \input{9 - weak conservation proofs/main.tex}
\end{document}

            \documentclass[12pt, a4paper]{report}

\input{template/main.tex}

\title{\BA{Title in Progress...}}
\author{Boris Andrews}
\affil{Mathematical Institute, University of Oxford}
\date{\today}


\begin{document}
    \pagenumbering{gobble}
    \maketitle
    
    
    \begin{abstract}
        Magnetic confinement reactors---in particular tokamaks---offer one of the most promising options for achieving practical nuclear fusion, with the potential to provide virtually limitless, clean energy. The theoretical and numerical modeling of tokamak plasmas is simultaneously an essential component of effective reactor design, and a great research barrier. Tokamak operational conditions exhibit comparatively low Knudsen numbers. Kinetic effects, including kinetic waves and instabilities, Landau damping, bump-on-tail instabilities and more, are therefore highly influential in tokamak plasma dynamics. Purely fluid models are inherently incapable of capturing these effects, whereas the high dimensionality in purely kinetic models render them practically intractable for most relevant purposes.

        We consider a $\delta\!f$ decomposition model, with a macroscopic fluid background and microscopic kinetic correction, both fully coupled to each other. A similar manner of discretization is proposed to that used in the recent \texttt{STRUPHY} code \cite{Holderied_Possanner_Wang_2021, Holderied_2022, Li_et_al_2023} with a finite-element model for the background and a pseudo-particle/PiC model for the correction.

        The fluid background satisfies the full, non-linear, resistive, compressible, Hall MHD equations. \cite{Laakmann_Hu_Farrell_2022} introduces finite-element(-in-space) implicit timesteppers for the incompressible analogue to this system with structure-preserving (SP) properties in the ideal case, alongside parameter-robust preconditioners. We show that these timesteppers can derive from a finite-element-in-time (FET) (and finite-element-in-space) interpretation. The benefits of this reformulation are discussed, including the derivation of timesteppers that are higher order in time, and the quantifiable dissipative SP properties in the non-ideal, resistive case.
        
        We discuss possible options for extending this FET approach to timesteppers for the compressible case.

        The kinetic corrections satisfy linearized Boltzmann equations. Using a Lénard--Bernstein collision operator, these take Fokker--Planck-like forms \cite{Fokker_1914, Planck_1917} wherein pseudo-particles in the numerical model obey the neoclassical transport equations, with particle-independent Brownian drift terms. This offers a rigorous methodology for incorporating collisions into the particle transport model, without coupling the equations of motions for each particle.
        
        Works by Chen, Chacón et al. \cite{Chen_Chacón_Barnes_2011, Chacón_Chen_Barnes_2013, Chen_Chacón_2014, Chen_Chacón_2015} have developed structure-preserving particle pushers for neoclassical transport in the Vlasov equations, derived from Crank--Nicolson integrators. We show these too can can derive from a FET interpretation, similarly offering potential extensions to higher-order-in-time particle pushers. The FET formulation is used also to consider how the stochastic drift terms can be incorporated into the pushers. Stochastic gyrokinetic expansions are also discussed.

        Different options for the numerical implementation of these schemes are considered.

        Due to the efficacy of FET in the development of SP timesteppers for both the fluid and kinetic component, we hope this approach will prove effective in the future for developing SP timesteppers for the full hybrid model. We hope this will give us the opportunity to incorporate previously inaccessible kinetic effects into the highly effective, modern, finite-element MHD models.
    \end{abstract}
    
    
    \newpage
    \tableofcontents
    
    
    \newpage
    \pagenumbering{arabic}
    %\linenumbers\renewcommand\thelinenumber{\color{black!50}\arabic{linenumber}}
            \input{0 - introduction/main.tex}
        \part{Research}
            \input{1 - low-noise PiC models/main.tex}
            \input{2 - kinetic component/main.tex}
            \input{3 - fluid component/main.tex}
            \input{4 - numerical implementation/main.tex}
        \part{Project Overview}
            \input{5 - research plan/main.tex}
            \input{6 - summary/main.tex}
    
    
    %\section{}
    \newpage
    \pagenumbering{gobble}
        \printbibliography


    \newpage
    \pagenumbering{roman}
    \appendix
        \part{Appendices}
            \input{8 - Hilbert complexes/main.tex}
            \input{9 - weak conservation proofs/main.tex}
\end{document}

            \documentclass[12pt, a4paper]{report}

\input{template/main.tex}

\title{\BA{Title in Progress...}}
\author{Boris Andrews}
\affil{Mathematical Institute, University of Oxford}
\date{\today}


\begin{document}
    \pagenumbering{gobble}
    \maketitle
    
    
    \begin{abstract}
        Magnetic confinement reactors---in particular tokamaks---offer one of the most promising options for achieving practical nuclear fusion, with the potential to provide virtually limitless, clean energy. The theoretical and numerical modeling of tokamak plasmas is simultaneously an essential component of effective reactor design, and a great research barrier. Tokamak operational conditions exhibit comparatively low Knudsen numbers. Kinetic effects, including kinetic waves and instabilities, Landau damping, bump-on-tail instabilities and more, are therefore highly influential in tokamak plasma dynamics. Purely fluid models are inherently incapable of capturing these effects, whereas the high dimensionality in purely kinetic models render them practically intractable for most relevant purposes.

        We consider a $\delta\!f$ decomposition model, with a macroscopic fluid background and microscopic kinetic correction, both fully coupled to each other. A similar manner of discretization is proposed to that used in the recent \texttt{STRUPHY} code \cite{Holderied_Possanner_Wang_2021, Holderied_2022, Li_et_al_2023} with a finite-element model for the background and a pseudo-particle/PiC model for the correction.

        The fluid background satisfies the full, non-linear, resistive, compressible, Hall MHD equations. \cite{Laakmann_Hu_Farrell_2022} introduces finite-element(-in-space) implicit timesteppers for the incompressible analogue to this system with structure-preserving (SP) properties in the ideal case, alongside parameter-robust preconditioners. We show that these timesteppers can derive from a finite-element-in-time (FET) (and finite-element-in-space) interpretation. The benefits of this reformulation are discussed, including the derivation of timesteppers that are higher order in time, and the quantifiable dissipative SP properties in the non-ideal, resistive case.
        
        We discuss possible options for extending this FET approach to timesteppers for the compressible case.

        The kinetic corrections satisfy linearized Boltzmann equations. Using a Lénard--Bernstein collision operator, these take Fokker--Planck-like forms \cite{Fokker_1914, Planck_1917} wherein pseudo-particles in the numerical model obey the neoclassical transport equations, with particle-independent Brownian drift terms. This offers a rigorous methodology for incorporating collisions into the particle transport model, without coupling the equations of motions for each particle.
        
        Works by Chen, Chacón et al. \cite{Chen_Chacón_Barnes_2011, Chacón_Chen_Barnes_2013, Chen_Chacón_2014, Chen_Chacón_2015} have developed structure-preserving particle pushers for neoclassical transport in the Vlasov equations, derived from Crank--Nicolson integrators. We show these too can can derive from a FET interpretation, similarly offering potential extensions to higher-order-in-time particle pushers. The FET formulation is used also to consider how the stochastic drift terms can be incorporated into the pushers. Stochastic gyrokinetic expansions are also discussed.

        Different options for the numerical implementation of these schemes are considered.

        Due to the efficacy of FET in the development of SP timesteppers for both the fluid and kinetic component, we hope this approach will prove effective in the future for developing SP timesteppers for the full hybrid model. We hope this will give us the opportunity to incorporate previously inaccessible kinetic effects into the highly effective, modern, finite-element MHD models.
    \end{abstract}
    
    
    \newpage
    \tableofcontents
    
    
    \newpage
    \pagenumbering{arabic}
    %\linenumbers\renewcommand\thelinenumber{\color{black!50}\arabic{linenumber}}
            \input{0 - introduction/main.tex}
        \part{Research}
            \input{1 - low-noise PiC models/main.tex}
            \input{2 - kinetic component/main.tex}
            \input{3 - fluid component/main.tex}
            \input{4 - numerical implementation/main.tex}
        \part{Project Overview}
            \input{5 - research plan/main.tex}
            \input{6 - summary/main.tex}
    
    
    %\section{}
    \newpage
    \pagenumbering{gobble}
        \printbibliography


    \newpage
    \pagenumbering{roman}
    \appendix
        \part{Appendices}
            \input{8 - Hilbert complexes/main.tex}
            \input{9 - weak conservation proofs/main.tex}
\end{document}

        \part{Project Overview}
            \documentclass[12pt, a4paper]{report}

\input{template/main.tex}

\title{\BA{Title in Progress...}}
\author{Boris Andrews}
\affil{Mathematical Institute, University of Oxford}
\date{\today}


\begin{document}
    \pagenumbering{gobble}
    \maketitle
    
    
    \begin{abstract}
        Magnetic confinement reactors---in particular tokamaks---offer one of the most promising options for achieving practical nuclear fusion, with the potential to provide virtually limitless, clean energy. The theoretical and numerical modeling of tokamak plasmas is simultaneously an essential component of effective reactor design, and a great research barrier. Tokamak operational conditions exhibit comparatively low Knudsen numbers. Kinetic effects, including kinetic waves and instabilities, Landau damping, bump-on-tail instabilities and more, are therefore highly influential in tokamak plasma dynamics. Purely fluid models are inherently incapable of capturing these effects, whereas the high dimensionality in purely kinetic models render them practically intractable for most relevant purposes.

        We consider a $\delta\!f$ decomposition model, with a macroscopic fluid background and microscopic kinetic correction, both fully coupled to each other. A similar manner of discretization is proposed to that used in the recent \texttt{STRUPHY} code \cite{Holderied_Possanner_Wang_2021, Holderied_2022, Li_et_al_2023} with a finite-element model for the background and a pseudo-particle/PiC model for the correction.

        The fluid background satisfies the full, non-linear, resistive, compressible, Hall MHD equations. \cite{Laakmann_Hu_Farrell_2022} introduces finite-element(-in-space) implicit timesteppers for the incompressible analogue to this system with structure-preserving (SP) properties in the ideal case, alongside parameter-robust preconditioners. We show that these timesteppers can derive from a finite-element-in-time (FET) (and finite-element-in-space) interpretation. The benefits of this reformulation are discussed, including the derivation of timesteppers that are higher order in time, and the quantifiable dissipative SP properties in the non-ideal, resistive case.
        
        We discuss possible options for extending this FET approach to timesteppers for the compressible case.

        The kinetic corrections satisfy linearized Boltzmann equations. Using a Lénard--Bernstein collision operator, these take Fokker--Planck-like forms \cite{Fokker_1914, Planck_1917} wherein pseudo-particles in the numerical model obey the neoclassical transport equations, with particle-independent Brownian drift terms. This offers a rigorous methodology for incorporating collisions into the particle transport model, without coupling the equations of motions for each particle.
        
        Works by Chen, Chacón et al. \cite{Chen_Chacón_Barnes_2011, Chacón_Chen_Barnes_2013, Chen_Chacón_2014, Chen_Chacón_2015} have developed structure-preserving particle pushers for neoclassical transport in the Vlasov equations, derived from Crank--Nicolson integrators. We show these too can can derive from a FET interpretation, similarly offering potential extensions to higher-order-in-time particle pushers. The FET formulation is used also to consider how the stochastic drift terms can be incorporated into the pushers. Stochastic gyrokinetic expansions are also discussed.

        Different options for the numerical implementation of these schemes are considered.

        Due to the efficacy of FET in the development of SP timesteppers for both the fluid and kinetic component, we hope this approach will prove effective in the future for developing SP timesteppers for the full hybrid model. We hope this will give us the opportunity to incorporate previously inaccessible kinetic effects into the highly effective, modern, finite-element MHD models.
    \end{abstract}
    
    
    \newpage
    \tableofcontents
    
    
    \newpage
    \pagenumbering{arabic}
    %\linenumbers\renewcommand\thelinenumber{\color{black!50}\arabic{linenumber}}
            \input{0 - introduction/main.tex}
        \part{Research}
            \input{1 - low-noise PiC models/main.tex}
            \input{2 - kinetic component/main.tex}
            \input{3 - fluid component/main.tex}
            \input{4 - numerical implementation/main.tex}
        \part{Project Overview}
            \input{5 - research plan/main.tex}
            \input{6 - summary/main.tex}
    
    
    %\section{}
    \newpage
    \pagenumbering{gobble}
        \printbibliography


    \newpage
    \pagenumbering{roman}
    \appendix
        \part{Appendices}
            \input{8 - Hilbert complexes/main.tex}
            \input{9 - weak conservation proofs/main.tex}
\end{document}

            \documentclass[12pt, a4paper]{report}

\input{template/main.tex}

\title{\BA{Title in Progress...}}
\author{Boris Andrews}
\affil{Mathematical Institute, University of Oxford}
\date{\today}


\begin{document}
    \pagenumbering{gobble}
    \maketitle
    
    
    \begin{abstract}
        Magnetic confinement reactors---in particular tokamaks---offer one of the most promising options for achieving practical nuclear fusion, with the potential to provide virtually limitless, clean energy. The theoretical and numerical modeling of tokamak plasmas is simultaneously an essential component of effective reactor design, and a great research barrier. Tokamak operational conditions exhibit comparatively low Knudsen numbers. Kinetic effects, including kinetic waves and instabilities, Landau damping, bump-on-tail instabilities and more, are therefore highly influential in tokamak plasma dynamics. Purely fluid models are inherently incapable of capturing these effects, whereas the high dimensionality in purely kinetic models render them practically intractable for most relevant purposes.

        We consider a $\delta\!f$ decomposition model, with a macroscopic fluid background and microscopic kinetic correction, both fully coupled to each other. A similar manner of discretization is proposed to that used in the recent \texttt{STRUPHY} code \cite{Holderied_Possanner_Wang_2021, Holderied_2022, Li_et_al_2023} with a finite-element model for the background and a pseudo-particle/PiC model for the correction.

        The fluid background satisfies the full, non-linear, resistive, compressible, Hall MHD equations. \cite{Laakmann_Hu_Farrell_2022} introduces finite-element(-in-space) implicit timesteppers for the incompressible analogue to this system with structure-preserving (SP) properties in the ideal case, alongside parameter-robust preconditioners. We show that these timesteppers can derive from a finite-element-in-time (FET) (and finite-element-in-space) interpretation. The benefits of this reformulation are discussed, including the derivation of timesteppers that are higher order in time, and the quantifiable dissipative SP properties in the non-ideal, resistive case.
        
        We discuss possible options for extending this FET approach to timesteppers for the compressible case.

        The kinetic corrections satisfy linearized Boltzmann equations. Using a Lénard--Bernstein collision operator, these take Fokker--Planck-like forms \cite{Fokker_1914, Planck_1917} wherein pseudo-particles in the numerical model obey the neoclassical transport equations, with particle-independent Brownian drift terms. This offers a rigorous methodology for incorporating collisions into the particle transport model, without coupling the equations of motions for each particle.
        
        Works by Chen, Chacón et al. \cite{Chen_Chacón_Barnes_2011, Chacón_Chen_Barnes_2013, Chen_Chacón_2014, Chen_Chacón_2015} have developed structure-preserving particle pushers for neoclassical transport in the Vlasov equations, derived from Crank--Nicolson integrators. We show these too can can derive from a FET interpretation, similarly offering potential extensions to higher-order-in-time particle pushers. The FET formulation is used also to consider how the stochastic drift terms can be incorporated into the pushers. Stochastic gyrokinetic expansions are also discussed.

        Different options for the numerical implementation of these schemes are considered.

        Due to the efficacy of FET in the development of SP timesteppers for both the fluid and kinetic component, we hope this approach will prove effective in the future for developing SP timesteppers for the full hybrid model. We hope this will give us the opportunity to incorporate previously inaccessible kinetic effects into the highly effective, modern, finite-element MHD models.
    \end{abstract}
    
    
    \newpage
    \tableofcontents
    
    
    \newpage
    \pagenumbering{arabic}
    %\linenumbers\renewcommand\thelinenumber{\color{black!50}\arabic{linenumber}}
            \input{0 - introduction/main.tex}
        \part{Research}
            \input{1 - low-noise PiC models/main.tex}
            \input{2 - kinetic component/main.tex}
            \input{3 - fluid component/main.tex}
            \input{4 - numerical implementation/main.tex}
        \part{Project Overview}
            \input{5 - research plan/main.tex}
            \input{6 - summary/main.tex}
    
    
    %\section{}
    \newpage
    \pagenumbering{gobble}
        \printbibliography


    \newpage
    \pagenumbering{roman}
    \appendix
        \part{Appendices}
            \input{8 - Hilbert complexes/main.tex}
            \input{9 - weak conservation proofs/main.tex}
\end{document}

    
    
    %\section{}
    \newpage
    \pagenumbering{gobble}
        \printbibliography


    \newpage
    \pagenumbering{roman}
    \appendix
        \part{Appendices}
            \documentclass[12pt, a4paper]{report}

\input{template/main.tex}

\title{\BA{Title in Progress...}}
\author{Boris Andrews}
\affil{Mathematical Institute, University of Oxford}
\date{\today}


\begin{document}
    \pagenumbering{gobble}
    \maketitle
    
    
    \begin{abstract}
        Magnetic confinement reactors---in particular tokamaks---offer one of the most promising options for achieving practical nuclear fusion, with the potential to provide virtually limitless, clean energy. The theoretical and numerical modeling of tokamak plasmas is simultaneously an essential component of effective reactor design, and a great research barrier. Tokamak operational conditions exhibit comparatively low Knudsen numbers. Kinetic effects, including kinetic waves and instabilities, Landau damping, bump-on-tail instabilities and more, are therefore highly influential in tokamak plasma dynamics. Purely fluid models are inherently incapable of capturing these effects, whereas the high dimensionality in purely kinetic models render them practically intractable for most relevant purposes.

        We consider a $\delta\!f$ decomposition model, with a macroscopic fluid background and microscopic kinetic correction, both fully coupled to each other. A similar manner of discretization is proposed to that used in the recent \texttt{STRUPHY} code \cite{Holderied_Possanner_Wang_2021, Holderied_2022, Li_et_al_2023} with a finite-element model for the background and a pseudo-particle/PiC model for the correction.

        The fluid background satisfies the full, non-linear, resistive, compressible, Hall MHD equations. \cite{Laakmann_Hu_Farrell_2022} introduces finite-element(-in-space) implicit timesteppers for the incompressible analogue to this system with structure-preserving (SP) properties in the ideal case, alongside parameter-robust preconditioners. We show that these timesteppers can derive from a finite-element-in-time (FET) (and finite-element-in-space) interpretation. The benefits of this reformulation are discussed, including the derivation of timesteppers that are higher order in time, and the quantifiable dissipative SP properties in the non-ideal, resistive case.
        
        We discuss possible options for extending this FET approach to timesteppers for the compressible case.

        The kinetic corrections satisfy linearized Boltzmann equations. Using a Lénard--Bernstein collision operator, these take Fokker--Planck-like forms \cite{Fokker_1914, Planck_1917} wherein pseudo-particles in the numerical model obey the neoclassical transport equations, with particle-independent Brownian drift terms. This offers a rigorous methodology for incorporating collisions into the particle transport model, without coupling the equations of motions for each particle.
        
        Works by Chen, Chacón et al. \cite{Chen_Chacón_Barnes_2011, Chacón_Chen_Barnes_2013, Chen_Chacón_2014, Chen_Chacón_2015} have developed structure-preserving particle pushers for neoclassical transport in the Vlasov equations, derived from Crank--Nicolson integrators. We show these too can can derive from a FET interpretation, similarly offering potential extensions to higher-order-in-time particle pushers. The FET formulation is used also to consider how the stochastic drift terms can be incorporated into the pushers. Stochastic gyrokinetic expansions are also discussed.

        Different options for the numerical implementation of these schemes are considered.

        Due to the efficacy of FET in the development of SP timesteppers for both the fluid and kinetic component, we hope this approach will prove effective in the future for developing SP timesteppers for the full hybrid model. We hope this will give us the opportunity to incorporate previously inaccessible kinetic effects into the highly effective, modern, finite-element MHD models.
    \end{abstract}
    
    
    \newpage
    \tableofcontents
    
    
    \newpage
    \pagenumbering{arabic}
    %\linenumbers\renewcommand\thelinenumber{\color{black!50}\arabic{linenumber}}
            \input{0 - introduction/main.tex}
        \part{Research}
            \input{1 - low-noise PiC models/main.tex}
            \input{2 - kinetic component/main.tex}
            \input{3 - fluid component/main.tex}
            \input{4 - numerical implementation/main.tex}
        \part{Project Overview}
            \input{5 - research plan/main.tex}
            \input{6 - summary/main.tex}
    
    
    %\section{}
    \newpage
    \pagenumbering{gobble}
        \printbibliography


    \newpage
    \pagenumbering{roman}
    \appendix
        \part{Appendices}
            \input{8 - Hilbert complexes/main.tex}
            \input{9 - weak conservation proofs/main.tex}
\end{document}

            \documentclass[12pt, a4paper]{report}

\input{template/main.tex}

\title{\BA{Title in Progress...}}
\author{Boris Andrews}
\affil{Mathematical Institute, University of Oxford}
\date{\today}


\begin{document}
    \pagenumbering{gobble}
    \maketitle
    
    
    \begin{abstract}
        Magnetic confinement reactors---in particular tokamaks---offer one of the most promising options for achieving practical nuclear fusion, with the potential to provide virtually limitless, clean energy. The theoretical and numerical modeling of tokamak plasmas is simultaneously an essential component of effective reactor design, and a great research barrier. Tokamak operational conditions exhibit comparatively low Knudsen numbers. Kinetic effects, including kinetic waves and instabilities, Landau damping, bump-on-tail instabilities and more, are therefore highly influential in tokamak plasma dynamics. Purely fluid models are inherently incapable of capturing these effects, whereas the high dimensionality in purely kinetic models render them practically intractable for most relevant purposes.

        We consider a $\delta\!f$ decomposition model, with a macroscopic fluid background and microscopic kinetic correction, both fully coupled to each other. A similar manner of discretization is proposed to that used in the recent \texttt{STRUPHY} code \cite{Holderied_Possanner_Wang_2021, Holderied_2022, Li_et_al_2023} with a finite-element model for the background and a pseudo-particle/PiC model for the correction.

        The fluid background satisfies the full, non-linear, resistive, compressible, Hall MHD equations. \cite{Laakmann_Hu_Farrell_2022} introduces finite-element(-in-space) implicit timesteppers for the incompressible analogue to this system with structure-preserving (SP) properties in the ideal case, alongside parameter-robust preconditioners. We show that these timesteppers can derive from a finite-element-in-time (FET) (and finite-element-in-space) interpretation. The benefits of this reformulation are discussed, including the derivation of timesteppers that are higher order in time, and the quantifiable dissipative SP properties in the non-ideal, resistive case.
        
        We discuss possible options for extending this FET approach to timesteppers for the compressible case.

        The kinetic corrections satisfy linearized Boltzmann equations. Using a Lénard--Bernstein collision operator, these take Fokker--Planck-like forms \cite{Fokker_1914, Planck_1917} wherein pseudo-particles in the numerical model obey the neoclassical transport equations, with particle-independent Brownian drift terms. This offers a rigorous methodology for incorporating collisions into the particle transport model, without coupling the equations of motions for each particle.
        
        Works by Chen, Chacón et al. \cite{Chen_Chacón_Barnes_2011, Chacón_Chen_Barnes_2013, Chen_Chacón_2014, Chen_Chacón_2015} have developed structure-preserving particle pushers for neoclassical transport in the Vlasov equations, derived from Crank--Nicolson integrators. We show these too can can derive from a FET interpretation, similarly offering potential extensions to higher-order-in-time particle pushers. The FET formulation is used also to consider how the stochastic drift terms can be incorporated into the pushers. Stochastic gyrokinetic expansions are also discussed.

        Different options for the numerical implementation of these schemes are considered.

        Due to the efficacy of FET in the development of SP timesteppers for both the fluid and kinetic component, we hope this approach will prove effective in the future for developing SP timesteppers for the full hybrid model. We hope this will give us the opportunity to incorporate previously inaccessible kinetic effects into the highly effective, modern, finite-element MHD models.
    \end{abstract}
    
    
    \newpage
    \tableofcontents
    
    
    \newpage
    \pagenumbering{arabic}
    %\linenumbers\renewcommand\thelinenumber{\color{black!50}\arabic{linenumber}}
            \input{0 - introduction/main.tex}
        \part{Research}
            \input{1 - low-noise PiC models/main.tex}
            \input{2 - kinetic component/main.tex}
            \input{3 - fluid component/main.tex}
            \input{4 - numerical implementation/main.tex}
        \part{Project Overview}
            \input{5 - research plan/main.tex}
            \input{6 - summary/main.tex}
    
    
    %\section{}
    \newpage
    \pagenumbering{gobble}
        \printbibliography


    \newpage
    \pagenumbering{roman}
    \appendix
        \part{Appendices}
            \input{8 - Hilbert complexes/main.tex}
            \input{9 - weak conservation proofs/main.tex}
\end{document}

\end{document}

            \documentclass[12pt, a4paper]{report}

\documentclass[12pt, a4paper]{report}

\input{template/main.tex}

\title{\BA{Title in Progress...}}
\author{Boris Andrews}
\affil{Mathematical Institute, University of Oxford}
\date{\today}


\begin{document}
    \pagenumbering{gobble}
    \maketitle
    
    
    \begin{abstract}
        Magnetic confinement reactors---in particular tokamaks---offer one of the most promising options for achieving practical nuclear fusion, with the potential to provide virtually limitless, clean energy. The theoretical and numerical modeling of tokamak plasmas is simultaneously an essential component of effective reactor design, and a great research barrier. Tokamak operational conditions exhibit comparatively low Knudsen numbers. Kinetic effects, including kinetic waves and instabilities, Landau damping, bump-on-tail instabilities and more, are therefore highly influential in tokamak plasma dynamics. Purely fluid models are inherently incapable of capturing these effects, whereas the high dimensionality in purely kinetic models render them practically intractable for most relevant purposes.

        We consider a $\delta\!f$ decomposition model, with a macroscopic fluid background and microscopic kinetic correction, both fully coupled to each other. A similar manner of discretization is proposed to that used in the recent \texttt{STRUPHY} code \cite{Holderied_Possanner_Wang_2021, Holderied_2022, Li_et_al_2023} with a finite-element model for the background and a pseudo-particle/PiC model for the correction.

        The fluid background satisfies the full, non-linear, resistive, compressible, Hall MHD equations. \cite{Laakmann_Hu_Farrell_2022} introduces finite-element(-in-space) implicit timesteppers for the incompressible analogue to this system with structure-preserving (SP) properties in the ideal case, alongside parameter-robust preconditioners. We show that these timesteppers can derive from a finite-element-in-time (FET) (and finite-element-in-space) interpretation. The benefits of this reformulation are discussed, including the derivation of timesteppers that are higher order in time, and the quantifiable dissipative SP properties in the non-ideal, resistive case.
        
        We discuss possible options for extending this FET approach to timesteppers for the compressible case.

        The kinetic corrections satisfy linearized Boltzmann equations. Using a Lénard--Bernstein collision operator, these take Fokker--Planck-like forms \cite{Fokker_1914, Planck_1917} wherein pseudo-particles in the numerical model obey the neoclassical transport equations, with particle-independent Brownian drift terms. This offers a rigorous methodology for incorporating collisions into the particle transport model, without coupling the equations of motions for each particle.
        
        Works by Chen, Chacón et al. \cite{Chen_Chacón_Barnes_2011, Chacón_Chen_Barnes_2013, Chen_Chacón_2014, Chen_Chacón_2015} have developed structure-preserving particle pushers for neoclassical transport in the Vlasov equations, derived from Crank--Nicolson integrators. We show these too can can derive from a FET interpretation, similarly offering potential extensions to higher-order-in-time particle pushers. The FET formulation is used also to consider how the stochastic drift terms can be incorporated into the pushers. Stochastic gyrokinetic expansions are also discussed.

        Different options for the numerical implementation of these schemes are considered.

        Due to the efficacy of FET in the development of SP timesteppers for both the fluid and kinetic component, we hope this approach will prove effective in the future for developing SP timesteppers for the full hybrid model. We hope this will give us the opportunity to incorporate previously inaccessible kinetic effects into the highly effective, modern, finite-element MHD models.
    \end{abstract}
    
    
    \newpage
    \tableofcontents
    
    
    \newpage
    \pagenumbering{arabic}
    %\linenumbers\renewcommand\thelinenumber{\color{black!50}\arabic{linenumber}}
            \input{0 - introduction/main.tex}
        \part{Research}
            \input{1 - low-noise PiC models/main.tex}
            \input{2 - kinetic component/main.tex}
            \input{3 - fluid component/main.tex}
            \input{4 - numerical implementation/main.tex}
        \part{Project Overview}
            \input{5 - research plan/main.tex}
            \input{6 - summary/main.tex}
    
    
    %\section{}
    \newpage
    \pagenumbering{gobble}
        \printbibliography


    \newpage
    \pagenumbering{roman}
    \appendix
        \part{Appendices}
            \input{8 - Hilbert complexes/main.tex}
            \input{9 - weak conservation proofs/main.tex}
\end{document}


\title{\BA{Title in Progress...}}
\author{Boris Andrews}
\affil{Mathematical Institute, University of Oxford}
\date{\today}


\begin{document}
    \pagenumbering{gobble}
    \maketitle
    
    
    \begin{abstract}
        Magnetic confinement reactors---in particular tokamaks---offer one of the most promising options for achieving practical nuclear fusion, with the potential to provide virtually limitless, clean energy. The theoretical and numerical modeling of tokamak plasmas is simultaneously an essential component of effective reactor design, and a great research barrier. Tokamak operational conditions exhibit comparatively low Knudsen numbers. Kinetic effects, including kinetic waves and instabilities, Landau damping, bump-on-tail instabilities and more, are therefore highly influential in tokamak plasma dynamics. Purely fluid models are inherently incapable of capturing these effects, whereas the high dimensionality in purely kinetic models render them practically intractable for most relevant purposes.

        We consider a $\delta\!f$ decomposition model, with a macroscopic fluid background and microscopic kinetic correction, both fully coupled to each other. A similar manner of discretization is proposed to that used in the recent \texttt{STRUPHY} code \cite{Holderied_Possanner_Wang_2021, Holderied_2022, Li_et_al_2023} with a finite-element model for the background and a pseudo-particle/PiC model for the correction.

        The fluid background satisfies the full, non-linear, resistive, compressible, Hall MHD equations. \cite{Laakmann_Hu_Farrell_2022} introduces finite-element(-in-space) implicit timesteppers for the incompressible analogue to this system with structure-preserving (SP) properties in the ideal case, alongside parameter-robust preconditioners. We show that these timesteppers can derive from a finite-element-in-time (FET) (and finite-element-in-space) interpretation. The benefits of this reformulation are discussed, including the derivation of timesteppers that are higher order in time, and the quantifiable dissipative SP properties in the non-ideal, resistive case.
        
        We discuss possible options for extending this FET approach to timesteppers for the compressible case.

        The kinetic corrections satisfy linearized Boltzmann equations. Using a Lénard--Bernstein collision operator, these take Fokker--Planck-like forms \cite{Fokker_1914, Planck_1917} wherein pseudo-particles in the numerical model obey the neoclassical transport equations, with particle-independent Brownian drift terms. This offers a rigorous methodology for incorporating collisions into the particle transport model, without coupling the equations of motions for each particle.
        
        Works by Chen, Chacón et al. \cite{Chen_Chacón_Barnes_2011, Chacón_Chen_Barnes_2013, Chen_Chacón_2014, Chen_Chacón_2015} have developed structure-preserving particle pushers for neoclassical transport in the Vlasov equations, derived from Crank--Nicolson integrators. We show these too can can derive from a FET interpretation, similarly offering potential extensions to higher-order-in-time particle pushers. The FET formulation is used also to consider how the stochastic drift terms can be incorporated into the pushers. Stochastic gyrokinetic expansions are also discussed.

        Different options for the numerical implementation of these schemes are considered.

        Due to the efficacy of FET in the development of SP timesteppers for both the fluid and kinetic component, we hope this approach will prove effective in the future for developing SP timesteppers for the full hybrid model. We hope this will give us the opportunity to incorporate previously inaccessible kinetic effects into the highly effective, modern, finite-element MHD models.
    \end{abstract}
    
    
    \newpage
    \tableofcontents
    
    
    \newpage
    \pagenumbering{arabic}
    %\linenumbers\renewcommand\thelinenumber{\color{black!50}\arabic{linenumber}}
            \documentclass[12pt, a4paper]{report}

\input{template/main.tex}

\title{\BA{Title in Progress...}}
\author{Boris Andrews}
\affil{Mathematical Institute, University of Oxford}
\date{\today}


\begin{document}
    \pagenumbering{gobble}
    \maketitle
    
    
    \begin{abstract}
        Magnetic confinement reactors---in particular tokamaks---offer one of the most promising options for achieving practical nuclear fusion, with the potential to provide virtually limitless, clean energy. The theoretical and numerical modeling of tokamak plasmas is simultaneously an essential component of effective reactor design, and a great research barrier. Tokamak operational conditions exhibit comparatively low Knudsen numbers. Kinetic effects, including kinetic waves and instabilities, Landau damping, bump-on-tail instabilities and more, are therefore highly influential in tokamak plasma dynamics. Purely fluid models are inherently incapable of capturing these effects, whereas the high dimensionality in purely kinetic models render them practically intractable for most relevant purposes.

        We consider a $\delta\!f$ decomposition model, with a macroscopic fluid background and microscopic kinetic correction, both fully coupled to each other. A similar manner of discretization is proposed to that used in the recent \texttt{STRUPHY} code \cite{Holderied_Possanner_Wang_2021, Holderied_2022, Li_et_al_2023} with a finite-element model for the background and a pseudo-particle/PiC model for the correction.

        The fluid background satisfies the full, non-linear, resistive, compressible, Hall MHD equations. \cite{Laakmann_Hu_Farrell_2022} introduces finite-element(-in-space) implicit timesteppers for the incompressible analogue to this system with structure-preserving (SP) properties in the ideal case, alongside parameter-robust preconditioners. We show that these timesteppers can derive from a finite-element-in-time (FET) (and finite-element-in-space) interpretation. The benefits of this reformulation are discussed, including the derivation of timesteppers that are higher order in time, and the quantifiable dissipative SP properties in the non-ideal, resistive case.
        
        We discuss possible options for extending this FET approach to timesteppers for the compressible case.

        The kinetic corrections satisfy linearized Boltzmann equations. Using a Lénard--Bernstein collision operator, these take Fokker--Planck-like forms \cite{Fokker_1914, Planck_1917} wherein pseudo-particles in the numerical model obey the neoclassical transport equations, with particle-independent Brownian drift terms. This offers a rigorous methodology for incorporating collisions into the particle transport model, without coupling the equations of motions for each particle.
        
        Works by Chen, Chacón et al. \cite{Chen_Chacón_Barnes_2011, Chacón_Chen_Barnes_2013, Chen_Chacón_2014, Chen_Chacón_2015} have developed structure-preserving particle pushers for neoclassical transport in the Vlasov equations, derived from Crank--Nicolson integrators. We show these too can can derive from a FET interpretation, similarly offering potential extensions to higher-order-in-time particle pushers. The FET formulation is used also to consider how the stochastic drift terms can be incorporated into the pushers. Stochastic gyrokinetic expansions are also discussed.

        Different options for the numerical implementation of these schemes are considered.

        Due to the efficacy of FET in the development of SP timesteppers for both the fluid and kinetic component, we hope this approach will prove effective in the future for developing SP timesteppers for the full hybrid model. We hope this will give us the opportunity to incorporate previously inaccessible kinetic effects into the highly effective, modern, finite-element MHD models.
    \end{abstract}
    
    
    \newpage
    \tableofcontents
    
    
    \newpage
    \pagenumbering{arabic}
    %\linenumbers\renewcommand\thelinenumber{\color{black!50}\arabic{linenumber}}
            \input{0 - introduction/main.tex}
        \part{Research}
            \input{1 - low-noise PiC models/main.tex}
            \input{2 - kinetic component/main.tex}
            \input{3 - fluid component/main.tex}
            \input{4 - numerical implementation/main.tex}
        \part{Project Overview}
            \input{5 - research plan/main.tex}
            \input{6 - summary/main.tex}
    
    
    %\section{}
    \newpage
    \pagenumbering{gobble}
        \printbibliography


    \newpage
    \pagenumbering{roman}
    \appendix
        \part{Appendices}
            \input{8 - Hilbert complexes/main.tex}
            \input{9 - weak conservation proofs/main.tex}
\end{document}

        \part{Research}
            \documentclass[12pt, a4paper]{report}

\input{template/main.tex}

\title{\BA{Title in Progress...}}
\author{Boris Andrews}
\affil{Mathematical Institute, University of Oxford}
\date{\today}


\begin{document}
    \pagenumbering{gobble}
    \maketitle
    
    
    \begin{abstract}
        Magnetic confinement reactors---in particular tokamaks---offer one of the most promising options for achieving practical nuclear fusion, with the potential to provide virtually limitless, clean energy. The theoretical and numerical modeling of tokamak plasmas is simultaneously an essential component of effective reactor design, and a great research barrier. Tokamak operational conditions exhibit comparatively low Knudsen numbers. Kinetic effects, including kinetic waves and instabilities, Landau damping, bump-on-tail instabilities and more, are therefore highly influential in tokamak plasma dynamics. Purely fluid models are inherently incapable of capturing these effects, whereas the high dimensionality in purely kinetic models render them practically intractable for most relevant purposes.

        We consider a $\delta\!f$ decomposition model, with a macroscopic fluid background and microscopic kinetic correction, both fully coupled to each other. A similar manner of discretization is proposed to that used in the recent \texttt{STRUPHY} code \cite{Holderied_Possanner_Wang_2021, Holderied_2022, Li_et_al_2023} with a finite-element model for the background and a pseudo-particle/PiC model for the correction.

        The fluid background satisfies the full, non-linear, resistive, compressible, Hall MHD equations. \cite{Laakmann_Hu_Farrell_2022} introduces finite-element(-in-space) implicit timesteppers for the incompressible analogue to this system with structure-preserving (SP) properties in the ideal case, alongside parameter-robust preconditioners. We show that these timesteppers can derive from a finite-element-in-time (FET) (and finite-element-in-space) interpretation. The benefits of this reformulation are discussed, including the derivation of timesteppers that are higher order in time, and the quantifiable dissipative SP properties in the non-ideal, resistive case.
        
        We discuss possible options for extending this FET approach to timesteppers for the compressible case.

        The kinetic corrections satisfy linearized Boltzmann equations. Using a Lénard--Bernstein collision operator, these take Fokker--Planck-like forms \cite{Fokker_1914, Planck_1917} wherein pseudo-particles in the numerical model obey the neoclassical transport equations, with particle-independent Brownian drift terms. This offers a rigorous methodology for incorporating collisions into the particle transport model, without coupling the equations of motions for each particle.
        
        Works by Chen, Chacón et al. \cite{Chen_Chacón_Barnes_2011, Chacón_Chen_Barnes_2013, Chen_Chacón_2014, Chen_Chacón_2015} have developed structure-preserving particle pushers for neoclassical transport in the Vlasov equations, derived from Crank--Nicolson integrators. We show these too can can derive from a FET interpretation, similarly offering potential extensions to higher-order-in-time particle pushers. The FET formulation is used also to consider how the stochastic drift terms can be incorporated into the pushers. Stochastic gyrokinetic expansions are also discussed.

        Different options for the numerical implementation of these schemes are considered.

        Due to the efficacy of FET in the development of SP timesteppers for both the fluid and kinetic component, we hope this approach will prove effective in the future for developing SP timesteppers for the full hybrid model. We hope this will give us the opportunity to incorporate previously inaccessible kinetic effects into the highly effective, modern, finite-element MHD models.
    \end{abstract}
    
    
    \newpage
    \tableofcontents
    
    
    \newpage
    \pagenumbering{arabic}
    %\linenumbers\renewcommand\thelinenumber{\color{black!50}\arabic{linenumber}}
            \input{0 - introduction/main.tex}
        \part{Research}
            \input{1 - low-noise PiC models/main.tex}
            \input{2 - kinetic component/main.tex}
            \input{3 - fluid component/main.tex}
            \input{4 - numerical implementation/main.tex}
        \part{Project Overview}
            \input{5 - research plan/main.tex}
            \input{6 - summary/main.tex}
    
    
    %\section{}
    \newpage
    \pagenumbering{gobble}
        \printbibliography


    \newpage
    \pagenumbering{roman}
    \appendix
        \part{Appendices}
            \input{8 - Hilbert complexes/main.tex}
            \input{9 - weak conservation proofs/main.tex}
\end{document}

            \documentclass[12pt, a4paper]{report}

\input{template/main.tex}

\title{\BA{Title in Progress...}}
\author{Boris Andrews}
\affil{Mathematical Institute, University of Oxford}
\date{\today}


\begin{document}
    \pagenumbering{gobble}
    \maketitle
    
    
    \begin{abstract}
        Magnetic confinement reactors---in particular tokamaks---offer one of the most promising options for achieving practical nuclear fusion, with the potential to provide virtually limitless, clean energy. The theoretical and numerical modeling of tokamak plasmas is simultaneously an essential component of effective reactor design, and a great research barrier. Tokamak operational conditions exhibit comparatively low Knudsen numbers. Kinetic effects, including kinetic waves and instabilities, Landau damping, bump-on-tail instabilities and more, are therefore highly influential in tokamak plasma dynamics. Purely fluid models are inherently incapable of capturing these effects, whereas the high dimensionality in purely kinetic models render them practically intractable for most relevant purposes.

        We consider a $\delta\!f$ decomposition model, with a macroscopic fluid background and microscopic kinetic correction, both fully coupled to each other. A similar manner of discretization is proposed to that used in the recent \texttt{STRUPHY} code \cite{Holderied_Possanner_Wang_2021, Holderied_2022, Li_et_al_2023} with a finite-element model for the background and a pseudo-particle/PiC model for the correction.

        The fluid background satisfies the full, non-linear, resistive, compressible, Hall MHD equations. \cite{Laakmann_Hu_Farrell_2022} introduces finite-element(-in-space) implicit timesteppers for the incompressible analogue to this system with structure-preserving (SP) properties in the ideal case, alongside parameter-robust preconditioners. We show that these timesteppers can derive from a finite-element-in-time (FET) (and finite-element-in-space) interpretation. The benefits of this reformulation are discussed, including the derivation of timesteppers that are higher order in time, and the quantifiable dissipative SP properties in the non-ideal, resistive case.
        
        We discuss possible options for extending this FET approach to timesteppers for the compressible case.

        The kinetic corrections satisfy linearized Boltzmann equations. Using a Lénard--Bernstein collision operator, these take Fokker--Planck-like forms \cite{Fokker_1914, Planck_1917} wherein pseudo-particles in the numerical model obey the neoclassical transport equations, with particle-independent Brownian drift terms. This offers a rigorous methodology for incorporating collisions into the particle transport model, without coupling the equations of motions for each particle.
        
        Works by Chen, Chacón et al. \cite{Chen_Chacón_Barnes_2011, Chacón_Chen_Barnes_2013, Chen_Chacón_2014, Chen_Chacón_2015} have developed structure-preserving particle pushers for neoclassical transport in the Vlasov equations, derived from Crank--Nicolson integrators. We show these too can can derive from a FET interpretation, similarly offering potential extensions to higher-order-in-time particle pushers. The FET formulation is used also to consider how the stochastic drift terms can be incorporated into the pushers. Stochastic gyrokinetic expansions are also discussed.

        Different options for the numerical implementation of these schemes are considered.

        Due to the efficacy of FET in the development of SP timesteppers for both the fluid and kinetic component, we hope this approach will prove effective in the future for developing SP timesteppers for the full hybrid model. We hope this will give us the opportunity to incorporate previously inaccessible kinetic effects into the highly effective, modern, finite-element MHD models.
    \end{abstract}
    
    
    \newpage
    \tableofcontents
    
    
    \newpage
    \pagenumbering{arabic}
    %\linenumbers\renewcommand\thelinenumber{\color{black!50}\arabic{linenumber}}
            \input{0 - introduction/main.tex}
        \part{Research}
            \input{1 - low-noise PiC models/main.tex}
            \input{2 - kinetic component/main.tex}
            \input{3 - fluid component/main.tex}
            \input{4 - numerical implementation/main.tex}
        \part{Project Overview}
            \input{5 - research plan/main.tex}
            \input{6 - summary/main.tex}
    
    
    %\section{}
    \newpage
    \pagenumbering{gobble}
        \printbibliography


    \newpage
    \pagenumbering{roman}
    \appendix
        \part{Appendices}
            \input{8 - Hilbert complexes/main.tex}
            \input{9 - weak conservation proofs/main.tex}
\end{document}

            \documentclass[12pt, a4paper]{report}

\input{template/main.tex}

\title{\BA{Title in Progress...}}
\author{Boris Andrews}
\affil{Mathematical Institute, University of Oxford}
\date{\today}


\begin{document}
    \pagenumbering{gobble}
    \maketitle
    
    
    \begin{abstract}
        Magnetic confinement reactors---in particular tokamaks---offer one of the most promising options for achieving practical nuclear fusion, with the potential to provide virtually limitless, clean energy. The theoretical and numerical modeling of tokamak plasmas is simultaneously an essential component of effective reactor design, and a great research barrier. Tokamak operational conditions exhibit comparatively low Knudsen numbers. Kinetic effects, including kinetic waves and instabilities, Landau damping, bump-on-tail instabilities and more, are therefore highly influential in tokamak plasma dynamics. Purely fluid models are inherently incapable of capturing these effects, whereas the high dimensionality in purely kinetic models render them practically intractable for most relevant purposes.

        We consider a $\delta\!f$ decomposition model, with a macroscopic fluid background and microscopic kinetic correction, both fully coupled to each other. A similar manner of discretization is proposed to that used in the recent \texttt{STRUPHY} code \cite{Holderied_Possanner_Wang_2021, Holderied_2022, Li_et_al_2023} with a finite-element model for the background and a pseudo-particle/PiC model for the correction.

        The fluid background satisfies the full, non-linear, resistive, compressible, Hall MHD equations. \cite{Laakmann_Hu_Farrell_2022} introduces finite-element(-in-space) implicit timesteppers for the incompressible analogue to this system with structure-preserving (SP) properties in the ideal case, alongside parameter-robust preconditioners. We show that these timesteppers can derive from a finite-element-in-time (FET) (and finite-element-in-space) interpretation. The benefits of this reformulation are discussed, including the derivation of timesteppers that are higher order in time, and the quantifiable dissipative SP properties in the non-ideal, resistive case.
        
        We discuss possible options for extending this FET approach to timesteppers for the compressible case.

        The kinetic corrections satisfy linearized Boltzmann equations. Using a Lénard--Bernstein collision operator, these take Fokker--Planck-like forms \cite{Fokker_1914, Planck_1917} wherein pseudo-particles in the numerical model obey the neoclassical transport equations, with particle-independent Brownian drift terms. This offers a rigorous methodology for incorporating collisions into the particle transport model, without coupling the equations of motions for each particle.
        
        Works by Chen, Chacón et al. \cite{Chen_Chacón_Barnes_2011, Chacón_Chen_Barnes_2013, Chen_Chacón_2014, Chen_Chacón_2015} have developed structure-preserving particle pushers for neoclassical transport in the Vlasov equations, derived from Crank--Nicolson integrators. We show these too can can derive from a FET interpretation, similarly offering potential extensions to higher-order-in-time particle pushers. The FET formulation is used also to consider how the stochastic drift terms can be incorporated into the pushers. Stochastic gyrokinetic expansions are also discussed.

        Different options for the numerical implementation of these schemes are considered.

        Due to the efficacy of FET in the development of SP timesteppers for both the fluid and kinetic component, we hope this approach will prove effective in the future for developing SP timesteppers for the full hybrid model. We hope this will give us the opportunity to incorporate previously inaccessible kinetic effects into the highly effective, modern, finite-element MHD models.
    \end{abstract}
    
    
    \newpage
    \tableofcontents
    
    
    \newpage
    \pagenumbering{arabic}
    %\linenumbers\renewcommand\thelinenumber{\color{black!50}\arabic{linenumber}}
            \input{0 - introduction/main.tex}
        \part{Research}
            \input{1 - low-noise PiC models/main.tex}
            \input{2 - kinetic component/main.tex}
            \input{3 - fluid component/main.tex}
            \input{4 - numerical implementation/main.tex}
        \part{Project Overview}
            \input{5 - research plan/main.tex}
            \input{6 - summary/main.tex}
    
    
    %\section{}
    \newpage
    \pagenumbering{gobble}
        \printbibliography


    \newpage
    \pagenumbering{roman}
    \appendix
        \part{Appendices}
            \input{8 - Hilbert complexes/main.tex}
            \input{9 - weak conservation proofs/main.tex}
\end{document}

            \documentclass[12pt, a4paper]{report}

\input{template/main.tex}

\title{\BA{Title in Progress...}}
\author{Boris Andrews}
\affil{Mathematical Institute, University of Oxford}
\date{\today}


\begin{document}
    \pagenumbering{gobble}
    \maketitle
    
    
    \begin{abstract}
        Magnetic confinement reactors---in particular tokamaks---offer one of the most promising options for achieving practical nuclear fusion, with the potential to provide virtually limitless, clean energy. The theoretical and numerical modeling of tokamak plasmas is simultaneously an essential component of effective reactor design, and a great research barrier. Tokamak operational conditions exhibit comparatively low Knudsen numbers. Kinetic effects, including kinetic waves and instabilities, Landau damping, bump-on-tail instabilities and more, are therefore highly influential in tokamak plasma dynamics. Purely fluid models are inherently incapable of capturing these effects, whereas the high dimensionality in purely kinetic models render them practically intractable for most relevant purposes.

        We consider a $\delta\!f$ decomposition model, with a macroscopic fluid background and microscopic kinetic correction, both fully coupled to each other. A similar manner of discretization is proposed to that used in the recent \texttt{STRUPHY} code \cite{Holderied_Possanner_Wang_2021, Holderied_2022, Li_et_al_2023} with a finite-element model for the background and a pseudo-particle/PiC model for the correction.

        The fluid background satisfies the full, non-linear, resistive, compressible, Hall MHD equations. \cite{Laakmann_Hu_Farrell_2022} introduces finite-element(-in-space) implicit timesteppers for the incompressible analogue to this system with structure-preserving (SP) properties in the ideal case, alongside parameter-robust preconditioners. We show that these timesteppers can derive from a finite-element-in-time (FET) (and finite-element-in-space) interpretation. The benefits of this reformulation are discussed, including the derivation of timesteppers that are higher order in time, and the quantifiable dissipative SP properties in the non-ideal, resistive case.
        
        We discuss possible options for extending this FET approach to timesteppers for the compressible case.

        The kinetic corrections satisfy linearized Boltzmann equations. Using a Lénard--Bernstein collision operator, these take Fokker--Planck-like forms \cite{Fokker_1914, Planck_1917} wherein pseudo-particles in the numerical model obey the neoclassical transport equations, with particle-independent Brownian drift terms. This offers a rigorous methodology for incorporating collisions into the particle transport model, without coupling the equations of motions for each particle.
        
        Works by Chen, Chacón et al. \cite{Chen_Chacón_Barnes_2011, Chacón_Chen_Barnes_2013, Chen_Chacón_2014, Chen_Chacón_2015} have developed structure-preserving particle pushers for neoclassical transport in the Vlasov equations, derived from Crank--Nicolson integrators. We show these too can can derive from a FET interpretation, similarly offering potential extensions to higher-order-in-time particle pushers. The FET formulation is used also to consider how the stochastic drift terms can be incorporated into the pushers. Stochastic gyrokinetic expansions are also discussed.

        Different options for the numerical implementation of these schemes are considered.

        Due to the efficacy of FET in the development of SP timesteppers for both the fluid and kinetic component, we hope this approach will prove effective in the future for developing SP timesteppers for the full hybrid model. We hope this will give us the opportunity to incorporate previously inaccessible kinetic effects into the highly effective, modern, finite-element MHD models.
    \end{abstract}
    
    
    \newpage
    \tableofcontents
    
    
    \newpage
    \pagenumbering{arabic}
    %\linenumbers\renewcommand\thelinenumber{\color{black!50}\arabic{linenumber}}
            \input{0 - introduction/main.tex}
        \part{Research}
            \input{1 - low-noise PiC models/main.tex}
            \input{2 - kinetic component/main.tex}
            \input{3 - fluid component/main.tex}
            \input{4 - numerical implementation/main.tex}
        \part{Project Overview}
            \input{5 - research plan/main.tex}
            \input{6 - summary/main.tex}
    
    
    %\section{}
    \newpage
    \pagenumbering{gobble}
        \printbibliography


    \newpage
    \pagenumbering{roman}
    \appendix
        \part{Appendices}
            \input{8 - Hilbert complexes/main.tex}
            \input{9 - weak conservation proofs/main.tex}
\end{document}

        \part{Project Overview}
            \documentclass[12pt, a4paper]{report}

\input{template/main.tex}

\title{\BA{Title in Progress...}}
\author{Boris Andrews}
\affil{Mathematical Institute, University of Oxford}
\date{\today}


\begin{document}
    \pagenumbering{gobble}
    \maketitle
    
    
    \begin{abstract}
        Magnetic confinement reactors---in particular tokamaks---offer one of the most promising options for achieving practical nuclear fusion, with the potential to provide virtually limitless, clean energy. The theoretical and numerical modeling of tokamak plasmas is simultaneously an essential component of effective reactor design, and a great research barrier. Tokamak operational conditions exhibit comparatively low Knudsen numbers. Kinetic effects, including kinetic waves and instabilities, Landau damping, bump-on-tail instabilities and more, are therefore highly influential in tokamak plasma dynamics. Purely fluid models are inherently incapable of capturing these effects, whereas the high dimensionality in purely kinetic models render them practically intractable for most relevant purposes.

        We consider a $\delta\!f$ decomposition model, with a macroscopic fluid background and microscopic kinetic correction, both fully coupled to each other. A similar manner of discretization is proposed to that used in the recent \texttt{STRUPHY} code \cite{Holderied_Possanner_Wang_2021, Holderied_2022, Li_et_al_2023} with a finite-element model for the background and a pseudo-particle/PiC model for the correction.

        The fluid background satisfies the full, non-linear, resistive, compressible, Hall MHD equations. \cite{Laakmann_Hu_Farrell_2022} introduces finite-element(-in-space) implicit timesteppers for the incompressible analogue to this system with structure-preserving (SP) properties in the ideal case, alongside parameter-robust preconditioners. We show that these timesteppers can derive from a finite-element-in-time (FET) (and finite-element-in-space) interpretation. The benefits of this reformulation are discussed, including the derivation of timesteppers that are higher order in time, and the quantifiable dissipative SP properties in the non-ideal, resistive case.
        
        We discuss possible options for extending this FET approach to timesteppers for the compressible case.

        The kinetic corrections satisfy linearized Boltzmann equations. Using a Lénard--Bernstein collision operator, these take Fokker--Planck-like forms \cite{Fokker_1914, Planck_1917} wherein pseudo-particles in the numerical model obey the neoclassical transport equations, with particle-independent Brownian drift terms. This offers a rigorous methodology for incorporating collisions into the particle transport model, without coupling the equations of motions for each particle.
        
        Works by Chen, Chacón et al. \cite{Chen_Chacón_Barnes_2011, Chacón_Chen_Barnes_2013, Chen_Chacón_2014, Chen_Chacón_2015} have developed structure-preserving particle pushers for neoclassical transport in the Vlasov equations, derived from Crank--Nicolson integrators. We show these too can can derive from a FET interpretation, similarly offering potential extensions to higher-order-in-time particle pushers. The FET formulation is used also to consider how the stochastic drift terms can be incorporated into the pushers. Stochastic gyrokinetic expansions are also discussed.

        Different options for the numerical implementation of these schemes are considered.

        Due to the efficacy of FET in the development of SP timesteppers for both the fluid and kinetic component, we hope this approach will prove effective in the future for developing SP timesteppers for the full hybrid model. We hope this will give us the opportunity to incorporate previously inaccessible kinetic effects into the highly effective, modern, finite-element MHD models.
    \end{abstract}
    
    
    \newpage
    \tableofcontents
    
    
    \newpage
    \pagenumbering{arabic}
    %\linenumbers\renewcommand\thelinenumber{\color{black!50}\arabic{linenumber}}
            \input{0 - introduction/main.tex}
        \part{Research}
            \input{1 - low-noise PiC models/main.tex}
            \input{2 - kinetic component/main.tex}
            \input{3 - fluid component/main.tex}
            \input{4 - numerical implementation/main.tex}
        \part{Project Overview}
            \input{5 - research plan/main.tex}
            \input{6 - summary/main.tex}
    
    
    %\section{}
    \newpage
    \pagenumbering{gobble}
        \printbibliography


    \newpage
    \pagenumbering{roman}
    \appendix
        \part{Appendices}
            \input{8 - Hilbert complexes/main.tex}
            \input{9 - weak conservation proofs/main.tex}
\end{document}

            \documentclass[12pt, a4paper]{report}

\input{template/main.tex}

\title{\BA{Title in Progress...}}
\author{Boris Andrews}
\affil{Mathematical Institute, University of Oxford}
\date{\today}


\begin{document}
    \pagenumbering{gobble}
    \maketitle
    
    
    \begin{abstract}
        Magnetic confinement reactors---in particular tokamaks---offer one of the most promising options for achieving practical nuclear fusion, with the potential to provide virtually limitless, clean energy. The theoretical and numerical modeling of tokamak plasmas is simultaneously an essential component of effective reactor design, and a great research barrier. Tokamak operational conditions exhibit comparatively low Knudsen numbers. Kinetic effects, including kinetic waves and instabilities, Landau damping, bump-on-tail instabilities and more, are therefore highly influential in tokamak plasma dynamics. Purely fluid models are inherently incapable of capturing these effects, whereas the high dimensionality in purely kinetic models render them practically intractable for most relevant purposes.

        We consider a $\delta\!f$ decomposition model, with a macroscopic fluid background and microscopic kinetic correction, both fully coupled to each other. A similar manner of discretization is proposed to that used in the recent \texttt{STRUPHY} code \cite{Holderied_Possanner_Wang_2021, Holderied_2022, Li_et_al_2023} with a finite-element model for the background and a pseudo-particle/PiC model for the correction.

        The fluid background satisfies the full, non-linear, resistive, compressible, Hall MHD equations. \cite{Laakmann_Hu_Farrell_2022} introduces finite-element(-in-space) implicit timesteppers for the incompressible analogue to this system with structure-preserving (SP) properties in the ideal case, alongside parameter-robust preconditioners. We show that these timesteppers can derive from a finite-element-in-time (FET) (and finite-element-in-space) interpretation. The benefits of this reformulation are discussed, including the derivation of timesteppers that are higher order in time, and the quantifiable dissipative SP properties in the non-ideal, resistive case.
        
        We discuss possible options for extending this FET approach to timesteppers for the compressible case.

        The kinetic corrections satisfy linearized Boltzmann equations. Using a Lénard--Bernstein collision operator, these take Fokker--Planck-like forms \cite{Fokker_1914, Planck_1917} wherein pseudo-particles in the numerical model obey the neoclassical transport equations, with particle-independent Brownian drift terms. This offers a rigorous methodology for incorporating collisions into the particle transport model, without coupling the equations of motions for each particle.
        
        Works by Chen, Chacón et al. \cite{Chen_Chacón_Barnes_2011, Chacón_Chen_Barnes_2013, Chen_Chacón_2014, Chen_Chacón_2015} have developed structure-preserving particle pushers for neoclassical transport in the Vlasov equations, derived from Crank--Nicolson integrators. We show these too can can derive from a FET interpretation, similarly offering potential extensions to higher-order-in-time particle pushers. The FET formulation is used also to consider how the stochastic drift terms can be incorporated into the pushers. Stochastic gyrokinetic expansions are also discussed.

        Different options for the numerical implementation of these schemes are considered.

        Due to the efficacy of FET in the development of SP timesteppers for both the fluid and kinetic component, we hope this approach will prove effective in the future for developing SP timesteppers for the full hybrid model. We hope this will give us the opportunity to incorporate previously inaccessible kinetic effects into the highly effective, modern, finite-element MHD models.
    \end{abstract}
    
    
    \newpage
    \tableofcontents
    
    
    \newpage
    \pagenumbering{arabic}
    %\linenumbers\renewcommand\thelinenumber{\color{black!50}\arabic{linenumber}}
            \input{0 - introduction/main.tex}
        \part{Research}
            \input{1 - low-noise PiC models/main.tex}
            \input{2 - kinetic component/main.tex}
            \input{3 - fluid component/main.tex}
            \input{4 - numerical implementation/main.tex}
        \part{Project Overview}
            \input{5 - research plan/main.tex}
            \input{6 - summary/main.tex}
    
    
    %\section{}
    \newpage
    \pagenumbering{gobble}
        \printbibliography


    \newpage
    \pagenumbering{roman}
    \appendix
        \part{Appendices}
            \input{8 - Hilbert complexes/main.tex}
            \input{9 - weak conservation proofs/main.tex}
\end{document}

    
    
    %\section{}
    \newpage
    \pagenumbering{gobble}
        \printbibliography


    \newpage
    \pagenumbering{roman}
    \appendix
        \part{Appendices}
            \documentclass[12pt, a4paper]{report}

\input{template/main.tex}

\title{\BA{Title in Progress...}}
\author{Boris Andrews}
\affil{Mathematical Institute, University of Oxford}
\date{\today}


\begin{document}
    \pagenumbering{gobble}
    \maketitle
    
    
    \begin{abstract}
        Magnetic confinement reactors---in particular tokamaks---offer one of the most promising options for achieving practical nuclear fusion, with the potential to provide virtually limitless, clean energy. The theoretical and numerical modeling of tokamak plasmas is simultaneously an essential component of effective reactor design, and a great research barrier. Tokamak operational conditions exhibit comparatively low Knudsen numbers. Kinetic effects, including kinetic waves and instabilities, Landau damping, bump-on-tail instabilities and more, are therefore highly influential in tokamak plasma dynamics. Purely fluid models are inherently incapable of capturing these effects, whereas the high dimensionality in purely kinetic models render them practically intractable for most relevant purposes.

        We consider a $\delta\!f$ decomposition model, with a macroscopic fluid background and microscopic kinetic correction, both fully coupled to each other. A similar manner of discretization is proposed to that used in the recent \texttt{STRUPHY} code \cite{Holderied_Possanner_Wang_2021, Holderied_2022, Li_et_al_2023} with a finite-element model for the background and a pseudo-particle/PiC model for the correction.

        The fluid background satisfies the full, non-linear, resistive, compressible, Hall MHD equations. \cite{Laakmann_Hu_Farrell_2022} introduces finite-element(-in-space) implicit timesteppers for the incompressible analogue to this system with structure-preserving (SP) properties in the ideal case, alongside parameter-robust preconditioners. We show that these timesteppers can derive from a finite-element-in-time (FET) (and finite-element-in-space) interpretation. The benefits of this reformulation are discussed, including the derivation of timesteppers that are higher order in time, and the quantifiable dissipative SP properties in the non-ideal, resistive case.
        
        We discuss possible options for extending this FET approach to timesteppers for the compressible case.

        The kinetic corrections satisfy linearized Boltzmann equations. Using a Lénard--Bernstein collision operator, these take Fokker--Planck-like forms \cite{Fokker_1914, Planck_1917} wherein pseudo-particles in the numerical model obey the neoclassical transport equations, with particle-independent Brownian drift terms. This offers a rigorous methodology for incorporating collisions into the particle transport model, without coupling the equations of motions for each particle.
        
        Works by Chen, Chacón et al. \cite{Chen_Chacón_Barnes_2011, Chacón_Chen_Barnes_2013, Chen_Chacón_2014, Chen_Chacón_2015} have developed structure-preserving particle pushers for neoclassical transport in the Vlasov equations, derived from Crank--Nicolson integrators. We show these too can can derive from a FET interpretation, similarly offering potential extensions to higher-order-in-time particle pushers. The FET formulation is used also to consider how the stochastic drift terms can be incorporated into the pushers. Stochastic gyrokinetic expansions are also discussed.

        Different options for the numerical implementation of these schemes are considered.

        Due to the efficacy of FET in the development of SP timesteppers for both the fluid and kinetic component, we hope this approach will prove effective in the future for developing SP timesteppers for the full hybrid model. We hope this will give us the opportunity to incorporate previously inaccessible kinetic effects into the highly effective, modern, finite-element MHD models.
    \end{abstract}
    
    
    \newpage
    \tableofcontents
    
    
    \newpage
    \pagenumbering{arabic}
    %\linenumbers\renewcommand\thelinenumber{\color{black!50}\arabic{linenumber}}
            \input{0 - introduction/main.tex}
        \part{Research}
            \input{1 - low-noise PiC models/main.tex}
            \input{2 - kinetic component/main.tex}
            \input{3 - fluid component/main.tex}
            \input{4 - numerical implementation/main.tex}
        \part{Project Overview}
            \input{5 - research plan/main.tex}
            \input{6 - summary/main.tex}
    
    
    %\section{}
    \newpage
    \pagenumbering{gobble}
        \printbibliography


    \newpage
    \pagenumbering{roman}
    \appendix
        \part{Appendices}
            \input{8 - Hilbert complexes/main.tex}
            \input{9 - weak conservation proofs/main.tex}
\end{document}

            \documentclass[12pt, a4paper]{report}

\input{template/main.tex}

\title{\BA{Title in Progress...}}
\author{Boris Andrews}
\affil{Mathematical Institute, University of Oxford}
\date{\today}


\begin{document}
    \pagenumbering{gobble}
    \maketitle
    
    
    \begin{abstract}
        Magnetic confinement reactors---in particular tokamaks---offer one of the most promising options for achieving practical nuclear fusion, with the potential to provide virtually limitless, clean energy. The theoretical and numerical modeling of tokamak plasmas is simultaneously an essential component of effective reactor design, and a great research barrier. Tokamak operational conditions exhibit comparatively low Knudsen numbers. Kinetic effects, including kinetic waves and instabilities, Landau damping, bump-on-tail instabilities and more, are therefore highly influential in tokamak plasma dynamics. Purely fluid models are inherently incapable of capturing these effects, whereas the high dimensionality in purely kinetic models render them practically intractable for most relevant purposes.

        We consider a $\delta\!f$ decomposition model, with a macroscopic fluid background and microscopic kinetic correction, both fully coupled to each other. A similar manner of discretization is proposed to that used in the recent \texttt{STRUPHY} code \cite{Holderied_Possanner_Wang_2021, Holderied_2022, Li_et_al_2023} with a finite-element model for the background and a pseudo-particle/PiC model for the correction.

        The fluid background satisfies the full, non-linear, resistive, compressible, Hall MHD equations. \cite{Laakmann_Hu_Farrell_2022} introduces finite-element(-in-space) implicit timesteppers for the incompressible analogue to this system with structure-preserving (SP) properties in the ideal case, alongside parameter-robust preconditioners. We show that these timesteppers can derive from a finite-element-in-time (FET) (and finite-element-in-space) interpretation. The benefits of this reformulation are discussed, including the derivation of timesteppers that are higher order in time, and the quantifiable dissipative SP properties in the non-ideal, resistive case.
        
        We discuss possible options for extending this FET approach to timesteppers for the compressible case.

        The kinetic corrections satisfy linearized Boltzmann equations. Using a Lénard--Bernstein collision operator, these take Fokker--Planck-like forms \cite{Fokker_1914, Planck_1917} wherein pseudo-particles in the numerical model obey the neoclassical transport equations, with particle-independent Brownian drift terms. This offers a rigorous methodology for incorporating collisions into the particle transport model, without coupling the equations of motions for each particle.
        
        Works by Chen, Chacón et al. \cite{Chen_Chacón_Barnes_2011, Chacón_Chen_Barnes_2013, Chen_Chacón_2014, Chen_Chacón_2015} have developed structure-preserving particle pushers for neoclassical transport in the Vlasov equations, derived from Crank--Nicolson integrators. We show these too can can derive from a FET interpretation, similarly offering potential extensions to higher-order-in-time particle pushers. The FET formulation is used also to consider how the stochastic drift terms can be incorporated into the pushers. Stochastic gyrokinetic expansions are also discussed.

        Different options for the numerical implementation of these schemes are considered.

        Due to the efficacy of FET in the development of SP timesteppers for both the fluid and kinetic component, we hope this approach will prove effective in the future for developing SP timesteppers for the full hybrid model. We hope this will give us the opportunity to incorporate previously inaccessible kinetic effects into the highly effective, modern, finite-element MHD models.
    \end{abstract}
    
    
    \newpage
    \tableofcontents
    
    
    \newpage
    \pagenumbering{arabic}
    %\linenumbers\renewcommand\thelinenumber{\color{black!50}\arabic{linenumber}}
            \input{0 - introduction/main.tex}
        \part{Research}
            \input{1 - low-noise PiC models/main.tex}
            \input{2 - kinetic component/main.tex}
            \input{3 - fluid component/main.tex}
            \input{4 - numerical implementation/main.tex}
        \part{Project Overview}
            \input{5 - research plan/main.tex}
            \input{6 - summary/main.tex}
    
    
    %\section{}
    \newpage
    \pagenumbering{gobble}
        \printbibliography


    \newpage
    \pagenumbering{roman}
    \appendix
        \part{Appendices}
            \input{8 - Hilbert complexes/main.tex}
            \input{9 - weak conservation proofs/main.tex}
\end{document}

\end{document}

            \documentclass[12pt, a4paper]{report}

\documentclass[12pt, a4paper]{report}

\input{template/main.tex}

\title{\BA{Title in Progress...}}
\author{Boris Andrews}
\affil{Mathematical Institute, University of Oxford}
\date{\today}


\begin{document}
    \pagenumbering{gobble}
    \maketitle
    
    
    \begin{abstract}
        Magnetic confinement reactors---in particular tokamaks---offer one of the most promising options for achieving practical nuclear fusion, with the potential to provide virtually limitless, clean energy. The theoretical and numerical modeling of tokamak plasmas is simultaneously an essential component of effective reactor design, and a great research barrier. Tokamak operational conditions exhibit comparatively low Knudsen numbers. Kinetic effects, including kinetic waves and instabilities, Landau damping, bump-on-tail instabilities and more, are therefore highly influential in tokamak plasma dynamics. Purely fluid models are inherently incapable of capturing these effects, whereas the high dimensionality in purely kinetic models render them practically intractable for most relevant purposes.

        We consider a $\delta\!f$ decomposition model, with a macroscopic fluid background and microscopic kinetic correction, both fully coupled to each other. A similar manner of discretization is proposed to that used in the recent \texttt{STRUPHY} code \cite{Holderied_Possanner_Wang_2021, Holderied_2022, Li_et_al_2023} with a finite-element model for the background and a pseudo-particle/PiC model for the correction.

        The fluid background satisfies the full, non-linear, resistive, compressible, Hall MHD equations. \cite{Laakmann_Hu_Farrell_2022} introduces finite-element(-in-space) implicit timesteppers for the incompressible analogue to this system with structure-preserving (SP) properties in the ideal case, alongside parameter-robust preconditioners. We show that these timesteppers can derive from a finite-element-in-time (FET) (and finite-element-in-space) interpretation. The benefits of this reformulation are discussed, including the derivation of timesteppers that are higher order in time, and the quantifiable dissipative SP properties in the non-ideal, resistive case.
        
        We discuss possible options for extending this FET approach to timesteppers for the compressible case.

        The kinetic corrections satisfy linearized Boltzmann equations. Using a Lénard--Bernstein collision operator, these take Fokker--Planck-like forms \cite{Fokker_1914, Planck_1917} wherein pseudo-particles in the numerical model obey the neoclassical transport equations, with particle-independent Brownian drift terms. This offers a rigorous methodology for incorporating collisions into the particle transport model, without coupling the equations of motions for each particle.
        
        Works by Chen, Chacón et al. \cite{Chen_Chacón_Barnes_2011, Chacón_Chen_Barnes_2013, Chen_Chacón_2014, Chen_Chacón_2015} have developed structure-preserving particle pushers for neoclassical transport in the Vlasov equations, derived from Crank--Nicolson integrators. We show these too can can derive from a FET interpretation, similarly offering potential extensions to higher-order-in-time particle pushers. The FET formulation is used also to consider how the stochastic drift terms can be incorporated into the pushers. Stochastic gyrokinetic expansions are also discussed.

        Different options for the numerical implementation of these schemes are considered.

        Due to the efficacy of FET in the development of SP timesteppers for both the fluid and kinetic component, we hope this approach will prove effective in the future for developing SP timesteppers for the full hybrid model. We hope this will give us the opportunity to incorporate previously inaccessible kinetic effects into the highly effective, modern, finite-element MHD models.
    \end{abstract}
    
    
    \newpage
    \tableofcontents
    
    
    \newpage
    \pagenumbering{arabic}
    %\linenumbers\renewcommand\thelinenumber{\color{black!50}\arabic{linenumber}}
            \input{0 - introduction/main.tex}
        \part{Research}
            \input{1 - low-noise PiC models/main.tex}
            \input{2 - kinetic component/main.tex}
            \input{3 - fluid component/main.tex}
            \input{4 - numerical implementation/main.tex}
        \part{Project Overview}
            \input{5 - research plan/main.tex}
            \input{6 - summary/main.tex}
    
    
    %\section{}
    \newpage
    \pagenumbering{gobble}
        \printbibliography


    \newpage
    \pagenumbering{roman}
    \appendix
        \part{Appendices}
            \input{8 - Hilbert complexes/main.tex}
            \input{9 - weak conservation proofs/main.tex}
\end{document}


\title{\BA{Title in Progress...}}
\author{Boris Andrews}
\affil{Mathematical Institute, University of Oxford}
\date{\today}


\begin{document}
    \pagenumbering{gobble}
    \maketitle
    
    
    \begin{abstract}
        Magnetic confinement reactors---in particular tokamaks---offer one of the most promising options for achieving practical nuclear fusion, with the potential to provide virtually limitless, clean energy. The theoretical and numerical modeling of tokamak plasmas is simultaneously an essential component of effective reactor design, and a great research barrier. Tokamak operational conditions exhibit comparatively low Knudsen numbers. Kinetic effects, including kinetic waves and instabilities, Landau damping, bump-on-tail instabilities and more, are therefore highly influential in tokamak plasma dynamics. Purely fluid models are inherently incapable of capturing these effects, whereas the high dimensionality in purely kinetic models render them practically intractable for most relevant purposes.

        We consider a $\delta\!f$ decomposition model, with a macroscopic fluid background and microscopic kinetic correction, both fully coupled to each other. A similar manner of discretization is proposed to that used in the recent \texttt{STRUPHY} code \cite{Holderied_Possanner_Wang_2021, Holderied_2022, Li_et_al_2023} with a finite-element model for the background and a pseudo-particle/PiC model for the correction.

        The fluid background satisfies the full, non-linear, resistive, compressible, Hall MHD equations. \cite{Laakmann_Hu_Farrell_2022} introduces finite-element(-in-space) implicit timesteppers for the incompressible analogue to this system with structure-preserving (SP) properties in the ideal case, alongside parameter-robust preconditioners. We show that these timesteppers can derive from a finite-element-in-time (FET) (and finite-element-in-space) interpretation. The benefits of this reformulation are discussed, including the derivation of timesteppers that are higher order in time, and the quantifiable dissipative SP properties in the non-ideal, resistive case.
        
        We discuss possible options for extending this FET approach to timesteppers for the compressible case.

        The kinetic corrections satisfy linearized Boltzmann equations. Using a Lénard--Bernstein collision operator, these take Fokker--Planck-like forms \cite{Fokker_1914, Planck_1917} wherein pseudo-particles in the numerical model obey the neoclassical transport equations, with particle-independent Brownian drift terms. This offers a rigorous methodology for incorporating collisions into the particle transport model, without coupling the equations of motions for each particle.
        
        Works by Chen, Chacón et al. \cite{Chen_Chacón_Barnes_2011, Chacón_Chen_Barnes_2013, Chen_Chacón_2014, Chen_Chacón_2015} have developed structure-preserving particle pushers for neoclassical transport in the Vlasov equations, derived from Crank--Nicolson integrators. We show these too can can derive from a FET interpretation, similarly offering potential extensions to higher-order-in-time particle pushers. The FET formulation is used also to consider how the stochastic drift terms can be incorporated into the pushers. Stochastic gyrokinetic expansions are also discussed.

        Different options for the numerical implementation of these schemes are considered.

        Due to the efficacy of FET in the development of SP timesteppers for both the fluid and kinetic component, we hope this approach will prove effective in the future for developing SP timesteppers for the full hybrid model. We hope this will give us the opportunity to incorporate previously inaccessible kinetic effects into the highly effective, modern, finite-element MHD models.
    \end{abstract}
    
    
    \newpage
    \tableofcontents
    
    
    \newpage
    \pagenumbering{arabic}
    %\linenumbers\renewcommand\thelinenumber{\color{black!50}\arabic{linenumber}}
            \documentclass[12pt, a4paper]{report}

\input{template/main.tex}

\title{\BA{Title in Progress...}}
\author{Boris Andrews}
\affil{Mathematical Institute, University of Oxford}
\date{\today}


\begin{document}
    \pagenumbering{gobble}
    \maketitle
    
    
    \begin{abstract}
        Magnetic confinement reactors---in particular tokamaks---offer one of the most promising options for achieving practical nuclear fusion, with the potential to provide virtually limitless, clean energy. The theoretical and numerical modeling of tokamak plasmas is simultaneously an essential component of effective reactor design, and a great research barrier. Tokamak operational conditions exhibit comparatively low Knudsen numbers. Kinetic effects, including kinetic waves and instabilities, Landau damping, bump-on-tail instabilities and more, are therefore highly influential in tokamak plasma dynamics. Purely fluid models are inherently incapable of capturing these effects, whereas the high dimensionality in purely kinetic models render them practically intractable for most relevant purposes.

        We consider a $\delta\!f$ decomposition model, with a macroscopic fluid background and microscopic kinetic correction, both fully coupled to each other. A similar manner of discretization is proposed to that used in the recent \texttt{STRUPHY} code \cite{Holderied_Possanner_Wang_2021, Holderied_2022, Li_et_al_2023} with a finite-element model for the background and a pseudo-particle/PiC model for the correction.

        The fluid background satisfies the full, non-linear, resistive, compressible, Hall MHD equations. \cite{Laakmann_Hu_Farrell_2022} introduces finite-element(-in-space) implicit timesteppers for the incompressible analogue to this system with structure-preserving (SP) properties in the ideal case, alongside parameter-robust preconditioners. We show that these timesteppers can derive from a finite-element-in-time (FET) (and finite-element-in-space) interpretation. The benefits of this reformulation are discussed, including the derivation of timesteppers that are higher order in time, and the quantifiable dissipative SP properties in the non-ideal, resistive case.
        
        We discuss possible options for extending this FET approach to timesteppers for the compressible case.

        The kinetic corrections satisfy linearized Boltzmann equations. Using a Lénard--Bernstein collision operator, these take Fokker--Planck-like forms \cite{Fokker_1914, Planck_1917} wherein pseudo-particles in the numerical model obey the neoclassical transport equations, with particle-independent Brownian drift terms. This offers a rigorous methodology for incorporating collisions into the particle transport model, without coupling the equations of motions for each particle.
        
        Works by Chen, Chacón et al. \cite{Chen_Chacón_Barnes_2011, Chacón_Chen_Barnes_2013, Chen_Chacón_2014, Chen_Chacón_2015} have developed structure-preserving particle pushers for neoclassical transport in the Vlasov equations, derived from Crank--Nicolson integrators. We show these too can can derive from a FET interpretation, similarly offering potential extensions to higher-order-in-time particle pushers. The FET formulation is used also to consider how the stochastic drift terms can be incorporated into the pushers. Stochastic gyrokinetic expansions are also discussed.

        Different options for the numerical implementation of these schemes are considered.

        Due to the efficacy of FET in the development of SP timesteppers for both the fluid and kinetic component, we hope this approach will prove effective in the future for developing SP timesteppers for the full hybrid model. We hope this will give us the opportunity to incorporate previously inaccessible kinetic effects into the highly effective, modern, finite-element MHD models.
    \end{abstract}
    
    
    \newpage
    \tableofcontents
    
    
    \newpage
    \pagenumbering{arabic}
    %\linenumbers\renewcommand\thelinenumber{\color{black!50}\arabic{linenumber}}
            \input{0 - introduction/main.tex}
        \part{Research}
            \input{1 - low-noise PiC models/main.tex}
            \input{2 - kinetic component/main.tex}
            \input{3 - fluid component/main.tex}
            \input{4 - numerical implementation/main.tex}
        \part{Project Overview}
            \input{5 - research plan/main.tex}
            \input{6 - summary/main.tex}
    
    
    %\section{}
    \newpage
    \pagenumbering{gobble}
        \printbibliography


    \newpage
    \pagenumbering{roman}
    \appendix
        \part{Appendices}
            \input{8 - Hilbert complexes/main.tex}
            \input{9 - weak conservation proofs/main.tex}
\end{document}

        \part{Research}
            \documentclass[12pt, a4paper]{report}

\input{template/main.tex}

\title{\BA{Title in Progress...}}
\author{Boris Andrews}
\affil{Mathematical Institute, University of Oxford}
\date{\today}


\begin{document}
    \pagenumbering{gobble}
    \maketitle
    
    
    \begin{abstract}
        Magnetic confinement reactors---in particular tokamaks---offer one of the most promising options for achieving practical nuclear fusion, with the potential to provide virtually limitless, clean energy. The theoretical and numerical modeling of tokamak plasmas is simultaneously an essential component of effective reactor design, and a great research barrier. Tokamak operational conditions exhibit comparatively low Knudsen numbers. Kinetic effects, including kinetic waves and instabilities, Landau damping, bump-on-tail instabilities and more, are therefore highly influential in tokamak plasma dynamics. Purely fluid models are inherently incapable of capturing these effects, whereas the high dimensionality in purely kinetic models render them practically intractable for most relevant purposes.

        We consider a $\delta\!f$ decomposition model, with a macroscopic fluid background and microscopic kinetic correction, both fully coupled to each other. A similar manner of discretization is proposed to that used in the recent \texttt{STRUPHY} code \cite{Holderied_Possanner_Wang_2021, Holderied_2022, Li_et_al_2023} with a finite-element model for the background and a pseudo-particle/PiC model for the correction.

        The fluid background satisfies the full, non-linear, resistive, compressible, Hall MHD equations. \cite{Laakmann_Hu_Farrell_2022} introduces finite-element(-in-space) implicit timesteppers for the incompressible analogue to this system with structure-preserving (SP) properties in the ideal case, alongside parameter-robust preconditioners. We show that these timesteppers can derive from a finite-element-in-time (FET) (and finite-element-in-space) interpretation. The benefits of this reformulation are discussed, including the derivation of timesteppers that are higher order in time, and the quantifiable dissipative SP properties in the non-ideal, resistive case.
        
        We discuss possible options for extending this FET approach to timesteppers for the compressible case.

        The kinetic corrections satisfy linearized Boltzmann equations. Using a Lénard--Bernstein collision operator, these take Fokker--Planck-like forms \cite{Fokker_1914, Planck_1917} wherein pseudo-particles in the numerical model obey the neoclassical transport equations, with particle-independent Brownian drift terms. This offers a rigorous methodology for incorporating collisions into the particle transport model, without coupling the equations of motions for each particle.
        
        Works by Chen, Chacón et al. \cite{Chen_Chacón_Barnes_2011, Chacón_Chen_Barnes_2013, Chen_Chacón_2014, Chen_Chacón_2015} have developed structure-preserving particle pushers for neoclassical transport in the Vlasov equations, derived from Crank--Nicolson integrators. We show these too can can derive from a FET interpretation, similarly offering potential extensions to higher-order-in-time particle pushers. The FET formulation is used also to consider how the stochastic drift terms can be incorporated into the pushers. Stochastic gyrokinetic expansions are also discussed.

        Different options for the numerical implementation of these schemes are considered.

        Due to the efficacy of FET in the development of SP timesteppers for both the fluid and kinetic component, we hope this approach will prove effective in the future for developing SP timesteppers for the full hybrid model. We hope this will give us the opportunity to incorporate previously inaccessible kinetic effects into the highly effective, modern, finite-element MHD models.
    \end{abstract}
    
    
    \newpage
    \tableofcontents
    
    
    \newpage
    \pagenumbering{arabic}
    %\linenumbers\renewcommand\thelinenumber{\color{black!50}\arabic{linenumber}}
            \input{0 - introduction/main.tex}
        \part{Research}
            \input{1 - low-noise PiC models/main.tex}
            \input{2 - kinetic component/main.tex}
            \input{3 - fluid component/main.tex}
            \input{4 - numerical implementation/main.tex}
        \part{Project Overview}
            \input{5 - research plan/main.tex}
            \input{6 - summary/main.tex}
    
    
    %\section{}
    \newpage
    \pagenumbering{gobble}
        \printbibliography


    \newpage
    \pagenumbering{roman}
    \appendix
        \part{Appendices}
            \input{8 - Hilbert complexes/main.tex}
            \input{9 - weak conservation proofs/main.tex}
\end{document}

            \documentclass[12pt, a4paper]{report}

\input{template/main.tex}

\title{\BA{Title in Progress...}}
\author{Boris Andrews}
\affil{Mathematical Institute, University of Oxford}
\date{\today}


\begin{document}
    \pagenumbering{gobble}
    \maketitle
    
    
    \begin{abstract}
        Magnetic confinement reactors---in particular tokamaks---offer one of the most promising options for achieving practical nuclear fusion, with the potential to provide virtually limitless, clean energy. The theoretical and numerical modeling of tokamak plasmas is simultaneously an essential component of effective reactor design, and a great research barrier. Tokamak operational conditions exhibit comparatively low Knudsen numbers. Kinetic effects, including kinetic waves and instabilities, Landau damping, bump-on-tail instabilities and more, are therefore highly influential in tokamak plasma dynamics. Purely fluid models are inherently incapable of capturing these effects, whereas the high dimensionality in purely kinetic models render them practically intractable for most relevant purposes.

        We consider a $\delta\!f$ decomposition model, with a macroscopic fluid background and microscopic kinetic correction, both fully coupled to each other. A similar manner of discretization is proposed to that used in the recent \texttt{STRUPHY} code \cite{Holderied_Possanner_Wang_2021, Holderied_2022, Li_et_al_2023} with a finite-element model for the background and a pseudo-particle/PiC model for the correction.

        The fluid background satisfies the full, non-linear, resistive, compressible, Hall MHD equations. \cite{Laakmann_Hu_Farrell_2022} introduces finite-element(-in-space) implicit timesteppers for the incompressible analogue to this system with structure-preserving (SP) properties in the ideal case, alongside parameter-robust preconditioners. We show that these timesteppers can derive from a finite-element-in-time (FET) (and finite-element-in-space) interpretation. The benefits of this reformulation are discussed, including the derivation of timesteppers that are higher order in time, and the quantifiable dissipative SP properties in the non-ideal, resistive case.
        
        We discuss possible options for extending this FET approach to timesteppers for the compressible case.

        The kinetic corrections satisfy linearized Boltzmann equations. Using a Lénard--Bernstein collision operator, these take Fokker--Planck-like forms \cite{Fokker_1914, Planck_1917} wherein pseudo-particles in the numerical model obey the neoclassical transport equations, with particle-independent Brownian drift terms. This offers a rigorous methodology for incorporating collisions into the particle transport model, without coupling the equations of motions for each particle.
        
        Works by Chen, Chacón et al. \cite{Chen_Chacón_Barnes_2011, Chacón_Chen_Barnes_2013, Chen_Chacón_2014, Chen_Chacón_2015} have developed structure-preserving particle pushers for neoclassical transport in the Vlasov equations, derived from Crank--Nicolson integrators. We show these too can can derive from a FET interpretation, similarly offering potential extensions to higher-order-in-time particle pushers. The FET formulation is used also to consider how the stochastic drift terms can be incorporated into the pushers. Stochastic gyrokinetic expansions are also discussed.

        Different options for the numerical implementation of these schemes are considered.

        Due to the efficacy of FET in the development of SP timesteppers for both the fluid and kinetic component, we hope this approach will prove effective in the future for developing SP timesteppers for the full hybrid model. We hope this will give us the opportunity to incorporate previously inaccessible kinetic effects into the highly effective, modern, finite-element MHD models.
    \end{abstract}
    
    
    \newpage
    \tableofcontents
    
    
    \newpage
    \pagenumbering{arabic}
    %\linenumbers\renewcommand\thelinenumber{\color{black!50}\arabic{linenumber}}
            \input{0 - introduction/main.tex}
        \part{Research}
            \input{1 - low-noise PiC models/main.tex}
            \input{2 - kinetic component/main.tex}
            \input{3 - fluid component/main.tex}
            \input{4 - numerical implementation/main.tex}
        \part{Project Overview}
            \input{5 - research plan/main.tex}
            \input{6 - summary/main.tex}
    
    
    %\section{}
    \newpage
    \pagenumbering{gobble}
        \printbibliography


    \newpage
    \pagenumbering{roman}
    \appendix
        \part{Appendices}
            \input{8 - Hilbert complexes/main.tex}
            \input{9 - weak conservation proofs/main.tex}
\end{document}

            \documentclass[12pt, a4paper]{report}

\input{template/main.tex}

\title{\BA{Title in Progress...}}
\author{Boris Andrews}
\affil{Mathematical Institute, University of Oxford}
\date{\today}


\begin{document}
    \pagenumbering{gobble}
    \maketitle
    
    
    \begin{abstract}
        Magnetic confinement reactors---in particular tokamaks---offer one of the most promising options for achieving practical nuclear fusion, with the potential to provide virtually limitless, clean energy. The theoretical and numerical modeling of tokamak plasmas is simultaneously an essential component of effective reactor design, and a great research barrier. Tokamak operational conditions exhibit comparatively low Knudsen numbers. Kinetic effects, including kinetic waves and instabilities, Landau damping, bump-on-tail instabilities and more, are therefore highly influential in tokamak plasma dynamics. Purely fluid models are inherently incapable of capturing these effects, whereas the high dimensionality in purely kinetic models render them practically intractable for most relevant purposes.

        We consider a $\delta\!f$ decomposition model, with a macroscopic fluid background and microscopic kinetic correction, both fully coupled to each other. A similar manner of discretization is proposed to that used in the recent \texttt{STRUPHY} code \cite{Holderied_Possanner_Wang_2021, Holderied_2022, Li_et_al_2023} with a finite-element model for the background and a pseudo-particle/PiC model for the correction.

        The fluid background satisfies the full, non-linear, resistive, compressible, Hall MHD equations. \cite{Laakmann_Hu_Farrell_2022} introduces finite-element(-in-space) implicit timesteppers for the incompressible analogue to this system with structure-preserving (SP) properties in the ideal case, alongside parameter-robust preconditioners. We show that these timesteppers can derive from a finite-element-in-time (FET) (and finite-element-in-space) interpretation. The benefits of this reformulation are discussed, including the derivation of timesteppers that are higher order in time, and the quantifiable dissipative SP properties in the non-ideal, resistive case.
        
        We discuss possible options for extending this FET approach to timesteppers for the compressible case.

        The kinetic corrections satisfy linearized Boltzmann equations. Using a Lénard--Bernstein collision operator, these take Fokker--Planck-like forms \cite{Fokker_1914, Planck_1917} wherein pseudo-particles in the numerical model obey the neoclassical transport equations, with particle-independent Brownian drift terms. This offers a rigorous methodology for incorporating collisions into the particle transport model, without coupling the equations of motions for each particle.
        
        Works by Chen, Chacón et al. \cite{Chen_Chacón_Barnes_2011, Chacón_Chen_Barnes_2013, Chen_Chacón_2014, Chen_Chacón_2015} have developed structure-preserving particle pushers for neoclassical transport in the Vlasov equations, derived from Crank--Nicolson integrators. We show these too can can derive from a FET interpretation, similarly offering potential extensions to higher-order-in-time particle pushers. The FET formulation is used also to consider how the stochastic drift terms can be incorporated into the pushers. Stochastic gyrokinetic expansions are also discussed.

        Different options for the numerical implementation of these schemes are considered.

        Due to the efficacy of FET in the development of SP timesteppers for both the fluid and kinetic component, we hope this approach will prove effective in the future for developing SP timesteppers for the full hybrid model. We hope this will give us the opportunity to incorporate previously inaccessible kinetic effects into the highly effective, modern, finite-element MHD models.
    \end{abstract}
    
    
    \newpage
    \tableofcontents
    
    
    \newpage
    \pagenumbering{arabic}
    %\linenumbers\renewcommand\thelinenumber{\color{black!50}\arabic{linenumber}}
            \input{0 - introduction/main.tex}
        \part{Research}
            \input{1 - low-noise PiC models/main.tex}
            \input{2 - kinetic component/main.tex}
            \input{3 - fluid component/main.tex}
            \input{4 - numerical implementation/main.tex}
        \part{Project Overview}
            \input{5 - research plan/main.tex}
            \input{6 - summary/main.tex}
    
    
    %\section{}
    \newpage
    \pagenumbering{gobble}
        \printbibliography


    \newpage
    \pagenumbering{roman}
    \appendix
        \part{Appendices}
            \input{8 - Hilbert complexes/main.tex}
            \input{9 - weak conservation proofs/main.tex}
\end{document}

            \documentclass[12pt, a4paper]{report}

\input{template/main.tex}

\title{\BA{Title in Progress...}}
\author{Boris Andrews}
\affil{Mathematical Institute, University of Oxford}
\date{\today}


\begin{document}
    \pagenumbering{gobble}
    \maketitle
    
    
    \begin{abstract}
        Magnetic confinement reactors---in particular tokamaks---offer one of the most promising options for achieving practical nuclear fusion, with the potential to provide virtually limitless, clean energy. The theoretical and numerical modeling of tokamak plasmas is simultaneously an essential component of effective reactor design, and a great research barrier. Tokamak operational conditions exhibit comparatively low Knudsen numbers. Kinetic effects, including kinetic waves and instabilities, Landau damping, bump-on-tail instabilities and more, are therefore highly influential in tokamak plasma dynamics. Purely fluid models are inherently incapable of capturing these effects, whereas the high dimensionality in purely kinetic models render them practically intractable for most relevant purposes.

        We consider a $\delta\!f$ decomposition model, with a macroscopic fluid background and microscopic kinetic correction, both fully coupled to each other. A similar manner of discretization is proposed to that used in the recent \texttt{STRUPHY} code \cite{Holderied_Possanner_Wang_2021, Holderied_2022, Li_et_al_2023} with a finite-element model for the background and a pseudo-particle/PiC model for the correction.

        The fluid background satisfies the full, non-linear, resistive, compressible, Hall MHD equations. \cite{Laakmann_Hu_Farrell_2022} introduces finite-element(-in-space) implicit timesteppers for the incompressible analogue to this system with structure-preserving (SP) properties in the ideal case, alongside parameter-robust preconditioners. We show that these timesteppers can derive from a finite-element-in-time (FET) (and finite-element-in-space) interpretation. The benefits of this reformulation are discussed, including the derivation of timesteppers that are higher order in time, and the quantifiable dissipative SP properties in the non-ideal, resistive case.
        
        We discuss possible options for extending this FET approach to timesteppers for the compressible case.

        The kinetic corrections satisfy linearized Boltzmann equations. Using a Lénard--Bernstein collision operator, these take Fokker--Planck-like forms \cite{Fokker_1914, Planck_1917} wherein pseudo-particles in the numerical model obey the neoclassical transport equations, with particle-independent Brownian drift terms. This offers a rigorous methodology for incorporating collisions into the particle transport model, without coupling the equations of motions for each particle.
        
        Works by Chen, Chacón et al. \cite{Chen_Chacón_Barnes_2011, Chacón_Chen_Barnes_2013, Chen_Chacón_2014, Chen_Chacón_2015} have developed structure-preserving particle pushers for neoclassical transport in the Vlasov equations, derived from Crank--Nicolson integrators. We show these too can can derive from a FET interpretation, similarly offering potential extensions to higher-order-in-time particle pushers. The FET formulation is used also to consider how the stochastic drift terms can be incorporated into the pushers. Stochastic gyrokinetic expansions are also discussed.

        Different options for the numerical implementation of these schemes are considered.

        Due to the efficacy of FET in the development of SP timesteppers for both the fluid and kinetic component, we hope this approach will prove effective in the future for developing SP timesteppers for the full hybrid model. We hope this will give us the opportunity to incorporate previously inaccessible kinetic effects into the highly effective, modern, finite-element MHD models.
    \end{abstract}
    
    
    \newpage
    \tableofcontents
    
    
    \newpage
    \pagenumbering{arabic}
    %\linenumbers\renewcommand\thelinenumber{\color{black!50}\arabic{linenumber}}
            \input{0 - introduction/main.tex}
        \part{Research}
            \input{1 - low-noise PiC models/main.tex}
            \input{2 - kinetic component/main.tex}
            \input{3 - fluid component/main.tex}
            \input{4 - numerical implementation/main.tex}
        \part{Project Overview}
            \input{5 - research plan/main.tex}
            \input{6 - summary/main.tex}
    
    
    %\section{}
    \newpage
    \pagenumbering{gobble}
        \printbibliography


    \newpage
    \pagenumbering{roman}
    \appendix
        \part{Appendices}
            \input{8 - Hilbert complexes/main.tex}
            \input{9 - weak conservation proofs/main.tex}
\end{document}

        \part{Project Overview}
            \documentclass[12pt, a4paper]{report}

\input{template/main.tex}

\title{\BA{Title in Progress...}}
\author{Boris Andrews}
\affil{Mathematical Institute, University of Oxford}
\date{\today}


\begin{document}
    \pagenumbering{gobble}
    \maketitle
    
    
    \begin{abstract}
        Magnetic confinement reactors---in particular tokamaks---offer one of the most promising options for achieving practical nuclear fusion, with the potential to provide virtually limitless, clean energy. The theoretical and numerical modeling of tokamak plasmas is simultaneously an essential component of effective reactor design, and a great research barrier. Tokamak operational conditions exhibit comparatively low Knudsen numbers. Kinetic effects, including kinetic waves and instabilities, Landau damping, bump-on-tail instabilities and more, are therefore highly influential in tokamak plasma dynamics. Purely fluid models are inherently incapable of capturing these effects, whereas the high dimensionality in purely kinetic models render them practically intractable for most relevant purposes.

        We consider a $\delta\!f$ decomposition model, with a macroscopic fluid background and microscopic kinetic correction, both fully coupled to each other. A similar manner of discretization is proposed to that used in the recent \texttt{STRUPHY} code \cite{Holderied_Possanner_Wang_2021, Holderied_2022, Li_et_al_2023} with a finite-element model for the background and a pseudo-particle/PiC model for the correction.

        The fluid background satisfies the full, non-linear, resistive, compressible, Hall MHD equations. \cite{Laakmann_Hu_Farrell_2022} introduces finite-element(-in-space) implicit timesteppers for the incompressible analogue to this system with structure-preserving (SP) properties in the ideal case, alongside parameter-robust preconditioners. We show that these timesteppers can derive from a finite-element-in-time (FET) (and finite-element-in-space) interpretation. The benefits of this reformulation are discussed, including the derivation of timesteppers that are higher order in time, and the quantifiable dissipative SP properties in the non-ideal, resistive case.
        
        We discuss possible options for extending this FET approach to timesteppers for the compressible case.

        The kinetic corrections satisfy linearized Boltzmann equations. Using a Lénard--Bernstein collision operator, these take Fokker--Planck-like forms \cite{Fokker_1914, Planck_1917} wherein pseudo-particles in the numerical model obey the neoclassical transport equations, with particle-independent Brownian drift terms. This offers a rigorous methodology for incorporating collisions into the particle transport model, without coupling the equations of motions for each particle.
        
        Works by Chen, Chacón et al. \cite{Chen_Chacón_Barnes_2011, Chacón_Chen_Barnes_2013, Chen_Chacón_2014, Chen_Chacón_2015} have developed structure-preserving particle pushers for neoclassical transport in the Vlasov equations, derived from Crank--Nicolson integrators. We show these too can can derive from a FET interpretation, similarly offering potential extensions to higher-order-in-time particle pushers. The FET formulation is used also to consider how the stochastic drift terms can be incorporated into the pushers. Stochastic gyrokinetic expansions are also discussed.

        Different options for the numerical implementation of these schemes are considered.

        Due to the efficacy of FET in the development of SP timesteppers for both the fluid and kinetic component, we hope this approach will prove effective in the future for developing SP timesteppers for the full hybrid model. We hope this will give us the opportunity to incorporate previously inaccessible kinetic effects into the highly effective, modern, finite-element MHD models.
    \end{abstract}
    
    
    \newpage
    \tableofcontents
    
    
    \newpage
    \pagenumbering{arabic}
    %\linenumbers\renewcommand\thelinenumber{\color{black!50}\arabic{linenumber}}
            \input{0 - introduction/main.tex}
        \part{Research}
            \input{1 - low-noise PiC models/main.tex}
            \input{2 - kinetic component/main.tex}
            \input{3 - fluid component/main.tex}
            \input{4 - numerical implementation/main.tex}
        \part{Project Overview}
            \input{5 - research plan/main.tex}
            \input{6 - summary/main.tex}
    
    
    %\section{}
    \newpage
    \pagenumbering{gobble}
        \printbibliography


    \newpage
    \pagenumbering{roman}
    \appendix
        \part{Appendices}
            \input{8 - Hilbert complexes/main.tex}
            \input{9 - weak conservation proofs/main.tex}
\end{document}

            \documentclass[12pt, a4paper]{report}

\input{template/main.tex}

\title{\BA{Title in Progress...}}
\author{Boris Andrews}
\affil{Mathematical Institute, University of Oxford}
\date{\today}


\begin{document}
    \pagenumbering{gobble}
    \maketitle
    
    
    \begin{abstract}
        Magnetic confinement reactors---in particular tokamaks---offer one of the most promising options for achieving practical nuclear fusion, with the potential to provide virtually limitless, clean energy. The theoretical and numerical modeling of tokamak plasmas is simultaneously an essential component of effective reactor design, and a great research barrier. Tokamak operational conditions exhibit comparatively low Knudsen numbers. Kinetic effects, including kinetic waves and instabilities, Landau damping, bump-on-tail instabilities and more, are therefore highly influential in tokamak plasma dynamics. Purely fluid models are inherently incapable of capturing these effects, whereas the high dimensionality in purely kinetic models render them practically intractable for most relevant purposes.

        We consider a $\delta\!f$ decomposition model, with a macroscopic fluid background and microscopic kinetic correction, both fully coupled to each other. A similar manner of discretization is proposed to that used in the recent \texttt{STRUPHY} code \cite{Holderied_Possanner_Wang_2021, Holderied_2022, Li_et_al_2023} with a finite-element model for the background and a pseudo-particle/PiC model for the correction.

        The fluid background satisfies the full, non-linear, resistive, compressible, Hall MHD equations. \cite{Laakmann_Hu_Farrell_2022} introduces finite-element(-in-space) implicit timesteppers for the incompressible analogue to this system with structure-preserving (SP) properties in the ideal case, alongside parameter-robust preconditioners. We show that these timesteppers can derive from a finite-element-in-time (FET) (and finite-element-in-space) interpretation. The benefits of this reformulation are discussed, including the derivation of timesteppers that are higher order in time, and the quantifiable dissipative SP properties in the non-ideal, resistive case.
        
        We discuss possible options for extending this FET approach to timesteppers for the compressible case.

        The kinetic corrections satisfy linearized Boltzmann equations. Using a Lénard--Bernstein collision operator, these take Fokker--Planck-like forms \cite{Fokker_1914, Planck_1917} wherein pseudo-particles in the numerical model obey the neoclassical transport equations, with particle-independent Brownian drift terms. This offers a rigorous methodology for incorporating collisions into the particle transport model, without coupling the equations of motions for each particle.
        
        Works by Chen, Chacón et al. \cite{Chen_Chacón_Barnes_2011, Chacón_Chen_Barnes_2013, Chen_Chacón_2014, Chen_Chacón_2015} have developed structure-preserving particle pushers for neoclassical transport in the Vlasov equations, derived from Crank--Nicolson integrators. We show these too can can derive from a FET interpretation, similarly offering potential extensions to higher-order-in-time particle pushers. The FET formulation is used also to consider how the stochastic drift terms can be incorporated into the pushers. Stochastic gyrokinetic expansions are also discussed.

        Different options for the numerical implementation of these schemes are considered.

        Due to the efficacy of FET in the development of SP timesteppers for both the fluid and kinetic component, we hope this approach will prove effective in the future for developing SP timesteppers for the full hybrid model. We hope this will give us the opportunity to incorporate previously inaccessible kinetic effects into the highly effective, modern, finite-element MHD models.
    \end{abstract}
    
    
    \newpage
    \tableofcontents
    
    
    \newpage
    \pagenumbering{arabic}
    %\linenumbers\renewcommand\thelinenumber{\color{black!50}\arabic{linenumber}}
            \input{0 - introduction/main.tex}
        \part{Research}
            \input{1 - low-noise PiC models/main.tex}
            \input{2 - kinetic component/main.tex}
            \input{3 - fluid component/main.tex}
            \input{4 - numerical implementation/main.tex}
        \part{Project Overview}
            \input{5 - research plan/main.tex}
            \input{6 - summary/main.tex}
    
    
    %\section{}
    \newpage
    \pagenumbering{gobble}
        \printbibliography


    \newpage
    \pagenumbering{roman}
    \appendix
        \part{Appendices}
            \input{8 - Hilbert complexes/main.tex}
            \input{9 - weak conservation proofs/main.tex}
\end{document}

    
    
    %\section{}
    \newpage
    \pagenumbering{gobble}
        \printbibliography


    \newpage
    \pagenumbering{roman}
    \appendix
        \part{Appendices}
            \documentclass[12pt, a4paper]{report}

\input{template/main.tex}

\title{\BA{Title in Progress...}}
\author{Boris Andrews}
\affil{Mathematical Institute, University of Oxford}
\date{\today}


\begin{document}
    \pagenumbering{gobble}
    \maketitle
    
    
    \begin{abstract}
        Magnetic confinement reactors---in particular tokamaks---offer one of the most promising options for achieving practical nuclear fusion, with the potential to provide virtually limitless, clean energy. The theoretical and numerical modeling of tokamak plasmas is simultaneously an essential component of effective reactor design, and a great research barrier. Tokamak operational conditions exhibit comparatively low Knudsen numbers. Kinetic effects, including kinetic waves and instabilities, Landau damping, bump-on-tail instabilities and more, are therefore highly influential in tokamak plasma dynamics. Purely fluid models are inherently incapable of capturing these effects, whereas the high dimensionality in purely kinetic models render them practically intractable for most relevant purposes.

        We consider a $\delta\!f$ decomposition model, with a macroscopic fluid background and microscopic kinetic correction, both fully coupled to each other. A similar manner of discretization is proposed to that used in the recent \texttt{STRUPHY} code \cite{Holderied_Possanner_Wang_2021, Holderied_2022, Li_et_al_2023} with a finite-element model for the background and a pseudo-particle/PiC model for the correction.

        The fluid background satisfies the full, non-linear, resistive, compressible, Hall MHD equations. \cite{Laakmann_Hu_Farrell_2022} introduces finite-element(-in-space) implicit timesteppers for the incompressible analogue to this system with structure-preserving (SP) properties in the ideal case, alongside parameter-robust preconditioners. We show that these timesteppers can derive from a finite-element-in-time (FET) (and finite-element-in-space) interpretation. The benefits of this reformulation are discussed, including the derivation of timesteppers that are higher order in time, and the quantifiable dissipative SP properties in the non-ideal, resistive case.
        
        We discuss possible options for extending this FET approach to timesteppers for the compressible case.

        The kinetic corrections satisfy linearized Boltzmann equations. Using a Lénard--Bernstein collision operator, these take Fokker--Planck-like forms \cite{Fokker_1914, Planck_1917} wherein pseudo-particles in the numerical model obey the neoclassical transport equations, with particle-independent Brownian drift terms. This offers a rigorous methodology for incorporating collisions into the particle transport model, without coupling the equations of motions for each particle.
        
        Works by Chen, Chacón et al. \cite{Chen_Chacón_Barnes_2011, Chacón_Chen_Barnes_2013, Chen_Chacón_2014, Chen_Chacón_2015} have developed structure-preserving particle pushers for neoclassical transport in the Vlasov equations, derived from Crank--Nicolson integrators. We show these too can can derive from a FET interpretation, similarly offering potential extensions to higher-order-in-time particle pushers. The FET formulation is used also to consider how the stochastic drift terms can be incorporated into the pushers. Stochastic gyrokinetic expansions are also discussed.

        Different options for the numerical implementation of these schemes are considered.

        Due to the efficacy of FET in the development of SP timesteppers for both the fluid and kinetic component, we hope this approach will prove effective in the future for developing SP timesteppers for the full hybrid model. We hope this will give us the opportunity to incorporate previously inaccessible kinetic effects into the highly effective, modern, finite-element MHD models.
    \end{abstract}
    
    
    \newpage
    \tableofcontents
    
    
    \newpage
    \pagenumbering{arabic}
    %\linenumbers\renewcommand\thelinenumber{\color{black!50}\arabic{linenumber}}
            \input{0 - introduction/main.tex}
        \part{Research}
            \input{1 - low-noise PiC models/main.tex}
            \input{2 - kinetic component/main.tex}
            \input{3 - fluid component/main.tex}
            \input{4 - numerical implementation/main.tex}
        \part{Project Overview}
            \input{5 - research plan/main.tex}
            \input{6 - summary/main.tex}
    
    
    %\section{}
    \newpage
    \pagenumbering{gobble}
        \printbibliography


    \newpage
    \pagenumbering{roman}
    \appendix
        \part{Appendices}
            \input{8 - Hilbert complexes/main.tex}
            \input{9 - weak conservation proofs/main.tex}
\end{document}

            \documentclass[12pt, a4paper]{report}

\input{template/main.tex}

\title{\BA{Title in Progress...}}
\author{Boris Andrews}
\affil{Mathematical Institute, University of Oxford}
\date{\today}


\begin{document}
    \pagenumbering{gobble}
    \maketitle
    
    
    \begin{abstract}
        Magnetic confinement reactors---in particular tokamaks---offer one of the most promising options for achieving practical nuclear fusion, with the potential to provide virtually limitless, clean energy. The theoretical and numerical modeling of tokamak plasmas is simultaneously an essential component of effective reactor design, and a great research barrier. Tokamak operational conditions exhibit comparatively low Knudsen numbers. Kinetic effects, including kinetic waves and instabilities, Landau damping, bump-on-tail instabilities and more, are therefore highly influential in tokamak plasma dynamics. Purely fluid models are inherently incapable of capturing these effects, whereas the high dimensionality in purely kinetic models render them practically intractable for most relevant purposes.

        We consider a $\delta\!f$ decomposition model, with a macroscopic fluid background and microscopic kinetic correction, both fully coupled to each other. A similar manner of discretization is proposed to that used in the recent \texttt{STRUPHY} code \cite{Holderied_Possanner_Wang_2021, Holderied_2022, Li_et_al_2023} with a finite-element model for the background and a pseudo-particle/PiC model for the correction.

        The fluid background satisfies the full, non-linear, resistive, compressible, Hall MHD equations. \cite{Laakmann_Hu_Farrell_2022} introduces finite-element(-in-space) implicit timesteppers for the incompressible analogue to this system with structure-preserving (SP) properties in the ideal case, alongside parameter-robust preconditioners. We show that these timesteppers can derive from a finite-element-in-time (FET) (and finite-element-in-space) interpretation. The benefits of this reformulation are discussed, including the derivation of timesteppers that are higher order in time, and the quantifiable dissipative SP properties in the non-ideal, resistive case.
        
        We discuss possible options for extending this FET approach to timesteppers for the compressible case.

        The kinetic corrections satisfy linearized Boltzmann equations. Using a Lénard--Bernstein collision operator, these take Fokker--Planck-like forms \cite{Fokker_1914, Planck_1917} wherein pseudo-particles in the numerical model obey the neoclassical transport equations, with particle-independent Brownian drift terms. This offers a rigorous methodology for incorporating collisions into the particle transport model, without coupling the equations of motions for each particle.
        
        Works by Chen, Chacón et al. \cite{Chen_Chacón_Barnes_2011, Chacón_Chen_Barnes_2013, Chen_Chacón_2014, Chen_Chacón_2015} have developed structure-preserving particle pushers for neoclassical transport in the Vlasov equations, derived from Crank--Nicolson integrators. We show these too can can derive from a FET interpretation, similarly offering potential extensions to higher-order-in-time particle pushers. The FET formulation is used also to consider how the stochastic drift terms can be incorporated into the pushers. Stochastic gyrokinetic expansions are also discussed.

        Different options for the numerical implementation of these schemes are considered.

        Due to the efficacy of FET in the development of SP timesteppers for both the fluid and kinetic component, we hope this approach will prove effective in the future for developing SP timesteppers for the full hybrid model. We hope this will give us the opportunity to incorporate previously inaccessible kinetic effects into the highly effective, modern, finite-element MHD models.
    \end{abstract}
    
    
    \newpage
    \tableofcontents
    
    
    \newpage
    \pagenumbering{arabic}
    %\linenumbers\renewcommand\thelinenumber{\color{black!50}\arabic{linenumber}}
            \input{0 - introduction/main.tex}
        \part{Research}
            \input{1 - low-noise PiC models/main.tex}
            \input{2 - kinetic component/main.tex}
            \input{3 - fluid component/main.tex}
            \input{4 - numerical implementation/main.tex}
        \part{Project Overview}
            \input{5 - research plan/main.tex}
            \input{6 - summary/main.tex}
    
    
    %\section{}
    \newpage
    \pagenumbering{gobble}
        \printbibliography


    \newpage
    \pagenumbering{roman}
    \appendix
        \part{Appendices}
            \input{8 - Hilbert complexes/main.tex}
            \input{9 - weak conservation proofs/main.tex}
\end{document}

\end{document}

        \part{Project Overview}
            \documentclass[12pt, a4paper]{report}

\documentclass[12pt, a4paper]{report}

\input{template/main.tex}

\title{\BA{Title in Progress...}}
\author{Boris Andrews}
\affil{Mathematical Institute, University of Oxford}
\date{\today}


\begin{document}
    \pagenumbering{gobble}
    \maketitle
    
    
    \begin{abstract}
        Magnetic confinement reactors---in particular tokamaks---offer one of the most promising options for achieving practical nuclear fusion, with the potential to provide virtually limitless, clean energy. The theoretical and numerical modeling of tokamak plasmas is simultaneously an essential component of effective reactor design, and a great research barrier. Tokamak operational conditions exhibit comparatively low Knudsen numbers. Kinetic effects, including kinetic waves and instabilities, Landau damping, bump-on-tail instabilities and more, are therefore highly influential in tokamak plasma dynamics. Purely fluid models are inherently incapable of capturing these effects, whereas the high dimensionality in purely kinetic models render them practically intractable for most relevant purposes.

        We consider a $\delta\!f$ decomposition model, with a macroscopic fluid background and microscopic kinetic correction, both fully coupled to each other. A similar manner of discretization is proposed to that used in the recent \texttt{STRUPHY} code \cite{Holderied_Possanner_Wang_2021, Holderied_2022, Li_et_al_2023} with a finite-element model for the background and a pseudo-particle/PiC model for the correction.

        The fluid background satisfies the full, non-linear, resistive, compressible, Hall MHD equations. \cite{Laakmann_Hu_Farrell_2022} introduces finite-element(-in-space) implicit timesteppers for the incompressible analogue to this system with structure-preserving (SP) properties in the ideal case, alongside parameter-robust preconditioners. We show that these timesteppers can derive from a finite-element-in-time (FET) (and finite-element-in-space) interpretation. The benefits of this reformulation are discussed, including the derivation of timesteppers that are higher order in time, and the quantifiable dissipative SP properties in the non-ideal, resistive case.
        
        We discuss possible options for extending this FET approach to timesteppers for the compressible case.

        The kinetic corrections satisfy linearized Boltzmann equations. Using a Lénard--Bernstein collision operator, these take Fokker--Planck-like forms \cite{Fokker_1914, Planck_1917} wherein pseudo-particles in the numerical model obey the neoclassical transport equations, with particle-independent Brownian drift terms. This offers a rigorous methodology for incorporating collisions into the particle transport model, without coupling the equations of motions for each particle.
        
        Works by Chen, Chacón et al. \cite{Chen_Chacón_Barnes_2011, Chacón_Chen_Barnes_2013, Chen_Chacón_2014, Chen_Chacón_2015} have developed structure-preserving particle pushers for neoclassical transport in the Vlasov equations, derived from Crank--Nicolson integrators. We show these too can can derive from a FET interpretation, similarly offering potential extensions to higher-order-in-time particle pushers. The FET formulation is used also to consider how the stochastic drift terms can be incorporated into the pushers. Stochastic gyrokinetic expansions are also discussed.

        Different options for the numerical implementation of these schemes are considered.

        Due to the efficacy of FET in the development of SP timesteppers for both the fluid and kinetic component, we hope this approach will prove effective in the future for developing SP timesteppers for the full hybrid model. We hope this will give us the opportunity to incorporate previously inaccessible kinetic effects into the highly effective, modern, finite-element MHD models.
    \end{abstract}
    
    
    \newpage
    \tableofcontents
    
    
    \newpage
    \pagenumbering{arabic}
    %\linenumbers\renewcommand\thelinenumber{\color{black!50}\arabic{linenumber}}
            \input{0 - introduction/main.tex}
        \part{Research}
            \input{1 - low-noise PiC models/main.tex}
            \input{2 - kinetic component/main.tex}
            \input{3 - fluid component/main.tex}
            \input{4 - numerical implementation/main.tex}
        \part{Project Overview}
            \input{5 - research plan/main.tex}
            \input{6 - summary/main.tex}
    
    
    %\section{}
    \newpage
    \pagenumbering{gobble}
        \printbibliography


    \newpage
    \pagenumbering{roman}
    \appendix
        \part{Appendices}
            \input{8 - Hilbert complexes/main.tex}
            \input{9 - weak conservation proofs/main.tex}
\end{document}


\title{\BA{Title in Progress...}}
\author{Boris Andrews}
\affil{Mathematical Institute, University of Oxford}
\date{\today}


\begin{document}
    \pagenumbering{gobble}
    \maketitle
    
    
    \begin{abstract}
        Magnetic confinement reactors---in particular tokamaks---offer one of the most promising options for achieving practical nuclear fusion, with the potential to provide virtually limitless, clean energy. The theoretical and numerical modeling of tokamak plasmas is simultaneously an essential component of effective reactor design, and a great research barrier. Tokamak operational conditions exhibit comparatively low Knudsen numbers. Kinetic effects, including kinetic waves and instabilities, Landau damping, bump-on-tail instabilities and more, are therefore highly influential in tokamak plasma dynamics. Purely fluid models are inherently incapable of capturing these effects, whereas the high dimensionality in purely kinetic models render them practically intractable for most relevant purposes.

        We consider a $\delta\!f$ decomposition model, with a macroscopic fluid background and microscopic kinetic correction, both fully coupled to each other. A similar manner of discretization is proposed to that used in the recent \texttt{STRUPHY} code \cite{Holderied_Possanner_Wang_2021, Holderied_2022, Li_et_al_2023} with a finite-element model for the background and a pseudo-particle/PiC model for the correction.

        The fluid background satisfies the full, non-linear, resistive, compressible, Hall MHD equations. \cite{Laakmann_Hu_Farrell_2022} introduces finite-element(-in-space) implicit timesteppers for the incompressible analogue to this system with structure-preserving (SP) properties in the ideal case, alongside parameter-robust preconditioners. We show that these timesteppers can derive from a finite-element-in-time (FET) (and finite-element-in-space) interpretation. The benefits of this reformulation are discussed, including the derivation of timesteppers that are higher order in time, and the quantifiable dissipative SP properties in the non-ideal, resistive case.
        
        We discuss possible options for extending this FET approach to timesteppers for the compressible case.

        The kinetic corrections satisfy linearized Boltzmann equations. Using a Lénard--Bernstein collision operator, these take Fokker--Planck-like forms \cite{Fokker_1914, Planck_1917} wherein pseudo-particles in the numerical model obey the neoclassical transport equations, with particle-independent Brownian drift terms. This offers a rigorous methodology for incorporating collisions into the particle transport model, without coupling the equations of motions for each particle.
        
        Works by Chen, Chacón et al. \cite{Chen_Chacón_Barnes_2011, Chacón_Chen_Barnes_2013, Chen_Chacón_2014, Chen_Chacón_2015} have developed structure-preserving particle pushers for neoclassical transport in the Vlasov equations, derived from Crank--Nicolson integrators. We show these too can can derive from a FET interpretation, similarly offering potential extensions to higher-order-in-time particle pushers. The FET formulation is used also to consider how the stochastic drift terms can be incorporated into the pushers. Stochastic gyrokinetic expansions are also discussed.

        Different options for the numerical implementation of these schemes are considered.

        Due to the efficacy of FET in the development of SP timesteppers for both the fluid and kinetic component, we hope this approach will prove effective in the future for developing SP timesteppers for the full hybrid model. We hope this will give us the opportunity to incorporate previously inaccessible kinetic effects into the highly effective, modern, finite-element MHD models.
    \end{abstract}
    
    
    \newpage
    \tableofcontents
    
    
    \newpage
    \pagenumbering{arabic}
    %\linenumbers\renewcommand\thelinenumber{\color{black!50}\arabic{linenumber}}
            \documentclass[12pt, a4paper]{report}

\input{template/main.tex}

\title{\BA{Title in Progress...}}
\author{Boris Andrews}
\affil{Mathematical Institute, University of Oxford}
\date{\today}


\begin{document}
    \pagenumbering{gobble}
    \maketitle
    
    
    \begin{abstract}
        Magnetic confinement reactors---in particular tokamaks---offer one of the most promising options for achieving practical nuclear fusion, with the potential to provide virtually limitless, clean energy. The theoretical and numerical modeling of tokamak plasmas is simultaneously an essential component of effective reactor design, and a great research barrier. Tokamak operational conditions exhibit comparatively low Knudsen numbers. Kinetic effects, including kinetic waves and instabilities, Landau damping, bump-on-tail instabilities and more, are therefore highly influential in tokamak plasma dynamics. Purely fluid models are inherently incapable of capturing these effects, whereas the high dimensionality in purely kinetic models render them practically intractable for most relevant purposes.

        We consider a $\delta\!f$ decomposition model, with a macroscopic fluid background and microscopic kinetic correction, both fully coupled to each other. A similar manner of discretization is proposed to that used in the recent \texttt{STRUPHY} code \cite{Holderied_Possanner_Wang_2021, Holderied_2022, Li_et_al_2023} with a finite-element model for the background and a pseudo-particle/PiC model for the correction.

        The fluid background satisfies the full, non-linear, resistive, compressible, Hall MHD equations. \cite{Laakmann_Hu_Farrell_2022} introduces finite-element(-in-space) implicit timesteppers for the incompressible analogue to this system with structure-preserving (SP) properties in the ideal case, alongside parameter-robust preconditioners. We show that these timesteppers can derive from a finite-element-in-time (FET) (and finite-element-in-space) interpretation. The benefits of this reformulation are discussed, including the derivation of timesteppers that are higher order in time, and the quantifiable dissipative SP properties in the non-ideal, resistive case.
        
        We discuss possible options for extending this FET approach to timesteppers for the compressible case.

        The kinetic corrections satisfy linearized Boltzmann equations. Using a Lénard--Bernstein collision operator, these take Fokker--Planck-like forms \cite{Fokker_1914, Planck_1917} wherein pseudo-particles in the numerical model obey the neoclassical transport equations, with particle-independent Brownian drift terms. This offers a rigorous methodology for incorporating collisions into the particle transport model, without coupling the equations of motions for each particle.
        
        Works by Chen, Chacón et al. \cite{Chen_Chacón_Barnes_2011, Chacón_Chen_Barnes_2013, Chen_Chacón_2014, Chen_Chacón_2015} have developed structure-preserving particle pushers for neoclassical transport in the Vlasov equations, derived from Crank--Nicolson integrators. We show these too can can derive from a FET interpretation, similarly offering potential extensions to higher-order-in-time particle pushers. The FET formulation is used also to consider how the stochastic drift terms can be incorporated into the pushers. Stochastic gyrokinetic expansions are also discussed.

        Different options for the numerical implementation of these schemes are considered.

        Due to the efficacy of FET in the development of SP timesteppers for both the fluid and kinetic component, we hope this approach will prove effective in the future for developing SP timesteppers for the full hybrid model. We hope this will give us the opportunity to incorporate previously inaccessible kinetic effects into the highly effective, modern, finite-element MHD models.
    \end{abstract}
    
    
    \newpage
    \tableofcontents
    
    
    \newpage
    \pagenumbering{arabic}
    %\linenumbers\renewcommand\thelinenumber{\color{black!50}\arabic{linenumber}}
            \input{0 - introduction/main.tex}
        \part{Research}
            \input{1 - low-noise PiC models/main.tex}
            \input{2 - kinetic component/main.tex}
            \input{3 - fluid component/main.tex}
            \input{4 - numerical implementation/main.tex}
        \part{Project Overview}
            \input{5 - research plan/main.tex}
            \input{6 - summary/main.tex}
    
    
    %\section{}
    \newpage
    \pagenumbering{gobble}
        \printbibliography


    \newpage
    \pagenumbering{roman}
    \appendix
        \part{Appendices}
            \input{8 - Hilbert complexes/main.tex}
            \input{9 - weak conservation proofs/main.tex}
\end{document}

        \part{Research}
            \documentclass[12pt, a4paper]{report}

\input{template/main.tex}

\title{\BA{Title in Progress...}}
\author{Boris Andrews}
\affil{Mathematical Institute, University of Oxford}
\date{\today}


\begin{document}
    \pagenumbering{gobble}
    \maketitle
    
    
    \begin{abstract}
        Magnetic confinement reactors---in particular tokamaks---offer one of the most promising options for achieving practical nuclear fusion, with the potential to provide virtually limitless, clean energy. The theoretical and numerical modeling of tokamak plasmas is simultaneously an essential component of effective reactor design, and a great research barrier. Tokamak operational conditions exhibit comparatively low Knudsen numbers. Kinetic effects, including kinetic waves and instabilities, Landau damping, bump-on-tail instabilities and more, are therefore highly influential in tokamak plasma dynamics. Purely fluid models are inherently incapable of capturing these effects, whereas the high dimensionality in purely kinetic models render them practically intractable for most relevant purposes.

        We consider a $\delta\!f$ decomposition model, with a macroscopic fluid background and microscopic kinetic correction, both fully coupled to each other. A similar manner of discretization is proposed to that used in the recent \texttt{STRUPHY} code \cite{Holderied_Possanner_Wang_2021, Holderied_2022, Li_et_al_2023} with a finite-element model for the background and a pseudo-particle/PiC model for the correction.

        The fluid background satisfies the full, non-linear, resistive, compressible, Hall MHD equations. \cite{Laakmann_Hu_Farrell_2022} introduces finite-element(-in-space) implicit timesteppers for the incompressible analogue to this system with structure-preserving (SP) properties in the ideal case, alongside parameter-robust preconditioners. We show that these timesteppers can derive from a finite-element-in-time (FET) (and finite-element-in-space) interpretation. The benefits of this reformulation are discussed, including the derivation of timesteppers that are higher order in time, and the quantifiable dissipative SP properties in the non-ideal, resistive case.
        
        We discuss possible options for extending this FET approach to timesteppers for the compressible case.

        The kinetic corrections satisfy linearized Boltzmann equations. Using a Lénard--Bernstein collision operator, these take Fokker--Planck-like forms \cite{Fokker_1914, Planck_1917} wherein pseudo-particles in the numerical model obey the neoclassical transport equations, with particle-independent Brownian drift terms. This offers a rigorous methodology for incorporating collisions into the particle transport model, without coupling the equations of motions for each particle.
        
        Works by Chen, Chacón et al. \cite{Chen_Chacón_Barnes_2011, Chacón_Chen_Barnes_2013, Chen_Chacón_2014, Chen_Chacón_2015} have developed structure-preserving particle pushers for neoclassical transport in the Vlasov equations, derived from Crank--Nicolson integrators. We show these too can can derive from a FET interpretation, similarly offering potential extensions to higher-order-in-time particle pushers. The FET formulation is used also to consider how the stochastic drift terms can be incorporated into the pushers. Stochastic gyrokinetic expansions are also discussed.

        Different options for the numerical implementation of these schemes are considered.

        Due to the efficacy of FET in the development of SP timesteppers for both the fluid and kinetic component, we hope this approach will prove effective in the future for developing SP timesteppers for the full hybrid model. We hope this will give us the opportunity to incorporate previously inaccessible kinetic effects into the highly effective, modern, finite-element MHD models.
    \end{abstract}
    
    
    \newpage
    \tableofcontents
    
    
    \newpage
    \pagenumbering{arabic}
    %\linenumbers\renewcommand\thelinenumber{\color{black!50}\arabic{linenumber}}
            \input{0 - introduction/main.tex}
        \part{Research}
            \input{1 - low-noise PiC models/main.tex}
            \input{2 - kinetic component/main.tex}
            \input{3 - fluid component/main.tex}
            \input{4 - numerical implementation/main.tex}
        \part{Project Overview}
            \input{5 - research plan/main.tex}
            \input{6 - summary/main.tex}
    
    
    %\section{}
    \newpage
    \pagenumbering{gobble}
        \printbibliography


    \newpage
    \pagenumbering{roman}
    \appendix
        \part{Appendices}
            \input{8 - Hilbert complexes/main.tex}
            \input{9 - weak conservation proofs/main.tex}
\end{document}

            \documentclass[12pt, a4paper]{report}

\input{template/main.tex}

\title{\BA{Title in Progress...}}
\author{Boris Andrews}
\affil{Mathematical Institute, University of Oxford}
\date{\today}


\begin{document}
    \pagenumbering{gobble}
    \maketitle
    
    
    \begin{abstract}
        Magnetic confinement reactors---in particular tokamaks---offer one of the most promising options for achieving practical nuclear fusion, with the potential to provide virtually limitless, clean energy. The theoretical and numerical modeling of tokamak plasmas is simultaneously an essential component of effective reactor design, and a great research barrier. Tokamak operational conditions exhibit comparatively low Knudsen numbers. Kinetic effects, including kinetic waves and instabilities, Landau damping, bump-on-tail instabilities and more, are therefore highly influential in tokamak plasma dynamics. Purely fluid models are inherently incapable of capturing these effects, whereas the high dimensionality in purely kinetic models render them practically intractable for most relevant purposes.

        We consider a $\delta\!f$ decomposition model, with a macroscopic fluid background and microscopic kinetic correction, both fully coupled to each other. A similar manner of discretization is proposed to that used in the recent \texttt{STRUPHY} code \cite{Holderied_Possanner_Wang_2021, Holderied_2022, Li_et_al_2023} with a finite-element model for the background and a pseudo-particle/PiC model for the correction.

        The fluid background satisfies the full, non-linear, resistive, compressible, Hall MHD equations. \cite{Laakmann_Hu_Farrell_2022} introduces finite-element(-in-space) implicit timesteppers for the incompressible analogue to this system with structure-preserving (SP) properties in the ideal case, alongside parameter-robust preconditioners. We show that these timesteppers can derive from a finite-element-in-time (FET) (and finite-element-in-space) interpretation. The benefits of this reformulation are discussed, including the derivation of timesteppers that are higher order in time, and the quantifiable dissipative SP properties in the non-ideal, resistive case.
        
        We discuss possible options for extending this FET approach to timesteppers for the compressible case.

        The kinetic corrections satisfy linearized Boltzmann equations. Using a Lénard--Bernstein collision operator, these take Fokker--Planck-like forms \cite{Fokker_1914, Planck_1917} wherein pseudo-particles in the numerical model obey the neoclassical transport equations, with particle-independent Brownian drift terms. This offers a rigorous methodology for incorporating collisions into the particle transport model, without coupling the equations of motions for each particle.
        
        Works by Chen, Chacón et al. \cite{Chen_Chacón_Barnes_2011, Chacón_Chen_Barnes_2013, Chen_Chacón_2014, Chen_Chacón_2015} have developed structure-preserving particle pushers for neoclassical transport in the Vlasov equations, derived from Crank--Nicolson integrators. We show these too can can derive from a FET interpretation, similarly offering potential extensions to higher-order-in-time particle pushers. The FET formulation is used also to consider how the stochastic drift terms can be incorporated into the pushers. Stochastic gyrokinetic expansions are also discussed.

        Different options for the numerical implementation of these schemes are considered.

        Due to the efficacy of FET in the development of SP timesteppers for both the fluid and kinetic component, we hope this approach will prove effective in the future for developing SP timesteppers for the full hybrid model. We hope this will give us the opportunity to incorporate previously inaccessible kinetic effects into the highly effective, modern, finite-element MHD models.
    \end{abstract}
    
    
    \newpage
    \tableofcontents
    
    
    \newpage
    \pagenumbering{arabic}
    %\linenumbers\renewcommand\thelinenumber{\color{black!50}\arabic{linenumber}}
            \input{0 - introduction/main.tex}
        \part{Research}
            \input{1 - low-noise PiC models/main.tex}
            \input{2 - kinetic component/main.tex}
            \input{3 - fluid component/main.tex}
            \input{4 - numerical implementation/main.tex}
        \part{Project Overview}
            \input{5 - research plan/main.tex}
            \input{6 - summary/main.tex}
    
    
    %\section{}
    \newpage
    \pagenumbering{gobble}
        \printbibliography


    \newpage
    \pagenumbering{roman}
    \appendix
        \part{Appendices}
            \input{8 - Hilbert complexes/main.tex}
            \input{9 - weak conservation proofs/main.tex}
\end{document}

            \documentclass[12pt, a4paper]{report}

\input{template/main.tex}

\title{\BA{Title in Progress...}}
\author{Boris Andrews}
\affil{Mathematical Institute, University of Oxford}
\date{\today}


\begin{document}
    \pagenumbering{gobble}
    \maketitle
    
    
    \begin{abstract}
        Magnetic confinement reactors---in particular tokamaks---offer one of the most promising options for achieving practical nuclear fusion, with the potential to provide virtually limitless, clean energy. The theoretical and numerical modeling of tokamak plasmas is simultaneously an essential component of effective reactor design, and a great research barrier. Tokamak operational conditions exhibit comparatively low Knudsen numbers. Kinetic effects, including kinetic waves and instabilities, Landau damping, bump-on-tail instabilities and more, are therefore highly influential in tokamak plasma dynamics. Purely fluid models are inherently incapable of capturing these effects, whereas the high dimensionality in purely kinetic models render them practically intractable for most relevant purposes.

        We consider a $\delta\!f$ decomposition model, with a macroscopic fluid background and microscopic kinetic correction, both fully coupled to each other. A similar manner of discretization is proposed to that used in the recent \texttt{STRUPHY} code \cite{Holderied_Possanner_Wang_2021, Holderied_2022, Li_et_al_2023} with a finite-element model for the background and a pseudo-particle/PiC model for the correction.

        The fluid background satisfies the full, non-linear, resistive, compressible, Hall MHD equations. \cite{Laakmann_Hu_Farrell_2022} introduces finite-element(-in-space) implicit timesteppers for the incompressible analogue to this system with structure-preserving (SP) properties in the ideal case, alongside parameter-robust preconditioners. We show that these timesteppers can derive from a finite-element-in-time (FET) (and finite-element-in-space) interpretation. The benefits of this reformulation are discussed, including the derivation of timesteppers that are higher order in time, and the quantifiable dissipative SP properties in the non-ideal, resistive case.
        
        We discuss possible options for extending this FET approach to timesteppers for the compressible case.

        The kinetic corrections satisfy linearized Boltzmann equations. Using a Lénard--Bernstein collision operator, these take Fokker--Planck-like forms \cite{Fokker_1914, Planck_1917} wherein pseudo-particles in the numerical model obey the neoclassical transport equations, with particle-independent Brownian drift terms. This offers a rigorous methodology for incorporating collisions into the particle transport model, without coupling the equations of motions for each particle.
        
        Works by Chen, Chacón et al. \cite{Chen_Chacón_Barnes_2011, Chacón_Chen_Barnes_2013, Chen_Chacón_2014, Chen_Chacón_2015} have developed structure-preserving particle pushers for neoclassical transport in the Vlasov equations, derived from Crank--Nicolson integrators. We show these too can can derive from a FET interpretation, similarly offering potential extensions to higher-order-in-time particle pushers. The FET formulation is used also to consider how the stochastic drift terms can be incorporated into the pushers. Stochastic gyrokinetic expansions are also discussed.

        Different options for the numerical implementation of these schemes are considered.

        Due to the efficacy of FET in the development of SP timesteppers for both the fluid and kinetic component, we hope this approach will prove effective in the future for developing SP timesteppers for the full hybrid model. We hope this will give us the opportunity to incorporate previously inaccessible kinetic effects into the highly effective, modern, finite-element MHD models.
    \end{abstract}
    
    
    \newpage
    \tableofcontents
    
    
    \newpage
    \pagenumbering{arabic}
    %\linenumbers\renewcommand\thelinenumber{\color{black!50}\arabic{linenumber}}
            \input{0 - introduction/main.tex}
        \part{Research}
            \input{1 - low-noise PiC models/main.tex}
            \input{2 - kinetic component/main.tex}
            \input{3 - fluid component/main.tex}
            \input{4 - numerical implementation/main.tex}
        \part{Project Overview}
            \input{5 - research plan/main.tex}
            \input{6 - summary/main.tex}
    
    
    %\section{}
    \newpage
    \pagenumbering{gobble}
        \printbibliography


    \newpage
    \pagenumbering{roman}
    \appendix
        \part{Appendices}
            \input{8 - Hilbert complexes/main.tex}
            \input{9 - weak conservation proofs/main.tex}
\end{document}

            \documentclass[12pt, a4paper]{report}

\input{template/main.tex}

\title{\BA{Title in Progress...}}
\author{Boris Andrews}
\affil{Mathematical Institute, University of Oxford}
\date{\today}


\begin{document}
    \pagenumbering{gobble}
    \maketitle
    
    
    \begin{abstract}
        Magnetic confinement reactors---in particular tokamaks---offer one of the most promising options for achieving practical nuclear fusion, with the potential to provide virtually limitless, clean energy. The theoretical and numerical modeling of tokamak plasmas is simultaneously an essential component of effective reactor design, and a great research barrier. Tokamak operational conditions exhibit comparatively low Knudsen numbers. Kinetic effects, including kinetic waves and instabilities, Landau damping, bump-on-tail instabilities and more, are therefore highly influential in tokamak plasma dynamics. Purely fluid models are inherently incapable of capturing these effects, whereas the high dimensionality in purely kinetic models render them practically intractable for most relevant purposes.

        We consider a $\delta\!f$ decomposition model, with a macroscopic fluid background and microscopic kinetic correction, both fully coupled to each other. A similar manner of discretization is proposed to that used in the recent \texttt{STRUPHY} code \cite{Holderied_Possanner_Wang_2021, Holderied_2022, Li_et_al_2023} with a finite-element model for the background and a pseudo-particle/PiC model for the correction.

        The fluid background satisfies the full, non-linear, resistive, compressible, Hall MHD equations. \cite{Laakmann_Hu_Farrell_2022} introduces finite-element(-in-space) implicit timesteppers for the incompressible analogue to this system with structure-preserving (SP) properties in the ideal case, alongside parameter-robust preconditioners. We show that these timesteppers can derive from a finite-element-in-time (FET) (and finite-element-in-space) interpretation. The benefits of this reformulation are discussed, including the derivation of timesteppers that are higher order in time, and the quantifiable dissipative SP properties in the non-ideal, resistive case.
        
        We discuss possible options for extending this FET approach to timesteppers for the compressible case.

        The kinetic corrections satisfy linearized Boltzmann equations. Using a Lénard--Bernstein collision operator, these take Fokker--Planck-like forms \cite{Fokker_1914, Planck_1917} wherein pseudo-particles in the numerical model obey the neoclassical transport equations, with particle-independent Brownian drift terms. This offers a rigorous methodology for incorporating collisions into the particle transport model, without coupling the equations of motions for each particle.
        
        Works by Chen, Chacón et al. \cite{Chen_Chacón_Barnes_2011, Chacón_Chen_Barnes_2013, Chen_Chacón_2014, Chen_Chacón_2015} have developed structure-preserving particle pushers for neoclassical transport in the Vlasov equations, derived from Crank--Nicolson integrators. We show these too can can derive from a FET interpretation, similarly offering potential extensions to higher-order-in-time particle pushers. The FET formulation is used also to consider how the stochastic drift terms can be incorporated into the pushers. Stochastic gyrokinetic expansions are also discussed.

        Different options for the numerical implementation of these schemes are considered.

        Due to the efficacy of FET in the development of SP timesteppers for both the fluid and kinetic component, we hope this approach will prove effective in the future for developing SP timesteppers for the full hybrid model. We hope this will give us the opportunity to incorporate previously inaccessible kinetic effects into the highly effective, modern, finite-element MHD models.
    \end{abstract}
    
    
    \newpage
    \tableofcontents
    
    
    \newpage
    \pagenumbering{arabic}
    %\linenumbers\renewcommand\thelinenumber{\color{black!50}\arabic{linenumber}}
            \input{0 - introduction/main.tex}
        \part{Research}
            \input{1 - low-noise PiC models/main.tex}
            \input{2 - kinetic component/main.tex}
            \input{3 - fluid component/main.tex}
            \input{4 - numerical implementation/main.tex}
        \part{Project Overview}
            \input{5 - research plan/main.tex}
            \input{6 - summary/main.tex}
    
    
    %\section{}
    \newpage
    \pagenumbering{gobble}
        \printbibliography


    \newpage
    \pagenumbering{roman}
    \appendix
        \part{Appendices}
            \input{8 - Hilbert complexes/main.tex}
            \input{9 - weak conservation proofs/main.tex}
\end{document}

        \part{Project Overview}
            \documentclass[12pt, a4paper]{report}

\input{template/main.tex}

\title{\BA{Title in Progress...}}
\author{Boris Andrews}
\affil{Mathematical Institute, University of Oxford}
\date{\today}


\begin{document}
    \pagenumbering{gobble}
    \maketitle
    
    
    \begin{abstract}
        Magnetic confinement reactors---in particular tokamaks---offer one of the most promising options for achieving practical nuclear fusion, with the potential to provide virtually limitless, clean energy. The theoretical and numerical modeling of tokamak plasmas is simultaneously an essential component of effective reactor design, and a great research barrier. Tokamak operational conditions exhibit comparatively low Knudsen numbers. Kinetic effects, including kinetic waves and instabilities, Landau damping, bump-on-tail instabilities and more, are therefore highly influential in tokamak plasma dynamics. Purely fluid models are inherently incapable of capturing these effects, whereas the high dimensionality in purely kinetic models render them practically intractable for most relevant purposes.

        We consider a $\delta\!f$ decomposition model, with a macroscopic fluid background and microscopic kinetic correction, both fully coupled to each other. A similar manner of discretization is proposed to that used in the recent \texttt{STRUPHY} code \cite{Holderied_Possanner_Wang_2021, Holderied_2022, Li_et_al_2023} with a finite-element model for the background and a pseudo-particle/PiC model for the correction.

        The fluid background satisfies the full, non-linear, resistive, compressible, Hall MHD equations. \cite{Laakmann_Hu_Farrell_2022} introduces finite-element(-in-space) implicit timesteppers for the incompressible analogue to this system with structure-preserving (SP) properties in the ideal case, alongside parameter-robust preconditioners. We show that these timesteppers can derive from a finite-element-in-time (FET) (and finite-element-in-space) interpretation. The benefits of this reformulation are discussed, including the derivation of timesteppers that are higher order in time, and the quantifiable dissipative SP properties in the non-ideal, resistive case.
        
        We discuss possible options for extending this FET approach to timesteppers for the compressible case.

        The kinetic corrections satisfy linearized Boltzmann equations. Using a Lénard--Bernstein collision operator, these take Fokker--Planck-like forms \cite{Fokker_1914, Planck_1917} wherein pseudo-particles in the numerical model obey the neoclassical transport equations, with particle-independent Brownian drift terms. This offers a rigorous methodology for incorporating collisions into the particle transport model, without coupling the equations of motions for each particle.
        
        Works by Chen, Chacón et al. \cite{Chen_Chacón_Barnes_2011, Chacón_Chen_Barnes_2013, Chen_Chacón_2014, Chen_Chacón_2015} have developed structure-preserving particle pushers for neoclassical transport in the Vlasov equations, derived from Crank--Nicolson integrators. We show these too can can derive from a FET interpretation, similarly offering potential extensions to higher-order-in-time particle pushers. The FET formulation is used also to consider how the stochastic drift terms can be incorporated into the pushers. Stochastic gyrokinetic expansions are also discussed.

        Different options for the numerical implementation of these schemes are considered.

        Due to the efficacy of FET in the development of SP timesteppers for both the fluid and kinetic component, we hope this approach will prove effective in the future for developing SP timesteppers for the full hybrid model. We hope this will give us the opportunity to incorporate previously inaccessible kinetic effects into the highly effective, modern, finite-element MHD models.
    \end{abstract}
    
    
    \newpage
    \tableofcontents
    
    
    \newpage
    \pagenumbering{arabic}
    %\linenumbers\renewcommand\thelinenumber{\color{black!50}\arabic{linenumber}}
            \input{0 - introduction/main.tex}
        \part{Research}
            \input{1 - low-noise PiC models/main.tex}
            \input{2 - kinetic component/main.tex}
            \input{3 - fluid component/main.tex}
            \input{4 - numerical implementation/main.tex}
        \part{Project Overview}
            \input{5 - research plan/main.tex}
            \input{6 - summary/main.tex}
    
    
    %\section{}
    \newpage
    \pagenumbering{gobble}
        \printbibliography


    \newpage
    \pagenumbering{roman}
    \appendix
        \part{Appendices}
            \input{8 - Hilbert complexes/main.tex}
            \input{9 - weak conservation proofs/main.tex}
\end{document}

            \documentclass[12pt, a4paper]{report}

\input{template/main.tex}

\title{\BA{Title in Progress...}}
\author{Boris Andrews}
\affil{Mathematical Institute, University of Oxford}
\date{\today}


\begin{document}
    \pagenumbering{gobble}
    \maketitle
    
    
    \begin{abstract}
        Magnetic confinement reactors---in particular tokamaks---offer one of the most promising options for achieving practical nuclear fusion, with the potential to provide virtually limitless, clean energy. The theoretical and numerical modeling of tokamak plasmas is simultaneously an essential component of effective reactor design, and a great research barrier. Tokamak operational conditions exhibit comparatively low Knudsen numbers. Kinetic effects, including kinetic waves and instabilities, Landau damping, bump-on-tail instabilities and more, are therefore highly influential in tokamak plasma dynamics. Purely fluid models are inherently incapable of capturing these effects, whereas the high dimensionality in purely kinetic models render them practically intractable for most relevant purposes.

        We consider a $\delta\!f$ decomposition model, with a macroscopic fluid background and microscopic kinetic correction, both fully coupled to each other. A similar manner of discretization is proposed to that used in the recent \texttt{STRUPHY} code \cite{Holderied_Possanner_Wang_2021, Holderied_2022, Li_et_al_2023} with a finite-element model for the background and a pseudo-particle/PiC model for the correction.

        The fluid background satisfies the full, non-linear, resistive, compressible, Hall MHD equations. \cite{Laakmann_Hu_Farrell_2022} introduces finite-element(-in-space) implicit timesteppers for the incompressible analogue to this system with structure-preserving (SP) properties in the ideal case, alongside parameter-robust preconditioners. We show that these timesteppers can derive from a finite-element-in-time (FET) (and finite-element-in-space) interpretation. The benefits of this reformulation are discussed, including the derivation of timesteppers that are higher order in time, and the quantifiable dissipative SP properties in the non-ideal, resistive case.
        
        We discuss possible options for extending this FET approach to timesteppers for the compressible case.

        The kinetic corrections satisfy linearized Boltzmann equations. Using a Lénard--Bernstein collision operator, these take Fokker--Planck-like forms \cite{Fokker_1914, Planck_1917} wherein pseudo-particles in the numerical model obey the neoclassical transport equations, with particle-independent Brownian drift terms. This offers a rigorous methodology for incorporating collisions into the particle transport model, without coupling the equations of motions for each particle.
        
        Works by Chen, Chacón et al. \cite{Chen_Chacón_Barnes_2011, Chacón_Chen_Barnes_2013, Chen_Chacón_2014, Chen_Chacón_2015} have developed structure-preserving particle pushers for neoclassical transport in the Vlasov equations, derived from Crank--Nicolson integrators. We show these too can can derive from a FET interpretation, similarly offering potential extensions to higher-order-in-time particle pushers. The FET formulation is used also to consider how the stochastic drift terms can be incorporated into the pushers. Stochastic gyrokinetic expansions are also discussed.

        Different options for the numerical implementation of these schemes are considered.

        Due to the efficacy of FET in the development of SP timesteppers for both the fluid and kinetic component, we hope this approach will prove effective in the future for developing SP timesteppers for the full hybrid model. We hope this will give us the opportunity to incorporate previously inaccessible kinetic effects into the highly effective, modern, finite-element MHD models.
    \end{abstract}
    
    
    \newpage
    \tableofcontents
    
    
    \newpage
    \pagenumbering{arabic}
    %\linenumbers\renewcommand\thelinenumber{\color{black!50}\arabic{linenumber}}
            \input{0 - introduction/main.tex}
        \part{Research}
            \input{1 - low-noise PiC models/main.tex}
            \input{2 - kinetic component/main.tex}
            \input{3 - fluid component/main.tex}
            \input{4 - numerical implementation/main.tex}
        \part{Project Overview}
            \input{5 - research plan/main.tex}
            \input{6 - summary/main.tex}
    
    
    %\section{}
    \newpage
    \pagenumbering{gobble}
        \printbibliography


    \newpage
    \pagenumbering{roman}
    \appendix
        \part{Appendices}
            \input{8 - Hilbert complexes/main.tex}
            \input{9 - weak conservation proofs/main.tex}
\end{document}

    
    
    %\section{}
    \newpage
    \pagenumbering{gobble}
        \printbibliography


    \newpage
    \pagenumbering{roman}
    \appendix
        \part{Appendices}
            \documentclass[12pt, a4paper]{report}

\input{template/main.tex}

\title{\BA{Title in Progress...}}
\author{Boris Andrews}
\affil{Mathematical Institute, University of Oxford}
\date{\today}


\begin{document}
    \pagenumbering{gobble}
    \maketitle
    
    
    \begin{abstract}
        Magnetic confinement reactors---in particular tokamaks---offer one of the most promising options for achieving practical nuclear fusion, with the potential to provide virtually limitless, clean energy. The theoretical and numerical modeling of tokamak plasmas is simultaneously an essential component of effective reactor design, and a great research barrier. Tokamak operational conditions exhibit comparatively low Knudsen numbers. Kinetic effects, including kinetic waves and instabilities, Landau damping, bump-on-tail instabilities and more, are therefore highly influential in tokamak plasma dynamics. Purely fluid models are inherently incapable of capturing these effects, whereas the high dimensionality in purely kinetic models render them practically intractable for most relevant purposes.

        We consider a $\delta\!f$ decomposition model, with a macroscopic fluid background and microscopic kinetic correction, both fully coupled to each other. A similar manner of discretization is proposed to that used in the recent \texttt{STRUPHY} code \cite{Holderied_Possanner_Wang_2021, Holderied_2022, Li_et_al_2023} with a finite-element model for the background and a pseudo-particle/PiC model for the correction.

        The fluid background satisfies the full, non-linear, resistive, compressible, Hall MHD equations. \cite{Laakmann_Hu_Farrell_2022} introduces finite-element(-in-space) implicit timesteppers for the incompressible analogue to this system with structure-preserving (SP) properties in the ideal case, alongside parameter-robust preconditioners. We show that these timesteppers can derive from a finite-element-in-time (FET) (and finite-element-in-space) interpretation. The benefits of this reformulation are discussed, including the derivation of timesteppers that are higher order in time, and the quantifiable dissipative SP properties in the non-ideal, resistive case.
        
        We discuss possible options for extending this FET approach to timesteppers for the compressible case.

        The kinetic corrections satisfy linearized Boltzmann equations. Using a Lénard--Bernstein collision operator, these take Fokker--Planck-like forms \cite{Fokker_1914, Planck_1917} wherein pseudo-particles in the numerical model obey the neoclassical transport equations, with particle-independent Brownian drift terms. This offers a rigorous methodology for incorporating collisions into the particle transport model, without coupling the equations of motions for each particle.
        
        Works by Chen, Chacón et al. \cite{Chen_Chacón_Barnes_2011, Chacón_Chen_Barnes_2013, Chen_Chacón_2014, Chen_Chacón_2015} have developed structure-preserving particle pushers for neoclassical transport in the Vlasov equations, derived from Crank--Nicolson integrators. We show these too can can derive from a FET interpretation, similarly offering potential extensions to higher-order-in-time particle pushers. The FET formulation is used also to consider how the stochastic drift terms can be incorporated into the pushers. Stochastic gyrokinetic expansions are also discussed.

        Different options for the numerical implementation of these schemes are considered.

        Due to the efficacy of FET in the development of SP timesteppers for both the fluid and kinetic component, we hope this approach will prove effective in the future for developing SP timesteppers for the full hybrid model. We hope this will give us the opportunity to incorporate previously inaccessible kinetic effects into the highly effective, modern, finite-element MHD models.
    \end{abstract}
    
    
    \newpage
    \tableofcontents
    
    
    \newpage
    \pagenumbering{arabic}
    %\linenumbers\renewcommand\thelinenumber{\color{black!50}\arabic{linenumber}}
            \input{0 - introduction/main.tex}
        \part{Research}
            \input{1 - low-noise PiC models/main.tex}
            \input{2 - kinetic component/main.tex}
            \input{3 - fluid component/main.tex}
            \input{4 - numerical implementation/main.tex}
        \part{Project Overview}
            \input{5 - research plan/main.tex}
            \input{6 - summary/main.tex}
    
    
    %\section{}
    \newpage
    \pagenumbering{gobble}
        \printbibliography


    \newpage
    \pagenumbering{roman}
    \appendix
        \part{Appendices}
            \input{8 - Hilbert complexes/main.tex}
            \input{9 - weak conservation proofs/main.tex}
\end{document}

            \documentclass[12pt, a4paper]{report}

\input{template/main.tex}

\title{\BA{Title in Progress...}}
\author{Boris Andrews}
\affil{Mathematical Institute, University of Oxford}
\date{\today}


\begin{document}
    \pagenumbering{gobble}
    \maketitle
    
    
    \begin{abstract}
        Magnetic confinement reactors---in particular tokamaks---offer one of the most promising options for achieving practical nuclear fusion, with the potential to provide virtually limitless, clean energy. The theoretical and numerical modeling of tokamak plasmas is simultaneously an essential component of effective reactor design, and a great research barrier. Tokamak operational conditions exhibit comparatively low Knudsen numbers. Kinetic effects, including kinetic waves and instabilities, Landau damping, bump-on-tail instabilities and more, are therefore highly influential in tokamak plasma dynamics. Purely fluid models are inherently incapable of capturing these effects, whereas the high dimensionality in purely kinetic models render them practically intractable for most relevant purposes.

        We consider a $\delta\!f$ decomposition model, with a macroscopic fluid background and microscopic kinetic correction, both fully coupled to each other. A similar manner of discretization is proposed to that used in the recent \texttt{STRUPHY} code \cite{Holderied_Possanner_Wang_2021, Holderied_2022, Li_et_al_2023} with a finite-element model for the background and a pseudo-particle/PiC model for the correction.

        The fluid background satisfies the full, non-linear, resistive, compressible, Hall MHD equations. \cite{Laakmann_Hu_Farrell_2022} introduces finite-element(-in-space) implicit timesteppers for the incompressible analogue to this system with structure-preserving (SP) properties in the ideal case, alongside parameter-robust preconditioners. We show that these timesteppers can derive from a finite-element-in-time (FET) (and finite-element-in-space) interpretation. The benefits of this reformulation are discussed, including the derivation of timesteppers that are higher order in time, and the quantifiable dissipative SP properties in the non-ideal, resistive case.
        
        We discuss possible options for extending this FET approach to timesteppers for the compressible case.

        The kinetic corrections satisfy linearized Boltzmann equations. Using a Lénard--Bernstein collision operator, these take Fokker--Planck-like forms \cite{Fokker_1914, Planck_1917} wherein pseudo-particles in the numerical model obey the neoclassical transport equations, with particle-independent Brownian drift terms. This offers a rigorous methodology for incorporating collisions into the particle transport model, without coupling the equations of motions for each particle.
        
        Works by Chen, Chacón et al. \cite{Chen_Chacón_Barnes_2011, Chacón_Chen_Barnes_2013, Chen_Chacón_2014, Chen_Chacón_2015} have developed structure-preserving particle pushers for neoclassical transport in the Vlasov equations, derived from Crank--Nicolson integrators. We show these too can can derive from a FET interpretation, similarly offering potential extensions to higher-order-in-time particle pushers. The FET formulation is used also to consider how the stochastic drift terms can be incorporated into the pushers. Stochastic gyrokinetic expansions are also discussed.

        Different options for the numerical implementation of these schemes are considered.

        Due to the efficacy of FET in the development of SP timesteppers for both the fluid and kinetic component, we hope this approach will prove effective in the future for developing SP timesteppers for the full hybrid model. We hope this will give us the opportunity to incorporate previously inaccessible kinetic effects into the highly effective, modern, finite-element MHD models.
    \end{abstract}
    
    
    \newpage
    \tableofcontents
    
    
    \newpage
    \pagenumbering{arabic}
    %\linenumbers\renewcommand\thelinenumber{\color{black!50}\arabic{linenumber}}
            \input{0 - introduction/main.tex}
        \part{Research}
            \input{1 - low-noise PiC models/main.tex}
            \input{2 - kinetic component/main.tex}
            \input{3 - fluid component/main.tex}
            \input{4 - numerical implementation/main.tex}
        \part{Project Overview}
            \input{5 - research plan/main.tex}
            \input{6 - summary/main.tex}
    
    
    %\section{}
    \newpage
    \pagenumbering{gobble}
        \printbibliography


    \newpage
    \pagenumbering{roman}
    \appendix
        \part{Appendices}
            \input{8 - Hilbert complexes/main.tex}
            \input{9 - weak conservation proofs/main.tex}
\end{document}

\end{document}

            \documentclass[12pt, a4paper]{report}

\documentclass[12pt, a4paper]{report}

\input{template/main.tex}

\title{\BA{Title in Progress...}}
\author{Boris Andrews}
\affil{Mathematical Institute, University of Oxford}
\date{\today}


\begin{document}
    \pagenumbering{gobble}
    \maketitle
    
    
    \begin{abstract}
        Magnetic confinement reactors---in particular tokamaks---offer one of the most promising options for achieving practical nuclear fusion, with the potential to provide virtually limitless, clean energy. The theoretical and numerical modeling of tokamak plasmas is simultaneously an essential component of effective reactor design, and a great research barrier. Tokamak operational conditions exhibit comparatively low Knudsen numbers. Kinetic effects, including kinetic waves and instabilities, Landau damping, bump-on-tail instabilities and more, are therefore highly influential in tokamak plasma dynamics. Purely fluid models are inherently incapable of capturing these effects, whereas the high dimensionality in purely kinetic models render them practically intractable for most relevant purposes.

        We consider a $\delta\!f$ decomposition model, with a macroscopic fluid background and microscopic kinetic correction, both fully coupled to each other. A similar manner of discretization is proposed to that used in the recent \texttt{STRUPHY} code \cite{Holderied_Possanner_Wang_2021, Holderied_2022, Li_et_al_2023} with a finite-element model for the background and a pseudo-particle/PiC model for the correction.

        The fluid background satisfies the full, non-linear, resistive, compressible, Hall MHD equations. \cite{Laakmann_Hu_Farrell_2022} introduces finite-element(-in-space) implicit timesteppers for the incompressible analogue to this system with structure-preserving (SP) properties in the ideal case, alongside parameter-robust preconditioners. We show that these timesteppers can derive from a finite-element-in-time (FET) (and finite-element-in-space) interpretation. The benefits of this reformulation are discussed, including the derivation of timesteppers that are higher order in time, and the quantifiable dissipative SP properties in the non-ideal, resistive case.
        
        We discuss possible options for extending this FET approach to timesteppers for the compressible case.

        The kinetic corrections satisfy linearized Boltzmann equations. Using a Lénard--Bernstein collision operator, these take Fokker--Planck-like forms \cite{Fokker_1914, Planck_1917} wherein pseudo-particles in the numerical model obey the neoclassical transport equations, with particle-independent Brownian drift terms. This offers a rigorous methodology for incorporating collisions into the particle transport model, without coupling the equations of motions for each particle.
        
        Works by Chen, Chacón et al. \cite{Chen_Chacón_Barnes_2011, Chacón_Chen_Barnes_2013, Chen_Chacón_2014, Chen_Chacón_2015} have developed structure-preserving particle pushers for neoclassical transport in the Vlasov equations, derived from Crank--Nicolson integrators. We show these too can can derive from a FET interpretation, similarly offering potential extensions to higher-order-in-time particle pushers. The FET formulation is used also to consider how the stochastic drift terms can be incorporated into the pushers. Stochastic gyrokinetic expansions are also discussed.

        Different options for the numerical implementation of these schemes are considered.

        Due to the efficacy of FET in the development of SP timesteppers for both the fluid and kinetic component, we hope this approach will prove effective in the future for developing SP timesteppers for the full hybrid model. We hope this will give us the opportunity to incorporate previously inaccessible kinetic effects into the highly effective, modern, finite-element MHD models.
    \end{abstract}
    
    
    \newpage
    \tableofcontents
    
    
    \newpage
    \pagenumbering{arabic}
    %\linenumbers\renewcommand\thelinenumber{\color{black!50}\arabic{linenumber}}
            \input{0 - introduction/main.tex}
        \part{Research}
            \input{1 - low-noise PiC models/main.tex}
            \input{2 - kinetic component/main.tex}
            \input{3 - fluid component/main.tex}
            \input{4 - numerical implementation/main.tex}
        \part{Project Overview}
            \input{5 - research plan/main.tex}
            \input{6 - summary/main.tex}
    
    
    %\section{}
    \newpage
    \pagenumbering{gobble}
        \printbibliography


    \newpage
    \pagenumbering{roman}
    \appendix
        \part{Appendices}
            \input{8 - Hilbert complexes/main.tex}
            \input{9 - weak conservation proofs/main.tex}
\end{document}


\title{\BA{Title in Progress...}}
\author{Boris Andrews}
\affil{Mathematical Institute, University of Oxford}
\date{\today}


\begin{document}
    \pagenumbering{gobble}
    \maketitle
    
    
    \begin{abstract}
        Magnetic confinement reactors---in particular tokamaks---offer one of the most promising options for achieving practical nuclear fusion, with the potential to provide virtually limitless, clean energy. The theoretical and numerical modeling of tokamak plasmas is simultaneously an essential component of effective reactor design, and a great research barrier. Tokamak operational conditions exhibit comparatively low Knudsen numbers. Kinetic effects, including kinetic waves and instabilities, Landau damping, bump-on-tail instabilities and more, are therefore highly influential in tokamak plasma dynamics. Purely fluid models are inherently incapable of capturing these effects, whereas the high dimensionality in purely kinetic models render them practically intractable for most relevant purposes.

        We consider a $\delta\!f$ decomposition model, with a macroscopic fluid background and microscopic kinetic correction, both fully coupled to each other. A similar manner of discretization is proposed to that used in the recent \texttt{STRUPHY} code \cite{Holderied_Possanner_Wang_2021, Holderied_2022, Li_et_al_2023} with a finite-element model for the background and a pseudo-particle/PiC model for the correction.

        The fluid background satisfies the full, non-linear, resistive, compressible, Hall MHD equations. \cite{Laakmann_Hu_Farrell_2022} introduces finite-element(-in-space) implicit timesteppers for the incompressible analogue to this system with structure-preserving (SP) properties in the ideal case, alongside parameter-robust preconditioners. We show that these timesteppers can derive from a finite-element-in-time (FET) (and finite-element-in-space) interpretation. The benefits of this reformulation are discussed, including the derivation of timesteppers that are higher order in time, and the quantifiable dissipative SP properties in the non-ideal, resistive case.
        
        We discuss possible options for extending this FET approach to timesteppers for the compressible case.

        The kinetic corrections satisfy linearized Boltzmann equations. Using a Lénard--Bernstein collision operator, these take Fokker--Planck-like forms \cite{Fokker_1914, Planck_1917} wherein pseudo-particles in the numerical model obey the neoclassical transport equations, with particle-independent Brownian drift terms. This offers a rigorous methodology for incorporating collisions into the particle transport model, without coupling the equations of motions for each particle.
        
        Works by Chen, Chacón et al. \cite{Chen_Chacón_Barnes_2011, Chacón_Chen_Barnes_2013, Chen_Chacón_2014, Chen_Chacón_2015} have developed structure-preserving particle pushers for neoclassical transport in the Vlasov equations, derived from Crank--Nicolson integrators. We show these too can can derive from a FET interpretation, similarly offering potential extensions to higher-order-in-time particle pushers. The FET formulation is used also to consider how the stochastic drift terms can be incorporated into the pushers. Stochastic gyrokinetic expansions are also discussed.

        Different options for the numerical implementation of these schemes are considered.

        Due to the efficacy of FET in the development of SP timesteppers for both the fluid and kinetic component, we hope this approach will prove effective in the future for developing SP timesteppers for the full hybrid model. We hope this will give us the opportunity to incorporate previously inaccessible kinetic effects into the highly effective, modern, finite-element MHD models.
    \end{abstract}
    
    
    \newpage
    \tableofcontents
    
    
    \newpage
    \pagenumbering{arabic}
    %\linenumbers\renewcommand\thelinenumber{\color{black!50}\arabic{linenumber}}
            \documentclass[12pt, a4paper]{report}

\input{template/main.tex}

\title{\BA{Title in Progress...}}
\author{Boris Andrews}
\affil{Mathematical Institute, University of Oxford}
\date{\today}


\begin{document}
    \pagenumbering{gobble}
    \maketitle
    
    
    \begin{abstract}
        Magnetic confinement reactors---in particular tokamaks---offer one of the most promising options for achieving practical nuclear fusion, with the potential to provide virtually limitless, clean energy. The theoretical and numerical modeling of tokamak plasmas is simultaneously an essential component of effective reactor design, and a great research barrier. Tokamak operational conditions exhibit comparatively low Knudsen numbers. Kinetic effects, including kinetic waves and instabilities, Landau damping, bump-on-tail instabilities and more, are therefore highly influential in tokamak plasma dynamics. Purely fluid models are inherently incapable of capturing these effects, whereas the high dimensionality in purely kinetic models render them practically intractable for most relevant purposes.

        We consider a $\delta\!f$ decomposition model, with a macroscopic fluid background and microscopic kinetic correction, both fully coupled to each other. A similar manner of discretization is proposed to that used in the recent \texttt{STRUPHY} code \cite{Holderied_Possanner_Wang_2021, Holderied_2022, Li_et_al_2023} with a finite-element model for the background and a pseudo-particle/PiC model for the correction.

        The fluid background satisfies the full, non-linear, resistive, compressible, Hall MHD equations. \cite{Laakmann_Hu_Farrell_2022} introduces finite-element(-in-space) implicit timesteppers for the incompressible analogue to this system with structure-preserving (SP) properties in the ideal case, alongside parameter-robust preconditioners. We show that these timesteppers can derive from a finite-element-in-time (FET) (and finite-element-in-space) interpretation. The benefits of this reformulation are discussed, including the derivation of timesteppers that are higher order in time, and the quantifiable dissipative SP properties in the non-ideal, resistive case.
        
        We discuss possible options for extending this FET approach to timesteppers for the compressible case.

        The kinetic corrections satisfy linearized Boltzmann equations. Using a Lénard--Bernstein collision operator, these take Fokker--Planck-like forms \cite{Fokker_1914, Planck_1917} wherein pseudo-particles in the numerical model obey the neoclassical transport equations, with particle-independent Brownian drift terms. This offers a rigorous methodology for incorporating collisions into the particle transport model, without coupling the equations of motions for each particle.
        
        Works by Chen, Chacón et al. \cite{Chen_Chacón_Barnes_2011, Chacón_Chen_Barnes_2013, Chen_Chacón_2014, Chen_Chacón_2015} have developed structure-preserving particle pushers for neoclassical transport in the Vlasov equations, derived from Crank--Nicolson integrators. We show these too can can derive from a FET interpretation, similarly offering potential extensions to higher-order-in-time particle pushers. The FET formulation is used also to consider how the stochastic drift terms can be incorporated into the pushers. Stochastic gyrokinetic expansions are also discussed.

        Different options for the numerical implementation of these schemes are considered.

        Due to the efficacy of FET in the development of SP timesteppers for both the fluid and kinetic component, we hope this approach will prove effective in the future for developing SP timesteppers for the full hybrid model. We hope this will give us the opportunity to incorporate previously inaccessible kinetic effects into the highly effective, modern, finite-element MHD models.
    \end{abstract}
    
    
    \newpage
    \tableofcontents
    
    
    \newpage
    \pagenumbering{arabic}
    %\linenumbers\renewcommand\thelinenumber{\color{black!50}\arabic{linenumber}}
            \input{0 - introduction/main.tex}
        \part{Research}
            \input{1 - low-noise PiC models/main.tex}
            \input{2 - kinetic component/main.tex}
            \input{3 - fluid component/main.tex}
            \input{4 - numerical implementation/main.tex}
        \part{Project Overview}
            \input{5 - research plan/main.tex}
            \input{6 - summary/main.tex}
    
    
    %\section{}
    \newpage
    \pagenumbering{gobble}
        \printbibliography


    \newpage
    \pagenumbering{roman}
    \appendix
        \part{Appendices}
            \input{8 - Hilbert complexes/main.tex}
            \input{9 - weak conservation proofs/main.tex}
\end{document}

        \part{Research}
            \documentclass[12pt, a4paper]{report}

\input{template/main.tex}

\title{\BA{Title in Progress...}}
\author{Boris Andrews}
\affil{Mathematical Institute, University of Oxford}
\date{\today}


\begin{document}
    \pagenumbering{gobble}
    \maketitle
    
    
    \begin{abstract}
        Magnetic confinement reactors---in particular tokamaks---offer one of the most promising options for achieving practical nuclear fusion, with the potential to provide virtually limitless, clean energy. The theoretical and numerical modeling of tokamak plasmas is simultaneously an essential component of effective reactor design, and a great research barrier. Tokamak operational conditions exhibit comparatively low Knudsen numbers. Kinetic effects, including kinetic waves and instabilities, Landau damping, bump-on-tail instabilities and more, are therefore highly influential in tokamak plasma dynamics. Purely fluid models are inherently incapable of capturing these effects, whereas the high dimensionality in purely kinetic models render them practically intractable for most relevant purposes.

        We consider a $\delta\!f$ decomposition model, with a macroscopic fluid background and microscopic kinetic correction, both fully coupled to each other. A similar manner of discretization is proposed to that used in the recent \texttt{STRUPHY} code \cite{Holderied_Possanner_Wang_2021, Holderied_2022, Li_et_al_2023} with a finite-element model for the background and a pseudo-particle/PiC model for the correction.

        The fluid background satisfies the full, non-linear, resistive, compressible, Hall MHD equations. \cite{Laakmann_Hu_Farrell_2022} introduces finite-element(-in-space) implicit timesteppers for the incompressible analogue to this system with structure-preserving (SP) properties in the ideal case, alongside parameter-robust preconditioners. We show that these timesteppers can derive from a finite-element-in-time (FET) (and finite-element-in-space) interpretation. The benefits of this reformulation are discussed, including the derivation of timesteppers that are higher order in time, and the quantifiable dissipative SP properties in the non-ideal, resistive case.
        
        We discuss possible options for extending this FET approach to timesteppers for the compressible case.

        The kinetic corrections satisfy linearized Boltzmann equations. Using a Lénard--Bernstein collision operator, these take Fokker--Planck-like forms \cite{Fokker_1914, Planck_1917} wherein pseudo-particles in the numerical model obey the neoclassical transport equations, with particle-independent Brownian drift terms. This offers a rigorous methodology for incorporating collisions into the particle transport model, without coupling the equations of motions for each particle.
        
        Works by Chen, Chacón et al. \cite{Chen_Chacón_Barnes_2011, Chacón_Chen_Barnes_2013, Chen_Chacón_2014, Chen_Chacón_2015} have developed structure-preserving particle pushers for neoclassical transport in the Vlasov equations, derived from Crank--Nicolson integrators. We show these too can can derive from a FET interpretation, similarly offering potential extensions to higher-order-in-time particle pushers. The FET formulation is used also to consider how the stochastic drift terms can be incorporated into the pushers. Stochastic gyrokinetic expansions are also discussed.

        Different options for the numerical implementation of these schemes are considered.

        Due to the efficacy of FET in the development of SP timesteppers for both the fluid and kinetic component, we hope this approach will prove effective in the future for developing SP timesteppers for the full hybrid model. We hope this will give us the opportunity to incorporate previously inaccessible kinetic effects into the highly effective, modern, finite-element MHD models.
    \end{abstract}
    
    
    \newpage
    \tableofcontents
    
    
    \newpage
    \pagenumbering{arabic}
    %\linenumbers\renewcommand\thelinenumber{\color{black!50}\arabic{linenumber}}
            \input{0 - introduction/main.tex}
        \part{Research}
            \input{1 - low-noise PiC models/main.tex}
            \input{2 - kinetic component/main.tex}
            \input{3 - fluid component/main.tex}
            \input{4 - numerical implementation/main.tex}
        \part{Project Overview}
            \input{5 - research plan/main.tex}
            \input{6 - summary/main.tex}
    
    
    %\section{}
    \newpage
    \pagenumbering{gobble}
        \printbibliography


    \newpage
    \pagenumbering{roman}
    \appendix
        \part{Appendices}
            \input{8 - Hilbert complexes/main.tex}
            \input{9 - weak conservation proofs/main.tex}
\end{document}

            \documentclass[12pt, a4paper]{report}

\input{template/main.tex}

\title{\BA{Title in Progress...}}
\author{Boris Andrews}
\affil{Mathematical Institute, University of Oxford}
\date{\today}


\begin{document}
    \pagenumbering{gobble}
    \maketitle
    
    
    \begin{abstract}
        Magnetic confinement reactors---in particular tokamaks---offer one of the most promising options for achieving practical nuclear fusion, with the potential to provide virtually limitless, clean energy. The theoretical and numerical modeling of tokamak plasmas is simultaneously an essential component of effective reactor design, and a great research barrier. Tokamak operational conditions exhibit comparatively low Knudsen numbers. Kinetic effects, including kinetic waves and instabilities, Landau damping, bump-on-tail instabilities and more, are therefore highly influential in tokamak plasma dynamics. Purely fluid models are inherently incapable of capturing these effects, whereas the high dimensionality in purely kinetic models render them practically intractable for most relevant purposes.

        We consider a $\delta\!f$ decomposition model, with a macroscopic fluid background and microscopic kinetic correction, both fully coupled to each other. A similar manner of discretization is proposed to that used in the recent \texttt{STRUPHY} code \cite{Holderied_Possanner_Wang_2021, Holderied_2022, Li_et_al_2023} with a finite-element model for the background and a pseudo-particle/PiC model for the correction.

        The fluid background satisfies the full, non-linear, resistive, compressible, Hall MHD equations. \cite{Laakmann_Hu_Farrell_2022} introduces finite-element(-in-space) implicit timesteppers for the incompressible analogue to this system with structure-preserving (SP) properties in the ideal case, alongside parameter-robust preconditioners. We show that these timesteppers can derive from a finite-element-in-time (FET) (and finite-element-in-space) interpretation. The benefits of this reformulation are discussed, including the derivation of timesteppers that are higher order in time, and the quantifiable dissipative SP properties in the non-ideal, resistive case.
        
        We discuss possible options for extending this FET approach to timesteppers for the compressible case.

        The kinetic corrections satisfy linearized Boltzmann equations. Using a Lénard--Bernstein collision operator, these take Fokker--Planck-like forms \cite{Fokker_1914, Planck_1917} wherein pseudo-particles in the numerical model obey the neoclassical transport equations, with particle-independent Brownian drift terms. This offers a rigorous methodology for incorporating collisions into the particle transport model, without coupling the equations of motions for each particle.
        
        Works by Chen, Chacón et al. \cite{Chen_Chacón_Barnes_2011, Chacón_Chen_Barnes_2013, Chen_Chacón_2014, Chen_Chacón_2015} have developed structure-preserving particle pushers for neoclassical transport in the Vlasov equations, derived from Crank--Nicolson integrators. We show these too can can derive from a FET interpretation, similarly offering potential extensions to higher-order-in-time particle pushers. The FET formulation is used also to consider how the stochastic drift terms can be incorporated into the pushers. Stochastic gyrokinetic expansions are also discussed.

        Different options for the numerical implementation of these schemes are considered.

        Due to the efficacy of FET in the development of SP timesteppers for both the fluid and kinetic component, we hope this approach will prove effective in the future for developing SP timesteppers for the full hybrid model. We hope this will give us the opportunity to incorporate previously inaccessible kinetic effects into the highly effective, modern, finite-element MHD models.
    \end{abstract}
    
    
    \newpage
    \tableofcontents
    
    
    \newpage
    \pagenumbering{arabic}
    %\linenumbers\renewcommand\thelinenumber{\color{black!50}\arabic{linenumber}}
            \input{0 - introduction/main.tex}
        \part{Research}
            \input{1 - low-noise PiC models/main.tex}
            \input{2 - kinetic component/main.tex}
            \input{3 - fluid component/main.tex}
            \input{4 - numerical implementation/main.tex}
        \part{Project Overview}
            \input{5 - research plan/main.tex}
            \input{6 - summary/main.tex}
    
    
    %\section{}
    \newpage
    \pagenumbering{gobble}
        \printbibliography


    \newpage
    \pagenumbering{roman}
    \appendix
        \part{Appendices}
            \input{8 - Hilbert complexes/main.tex}
            \input{9 - weak conservation proofs/main.tex}
\end{document}

            \documentclass[12pt, a4paper]{report}

\input{template/main.tex}

\title{\BA{Title in Progress...}}
\author{Boris Andrews}
\affil{Mathematical Institute, University of Oxford}
\date{\today}


\begin{document}
    \pagenumbering{gobble}
    \maketitle
    
    
    \begin{abstract}
        Magnetic confinement reactors---in particular tokamaks---offer one of the most promising options for achieving practical nuclear fusion, with the potential to provide virtually limitless, clean energy. The theoretical and numerical modeling of tokamak plasmas is simultaneously an essential component of effective reactor design, and a great research barrier. Tokamak operational conditions exhibit comparatively low Knudsen numbers. Kinetic effects, including kinetic waves and instabilities, Landau damping, bump-on-tail instabilities and more, are therefore highly influential in tokamak plasma dynamics. Purely fluid models are inherently incapable of capturing these effects, whereas the high dimensionality in purely kinetic models render them practically intractable for most relevant purposes.

        We consider a $\delta\!f$ decomposition model, with a macroscopic fluid background and microscopic kinetic correction, both fully coupled to each other. A similar manner of discretization is proposed to that used in the recent \texttt{STRUPHY} code \cite{Holderied_Possanner_Wang_2021, Holderied_2022, Li_et_al_2023} with a finite-element model for the background and a pseudo-particle/PiC model for the correction.

        The fluid background satisfies the full, non-linear, resistive, compressible, Hall MHD equations. \cite{Laakmann_Hu_Farrell_2022} introduces finite-element(-in-space) implicit timesteppers for the incompressible analogue to this system with structure-preserving (SP) properties in the ideal case, alongside parameter-robust preconditioners. We show that these timesteppers can derive from a finite-element-in-time (FET) (and finite-element-in-space) interpretation. The benefits of this reformulation are discussed, including the derivation of timesteppers that are higher order in time, and the quantifiable dissipative SP properties in the non-ideal, resistive case.
        
        We discuss possible options for extending this FET approach to timesteppers for the compressible case.

        The kinetic corrections satisfy linearized Boltzmann equations. Using a Lénard--Bernstein collision operator, these take Fokker--Planck-like forms \cite{Fokker_1914, Planck_1917} wherein pseudo-particles in the numerical model obey the neoclassical transport equations, with particle-independent Brownian drift terms. This offers a rigorous methodology for incorporating collisions into the particle transport model, without coupling the equations of motions for each particle.
        
        Works by Chen, Chacón et al. \cite{Chen_Chacón_Barnes_2011, Chacón_Chen_Barnes_2013, Chen_Chacón_2014, Chen_Chacón_2015} have developed structure-preserving particle pushers for neoclassical transport in the Vlasov equations, derived from Crank--Nicolson integrators. We show these too can can derive from a FET interpretation, similarly offering potential extensions to higher-order-in-time particle pushers. The FET formulation is used also to consider how the stochastic drift terms can be incorporated into the pushers. Stochastic gyrokinetic expansions are also discussed.

        Different options for the numerical implementation of these schemes are considered.

        Due to the efficacy of FET in the development of SP timesteppers for both the fluid and kinetic component, we hope this approach will prove effective in the future for developing SP timesteppers for the full hybrid model. We hope this will give us the opportunity to incorporate previously inaccessible kinetic effects into the highly effective, modern, finite-element MHD models.
    \end{abstract}
    
    
    \newpage
    \tableofcontents
    
    
    \newpage
    \pagenumbering{arabic}
    %\linenumbers\renewcommand\thelinenumber{\color{black!50}\arabic{linenumber}}
            \input{0 - introduction/main.tex}
        \part{Research}
            \input{1 - low-noise PiC models/main.tex}
            \input{2 - kinetic component/main.tex}
            \input{3 - fluid component/main.tex}
            \input{4 - numerical implementation/main.tex}
        \part{Project Overview}
            \input{5 - research plan/main.tex}
            \input{6 - summary/main.tex}
    
    
    %\section{}
    \newpage
    \pagenumbering{gobble}
        \printbibliography


    \newpage
    \pagenumbering{roman}
    \appendix
        \part{Appendices}
            \input{8 - Hilbert complexes/main.tex}
            \input{9 - weak conservation proofs/main.tex}
\end{document}

            \documentclass[12pt, a4paper]{report}

\input{template/main.tex}

\title{\BA{Title in Progress...}}
\author{Boris Andrews}
\affil{Mathematical Institute, University of Oxford}
\date{\today}


\begin{document}
    \pagenumbering{gobble}
    \maketitle
    
    
    \begin{abstract}
        Magnetic confinement reactors---in particular tokamaks---offer one of the most promising options for achieving practical nuclear fusion, with the potential to provide virtually limitless, clean energy. The theoretical and numerical modeling of tokamak plasmas is simultaneously an essential component of effective reactor design, and a great research barrier. Tokamak operational conditions exhibit comparatively low Knudsen numbers. Kinetic effects, including kinetic waves and instabilities, Landau damping, bump-on-tail instabilities and more, are therefore highly influential in tokamak plasma dynamics. Purely fluid models are inherently incapable of capturing these effects, whereas the high dimensionality in purely kinetic models render them practically intractable for most relevant purposes.

        We consider a $\delta\!f$ decomposition model, with a macroscopic fluid background and microscopic kinetic correction, both fully coupled to each other. A similar manner of discretization is proposed to that used in the recent \texttt{STRUPHY} code \cite{Holderied_Possanner_Wang_2021, Holderied_2022, Li_et_al_2023} with a finite-element model for the background and a pseudo-particle/PiC model for the correction.

        The fluid background satisfies the full, non-linear, resistive, compressible, Hall MHD equations. \cite{Laakmann_Hu_Farrell_2022} introduces finite-element(-in-space) implicit timesteppers for the incompressible analogue to this system with structure-preserving (SP) properties in the ideal case, alongside parameter-robust preconditioners. We show that these timesteppers can derive from a finite-element-in-time (FET) (and finite-element-in-space) interpretation. The benefits of this reformulation are discussed, including the derivation of timesteppers that are higher order in time, and the quantifiable dissipative SP properties in the non-ideal, resistive case.
        
        We discuss possible options for extending this FET approach to timesteppers for the compressible case.

        The kinetic corrections satisfy linearized Boltzmann equations. Using a Lénard--Bernstein collision operator, these take Fokker--Planck-like forms \cite{Fokker_1914, Planck_1917} wherein pseudo-particles in the numerical model obey the neoclassical transport equations, with particle-independent Brownian drift terms. This offers a rigorous methodology for incorporating collisions into the particle transport model, without coupling the equations of motions for each particle.
        
        Works by Chen, Chacón et al. \cite{Chen_Chacón_Barnes_2011, Chacón_Chen_Barnes_2013, Chen_Chacón_2014, Chen_Chacón_2015} have developed structure-preserving particle pushers for neoclassical transport in the Vlasov equations, derived from Crank--Nicolson integrators. We show these too can can derive from a FET interpretation, similarly offering potential extensions to higher-order-in-time particle pushers. The FET formulation is used also to consider how the stochastic drift terms can be incorporated into the pushers. Stochastic gyrokinetic expansions are also discussed.

        Different options for the numerical implementation of these schemes are considered.

        Due to the efficacy of FET in the development of SP timesteppers for both the fluid and kinetic component, we hope this approach will prove effective in the future for developing SP timesteppers for the full hybrid model. We hope this will give us the opportunity to incorporate previously inaccessible kinetic effects into the highly effective, modern, finite-element MHD models.
    \end{abstract}
    
    
    \newpage
    \tableofcontents
    
    
    \newpage
    \pagenumbering{arabic}
    %\linenumbers\renewcommand\thelinenumber{\color{black!50}\arabic{linenumber}}
            \input{0 - introduction/main.tex}
        \part{Research}
            \input{1 - low-noise PiC models/main.tex}
            \input{2 - kinetic component/main.tex}
            \input{3 - fluid component/main.tex}
            \input{4 - numerical implementation/main.tex}
        \part{Project Overview}
            \input{5 - research plan/main.tex}
            \input{6 - summary/main.tex}
    
    
    %\section{}
    \newpage
    \pagenumbering{gobble}
        \printbibliography


    \newpage
    \pagenumbering{roman}
    \appendix
        \part{Appendices}
            \input{8 - Hilbert complexes/main.tex}
            \input{9 - weak conservation proofs/main.tex}
\end{document}

        \part{Project Overview}
            \documentclass[12pt, a4paper]{report}

\input{template/main.tex}

\title{\BA{Title in Progress...}}
\author{Boris Andrews}
\affil{Mathematical Institute, University of Oxford}
\date{\today}


\begin{document}
    \pagenumbering{gobble}
    \maketitle
    
    
    \begin{abstract}
        Magnetic confinement reactors---in particular tokamaks---offer one of the most promising options for achieving practical nuclear fusion, with the potential to provide virtually limitless, clean energy. The theoretical and numerical modeling of tokamak plasmas is simultaneously an essential component of effective reactor design, and a great research barrier. Tokamak operational conditions exhibit comparatively low Knudsen numbers. Kinetic effects, including kinetic waves and instabilities, Landau damping, bump-on-tail instabilities and more, are therefore highly influential in tokamak plasma dynamics. Purely fluid models are inherently incapable of capturing these effects, whereas the high dimensionality in purely kinetic models render them practically intractable for most relevant purposes.

        We consider a $\delta\!f$ decomposition model, with a macroscopic fluid background and microscopic kinetic correction, both fully coupled to each other. A similar manner of discretization is proposed to that used in the recent \texttt{STRUPHY} code \cite{Holderied_Possanner_Wang_2021, Holderied_2022, Li_et_al_2023} with a finite-element model for the background and a pseudo-particle/PiC model for the correction.

        The fluid background satisfies the full, non-linear, resistive, compressible, Hall MHD equations. \cite{Laakmann_Hu_Farrell_2022} introduces finite-element(-in-space) implicit timesteppers for the incompressible analogue to this system with structure-preserving (SP) properties in the ideal case, alongside parameter-robust preconditioners. We show that these timesteppers can derive from a finite-element-in-time (FET) (and finite-element-in-space) interpretation. The benefits of this reformulation are discussed, including the derivation of timesteppers that are higher order in time, and the quantifiable dissipative SP properties in the non-ideal, resistive case.
        
        We discuss possible options for extending this FET approach to timesteppers for the compressible case.

        The kinetic corrections satisfy linearized Boltzmann equations. Using a Lénard--Bernstein collision operator, these take Fokker--Planck-like forms \cite{Fokker_1914, Planck_1917} wherein pseudo-particles in the numerical model obey the neoclassical transport equations, with particle-independent Brownian drift terms. This offers a rigorous methodology for incorporating collisions into the particle transport model, without coupling the equations of motions for each particle.
        
        Works by Chen, Chacón et al. \cite{Chen_Chacón_Barnes_2011, Chacón_Chen_Barnes_2013, Chen_Chacón_2014, Chen_Chacón_2015} have developed structure-preserving particle pushers for neoclassical transport in the Vlasov equations, derived from Crank--Nicolson integrators. We show these too can can derive from a FET interpretation, similarly offering potential extensions to higher-order-in-time particle pushers. The FET formulation is used also to consider how the stochastic drift terms can be incorporated into the pushers. Stochastic gyrokinetic expansions are also discussed.

        Different options for the numerical implementation of these schemes are considered.

        Due to the efficacy of FET in the development of SP timesteppers for both the fluid and kinetic component, we hope this approach will prove effective in the future for developing SP timesteppers for the full hybrid model. We hope this will give us the opportunity to incorporate previously inaccessible kinetic effects into the highly effective, modern, finite-element MHD models.
    \end{abstract}
    
    
    \newpage
    \tableofcontents
    
    
    \newpage
    \pagenumbering{arabic}
    %\linenumbers\renewcommand\thelinenumber{\color{black!50}\arabic{linenumber}}
            \input{0 - introduction/main.tex}
        \part{Research}
            \input{1 - low-noise PiC models/main.tex}
            \input{2 - kinetic component/main.tex}
            \input{3 - fluid component/main.tex}
            \input{4 - numerical implementation/main.tex}
        \part{Project Overview}
            \input{5 - research plan/main.tex}
            \input{6 - summary/main.tex}
    
    
    %\section{}
    \newpage
    \pagenumbering{gobble}
        \printbibliography


    \newpage
    \pagenumbering{roman}
    \appendix
        \part{Appendices}
            \input{8 - Hilbert complexes/main.tex}
            \input{9 - weak conservation proofs/main.tex}
\end{document}

            \documentclass[12pt, a4paper]{report}

\input{template/main.tex}

\title{\BA{Title in Progress...}}
\author{Boris Andrews}
\affil{Mathematical Institute, University of Oxford}
\date{\today}


\begin{document}
    \pagenumbering{gobble}
    \maketitle
    
    
    \begin{abstract}
        Magnetic confinement reactors---in particular tokamaks---offer one of the most promising options for achieving practical nuclear fusion, with the potential to provide virtually limitless, clean energy. The theoretical and numerical modeling of tokamak plasmas is simultaneously an essential component of effective reactor design, and a great research barrier. Tokamak operational conditions exhibit comparatively low Knudsen numbers. Kinetic effects, including kinetic waves and instabilities, Landau damping, bump-on-tail instabilities and more, are therefore highly influential in tokamak plasma dynamics. Purely fluid models are inherently incapable of capturing these effects, whereas the high dimensionality in purely kinetic models render them practically intractable for most relevant purposes.

        We consider a $\delta\!f$ decomposition model, with a macroscopic fluid background and microscopic kinetic correction, both fully coupled to each other. A similar manner of discretization is proposed to that used in the recent \texttt{STRUPHY} code \cite{Holderied_Possanner_Wang_2021, Holderied_2022, Li_et_al_2023} with a finite-element model for the background and a pseudo-particle/PiC model for the correction.

        The fluid background satisfies the full, non-linear, resistive, compressible, Hall MHD equations. \cite{Laakmann_Hu_Farrell_2022} introduces finite-element(-in-space) implicit timesteppers for the incompressible analogue to this system with structure-preserving (SP) properties in the ideal case, alongside parameter-robust preconditioners. We show that these timesteppers can derive from a finite-element-in-time (FET) (and finite-element-in-space) interpretation. The benefits of this reformulation are discussed, including the derivation of timesteppers that are higher order in time, and the quantifiable dissipative SP properties in the non-ideal, resistive case.
        
        We discuss possible options for extending this FET approach to timesteppers for the compressible case.

        The kinetic corrections satisfy linearized Boltzmann equations. Using a Lénard--Bernstein collision operator, these take Fokker--Planck-like forms \cite{Fokker_1914, Planck_1917} wherein pseudo-particles in the numerical model obey the neoclassical transport equations, with particle-independent Brownian drift terms. This offers a rigorous methodology for incorporating collisions into the particle transport model, without coupling the equations of motions for each particle.
        
        Works by Chen, Chacón et al. \cite{Chen_Chacón_Barnes_2011, Chacón_Chen_Barnes_2013, Chen_Chacón_2014, Chen_Chacón_2015} have developed structure-preserving particle pushers for neoclassical transport in the Vlasov equations, derived from Crank--Nicolson integrators. We show these too can can derive from a FET interpretation, similarly offering potential extensions to higher-order-in-time particle pushers. The FET formulation is used also to consider how the stochastic drift terms can be incorporated into the pushers. Stochastic gyrokinetic expansions are also discussed.

        Different options for the numerical implementation of these schemes are considered.

        Due to the efficacy of FET in the development of SP timesteppers for both the fluid and kinetic component, we hope this approach will prove effective in the future for developing SP timesteppers for the full hybrid model. We hope this will give us the opportunity to incorporate previously inaccessible kinetic effects into the highly effective, modern, finite-element MHD models.
    \end{abstract}
    
    
    \newpage
    \tableofcontents
    
    
    \newpage
    \pagenumbering{arabic}
    %\linenumbers\renewcommand\thelinenumber{\color{black!50}\arabic{linenumber}}
            \input{0 - introduction/main.tex}
        \part{Research}
            \input{1 - low-noise PiC models/main.tex}
            \input{2 - kinetic component/main.tex}
            \input{3 - fluid component/main.tex}
            \input{4 - numerical implementation/main.tex}
        \part{Project Overview}
            \input{5 - research plan/main.tex}
            \input{6 - summary/main.tex}
    
    
    %\section{}
    \newpage
    \pagenumbering{gobble}
        \printbibliography


    \newpage
    \pagenumbering{roman}
    \appendix
        \part{Appendices}
            \input{8 - Hilbert complexes/main.tex}
            \input{9 - weak conservation proofs/main.tex}
\end{document}

    
    
    %\section{}
    \newpage
    \pagenumbering{gobble}
        \printbibliography


    \newpage
    \pagenumbering{roman}
    \appendix
        \part{Appendices}
            \documentclass[12pt, a4paper]{report}

\input{template/main.tex}

\title{\BA{Title in Progress...}}
\author{Boris Andrews}
\affil{Mathematical Institute, University of Oxford}
\date{\today}


\begin{document}
    \pagenumbering{gobble}
    \maketitle
    
    
    \begin{abstract}
        Magnetic confinement reactors---in particular tokamaks---offer one of the most promising options for achieving practical nuclear fusion, with the potential to provide virtually limitless, clean energy. The theoretical and numerical modeling of tokamak plasmas is simultaneously an essential component of effective reactor design, and a great research barrier. Tokamak operational conditions exhibit comparatively low Knudsen numbers. Kinetic effects, including kinetic waves and instabilities, Landau damping, bump-on-tail instabilities and more, are therefore highly influential in tokamak plasma dynamics. Purely fluid models are inherently incapable of capturing these effects, whereas the high dimensionality in purely kinetic models render them practically intractable for most relevant purposes.

        We consider a $\delta\!f$ decomposition model, with a macroscopic fluid background and microscopic kinetic correction, both fully coupled to each other. A similar manner of discretization is proposed to that used in the recent \texttt{STRUPHY} code \cite{Holderied_Possanner_Wang_2021, Holderied_2022, Li_et_al_2023} with a finite-element model for the background and a pseudo-particle/PiC model for the correction.

        The fluid background satisfies the full, non-linear, resistive, compressible, Hall MHD equations. \cite{Laakmann_Hu_Farrell_2022} introduces finite-element(-in-space) implicit timesteppers for the incompressible analogue to this system with structure-preserving (SP) properties in the ideal case, alongside parameter-robust preconditioners. We show that these timesteppers can derive from a finite-element-in-time (FET) (and finite-element-in-space) interpretation. The benefits of this reformulation are discussed, including the derivation of timesteppers that are higher order in time, and the quantifiable dissipative SP properties in the non-ideal, resistive case.
        
        We discuss possible options for extending this FET approach to timesteppers for the compressible case.

        The kinetic corrections satisfy linearized Boltzmann equations. Using a Lénard--Bernstein collision operator, these take Fokker--Planck-like forms \cite{Fokker_1914, Planck_1917} wherein pseudo-particles in the numerical model obey the neoclassical transport equations, with particle-independent Brownian drift terms. This offers a rigorous methodology for incorporating collisions into the particle transport model, without coupling the equations of motions for each particle.
        
        Works by Chen, Chacón et al. \cite{Chen_Chacón_Barnes_2011, Chacón_Chen_Barnes_2013, Chen_Chacón_2014, Chen_Chacón_2015} have developed structure-preserving particle pushers for neoclassical transport in the Vlasov equations, derived from Crank--Nicolson integrators. We show these too can can derive from a FET interpretation, similarly offering potential extensions to higher-order-in-time particle pushers. The FET formulation is used also to consider how the stochastic drift terms can be incorporated into the pushers. Stochastic gyrokinetic expansions are also discussed.

        Different options for the numerical implementation of these schemes are considered.

        Due to the efficacy of FET in the development of SP timesteppers for both the fluid and kinetic component, we hope this approach will prove effective in the future for developing SP timesteppers for the full hybrid model. We hope this will give us the opportunity to incorporate previously inaccessible kinetic effects into the highly effective, modern, finite-element MHD models.
    \end{abstract}
    
    
    \newpage
    \tableofcontents
    
    
    \newpage
    \pagenumbering{arabic}
    %\linenumbers\renewcommand\thelinenumber{\color{black!50}\arabic{linenumber}}
            \input{0 - introduction/main.tex}
        \part{Research}
            \input{1 - low-noise PiC models/main.tex}
            \input{2 - kinetic component/main.tex}
            \input{3 - fluid component/main.tex}
            \input{4 - numerical implementation/main.tex}
        \part{Project Overview}
            \input{5 - research plan/main.tex}
            \input{6 - summary/main.tex}
    
    
    %\section{}
    \newpage
    \pagenumbering{gobble}
        \printbibliography


    \newpage
    \pagenumbering{roman}
    \appendix
        \part{Appendices}
            \input{8 - Hilbert complexes/main.tex}
            \input{9 - weak conservation proofs/main.tex}
\end{document}

            \documentclass[12pt, a4paper]{report}

\input{template/main.tex}

\title{\BA{Title in Progress...}}
\author{Boris Andrews}
\affil{Mathematical Institute, University of Oxford}
\date{\today}


\begin{document}
    \pagenumbering{gobble}
    \maketitle
    
    
    \begin{abstract}
        Magnetic confinement reactors---in particular tokamaks---offer one of the most promising options for achieving practical nuclear fusion, with the potential to provide virtually limitless, clean energy. The theoretical and numerical modeling of tokamak plasmas is simultaneously an essential component of effective reactor design, and a great research barrier. Tokamak operational conditions exhibit comparatively low Knudsen numbers. Kinetic effects, including kinetic waves and instabilities, Landau damping, bump-on-tail instabilities and more, are therefore highly influential in tokamak plasma dynamics. Purely fluid models are inherently incapable of capturing these effects, whereas the high dimensionality in purely kinetic models render them practically intractable for most relevant purposes.

        We consider a $\delta\!f$ decomposition model, with a macroscopic fluid background and microscopic kinetic correction, both fully coupled to each other. A similar manner of discretization is proposed to that used in the recent \texttt{STRUPHY} code \cite{Holderied_Possanner_Wang_2021, Holderied_2022, Li_et_al_2023} with a finite-element model for the background and a pseudo-particle/PiC model for the correction.

        The fluid background satisfies the full, non-linear, resistive, compressible, Hall MHD equations. \cite{Laakmann_Hu_Farrell_2022} introduces finite-element(-in-space) implicit timesteppers for the incompressible analogue to this system with structure-preserving (SP) properties in the ideal case, alongside parameter-robust preconditioners. We show that these timesteppers can derive from a finite-element-in-time (FET) (and finite-element-in-space) interpretation. The benefits of this reformulation are discussed, including the derivation of timesteppers that are higher order in time, and the quantifiable dissipative SP properties in the non-ideal, resistive case.
        
        We discuss possible options for extending this FET approach to timesteppers for the compressible case.

        The kinetic corrections satisfy linearized Boltzmann equations. Using a Lénard--Bernstein collision operator, these take Fokker--Planck-like forms \cite{Fokker_1914, Planck_1917} wherein pseudo-particles in the numerical model obey the neoclassical transport equations, with particle-independent Brownian drift terms. This offers a rigorous methodology for incorporating collisions into the particle transport model, without coupling the equations of motions for each particle.
        
        Works by Chen, Chacón et al. \cite{Chen_Chacón_Barnes_2011, Chacón_Chen_Barnes_2013, Chen_Chacón_2014, Chen_Chacón_2015} have developed structure-preserving particle pushers for neoclassical transport in the Vlasov equations, derived from Crank--Nicolson integrators. We show these too can can derive from a FET interpretation, similarly offering potential extensions to higher-order-in-time particle pushers. The FET formulation is used also to consider how the stochastic drift terms can be incorporated into the pushers. Stochastic gyrokinetic expansions are also discussed.

        Different options for the numerical implementation of these schemes are considered.

        Due to the efficacy of FET in the development of SP timesteppers for both the fluid and kinetic component, we hope this approach will prove effective in the future for developing SP timesteppers for the full hybrid model. We hope this will give us the opportunity to incorporate previously inaccessible kinetic effects into the highly effective, modern, finite-element MHD models.
    \end{abstract}
    
    
    \newpage
    \tableofcontents
    
    
    \newpage
    \pagenumbering{arabic}
    %\linenumbers\renewcommand\thelinenumber{\color{black!50}\arabic{linenumber}}
            \input{0 - introduction/main.tex}
        \part{Research}
            \input{1 - low-noise PiC models/main.tex}
            \input{2 - kinetic component/main.tex}
            \input{3 - fluid component/main.tex}
            \input{4 - numerical implementation/main.tex}
        \part{Project Overview}
            \input{5 - research plan/main.tex}
            \input{6 - summary/main.tex}
    
    
    %\section{}
    \newpage
    \pagenumbering{gobble}
        \printbibliography


    \newpage
    \pagenumbering{roman}
    \appendix
        \part{Appendices}
            \input{8 - Hilbert complexes/main.tex}
            \input{9 - weak conservation proofs/main.tex}
\end{document}

\end{document}

    
    
    %\section{}
    \newpage
    \pagenumbering{gobble}
        \printbibliography


    \newpage
    \pagenumbering{roman}
    \appendix
        \part{Appendices}
            \documentclass[12pt, a4paper]{report}

\documentclass[12pt, a4paper]{report}

\input{template/main.tex}

\title{\BA{Title in Progress...}}
\author{Boris Andrews}
\affil{Mathematical Institute, University of Oxford}
\date{\today}


\begin{document}
    \pagenumbering{gobble}
    \maketitle
    
    
    \begin{abstract}
        Magnetic confinement reactors---in particular tokamaks---offer one of the most promising options for achieving practical nuclear fusion, with the potential to provide virtually limitless, clean energy. The theoretical and numerical modeling of tokamak plasmas is simultaneously an essential component of effective reactor design, and a great research barrier. Tokamak operational conditions exhibit comparatively low Knudsen numbers. Kinetic effects, including kinetic waves and instabilities, Landau damping, bump-on-tail instabilities and more, are therefore highly influential in tokamak plasma dynamics. Purely fluid models are inherently incapable of capturing these effects, whereas the high dimensionality in purely kinetic models render them practically intractable for most relevant purposes.

        We consider a $\delta\!f$ decomposition model, with a macroscopic fluid background and microscopic kinetic correction, both fully coupled to each other. A similar manner of discretization is proposed to that used in the recent \texttt{STRUPHY} code \cite{Holderied_Possanner_Wang_2021, Holderied_2022, Li_et_al_2023} with a finite-element model for the background and a pseudo-particle/PiC model for the correction.

        The fluid background satisfies the full, non-linear, resistive, compressible, Hall MHD equations. \cite{Laakmann_Hu_Farrell_2022} introduces finite-element(-in-space) implicit timesteppers for the incompressible analogue to this system with structure-preserving (SP) properties in the ideal case, alongside parameter-robust preconditioners. We show that these timesteppers can derive from a finite-element-in-time (FET) (and finite-element-in-space) interpretation. The benefits of this reformulation are discussed, including the derivation of timesteppers that are higher order in time, and the quantifiable dissipative SP properties in the non-ideal, resistive case.
        
        We discuss possible options for extending this FET approach to timesteppers for the compressible case.

        The kinetic corrections satisfy linearized Boltzmann equations. Using a Lénard--Bernstein collision operator, these take Fokker--Planck-like forms \cite{Fokker_1914, Planck_1917} wherein pseudo-particles in the numerical model obey the neoclassical transport equations, with particle-independent Brownian drift terms. This offers a rigorous methodology for incorporating collisions into the particle transport model, without coupling the equations of motions for each particle.
        
        Works by Chen, Chacón et al. \cite{Chen_Chacón_Barnes_2011, Chacón_Chen_Barnes_2013, Chen_Chacón_2014, Chen_Chacón_2015} have developed structure-preserving particle pushers for neoclassical transport in the Vlasov equations, derived from Crank--Nicolson integrators. We show these too can can derive from a FET interpretation, similarly offering potential extensions to higher-order-in-time particle pushers. The FET formulation is used also to consider how the stochastic drift terms can be incorporated into the pushers. Stochastic gyrokinetic expansions are also discussed.

        Different options for the numerical implementation of these schemes are considered.

        Due to the efficacy of FET in the development of SP timesteppers for both the fluid and kinetic component, we hope this approach will prove effective in the future for developing SP timesteppers for the full hybrid model. We hope this will give us the opportunity to incorporate previously inaccessible kinetic effects into the highly effective, modern, finite-element MHD models.
    \end{abstract}
    
    
    \newpage
    \tableofcontents
    
    
    \newpage
    \pagenumbering{arabic}
    %\linenumbers\renewcommand\thelinenumber{\color{black!50}\arabic{linenumber}}
            \input{0 - introduction/main.tex}
        \part{Research}
            \input{1 - low-noise PiC models/main.tex}
            \input{2 - kinetic component/main.tex}
            \input{3 - fluid component/main.tex}
            \input{4 - numerical implementation/main.tex}
        \part{Project Overview}
            \input{5 - research plan/main.tex}
            \input{6 - summary/main.tex}
    
    
    %\section{}
    \newpage
    \pagenumbering{gobble}
        \printbibliography


    \newpage
    \pagenumbering{roman}
    \appendix
        \part{Appendices}
            \input{8 - Hilbert complexes/main.tex}
            \input{9 - weak conservation proofs/main.tex}
\end{document}


\title{\BA{Title in Progress...}}
\author{Boris Andrews}
\affil{Mathematical Institute, University of Oxford}
\date{\today}


\begin{document}
    \pagenumbering{gobble}
    \maketitle
    
    
    \begin{abstract}
        Magnetic confinement reactors---in particular tokamaks---offer one of the most promising options for achieving practical nuclear fusion, with the potential to provide virtually limitless, clean energy. The theoretical and numerical modeling of tokamak plasmas is simultaneously an essential component of effective reactor design, and a great research barrier. Tokamak operational conditions exhibit comparatively low Knudsen numbers. Kinetic effects, including kinetic waves and instabilities, Landau damping, bump-on-tail instabilities and more, are therefore highly influential in tokamak plasma dynamics. Purely fluid models are inherently incapable of capturing these effects, whereas the high dimensionality in purely kinetic models render them practically intractable for most relevant purposes.

        We consider a $\delta\!f$ decomposition model, with a macroscopic fluid background and microscopic kinetic correction, both fully coupled to each other. A similar manner of discretization is proposed to that used in the recent \texttt{STRUPHY} code \cite{Holderied_Possanner_Wang_2021, Holderied_2022, Li_et_al_2023} with a finite-element model for the background and a pseudo-particle/PiC model for the correction.

        The fluid background satisfies the full, non-linear, resistive, compressible, Hall MHD equations. \cite{Laakmann_Hu_Farrell_2022} introduces finite-element(-in-space) implicit timesteppers for the incompressible analogue to this system with structure-preserving (SP) properties in the ideal case, alongside parameter-robust preconditioners. We show that these timesteppers can derive from a finite-element-in-time (FET) (and finite-element-in-space) interpretation. The benefits of this reformulation are discussed, including the derivation of timesteppers that are higher order in time, and the quantifiable dissipative SP properties in the non-ideal, resistive case.
        
        We discuss possible options for extending this FET approach to timesteppers for the compressible case.

        The kinetic corrections satisfy linearized Boltzmann equations. Using a Lénard--Bernstein collision operator, these take Fokker--Planck-like forms \cite{Fokker_1914, Planck_1917} wherein pseudo-particles in the numerical model obey the neoclassical transport equations, with particle-independent Brownian drift terms. This offers a rigorous methodology for incorporating collisions into the particle transport model, without coupling the equations of motions for each particle.
        
        Works by Chen, Chacón et al. \cite{Chen_Chacón_Barnes_2011, Chacón_Chen_Barnes_2013, Chen_Chacón_2014, Chen_Chacón_2015} have developed structure-preserving particle pushers for neoclassical transport in the Vlasov equations, derived from Crank--Nicolson integrators. We show these too can can derive from a FET interpretation, similarly offering potential extensions to higher-order-in-time particle pushers. The FET formulation is used also to consider how the stochastic drift terms can be incorporated into the pushers. Stochastic gyrokinetic expansions are also discussed.

        Different options for the numerical implementation of these schemes are considered.

        Due to the efficacy of FET in the development of SP timesteppers for both the fluid and kinetic component, we hope this approach will prove effective in the future for developing SP timesteppers for the full hybrid model. We hope this will give us the opportunity to incorporate previously inaccessible kinetic effects into the highly effective, modern, finite-element MHD models.
    \end{abstract}
    
    
    \newpage
    \tableofcontents
    
    
    \newpage
    \pagenumbering{arabic}
    %\linenumbers\renewcommand\thelinenumber{\color{black!50}\arabic{linenumber}}
            \documentclass[12pt, a4paper]{report}

\input{template/main.tex}

\title{\BA{Title in Progress...}}
\author{Boris Andrews}
\affil{Mathematical Institute, University of Oxford}
\date{\today}


\begin{document}
    \pagenumbering{gobble}
    \maketitle
    
    
    \begin{abstract}
        Magnetic confinement reactors---in particular tokamaks---offer one of the most promising options for achieving practical nuclear fusion, with the potential to provide virtually limitless, clean energy. The theoretical and numerical modeling of tokamak plasmas is simultaneously an essential component of effective reactor design, and a great research barrier. Tokamak operational conditions exhibit comparatively low Knudsen numbers. Kinetic effects, including kinetic waves and instabilities, Landau damping, bump-on-tail instabilities and more, are therefore highly influential in tokamak plasma dynamics. Purely fluid models are inherently incapable of capturing these effects, whereas the high dimensionality in purely kinetic models render them practically intractable for most relevant purposes.

        We consider a $\delta\!f$ decomposition model, with a macroscopic fluid background and microscopic kinetic correction, both fully coupled to each other. A similar manner of discretization is proposed to that used in the recent \texttt{STRUPHY} code \cite{Holderied_Possanner_Wang_2021, Holderied_2022, Li_et_al_2023} with a finite-element model for the background and a pseudo-particle/PiC model for the correction.

        The fluid background satisfies the full, non-linear, resistive, compressible, Hall MHD equations. \cite{Laakmann_Hu_Farrell_2022} introduces finite-element(-in-space) implicit timesteppers for the incompressible analogue to this system with structure-preserving (SP) properties in the ideal case, alongside parameter-robust preconditioners. We show that these timesteppers can derive from a finite-element-in-time (FET) (and finite-element-in-space) interpretation. The benefits of this reformulation are discussed, including the derivation of timesteppers that are higher order in time, and the quantifiable dissipative SP properties in the non-ideal, resistive case.
        
        We discuss possible options for extending this FET approach to timesteppers for the compressible case.

        The kinetic corrections satisfy linearized Boltzmann equations. Using a Lénard--Bernstein collision operator, these take Fokker--Planck-like forms \cite{Fokker_1914, Planck_1917} wherein pseudo-particles in the numerical model obey the neoclassical transport equations, with particle-independent Brownian drift terms. This offers a rigorous methodology for incorporating collisions into the particle transport model, without coupling the equations of motions for each particle.
        
        Works by Chen, Chacón et al. \cite{Chen_Chacón_Barnes_2011, Chacón_Chen_Barnes_2013, Chen_Chacón_2014, Chen_Chacón_2015} have developed structure-preserving particle pushers for neoclassical transport in the Vlasov equations, derived from Crank--Nicolson integrators. We show these too can can derive from a FET interpretation, similarly offering potential extensions to higher-order-in-time particle pushers. The FET formulation is used also to consider how the stochastic drift terms can be incorporated into the pushers. Stochastic gyrokinetic expansions are also discussed.

        Different options for the numerical implementation of these schemes are considered.

        Due to the efficacy of FET in the development of SP timesteppers for both the fluid and kinetic component, we hope this approach will prove effective in the future for developing SP timesteppers for the full hybrid model. We hope this will give us the opportunity to incorporate previously inaccessible kinetic effects into the highly effective, modern, finite-element MHD models.
    \end{abstract}
    
    
    \newpage
    \tableofcontents
    
    
    \newpage
    \pagenumbering{arabic}
    %\linenumbers\renewcommand\thelinenumber{\color{black!50}\arabic{linenumber}}
            \input{0 - introduction/main.tex}
        \part{Research}
            \input{1 - low-noise PiC models/main.tex}
            \input{2 - kinetic component/main.tex}
            \input{3 - fluid component/main.tex}
            \input{4 - numerical implementation/main.tex}
        \part{Project Overview}
            \input{5 - research plan/main.tex}
            \input{6 - summary/main.tex}
    
    
    %\section{}
    \newpage
    \pagenumbering{gobble}
        \printbibliography


    \newpage
    \pagenumbering{roman}
    \appendix
        \part{Appendices}
            \input{8 - Hilbert complexes/main.tex}
            \input{9 - weak conservation proofs/main.tex}
\end{document}

        \part{Research}
            \documentclass[12pt, a4paper]{report}

\input{template/main.tex}

\title{\BA{Title in Progress...}}
\author{Boris Andrews}
\affil{Mathematical Institute, University of Oxford}
\date{\today}


\begin{document}
    \pagenumbering{gobble}
    \maketitle
    
    
    \begin{abstract}
        Magnetic confinement reactors---in particular tokamaks---offer one of the most promising options for achieving practical nuclear fusion, with the potential to provide virtually limitless, clean energy. The theoretical and numerical modeling of tokamak plasmas is simultaneously an essential component of effective reactor design, and a great research barrier. Tokamak operational conditions exhibit comparatively low Knudsen numbers. Kinetic effects, including kinetic waves and instabilities, Landau damping, bump-on-tail instabilities and more, are therefore highly influential in tokamak plasma dynamics. Purely fluid models are inherently incapable of capturing these effects, whereas the high dimensionality in purely kinetic models render them practically intractable for most relevant purposes.

        We consider a $\delta\!f$ decomposition model, with a macroscopic fluid background and microscopic kinetic correction, both fully coupled to each other. A similar manner of discretization is proposed to that used in the recent \texttt{STRUPHY} code \cite{Holderied_Possanner_Wang_2021, Holderied_2022, Li_et_al_2023} with a finite-element model for the background and a pseudo-particle/PiC model for the correction.

        The fluid background satisfies the full, non-linear, resistive, compressible, Hall MHD equations. \cite{Laakmann_Hu_Farrell_2022} introduces finite-element(-in-space) implicit timesteppers for the incompressible analogue to this system with structure-preserving (SP) properties in the ideal case, alongside parameter-robust preconditioners. We show that these timesteppers can derive from a finite-element-in-time (FET) (and finite-element-in-space) interpretation. The benefits of this reformulation are discussed, including the derivation of timesteppers that are higher order in time, and the quantifiable dissipative SP properties in the non-ideal, resistive case.
        
        We discuss possible options for extending this FET approach to timesteppers for the compressible case.

        The kinetic corrections satisfy linearized Boltzmann equations. Using a Lénard--Bernstein collision operator, these take Fokker--Planck-like forms \cite{Fokker_1914, Planck_1917} wherein pseudo-particles in the numerical model obey the neoclassical transport equations, with particle-independent Brownian drift terms. This offers a rigorous methodology for incorporating collisions into the particle transport model, without coupling the equations of motions for each particle.
        
        Works by Chen, Chacón et al. \cite{Chen_Chacón_Barnes_2011, Chacón_Chen_Barnes_2013, Chen_Chacón_2014, Chen_Chacón_2015} have developed structure-preserving particle pushers for neoclassical transport in the Vlasov equations, derived from Crank--Nicolson integrators. We show these too can can derive from a FET interpretation, similarly offering potential extensions to higher-order-in-time particle pushers. The FET formulation is used also to consider how the stochastic drift terms can be incorporated into the pushers. Stochastic gyrokinetic expansions are also discussed.

        Different options for the numerical implementation of these schemes are considered.

        Due to the efficacy of FET in the development of SP timesteppers for both the fluid and kinetic component, we hope this approach will prove effective in the future for developing SP timesteppers for the full hybrid model. We hope this will give us the opportunity to incorporate previously inaccessible kinetic effects into the highly effective, modern, finite-element MHD models.
    \end{abstract}
    
    
    \newpage
    \tableofcontents
    
    
    \newpage
    \pagenumbering{arabic}
    %\linenumbers\renewcommand\thelinenumber{\color{black!50}\arabic{linenumber}}
            \input{0 - introduction/main.tex}
        \part{Research}
            \input{1 - low-noise PiC models/main.tex}
            \input{2 - kinetic component/main.tex}
            \input{3 - fluid component/main.tex}
            \input{4 - numerical implementation/main.tex}
        \part{Project Overview}
            \input{5 - research plan/main.tex}
            \input{6 - summary/main.tex}
    
    
    %\section{}
    \newpage
    \pagenumbering{gobble}
        \printbibliography


    \newpage
    \pagenumbering{roman}
    \appendix
        \part{Appendices}
            \input{8 - Hilbert complexes/main.tex}
            \input{9 - weak conservation proofs/main.tex}
\end{document}

            \documentclass[12pt, a4paper]{report}

\input{template/main.tex}

\title{\BA{Title in Progress...}}
\author{Boris Andrews}
\affil{Mathematical Institute, University of Oxford}
\date{\today}


\begin{document}
    \pagenumbering{gobble}
    \maketitle
    
    
    \begin{abstract}
        Magnetic confinement reactors---in particular tokamaks---offer one of the most promising options for achieving practical nuclear fusion, with the potential to provide virtually limitless, clean energy. The theoretical and numerical modeling of tokamak plasmas is simultaneously an essential component of effective reactor design, and a great research barrier. Tokamak operational conditions exhibit comparatively low Knudsen numbers. Kinetic effects, including kinetic waves and instabilities, Landau damping, bump-on-tail instabilities and more, are therefore highly influential in tokamak plasma dynamics. Purely fluid models are inherently incapable of capturing these effects, whereas the high dimensionality in purely kinetic models render them practically intractable for most relevant purposes.

        We consider a $\delta\!f$ decomposition model, with a macroscopic fluid background and microscopic kinetic correction, both fully coupled to each other. A similar manner of discretization is proposed to that used in the recent \texttt{STRUPHY} code \cite{Holderied_Possanner_Wang_2021, Holderied_2022, Li_et_al_2023} with a finite-element model for the background and a pseudo-particle/PiC model for the correction.

        The fluid background satisfies the full, non-linear, resistive, compressible, Hall MHD equations. \cite{Laakmann_Hu_Farrell_2022} introduces finite-element(-in-space) implicit timesteppers for the incompressible analogue to this system with structure-preserving (SP) properties in the ideal case, alongside parameter-robust preconditioners. We show that these timesteppers can derive from a finite-element-in-time (FET) (and finite-element-in-space) interpretation. The benefits of this reformulation are discussed, including the derivation of timesteppers that are higher order in time, and the quantifiable dissipative SP properties in the non-ideal, resistive case.
        
        We discuss possible options for extending this FET approach to timesteppers for the compressible case.

        The kinetic corrections satisfy linearized Boltzmann equations. Using a Lénard--Bernstein collision operator, these take Fokker--Planck-like forms \cite{Fokker_1914, Planck_1917} wherein pseudo-particles in the numerical model obey the neoclassical transport equations, with particle-independent Brownian drift terms. This offers a rigorous methodology for incorporating collisions into the particle transport model, without coupling the equations of motions for each particle.
        
        Works by Chen, Chacón et al. \cite{Chen_Chacón_Barnes_2011, Chacón_Chen_Barnes_2013, Chen_Chacón_2014, Chen_Chacón_2015} have developed structure-preserving particle pushers for neoclassical transport in the Vlasov equations, derived from Crank--Nicolson integrators. We show these too can can derive from a FET interpretation, similarly offering potential extensions to higher-order-in-time particle pushers. The FET formulation is used also to consider how the stochastic drift terms can be incorporated into the pushers. Stochastic gyrokinetic expansions are also discussed.

        Different options for the numerical implementation of these schemes are considered.

        Due to the efficacy of FET in the development of SP timesteppers for both the fluid and kinetic component, we hope this approach will prove effective in the future for developing SP timesteppers for the full hybrid model. We hope this will give us the opportunity to incorporate previously inaccessible kinetic effects into the highly effective, modern, finite-element MHD models.
    \end{abstract}
    
    
    \newpage
    \tableofcontents
    
    
    \newpage
    \pagenumbering{arabic}
    %\linenumbers\renewcommand\thelinenumber{\color{black!50}\arabic{linenumber}}
            \input{0 - introduction/main.tex}
        \part{Research}
            \input{1 - low-noise PiC models/main.tex}
            \input{2 - kinetic component/main.tex}
            \input{3 - fluid component/main.tex}
            \input{4 - numerical implementation/main.tex}
        \part{Project Overview}
            \input{5 - research plan/main.tex}
            \input{6 - summary/main.tex}
    
    
    %\section{}
    \newpage
    \pagenumbering{gobble}
        \printbibliography


    \newpage
    \pagenumbering{roman}
    \appendix
        \part{Appendices}
            \input{8 - Hilbert complexes/main.tex}
            \input{9 - weak conservation proofs/main.tex}
\end{document}

            \documentclass[12pt, a4paper]{report}

\input{template/main.tex}

\title{\BA{Title in Progress...}}
\author{Boris Andrews}
\affil{Mathematical Institute, University of Oxford}
\date{\today}


\begin{document}
    \pagenumbering{gobble}
    \maketitle
    
    
    \begin{abstract}
        Magnetic confinement reactors---in particular tokamaks---offer one of the most promising options for achieving practical nuclear fusion, with the potential to provide virtually limitless, clean energy. The theoretical and numerical modeling of tokamak plasmas is simultaneously an essential component of effective reactor design, and a great research barrier. Tokamak operational conditions exhibit comparatively low Knudsen numbers. Kinetic effects, including kinetic waves and instabilities, Landau damping, bump-on-tail instabilities and more, are therefore highly influential in tokamak plasma dynamics. Purely fluid models are inherently incapable of capturing these effects, whereas the high dimensionality in purely kinetic models render them practically intractable for most relevant purposes.

        We consider a $\delta\!f$ decomposition model, with a macroscopic fluid background and microscopic kinetic correction, both fully coupled to each other. A similar manner of discretization is proposed to that used in the recent \texttt{STRUPHY} code \cite{Holderied_Possanner_Wang_2021, Holderied_2022, Li_et_al_2023} with a finite-element model for the background and a pseudo-particle/PiC model for the correction.

        The fluid background satisfies the full, non-linear, resistive, compressible, Hall MHD equations. \cite{Laakmann_Hu_Farrell_2022} introduces finite-element(-in-space) implicit timesteppers for the incompressible analogue to this system with structure-preserving (SP) properties in the ideal case, alongside parameter-robust preconditioners. We show that these timesteppers can derive from a finite-element-in-time (FET) (and finite-element-in-space) interpretation. The benefits of this reformulation are discussed, including the derivation of timesteppers that are higher order in time, and the quantifiable dissipative SP properties in the non-ideal, resistive case.
        
        We discuss possible options for extending this FET approach to timesteppers for the compressible case.

        The kinetic corrections satisfy linearized Boltzmann equations. Using a Lénard--Bernstein collision operator, these take Fokker--Planck-like forms \cite{Fokker_1914, Planck_1917} wherein pseudo-particles in the numerical model obey the neoclassical transport equations, with particle-independent Brownian drift terms. This offers a rigorous methodology for incorporating collisions into the particle transport model, without coupling the equations of motions for each particle.
        
        Works by Chen, Chacón et al. \cite{Chen_Chacón_Barnes_2011, Chacón_Chen_Barnes_2013, Chen_Chacón_2014, Chen_Chacón_2015} have developed structure-preserving particle pushers for neoclassical transport in the Vlasov equations, derived from Crank--Nicolson integrators. We show these too can can derive from a FET interpretation, similarly offering potential extensions to higher-order-in-time particle pushers. The FET formulation is used also to consider how the stochastic drift terms can be incorporated into the pushers. Stochastic gyrokinetic expansions are also discussed.

        Different options for the numerical implementation of these schemes are considered.

        Due to the efficacy of FET in the development of SP timesteppers for both the fluid and kinetic component, we hope this approach will prove effective in the future for developing SP timesteppers for the full hybrid model. We hope this will give us the opportunity to incorporate previously inaccessible kinetic effects into the highly effective, modern, finite-element MHD models.
    \end{abstract}
    
    
    \newpage
    \tableofcontents
    
    
    \newpage
    \pagenumbering{arabic}
    %\linenumbers\renewcommand\thelinenumber{\color{black!50}\arabic{linenumber}}
            \input{0 - introduction/main.tex}
        \part{Research}
            \input{1 - low-noise PiC models/main.tex}
            \input{2 - kinetic component/main.tex}
            \input{3 - fluid component/main.tex}
            \input{4 - numerical implementation/main.tex}
        \part{Project Overview}
            \input{5 - research plan/main.tex}
            \input{6 - summary/main.tex}
    
    
    %\section{}
    \newpage
    \pagenumbering{gobble}
        \printbibliography


    \newpage
    \pagenumbering{roman}
    \appendix
        \part{Appendices}
            \input{8 - Hilbert complexes/main.tex}
            \input{9 - weak conservation proofs/main.tex}
\end{document}

            \documentclass[12pt, a4paper]{report}

\input{template/main.tex}

\title{\BA{Title in Progress...}}
\author{Boris Andrews}
\affil{Mathematical Institute, University of Oxford}
\date{\today}


\begin{document}
    \pagenumbering{gobble}
    \maketitle
    
    
    \begin{abstract}
        Magnetic confinement reactors---in particular tokamaks---offer one of the most promising options for achieving practical nuclear fusion, with the potential to provide virtually limitless, clean energy. The theoretical and numerical modeling of tokamak plasmas is simultaneously an essential component of effective reactor design, and a great research barrier. Tokamak operational conditions exhibit comparatively low Knudsen numbers. Kinetic effects, including kinetic waves and instabilities, Landau damping, bump-on-tail instabilities and more, are therefore highly influential in tokamak plasma dynamics. Purely fluid models are inherently incapable of capturing these effects, whereas the high dimensionality in purely kinetic models render them practically intractable for most relevant purposes.

        We consider a $\delta\!f$ decomposition model, with a macroscopic fluid background and microscopic kinetic correction, both fully coupled to each other. A similar manner of discretization is proposed to that used in the recent \texttt{STRUPHY} code \cite{Holderied_Possanner_Wang_2021, Holderied_2022, Li_et_al_2023} with a finite-element model for the background and a pseudo-particle/PiC model for the correction.

        The fluid background satisfies the full, non-linear, resistive, compressible, Hall MHD equations. \cite{Laakmann_Hu_Farrell_2022} introduces finite-element(-in-space) implicit timesteppers for the incompressible analogue to this system with structure-preserving (SP) properties in the ideal case, alongside parameter-robust preconditioners. We show that these timesteppers can derive from a finite-element-in-time (FET) (and finite-element-in-space) interpretation. The benefits of this reformulation are discussed, including the derivation of timesteppers that are higher order in time, and the quantifiable dissipative SP properties in the non-ideal, resistive case.
        
        We discuss possible options for extending this FET approach to timesteppers for the compressible case.

        The kinetic corrections satisfy linearized Boltzmann equations. Using a Lénard--Bernstein collision operator, these take Fokker--Planck-like forms \cite{Fokker_1914, Planck_1917} wherein pseudo-particles in the numerical model obey the neoclassical transport equations, with particle-independent Brownian drift terms. This offers a rigorous methodology for incorporating collisions into the particle transport model, without coupling the equations of motions for each particle.
        
        Works by Chen, Chacón et al. \cite{Chen_Chacón_Barnes_2011, Chacón_Chen_Barnes_2013, Chen_Chacón_2014, Chen_Chacón_2015} have developed structure-preserving particle pushers for neoclassical transport in the Vlasov equations, derived from Crank--Nicolson integrators. We show these too can can derive from a FET interpretation, similarly offering potential extensions to higher-order-in-time particle pushers. The FET formulation is used also to consider how the stochastic drift terms can be incorporated into the pushers. Stochastic gyrokinetic expansions are also discussed.

        Different options for the numerical implementation of these schemes are considered.

        Due to the efficacy of FET in the development of SP timesteppers for both the fluid and kinetic component, we hope this approach will prove effective in the future for developing SP timesteppers for the full hybrid model. We hope this will give us the opportunity to incorporate previously inaccessible kinetic effects into the highly effective, modern, finite-element MHD models.
    \end{abstract}
    
    
    \newpage
    \tableofcontents
    
    
    \newpage
    \pagenumbering{arabic}
    %\linenumbers\renewcommand\thelinenumber{\color{black!50}\arabic{linenumber}}
            \input{0 - introduction/main.tex}
        \part{Research}
            \input{1 - low-noise PiC models/main.tex}
            \input{2 - kinetic component/main.tex}
            \input{3 - fluid component/main.tex}
            \input{4 - numerical implementation/main.tex}
        \part{Project Overview}
            \input{5 - research plan/main.tex}
            \input{6 - summary/main.tex}
    
    
    %\section{}
    \newpage
    \pagenumbering{gobble}
        \printbibliography


    \newpage
    \pagenumbering{roman}
    \appendix
        \part{Appendices}
            \input{8 - Hilbert complexes/main.tex}
            \input{9 - weak conservation proofs/main.tex}
\end{document}

        \part{Project Overview}
            \documentclass[12pt, a4paper]{report}

\input{template/main.tex}

\title{\BA{Title in Progress...}}
\author{Boris Andrews}
\affil{Mathematical Institute, University of Oxford}
\date{\today}


\begin{document}
    \pagenumbering{gobble}
    \maketitle
    
    
    \begin{abstract}
        Magnetic confinement reactors---in particular tokamaks---offer one of the most promising options for achieving practical nuclear fusion, with the potential to provide virtually limitless, clean energy. The theoretical and numerical modeling of tokamak plasmas is simultaneously an essential component of effective reactor design, and a great research barrier. Tokamak operational conditions exhibit comparatively low Knudsen numbers. Kinetic effects, including kinetic waves and instabilities, Landau damping, bump-on-tail instabilities and more, are therefore highly influential in tokamak plasma dynamics. Purely fluid models are inherently incapable of capturing these effects, whereas the high dimensionality in purely kinetic models render them practically intractable for most relevant purposes.

        We consider a $\delta\!f$ decomposition model, with a macroscopic fluid background and microscopic kinetic correction, both fully coupled to each other. A similar manner of discretization is proposed to that used in the recent \texttt{STRUPHY} code \cite{Holderied_Possanner_Wang_2021, Holderied_2022, Li_et_al_2023} with a finite-element model for the background and a pseudo-particle/PiC model for the correction.

        The fluid background satisfies the full, non-linear, resistive, compressible, Hall MHD equations. \cite{Laakmann_Hu_Farrell_2022} introduces finite-element(-in-space) implicit timesteppers for the incompressible analogue to this system with structure-preserving (SP) properties in the ideal case, alongside parameter-robust preconditioners. We show that these timesteppers can derive from a finite-element-in-time (FET) (and finite-element-in-space) interpretation. The benefits of this reformulation are discussed, including the derivation of timesteppers that are higher order in time, and the quantifiable dissipative SP properties in the non-ideal, resistive case.
        
        We discuss possible options for extending this FET approach to timesteppers for the compressible case.

        The kinetic corrections satisfy linearized Boltzmann equations. Using a Lénard--Bernstein collision operator, these take Fokker--Planck-like forms \cite{Fokker_1914, Planck_1917} wherein pseudo-particles in the numerical model obey the neoclassical transport equations, with particle-independent Brownian drift terms. This offers a rigorous methodology for incorporating collisions into the particle transport model, without coupling the equations of motions for each particle.
        
        Works by Chen, Chacón et al. \cite{Chen_Chacón_Barnes_2011, Chacón_Chen_Barnes_2013, Chen_Chacón_2014, Chen_Chacón_2015} have developed structure-preserving particle pushers for neoclassical transport in the Vlasov equations, derived from Crank--Nicolson integrators. We show these too can can derive from a FET interpretation, similarly offering potential extensions to higher-order-in-time particle pushers. The FET formulation is used also to consider how the stochastic drift terms can be incorporated into the pushers. Stochastic gyrokinetic expansions are also discussed.

        Different options for the numerical implementation of these schemes are considered.

        Due to the efficacy of FET in the development of SP timesteppers for both the fluid and kinetic component, we hope this approach will prove effective in the future for developing SP timesteppers for the full hybrid model. We hope this will give us the opportunity to incorporate previously inaccessible kinetic effects into the highly effective, modern, finite-element MHD models.
    \end{abstract}
    
    
    \newpage
    \tableofcontents
    
    
    \newpage
    \pagenumbering{arabic}
    %\linenumbers\renewcommand\thelinenumber{\color{black!50}\arabic{linenumber}}
            \input{0 - introduction/main.tex}
        \part{Research}
            \input{1 - low-noise PiC models/main.tex}
            \input{2 - kinetic component/main.tex}
            \input{3 - fluid component/main.tex}
            \input{4 - numerical implementation/main.tex}
        \part{Project Overview}
            \input{5 - research plan/main.tex}
            \input{6 - summary/main.tex}
    
    
    %\section{}
    \newpage
    \pagenumbering{gobble}
        \printbibliography


    \newpage
    \pagenumbering{roman}
    \appendix
        \part{Appendices}
            \input{8 - Hilbert complexes/main.tex}
            \input{9 - weak conservation proofs/main.tex}
\end{document}

            \documentclass[12pt, a4paper]{report}

\input{template/main.tex}

\title{\BA{Title in Progress...}}
\author{Boris Andrews}
\affil{Mathematical Institute, University of Oxford}
\date{\today}


\begin{document}
    \pagenumbering{gobble}
    \maketitle
    
    
    \begin{abstract}
        Magnetic confinement reactors---in particular tokamaks---offer one of the most promising options for achieving practical nuclear fusion, with the potential to provide virtually limitless, clean energy. The theoretical and numerical modeling of tokamak plasmas is simultaneously an essential component of effective reactor design, and a great research barrier. Tokamak operational conditions exhibit comparatively low Knudsen numbers. Kinetic effects, including kinetic waves and instabilities, Landau damping, bump-on-tail instabilities and more, are therefore highly influential in tokamak plasma dynamics. Purely fluid models are inherently incapable of capturing these effects, whereas the high dimensionality in purely kinetic models render them practically intractable for most relevant purposes.

        We consider a $\delta\!f$ decomposition model, with a macroscopic fluid background and microscopic kinetic correction, both fully coupled to each other. A similar manner of discretization is proposed to that used in the recent \texttt{STRUPHY} code \cite{Holderied_Possanner_Wang_2021, Holderied_2022, Li_et_al_2023} with a finite-element model for the background and a pseudo-particle/PiC model for the correction.

        The fluid background satisfies the full, non-linear, resistive, compressible, Hall MHD equations. \cite{Laakmann_Hu_Farrell_2022} introduces finite-element(-in-space) implicit timesteppers for the incompressible analogue to this system with structure-preserving (SP) properties in the ideal case, alongside parameter-robust preconditioners. We show that these timesteppers can derive from a finite-element-in-time (FET) (and finite-element-in-space) interpretation. The benefits of this reformulation are discussed, including the derivation of timesteppers that are higher order in time, and the quantifiable dissipative SP properties in the non-ideal, resistive case.
        
        We discuss possible options for extending this FET approach to timesteppers for the compressible case.

        The kinetic corrections satisfy linearized Boltzmann equations. Using a Lénard--Bernstein collision operator, these take Fokker--Planck-like forms \cite{Fokker_1914, Planck_1917} wherein pseudo-particles in the numerical model obey the neoclassical transport equations, with particle-independent Brownian drift terms. This offers a rigorous methodology for incorporating collisions into the particle transport model, without coupling the equations of motions for each particle.
        
        Works by Chen, Chacón et al. \cite{Chen_Chacón_Barnes_2011, Chacón_Chen_Barnes_2013, Chen_Chacón_2014, Chen_Chacón_2015} have developed structure-preserving particle pushers for neoclassical transport in the Vlasov equations, derived from Crank--Nicolson integrators. We show these too can can derive from a FET interpretation, similarly offering potential extensions to higher-order-in-time particle pushers. The FET formulation is used also to consider how the stochastic drift terms can be incorporated into the pushers. Stochastic gyrokinetic expansions are also discussed.

        Different options for the numerical implementation of these schemes are considered.

        Due to the efficacy of FET in the development of SP timesteppers for both the fluid and kinetic component, we hope this approach will prove effective in the future for developing SP timesteppers for the full hybrid model. We hope this will give us the opportunity to incorporate previously inaccessible kinetic effects into the highly effective, modern, finite-element MHD models.
    \end{abstract}
    
    
    \newpage
    \tableofcontents
    
    
    \newpage
    \pagenumbering{arabic}
    %\linenumbers\renewcommand\thelinenumber{\color{black!50}\arabic{linenumber}}
            \input{0 - introduction/main.tex}
        \part{Research}
            \input{1 - low-noise PiC models/main.tex}
            \input{2 - kinetic component/main.tex}
            \input{3 - fluid component/main.tex}
            \input{4 - numerical implementation/main.tex}
        \part{Project Overview}
            \input{5 - research plan/main.tex}
            \input{6 - summary/main.tex}
    
    
    %\section{}
    \newpage
    \pagenumbering{gobble}
        \printbibliography


    \newpage
    \pagenumbering{roman}
    \appendix
        \part{Appendices}
            \input{8 - Hilbert complexes/main.tex}
            \input{9 - weak conservation proofs/main.tex}
\end{document}

    
    
    %\section{}
    \newpage
    \pagenumbering{gobble}
        \printbibliography


    \newpage
    \pagenumbering{roman}
    \appendix
        \part{Appendices}
            \documentclass[12pt, a4paper]{report}

\input{template/main.tex}

\title{\BA{Title in Progress...}}
\author{Boris Andrews}
\affil{Mathematical Institute, University of Oxford}
\date{\today}


\begin{document}
    \pagenumbering{gobble}
    \maketitle
    
    
    \begin{abstract}
        Magnetic confinement reactors---in particular tokamaks---offer one of the most promising options for achieving practical nuclear fusion, with the potential to provide virtually limitless, clean energy. The theoretical and numerical modeling of tokamak plasmas is simultaneously an essential component of effective reactor design, and a great research barrier. Tokamak operational conditions exhibit comparatively low Knudsen numbers. Kinetic effects, including kinetic waves and instabilities, Landau damping, bump-on-tail instabilities and more, are therefore highly influential in tokamak plasma dynamics. Purely fluid models are inherently incapable of capturing these effects, whereas the high dimensionality in purely kinetic models render them practically intractable for most relevant purposes.

        We consider a $\delta\!f$ decomposition model, with a macroscopic fluid background and microscopic kinetic correction, both fully coupled to each other. A similar manner of discretization is proposed to that used in the recent \texttt{STRUPHY} code \cite{Holderied_Possanner_Wang_2021, Holderied_2022, Li_et_al_2023} with a finite-element model for the background and a pseudo-particle/PiC model for the correction.

        The fluid background satisfies the full, non-linear, resistive, compressible, Hall MHD equations. \cite{Laakmann_Hu_Farrell_2022} introduces finite-element(-in-space) implicit timesteppers for the incompressible analogue to this system with structure-preserving (SP) properties in the ideal case, alongside parameter-robust preconditioners. We show that these timesteppers can derive from a finite-element-in-time (FET) (and finite-element-in-space) interpretation. The benefits of this reformulation are discussed, including the derivation of timesteppers that are higher order in time, and the quantifiable dissipative SP properties in the non-ideal, resistive case.
        
        We discuss possible options for extending this FET approach to timesteppers for the compressible case.

        The kinetic corrections satisfy linearized Boltzmann equations. Using a Lénard--Bernstein collision operator, these take Fokker--Planck-like forms \cite{Fokker_1914, Planck_1917} wherein pseudo-particles in the numerical model obey the neoclassical transport equations, with particle-independent Brownian drift terms. This offers a rigorous methodology for incorporating collisions into the particle transport model, without coupling the equations of motions for each particle.
        
        Works by Chen, Chacón et al. \cite{Chen_Chacón_Barnes_2011, Chacón_Chen_Barnes_2013, Chen_Chacón_2014, Chen_Chacón_2015} have developed structure-preserving particle pushers for neoclassical transport in the Vlasov equations, derived from Crank--Nicolson integrators. We show these too can can derive from a FET interpretation, similarly offering potential extensions to higher-order-in-time particle pushers. The FET formulation is used also to consider how the stochastic drift terms can be incorporated into the pushers. Stochastic gyrokinetic expansions are also discussed.

        Different options for the numerical implementation of these schemes are considered.

        Due to the efficacy of FET in the development of SP timesteppers for both the fluid and kinetic component, we hope this approach will prove effective in the future for developing SP timesteppers for the full hybrid model. We hope this will give us the opportunity to incorporate previously inaccessible kinetic effects into the highly effective, modern, finite-element MHD models.
    \end{abstract}
    
    
    \newpage
    \tableofcontents
    
    
    \newpage
    \pagenumbering{arabic}
    %\linenumbers\renewcommand\thelinenumber{\color{black!50}\arabic{linenumber}}
            \input{0 - introduction/main.tex}
        \part{Research}
            \input{1 - low-noise PiC models/main.tex}
            \input{2 - kinetic component/main.tex}
            \input{3 - fluid component/main.tex}
            \input{4 - numerical implementation/main.tex}
        \part{Project Overview}
            \input{5 - research plan/main.tex}
            \input{6 - summary/main.tex}
    
    
    %\section{}
    \newpage
    \pagenumbering{gobble}
        \printbibliography


    \newpage
    \pagenumbering{roman}
    \appendix
        \part{Appendices}
            \input{8 - Hilbert complexes/main.tex}
            \input{9 - weak conservation proofs/main.tex}
\end{document}

            \documentclass[12pt, a4paper]{report}

\input{template/main.tex}

\title{\BA{Title in Progress...}}
\author{Boris Andrews}
\affil{Mathematical Institute, University of Oxford}
\date{\today}


\begin{document}
    \pagenumbering{gobble}
    \maketitle
    
    
    \begin{abstract}
        Magnetic confinement reactors---in particular tokamaks---offer one of the most promising options for achieving practical nuclear fusion, with the potential to provide virtually limitless, clean energy. The theoretical and numerical modeling of tokamak plasmas is simultaneously an essential component of effective reactor design, and a great research barrier. Tokamak operational conditions exhibit comparatively low Knudsen numbers. Kinetic effects, including kinetic waves and instabilities, Landau damping, bump-on-tail instabilities and more, are therefore highly influential in tokamak plasma dynamics. Purely fluid models are inherently incapable of capturing these effects, whereas the high dimensionality in purely kinetic models render them practically intractable for most relevant purposes.

        We consider a $\delta\!f$ decomposition model, with a macroscopic fluid background and microscopic kinetic correction, both fully coupled to each other. A similar manner of discretization is proposed to that used in the recent \texttt{STRUPHY} code \cite{Holderied_Possanner_Wang_2021, Holderied_2022, Li_et_al_2023} with a finite-element model for the background and a pseudo-particle/PiC model for the correction.

        The fluid background satisfies the full, non-linear, resistive, compressible, Hall MHD equations. \cite{Laakmann_Hu_Farrell_2022} introduces finite-element(-in-space) implicit timesteppers for the incompressible analogue to this system with structure-preserving (SP) properties in the ideal case, alongside parameter-robust preconditioners. We show that these timesteppers can derive from a finite-element-in-time (FET) (and finite-element-in-space) interpretation. The benefits of this reformulation are discussed, including the derivation of timesteppers that are higher order in time, and the quantifiable dissipative SP properties in the non-ideal, resistive case.
        
        We discuss possible options for extending this FET approach to timesteppers for the compressible case.

        The kinetic corrections satisfy linearized Boltzmann equations. Using a Lénard--Bernstein collision operator, these take Fokker--Planck-like forms \cite{Fokker_1914, Planck_1917} wherein pseudo-particles in the numerical model obey the neoclassical transport equations, with particle-independent Brownian drift terms. This offers a rigorous methodology for incorporating collisions into the particle transport model, without coupling the equations of motions for each particle.
        
        Works by Chen, Chacón et al. \cite{Chen_Chacón_Barnes_2011, Chacón_Chen_Barnes_2013, Chen_Chacón_2014, Chen_Chacón_2015} have developed structure-preserving particle pushers for neoclassical transport in the Vlasov equations, derived from Crank--Nicolson integrators. We show these too can can derive from a FET interpretation, similarly offering potential extensions to higher-order-in-time particle pushers. The FET formulation is used also to consider how the stochastic drift terms can be incorporated into the pushers. Stochastic gyrokinetic expansions are also discussed.

        Different options for the numerical implementation of these schemes are considered.

        Due to the efficacy of FET in the development of SP timesteppers for both the fluid and kinetic component, we hope this approach will prove effective in the future for developing SP timesteppers for the full hybrid model. We hope this will give us the opportunity to incorporate previously inaccessible kinetic effects into the highly effective, modern, finite-element MHD models.
    \end{abstract}
    
    
    \newpage
    \tableofcontents
    
    
    \newpage
    \pagenumbering{arabic}
    %\linenumbers\renewcommand\thelinenumber{\color{black!50}\arabic{linenumber}}
            \input{0 - introduction/main.tex}
        \part{Research}
            \input{1 - low-noise PiC models/main.tex}
            \input{2 - kinetic component/main.tex}
            \input{3 - fluid component/main.tex}
            \input{4 - numerical implementation/main.tex}
        \part{Project Overview}
            \input{5 - research plan/main.tex}
            \input{6 - summary/main.tex}
    
    
    %\section{}
    \newpage
    \pagenumbering{gobble}
        \printbibliography


    \newpage
    \pagenumbering{roman}
    \appendix
        \part{Appendices}
            \input{8 - Hilbert complexes/main.tex}
            \input{9 - weak conservation proofs/main.tex}
\end{document}

\end{document}

            \documentclass[12pt, a4paper]{report}

\documentclass[12pt, a4paper]{report}

\input{template/main.tex}

\title{\BA{Title in Progress...}}
\author{Boris Andrews}
\affil{Mathematical Institute, University of Oxford}
\date{\today}


\begin{document}
    \pagenumbering{gobble}
    \maketitle
    
    
    \begin{abstract}
        Magnetic confinement reactors---in particular tokamaks---offer one of the most promising options for achieving practical nuclear fusion, with the potential to provide virtually limitless, clean energy. The theoretical and numerical modeling of tokamak plasmas is simultaneously an essential component of effective reactor design, and a great research barrier. Tokamak operational conditions exhibit comparatively low Knudsen numbers. Kinetic effects, including kinetic waves and instabilities, Landau damping, bump-on-tail instabilities and more, are therefore highly influential in tokamak plasma dynamics. Purely fluid models are inherently incapable of capturing these effects, whereas the high dimensionality in purely kinetic models render them practically intractable for most relevant purposes.

        We consider a $\delta\!f$ decomposition model, with a macroscopic fluid background and microscopic kinetic correction, both fully coupled to each other. A similar manner of discretization is proposed to that used in the recent \texttt{STRUPHY} code \cite{Holderied_Possanner_Wang_2021, Holderied_2022, Li_et_al_2023} with a finite-element model for the background and a pseudo-particle/PiC model for the correction.

        The fluid background satisfies the full, non-linear, resistive, compressible, Hall MHD equations. \cite{Laakmann_Hu_Farrell_2022} introduces finite-element(-in-space) implicit timesteppers for the incompressible analogue to this system with structure-preserving (SP) properties in the ideal case, alongside parameter-robust preconditioners. We show that these timesteppers can derive from a finite-element-in-time (FET) (and finite-element-in-space) interpretation. The benefits of this reformulation are discussed, including the derivation of timesteppers that are higher order in time, and the quantifiable dissipative SP properties in the non-ideal, resistive case.
        
        We discuss possible options for extending this FET approach to timesteppers for the compressible case.

        The kinetic corrections satisfy linearized Boltzmann equations. Using a Lénard--Bernstein collision operator, these take Fokker--Planck-like forms \cite{Fokker_1914, Planck_1917} wherein pseudo-particles in the numerical model obey the neoclassical transport equations, with particle-independent Brownian drift terms. This offers a rigorous methodology for incorporating collisions into the particle transport model, without coupling the equations of motions for each particle.
        
        Works by Chen, Chacón et al. \cite{Chen_Chacón_Barnes_2011, Chacón_Chen_Barnes_2013, Chen_Chacón_2014, Chen_Chacón_2015} have developed structure-preserving particle pushers for neoclassical transport in the Vlasov equations, derived from Crank--Nicolson integrators. We show these too can can derive from a FET interpretation, similarly offering potential extensions to higher-order-in-time particle pushers. The FET formulation is used also to consider how the stochastic drift terms can be incorporated into the pushers. Stochastic gyrokinetic expansions are also discussed.

        Different options for the numerical implementation of these schemes are considered.

        Due to the efficacy of FET in the development of SP timesteppers for both the fluid and kinetic component, we hope this approach will prove effective in the future for developing SP timesteppers for the full hybrid model. We hope this will give us the opportunity to incorporate previously inaccessible kinetic effects into the highly effective, modern, finite-element MHD models.
    \end{abstract}
    
    
    \newpage
    \tableofcontents
    
    
    \newpage
    \pagenumbering{arabic}
    %\linenumbers\renewcommand\thelinenumber{\color{black!50}\arabic{linenumber}}
            \input{0 - introduction/main.tex}
        \part{Research}
            \input{1 - low-noise PiC models/main.tex}
            \input{2 - kinetic component/main.tex}
            \input{3 - fluid component/main.tex}
            \input{4 - numerical implementation/main.tex}
        \part{Project Overview}
            \input{5 - research plan/main.tex}
            \input{6 - summary/main.tex}
    
    
    %\section{}
    \newpage
    \pagenumbering{gobble}
        \printbibliography


    \newpage
    \pagenumbering{roman}
    \appendix
        \part{Appendices}
            \input{8 - Hilbert complexes/main.tex}
            \input{9 - weak conservation proofs/main.tex}
\end{document}


\title{\BA{Title in Progress...}}
\author{Boris Andrews}
\affil{Mathematical Institute, University of Oxford}
\date{\today}


\begin{document}
    \pagenumbering{gobble}
    \maketitle
    
    
    \begin{abstract}
        Magnetic confinement reactors---in particular tokamaks---offer one of the most promising options for achieving practical nuclear fusion, with the potential to provide virtually limitless, clean energy. The theoretical and numerical modeling of tokamak plasmas is simultaneously an essential component of effective reactor design, and a great research barrier. Tokamak operational conditions exhibit comparatively low Knudsen numbers. Kinetic effects, including kinetic waves and instabilities, Landau damping, bump-on-tail instabilities and more, are therefore highly influential in tokamak plasma dynamics. Purely fluid models are inherently incapable of capturing these effects, whereas the high dimensionality in purely kinetic models render them practically intractable for most relevant purposes.

        We consider a $\delta\!f$ decomposition model, with a macroscopic fluid background and microscopic kinetic correction, both fully coupled to each other. A similar manner of discretization is proposed to that used in the recent \texttt{STRUPHY} code \cite{Holderied_Possanner_Wang_2021, Holderied_2022, Li_et_al_2023} with a finite-element model for the background and a pseudo-particle/PiC model for the correction.

        The fluid background satisfies the full, non-linear, resistive, compressible, Hall MHD equations. \cite{Laakmann_Hu_Farrell_2022} introduces finite-element(-in-space) implicit timesteppers for the incompressible analogue to this system with structure-preserving (SP) properties in the ideal case, alongside parameter-robust preconditioners. We show that these timesteppers can derive from a finite-element-in-time (FET) (and finite-element-in-space) interpretation. The benefits of this reformulation are discussed, including the derivation of timesteppers that are higher order in time, and the quantifiable dissipative SP properties in the non-ideal, resistive case.
        
        We discuss possible options for extending this FET approach to timesteppers for the compressible case.

        The kinetic corrections satisfy linearized Boltzmann equations. Using a Lénard--Bernstein collision operator, these take Fokker--Planck-like forms \cite{Fokker_1914, Planck_1917} wherein pseudo-particles in the numerical model obey the neoclassical transport equations, with particle-independent Brownian drift terms. This offers a rigorous methodology for incorporating collisions into the particle transport model, without coupling the equations of motions for each particle.
        
        Works by Chen, Chacón et al. \cite{Chen_Chacón_Barnes_2011, Chacón_Chen_Barnes_2013, Chen_Chacón_2014, Chen_Chacón_2015} have developed structure-preserving particle pushers for neoclassical transport in the Vlasov equations, derived from Crank--Nicolson integrators. We show these too can can derive from a FET interpretation, similarly offering potential extensions to higher-order-in-time particle pushers. The FET formulation is used also to consider how the stochastic drift terms can be incorporated into the pushers. Stochastic gyrokinetic expansions are also discussed.

        Different options for the numerical implementation of these schemes are considered.

        Due to the efficacy of FET in the development of SP timesteppers for both the fluid and kinetic component, we hope this approach will prove effective in the future for developing SP timesteppers for the full hybrid model. We hope this will give us the opportunity to incorporate previously inaccessible kinetic effects into the highly effective, modern, finite-element MHD models.
    \end{abstract}
    
    
    \newpage
    \tableofcontents
    
    
    \newpage
    \pagenumbering{arabic}
    %\linenumbers\renewcommand\thelinenumber{\color{black!50}\arabic{linenumber}}
            \documentclass[12pt, a4paper]{report}

\input{template/main.tex}

\title{\BA{Title in Progress...}}
\author{Boris Andrews}
\affil{Mathematical Institute, University of Oxford}
\date{\today}


\begin{document}
    \pagenumbering{gobble}
    \maketitle
    
    
    \begin{abstract}
        Magnetic confinement reactors---in particular tokamaks---offer one of the most promising options for achieving practical nuclear fusion, with the potential to provide virtually limitless, clean energy. The theoretical and numerical modeling of tokamak plasmas is simultaneously an essential component of effective reactor design, and a great research barrier. Tokamak operational conditions exhibit comparatively low Knudsen numbers. Kinetic effects, including kinetic waves and instabilities, Landau damping, bump-on-tail instabilities and more, are therefore highly influential in tokamak plasma dynamics. Purely fluid models are inherently incapable of capturing these effects, whereas the high dimensionality in purely kinetic models render them practically intractable for most relevant purposes.

        We consider a $\delta\!f$ decomposition model, with a macroscopic fluid background and microscopic kinetic correction, both fully coupled to each other. A similar manner of discretization is proposed to that used in the recent \texttt{STRUPHY} code \cite{Holderied_Possanner_Wang_2021, Holderied_2022, Li_et_al_2023} with a finite-element model for the background and a pseudo-particle/PiC model for the correction.

        The fluid background satisfies the full, non-linear, resistive, compressible, Hall MHD equations. \cite{Laakmann_Hu_Farrell_2022} introduces finite-element(-in-space) implicit timesteppers for the incompressible analogue to this system with structure-preserving (SP) properties in the ideal case, alongside parameter-robust preconditioners. We show that these timesteppers can derive from a finite-element-in-time (FET) (and finite-element-in-space) interpretation. The benefits of this reformulation are discussed, including the derivation of timesteppers that are higher order in time, and the quantifiable dissipative SP properties in the non-ideal, resistive case.
        
        We discuss possible options for extending this FET approach to timesteppers for the compressible case.

        The kinetic corrections satisfy linearized Boltzmann equations. Using a Lénard--Bernstein collision operator, these take Fokker--Planck-like forms \cite{Fokker_1914, Planck_1917} wherein pseudo-particles in the numerical model obey the neoclassical transport equations, with particle-independent Brownian drift terms. This offers a rigorous methodology for incorporating collisions into the particle transport model, without coupling the equations of motions for each particle.
        
        Works by Chen, Chacón et al. \cite{Chen_Chacón_Barnes_2011, Chacón_Chen_Barnes_2013, Chen_Chacón_2014, Chen_Chacón_2015} have developed structure-preserving particle pushers for neoclassical transport in the Vlasov equations, derived from Crank--Nicolson integrators. We show these too can can derive from a FET interpretation, similarly offering potential extensions to higher-order-in-time particle pushers. The FET formulation is used also to consider how the stochastic drift terms can be incorporated into the pushers. Stochastic gyrokinetic expansions are also discussed.

        Different options for the numerical implementation of these schemes are considered.

        Due to the efficacy of FET in the development of SP timesteppers for both the fluid and kinetic component, we hope this approach will prove effective in the future for developing SP timesteppers for the full hybrid model. We hope this will give us the opportunity to incorporate previously inaccessible kinetic effects into the highly effective, modern, finite-element MHD models.
    \end{abstract}
    
    
    \newpage
    \tableofcontents
    
    
    \newpage
    \pagenumbering{arabic}
    %\linenumbers\renewcommand\thelinenumber{\color{black!50}\arabic{linenumber}}
            \input{0 - introduction/main.tex}
        \part{Research}
            \input{1 - low-noise PiC models/main.tex}
            \input{2 - kinetic component/main.tex}
            \input{3 - fluid component/main.tex}
            \input{4 - numerical implementation/main.tex}
        \part{Project Overview}
            \input{5 - research plan/main.tex}
            \input{6 - summary/main.tex}
    
    
    %\section{}
    \newpage
    \pagenumbering{gobble}
        \printbibliography


    \newpage
    \pagenumbering{roman}
    \appendix
        \part{Appendices}
            \input{8 - Hilbert complexes/main.tex}
            \input{9 - weak conservation proofs/main.tex}
\end{document}

        \part{Research}
            \documentclass[12pt, a4paper]{report}

\input{template/main.tex}

\title{\BA{Title in Progress...}}
\author{Boris Andrews}
\affil{Mathematical Institute, University of Oxford}
\date{\today}


\begin{document}
    \pagenumbering{gobble}
    \maketitle
    
    
    \begin{abstract}
        Magnetic confinement reactors---in particular tokamaks---offer one of the most promising options for achieving practical nuclear fusion, with the potential to provide virtually limitless, clean energy. The theoretical and numerical modeling of tokamak plasmas is simultaneously an essential component of effective reactor design, and a great research barrier. Tokamak operational conditions exhibit comparatively low Knudsen numbers. Kinetic effects, including kinetic waves and instabilities, Landau damping, bump-on-tail instabilities and more, are therefore highly influential in tokamak plasma dynamics. Purely fluid models are inherently incapable of capturing these effects, whereas the high dimensionality in purely kinetic models render them practically intractable for most relevant purposes.

        We consider a $\delta\!f$ decomposition model, with a macroscopic fluid background and microscopic kinetic correction, both fully coupled to each other. A similar manner of discretization is proposed to that used in the recent \texttt{STRUPHY} code \cite{Holderied_Possanner_Wang_2021, Holderied_2022, Li_et_al_2023} with a finite-element model for the background and a pseudo-particle/PiC model for the correction.

        The fluid background satisfies the full, non-linear, resistive, compressible, Hall MHD equations. \cite{Laakmann_Hu_Farrell_2022} introduces finite-element(-in-space) implicit timesteppers for the incompressible analogue to this system with structure-preserving (SP) properties in the ideal case, alongside parameter-robust preconditioners. We show that these timesteppers can derive from a finite-element-in-time (FET) (and finite-element-in-space) interpretation. The benefits of this reformulation are discussed, including the derivation of timesteppers that are higher order in time, and the quantifiable dissipative SP properties in the non-ideal, resistive case.
        
        We discuss possible options for extending this FET approach to timesteppers for the compressible case.

        The kinetic corrections satisfy linearized Boltzmann equations. Using a Lénard--Bernstein collision operator, these take Fokker--Planck-like forms \cite{Fokker_1914, Planck_1917} wherein pseudo-particles in the numerical model obey the neoclassical transport equations, with particle-independent Brownian drift terms. This offers a rigorous methodology for incorporating collisions into the particle transport model, without coupling the equations of motions for each particle.
        
        Works by Chen, Chacón et al. \cite{Chen_Chacón_Barnes_2011, Chacón_Chen_Barnes_2013, Chen_Chacón_2014, Chen_Chacón_2015} have developed structure-preserving particle pushers for neoclassical transport in the Vlasov equations, derived from Crank--Nicolson integrators. We show these too can can derive from a FET interpretation, similarly offering potential extensions to higher-order-in-time particle pushers. The FET formulation is used also to consider how the stochastic drift terms can be incorporated into the pushers. Stochastic gyrokinetic expansions are also discussed.

        Different options for the numerical implementation of these schemes are considered.

        Due to the efficacy of FET in the development of SP timesteppers for both the fluid and kinetic component, we hope this approach will prove effective in the future for developing SP timesteppers for the full hybrid model. We hope this will give us the opportunity to incorporate previously inaccessible kinetic effects into the highly effective, modern, finite-element MHD models.
    \end{abstract}
    
    
    \newpage
    \tableofcontents
    
    
    \newpage
    \pagenumbering{arabic}
    %\linenumbers\renewcommand\thelinenumber{\color{black!50}\arabic{linenumber}}
            \input{0 - introduction/main.tex}
        \part{Research}
            \input{1 - low-noise PiC models/main.tex}
            \input{2 - kinetic component/main.tex}
            \input{3 - fluid component/main.tex}
            \input{4 - numerical implementation/main.tex}
        \part{Project Overview}
            \input{5 - research plan/main.tex}
            \input{6 - summary/main.tex}
    
    
    %\section{}
    \newpage
    \pagenumbering{gobble}
        \printbibliography


    \newpage
    \pagenumbering{roman}
    \appendix
        \part{Appendices}
            \input{8 - Hilbert complexes/main.tex}
            \input{9 - weak conservation proofs/main.tex}
\end{document}

            \documentclass[12pt, a4paper]{report}

\input{template/main.tex}

\title{\BA{Title in Progress...}}
\author{Boris Andrews}
\affil{Mathematical Institute, University of Oxford}
\date{\today}


\begin{document}
    \pagenumbering{gobble}
    \maketitle
    
    
    \begin{abstract}
        Magnetic confinement reactors---in particular tokamaks---offer one of the most promising options for achieving practical nuclear fusion, with the potential to provide virtually limitless, clean energy. The theoretical and numerical modeling of tokamak plasmas is simultaneously an essential component of effective reactor design, and a great research barrier. Tokamak operational conditions exhibit comparatively low Knudsen numbers. Kinetic effects, including kinetic waves and instabilities, Landau damping, bump-on-tail instabilities and more, are therefore highly influential in tokamak plasma dynamics. Purely fluid models are inherently incapable of capturing these effects, whereas the high dimensionality in purely kinetic models render them practically intractable for most relevant purposes.

        We consider a $\delta\!f$ decomposition model, with a macroscopic fluid background and microscopic kinetic correction, both fully coupled to each other. A similar manner of discretization is proposed to that used in the recent \texttt{STRUPHY} code \cite{Holderied_Possanner_Wang_2021, Holderied_2022, Li_et_al_2023} with a finite-element model for the background and a pseudo-particle/PiC model for the correction.

        The fluid background satisfies the full, non-linear, resistive, compressible, Hall MHD equations. \cite{Laakmann_Hu_Farrell_2022} introduces finite-element(-in-space) implicit timesteppers for the incompressible analogue to this system with structure-preserving (SP) properties in the ideal case, alongside parameter-robust preconditioners. We show that these timesteppers can derive from a finite-element-in-time (FET) (and finite-element-in-space) interpretation. The benefits of this reformulation are discussed, including the derivation of timesteppers that are higher order in time, and the quantifiable dissipative SP properties in the non-ideal, resistive case.
        
        We discuss possible options for extending this FET approach to timesteppers for the compressible case.

        The kinetic corrections satisfy linearized Boltzmann equations. Using a Lénard--Bernstein collision operator, these take Fokker--Planck-like forms \cite{Fokker_1914, Planck_1917} wherein pseudo-particles in the numerical model obey the neoclassical transport equations, with particle-independent Brownian drift terms. This offers a rigorous methodology for incorporating collisions into the particle transport model, without coupling the equations of motions for each particle.
        
        Works by Chen, Chacón et al. \cite{Chen_Chacón_Barnes_2011, Chacón_Chen_Barnes_2013, Chen_Chacón_2014, Chen_Chacón_2015} have developed structure-preserving particle pushers for neoclassical transport in the Vlasov equations, derived from Crank--Nicolson integrators. We show these too can can derive from a FET interpretation, similarly offering potential extensions to higher-order-in-time particle pushers. The FET formulation is used also to consider how the stochastic drift terms can be incorporated into the pushers. Stochastic gyrokinetic expansions are also discussed.

        Different options for the numerical implementation of these schemes are considered.

        Due to the efficacy of FET in the development of SP timesteppers for both the fluid and kinetic component, we hope this approach will prove effective in the future for developing SP timesteppers for the full hybrid model. We hope this will give us the opportunity to incorporate previously inaccessible kinetic effects into the highly effective, modern, finite-element MHD models.
    \end{abstract}
    
    
    \newpage
    \tableofcontents
    
    
    \newpage
    \pagenumbering{arabic}
    %\linenumbers\renewcommand\thelinenumber{\color{black!50}\arabic{linenumber}}
            \input{0 - introduction/main.tex}
        \part{Research}
            \input{1 - low-noise PiC models/main.tex}
            \input{2 - kinetic component/main.tex}
            \input{3 - fluid component/main.tex}
            \input{4 - numerical implementation/main.tex}
        \part{Project Overview}
            \input{5 - research plan/main.tex}
            \input{6 - summary/main.tex}
    
    
    %\section{}
    \newpage
    \pagenumbering{gobble}
        \printbibliography


    \newpage
    \pagenumbering{roman}
    \appendix
        \part{Appendices}
            \input{8 - Hilbert complexes/main.tex}
            \input{9 - weak conservation proofs/main.tex}
\end{document}

            \documentclass[12pt, a4paper]{report}

\input{template/main.tex}

\title{\BA{Title in Progress...}}
\author{Boris Andrews}
\affil{Mathematical Institute, University of Oxford}
\date{\today}


\begin{document}
    \pagenumbering{gobble}
    \maketitle
    
    
    \begin{abstract}
        Magnetic confinement reactors---in particular tokamaks---offer one of the most promising options for achieving practical nuclear fusion, with the potential to provide virtually limitless, clean energy. The theoretical and numerical modeling of tokamak plasmas is simultaneously an essential component of effective reactor design, and a great research barrier. Tokamak operational conditions exhibit comparatively low Knudsen numbers. Kinetic effects, including kinetic waves and instabilities, Landau damping, bump-on-tail instabilities and more, are therefore highly influential in tokamak plasma dynamics. Purely fluid models are inherently incapable of capturing these effects, whereas the high dimensionality in purely kinetic models render them practically intractable for most relevant purposes.

        We consider a $\delta\!f$ decomposition model, with a macroscopic fluid background and microscopic kinetic correction, both fully coupled to each other. A similar manner of discretization is proposed to that used in the recent \texttt{STRUPHY} code \cite{Holderied_Possanner_Wang_2021, Holderied_2022, Li_et_al_2023} with a finite-element model for the background and a pseudo-particle/PiC model for the correction.

        The fluid background satisfies the full, non-linear, resistive, compressible, Hall MHD equations. \cite{Laakmann_Hu_Farrell_2022} introduces finite-element(-in-space) implicit timesteppers for the incompressible analogue to this system with structure-preserving (SP) properties in the ideal case, alongside parameter-robust preconditioners. We show that these timesteppers can derive from a finite-element-in-time (FET) (and finite-element-in-space) interpretation. The benefits of this reformulation are discussed, including the derivation of timesteppers that are higher order in time, and the quantifiable dissipative SP properties in the non-ideal, resistive case.
        
        We discuss possible options for extending this FET approach to timesteppers for the compressible case.

        The kinetic corrections satisfy linearized Boltzmann equations. Using a Lénard--Bernstein collision operator, these take Fokker--Planck-like forms \cite{Fokker_1914, Planck_1917} wherein pseudo-particles in the numerical model obey the neoclassical transport equations, with particle-independent Brownian drift terms. This offers a rigorous methodology for incorporating collisions into the particle transport model, without coupling the equations of motions for each particle.
        
        Works by Chen, Chacón et al. \cite{Chen_Chacón_Barnes_2011, Chacón_Chen_Barnes_2013, Chen_Chacón_2014, Chen_Chacón_2015} have developed structure-preserving particle pushers for neoclassical transport in the Vlasov equations, derived from Crank--Nicolson integrators. We show these too can can derive from a FET interpretation, similarly offering potential extensions to higher-order-in-time particle pushers. The FET formulation is used also to consider how the stochastic drift terms can be incorporated into the pushers. Stochastic gyrokinetic expansions are also discussed.

        Different options for the numerical implementation of these schemes are considered.

        Due to the efficacy of FET in the development of SP timesteppers for both the fluid and kinetic component, we hope this approach will prove effective in the future for developing SP timesteppers for the full hybrid model. We hope this will give us the opportunity to incorporate previously inaccessible kinetic effects into the highly effective, modern, finite-element MHD models.
    \end{abstract}
    
    
    \newpage
    \tableofcontents
    
    
    \newpage
    \pagenumbering{arabic}
    %\linenumbers\renewcommand\thelinenumber{\color{black!50}\arabic{linenumber}}
            \input{0 - introduction/main.tex}
        \part{Research}
            \input{1 - low-noise PiC models/main.tex}
            \input{2 - kinetic component/main.tex}
            \input{3 - fluid component/main.tex}
            \input{4 - numerical implementation/main.tex}
        \part{Project Overview}
            \input{5 - research plan/main.tex}
            \input{6 - summary/main.tex}
    
    
    %\section{}
    \newpage
    \pagenumbering{gobble}
        \printbibliography


    \newpage
    \pagenumbering{roman}
    \appendix
        \part{Appendices}
            \input{8 - Hilbert complexes/main.tex}
            \input{9 - weak conservation proofs/main.tex}
\end{document}

            \documentclass[12pt, a4paper]{report}

\input{template/main.tex}

\title{\BA{Title in Progress...}}
\author{Boris Andrews}
\affil{Mathematical Institute, University of Oxford}
\date{\today}


\begin{document}
    \pagenumbering{gobble}
    \maketitle
    
    
    \begin{abstract}
        Magnetic confinement reactors---in particular tokamaks---offer one of the most promising options for achieving practical nuclear fusion, with the potential to provide virtually limitless, clean energy. The theoretical and numerical modeling of tokamak plasmas is simultaneously an essential component of effective reactor design, and a great research barrier. Tokamak operational conditions exhibit comparatively low Knudsen numbers. Kinetic effects, including kinetic waves and instabilities, Landau damping, bump-on-tail instabilities and more, are therefore highly influential in tokamak plasma dynamics. Purely fluid models are inherently incapable of capturing these effects, whereas the high dimensionality in purely kinetic models render them practically intractable for most relevant purposes.

        We consider a $\delta\!f$ decomposition model, with a macroscopic fluid background and microscopic kinetic correction, both fully coupled to each other. A similar manner of discretization is proposed to that used in the recent \texttt{STRUPHY} code \cite{Holderied_Possanner_Wang_2021, Holderied_2022, Li_et_al_2023} with a finite-element model for the background and a pseudo-particle/PiC model for the correction.

        The fluid background satisfies the full, non-linear, resistive, compressible, Hall MHD equations. \cite{Laakmann_Hu_Farrell_2022} introduces finite-element(-in-space) implicit timesteppers for the incompressible analogue to this system with structure-preserving (SP) properties in the ideal case, alongside parameter-robust preconditioners. We show that these timesteppers can derive from a finite-element-in-time (FET) (and finite-element-in-space) interpretation. The benefits of this reformulation are discussed, including the derivation of timesteppers that are higher order in time, and the quantifiable dissipative SP properties in the non-ideal, resistive case.
        
        We discuss possible options for extending this FET approach to timesteppers for the compressible case.

        The kinetic corrections satisfy linearized Boltzmann equations. Using a Lénard--Bernstein collision operator, these take Fokker--Planck-like forms \cite{Fokker_1914, Planck_1917} wherein pseudo-particles in the numerical model obey the neoclassical transport equations, with particle-independent Brownian drift terms. This offers a rigorous methodology for incorporating collisions into the particle transport model, without coupling the equations of motions for each particle.
        
        Works by Chen, Chacón et al. \cite{Chen_Chacón_Barnes_2011, Chacón_Chen_Barnes_2013, Chen_Chacón_2014, Chen_Chacón_2015} have developed structure-preserving particle pushers for neoclassical transport in the Vlasov equations, derived from Crank--Nicolson integrators. We show these too can can derive from a FET interpretation, similarly offering potential extensions to higher-order-in-time particle pushers. The FET formulation is used also to consider how the stochastic drift terms can be incorporated into the pushers. Stochastic gyrokinetic expansions are also discussed.

        Different options for the numerical implementation of these schemes are considered.

        Due to the efficacy of FET in the development of SP timesteppers for both the fluid and kinetic component, we hope this approach will prove effective in the future for developing SP timesteppers for the full hybrid model. We hope this will give us the opportunity to incorporate previously inaccessible kinetic effects into the highly effective, modern, finite-element MHD models.
    \end{abstract}
    
    
    \newpage
    \tableofcontents
    
    
    \newpage
    \pagenumbering{arabic}
    %\linenumbers\renewcommand\thelinenumber{\color{black!50}\arabic{linenumber}}
            \input{0 - introduction/main.tex}
        \part{Research}
            \input{1 - low-noise PiC models/main.tex}
            \input{2 - kinetic component/main.tex}
            \input{3 - fluid component/main.tex}
            \input{4 - numerical implementation/main.tex}
        \part{Project Overview}
            \input{5 - research plan/main.tex}
            \input{6 - summary/main.tex}
    
    
    %\section{}
    \newpage
    \pagenumbering{gobble}
        \printbibliography


    \newpage
    \pagenumbering{roman}
    \appendix
        \part{Appendices}
            \input{8 - Hilbert complexes/main.tex}
            \input{9 - weak conservation proofs/main.tex}
\end{document}

        \part{Project Overview}
            \documentclass[12pt, a4paper]{report}

\input{template/main.tex}

\title{\BA{Title in Progress...}}
\author{Boris Andrews}
\affil{Mathematical Institute, University of Oxford}
\date{\today}


\begin{document}
    \pagenumbering{gobble}
    \maketitle
    
    
    \begin{abstract}
        Magnetic confinement reactors---in particular tokamaks---offer one of the most promising options for achieving practical nuclear fusion, with the potential to provide virtually limitless, clean energy. The theoretical and numerical modeling of tokamak plasmas is simultaneously an essential component of effective reactor design, and a great research barrier. Tokamak operational conditions exhibit comparatively low Knudsen numbers. Kinetic effects, including kinetic waves and instabilities, Landau damping, bump-on-tail instabilities and more, are therefore highly influential in tokamak plasma dynamics. Purely fluid models are inherently incapable of capturing these effects, whereas the high dimensionality in purely kinetic models render them practically intractable for most relevant purposes.

        We consider a $\delta\!f$ decomposition model, with a macroscopic fluid background and microscopic kinetic correction, both fully coupled to each other. A similar manner of discretization is proposed to that used in the recent \texttt{STRUPHY} code \cite{Holderied_Possanner_Wang_2021, Holderied_2022, Li_et_al_2023} with a finite-element model for the background and a pseudo-particle/PiC model for the correction.

        The fluid background satisfies the full, non-linear, resistive, compressible, Hall MHD equations. \cite{Laakmann_Hu_Farrell_2022} introduces finite-element(-in-space) implicit timesteppers for the incompressible analogue to this system with structure-preserving (SP) properties in the ideal case, alongside parameter-robust preconditioners. We show that these timesteppers can derive from a finite-element-in-time (FET) (and finite-element-in-space) interpretation. The benefits of this reformulation are discussed, including the derivation of timesteppers that are higher order in time, and the quantifiable dissipative SP properties in the non-ideal, resistive case.
        
        We discuss possible options for extending this FET approach to timesteppers for the compressible case.

        The kinetic corrections satisfy linearized Boltzmann equations. Using a Lénard--Bernstein collision operator, these take Fokker--Planck-like forms \cite{Fokker_1914, Planck_1917} wherein pseudo-particles in the numerical model obey the neoclassical transport equations, with particle-independent Brownian drift terms. This offers a rigorous methodology for incorporating collisions into the particle transport model, without coupling the equations of motions for each particle.
        
        Works by Chen, Chacón et al. \cite{Chen_Chacón_Barnes_2011, Chacón_Chen_Barnes_2013, Chen_Chacón_2014, Chen_Chacón_2015} have developed structure-preserving particle pushers for neoclassical transport in the Vlasov equations, derived from Crank--Nicolson integrators. We show these too can can derive from a FET interpretation, similarly offering potential extensions to higher-order-in-time particle pushers. The FET formulation is used also to consider how the stochastic drift terms can be incorporated into the pushers. Stochastic gyrokinetic expansions are also discussed.

        Different options for the numerical implementation of these schemes are considered.

        Due to the efficacy of FET in the development of SP timesteppers for both the fluid and kinetic component, we hope this approach will prove effective in the future for developing SP timesteppers for the full hybrid model. We hope this will give us the opportunity to incorporate previously inaccessible kinetic effects into the highly effective, modern, finite-element MHD models.
    \end{abstract}
    
    
    \newpage
    \tableofcontents
    
    
    \newpage
    \pagenumbering{arabic}
    %\linenumbers\renewcommand\thelinenumber{\color{black!50}\arabic{linenumber}}
            \input{0 - introduction/main.tex}
        \part{Research}
            \input{1 - low-noise PiC models/main.tex}
            \input{2 - kinetic component/main.tex}
            \input{3 - fluid component/main.tex}
            \input{4 - numerical implementation/main.tex}
        \part{Project Overview}
            \input{5 - research plan/main.tex}
            \input{6 - summary/main.tex}
    
    
    %\section{}
    \newpage
    \pagenumbering{gobble}
        \printbibliography


    \newpage
    \pagenumbering{roman}
    \appendix
        \part{Appendices}
            \input{8 - Hilbert complexes/main.tex}
            \input{9 - weak conservation proofs/main.tex}
\end{document}

            \documentclass[12pt, a4paper]{report}

\input{template/main.tex}

\title{\BA{Title in Progress...}}
\author{Boris Andrews}
\affil{Mathematical Institute, University of Oxford}
\date{\today}


\begin{document}
    \pagenumbering{gobble}
    \maketitle
    
    
    \begin{abstract}
        Magnetic confinement reactors---in particular tokamaks---offer one of the most promising options for achieving practical nuclear fusion, with the potential to provide virtually limitless, clean energy. The theoretical and numerical modeling of tokamak plasmas is simultaneously an essential component of effective reactor design, and a great research barrier. Tokamak operational conditions exhibit comparatively low Knudsen numbers. Kinetic effects, including kinetic waves and instabilities, Landau damping, bump-on-tail instabilities and more, are therefore highly influential in tokamak plasma dynamics. Purely fluid models are inherently incapable of capturing these effects, whereas the high dimensionality in purely kinetic models render them practically intractable for most relevant purposes.

        We consider a $\delta\!f$ decomposition model, with a macroscopic fluid background and microscopic kinetic correction, both fully coupled to each other. A similar manner of discretization is proposed to that used in the recent \texttt{STRUPHY} code \cite{Holderied_Possanner_Wang_2021, Holderied_2022, Li_et_al_2023} with a finite-element model for the background and a pseudo-particle/PiC model for the correction.

        The fluid background satisfies the full, non-linear, resistive, compressible, Hall MHD equations. \cite{Laakmann_Hu_Farrell_2022} introduces finite-element(-in-space) implicit timesteppers for the incompressible analogue to this system with structure-preserving (SP) properties in the ideal case, alongside parameter-robust preconditioners. We show that these timesteppers can derive from a finite-element-in-time (FET) (and finite-element-in-space) interpretation. The benefits of this reformulation are discussed, including the derivation of timesteppers that are higher order in time, and the quantifiable dissipative SP properties in the non-ideal, resistive case.
        
        We discuss possible options for extending this FET approach to timesteppers for the compressible case.

        The kinetic corrections satisfy linearized Boltzmann equations. Using a Lénard--Bernstein collision operator, these take Fokker--Planck-like forms \cite{Fokker_1914, Planck_1917} wherein pseudo-particles in the numerical model obey the neoclassical transport equations, with particle-independent Brownian drift terms. This offers a rigorous methodology for incorporating collisions into the particle transport model, without coupling the equations of motions for each particle.
        
        Works by Chen, Chacón et al. \cite{Chen_Chacón_Barnes_2011, Chacón_Chen_Barnes_2013, Chen_Chacón_2014, Chen_Chacón_2015} have developed structure-preserving particle pushers for neoclassical transport in the Vlasov equations, derived from Crank--Nicolson integrators. We show these too can can derive from a FET interpretation, similarly offering potential extensions to higher-order-in-time particle pushers. The FET formulation is used also to consider how the stochastic drift terms can be incorporated into the pushers. Stochastic gyrokinetic expansions are also discussed.

        Different options for the numerical implementation of these schemes are considered.

        Due to the efficacy of FET in the development of SP timesteppers for both the fluid and kinetic component, we hope this approach will prove effective in the future for developing SP timesteppers for the full hybrid model. We hope this will give us the opportunity to incorporate previously inaccessible kinetic effects into the highly effective, modern, finite-element MHD models.
    \end{abstract}
    
    
    \newpage
    \tableofcontents
    
    
    \newpage
    \pagenumbering{arabic}
    %\linenumbers\renewcommand\thelinenumber{\color{black!50}\arabic{linenumber}}
            \input{0 - introduction/main.tex}
        \part{Research}
            \input{1 - low-noise PiC models/main.tex}
            \input{2 - kinetic component/main.tex}
            \input{3 - fluid component/main.tex}
            \input{4 - numerical implementation/main.tex}
        \part{Project Overview}
            \input{5 - research plan/main.tex}
            \input{6 - summary/main.tex}
    
    
    %\section{}
    \newpage
    \pagenumbering{gobble}
        \printbibliography


    \newpage
    \pagenumbering{roman}
    \appendix
        \part{Appendices}
            \input{8 - Hilbert complexes/main.tex}
            \input{9 - weak conservation proofs/main.tex}
\end{document}

    
    
    %\section{}
    \newpage
    \pagenumbering{gobble}
        \printbibliography


    \newpage
    \pagenumbering{roman}
    \appendix
        \part{Appendices}
            \documentclass[12pt, a4paper]{report}

\input{template/main.tex}

\title{\BA{Title in Progress...}}
\author{Boris Andrews}
\affil{Mathematical Institute, University of Oxford}
\date{\today}


\begin{document}
    \pagenumbering{gobble}
    \maketitle
    
    
    \begin{abstract}
        Magnetic confinement reactors---in particular tokamaks---offer one of the most promising options for achieving practical nuclear fusion, with the potential to provide virtually limitless, clean energy. The theoretical and numerical modeling of tokamak plasmas is simultaneously an essential component of effective reactor design, and a great research barrier. Tokamak operational conditions exhibit comparatively low Knudsen numbers. Kinetic effects, including kinetic waves and instabilities, Landau damping, bump-on-tail instabilities and more, are therefore highly influential in tokamak plasma dynamics. Purely fluid models are inherently incapable of capturing these effects, whereas the high dimensionality in purely kinetic models render them practically intractable for most relevant purposes.

        We consider a $\delta\!f$ decomposition model, with a macroscopic fluid background and microscopic kinetic correction, both fully coupled to each other. A similar manner of discretization is proposed to that used in the recent \texttt{STRUPHY} code \cite{Holderied_Possanner_Wang_2021, Holderied_2022, Li_et_al_2023} with a finite-element model for the background and a pseudo-particle/PiC model for the correction.

        The fluid background satisfies the full, non-linear, resistive, compressible, Hall MHD equations. \cite{Laakmann_Hu_Farrell_2022} introduces finite-element(-in-space) implicit timesteppers for the incompressible analogue to this system with structure-preserving (SP) properties in the ideal case, alongside parameter-robust preconditioners. We show that these timesteppers can derive from a finite-element-in-time (FET) (and finite-element-in-space) interpretation. The benefits of this reformulation are discussed, including the derivation of timesteppers that are higher order in time, and the quantifiable dissipative SP properties in the non-ideal, resistive case.
        
        We discuss possible options for extending this FET approach to timesteppers for the compressible case.

        The kinetic corrections satisfy linearized Boltzmann equations. Using a Lénard--Bernstein collision operator, these take Fokker--Planck-like forms \cite{Fokker_1914, Planck_1917} wherein pseudo-particles in the numerical model obey the neoclassical transport equations, with particle-independent Brownian drift terms. This offers a rigorous methodology for incorporating collisions into the particle transport model, without coupling the equations of motions for each particle.
        
        Works by Chen, Chacón et al. \cite{Chen_Chacón_Barnes_2011, Chacón_Chen_Barnes_2013, Chen_Chacón_2014, Chen_Chacón_2015} have developed structure-preserving particle pushers for neoclassical transport in the Vlasov equations, derived from Crank--Nicolson integrators. We show these too can can derive from a FET interpretation, similarly offering potential extensions to higher-order-in-time particle pushers. The FET formulation is used also to consider how the stochastic drift terms can be incorporated into the pushers. Stochastic gyrokinetic expansions are also discussed.

        Different options for the numerical implementation of these schemes are considered.

        Due to the efficacy of FET in the development of SP timesteppers for both the fluid and kinetic component, we hope this approach will prove effective in the future for developing SP timesteppers for the full hybrid model. We hope this will give us the opportunity to incorporate previously inaccessible kinetic effects into the highly effective, modern, finite-element MHD models.
    \end{abstract}
    
    
    \newpage
    \tableofcontents
    
    
    \newpage
    \pagenumbering{arabic}
    %\linenumbers\renewcommand\thelinenumber{\color{black!50}\arabic{linenumber}}
            \input{0 - introduction/main.tex}
        \part{Research}
            \input{1 - low-noise PiC models/main.tex}
            \input{2 - kinetic component/main.tex}
            \input{3 - fluid component/main.tex}
            \input{4 - numerical implementation/main.tex}
        \part{Project Overview}
            \input{5 - research plan/main.tex}
            \input{6 - summary/main.tex}
    
    
    %\section{}
    \newpage
    \pagenumbering{gobble}
        \printbibliography


    \newpage
    \pagenumbering{roman}
    \appendix
        \part{Appendices}
            \input{8 - Hilbert complexes/main.tex}
            \input{9 - weak conservation proofs/main.tex}
\end{document}

            \documentclass[12pt, a4paper]{report}

\input{template/main.tex}

\title{\BA{Title in Progress...}}
\author{Boris Andrews}
\affil{Mathematical Institute, University of Oxford}
\date{\today}


\begin{document}
    \pagenumbering{gobble}
    \maketitle
    
    
    \begin{abstract}
        Magnetic confinement reactors---in particular tokamaks---offer one of the most promising options for achieving practical nuclear fusion, with the potential to provide virtually limitless, clean energy. The theoretical and numerical modeling of tokamak plasmas is simultaneously an essential component of effective reactor design, and a great research barrier. Tokamak operational conditions exhibit comparatively low Knudsen numbers. Kinetic effects, including kinetic waves and instabilities, Landau damping, bump-on-tail instabilities and more, are therefore highly influential in tokamak plasma dynamics. Purely fluid models are inherently incapable of capturing these effects, whereas the high dimensionality in purely kinetic models render them practically intractable for most relevant purposes.

        We consider a $\delta\!f$ decomposition model, with a macroscopic fluid background and microscopic kinetic correction, both fully coupled to each other. A similar manner of discretization is proposed to that used in the recent \texttt{STRUPHY} code \cite{Holderied_Possanner_Wang_2021, Holderied_2022, Li_et_al_2023} with a finite-element model for the background and a pseudo-particle/PiC model for the correction.

        The fluid background satisfies the full, non-linear, resistive, compressible, Hall MHD equations. \cite{Laakmann_Hu_Farrell_2022} introduces finite-element(-in-space) implicit timesteppers for the incompressible analogue to this system with structure-preserving (SP) properties in the ideal case, alongside parameter-robust preconditioners. We show that these timesteppers can derive from a finite-element-in-time (FET) (and finite-element-in-space) interpretation. The benefits of this reformulation are discussed, including the derivation of timesteppers that are higher order in time, and the quantifiable dissipative SP properties in the non-ideal, resistive case.
        
        We discuss possible options for extending this FET approach to timesteppers for the compressible case.

        The kinetic corrections satisfy linearized Boltzmann equations. Using a Lénard--Bernstein collision operator, these take Fokker--Planck-like forms \cite{Fokker_1914, Planck_1917} wherein pseudo-particles in the numerical model obey the neoclassical transport equations, with particle-independent Brownian drift terms. This offers a rigorous methodology for incorporating collisions into the particle transport model, without coupling the equations of motions for each particle.
        
        Works by Chen, Chacón et al. \cite{Chen_Chacón_Barnes_2011, Chacón_Chen_Barnes_2013, Chen_Chacón_2014, Chen_Chacón_2015} have developed structure-preserving particle pushers for neoclassical transport in the Vlasov equations, derived from Crank--Nicolson integrators. We show these too can can derive from a FET interpretation, similarly offering potential extensions to higher-order-in-time particle pushers. The FET formulation is used also to consider how the stochastic drift terms can be incorporated into the pushers. Stochastic gyrokinetic expansions are also discussed.

        Different options for the numerical implementation of these schemes are considered.

        Due to the efficacy of FET in the development of SP timesteppers for both the fluid and kinetic component, we hope this approach will prove effective in the future for developing SP timesteppers for the full hybrid model. We hope this will give us the opportunity to incorporate previously inaccessible kinetic effects into the highly effective, modern, finite-element MHD models.
    \end{abstract}
    
    
    \newpage
    \tableofcontents
    
    
    \newpage
    \pagenumbering{arabic}
    %\linenumbers\renewcommand\thelinenumber{\color{black!50}\arabic{linenumber}}
            \input{0 - introduction/main.tex}
        \part{Research}
            \input{1 - low-noise PiC models/main.tex}
            \input{2 - kinetic component/main.tex}
            \input{3 - fluid component/main.tex}
            \input{4 - numerical implementation/main.tex}
        \part{Project Overview}
            \input{5 - research plan/main.tex}
            \input{6 - summary/main.tex}
    
    
    %\section{}
    \newpage
    \pagenumbering{gobble}
        \printbibliography


    \newpage
    \pagenumbering{roman}
    \appendix
        \part{Appendices}
            \input{8 - Hilbert complexes/main.tex}
            \input{9 - weak conservation proofs/main.tex}
\end{document}

\end{document}

\end{document}
