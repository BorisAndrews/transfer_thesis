\subsubsection*{Assumption 2: EM vs. Collisions Dominant Balance}
    Consider now the scale of the dimensionless quantities in (\ref{eqn:Boltzmann equation non-dimensionalized}), to consider which terms are in dominant balance.
    
    Consider a typical JET reactor pulse, with a predominant (positive) deuterium ion (indexed via $s  \mapsto  +$) and (negative) electron (indexed via $s  \mapsto  -$) phase:
    \begin{equation}
        \rmq_{\pm}  \approx  \pm 1.602\ldots\times 10^{- 19}{\rm kg}\rms^{- 1}\rmT^{- 1},
    \end{equation}
    \begin{align}
        \rmm_{+}    \approx  3.344\ldots\times 10^{- 27}{\rm kg},  &&
        \rmm_{-}    \approx  9.109\ldots\times 10^{- 31}{\rm kg}.
    \end{align}
    The following parameters for the JET reactor are listed in Chapters 2 and 4 of \cite{Wes00}:\footnote{$\overline{\bfx}$ is evaluated as twice the minor radius, $1.25\rmm$.}
    \begin{align}
        \overline{\bfx}     \approx  2.5\rmm,  &&
        \overline{n}_{\pm}  \approx  10^{19}\rmm^{- 3},  &&
        \overline{\bfB}     \approx  3.5\rmT.
    \end{align}
    For the remaining two scales:
    \begin{itemize}
        \item  $\overline{t}$ shall be taken as equal to $\overline{\bfx}/\overline{\bfv}$, such that $\rmSt  =  1$, i.e. the model is on the convective timescale.

        \item  $\overline{\bfv}$ shall be taken, as stated above, to achieve dominant balance.
        
        While the Knudsen numbers, $\rmKn_{\pm\pm}$, scale \emph{with} $\overline{\bfv}$, the cyclotron numbers $\rmCy\!_{\pm}$ scale \emph{inversely}. At a certain velocity scale, $\overline{\bfv}$, therefore, a balance will be achieved, where $\max\{|\rmCy\!_{\pm}|\}  =  \max\{\rmKn_{\pm\pm}\}$. When such a balance is achieved, as will be seen after their evaluation below, $\max\{|\rmCy\!_{\pm}|\}  =  \max\{\rmKn_{\pm\pm}\}  \gg  1$, implying this balance is dominant. This equilibrium velocity scale can be evaluated as the thermal velocity scale at operational temperature,
        \begin{equation}
            \overline{\bfv}  \approx  \sqrt{\frac{\rmk_{\rmB}}{\rmm_{+}}\cdot\overline{\theta}},
        \end{equation}
        where $\rmk_{\rmB}$ is the Boltzmann constant, $\rmk_{\rmB}  \approx  1.381\ldots\times 10^{- 23}\rmm^{2}{\rm kg}\rmK^{- 1}\rms^{- 2}$, and $\theta$ denotes the temperature. For $\overline{\theta}  \approx  1.5\ldots\times 10^{8}$ as given by \cite{Wes00}, $\overline{\bfv}  \approx  7.869\ldots\times 10^{5}\rmm\rms^{- 1}$.
    \end{itemize}

    The dimensionless quantities in Figure \ref{fig:kinetic dimensionless quantities} thus evaluate as:
    \begin{align}
                                                           \rmM  &\approx  2.625\ldots\times 10^{- 3}  \\
                                                          \rmSt  &=        1                           \\
        |\rmCy\!_{+}|  \approx  \max\{\rmKn_{++},  \rmKn_{+-}\}  &\approx  5.327\ldots\times 10^{2}    \\
        |\rmCy\!_{-}|  \approx  \max\{\rmKn_{--},  \rmKn_{-+}\}  &\approx  1.957\ldots\times 10^{6}
    \end{align}

    ...

    Under typical tokamak conditions therefore, one can expect the Boltzmann equations for either phase are dominated by the EM terms, $\rmCy\!_{\pm}\nabla_{\bfv}\cdot[f_{\pm}\left(\bfE + \bfv\wedge\bfB\right)]$, and the collisional terms, $\rmKn_{\pm_{1}\pm_{2}}\nabla_{\bfv}\cdot\bfC_{\pm_{1}\pm_{2}}$, by a factor of $\rmCy_{\pm}$. The Boltzmann equations then take the leading-order forms:
    \begin{equation}
        \rmCy\!_{\pm}\nabla_{\bfv}\cdot[f_{\pm}(\bfE + \bfv\wedge\bfB)]  =  \rmKn_{\pm\pm}\nabla_{\bfv}\cdot\bfC_{\pm\pm} + \rmKn_{\pm\mp}\nabla_{\bfv}\cdot\bfC_{\pm\mp} + \calO[1]
    \end{equation}
    \begin{equation}
        \nabla_{\bfv}\cdot[\rmCy\!_{\pm}f_{\pm}(\bfE + \bfv\wedge\bfB) - \rmKn_{\pm\pm}\bfC_{\pm\pm} - \rmKn_{\pm\mp}\bfC_{\pm\mp}]  =  \calO[1]  \label{eqn:leading-order Boltzmann equation}
    \end{equation}
    This approximation holds true only to an accuracy of $1/\rmCy\!_{\pm}$. While it is true that in each phase, $\rmCy\!_{\pm} > 1$, these values are \emph{very small} in comparison to those that would typically be found in other comparable kinetic systems. This is particularly true in the ion phase, with $\rmCy\!_{+}$ at less than $10^{3}$ in magnitude; compare with the equivalent factor of around $10^{11}$ for the hydrogen phase in the full solar corona.

    It is primarily this factor that is responsible for the shortcoming of the fluid approximation for tokamak plasmas, and the high influence of kinetic effects on their dynamics.

    \begin{remark}[Justification for the non-relativistic model]
        Up until now, a non-relativistic model has been assumed. Observing that $\rmM  \ll  1$, i.e.. $\overline{\bfv}  \ll  c$, we see that this is \emph{in general} a fair assumption. Relativistic effects are known to have some effects on tokamak plasma dynamics however, due to their impact on the very-high energy/speed tails of the distribution functions. \BA{[Ref]}
    \end{remark}
    